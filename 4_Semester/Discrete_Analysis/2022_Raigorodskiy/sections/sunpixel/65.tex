Введём такое обозначение:
$$
m(n) = \min \{ k : \exists n-\text{однородный гиперграф } H \text{ с } 
k \text{ рёбрами и } \chi(H) > 2 \} 
$$

Напомним, что $\chi(H)$ -- минимальное количество цветов, в которое можно покрасить вершины гиперграфа так, чтобы ни одно ребро не было одноцветным.

\textbf{Теорема 1.}

$$
    m(n) \geqslant 2^{n - 1}
$$

\textit{Доказательство.}

Чтобы доказать это неравенство нужно показать, что для любого гиперграфа $H$ с количеством рёбер $k < 2^{n - 1}$ $\chi(H) \leqslant 2$. Докажем, что такой гиперграф можно раскрасить в 2 цвета правильным образом.

Пусть $A_i$ -- событие, которое состоит в том, что $i$-е ребро одноцветно.
$$
    P(A_i) = 2 \cdot \frac{1}{2^n} = 2^{1 - n} \\
$$

$$
P \left( \bigcup_{i = 1}^{k} A_i \right) \leqslant k \cdot P(A_i) < 
    2^{n - 1} \cdot 2^{1 - n} = 1
$$

$$
P \left( \bigcap_{i = 1}^{k} \overline{A_i} \right) > 0
$$

Это означает, что правильная раскраска из двух цветов существует и $\chi(H) \leqslant 2$.

\textbf{Теорема 2.}

$$
    m(n) \leqslant C_{2n - 1}^{n}
$$

\textit{Доказательство.}

Чтобы доказать это неравенство нужно показать, что существует $n$-однородный гиперграф с
$C_{2n - 1}^{n}$ рёбрами, который нельзя раскрасить в 2 цвета правильным образом.

Возьмём гиперграф с $2n - 1$ вершинами со всеми возможными рёбрами (то есть полный $n$-однородный гиперграф с $2n - 1$ вершинами). Если мы $2n - 1$ вершину покрасим в два цвета, то по принципу Дирихле найдутся $n$ вершин, покрашенных в один цвет. Так как наш граф полный, то эти $n$ вершин образуют ребро, а значит оно получилось одноцветным. 