\section{Строение групп}

\subsection{Теоремы Силова}

\begin{definition}
	Пусть $G$ "--- конечная группа, $|G| = n$, $p$ "--- простой делитель числа $n$, $n = p^ks$, $(p, s) = 1$. Тогда подгруппа $H \le G$ такая, что $|H| = p^k$, называется \textit{силовской $p$-подгруппой} группы $G$.
\end{definition}

\begin{note}
	Если $G$ "--- конечная группа, $|G| = n$, $t \mid n$, то необязательно в $G$ есть подгруппа порядка $t$. Например, в группе $A_5$, $|A_5| = 60$ нет подгруппы порядка 30, поскольку такая подгруппа была бы нормальной, а $A_5$ "--- простая.
\end{note}

\begin{theorem}[Теоремы Силова]
	Пусть $G$ "--- конечная группа, $|G| \hm= n$, $p$ "--- простой делитель числа $n$, $n = p^ks$, $(p, s) = 1$. Обозначим через $N_p$ количество силовских $p$-подгрупп в $G$ Тогда:
	\begin{enumerate}[align=left, leftmargin=15pt]
		\item В $G$ существует силовская $p$-подгруппа, то есть $N_p > 0$
		\item[1'.] Любая $p$-подгруппа в $G$ содержится в некоторой силовской
		\item Все силовские $p$-подгруппы в $G$ сопряжены
		\item $N_p \equiv_p 1$
		\item[3'.] $N_p \mid s$
	\end{enumerate}
\end{theorem}

\begin{proof}[Доказательство 1 и 3]
	Положим $\Omega := \{M  \subset G: |M| = p^k\}$ и рассмотрим действие $G$ на $\Omega$ левыми сдвигами: $\forall g \in G: \forall M \in \Omega: g(M) = gM$. Если для некоторого $M \hm\in \Omega$ его стабилизатор "--- это $H \le G$, то $M = HM \hm= \bigcup_{m \in M}Hm$, то есть $M$ разбивается на непересекающиеся классы. В частности, это означает, что $|H| \mid |M| = p^k$, то есть $|H| = p^l, l \le k$. Тогда $|H| = p^k \lra M = Hg, g \in G \lra |G(M)| \hm= |G : H| = s$. Если же $|H| = p^l, l < k$, то $|G(M)| = |G : H| \hm= p^{k-l}s \equiv_{ps} 0$. Обозначим через $\Omega_1, \dotsc, \Omega_r$ орбиты действия и воспользуемся формулой орбит:
	\[\binom{n}{p^k} = |\Omega| = \sum\limits_{i = 1}^r|\Omega_i| = \sum\limits_{i = 1}^r|G : \St(M_i)| \equiv_{ps} N_ps\]
	
	Поскольку $\binom{n}{p^k} \equiv_{ps} N_ps$ выполнено в любой группе $G$ порядка $n$, мы можем рассмотреть конкретную группу $\Z_n$. Единственная силовская $p$-подгруппа в $\Z_n$ "--- это $s\Z_n$, поэтому в данном случае $N_p = 1$ и $\binom{n}{p^k} \equiv_{ps} s$. Возвращаясь к случаю произвольной группы, получаем, что $s \equiv_{ps} \binom{n}{p^k} \equiv_{ps} N_ps \ra ps\mid(N_p - 1)s \ra N_p \equiv_p 1$.
\end{proof}

\begin{proof}[Доказательство 1' и 2]
	Пусть $P \le G$ "--- силовская $p$-подгруппа, а $Q$ "--- $p$-подгруппа в $G$. Рассмотрим действие $Q$ на $G / P$ левыми сдвигами: $\forall q \in Q: \forall gP \in G / P: q(gP) \hm= qgP$. Обозначим через $\Omega_1, \dotsc, \Omega_r$ орбиты действия и воспользуемся формулой орбит:
	\[s = |G / P| = \sum\limits_{i = 1}^r |\Omega_i| = \sum\limits_{i = 1}^r |Q : \St(\omega_i)|\]
	
	Поскольку правая часть "--- это сумма выражений вида $p^l$, $l \in \Z$, и $p \nmid s$, то $\exists i \hm\in \{1, \dotsc, r\}: |Q: \St(\omega_i)| = 1 \lra \St(\omega_i) = Q$. Положим $\omega_i := gP$, тогда $QgP = gP \hm\ra QP^{g^{-1}} = P^{g^{-1}} \ra Q \le P^{g^{-1}}$, причем подгруппа $P^{g^{-1}}$ "--- силовская. Если же $Q$ "--- тоже силовская $p$-подгруппа в $G$, то $|Q| = |P|$, поэтому $Q = P^{g^{-1}}$.
\end{proof}

\begin{proof}[Доказательство 3']
	Пусть $P \le G$ "--- силовская $p$-подгруппа. Тогда все силовские $p$-подгруппы имеют вид $P^g$, $g \in G$. Они образуют орбиту $P$ при действии $G$ на множестве своих подгрупп сопряжениями. Тогда $N_p = \frac{|G|}{|N_G(P)|}$ и, поскольку $P \le N_G(P)$, $N_p \mid s$.
\end{proof}

\begin{note}
	Пусть $G$ "--- конечная группа, $|G| = n$, $p$ "--- простой делитель числа $n$, $n \hm= p^ks$, $(p, s) = 1$. Обозначим через $N_p(l)$ число подгрупп порядка $p^l$, $l \le k$. Аналогично доказательству выше, можно показать, что $N_p(l) \equiv_p 1$.
\end{note}

\begin{example}
	Пусть $G$ "--- группа, $|G| = pq$, где $p > q$ "--- простые числа. Покажем, что $G$ разрешима. Пусть $P$ "--- силовская $p$-подгруппа в $G$. Тогда либо $N_p = 1$, либо $N_p \ge p + 1$. С другой стороны, $N_p\mid q$, поэтому второй случай невозможен. Значит, $P$ "--- единственная силовская $p$-подгруппа. В частности, $\forall g \in G: P^g = P$, поэтому $P \normal G$, причем $P \cong \Z_p$, $G / P \cong \Z_q$. Обе эти группы абелевы и потому разрешимы, поэтому $G$ разрешима. Более того, если $Q$ "--- силовская $q$-подгруппа, то $P \cap Q = \{e\} \ra PQ = G$, поэтому $G = P \sd Q \cong \Z_p \sd \Z_q$.
\end{example}

\begin{theorem} Пусть $G$ "--- группа. Тогда:
	\begin{enumerate}
		\item Если $P$ "--- силовская $p$-подгруппа в $G$, то $P \normal G \hm\lra N_p = 1$
		\item Если $|G| = \prod_{i = 1}^mp_i^{\alpha_i}$ и $\forall i \in \{1, \dotsc, m\}: P_i$ "--- силовская $p_i$-подгруппа в $G$, то $\forall i \hm\in \{1, \dotsc, m\}: P_i \normal G$ тогда и только тогда, когда $G = P_1 \times \dotsb \times P_m$
	\end{enumerate}
\end{theorem}

\begin{proof}~
	\begin{enumerate}
		\item \begin{itemize}
			\item[$\la$] Если $N_p = 1$, то $\forall g \in G: P^g = P$, поэтому $P \normal G$.
			
			\item[$\ra$] Если $P \normal G$, то любая другая силовская $p$-подгруппа в $G$ имеет вид $P^g, g \in G$, тогда, поскольку $P^g = P$, $N_p = 1$.
		\end{itemize}
		
		\item \begin{itemize}
			\item[$\la$] Если $G = P_1 \times \dotsb \times P_m$, то, по определению прямого произведения, $\forall i \hm\in \{1, \dotsc, m\}: P_i \normal G$.
			
			\item[$\ra$] Проведем индукцию по $m$. База, $m = 1$, тривиальна. Пусть теперь $m > 1$. Положим $H := P_1\dots P_{m - 1} \normal G$. Тогда в $H$ есть силовские подгруппы $P_1, \dotsc, P_{m - 1}$, и, более того, все они нормальны в $H$. По предположению индукции, $H \hm= P_1 \times \dotsb \times P_{m-1}$. Поскольку $(|H|, |P_m|) = 1$, то $H \cap P_m \hm= \{e\} \ra HP_m \hm= G$, и $G = P_1 \times \dotsb \times P_m$.
		\end{itemize}
	\end{enumerate}
\end{proof}