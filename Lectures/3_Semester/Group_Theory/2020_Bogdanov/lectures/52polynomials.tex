\subsection{Кольцо многочленов над полем}

\begin{reminder}
	\textit{Кольцо многочленов над полем} $F$ "--- это множество формальных записей вида $a_nx^n + \dotsc + a_0$, $\forall i \in \{0, \dotsc, n\} : a_i \in F$. Обозначение "--- $F[x]$.
\end{reminder}

\begin{theorem} Пусть $F$ "--- поле. Тогда:
	\begin{enumerate}
		\item Любой идеал в $F[x]$ "--- главный
		\item Если $p \in F[x]$, то идеал $(p)$ максимален $\lra$ многочлен $p$ неприводим над $F$.
	\end{enumerate}
\end{theorem}

\begin{proof}~
	\begin{enumerate}
		\item Пусть $I \normal F[x]$. Если $I = \{0\}$, то $I = (0)$. В противном случае выберем многочлен $p \in I$, $p \ne 0$, наименьшей степени. Тогда, из соображений деления с остатком, $\forall g \hm\in I: p \mid g$. Значит, $(p) = pF[x] = I$.
		\item Если $p \in F$, $p \ne 0$, то $(p) = (1) = F[x]$ "--- не максимальный идеал. Если $p = p_1p_2$, $\deg{p_1}, \deg{p_2} < \deg{p}$, то $(p_1), (p_2) \supset (p)$. Пусть теперь $p$ неприводим. Тогда рассмотрим $J = (q) \normal F[x]$, $J \supset (p)$. Значит, $q \mid p$, и либо $p, q$ ассоциированы, либо $q \in F$. В первом случае $(p) = (q)$, а во втором "--- $(q) = (1) = F[x]$.
	\end{enumerate}
\end{proof}

\begin{note}
	Если $p \in F[x]$ неприводим, то в поле $K := F[x] / p$ есть подполе, изоморфное $p$, имеющее вид $\{\overline{\alpha} = \alpha + (p): \alpha \in F\}$. Поэтому можно считать, что $F \le K$. Кроме того, $\overline{x} = x + (p) \in K$, и $p(\overline{x}) = p(x) + (p) = (p)$. Поскольку $(p)$ "--- нулевой элемент в $K$, то $\overline{x}$ "--- это корень $p$ в поле $K$.
\end{note}

\begin{definition}
	Пусть $R$ "--- коммутативное кольцо, $S \le R$. Тогда:
	\begin{itemize}
		\item $R$ называется \textit{расширением} кольца $S$.
		\item \textit{Расширением кольца $S$ элементами $a_1, \dotsc, a_k \in R$} называется наименьшее по включению подкольцо в $R$, которое содержит $S$ и $a_1, \dotsc, a_k$. Обозначение "--- $S[a_1, \dotsc, a_k]$.
	\end{itemize}
\end{definition}

\begin{note}
	$S[a_1, \dotsc, a_k] = \{p(a_1, \dotsc, a_k) : p \in S[x_1, \dotsc, x_k]\}$.
\end{note}

\begin{definition}
	Пусть $K$ "--- поле, $F \le K$. Тогда:
	\begin{itemize}
		\item $K$ называется \textit{расширением} поля $F$.
		\item \textit{Расширением поля $F$ элементами $a_1, \dotsc, a_k \in K$} называется наименьшее по включению подполе в $K$, которое содержит $F$ и $a_1, \dotsc, a_k$. Обозначение "--- $F(a_1, \dotsc, a_k)$.
	\end{itemize}
\end{definition}

\begin{note}
	Аналогично случаю колец, выполнено равенство:
	\[F(a_1, \dotsc, a_k) = \left\{\frac{p(a_1, \dotsc, a_k)}{q(a_1, \dotsc, a_k)} : p, q \in F[x_1, \dotsc, x_k], q(a_1, \dotsc, a_k) \ne 0\right\}\]
\end{note}

\begin{definition}
	Если поле $K$ "--- расширение поля $F$, то $K$ "--- это линейное пространство над $F$. Размерность $K$ называется \textit{степенью расширения}. Обозначение "--- $[K : F]$. Расширение называется \textit{конечным}, если его степень конечна.
\end{definition}

\begin{note}
	Если $F, K, L$ "--- поля такие, что $F \le K \le L$ и оба расширения конечны, то $[L : F] = [L : K][K : F]$.
\end{note}

\begin{definition}
	Пусть $K$ "--- расширение поля $F$, $a \in K$. Элемент $a$ называется \textit{алгебраическим над $F$}, если $\exists p \in F[x], p \ne 0: p(a) = 0$. Многочлен наименьшей степени, удовлетворяющий этому условию, называется \textit{минимальным многочленом} элемента $a$.
\end{definition}

\begin{theorem}
	Пусть $K$ "--- расширение поля $F$, $a \in K$. Тогда:
	\begin{enumerate}
		\item $a$ "--- алгебраический над $F$ $\lra$ расширение $F(a)$ конечно
		\item Если расширение $F(a)$ конечно, то $F(a) = F[a] \cong F[x] / (p)$, где $p$ "--- минимальный многочлен элемента $a$
	\end{enumerate}
\end{theorem}

\begin{proof}~
	\begin{enumerate}
		\item Рассмотрим поле $F(a)$ и систему элементов $\{1, a, a^2, \dotsc\}$ в нем. Если $a$ "--- не алгебраический над $F$, то эта система линейно независима над $F$, поэтому $[F(a) : F] = \infty$. Если же $a$ "--- алгебраический над $F$, то для некоторого $p \in F[x], \deg{p} = n \ge 1$ выполнено $p(a) = 0$. Отсюда $a^n$ выражается через $1, \dotsc, a^{n-1}$, и, по индукции, $\forall k \in \N:$ $a^{n + k}$ выражается через $1, \dotsc, a^{n-1}$. Значит, $\dim{F[a]} \le n$, и, в силу следующего пункта, $\dim{F(a)} \le n$.
		\item Рассмотрим $p$ "--- минимальный многочлен элемента $a$. В силу минимальности, $p$ неприводим над $F$, и $F[x] / (p)$ "--- это поле. Рассмотрим гомоморфизм колец $\phi: F[x] \to K$, $\forall q \in F[x]: \phi(q) \hm= q(a)$. Пусть $I := \ke\phi$, тогда $p \in I$ и $(p) \subset I$. Поскольку идеал $(p)$ максимален и $I \ne F[x]$, то $I = (p)$. Тогда, по основной теореме о гомоморфизмах колец, $F[x] / (p) \cong \im\phi = F[a]$. Но $F[x] / (p)$ "--- поле, поэтому $F[a]$ "--- тоже поле и $F[a] \hm= F(a)$.
	\end{enumerate}
\end{proof}

\begin{corollary}
	Если $K$ "--- расширение поля $F$ и $a_1, \dotsc, a_k \in K$ "--- алгебраические над $F$, то и все элементы в $F(a_1, \dotsc, a_k)$ "--- тоже алгебраические, поскольку степень $[F(a_1, \dotsc, a_k) : F]$ конечна.
\end{corollary}

\begin{note}
	Мы доказали, что расширение поля корнем неприводимого многочлена единственно с точностью до изоморфизма. Отсюда можно получить индукцией по степени многочлена, что поле разложения любого многочлена единственно с точностью до изоморфизма. Следовательно, поле порядка $p^n$, полученное как поле разложения многочлена $x^{p^n} - x$ над $\Z_p$, единственно с точностью до изоморфизма.
\end{note}