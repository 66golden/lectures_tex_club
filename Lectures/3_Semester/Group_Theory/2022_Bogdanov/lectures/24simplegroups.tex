\subsection{Простые группы}

\begin{definition}
	Пусть $G$ --- группа, $|G| > 1$. $G$ называется \textit{простой}, если в ней нет нормальных подгрупп, отличных от $\{e\}$ и $G$.
\end{definition}

\begin{note}
	Пусть $G$ --- конечная разрешимая группа. Рассмотрим максимальный субнормальный ряд (факторы не обязательно абелевы, поэтому разрешимость мы не затрагиваем. Требуется лишь нормальность) $G = G_0 \normalr G_1 \normalr \dots \normalr G_n = \{e\}$, в котором все группы различны. Тогда $\forall i \in \{1, \ldots, n\}\ G_{i-1} / G_i$ --- простая, поскольку иначе $\exists H / G_i \normal G_{i - 1} / G_i$ и, по второй теореме об изоморфизме, $H \normal G_{i - 1}$. При этом $G_i \normal H$, коль скоро $G_i$ нормальна в б\'{о}льшей группе $G_{i - 1}$. Стало быть, можем вставить $H$ в наш ряд, а это невозможно по максимальности.
\end{note}

\begin{note}
	Существует \textit{теорема Жордана-Гёльдера}, гласящая, что в любых двух наибольших по включению нормальных (субнормальных) рядах соответствующие факторгруппы изоморфны (с точностью до перестановки).
\end{note}

\begin{proposition}
	Абелева группа $A$ --- простая $\Lra$ $A \cong \Z_p$, где число $p$ --- простое.
\end{proposition}

\begin{proof}~
	\begin{itemize}
		\item[$\ra$] Коль скоро $A$ --- простая, то $\exists a \in A \bs \{e\}$. Рассмотрим 2 ситуации:
		\begin{itemize}
			\item $\ord a = n$. Тогда возьмём любой простой множитель $p \mid n$. Положим $b = a^{n / p} \in A$, для которого $\ord b = p$. В силу абелевости $A$, имеем нормальную подгруппу $\trb{b} \normal A$. В силу простоты, $A \cong \Z_p$ (и, соответственно, $n = p$).
			
			\item $\ord a = \infty$. Поймём, что такой ситуации не может быть в принципе. Посмотрим на $H = \trb{a^2}$. Тогда $a \notin H$ и, следовательно, $H \neq \{e\} \wedge H \neq G$, то есть $G$ непроста.
		\end{itemize}
		
		\item[$\la$] Если $A \cong \Z_p$, то $\forall B \le A\ |B| = 1$ или $|B| = p$, то есть $B = \{e\}$ или $B = A$, поэтому $A$ "--- простая.
	\end{itemize}
\end{proof}

\begin{proposition}
	Пусть $G$ "--- конечная группа, $H \le G$, причем $|G : H| = 2$, и $h \in H$. Если $C_G(h) \ne C_H(h)$, то $h^G = h^H$. В противном случае, $|h^G| = 2|h^H|$.
\end{proposition}

\begin{proof}
	Коль скоро $|G \colon H| = 2$, то $H \normal G$ и верно такое разбиение $G$:
	\[
		\forall g \in G \bs H\ \ G = H \sqcup gH = H \sqcup Hg
	\]
	\begin{itemize}
		\item $C_G(h) \neq C_H(h)$. Возьмём $g \in C_G(h) \bs C_H(h)$, для которого $gh = hg$ и $g \notin H$. Значит, с его помощью мы разбиваем $g$ на 2 класса: $G = H \sqcup gH$. Осталась цепочка равенств:
		\[
			h^G = h^{H \sqcup gH} = h^H \cup h^{gH} = h^H \cup (h^g)^H = h^H
		\]
		
		\item Теперь $C_G(h) = C_H(h)$. Тогда $\forall g \in G \bs H\ h^g \notin h^H$. Действительно, если бы это было так, то:
		\[
			h^g = h^x,\ x \in H \ra h^{gx^{-1}} = h^{xx^{-1}} = h \ra gx^{-1} \in C_G(h) = C_H(h) \le H
		\]
		Но в то же время $gx^{-1} \notin H$. Значит, для такого элемента $g$ имеет место равенство $h^G = h^H \sqcup h^{gH}$ и остаётся доказать, что классы равномощны. Для этого построим явно биекцию между $h^H$ и $h^{Hg} = h^{gH}$ как $x \mapsto x^g$ (обратное отображение имеет явный вид $y \mapsto y^{g^{-1}}$, что доказывает биективность).
	\end{itemize}
\end{proof}

\begin{theorem}
	$A_5$ является простой группой
\end{theorem}

\begin{proof}
	Отметим, что $|A_5| = 60$. Доказательство состоит в том, чтобы рассмотреть всевозможные $\{e\} \neq H \normal A_5$, причём мы знаем их вид --- объединение классов сопряженности и, более того, $|H| \mid 60$. Пользуясь доказанным утверждением, перечислим все классы сопряженности элементов $h \in A_5$ в $A_5$ и $S_5$:
	\def\arraystretch{1.5}
	\begin{center}
		\begin{tabular}{c|c|c|c}
			$h^{S_5}$ & $g \in C_{S_5}(h) \backslash C_{A_5}(h)$ & $h^{A_5}$ & |$h^{A_5}$| \\
			\hline
			$\id^{S_5}$ & $(12)$ & $\id^{S_5}$ & 1\\
			\hline
			$(123)^{S_5}$ & $(45)$ & $(123)^{S_5}$ & $\frac{A_5^3}3 = 20$\\
			\hline
			$((12)(34))^{S_5}$ & $(12)$ & $((12)(34))^{S_5}$ & $\frac{C_5^2C_3^2}2 = 15$ \\
			\hline
			\multirow{2}{*}{$(12345)^{S_5}$}& \multirow{2}{*}{---} & $(12345)^{A_5}$ & $\frac12\frac{P_5}{5} = 12$ \\
			\cline{3-4}
			&  & $(12354)^{A_5}$ & $\frac12\frac{P_5}{5} = 12$ \\
		\end{tabular}
	\end{center}
	\def\arraystretch{1}
	
	Поясним отсутствие элемента из разности централизаторов для последнего класса: $|(12345)^{S_5}| = 24 \centernot\mid 60$, то есть по формуле мощности орбиты это точно не один класс в $A_5$. Остается непосредственно убедиться, что сумма мощностей никаких двух и более классов сопряженности в $A_5$ не является собственным делителем $|A_5| = 60$:
	\begin{itemize}
		\item $H \supseteq (123)^{S_5}$. Тогда $|H| \ge 21$ и, чтобы $|H|$ делило $60$, оно должно стать хотя бы $30$, то есть содержать минимум ещё один класс сопряженности, а это уже точно больше $30$ и такое возможно лишь при $H = G$.
		
		\item Если $H \supseteq ((12)(34))^{S_5}$ и не содержит $(123)^{S_5}$, то $|H| \ge 16$ и нам снова нужен как минимум ещё один из оставшихся классов, но ни сумма с 12, ни сумма с 24 элементами не даёт делитель 60.
		
		\item Остаётся 3 варианта с классами по 12 элементов, но ни один из них не подойдёт по соображениям делимости.
	\end{itemize}
	Значит, в $A_5$ нет подгрупп, отличных от единицы или всей группы, то есть $A_5$ проста.
\end{proof}

\begin{theorem}
	$\forall n \in \N, n \ge 5$ группа $A_n$ --- простая.
\end{theorem}

\begin{proof}
	Проведем теперь индукцию по $n$:
	\begin{itemize}
		\item База $n = 5$: доказано предыдущей теоремой.
		
		\item Переход $n > 5$: Рассмотрим произвольную $H \normal A_n$, $H \neq \{\id\}$. Сделаем переход в 2 шага:
		\begin{enumerate}
			\item Покажем, что $\exists \tau \in H \colon \exists i \in \{1, \ldots, n\},\ \tau(i) = i$ --- в $H$ есть перестановка с неподвижной точкой. Без ограничения общности, пусть $\sigma \in H$ и $\sigma(1) = 2$. Если мы найдём $\sigma' \in H \colon \sigma' \neq \sigma \wedge \sigma'(1) = 2$, то $\tau := \sigma' \circ \sigma^{-1}$. В силу того, что $n \ge 6$, у нас есть $i \notin \{1, 2\}$ такое, что $\sigma(i) = j \notin \{1, 2\}$. Если $\sigma(i) = i = j$, то мы уже нашли требуемое ($\tau = \sigma$). Иначе дополнительно возьмём $k, l \notin \{1, 2, i, j\}$. Утверждается, что подходящей $\sigma'$ будет такая перестановка:
			\[
				\sigma' = \sigma^{(jkl)} = (jkl)^{-1} \sigma (jkl)
			\]
			Действительно, $\sigma' \in H$ из-за нормальности $H$ и $\sigma'(1) = 2$, ибо сопряжение никак не задевает эти значения в силу выбора. С другой стороны, $\sigma' \neq \sigma$, ибо $\sigma'(i) = l \neq j = \sigma(i)$.
			
			\item Без ограничения общности, пусть $\exists \sigma \in H \bs \{\id\} \such \sigma(n) = n$, то есть $\sigma \in A_{n - 1}$. Тогда посмотрим на $L := H \cap A_{n - 1} \normal A_{n - 1}$. По предположению индукции, $L = A_{n - 1}$, поскольку $L$ нетривиальна, и, следовательно, $(123) \in L \subset H \ra \gl(123)^{A_n}\gr = A_n \le H$, то есть $H = A_n$
		\end{enumerate}
	\end{itemize}
\end{proof}

\begin{note}
	Группа $A_4$ "--- уже не простая, поскольку $V_4 \normal A_4$.
\end{note}

\begin{note}
	Можно показать, что если $\F$ "--- поле, $k \ge 2$, то группа $\PSL_k(\F) := \SL_k(\F) / Z(\SL_k(\F))$ проста при $|\F| \ge 4$ или $k \ge 3$ ($Z(\SL_k(\F)) = \{\alpha E: \alpha \in \F, \alpha^k = 1\}$).
\end{note}

\begin{exercise}
	Докажите, что $\PSL_2(\Z_2) \cong S_3$.
\end{exercise}

\begin{note}
	Можно также показать, что $\PSL_2(\Z_3) \cong A_4$ и, кроме того, $\PSL_2(\F_4) \cong \PSL_2(\Z_5) \cong A_5$.
\end{note}

\begin{note}
	На данный момент классификация конечных простых групп \textit{считается} завершенной. Она состоит из конечного числа бесконечных серий и конечного числа так называемых \textit{спорадических} групп, не попадающих ни в одну из серий.
\end{note}

\begin{proposition}
	Пусть $G$ "--- неабелева простая группа. Тогда $G' = G$.
\end{proposition}

\begin{proof}
	$G' \ne \{e\}$ в силу неабелевости $G$, и $G' \normal G$, следовательно, $G' = G$.
\end{proof}

\begin{corollary}
	$\forall n \in \N, n \ge 5:$ группа $A_n$ неразрешима.
\end{corollary}

\begin{theorem}
	Группа $\mathrm{SO}_3 = \{A \in \mathcal{O}_3: \det{A} = 1\}$ "--- простая.
\end{theorem}

\begin{proof}
	Воспользуемся следующим фактом из линейной алгебры: $\forall A \in \mathrm{SO}_3: \exists S \in \mathrm{SO_3}:$
	\[S^{-1}AS = \begin{pmatrix}
		1 & 0 & 0 \\
		0 & \cos\alpha & -\sin\alpha\\
		0 & \sin\alpha & \cos\alpha\\
	\end{pmatrix}\]
	
	Рассмотрим $H \normal \mathrm{SO}_3, H \ne \{E\}$. Пусть $A \in H, A \ne E$ "--- вращение на угол $\alpha$ вокруг некоторой оси, тогда $A^{\mathrm{SO}_3}$ "--- это все вращения на угол $\alpha$ (то есть вокруг любой оси в пространстве), и $A^{\mathrm{SO}_3} \subset H$.
	
	Пусть $B(\beta)$ "--- вращение на угол $\beta$ вокруг оси, подобранной так, что $B(\beta)$ не коммутирует с $A$ при $\beta = \beta_0$, то есть $A^{B(\beta_0)} \ne A$. Положим $C(\beta) := (A^{B(\beta)})^{-1}A$. Заметим, что тогда $C(0) = E$ и $C(\beta_0) \ne E$, причем элементы $C$ "--- непрерывные функции $\beta$. Если $C(\beta)$ "--- это вращение на угол $\mu(\beta)$, то, поскольку след матрицы инвариантен относительно замены базиса и $1 + 2\cos\mu = \tr C(\beta)$, $\mu(\beta)$ "--- тоже непрерывная функция $\beta$. Значит, $H$ содержит хотя бы по одному повороту на каждый угол из отрезка $[0, \mu(\beta_0)]$, $\mu(\beta_0) \ne 0$, и, следовательно, содержит все такие повороты. Но тогда $H$ содержит и все повороты на углы из отрезков $[0, 2\mu(\beta_0)], [0, 3\mu(\beta_0)], \ldots, [0, 2\pi]$. Значит, $H = \mathrm{SO}_3$.
\end{proof}

\begin{note}
	Группа $SO_n$ проста при $n \ge 3$, кроме $n = 4$.
\end{note}