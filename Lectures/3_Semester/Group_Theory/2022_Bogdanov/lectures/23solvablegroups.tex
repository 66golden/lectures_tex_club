\subsection{Разрешимые группы}

\begin{definition}
	Группа $G$ называется \textit{разрешимой}, если $\exists n \in \N: G^{(n)} = \{e\}$. Наименьшее $n \in \N$, для которого это выполнено, называется \textit{степенью разрешимости} $G$.
\end{definition}

\begin{note}
	Разрешимые группы степени 1 "--- это абелевы группы. Разрешимые группы степени 2 часто называют \textit{метаабелевыми}.
\end{note}

\begin{note}
	Если $G$ "--- конечная группа, то последовательность вида $G \ge G' \ge G'' \ge \ldots$ обязательно стабилизируется, но необязательно на $\{e\}$.
\end{note}

\begin{proposition}
	Пусть $G$ "--- разрешимая группа, $H \le G$. Тогда $H$ тоже разрешима.
\end{proposition}

\begin{proof}
	Достаточно заметить, что $H \le G \ra H' \le G' \ra \dotsb \ra H^{(n)} \le G^{(n)} = \{e\}$.
\end{proof}

\begin{theorem}
	Пусть $G$ "--- группа, $K \normal G$. Тогда $G$ разрешима $\Lra$ $K$ и $G / K$ разрешимы.
\end{theorem}

\begin{proof}~
	\begin{itemize}
		\item[$\ra$] Если $G$ разрешима, то $K \le G$ разрешима, и, поскольку при каноническом эпиморфизме $\pi \colon G \to G/K$ выполнено равенство $\pi(G') = (G / K)'$, то, по индукции, $(G / K)^{(n)} = \pi(G^{(n)}) = \{K\}$
		\item[$\la$] Если $K^{(m)} = \{e\}$ и $(G / K)^{(n)} = \{K\}$, то, снова рассматриавя канонический эпиморфизм, получаем, что $\pi(G^{(n)}) = (G / K)^{(n)} = \{K\} \ra G^{(n)} \le \ke\pi = K \ra G^{(n + m)} \le K^{(m)} = \{e\}$
	\end{itemize}
\end{proof}

\begin{corollary}
	Пусть $G$ "--- группа, $K_1, K_2 \normal G$ разрешимы. Тогда группа $K_1K_2$ разрешима.
\end{corollary}

\begin{proof}
	Заметим, что $K_1 \normal K_1K_2$ и, по первой теореме об изоморфизме, $K_1K_2/K_1 \cong K_2/(K_1 \cap K_2)$. Группы $K_1$ и $K_1K_2 / K_1$ разрешимы, поэтому $K_1K_2$ разрешима.
\end{proof}

\begin{corollary}
	Пусть $G$ "--- конечная группа. Тогда в $G$ существует наибольшая по включению разрешимая нормальная подгруппа $K$.
\end{corollary}

\begin{proof}
	Пусть $K_1, \ldots, K_m \normal G$ "--- это все разрешимые нормальные подгруппы в $G$. Тогда, обобщая предыдущее следствие на случай $m$ нормальных подгрупп в $G$ по индукции, получаем, что $K = K_1\dotsm K_m \normal G$ "--- нормальная разрешимая подгруппа, причем $K_1, \ldots, K_m \le K$.
\end{proof}

\begin{example} (Применение доказанных следствий для разрешимости групп)
	\item $A_4$ --- разрешимая подгруппа. Действительно, $V_4 \normal A_4$ --- абелева, а также $A_4 / V_4 \cong \Z_3$ --- тоже абелева.
	
	\item $S_4$ --- разрешимая группа. В самом деле, теперь $A_4 \normal S_4$, причём $S_4 / A_4 \cong \Z_2$ --- абелева группа.
\end{example}

\begin{proposition}
	Если $G$ "--- $p$-группа, то $G$ разрешима.
\end{proposition}

\begin{proof}
	Пусть $|G| = p^n$. Будем вести индукцию по $n$:
	\begin{itemize}
		\item База $n = 1$: если $|G| = p$, то $G \cong \Z_p$ --- циклическая и, в частности, абелева, а потому разрешима.
		
		\item Переход $n > 1$: по теореме о центре $p$-группы уже знаем, что $Z(G) \neq \{e\}$. Если $Z(G) = G$, то группа абелева и всё доказано. Иначе $|Z(G)|, |G / Z(G)| < p^n$ --- это $p$-группы меньшего порядка. Пользуясь предположением индукции, получаем, что они обе разрешимы. Тогда и $G$ разрешима.
	\end{itemize}
\end{proof}

\begin{theorem}
	Пусть $G$ "--- группа. Тогда следующие утверждения эквивалентны:
	\begin{enumerate}
		\item $G$ разрешима
		\item В $G$ существует ряд $G = G_0 \ge G_1 \ge \ldots \ge G_n = \{e\}$ такой, что $\forall i \in \{1, \ldots, n\}\ \ G_i \normal G$ и $G_{i-1}/G_i$ "--- абелева (нормальный ряд с абелевыми факторами)
		\item В $G$ существует ряд $G = G_0 \ge G_1 \ge \ldots \ge G_n = \{e\}$ такой, что $\forall i \in \{1, \ldots, n\}\ \ G_i \normal G_{i-1}$ и $G_{i-1}/G_i$ "--- абелева (субнормальный ряд с абелевыми факторами)
	\end{enumerate}
\end{theorem}

\begin{proof}~
	\begin{itemize}
		\item ($1 \ra 2$) Достаточно расмотреть ряд $G \ge G' \ge \ldots \ge G^{(n)} = \{e\}$, и по свойствам коммутантов все необходимые свойства будут выполнены.
		\item ($2 \ra 3$) Заметим, что нормальный ряд также является и субнормальным: если $\forall i \in \{1, \ldots, n\}\ G_i \normal G$, то $\forall i \in \{1, \ldots, n\}\ G_i \normal G_{i-1}$.
		\item ($3 \ra 1$) Докажем, что $\forall i \in \{0, \ldots, n\}\ G^{(i)} \le G_i$, по индукцией по $i$:
		\begin{itemize}
			\item База $i = 0$: тривиально
			
			\item Переход $i > 0$: поскольку $G_i \normal G_{i - 1}$ и $G_{i - 1} / G_i$ --- абелева, то $(G_{i - 1})' \le G_i$. Это доказывает переход индукции:
			\[
				G^{(i)} = (G^{(i - 1)})' \le (G_{i - 1})' \le G_i
			\]
		\end{itemize}
		Остаётся посмотреть на последний член субнормального ряда: $\{e\} = G_n \ge G^{(n)}$ по доказанной индукции.
	\end{itemize}
\end{proof}

\begin{note}
	В доказательстве выше мы предъявили существование ряда, явно построив его из коммутантов группы. Несложно понять, что наименьшая длина нормального или субнормального ряда c абелевыми факторами --- всегда степень разрешимости $G$.
\end{note}

\begin{note}
	Мы могли бы доказать разрешимость $S_4$, сославшись на соответствующий ряд:
	\[
		\{e\} \normal V_4 \normal S_4
	\]
\end{note}

\begin{corollary}
	Если $G$ --- разрешимая группа, то $G' \neq G$.
\end{corollary}

\begin{theorem}
	Пусть $G$ --- $p$-группа, $|G| = p^n$. Тогда $\forall k \in \N, k \le n\ \exists H \normal G\ |H| = p^k$.
\end{theorem}

\begin{proof}
	Проведём индукцию по $n$:
	\begin{itemize}
		\item База $n = 1$: доказывать нечего
		
		\item Переход $n > 1$: пусть $Z = Z(G) \neq \{e\}$. Пусть $e \neq a \in Z$. Тогда $\ord a = p^l$. Стало быть, $\ord a^{p^{l - 1}} = p$ --- нашли элемент порядка $p$. Обозначим $b = a^{p^{l - 1}}$, тогда $K = \trb{b} \le Z$, $|K| = p$ и $K \normal G$ как подгруппа центра.
		
		Если надо было найти ответ для $k = 1$, то мы это уже сделали. Иначе рассмотрим канонический эпиморфизм $\pi \colon G \to G / K$, тогда $\ol{G} = G / K$, $|\ol{G}| = p^{n - 1}$. По предположению индукции, существует $\ol{L} \normal \ol{G}$, $|\ol{L}| = p^{k - 1}$. По второй теореме об изоморфизме $\ol{L} \mapsto L$, причём $L \normal G$ из-за $\ol{L}$. Остаётся заметить, что $|L| = p^{k - 1} \cdot p = p^k$
	\end{itemize}
\end{proof}

\begin{note}
	Термин <<разрешимости>> пришёл из теории Галуа. Эваристу Галуа удалось построить для уравнений с рациональными многочленами некоторую группу и показать, что уравнение разрешимо в радикалах тогда и только тогда, когда соответствующая группа будет разрешимой.
\end{note}