\subsection{Коммутант группы}

\begin{definition}
	Пусть $G$ "--- группа, $x, y \in G$. \textit{Коммутатором} элементов $x$ и $y$ называется элемент $[x, y] := xyx^{-1}y^{-1}$.
\end{definition}

\begin{proposition} Пусть $G$ "--- группа, $x, y \in G$. Тогда:
	\begin{enumerate}
		\item $xy = [x, y]yx$
		\item $xy = yx \Lra [x, y] = e$
		\item $[x, y]^{-1} = [y, x]$
		\item $\forall g \in G: [x, y]^g = [x^g, y^g]$ 
	\end{enumerate}
\end{proposition}

\begin{proof}~
	\begin{enumerate}
		\item $[x, y]yx = xyx^{-1}y^{-1}yx = xy$
		\item $xy = yx \Lra xyx^{-1}y^{-1} = e \Lra [x, y] = e$
		\item $[x, y]^{-1} = (xyx^{-1}y^{-1})^{-1} = yxy^{-1}x^{-1} = [y, x]$
		\item Данное равенство можно проверить непосредственно, но оно следует из того, что сопряжение с помощью $g \in G$ "--- это автоморфизм $G$
	\end{enumerate}
\end{proof}

\begin{note}
	Пусть $\phi : G \to A$ "--- гомоморфизм групп $G$ и $A$, причем $A$ "--- абелева. Тогда $\phi([x, y]) = [\phi(x), \phi(y)] = e$, поскольку $\phi(x)$ и $\phi(y)$ коммутируют.
\end{note}

\begin{definition}
	Пусть $G$ "--- группа. \textit{Коммутантом} группы $G$ называется $G' := \gl[x, y]: x, y \in G\gr \le G$. Рекурсивно определяется \textit{$n$-ный коммутант} $G^{(n)} := (G^{(n - 1)})' \le G^{(n - 1)}$.
\end{definition}

\begin{definition}
	Пусть $G$ "--- группа, $K, H \le G$. \textit{Взаимным коммутантом} $K$ и $H$ называется $[K, H] := \gl[k, h]: k \in K,  h \in H\gr$.
\end{definition}

\begin{note}
	Определения имеют именно такой вид, потому что возможна ситуация, когда $\{[x, y]: x, y \in G\}$ не является подгруппой в $G$. Отметим также, что $G^{(n)} = [G^{(n - 1)}, G^{(n - 1)}]$.
\end{note}

\begin{proposition}
	Пусть $\phi: G \to H$ "--- гомоморфизм групп $G$ и $H$. Тогда $\phi(G') \le H'$. Более того, если $\phi$ "--- эпиморфизм, то $\phi(G') = H'$.
\end{proposition}

\begin{proof}
	Поскольку $G' = \gl[x, y]: x, y \in G\gr$, то $\phi(G') = \gl\phi([x, y]): x, y \in G\gr = \gl[\phi(x), \phi(y)]: x, y \in G\gr \le \gl[p, q]: p, q \in H\gr$. Если же $\phi$ "--- эпиморфизм, то последнее включение тоже становится равенством.
\end{proof}

\begin{corollary}
	Пусть $G$ "--- группа, $K \normal G$. Тогда $K' \normal G$.
\end{corollary}

\begin{proof}
	Рассмотрим $g \in G$. Сопряжение с помощью $g$ "--- это автоморфизм $I_g: G \to G$, $\forall x \in G: I_g(x) = x^g$. Следовательно, $I_g(K) = K$, то есть $I_g|_K : K \to K$ "--- автоморфизм, и, по утверждению выше, $I_g(K') = K'$, тогда, в силу произвольности $g$, $K' \normal G$.
\end{proof}

\begin{corollary}
	$G' \normal G$, и, по индукции, $\forall n \in \N: G^{(n)} \normal G$.
\end{corollary}

\begin{theorem} Пусть $G$ "--- группа. Тогда:
	\begin{enumerate}
		\item Если $G' \le K \le G$, то $K \normal G$ и $G / K$ "--- абелева.
		\item Если $K \normal G$ и $G / K$ "--- абелева, то $G' \le K$.
	\end{enumerate}
\end{theorem}

\begin{proof}~
	\begin{enumerate}
		\item Рассмотрим канонический эпиморфизм $\pi: G \to G / G'$. Тогда $\{G'\} = \pi(G') = (G / G')'$. Поскольку $G'$ "--- нейтральный элемент в $G / G'$, все коммутаторы в $G / G'$ единичны, поэтому $G / G'$ "--- абелева. По второй теореме об изоморфизме, подгруппе $G' \le K \le G$ соответствует $\ol{K} = K / G' \le \ol{G} = G / G'$, и, более того, $\ol{K} \normal \ol{G} \Lra K \normal G$. Поскольку $\ol{G}$ "--- абелева, $\ol{K} \normal \ol{G}$, значит, и $K \normal G$. Наконец, $G / K \cong \ol{G} / \ol{K}$ "--- абелева группа.
		
		\item Рассмотрим канонический эпиморфизм $\pi: G \to G / K$. Тогда $\pi(G') = (G / K)' = \{K\}$. Значит, $G' \le \ke\pi = K$.
	\end{enumerate}
\end{proof}

\begin{note}
	Согласно теореме выше, $G'$ "--- наименьшая по включению нормальная подгруппа в $G$ такая, что $G / G'$ "--- абелева.
\end{note}

\begin{exercise}
	Пусть $G$ --- группа, $H \normal G$, $K = [G, H]$. Докажите, что $K$ --- это наименьшая нормальная подгруппа в $G$ такая, что $H / K \le Z (G / K)$.
\end{exercise}

\begin{definition}
	Пусть $G$ "--- группа, $M \subset G$. \textit{Нормальной подгруппой, порожденной $M$,} называется $\gl M\gr_{norm} = \bigcap_{H \normal G \atop {M \subset H}} H$
\end{definition}

\begin{note}
	Конечно, $\gl M\gr_{norm} \normal G$ как пересечение некоторого числа нормальных подгрупп.
\end{note}

\begin{proposition}
	Пусть $G$ "--- группа, $M \subset G$. Тогда $\gl M\gr_{norm} = \gl M^G\gr$.
\end{proposition}

\begin{proof}~
	\begin{itemize}
		\item[$\ge$] Если $H \normal G$, $M \subset H$, то, в силу нормальности группы $H$, $M^G \subset H$, тогда $M^G \subset \gl M\gr_{norm}$, и, так как $\gl M\gr_{norm}$ "--- группа, $\gl M^G\gr \le \gl M\gr_{norm}$
		
		\item[$\le$] Заметим, что $\forall g \in G\ \ (M^G)^g = M^G$. Следовательно:
		\[
			\forall g \in G\ \ \trb{M^G} = \bigcap_{H \le G \atop {M^G \subset H}} H = \bigcap_{H \le G \atop {M^G \subset H^{(g^{-1})}}} H = \bigcap_{K \le G \atop {M^G \subset K}} K^g = \Bigg(\bigcap_{K \le G \atop {M^G \subset K}} K\Bigg)^g = \trb{M^G}^g
		\]
		То есть $\gl M^G\gr \normal G$, и поэтому $\gl M\gr_{norm} \le \gl M^G\gr$
	\end{itemize}
	Поскольку доказаны оба включения, $\gl M\gr_{norm} = \gl M^G\gr$.
\end{proof}

\begin{proposition}
	Пусть $G$ "--- группа, $M \subset G$, $G = \gl M\gr$. Тогда $G' = \gl [m_1, m_2] \colon m_1, m_2 \in M\gr_{norm}$.
\end{proposition}

\begin{proof}
	Обозначим $\gl[m_1, m_2] \colon m_1, m_2 \in M\gr_{norm}$ через $H$ и докажем, что $H = G'$.
	\begin{itemize}
		\item[$\le$] Поскольку все коммутаторы лежат в $G'$ и $G' \normal G$, то $H \le G'$
		
		\item[$\ge$] Рассмотрим канонический эпиморфизм $\pi \colon G \to G/ H$, тогда $\forall m_1, m_2 \in M \colon [\ol{m_1}, \ol{m_2}] = \ol{[m_1, m_2]} = e$, тогда, поскольку $G / H = \gl \ol{m} \colon m \in M\gr$, то $G / H$ "--- абелева, и потому $G' \le H$
	\end{itemize}
\end{proof}

\begin{note}
	Центр и коммутант группы $G$ показывают, насколько $G$ <<близка>> к абелевой группе: для абелевой группы $A$ верно, что $Z(A) = A$, $A' = \{e\}$.
\end{note}