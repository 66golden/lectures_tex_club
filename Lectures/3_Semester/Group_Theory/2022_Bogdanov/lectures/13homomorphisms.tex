\subsection{Гомоморфизмы групп и факторгруппа}

\begin{definition}
	Пусть $G, H$ "--- группы. \textit{Гомоморфизмом групп} $G$ и $H$ называется отображение $\phi \colon G \to H$ такое, что
	\[
		\forall g_1, g_2 \in G \quad\phi(g_1g_2) = \phi(g_1)\phi(g_2)
	\]
\end{definition}

\begin{note}
	Дадим определение всех остальных нужных морфизмов:
	\begin{itemize}
		\item \textit{Эпиморфизм} "--- это сюръективный гомоморфизм
		
		\item \textit{Мономорфизм} "--- это инъективный гомоморфизм
		
		\item \textit{Изоморфизм} "--- это биективный гомоморфизм
		
		\item \textit{Эндоморфизм} "--- это гомоморфизм группы в себя
		
		\item \textit{Автоморфизм} "--- это изоморфизм группы в себя
	\end{itemize}
\end{note}

\begin{example}
	Рассмотрим несколько примеров гомоморфизмов групп:
	\begin{enumerate}
		\item Любой изоморфизм групп, в частности, автоморфизм $\phi(g) = g$
		
		\item $\phi: G \to H$, $\forall g \in G: \phi(g) = e$, где $G, H$ "--- произвольные группы
		
		\item Сопряжение при помощи $x \in G$, где $G$ "--- произвольная группа, поскольку $\forall g_1, g_2 \in G: (g_1g_2)^x = g_1^xg_2^x$ (более того, сопряжение "--- это автоморфизм, поскольку существует обратное отображение: $\forall g \in G: \phi^{-1}(g) = g^{x^{-1}}$)
		
		\item $\det: \GL_n(F) \to F^*$, поскольку $\forall A, B \in \GL_n(F): \det(AB) = \det A \det B$
		
		\item $\sgn: S_n \to \Q^*$, поскольку $\forall \sigma, \tau \in S_n: \sgn(\sigma\tau) = \sgn\sigma\sgn\tau$
		
		\item Отображение $\phi: \Z \to \Z_n$ такое, что $\forall a \in \Z: \phi(a) = a + n\Z$, поскольку $\forall a, b \in \Z: \phi(a + b) = (a + b) + n\Z = \phi(a) + \phi(b)$
	\end{enumerate}
\end{example}

\begin{proposition}
	Пусть $\phi \colon G_1 \to G_2$ - гомоморфизм. Тогда верно 2 утверждения:
	\begin{enumerate}
		\item $\phi(e_1) = e_2$ - нейтральный элемент переходит в нейтральный элемент
		
		\item $\forall a \in G_1\ \ \phi(a^{-1}) = \phi(a)^{-1}$
	\end{enumerate}
\end{proposition}

\begin{proof}~
	\begin{enumerate}
		\item \(\phi(e_1) = \phi(e_1^2) = \phi(e_1) \cdot \phi(e_1) \ra \phi(e_1) = \phi(e_1) \cdot \phi(e_1)^{-1} = e_2\)
		
		\item \(\phi(e_1) = \phi(a \cdot a^{-1}) = \phi(a) \cdot \phi(a^{-1}) = e_2 \ra \phi(a^{-1}) = \phi(a)^{-1}\)
	\end{enumerate}
\end{proof}

\begin{definition}
	Пусть $G, H$ "--- группы, $\phi : G \to H$ "--- гомоморфизм $G$ и $H$. Тогда:
	\begin{itemize}
		\item \textit{Образом} $\phi$ называется $\im \phi := \{\phi(g): g \in G\} = \phi(G)$
		\item \textit{Ядром} $\phi$ называется $\ke \phi := \{g \in G: \phi(g) = e\} = \phi^{-1}(e)$
	\end{itemize}
\end{definition}

\begin{note}
	Далее мы часто будем обозначать $\phi(g)$ как $\ol{g}$.
\end{note}

\begin{proposition}
	Пусть $G, H$ "--- группы, $\phi: G \to H$ "--- гомоморфизм, $K := \ke \phi$. Тогда $\forall g \in G: \phi^{-1}(\ol{g}) = gK = Kg$.
\end{proposition}

\begin{proof}
	Пусть $a \in G$. Тогда $a \in \phi^{-1}(\ol{g}) \Leftrightarrow \ol{a} = \ol{g} \Leftrightarrow e \hm= \ol{g}^{-1}\ol{a} = \ol{g^{-1}a} \Leftrightarrow g^{-1}a \hm\in \ke \phi = K \Leftrightarrow a \in gK$. Аналогично доказывается, что $a \in \phi^{-1}(\ol{g}) \Leftrightarrow a \in Kg$.
\end{proof}

\begin{corollary}
	$\phi$ "--- мономорфизм $\Leftrightarrow \ke \phi = \{e\}$.
\end{corollary}

\begin{proposition}
	Пусть $G, H$ "--- группы, $\phi: G \to H$ "--- гомоморфизм. Тогда:
	\begin{enumerate}
		\item $\im \phi \le H$
		\item $\ke \phi \normal G$
	\end{enumerate}
\end{proposition}

\begin{proof}~
	\begin{enumerate}
		\item Если $h_1, h_2 \in \im \phi$, то $h_1 = \ol{g_1}$, $h_2 = \ol{g_2}$, откуда $h_1h_2 = \ol{g_1g_2} \hm\in \im \phi$ и $h_1^{-1} = \ol{g_1^{-1}} \in \im \phi$
		\item Если $g_1, g_2 \in \ke \phi$, то $\ol{g_1} = \ol{g_2} = e$, откуда $\ol{g_1g_2} = e$ и $\ol{g_1^{-1}} = e$, и, более того, $\forall g \in G\ \ gK = Kg = \phi^{-1}(\ol{g})$
	\end{enumerate}
\end{proof}

\begin{note}
	Пусть $H \le G$. Тогда существует гомоморфизм $\phi \colon H \to G$ тривиального вида:
	\[
		\forall h \in H\ \ \phi(h) = h \ra \im \phi = H
	\]
\end{note}

\begin{note}
	Если $G, H$ "--- группы, $\phi: G \to H$ "--- гомоморфизм и $G' \le G$, то $\phi|_{G'} \colon G' \hm\to H$ "--- тоже гомоморфизм, поэтому $\phi(G') \hm= \im \phi|_{G'} \le H$. С другой стороны, если $H' \le H$, то существует гомоморфизм $\psi = \id|_{H'}: H' \to H$ такой, что $H' = \im\psi$.
\end{note}

\begin{definition}
	Пусть $G$ "--- группа, $K \normal G$. Определим операцию на $G / K$ следующим образом:
	\[
		\forall g_1, g_2 \in G \quad g_1K\cdot g_2K \hm= g_1(Kg_2)K = g_1(g_2K)K = g_1g_2K
	\]
\end{definition}

\begin{note}
	Определение опирается на конкретных представителей классов, а поэтому необходимо проверить его корректность, то есть независимость от выбора этих представителей. Пусть $g'_1 \in g_1 K$, $g'_2 \in g_2 K$, тогда нужно показать, что $g'_1 g'_2 K = g_1 g_2 K$. Действительно, $g'_1 = g_1 k_1$ и $g'_2 = g_2 k_2$. Стало быть
	\[
		g'_1 g'_2 e = g_1 g_2 k_1 k_2 \in g'_1 g'_2 K \cap g_1 g_2 K \ra g'_1 g'_2 K = g_1 g_2 K
	\]
\end{note}

\begin{proposition}
	Пусть $G$ "--- группа, $K \normal G$. Тогда $(G / K, \cdot)$ "--- группа.
\end{proposition}

\begin{proof}
	Проверим непосредственно, что множество $G / K$ является группой:
	\begin{itemize}
		\item (Ассоциативность) \[
			\forall g_1K, g_2K, g_3K \in G / K \quad (g_1Kg_2K)g_3K = (g_1g_2g_3)K = g_1K(g_2Kg_3K)
		\]
		\item (Нейтральный элемент) \(
			\exists eK = K \in G / K \such \forall gK \in G / K \quad (gK)K = K(gK) = gK
		\)
		\item (Обратный элемент) \[
			\forall gK \in G / K\ \exists (gK)^{-1} = g^{-1}K \in G / K \such (gK)(gK)^{-1} \hm= (gK)^{-1}(gK) = K
		\]
	\end{itemize}
\end{proof}

\begin{definition}
	Пусть $G$ "--- группа, $K \normal G$. Группа $G / K$ называется \textit{факторгруппой} $G$ \textit{по} $K$.
\end{definition}

\begin{definition}
	\textit{Коммутативной диаграммой} называется рисунок, где отображены алгебраические структуры и функции, отображающие их друг в друга - морфизмы. Коммутативность диаграммы означает, что композиция морфизмов вдоль любого направленного пути зависит только от начал и конца пути (то есть композиции, полученные разными путями, должны давать равный результат).
\end{definition}

\begin{theorem}[Основная теорема о гомоморфизме]~
	\begin{enumerate}
		\item Пусть $G$ "--- группа, $K \normal G$. Тогда $\exists \pi: G \to G / K$ "--- эпиморфизм такой, что $\ke\pi \hm= K$.
		\item Пусть $G, H$ "--- группы, $\phi: G \to H$ "--- гомоморфизм. Тогда $\im \phi \cong G / K$.
	\end{enumerate}
\end{theorem}

\begin{proof}
	Традиционно эту теорему показывают при помощи \textit{коммутативной диаграммы}, которая отражает 2 соотношения: $\phi = \Theta \circ \pi$ и $\pi = \psi \circ \phi$
	\[
	\begin{tikzcd}[row sep = huge]
		G \arrow{rr}{\phi} \arrow[swap]{dr}{\pi} && \im \phi \le H \arrow[dashrightarrow, swap, xshift = -4pt]{dl}{\psi}\\
		& G / K \arrow[dashrightarrow, swap, xshift = 4pt]{ur}{\Theta} &
	\end{tikzcd}
	\]
	\begin{enumerate}
		\item Очевидным образом хочется определить $\pi \colon G \to G / K$ следующим образом:
		\[
			\forall g \in G \quad \pi(g) = gK
		\]
		Это уже эпиморфизм из доказанных выше свойств. Остаётся показать, что ядро действительно совпадает с подгруппой:
		\[
			\ke \pi = \{g \in G \colon \pi(g) = gK = K\} = K
		\]
		
		\item Итак, нам нужно построить $\psi \colon \im \phi \to G / K$ - изоморфизм. Как его сделать, снова интуитивно понятно:
		\[
			\forall g \in G \quad \psi(\ol{g}) = gK
		\]
		Проверим, что полученная функция задаёт изоморфизм:
		\begin{itemize}
			\item (Гомоморфизм)
			\[
				\forall \ol{g_1}, \ol{g_2} \in \im \phi \quad \psi(\ol{g_1} \cdot \ol{g_2}) = \psi(\ol{g_1 g_2}) = g_1 g_2 K = (g_1 K)(g_2 K) = \psi(\ol{g_1}) \psi(\ol{g_2})
			\]
			
			\item (Сюръективность)
			\[
				\forall gK \in G / K \quad \psi(\ol{g}) = gK
			\]
			
			\item (Инъективность) Как уже известно, достаточно проверить тривиальность ядра:
			\[
				\forall \ol{g} \in \im \phi \quad \psi(\ol{g}) = K \ra gK = K \ra g \in K \ra \ol{g} = \ol{e}
			\]
		\end{itemize}
	\end{enumerate}
	Теорема доказана, но убедимся непосредственно в том, что диаграмма коммутативна, то есть $\Theta \circ \pi = \phi$, или $\Theta^{-1} \circ \phi = \psi \circ \phi = \pi$:
	\[
		\forall g \in G: \psi(\phi(g)) = \psi(\ol{g}) = gK = \pi(g)
	\]
\end{proof}

\begin{note}
	<<Гомоморфный образ группы, будь во имя коммунизма изоморфен факторгруппе по ядру гомоморфизма!>>
\end{note}

\begin{note}
	Есть и другая версия стишка, позволяющая доказать недостижимость коммунизма:
	
	<<Гомоморфный образ группы изоморфен факторгруппе по ядру гомоморфизма до победы коммунизма>>
	
	Так как математическая истина вечна, то коммунизм никогда не победит.
\end{note}

\begin{note}
	Эпиморфизм $\pi$ называется \textit{каноническим эпиморфизмом}.
\end{note}

\begin{example}
	В группе перестановок у нас есть функция знака, которая тоже является гомоморфизмом (при $n \ge 2$):
	\[
		\sgn \colon S_n \to \Cm_2 = (\{\pm 1\}, \cdot) \cong \Z_2
	\]
	Верно 2 вещи: $\im \sgn = \Cm_2$ и $\ke \sgn = A_n \normal S_n$. Стало быть
	\[
		S_n / A_n \cong \Cm_2 \cong \Z_2и
	\]
\end{example}

\begin{theorem}[Первая теорема об изоморфизме]
	Пусть $G$ "--- группа, $K \normal G$, $H \le G$. Тогда $HK = KH \le G$, $K \cap H \normal H$ и $HK / K \cong H / (K \cap H)$.
\end{theorem}

\begin{proof}
	Первое утверждение теоремы уже было доказано, поэтому докажем оставшиеся два. Для этого рассмотрим канонический эпиморфизм $\pi: G \hm\to G / K$ и $\forall g \in G$ обозначим $\ol{g} := \pi(g)$.
	
	Пусть $\phi := \pi|_H : H \to G/ K$. Тогда $\ke \phi = \ke\pi \cap H = K \cap H$, откуда $K \cap H \normal H$. $\im \phi = \{\ol{h}: h \in H\} = \{hK: h \hm\in H\} = HK / K$, поскольку $HK / K \hm= \{hkK: h \in H, k \hm\in K\} = \{hK: h \in H\}$. По основной теореме о гомоморфизме, $HK / K \cong H / (K \cap H)$.
\end{proof}

\begin{example}
	Положим $G = S_4$, за $H = S_3$, а $K = V_4 = \{e, (1\ 2)(3\ 4), (1\ 3)(2\ 4), (1\ 4)(2\ 3)\}$ - четверная группа Клейна. Заметим, что $V_4 \normal S_4$ - это так, потому что $V_4$ состоит из двух классов сопряженности. Более того, $HK = S_3 V_4 = S_4$. Стало быть
	\[
		S_4 / V_4 = HK / K \cong H / (H \cap K) = S_3 / \{e\} \cong S_3
	\]
\end{example}

\begin{theorem}[Вторая теорема об изоморфизме, или теорема о соответствии]
	Пусть $G$ "--- группа, $K \normal G$. 
	\begin{enumerate}
		\item Для любой подгруппы $H$ такой, что $K \le H \le G$, можно поставить в соответствие другую, $\ol{H}$:
		\[
			\ol{H} = H / K \le G / K = \ol{G}
		\]
		При этом соответствие $H \mapsto \ol{H}$ - это биекция между между множством подгрупп $H$ и подгрупп в $\ol{G}$.
		
		\item При этом $H \normal G \lra \ol{H} \normal \ol{G}$, и в этом случае
		\[
			G / H \cong \ol{G} / \ol{H}
		\]
	\end{enumerate}
\end{theorem}

\begin{proof}~
	\begin{enumerate}
		\item Рассмотрим канонический эпиморфизм $\pi: G \to G / K = \ol{G}$. Тогда
		\[
			\forall K \le H \le G: \pi(H) =\ol{H} = H / K \le G / K
		\]
		С другой стороны, $\forall L \le \ol{G}\ \ \pi^{-1}(L) = \bigcup_{gK \in L}gK \hm\le G$. Проверим, что $\pi$ осуществляет требуемую биекцию. Действительно, $\pi^{-1}\circ\pi \hm= \id$, поскольку $\forall K \le H \le G: \pi^{-1}(\pi(H)) = H$ ($H$ "--- объединение нескольких левых смежных классов по $K$), и $\pi\circ\pi^{-1} = \id$, поскольку $\forall L \le \ol{G}: \pi(\pi^{-1}(L)) = L$.
		
		\item Если $H \normal G$, то $\forall g \in G: gH = Hg$, поэтому, применяя эпиморфизм $\pi$, получаем, что $\forall \ol{g} \in \ol{G}: \ol{g}\ol{H} = \ol{H}\ol{g}$, то есть $\ol{H} \normal \ol{G}$. Пусть теперь, наоборот, $\ol{H} \normal \ol{G}$. Рассмотрим канонический эпиморфизм $\pi': \ol{G} \to \ol{G} / \ol{H}$. Тогда $\phi := \pi'\circ\pi: G \hm\to \ol{G} \hm\to \ol{G} / \ol{H}$ "--- тоже эпиморфизм, причем $\ke \phi = \pi^{-1}(\pi'^{-1}(\ol{H})) \hm= \pi^{-1}(\ol{H}) = H$. Значит, $H \normal G$, и, по основной теореме о гомоморфизме, $\ol{G} / \ol{H} \cong G / H$.
	\end{enumerate}
\end{proof}

\begin{example}
	Рассмотрим $G = \Z$ и $G \ge n\Z \ge k\Z$, тогда $n \div k$, $\ol{H} = n\Z / k\Z = n \cdot \Z_k$. Вторая теорема утверждает, что
	\[
		\ol{G} / \ol{H} = \Z_k / n\Z_k \cong \Z_n
	\]
\end{example}

\begin{exercise}
	Укажите в явном виде изоморфизм $G / H \hm\to \ol{G} / \ol{H}$ из предыдущей теоремы.
\end{exercise}

\begin{proposition}
	Пусть $G$ "--- конечная группа, $H, K \le G$. Тогда $|HK| = \frac{|H||K|}{|H\cap K|}$.
\end{proposition}

\begin{proof}
	Рассмотрим отображение $\delta: H\times K \to G$ такое, что $\forall h \hm\in H: \forall k \in K: \delta(h, k) = hk$. Тогда $\delta(h_1, k_1) = \delta(h_2, k_2) \hm\Leftrightarrow h_1k_1 = h_2k_2 \hm\Leftrightarrow h_2^{-1}h_1 = k_2k_1^{-1} = x \in H \cap K$, то есть $h_1 = h_2x$, $k_1 = x^{-1}k_2$. Значит, $|HK||H \cap K| = |H \times K| = |H||K|$.
\end{proof}