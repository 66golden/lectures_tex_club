\section{Задание групп}

\subsection{Свободные группы}

\begin{definition}
	Пусть $F_n$ "--- группа, $F_n = \gl f_1, \dotsc, f_n\gr$. $F_n$ называется \textit{свободной} со \textit{свободными образующими} $f_1, \dotsc, f_n \hm\in F_n$, если для любой группы $G$ выполнено, что $\forall g_1, \dotsc, g_n \in G: \exists \phi: F_n \to G$ "--- гомоморфизм: $\forall i \in \{1, \dotsc, n\}: \phi(f_i) = g_i$.
\end{definition}

\begin{note}
	Если такой гомоморфизм существует, то он единственен, поскольку $F_n$ порождена элементами $f_1, \dotsc, f_n$.
\end{note}

\begin{note}
	Аналогичным образом можно определить и свободные группы с бесконечным количеством свободных образующих.
\end{note}

\begin{theorem}
	Свободная группа $F_n$ со свободными образующими $f_1, \dotsc, f_n$ существует.
\end{theorem}

\begin{proof}
	Считая $\{f_1, \dotsc, f_n, f_1^{-1}, \dotsc, f_n^{-1}\}$ алфавитом, положим $F_n := \{w \hm\in \{f_1, \dotsc, f_n, f_1^{-1}, \dotsc, f_n^{-1}\}^*: \forall i \in \{1, \dotsc, n\}: w \text{ не содержит }f_if_i^{-1}, f_i^{-1}f_i\}$. Определим операцию на $F_n$ следующим образом: если $w_1, w_2 \in F_n$, то сократим взаимно обратные элементы алфавита с конца $w_1$ и начала $w_2$, получив $w_1'$ и $w_2'$, и положим $w_1\cdot w_2 := w_1'w_2'$. Докажем, что $F_n$ "--- действительно группа:
	\begin{itemize}
		\item (Нейтральный элемент) $\exists \epsilon \in F_n: \forall w \in F_n: w\cdot \epsilon = \epsilon\cdot w = w$.
		\item (Обратный элемент) Пусть $w \in F_n, w = f_{i_1}^{\alpha_1}\dotsc f_{i_k}^{\alpha_k}, \alpha_1, \dotsc, \alpha_k \hm\in \{\pm1\}$, тогда $\exists w^{-1} \hm= f_{i_k}^{-\alpha_k}\dotsc f_{i_1}^{-\alpha_1} \in F_n : \forall w \in F_n: w\cdot w^{-1} \hm= w^{-1} \cdot w = \epsilon$.
		\item (Ассоциативность) Докажем, что $\forall a, b, c \in F_n: (ab)c = a(bc)$ индукцией по $|b|$. База индукции, $b = \epsilon$, тривиальна. Рассмотрим отдельно случай, когда $|b| = 1$. Непосредственный перебор всех случаев (в операциях $ab$ и $bc$ нет сокращений, одно или два сокращения) позволяет убедиться, что и в этом случае $a(bc) \hm= (ab)c$. Пусть теперь $|b| > 1$, тогда $b = xy$, $|x| < |b|$, $|y| < |b|$. Тогда, по предположению индукции, $(ab)c \hm= (a(xy))c = ((ax)y)c \hm= (ax)(yc) = a(x(yc)) = a((xy)c) = a(bc)$.
	\end{itemize}
	
	Проверим теперь, что $F_n$ "--- свободная группа. Пусть $G$ "--- произвольная группа, $g_1, \dotsc, g_n \in G$. Определим $\phi: F_n \to G$ следующим образом: $\phi(f_{i_1}^{\alpha_1}\dotsc f_{i_k}^{\alpha_k}) = g_{i_1}^{\alpha_1}\dotsc g_{i_k}^{\alpha_k}$. Тогда, по определению, $\forall i \in \{1, \dotsc, n\}: \phi(f_i) = g_i$. Наконец, $\phi$ "--- гомоморфизм, поскольку $\forall w_1, w_2 \hm\in F_n$ в записях $\phi(w_1w_2)$ и $\phi(w_1)\phi(w_2)$ сокращаются одни и те же элементы.
\end{proof}

\begin{theorem}
	Пусть $F_n$ "--- свободная группа со свободными образующими $f_1, \dotsc, f_n$, $G_n$ "--- свободная группа со свободными образующими $g_1, \dotsc, g_n$. Тогда существует изоморфизм $\phi: F_n \to G_n$ такой, что $\forall i \in \{1, \dotsc, n\}: \phi(f_i) = g_i$.
\end{theorem}

\begin{proof}
	По определению свободной группы, существует гомоморфизм $\phi: F_n \hm\to G_n$ такой, что $\forall i \in \{1, \dotsc, n\}: \phi(f_i) = g_i$, и, аналогично, существует гомоморфизм $\psi: G_n \to F_n$ такой, что $\forall i \in \{1, \dotsc, n\}: \psi(g_i) = f_i$. Тогда $\psi \circ \phi = \id_{F_n}$, $\phi \circ \psi = \id_{G_n}$, поэтому эти гомоморфизмы биективны и взаимно обратны.
\end{proof}

\begin{example}
	Рассмотрим $F_1 = \{f_1^n: n \in \Z\}$. Легко видеть, что $F_1 \cong \Z$.
\end{example}

\begin{note}
	При $n \ge 2$ группа $F_n$ "--- уже неабелева, например, потому, что $f_1f_2 \ne f_2f_1$.
\end{note}

\begin{exercise}
	Пусть $G = \SL_2(\Z[x])$. Рассмотрим следующую подгруппу в $G$:
	\[F = \left\gl \begin{pmatrix}1&0\\x&1\end{pmatrix}, \begin{pmatrix}1&x\\0&1\end{pmatrix} \right\gr \le G\]
	
	Докажите, что $F$ "--- свободная группа с соответствующими свободными образующими.
\end{exercise}

\begin{proof}
	Обозначим образующие $F$ через $f'_1$, $f'_2$ и рассмотрим гомоморфизм $\phi: F_2 \to F$ такой, что $\phi(f_1) = f'_1$, $\phi(f_2) = f'_2$. Сюръективность $\phi$ очевидна. Покажем, что $\ke\phi = \{\epsilon\}$. Для этого заметим следующее:
	\[\forall k \in \Z: \begin{pmatrix}1&0\\x&1\end{pmatrix}^k = \begin{pmatrix}1&0\\kx&1\end{pmatrix}, \begin{pmatrix}1&x\\0&1\end{pmatrix}^k = \begin{pmatrix}1&kx\\0&1\end{pmatrix}\]
	
	Пользуясь равенствами выше, можно убедиться индукцией по длине слова, что $\forall w \hm\in F_2, |w| \ge 1: \phi(w) \ne E$.
\end{proof}

\begin{definition}
	Пусть $w \in F_n$, $G$ "--- группа, $g_1, \dotsc, g_n \in G$. Тогда \textit{значение $w$ в группе $G$} "--- это $w(g_1, \dotsc, g_n) := \phi(w)$, где $\phi$ "--- гомоморфизм $F_n$ и $G$ такой, что $\forall i \in \{1, \dotsc, n\}: \phi(f_i) = g_i$.
\end{definition}