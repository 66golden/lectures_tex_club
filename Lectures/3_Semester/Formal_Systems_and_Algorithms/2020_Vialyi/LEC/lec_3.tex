\section{Лекция 3}
\subsection{Несовместные и противоречивые множества гипотез}
\begin{definition} \tit{Противоречивое множества формул} \\
Множество формул называентся \tit{противоречивым}, если из него можно одновременно вывести $A, \neg A$ 
\end{definition}
\begin{remark}
Противоречивость это синтаксическое свойство
\end{remark}
\begin{definition} \tit {Несовместность множества формул} \\
Множество формул является  \tit{несовместным}, если нет такого набора значений переменных, на котором все формулы обращяются в истину
\end{definition}
\begin{remark}
несовместность это семантическе свойство
\end{remark}
\begin{remark}
Из несовмместного множества формул семантически следует любая формула, очевидно в силу определения семантического следствия
\end{remark}
\begin{lemma}
если любая формула выводится из множества $\Ga$, то $\Ga$ несовместна
\end{lemma}
\beginproof
От проивного, пусть $\Ga$ совместна, то есть набор $\al$, что все формулы из $\Ga$ = 1, но тогда выберим произвольную $f \in \Ga$, тогда $\neg x^{\al}$(отрицание альфа--версии переменной) не выводится из $\Ga$
\begin{lemma} \label{lem:consistent_set_and_union}
$\Ga$ непротиворечиво, из $\Ga$ невозможно вывести ни $A$, ни $\neg A$, тогда $\Ga \cup \figbr{\neg A} $ -- непротиворечиво
\end{lemma}
Докажем от противного, пусть $\Ga \cup \figbr{\neg A} $ -- противоречиво, тогда выводятся $B , \neg B$, тогда по т. дедукции: \begin{equation}
 \left\{    \begin{aligned}
    &\Ga \vdash \neg A \to B \\
    &\Ga \vdash \neg A \to \neg B
    \end{aligned} \right. \Rar \Ga \vdash \neg A
\end{equation}
\subsection{Полные множества формул}
\begin{definition} \tit{полное множество формул} \\
$\Ga$ -- полное, если $\forall A : \Ga \vdash A \lor \Ga \vdash \neg A$
\end{definition}
\begin{theorem} \label{th:embedding a consistent into a complete} \tit{вложение непротиворечивого в полное} \\
Любое непротиворечивое множество $\Ga$ можно вложить в полное непротиворечивое множество
\end{theorem}
\beginproof
 Занумеруем все формулы: $F_1, F_2 ,....$, (множество всех формул счетно как подмножество всех слов над конечным алфавитом). По этой нумерации построим последовательность множеств формул: $\Ga_0 = \Ga \subseteq \Ga_1 \subseteq \Ga_2 ... \subseteq $\\
\begin{equation}
    \Ga_k = \begin{cases}
        &\Ga_{k-1}, \text{ если } \Ga_{k-1} \vdash F_{k} \lor \Ga_{k-1} \vdash  \neg F_{k} \\
        & \Ga_{k-1} \cup F_{k} \text{ в противном случае}
    \end{cases}
\end{equation}
любое $\Ga_{k}$ непротиворечиво в силу леммы \ref{lem:consistent_set_and_union} \\
Рассмотрим $\wt{\Ga} = \bigcup_{n=0}^{\INF} \Ga_n$,  $\wt{\Ga}$ непротиворечиво, обоснуем это. Предположим противное, существует такое $A$, что $\wt{\Ga}  \vdash A \land \wt{\Ga} \vdash \neg A$, т.к. вывод это конечный набор строк, то существует такое $\Ga_n$, что : $\Ga_n \vdash A \land \Ga_n \vdash \neg A$ -- противоречие.\\
Итак, $\wt{\Ga}$ непротиворечиво, полно (действительно, каждая $F_k$ или включена в $\wt{\Ga}$, или выводится из какого-то $\Ga_n \subseteq \wt{\Ga}$) и содержит $\Ga$ \qed
\subsection{Главная теорема}
\begin{lemma}
Полное непротиворечивое множество формул совместно
\end{lemma}
\beginproof
Пусть $\Ga$  --полное непротиворечивое множество формул. Построим булевый вектор $\al$ следующим образом: Если $\Ga \vdash x_i$, то $\al_i = 1$, иначе $\al_i = 0$, тогда по построению $\Ga \vdash x_i^{\al}, ~ \forall x_i$.\\
Рассмотрим произвольное $F \in \Ga$, тогда по лемме Кльмара \ref{th:Kalmar} : $\Ga \vdash F^{\al}$, но с другой стороны, $\Ga \vdash F$, значит $F(\al) = F^{\al}(\al) = 1$, в силу произвольности выбора $F$, получаем что на наборе $\al$ все формулы в $\Ga$ истины, таким образом $\Ga$ -- совместно \qed

\begin{theorem} \label{th:main_theorem} \tit{главная теорема} \\
$\Ga$ -- непротиворечива $\LRar$ $\Ga$ -- совместна
\end{theorem}
\beginproof Сначала докажем, что противоречивость влечет несовместность: \\
От противного: пусть $\Ga$ -- противоречива, но совместна, тогда $\Ga \vdash A, \Ga \vdash \neg A$ для какой-то формулы $A$. Подставим набор $\al$ такой, что все формулы $\Ga$ обращаются в 1, но тогда $\neg A(\al) =  A(\al) = 1$ -- противоречие\\
теперь докажем, что несовместность влечет противоречивость \\
Воспользуемся теоремой о полонении \ref{th:embedding a consistent into a complete}, пополним $\Ga$, пусть пополненное $\Ga = \wt{\Ga}$, $\wt{\Ga}$ совместно в силу выше доказанной леммы, значит и $\Ga$ совместно \qed
\subsection{Завершение доказательства основной теоремы}
\begin{theorem}  \label{th:completeness} \tit{теорема полноты}
\begin{equation}
    \Ga \vDash A \Rar \Ga \vdash A
\end{equation}
\end{theorem}
\begin{proposition} \label{pr:th_completeness_color_th_main}
теорема полноты выводится из главной
\end{proposition}
\tit{Обоснование:} Пусть $\Ga \vDash A$, значит $\Ga \cup \neg A$ --несовместно, тогда по теорема \ref{th:main_theorem} $\Ga \cup \neg A$ -- противоречиво: воводятся $B, \neg B$ одновременно, тогда по теореме дедукции (теорема \ref{th:deduction}) получаем $\Ga \vdash  \neg A \to B, ~ \Ga \vdash \neg A \to \neg B$. В таком случае множно построить $\Ga \vdash A$, применив третью схему аксиом  к $\neg A \to B, ~ \neg A \to \neg B$: $\br{\neg A \to \neg B} \to \br{ \br{\neg A \to B} \to A }$, применяя дважды MP получаем $A$ \qed \\
Объединение теоремы корректности \ref{th:correct} и теоремы полноты \ref{th:completeness}  означают равносильность симантического и синтаксического следствия, что и называется основной теоремой \\
\subsection{Метод доказательства невыводимости, контромодели}
Построим новую формульную систему : оставим первую и вторую схему аксиом, плюс правило вывода $MP$, определение формул тоже.\\
В этой ФС уже не все тавтологии выводимы, докажем, что третья схема аксиом не выводится.\\
Переопределим операцию отрицания :
\begin{table}[!h]
    \centering
    \begin{tabular}{c|c}
        $A$ & $\neg A$  \\
        \hline
         0 & 0 \\
         1 & 1 \\
    \end{tabular}
    \caption{Переопределение отрицания}
    \label{tab:my_label}
\end{table}

Заметим, что модели аксиом все так же остались тавтологиями, MP так же не нарушается, таким образом, все формулы, которые мы можем вывести в данной ФС это только тавтологии, по теореме корректности \ref{th:correct} \\
Рассмотри третью схему аксиом : $\br{B \to A} \to  \br{\br{B \to A} \to B}$. Пусть $A  = 0, B = 0$, легко проверить, что данное  выражение с переопределенной операцией отрицания не тавтология, таким образом третья схема аксиом не выводими из первых двух, таким образом мы построили контромодель\\
Так же можно доказать это же утверждение для первых двух схем аксиом, это можно посмотреть в учебнике лектора, сами контромодели более сложны\\
Рассмотрим тавтологию, закон Пирса : $A \to B \to A \to A$, оказывается и эта тавтология так же не выводима в рассматриваемой ФС.\\
Контро-модель строить уже сложнее, очевидно, что необходимо каким-то образом переопределять операцию импликации.\\
Будем рассматривать теперь трех значную логику $0 , 1/2, 1$ : <<никогда>>, <<иногда>>, <<всегда>>. Значчения ипликации для $0, 1$ сохраняются, таким образом, некоторые значения имплицаций:
\begin{equation}
    \begin{aligned}
    1 \to x = x \\
    x \to 1 = 1 \\
    x \to x = 1 \\
    0 \to x = 1 \\
    1/2 \to 0 = 0
    \end{aligned}
\end{equation}
Этой таблицы значений нам будет достаточно.\\
Будем называть и-тавтологиями, такие функции $F$ над трехзначной алгеброй, всюду равные $1$\\
Покажем, что закон Пирса не и-тавтология: $A = 1/2, B = 0$:
$\br{ \br{1/2 \to 0} \to 0 } \to 0 \hm{=} \br{0 \to 0} \to 0 = 1 \to 0 = 0 $\\
Осталось показать, что формулы задающий тавтлогии и-тавтологии и правило MP сохраняется, в таком случае по теореме корректности \ref{th:correct}, все выводиме формулы будут и-тавтологии. Это утверждение легко проверяется составлениями таблиц истинности
