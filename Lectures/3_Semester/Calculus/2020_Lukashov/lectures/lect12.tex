Дифференциальную форму валентности $n$ в $D\subset\R^n$ можно записать в виде $\Omega(x)=\alpha(x)V_{e^0}$, так как пространство полилинейных форм валентности $n$ в $n$-мерном пространстве одномерное.

\begin{Def}
	Интегралом от формы $\Omega(x)=\alpha(x)V_{e^0}$ по области $D\subset\R^n$ называется $\int\limits_{D}\Omega=\int\limits_{D}\alpha(x)d\mu(x)$, где $\alpha$ --- суммируемая на $D$.
\end{Def}

\begin{prop}
	Если $\varphi:G\to D$ --- гладкое отображение области $G\subset \R^n$ на $D$, такое, что $\det \varphi'(u)\ne 0\ \ \forall u\in G$ (то есть $\varphi$ --- неособое), то $\int\limits_{G}\varphi^*\Omega=\pm\int\limits_{D}\Omega$
\end{prop}

\begin{proof}
	\begin{multline*}
		\varphi^*\Omega(u)(H_1,\ldots, H_n)=\Omega(\varphi(u))(\varphi'(u)H_1, \ldots, \varphi'(u)H_n)=\\=\alpha(\varphi(u))V_{e^0}(\varphi'(u)H_1, \ldots, \varphi'(u)H_n)=\alpha(\varphi(u))dx_0^1\wedge\ldots\wedge dx^n_0(\varphi'(u)H_1, \ldots, \varphi'(u)H_n)=\\=\alpha(\varphi(u))\det((\varphi'(u)H_i)^j)=\alpha(\varphi(u))\det\varphi'(u)det(H^j_i)=\alpha(\varphi(u))\det\varphi'(u)\widetilde{V_{e^0}}(H_1,\ldots,H_n).
	\end{multline*}
	
	$\int\limits_{G}\varphi^*\Omega=\int\limits_{G}\alpha(\varphi(u))\det\varphi'(u)d\mu(u)$.
	
	$\int\limits_{D}\Omega=\int\limits_{D}\alpha(x)d\mu(x)=[x=\varphi(x)]=\int\limits_{G}\alpha(\varphi(u))|\det\varphi'(u)|d\mu(u)=\pm\int\limits_{G}\alpha(\varphi(u))\det\varphi'(u)d\mu(u)$
\end{proof}

\begin{Def}
	Стандартным кубом $K$ называется множество $\{0<x^i_0<1\},i=1,\ldots,n$ в ортонормированном базисе $e^0_1,\ldots,e_n^0$.
	
	Цепь стандартных кубов --- это линейная комбинация $\Pi=\sum\limits_{j=1}^mn_jK_j$, где $n_j\in \Z, K_j$ --- стандартные кубы.
\end{Def}

\begin{Def}
	Интеграл от $n$-формы $\Omega$ по цепи кубов определяется как $\int\limits_{\Pi}\Omega=\sum\limits_{j=1}^m n_j\int\limits_{K_j}\Omega$
\end{Def}

Граница стандартного куба --- цепь стандартных $(n-1)$-мерных кубов.

$\partial K=\sum\limits_{j=1, \alpha=0,1}^m\varepsilon_j^\alpha K_{j,\alpha},\ \ K_{j,\alpha}=\{0<x_0^i<1;i\ne j, x_0^j=\alpha\}$.

Правило ориентации граней куба: грань $K_{j,\alpha}$ ориентирована положительным базисом пространства $E_j=\{(x_0^1,\ldots,x_0^{j-1},x_0^{j+1},\ldots,x_0^n)\}$ таким, что дополнив его первым вектором, являющимся нормалью к грани $K_{j,\alpha}$, выходящей из куба $K$, получим положительный базис $\R^n$.












