\subsection{Основные свойства интеграла Лебега}

\begin{anote}
	Если мы, например, заявляем, что $f \colon E \to \R$ --- суммируемая функция на $E$, то надо понимать, что тогда $f$ обязана своими свойствами подходить под хотя бы одно определение суммируемой функции. В частности, для всех классов функций, для которых мы вводили понятие суммирования, мы требовали измеримость и определенность на $E$. Стало быть, если говорится, что $f$ суммируема на $E$, то она по условию уже измерима и определена на этом множестве. Более того, само $E$ имеет конечную меру.
\end{anote}

\begin{theorem} (Линейность и монотонность интеграла Лебега)
	Имеют место 2 свойства:
	\begin{enumerate}
		\item (Линейность) Если $f_1$ и $f_2$ суммируемы на $E$, то $\forall c_1, c_2 \in \R$ функция $c_1 f_1 + c_2 f_2$ тоже суммируема на $E$, причём
		\[
			\int_E (c_1f_1(x) + c_2f_2(x))d\mu(x) = c_1\int_E f_1(x)d\mu(x) + c_2\int_E f_2(x)d\mu(x)
		\]
		
		\item (Монотонность) Если функции $f_1, f_2$ суммируемы на $E$ и $\forall x \in E\ \ f_1(x) \le f_2(x)$, то
		\[
			\int_E f_1(x)d\mu(x) \le \int_E f_2(x)d\mu(x)
		\]
	\end{enumerate}
\end{theorem}

\begin{proof}~
	\begin{enumerate}
		\item Для доказательства линейности, достаточно рассмотреть доказать 2 факта:
		\begin{itemize}
			\item Линейность при $c_1 = c_2 = 1$. Разберём случаи:
			\begin{enumerate}
				\item $f_1, f_2 \colon E \to \R$ --- ограниченные. Тогда по эквивалентному условию суммирования
				\[
					\forall \eps > 0\ \exists P_1(E), P_2(E) \such U(P_1, f_1) - L(P_1, f_1) < \eps;\ \ U(P_2, f_2) - L(P_2, f_2) < \eps
				\]
				Положим $P := P_1 \cup P_2$ и увидим следующую цепочку неравенств:
				\[
					L(P, f_1) + L(P, f_2) \le L(P, f_1 + f_2) \le U(P, f_1 + f_2) \le U(P, f_1) + U(P, f_2)
				\]
				Отсюда понятным образом $f_1 + f_2$ тоже суммируется на $E$. Так как интегралы зажаты между соответствующими суммами (из-за суммируемости они конечны), то предельный переход даёт нужное равенство.
				
				\item $f_1, f_2 \colon E \to \sxR_{\ge 0}$ --- неотрицательные и неограниченные функции. Тогда предыдущее доказательство сработает, если мы покажем, что множество точек с бесконечным значением функции $f_1 + f_2$ имеет меру ноль. Действительно, имеет место равенство $E_0 = E_{1, 0} \cup E_{2, 0}$:
				\[
					\mu(E_0) = \mu(E_{1, 0} \cup E_{2, 0}) \le \mu(E_{1, 0}) + \mu(E_{2, 0}) = 0
				\]
				Чтобы воспользоваться этим фактом формально, нужно уже брать не просто измельчение $P := P_1 \cup P_2$, но и получить по нему $P'$, в котором все точки $E_0$ отсоединены от остальных частей разбиения, а далее мы пользуемся уже полученным выше фактом и дополненным умножением, дело в шляпе.
				
				\item $f_1, f_2 \colon E \to \R$ --- произвольные функции. Тогда $f_1 = f_1^+ - f_1^-$, $f_2 = f_2^+ - f_2^-$ --- все эти функции суммируются на $E$. Тогда
				\[
					f_1 + f_2 = f_1^+ - f_1^- + f_2^+ - f_2^- = (f_1 + f_2)^+ - (f_1 + f_2)^-
				\]
				Стало быть, $f_1 + f_2$ суммируется на $E$. Раз так, то по определению
				\[
					\int_E (f_1 + f_2)(x)d\mu(x) = \int_E (f_1 + f_2)^+ d\mu(x) - \int_E (f_1 + f_2)^- d\mu(x)
				\]
				Более того, для интегралов справа по предыдущему пункту уже доказана дистрибутивность интеграла по сложению (это и есть то, что получается при $c_1 = c_2 = 1$), поэтому
				\begin{multline*}
					\int_E (f_1 + f_2)^+ d\mu(x) - \int_E (f_1 + f_2)^- d\mu(x) =
					\\
					\int_E f_1^+ d\mu(x) + \int_E f_2^+ d\mu(x) - \int_E f_1^- d\mu(x) - \int_E f_2^- d\mu(x) =
					\\
					\ps{\int_E f_1^+ d\mu(x) - \int_E f_1^- d\mu(x)} + \ps{\int_E f_2^+ d\mu(x) - \int_E f_2^- d\mu(x)}
				\end{multline*}
				Выражения в скобках по определению равны интегралам от $f_1$ и $f_2$ соответственно.
			\end{enumerate}
		
			\item Вынесение константы за знак интеграла. Для $c = 0$ всё тривиально верно, а для остальных значений сделаем разбор случаев:
			\begin{enumerate}
				\item $f \colon E \to \R$ --- ограниченная. Тогда заметим следующие соотношения при любом разбиении $P(E)$:
				\begin{align*}
					&{\forall c > 0\ L(P, cf) = cL(P, f);\ U(P, cf) = cU(P, f)}
					\\
					&{\forall c < 0\ L(P, cf) = cU(P, f);\ U(P, cf) = cL(P, f)}
				\end{align*}
				Отсюда по эквивалентному свойству суммирования следует суммирование $cf$ на $E$, ну и в силу соотношений между суммами Дарбу-Лебега равенство между значениями интегралов тоже тривиально.
				
				\item $f \colon E \to \sxR_{\ge 0}$ --- неотрицательная неограниченная функция. Всё сказанное выше будет верным для измельчения произвольного разбиения, в котором мы выделяем $E_0$, тем самым убирая бесконечные слагаемые.
				
				\item $f \colon E \to \R$ --- произвольная функция. Тогда мы просто замечаем, что $(cf)^+ = cf^+$ и $(cf)^- = cf^-$ для $c \in \R$.
			\end{enumerate}
		\end{itemize}
	
		\item Здесь классический приём с разбором разных $f_1$:
		\begin{itemize}
			\item Если $f_1 = 0$, то интеграл тривиально принимает значение ноль, $f_2$ оказывается неотрицательной функцией, для которой неравенство очевидно.
			
			\item Если $f_1 \neq 0$, то сведём ситуацию к первой, рассмотрев функцию $\forall x \in E\ f(x) = f_2(x) - f_1(x) \ge 0$
		\end{itemize}
	\end{enumerate}
\end{proof}

\begin{definition}
	Пусть $f \colon E \to \sxR_{\ge 0}$ --- неотрицательная функция. Тогда определим функцию $f_{[N]}(x) \colon E \to \R$, называемую \textit{срезкой}, следующим образом:
	\[
		f_{[N]}(x) = \System{
			&{f(x),\ f(x) \le N}
			\\
			&{N,\ f(x) > N}
		}
	\]
	или же $f_{[N]}(x) = min(f(x), N)$
\end{definition}

\begin{note}
	$f_{[N]}(x)$ является неотрицательной и ограниченной функцией.
\end{note}

\begin{proposition}
	Если $f \colon E \to \sxR_{\ge 0}$ --- неотрицательная измеримая функция, то $f_{[N]}$ тоже измерима.
\end{proposition}

\begin{proof}
	По определению одному из эквивалентных определений измеримости
	\[
		\forall c \in \R\ \ f^{-1}\rsi{c; +\infty} \text{ --- измеримо по Лебегу}
	\]
	Разберём разные случаи в контексте срезки $g := f_{[N]}$:
	\begin{itemize}
		\item $c < N$. Тогда $f(x) > c \Lra g(x) > c$, поэтому $g^{-1}(c; +\infty) = f^{-1}\rsi{c; +\infty}$ --- измеримое множество.
		
		\item $c \ge N$. Тогда $g^{-1}(c; +\infty) = \emptyset$, а для пустого множества верно всё.
	\end{itemize}
\end{proof}

\begin{theorem} (Интеграл Лебега от неотрицательных, неограниченных и измеримых функций как предел интегралов от срезок)
	Если $f \colon E \to \sxR_{\ge 0}$ --- неотрицательная, неограниченная и измеримая функция, то
	\[
		\int_E f(x)d\mu(x) = \lim_{N \to \infty} \int_E f_{[N]}(x)d\mu(x)
	\]
\end{theorem}

\begin{note}
	Интегралы слева и справа существуют по основной теореме.
\end{note}

\begin{proof}
	Для начала заметим, что $\forall N \in \N, x \in E\ \ f_{[N]}(x) \le f_{[N + 1]}(x)$. По свойству интеграла Лебега это даёт утверждение
	\[
		\forall N \in \N\ \int_E f_{[N]}(x)d\mu(x) \le \int_E f_{[N + 1]}(x)d\mu(x)
	\]
	Коль скоро это верно, у последовательности интегралов существует предел $I$ при $N \to \infty$, $I \in \R \cup \{+\infty\}$.
	
	Остаётся показать равенство интеграла и предела. Имеет место неравенство $\forall N \in \N\ \  f_{[N]}(x) \le f(x)$, а поэтому $I \le \int_E f(x)d\mu(x)$. Предположим, что в последнее неравенство строгое. В частности, это означает, что $I < +\infty$, то есть предел конечен. Сделаем разбор случаев:
	\begin{itemize}
		\item $\mu(E_0) > 0$ Тогда, рассмотрим разбиение $P_1(E) = \{E_0, E \bs E_0\}$. Имеет место следующая цепочка неравенств:
		\[
			\int_E f_{[N]}(x)d\mu(x) \ge L(P_1, f_{[N]}) \ge N\mu(E_0)
		\]
		Если устремить $N \to \infty$, то получим $I = +\infty$, что противоречит предположению.
		
		\item $\mu(E_0) = 0$ В таком случае $L(P, f) = \sum_{k = 1}^\infty m_k \mu(E_k)$ --- ряд, у которого все слагаемые конечные неотрицательные числа. В силу предположения, мы можем найти такое $P$, что $L(P, f) > I$. Отсюда можно получить ещё одну вещь:
		\[
			\exists K \in \N \such \sum_{k = 1}^K m_k \mu(E_k) > I
		\]
		Выберем любое $N > \max \{m_1, \ldots, m_k\}$. Тогда, если $m'_k$ --- это инфинумы разбиения $P$ по функции $f_{[N]}$, то $\forall k \in \range{1}{K}\ m'_k = m_k$. Остаётся посмотреть соответствующее разбиение $P_2 \colon E = \ps{\bscup_{k = 0}^K E_k} \sqcup \ps{E \bs \bscup_{k = 0}^K E_k}$ и то же неравенство, что и в предыдущем случае:
		\[
			\int_E f_{[N]}(x)d\mu(x) \ge L(P_2, f_{[N]}) \ge \sum_{k = 1}^K m_k \mu(E_k) > I
		\]
		Получили противоречие в силу монотонного возрастания последовательности интегралов, сходящейся к $I$.
	\end{itemize}
\end{proof}

\begin{lemma} (Признак суммируемости)
	Если заданы произвольные функции $f, F \colon E \to \sxR$ со следующими свойствами:
	\begin{enumerate}
		\item $f$ измерима на $E$
		
		\item $f$ неограничена на $E$
		
		\item $F$ неотрицательна
		
		\item $F$ суммируема на $E$
		
		\item $\forall x \in E\ \ |f(x)| \le F(x)$
	\end{enumerate}
	Тогда $f$ тоже суммируема на $E$.
\end{lemma}

\begin{proof}
	Заметим, что суммирование $f$ равносильно суммированию $|f|$. Действительно, если $f = f^+ - f^-$, то $|f| = f^+ + f^-$. Притом $|f|$ --- это неотрицательная функция, можно рассмотреть её срезки:
	\[
		\forall N \in \N\ \forall x \in E\ \ |f|_{[N]}(x) \le |f|(x) \le F(x)
	\]
	При этом срезки уже интегрируемы по Лебегу. По монотонности интеграла Лебега получается следующее:
	\[
		\forall N \in \N\ \int_E |f|_{[N]}(x)d\mu(x) \le \int_E F(x)d\mu(x) < +\infty
	\]
	По теореме о представлении неотрицательной функции получим требуемое:
	\[
		\int_E |f|(x)d\mu(x) = \lim_{N \to \infty} \int_E |f|_{[N]}(x)d\mu(x) \le \int_E F(x)d\mu(x) < +\infty
	\]
\end{proof}

\begin{lemma} (Конечная аддитивность интеграла Лебега)
	Если $f \colon E \to \sxR$ --- произвольная функция, $E = \bscup_{k = 1}^K E_k$ --- произвольное конечное разбиение $E$ на измеримые по Лебегу множества $E_k$ конечной меры. Тогда
	\[
		\text{f суммируется на $E$} \Lolra \forall k \in \N\ \ \text{f суммируется на $E_k$}
	\]
	При этом имеет место равенство:
	\[
		\int_E f(x)d\mu(x) = \sum_{k = 1}^K \int_{E_k} f(x)d\mu(x)
	\]
\end{lemma}

\begin{proof}
	Чтобы доказать это свойство, необходимо и достаточно доказать эквивалентность для разбиения $E = E_1 \sqcup E_2$. Тогда тот факт, что $f$ измерима на $E \Lra f$ измерима на $E_1$ и $E_2$ тривиален. Стало быть, по основной теореме у нас есть эквивалентность суммирований. Остаётся показать, что равенство между интегралами выполнено. Сделаем это путём разбора случаев:
	\begin{enumerate}
		\item $f \colon E \to \R$ --- ограниченная функция. Пусть $P(E)$ --- это разбиение Лебега, соответсвующее разбиению $Q$. $P_1, P_2$ - тоже разбиения Лебега, полученные из $P$ для множества $E_1, E_2$ соответственно. По основной теореме
		\[
			\forall i \in \{1, 2\} \quad \lim_{\Delta Q \to 0} S(P_i, f, \{t_{i, k}\}) = \int_{E_i} f(x)d\mu(x)
		\]
		Если за наборы $\{t_{i, k}\}$ взять те точки, которые соответствуют точным верхним граням \textcolor{red}{А так точно можно?}, то получим такие равенства:
		\[
			\forall i \in \{1, 2\}\ \ \lim_{\Delta Q \to 0} L(P_i, f) = \lim_{\Delta Q \to 0} U(P_i, f) = \int_{E_i} f(x)d\mu(x)
		\]
		Аналогичные верны и для $L(P, f), U(P, f)$ с интегралом по $E$. Остаётся заметить неравенство на верхние и нижние суммы:
		\[
			L(P, f) \le L(P_1, f) + L(P_2, f) \le U(P_1, f) + U(P_2, f) \le U(P, f)
		\]
		Предельный переход $\Delta Q \to 0$ в нём даст требуемое.
		
		\item $f \colon E \to \sxR_{\ge 0}$ --- неотрицательная, неограниченная функция. В таком случае, мы можем пользоваться доказанным свойством для срезок $f$:
		\[
			\forall N \in \N \quad \int_E f_{[N]}(x)d\mu(x) = \sum_{i = 1}^2 \int_{E_i} f_{[N]}(x)d\mu(x)
		\]
		В силу теоремы о представлении неотрицательных, неограниченных и измеримых функций как предел интегралов от срезок, имеем:
		\begin{multline*}
			\int_E f(x)d\mu(x) = \lim_{N \to \infty} \int_E f_{[N]}(x)d\mu(x) = \lim_{N \to \infty} \ps{\int_{E_1} f_{[N]}(x)d\mu(x) + \int_{E_2} f_{[N]}(x)d\mu(x)} =
			\\
			\lim_{N \to \infty} \int_{E_1} f_{[N]}(x)d\mu(x) + \lim_{N \to \infty} \int_{E_2} f_{[N]}(x)d\mu(x) = \int_{E_1} f(x)d\mu(x) + \int_{E_2} f(x)d\mu(x)
		\end{multline*}
		
		\item $f \colon E \to \sxR$ --- произвольная функция. Тогда мы воспользуемся её представлением в виде двух неотрицательных, для которых уже всё доказано (ну и плюс линейность интеграла Лебега может понадобится, это тоже есть).
	\end{enumerate}
\end{proof}

\begin{theorem}
	Помимо линейности и монотонности, у интеграла Лебега есть ещё и такие свойства:
	\begin{enumerate}
		\item (Интегрируемость постоянной) Если $f \colon E \to \{c\},\ c \in \R$, то эта функция суммируема, причём имеет место равенство:
		\[
			\int_E cd\mu(x) = c\mu(E)
		\]
		
		\item ($\sigma$-аддитивность интеграла Лебега) Пусть $f \colon E \to \sxR$ --- произвольная функция, $E = \bscup_{k = 1}^\infty E_k$ - произвольное счётное разбиение $E$ на измеримые по Лебегу множества $E_k$ конечной меры. Тогда
		\[
			f \text{ суммируется на $E$} \Lolra \forall k \in \N\ f \text{ суммируется на $E_k$}
		\]
		При этом выполнено равенство:
		\[
			\int_E f(x)d\mu(x) = \sum_{k = 1}^\infty \int_{E_k} f(x)d\mu(x)
		\]
		
		\item (Абсолютная непрерывность интеграла Лебега) Если $f \colon E \to \sxR$ --- произвольная функция, суммирующаяся на $E$, то
		\[
			\forall \eps > 0\ \exists \delta > 0 \such \forall e \subseteq E,\ \mu(e) < \delta\ \ \ \mo{\int_e f(x)d\mu(x)} < \eps
		\]
	\end{enumerate}
\end{theorem}

\begin{proof}~
	\begin{enumerate}
		\item Для любого разбиения $P$ верно равенство $L(P, c) = U(P, c) = c\mu(E)$
		
		\item Для эквивалентности суммируемости нам необходима и достаточна соответствующая эквивалентность по измеримости, она у нас тривиальным образом есть. Что же касается равенства интегралов, то нужно расписать всё отдельно:
		\begin{itemize}
			\item $f \colon E \to \R$ --- ограниченная функция. В силу того, что мера Лебега $\sigma$-аддитивна, имеет место равенство:
			\[
				\mu(E) = \sum_{k = 1}^\infty \mu(E_k)
			\]
			В частности, все слагаемые неотрицательные и ряд сходится. Это даёт нам возможность оценить <<хвост>> ряда:
			\[
				\forall \eps > 0\ \exists K \in \N \such \sum_{k = K + 1}^\infty \mu(E_k) < \eps
			\]
			Обозначим $R_K := \bscup_{k = K + 1}^\infty E_k$, тогда в утверждении выше написано $\mu(R_k) < \eps$. По конечной аддитивности интеграла Лебега:
			\[
				\int_E f(x)d\mu(x) = \sum_{k = 1}^K \int_{E_k} f(x)d\mu(x) + \int_{R_K} f(x)d\mu(x)
			\]
			Коль скоро $f$ ограничена, то $\exists M > 0 \such  \forall x \in E\ |f(x)| < M$. Тогда по монотонности интеграла Лебега:
			\[
				\mo{\int_{R_k} f(x)d\mu(x)} \le \int_{R_k} |f(x)|d\mu(x) < M\eps
			\]
			Устремляя $\eps$ к нулю и $K$ при необходимости в бесконечность, получаем требуемое. 
			
			\item $f \colon E \to \sxR_{\ge 0}$ --- неотрицательная неограниченная функция. Воспользуемся для срезок тем же, что и в предыдущем пункте:
			\[
				\forall N \in \N \quad \int_E f_{[N]}(x)d\mu(x) = \sum_{k = 1}^K \int_{E_k} f_{[N]}(x)d\mu(x) + \int_{R_K} f_{[N]}(x)d\mu(x)
			\]
			В силу неотрицательности срезки, получаем неравенство
			\[
				\forall N \in \N \quad \int_E f_{[N]}(x)d\mu(x) \ge \sum_{k = 1}^K \int_{E_k} f_{[N]}(x)d\mu(x)
			\]
			Предельный переход $N \to \infty$ позволяет заменить срезку на саму $f$. Так как $K$ может быть сколь угодно большим, получаем одно из двух итоговых неравенств:
			\[
				\sum_{k = 1}^\infty \int_{E_k} f(x)d\mu(x) \le \int_E f(x)d\mu(x)
			\]
			Более того, правый интеграл конечен по определению суммирования. Значит, ряд интегралов слева сходится. Чтобы получить второе неравенство, воспользуемся уже доказанной $\sigma$-аддитивностью для срезок и монотонностью интеграла Лебега по отношению к срезкам ($\forall x \in E\ f_{[N]}(x) \le f(x)$):
			\[
				\forall N \in \N \quad \int_E f_{[N]}(x)d\mu(x) = \sum_{k = 1}^\infty \int_{E_k} f_{[N]}(x)d\mu(x) \le \sum_{k = 1}^\infty \int_{E_k} f(x)d\mu(x)
			\]
			Устремляя $N$ в бесконечность, получаем второе неравенство и, как следствие, итоговое равенство.
			
			\item $f \colon E \to \sxR$ --- произвольная функция. Суммируемость $f$ эквивалентна суммируемости $f^+$ и $f^-$ по определению, этим всё доказано. Далее мы берём равенство для $f^+$ и вычитаем из него аналогичное для $f^-$, получая тем самым требуемое.
		\end{itemize}
	
		\item Абсолютную непрерывность будем также доказывать разбором случаев:
		\begin{itemize}
			\item $f \colon E \to \R$ --- ограниченная функция. Пусть $\forall x \in E\ |f(x)| \le M$. Тогда
			\[
				\mo{\int_e f(x)d\mu(x)} \le M\mu(e)
			\]
			Положив $\delta = \eps / (M + 1)$, добьёмся требуемого
			
			\item $f \colon E \to \sxR_{\ge 0}$ --- неотрицательная неограниченная функция. По теореме о представлении интеграла такой функции в виде предела интегралов срезок:
			\[
				\exists N \in \N \such \int_E (f(x) - f_{[N]}(x))d\mu(x) \le \frac{\eps}{2}
			\]
			Тогда аналогичное неравенство верно и для $\forall e \subseteq E$: $\int_e (f(x) - f_{[N]}(x))d\mu(x) \le \frac{\eps}{2}$. Остаётся увидеть следующее соотношение:
			\[
				\mo{\int_e f(x)d\mu(x)} = \int_e f_{[N]}xd\mu(x) + \int_e (f(x) - f_{[N]}(x))d\mu(x) \le \int_e f_{[N]}(x)d\mu(x) + \frac{\eps}{2}
			\]
			Если взять $\delta = \eps_{2N + 1}$, то победа у нас в кармане (по неравенству из предыдущего пункта в контексте срезки).
			
			\item $f \colon E \to \R$ --- произвольная функция. Заметим такое неравенство, которого достаточно для доказательства:
			\[
				\mo{\int_e f(x)d\mu(x)} \le \int_e |f(x)|d\mu(x) = \int_e f^+(x)d\mu(x) + \int_e f^-(x)d\mu(x)
			\]
			Если левой части надо $\eps$, то для интегралов $f^+$ и $f^-$ требуем $\delta_1, \delta_2$ при $\eps / 2$, тогда $\delta = \min(\delta_1, \delta_2)$.
		\end{itemize}
	\end{enumerate}
\end{proof}

\begin{corollary}
	Пусть даны $f, g \colon E \to \sxR$ --- произвольные функции. Если $f(x) = g(x)$ почти всюду на $E$, то одна из них суммируема тогда и только тогда, когда суммируема другая. Более того, их интегралы по $E$ совпадают.
\end{corollary}

\begin{proof}
	Для начала заметим, что любая функция, заданная на множестве нулевой меры, всегда суммируема на нём и интеграл по этому множеству равен нулю.
	
	Не умаляя общности, считаем $f$ суммируемой на $E$. Обозначим $e := \{x \in E \such f(x) \neq g(x)\}$. Тогда, $\mu(e) = 0$ и мы запишем конечную аддитивность интеграла Лебега от $f$ на разбиении $E = (E \bs e) \sqcup e$:
	\[
		\int_E f(x)d\mu(x) = \int_{E \bs e} f(x)d\mu(x) + \underbrace{\int_e f(x)d\mu(x)}_{0}
	\]
	Из введённого определения, $\forall x \in E \bs e\ f(x) = g(x)$. Стало быть, $g$ суммируема на $E \bs e$. Однако, то же можно сказать и про её суммируемость на $e$. По конечной аддитивности получаем, что $g$ также суммируема на $E$.
\end{proof}

\begin{corollary}
	Если $f \colon E \to \sxR_{\ge 0}$ --- функция и выполнены следующие условия:
	\begin{enumerate}
		\item $f$ - неотрицательная
		
		\item $f$ - измеримая на $E$
		
		\item $\int_E f(x)d\mu(x) = 0$
	\end{enumerate}
	Тогда $f(x) = 0$ почти всюду на $E$.
\end{corollary}

\begin{proof}
	От противного. Пусть это не так. Тогда мера этого множества положительна. Заметим, что имеет место предел:
	\[
		\{x \in E \such f(x) > 0\} = \lim_{m \to \infty} \set{x \in E \such f(x) > \frac{1}{m}}
	\]
	По свойствам пределов, должно существовать $m_0 \in \N$ такое, что мера соответствующего множества $B$ тоже положительна. Следовательно
	\[
		\int_E f(x)d\mu(x) \ge \frac{1}{m_0}\mu(B) > 0
	\]
	Противоречие условию.
\end{proof}

\begin{definition}
	Функции $f, g \colon E \to \ole{\R}$ называются \textit{эквивалентными}, если $f(x) = g(x)$ почти всюду на $E$.
\end{definition}

\begin{proposition}
	Эквивалентность функций задаёт отношение эквивалентности.
\end{proposition}

\begin{proof}
	Тривиально.
\end{proof}

\begin{corollary}
	Множество классов эквивалентностей суммируемых на $E$ функций обозначается за $L_1(E)$ и является линейно нормированным пространством с нормой следующего вида:
	\[
		\|f\|_{L_1(E)} := \int_E |f(x)|d\mu(x)
	\]
	Корректность определения нормы получается из последнего следствия.
\end{corollary}

\begin{note}
	Отныне и далее мы будем работать с целым таким классом функций точно так же, как и с одной. Нулевая функция --- это естественно представитель класса функций, которые почти всюду равны нулю на $E$.
\end{note}

\begin{definition}
	Пусть $\{f_m\}_{m = 1}^\infty \subseteq L_1(E)$, $f \in L_1(E)$. Тогда говорят, что $f_m \to f$ в $L_1(E)$ (самые важные слова в конце, ибо обозначение совпадает с поточечной сходимостью), если сходится предел следующих метрик:
	\[
		\lim_{m \to \infty} \|f_m - f\|_{L_1(E)} = \lim_{m \to \infty} \int_{E} |f_m(x) - f(x)|d\mu(x) = 0
	\]
\end{definition}