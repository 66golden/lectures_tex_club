\section{Кратные интегралы}

\subsection{Определение кратного интеграла}

\begin{designation}
	В этом параграфе, если не обговорено обратного, обозначение $E$ используется для подмножества $\R^n$ \textbf{конечной меры}.
\end{designation}

\begin{reminder}
	Ограниченность функции $f$ на $E$ подразумевает, что $f$ определена в каждой точке из $E$. Аналогично, если пишется $f \colon E \to \R$ и не обговорено обратного.
\end{reminder}

\begin{definition}
	\textit{Разбиением $P$ множества $E$} называется его произвольное представление в виде конечного числа непересекающихся измеримых по Лебегу (Жордану) множеств. Обозначение такое:
	\[
		P \colon E = \bscup_{i = 1}^N E_i
	\]
\end{definition}

\begin{anote}
	Когда мы не обращаемся к конкретному виду разбиения, я буду обозначать разбиение $P$ над $E$ как $P(E)$.
	
	Дополнительно будет полезно знать, что разбиение $P$ можно воспринимать как $P = \{E_i\}$
\end{anote}

\begin{designation}
	Если $f$ определена на $E$, то инфинумы и супремумы функции $f$ на разбиении $P(E)$ будут обозначаться так:
	\begin{align*}
		&{M_k := \sup_{x \in E_k} f(x)}
		\\
		&{m_k := \inf_{x \in E_k} f(x)}
	\end{align*}
	где $E_k$ - это какой-то элемент разбиения $P$.
	
	Аналогично $M$ и $m$ - это супремум и инфинум функции по всему $E$.
\end{designation}

\subsubsection*{Определение интеграла Лебега (Римана) в случае ограниченной функции}

\begin{note}
	В рамках этой части, если не оговорено обратного, $f \colon E \to \R$ --- ограниченная функция.
\end{note}

\begin{definition}
	Пусть $P \colon E = \bscup_{i = 1}^N E_i$. Тогда \textit{нижней суммой Дарбу (Дарбу-Лебега)} мы будем называть следующую величину:
	\[
		L(P, f) := \sum_{k = 1}^N m_k \cdot \mu(E_k)
	\]
	Аналогично есть \textit{нижняя сумма Дарбу-Римана}:
	\[
		L_R(P, f) := \sum_{k = 1}^N m_k \cdot \jm(E_k)
	\]
\end{definition}

\begin{definition}
	Пусть $P \colon E = \bscup_{i = 1}^N E_i$. Тогда \textit{верхней суммой Дарбу (Дарбу-Лебега)} мы будем называть следующую величину:
	\[
	U(P, f) := \sum_{k = 1}^N M_k \cdot \mu(E_k)
	\]
	Аналогично есть \textit{верхняя сумма Дарбу-Римана}:
	\[
	U_R(P, f) := \sum_{k = 1}^N M_k \cdot \jm(E_k)
	\]
\end{definition}

\begin{note}
	Чтобы не дублировать один и тот же текст, далее мы будем сжимать определения для Лебега и Римана. Чтобы получить из определения Лебега определение Римана, надо вместо соответствующих частей исходного текста читать то, что в скобках. С формулами аналогично мерам: $U_\comR, L_\comR$ и так далее.
\end{note}

\begin{definition}
	\textit{Верхним интегралом Лебега (Римана) функции $f$ по множеству $E$} называется инфинум по всем разбиениям верхних сумм Дарбу-Лебега (-Римана):
	\[
		\ole{I}_\comR := \inf_{P(E)} U_\comR(P, f)
	\]
\end{definition}

\begin{definition}
	\textit{Нижним интегралом Лебега (Римана) функции $f$ по множеству $E$} называется супремум по всем разбиениям нижних сумм Дарбу-Лебега (-Римана):
	\[
		\ule{I}_\comR := \sup_{P(E)} L_\comR(P, f)
	\]
\end{definition}

\begin{definition} (Кратный интеграл через суммы Дарбу)
	Если нижний и верхние интегралы Лебега (Римана) $f$ по $E$ существуют и они равны друг другу, то их общее значение называется \textit{интегралом Лебега (Римана) функции $f$ по множеству $E$}. Обозначается так:
`	\[
		\ole{I} = \ule{I} = \int_E f(x) d\mu(x); \quad \ole{I}_R = \ule{I}_R = \int_E f(x)dx
	\]
\end{definition}

\begin{definition}
	Пусть есть разбиения $P_1 \colon E = \bscup_{i = 1}^{N_1} E_{i, 1}$ и $P_2 \colon E = \bscup_{i = 1}^{N_2} E_{i, 2}$. Тогда \textit{измельчением} $P_1$ и $P_2$ мы назовём следующее разбиение:
	\[
		P_1 \cup P_2 \colon E = \bscup_{i = 1}^{N_1} \bscup_{j = 1}^{N_2} (E_{i, 1} \cap E_{j, 2})
	\]
\end{definition}

\begin{proposition}
	С обозначениями из определения измельчения имеют смысл такие неравенства:
	\[
		L_\comR(P_1, f) \le L_\comR(P_1 \cup P_2, f) \le U_\comR(P_1 \cup P_2, f) \le U_\comR(P_1, f)
	\]
	На концах, естественно, можно поставить и $P_2$ вместо $P_1$.
\end{proposition}

\begin{proof}
	\textcolor{red}{Когда-нибудь к сессии. Или пришлите мне в личку.}
\end{proof}

\begin{proposition} \label{commonEquivLebeg}
	$f$ интегрируема по Лебегу (Риману) по $E$ тогда и только тогда, когда верно следующее условие:
	\[
		\forall \eps > 0\ \exists P(E) \such U_\comR(P, f) - L_\comR(P, f) < \eps
	\]
\end{proposition}

\begin{proof}
	\textcolor{red}{Возможно подойдёт доказательство из одномерного случая, но я не уверен. Когда-нибудь к сессии}
\end{proof}

\begin{proposition}
	Произвольная функция $f \colon [a; b] \to \R$ интегрируема по Риману на $[a; b]$ в старом смысле тогда и только тогда, когда она интегрируема по Риману по $[a; b]$ в новом смысле.
\end{proposition}

\begin{proof}
	\textcolor{red}{Когда-нибудь обязательно.}
\end{proof}

\begin{proposition}
	Если произвольная функция $f \colon V \to \R, V \subset \R$ интегрируема по Риману по $V$, то она интегрируема по Лебегу по $V$.
\end{proposition}

\begin{proof}
	\textcolor{red}{Когда-нибудь обязательно.}
\end{proof}

\begin{definition}
	Пусть $f \colon E \to \R$ --- ограниченная и измеримая на измеримом множестве $E \subset \R^n$ конечной меры функция. \textit{Разбиением Лебега}, соответствующим разбиению $Q = \{m = y_0 < \ldots < y_q = M\}$, называется разбиение $P(E)$ следующего вида:
	\[
		P \colon E = \ps{\bscup_{i = 1}^{q - 1} \{x \in E \colon f(x) \in \lsi{y_{i - 1}; y_i}\}} \sqcup \{x \in E \colon f(x) \in [y_{q - 1}; y_q]\}
	\]
\end{definition}

\begin{proposition} \label{uminl-le-dqe}
	Для разбиения Лебега множества $E$ и функции $f$, ограниченной и измеримой на измеримом множестве $E \subset \R^n$ конечной меры, верно следующее неравенство:
	\[
		U(P, f) - L(P, f) \le \Delta Q \mu(E)
	\]
	где $\Delta(Q)$ --- диаметр разбиения $Q$.
\end{proposition}

\begin{proof}
	Распишем левую часть по определению:
	\[
		U(P, f) - L(P, f) = \sum_{i = 1}^q (M_i - m_i)\mu(E_i) \le \sum_{i = 1}^q \Delta Q\mu(E_k) = \Delta Q \mu(E)
	\]
\end{proof}

\begin{definition}
	\textit{Интегральной суммой Лебега (Римана)} называется величина $S(P, f, \{t_k\})$, где $P \colon E = \bscup_{k = 1}^N E_k$, $t_k \in E_k$:
	\[
		S(P, f, \{t_k\}) := \sum_{k = 1}^N f(t_k) \jlm(E_k)
	\]
\end{definition}

\begin{theorem} (Основная теорема об интеграле Лебега для ограниченных и измеримых функций)
	Если $f \colon E \to \R$ и верны условия:
	\begin{enumerate}
		\item $E \subset \R^n$ --- измеримое по Лебегу множество конечной меры
		
		\item $f$ ограничена
		
		\item $f$ измерима на $E$
	\end{enumerate}
	Тогда $f$ интегрируема по Лебегу по множеству $E$, причём
	\[
		\int_E f(x)d\mu(x) = \lim_{\Delta Q \to 0} S(P, f, \{t_k\})
	\]
	где $P$ --- это разбиение Лебега. Равенство в кванторной форме имеет вид:
	\begin{multline*}
		\forall \eps > 0\ \exists \delta > 0 \such \forall Q, \Delta(Q) < \delta;\ \ \forall P \colon E = \bscup_{k = 1}^N E_k;\ \ \forall \{t_k\}, t_k \in E_k
		\\
		\mo{\int_E f(x)d\mu(x) - S(P, f, \{t_k\})} < \eps
	\end{multline*}
\end{theorem}

\textcolor{red}{Кажется, мы можем выкинуть требование измеримости функции}

\begin{reminder}
	Если поменять в определении предела последнее неравенство на нестрогое, то получится эквивалентное определение.
\end{reminder}

\begin{proof}
	Если $\mu(E) = 0$, то все верхние и нижние суммы равны нулю, и утверждение теоремы очевидно. Далее будем считать $\mu(E) > 0$.
	\begin{enumerate}
		\item Существование интеграла. Для этого воспользуемся утверждением $\ref{uminl-le-dqe}$. Для любого $\eps > 0$ нам достаточно взять разбиение $Q$, что $\Delta Q < \delta = \eps / \mu(E)$, ибо тогда
		\[
			U(P, f) - L(P, f) \le \Delta Q \mu(E) = \eps
		\]
		Тогда, по утверждению \ref{commonEquivLebeg} $f$ интегрируема по Лебегу.
		
		\item Сходимость предела к интегралу. Заметим такое соотношение:
		\[
			\forall P(E), \{t_k\} \quad L(P, f) \le S(P, f, \{t_k\}) \le U(P, f)
		\]
		Стало быть, предел сходится к общему значению верхнего и нижнего интегралов в силу зажатости между верхней и нижней суммами.
	\end{enumerate}
\end{proof}

\subsubsection*{Определение интеграла Лебега в случае неотрицательной функции}

\begin{note}
	В рамках этой части, если не оговорено обратного, $f \colon E \to \R_{\ge 0} \cup \{+\infty\}$ --- неотрицательная функция, определенная в каждой точке и способная принимать значение $+\infty$.
\end{note}

\begin{note}
	Чтобы ввести интеграл здесь, нам потребуются уже счётные разбиения (как обычное, так и лебегово) и соответствующие суммы:
	\begin{itemize}
		\item Счётное простое разбиение имеет вид $P \colon E = \bscup_{i = 1}^\infty E_i$
		
		\item Счётное разбиение Лебега, соответствующее, естественно, счётному разбиению $Q \colon m = y_0 < \ldots < \ldots$, будет выглядеть как $P \colon E = \bscup_{i = 1}^\infty \underbrace{\{x \in E \colon f(x) \in \lsi{y_{i - 1}; y_i}\}}_{E_i}$
		
		\item Нижние и верхние суммы Дарбу-Лебега имеют соответственно виды 
		\[
			L(P, f) = \sum_{i = 0}^\infty m_i \mu(E_i); \quad U(P, f) = \sum_{i = 0}^\infty M_i \mu(E_i)
		\]
		где $E_0 := \{x \in E \colon f(x) = +\infty\}$, $m_0 = M_0 = +\infty$ Дополнительно мы примем соглашение, что $(+\infty) \mu(E_0) = 0$ \textcolor{red}{Почему это корректно? Достаточно ли последующих теорем, утверждений?}
		
		\item Интегральная сумма $S(P, f, \{t_k\})$ принимает вид
		\[
			S(P, f, \{t_k\}) = \sum_{k = 0}^\infty f(t_k) \mu(E_k)
		\]
	\end{itemize}
	Все утверждения (не теоремы!) остаются верными для таких расширений предыдущих определений.
\end{note}

\begin{theorem} (Основная теорема для неотрицательных и измеримых функций)
	Если дана функция $f \colon E \to \R$ и верны условия:
	\begin{enumerate}
		\item $E$ --- измеримое по Лебегу множество конечной меры
		
		\item $f$ --- неотрицательная функция (то есть можно написать, что $f \colon E \to \R_{\ge 0} \cup \{+\infty\}$)
		
		\item $f$ --- измеримая на $E$ функция
	\end{enumerate}
	Тогда $f$ интегрируема по Лебегу на $E$, причём интеграл равен пределу интегральных сумм:
	\[
		\int_E f(x)d\mu(x) = \lim_{\Delta Q \to 0} S(P, f, \{t_k\})
	\]
\end{theorem}

\begin{proof}~
	\begin{enumerate}
		\item Существование интеграла. Работает ровно то же самое, что написано в основной теореме для ограниченных функций.
		
		\item Сходимость предела к интегралу. В конечном случае всё тривиально, а в бесконечном $\ule{I} = \sup_{P(E)} = +\infty$. Так как следующее неравенство всё ещё в силе:
		\[
			L(P, f) \le S(P, f, \{t_k\}) \le U(P, f)
		\]
		то предел интегральных сумм тоже будет в бесконечности и соответственно равен интегралу.
	\end{enumerate}
\end{proof}

\begin{definition}
	Если $f$ --- неотрицательная функция на $E$ и интеграл $\int_E f(x)d\mu(x)$ конечен, то $f$ называется \textit{суммируемой на $E$}.
\end{definition}

\begin{proposition}
	$f$ суммируется на $E$ тогда и только тогда, когда верно условие:
	\[
		\forall \eps > 0\ \exists P(E) \such U(P, f) - L(P, f) < \eps
	\]
\end{proposition}

\subsubsection*{Определение интеграла Лебега для произвольных функций}

\begin{reminder}
	$\ole{\R} := \R \cup \{\pm \infty\}$
\end{reminder}

\begin{definition}
	Пусть $f \colon E \to \ole{\R}$ --- произвольная функция. Тогда, мы положим по определению за $f_+$ и $f_-$ следующие функции:
	\begin{align*}
		&{f_+(x) := \max(f(x), 0)}
		\\
		&{f_-(x) := \min(-f(x), 0)}
	\end{align*}
\end{definition}

\begin{proposition}
	Если дана функция $f \colon E \to \ole{R}$, то $f = f_+ - f_-$
\end{proposition}

\begin{proof}
	Тривиально.
\end{proof}

\begin{definition}
	Если есть произвольная функция $f \colon E \to \ole{\R}$, причём выполнены следующие условия:
	\begin{enumerate}
		\item $E$ --- измеримое по Лебегу множество конечной меры
		
		\item $f$ --- измеримая на $E$
	\end{enumerate}
	Тогда, $f$ называется \textit{интегрируемой по Лебегу по $E$}, если хотя бы одна из функций $f_+, f_-$ суммируется на $E$. Значением интеграла по определению считается следующее:
	\[
		\int_E f(x)d\mu(x) := \int_E f_+(x)d\mu(x) - \int_E f_-(x)d\mu(x)
	\]
\end{definition}

\begin{definition}
	Если в рамках последнего определения и $f_+$, и $f_-$ оказались суммируемыми на $E$, то говорят, что \textit{$f$ суммируется на $E$}.
\end{definition}

\begin{example} (Интеграл Лебега функции Дирихле)
	Напомним, что из себя представляет $D(x)$:
	\[
		D(x) := \System{
			&{1,\ x \in \Q}
			\\
			&{0,\ x \in \R \bs \Q}
		}
	\]
	Найдём интеграл от этой функции на отрезке $[0; 1]$ (она точно интегрируема по основной теореме для ограниченных функций): $m = 0$, $M = 1$. Для лебегова разбиения $P$ точек со значениями из $(0; 1)$ нет, если же берётся часть $E_1 = \{x \in E \colon f(x) \in \lsi{0; y_1}\}$, у любой точки из $E_1$ значение 0, а для $E_q = \{x \in E \colon f(x) \in [y_{q - 1}; 1]\}$ значение в любой точке будет 1, но $\mu(\Q \cap [0; 1]) = 0$. Следовательно
	\[
		\int_{[0; 1]} D(x)d\mu(x) = \lim_{\Delta Q \to 0} S(P, f, \{t_k\}) = 0
	\]
\end{example}
