\begin{note}
	В теории функций одного переменного мы доказали замечательный результат, формулу Ньютона-Лейбнщица:
	\[
		\int_{[a; b]} F'(x)dx = F(b) - F(a)
	\]
	Неспроста подинтегральная функция написана в таком виде, ведь $dF(x) = F'(x)dx$, и это равенство можно интерпретировать как дифференциальную форму. Следующая часть текущего параграфа будет направлена на доказательство \textit{теоремы Стокса-Пуанкаре}, заявляющаяя в нашем случае такое равенство:
	\[
		\int_{[a; b]} dF = \int_{\vdelta([a; b])} F
	\]
\end{note}

\begin{definition}
	\textit{Цепью $k$-мерных кубов} называется формальная целочисленная комбинация $\sum_{i = 1}^N \alpha_i K_i$, где $\alpha_i \in \Z$, а $K_i$ --- соответствующий $k$-мерный куб в $\R^n$, $n \ge k$.
\end{definition}

\begin{anote}
	Формальность подразумевает, что это ни в коем случае не сумма множеств, а скорее просто кортеж.
\end{anote}

\begin{note}
	Грань $n$-мерного куба является просто $(n - 1)$-мерным кубом, живущим в гиперплоскости $\{x \in \R^n \colon x^j = \alpha\}$, поэтому новое определение очень даже согласуется с предыдущим.
\end{note}

\begin{anote}
	Задать ориентацию для какой-то области $U \subset \R^n$ означает задать ориентацию на всём пространстве. Открытый куб --- тоже область.
\end{anote}

\begin{definition} (Правило ориентации)
	\begin{enumerate}
		\item Чтобы задать ориентацию для стандартного открытого куба $K$:
		\[
			K = \{x \in \R^n \colon \forall i \in \range{1}{n}\ 0 < x^i < 1\}
		\]
		будем считать $e_0$ положительным базисом (как это и было раньше).
		
		\item Рассматривая грань $K_\alpha^j$ как $(n - 1)$-мерное пространство, зададим на нёй положительный ортонормированный базис. Выбор этого базиса произведём из векторов $e_0$ с, возможно, добавлением минусов так, чтобы если его дополнить слева выходящей из куба $K$ нормалью к этой грани, то получается положительно ориентированный базис.
	\end{enumerate}
\end{definition}

\begin{anote}
	В случае произвольного куба нам нужен некоторый базис $e_0$, где каждый вектор является каким-то ориентированным ребром.
\end{anote}

\begin{anote}
	Если бы мы не обращали внимание на правило, а взяли для грани $K_\alpha^j$ базис $e_0$ без вектора $e_j^0$, то в грани была бы определена форма ориентированного объёма $V^j$:
	\[
		V^j = dx^1 \wedge \ldots \wedge dx^{j - 1} \wedge dx^{j + 1} \wedge \ldots \wedge dx^n
	\]
	Эту форму можно понимать как форму в подпространстве, соответствующем грани $K_\alpha^j$, так и в $\R^n$.
\end{anote}

\textcolor{red}{Здесь надо картинку с кубом в трёхмерном пространстве}

\begin{proposition}
	Обозначим за $V_\alpha^j$ форму ориентированного объёма для грани $K_\alpha^j$. Тогда её можно записать следующим образом:
	\[
		V_\alpha^j = (-1)^{\alpha + j} dx^1 \wedge \ldots \wedge dx^{j - 1} \wedge dx^{j + 1} \wedge \ldots \wedge dx^n
	\]
\end{proposition}

\begin{proof}
	В силу того, как мы выбирали базис в грани $K_\alpha^j$, должно быть очевидным следующее равенство:
	\[
		V_\alpha^j = \eps_\alpha^j V^j,\ \eps_\alpha^j = \pm 1
	\]
	Заметим, что выходящей из куба нормалью к грани $K_\alpha^j$ можно всегда брать вектор $(-1)^{1 - \alpha}e_j^0$. Так как новый базис является просто перестановкой $e_0$ и мы знаем, что он положительно ориентирован, то должно быть справедливо равенство:
	\[
		(-1)^{1 - \alpha} dx^j \wedge \eps_\alpha^j V^j = V = dx^1 \wedge \ldots \wedge dx^n
	\]
	Остаётся расписать левую часть так, чтобы она отличалась от правой лишь видом коэффициента:
	\[
		(-1)^{1 - \alpha} \cdot \eps_\alpha^j \cdot (-1)^{j - 1} dx^1 \wedge \ldots \wedge dx^n = dx^1 \wedge \ldots dx^n \Lora (-1)^{-\alpha + j} \cdot \eps_\alpha^j;\ \eps_\alpha^j = (-1)^{-\alpha + j + 2\alpha} = (-1)^{\alpha + j}
	\]
\end{proof}

\begin{definition}
	\textit{Границей (краем) куба} $K$ назовём формальную сумму граней $K_\alpha^j$ с коэффициентами, соответствующими записи их форм ориентированного объёма:
	\[
		\vdelta K = \sum_{i \in \range{1}{n} \over {\alpha \in \{0, 1\}}} (-1)^{\alpha + i} K_\alpha^i
	\]
\end{definition}

\textcolor{red}{Тут нужно продемонстрировать границу $\vdelta([0; 1]) = \{1\} - \{0\}$ в одномерном случае}

\begin{theorem} (Стокса-Пуанкаре для стандартного куба)
	Если $\Omega$ --- гладкая $(n - 1)$-форма, заданная на стандартном кубе $K$, то имеет место формула:
	\[
		\int_K d\Omega = \int_{\vdelta K} \Omega
	\]
	где интеграл по границе куба нужно понимать как сумму интегралов от кубов в формальном ряде.
\end{theorem}

\begin{proof}
	Поскольку обе части равенства зависят от $(n - 1)$-формы линейно, то достаточно проверить его для базисных форм с коэффициентами следующего вида:
	\[
		\Omega(x) = f_j(x) dx^1 \wedge \ldots \wedge dx^{j - 1} \wedge dx^{j + 1} \wedge \ldots \wedge dx^n
	\]
	Тогда запишем дифференциал формы:
	\[
		d\Omega(x) = df_j(x)dx^1 \wedge \ldots \wedge dx^{j - 1} \wedge dx^{j + 1} \wedge \ldots \wedge dx^n = (-1)^{j - 1}\pd{f_j}{x_j}(x) \underbrace{dx^1 \wedge \ldots \wedge dx^n}_{V(x)}
	\]
	По определению интеграла от дифференциальной формы мы имеем право записать следующее:
	\[
		\int_K d\Omega = (-1)^{j - 1}\int_K \pd{f_j}{x_j}(x)d\mu(x)
	\]
	По теореме Фубини этот интеграл можно свести к повторному. Обозначим $E_j = \{x \in \R^n \colon x^j = 0\}$. Тогда:
	\begin{multline*}
		\int_K \pd{f_j}{x_j}(x)d\mu(x) = \int_{E_j \cap K} d\mu(y) \int_{[0; 1]} \pd{f_j}{x_j} (y, x_j) d\mu(x_j) = \int_{E_j \cap K} (f_j(y, 1) - f_j(y, 0))d\mu(y) =
		\\
		\int_{K_1^j} f_j(y, 1)d\mu(y) - \int_{K_0^j} f_j(y, 0)d\mu(y)
	\end{multline*}
	где $E_j \cap K$ нужно понимать как $(n - 1)$-мерное пространство, а запись аргументов $(y, x_j)$ неформально, то есть по-хорошему $x_j$ должен быть на $j$-й позиции среди координат $y$. По итогу получили такое равенство:
	\[
		\int_K d\Omega = (-1)^{j - 1}\int_{K_1^j} f_j(y, 1)d\mu(y) + (-1)^j \int_{K_0^j} f_j(y, 0)d\mu(y)
	\]
	А теперь посмотрим на то, что такое интеграл по границе:
	\[
		\int_{\vdelta K} \Omega := \sum_{i \in \range{1}{n} \over {\alpha \in \{0, 1\}}} (-1)^{\alpha + i} \int_{K_\alpha^j} \Omega = (-1)^{j + 1} \int_{K_1^j} f_j(y, 1)d\mu(y) + (-1)^j \int_{K_0^j} f_j(y, 0)d\mu(y) = \int_K d\Omega
	\]
\end{proof}