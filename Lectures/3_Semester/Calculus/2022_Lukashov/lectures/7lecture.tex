\section{Дифференциальные формы. Теория поля}

\subsection{Основы тензорной алгербы}

\begin{note}
	Для упрощения дальнейших формул, обозначим линейное пространство $E = \R^n$, а также примем стандартное соглашение, что $E^*$ --- \textit{сопряженное пространство} (линейное пространство функций $f \colon E \to \R$).
\end{note}

\begin{definition}
	\textit{Полилинейной формой валентности} $(p, q)$ называется отображение $w \colon E^p \times (E^*)^q$, линейное по каждому аргументу.
\end{definition}

\begin{note}
	Далее мы будем придерживаться следующих обозначений, принятых в данной области математики:
	\begin{enumerate}
		\item Индексация элементов из $E$ всегда снизу ($x_i$).
		
		\item Индексация элементов из $E^*$ всегда сверху ($f^j = f^j(x)$)
	\end{enumerate}
\end{note}

\begin{note}
	Зафиксируем базис $\{e_i\}_{i = 1}^n \subset E$. Тогда $\{e^j\}_{j = 1}^n \subset E^*$ --- соответствующий \textit{сопряжённый базис}.
\end{note}

\begin{note}
	Если не обговорено иного, то мы отождествляем пространство $E$ и $E^{**}$.
\end{note}

\begin{note}
	Если мы говорим о $j$-й координате вектора $x_i$, которая равна $e^j(x_i)$, то мы будем обозначать это как $x_i^j$. Если вектор $x_i$ записан как линейная комбинация базисных векторов, то знак суммы опускается (он однозначно восстанавливается по индексам. Пределы суммирования либо обговариваются, либо тривиальны):
	\[
		x_i = \sum_{j = 1}^n \xi_i^j e_j =: \xi_i^j e_j
	\]
	Аналогично с элементами сопряженного пространства:
	\[
		f^i = \sum_{j = 1}^n \eta_j^i e^j =: \eta_j^i e^j
	\]
\end{note}

\begin{proposition}
	Полулинейная форма $w$ валентности $(p, q)$ однозначно определяется заданием своих значения на следующих наборах:
	\[
		w_{i_1, \ldots, i_p}^{j_1, \ldots, j_q} := w(e_{i_1}, \ldots, e_{i_p}, e^{j_1}, \ldots, e^{j_q})
	\]
	где $i_k, j_t \in \range{n}$
\end{proposition}

\begin{proof}
	Действительно, если мы хотим узнать значение на произвольном наборе, то каждый аргумент можно переписать в виде линейной комбинации:
	\begin{multline*}
		w(x_1, \ldots, x_p, f^1, \ldots, f^q) = w(\xi_1^{i_1} e_{i_1}, \ldots, \xi_p^{i_p} e_{i_p}, \eta_{j_1}^1 e^{j_1}, \ldots, \eta_{j_q}^q e^{j_q}) =
		\\
		\xi_1^{i_1} \cdots \xi_p^{i_p} \eta_{j_1}^1 \cdots \eta_{j_q}^q w(e_{i_1}, \ldots, e_{i_p}, e^{j_1}, \ldots, e^{j_q}) = \xi_1^{i_1} \cdots \xi_p^{i_p} \cdot w_{e_{i_1}, \ldots, e_{i_p}}^{e^{j_1}, \ldots, e^{j_q}} \cdot \eta_{j_1}^1 \cdots \eta_{j_q}^q
	\end{multline*}
\end{proof}

\begin{definition}
	Набор значений $\{w_{e_{i_1}, \ldots, e_{i_p}}^{e^{j_1}, \ldots, e^{j_q}}\}$ называется \textit{тензором валентности (типа) $(p, q)$}.
\end{definition}

\begin{definition}
	\textit{Координатой тензора} называется одно из его многочисленных значений.
\end{definition}

\begin{definition}
	Множество полилинейных форм одной валентности $(p, q)$ образует линейное пространство $\Omega_p^q$ с естественными операциями:
	\[
		(\alpha w_1 + \beta w_2) = \alpha w_1 + \beta w_2
	\]
\end{definition}

\begin{note}
	Согласно тому, что мы поняли про задание полулинейной формы, базисом в $\Omega_p^q$ будет $n^p \cdot n^q$ форм, принимающих ненулевое значение лишь на своём наборе $(e_{i_1}, \ldots, e_{i_p}, e^{j_1}, \ldots, e^{j_q})$.
\end{note}

\begin{definition}
	\textit{Символом Кронекера} называется функция $\delta_i^j$ следующего вида:
	\[
		\delta_i^j = \System{
			&{1,\ i = j}
			\\
			&{0,\ i \neq j}
		}
	\]
\end{definition}

\begin{note}
	Символ Кронекера --- это естественная замена записи $e^j(e_i)$. Однако, при помощи этой записи мы также можем дать явный вид базисной формы:
	\[
		w(e_{k_1}, \ldots, e_{k_p}, e^{l_1}, \ldots, e^{l_q}) = \delta_{k_1}^{i_1} \cdot \ldots \cdot \delta_{k_p}^{i_p} \cdot \delta_{j_1}^{l_1} \cdot \ldots \cdot \delta_{j_q}^{l_q}
	\]
\end{note}

\begin{definition}
	Пусть $w_1 \in \Omega_{p_1}^{q_1}$, $w_2 \in \Omega_{p_2}^{q_2}$. Тогда их \textit{тензорным произведением} называется полилинейная форма $w_1 \otimes w_2 \in \Omega_{p_1 + p_2}^{q_1 + q_2}$, задаваемая следующим соотношением:
	\begin{multline*}
		w_1 \otimes w_2(x_1, \ldots, x_{p_1 + p_2}, f^1, \ldots, f^{q_1 + q_2}) =
		\\
		w_1(x_1, \ldots, x_{p_1}, f^1, \ldots f^{q_1}) \cdot w_2(x_{p_1 + 1}, \ldots, x_{p_1 + p_2}, f^{q_1 + 1}, \ldots, f^{q_1 + q_2})
	\end{multline*}
\end{definition}

\begin{proposition}
	Если $w$ --- базисная форма в пространстве $\Omega_p^q$, то её можно записать так:
	\[
		w = e^{j_1} \otimes \ldots \otimes e^{j_p} \otimes e_{i_1} \otimes \ldots \otimes e_{i_q}
	\]
\end{proposition}

\begin{proof}
	Очевидно.
\end{proof}

\begin{note}
	Что происходит при смене базиса? Посмотрим на замену $e' = eA$. Это эквивалентно всем допустимым равенствам $e'_i = \alpha_i^k e_k$. Да, это означает, что матрица $A$ выглядит так:
	\[
		e' = e\ps{\begin{aligned}
			&{\alpha_1^1 \cdots \alpha_n^1}
			\\
			&{\ \vdots\ \ \ddots}
			\\
			&{\alpha_1^n \cdots \alpha_n^n}
		\end{aligned}}
	\]
	Если посмотреть, куда переходит сопряжённый базис при смене лишь основного базиса, то получим закон $e^j = \alpha_k^j e'^k$. Наконец, мы можем посмотреть, что происходит с координатами произвольного вектора:
	\[
		x = \xi^i e_i = \xi'^i e'_i = \xi'^i \alpha_i^k e_k \Lora \xi^i = \xi'^j\alpha_j^i
	\]
\end{note}

\begin{definition}
	Если при переходе от базиса $e$ к $e'$ по матрице $A$ переход для связанной величины записывается в ту же сторону (то есть величина в базисе $e'$ выразилась явно) без применения операции обращения, то преобразование называют \textit{ковариантным}. В противном случае --- \textit{контрвариантным}.
\end{definition}

\begin{example}
	Базисные векторы преобразуются ковариантно, а координаты векторов --- контрвариантно.
\end{example}

\begin{proposition}
	Координаты тензора типа $(p, q)$ преобразуются ковариантно по первым $p$ аргументам и контрвариантно по последним $q$.
\end{proposition}

\begin{proof}
	Рассмотрим смену базиса по закону $e' = eA$ и тензор $w$. Всё, что нам необходимо сделать --- это расписать значение тензора на комбинации новых базисов. Проблема состоит в том, что расписать $e'^j$ без обратной матрицы мы не можем. Обозначим $B = A^{-1} = (\beta_i^j)$. Тогда в общем виде переход имеет вид $Be^* = e'^*$, стало быть:
	\begin{multline*}
		w(e'_{i_1}, \ldots, e'_{i_p}, e'^{j_1}, \ldots, e'^{j_q}) = w(\alpha_{i_1}^{k_1} e_{k_1}, \ldots \alpha_{i_p}^{k_p} e_{k_p}, \beta_{l_1}^{j_1} e'^{l_1}, \ldots, \beta_{l_q}^{j_q} e'^{l_q}) =
		\\
		\alpha_{i_1}^{k_1} \cdots \alpha_{i_p}^{k_p} \cdot w_{k_1, \ldots, k_p}^{l_1, \ldots, l_q} \cdot \beta_{l_1}^{j_1} \cdots \beta_{l_q}^{j_q}
	\end{multline*}
\end{proof}

\begin{proposition}
	Если задан базис $e \subset E$, то любой тензор $\{w_{i_1, \ldots, i_p}^{j_1, \ldots, j_q}\}$ порождает полилинейную форму $w \in \Omega_p^q$, задаваемую следующей формулой:
	\[
		w(x_1, \ldots, x_p, f^1, \ldots, f^q) = w_{i_1, \ldots, i_p}^{j_1, \ldots, j_q} (e^{i_1} \otimes \ldots \otimes e^{i_p} \otimes e_{j_1} \otimes \ldots \otimes e_{j_q})(x_1, \ldots, x_p, f^1 \ldots, f^q)
	\]
\end{proposition}

\begin{proof}
	Тут стоило бы проверить полилинейность, но это очевидно, коль скоро мы существуем в $\R^n$.
\end{proof}

\begin{example}
	А теперь поговорим о наглядных примерах форм разной валентности:
	\begin{enumerate}
		\item Вектор изоморфен (ну или является, если отождествлять) форме типа $(0, 1)$
		
		\item Линейный функционал является формой типа $(1, 0)$
		
		\item Чем является форма валентности $(1, 1)$? Если взять такую форму $T \colon E \times E^* \to \R$, то при фиксировании первого аргумента мы получаем элемент из $E^{**}$, то есть вектор. Стало быть, $T$ --- линейное преобразование.
		
		\item Билинейная форма является формой типа $(2, 0)$
	\end{enumerate}
\end{example}

\begin{definition}
	Форма $w \in \Omega_p^0$ называется \textit{симметричной}, если её значение на наборе не зависит от перестановки аргументов.
\end{definition}

\begin{definition}
	Форма $w \in \Omega_p^0$ называется \textit{кососимметричной}, если её значение на наборе равно нулю, при условии наличия хотя бы двух одинаковых аргументов.
\end{definition}

\begin{proposition}
	Пусть имеется форма $w \in \Omega_p^0$. Тогда она кососимметрична тогда и только тогда, когда при перестановке любых двух своих аргументов она меняет знак.
\end{proposition}

\begin{proof}~
	\begin{itemize}
		\item $\Ra$ Возьмём форму на каком-то наборе, переставим два аргумента и сложим эти значения. В силу кососимметричности их сумма равна нулю, что и свидетельствует о равенстве с точностью до знака.
		
		\item $\La$ От перестановки двух одинаковых аргументов значение формы поменяет знак, хотя запись значения не изменилась. Это возможно тогда и только тогда, когда значение равно нулю.
	\end{itemize}
\end{proof}

\begin{definition}
	Если задана форма $w \in \Omega_p^0$ и перестановка $\sigma \in S_p$, то определим произведение перестановки на форму как форму, заданную следующим соотношением:
	\[
		(\sigma w)(x_1, \ldots, x_p) = w(x_{\sigma(1)}, \ldots, x_{\sigma(p)})
	\]
\end{definition}

\begin{note}
	Произведение перестановки на форму обладает свойством дистрибутивности относительно сложения.
\end{note}

\begin{proposition}
	Если $w \in \Omega_p^0$ --- кососимметричная форма, а также имеется $\sigma \in S_p$, то верно равенство:
	\[
		\sigma w = w \cdot \sgn \sigma
	\]
\end{proposition}

\begin{proof}
	Тривиально.
\end{proof}

\begin{proposition}
	Если $w \in \Omega_p^0$ --- симметричная форма, а также имеется $\sigma \in S_p$, то верно равенство:
	\[
		\sigma w = w
	\]
\end{proposition}

\begin{note}
	Чтобы понять, является ли форма симметричной/кососимметричной, может быть достаточно просто посмотреть её поведение с любой перестановкой.
\end{note}

\begin{proposition}
	Если имеется $w \in \Omega_p^0$, то
	\begin{itemize}
		\item Форма $U = \frac{1}{p!} \sum_{\sigma \in S_p} \sigma w$ симметрична
		
		\item Форма $V = \frac{1}{p!} \sum_{\sigma \in S_p} (\sgn \sigma) \sigma w$ кососимметрична
	\end{itemize}
\end{proposition}

\begin{proof}~
	\begin{itemize}
		\item Поймём, что в сумме по сути в $w$ подставляются всевозможные перестановки набора аргументов. Стало быть, от их перемены местами ничего не изменится.
		
		\item Воспользуемся вышеупомянутой идеей и рассмотрим форму $\tau V$, $\tau \in S_p$:
		\[
		\tau V = \frac{1}{p!} \sum_{\sigma \in S_p} (\sgn \sigma) \tau \sigma w
		\]
		Обозначим $\zeta = \tau \circ \sigma$. Тогда $\sgn \sigma = \sgn \zeta \cdot \sgn \tau$ и нашу форму можно переписать так:
		\[
		\tau V = \frac{\sgn \tau}{p!} \sum_{\zeta \in S_p} (\sgn \zeta) \zeta w = (\sgn \tau) \tau V
		\]
	\end{itemize}
\end{proof}

\begin{definition}
	Рассмотренная выше форма $U$ называется \textit{симметризацией} $w$, а $V$ называется \textit{ассиметризацией} соответсвенно. Обозначаются как $U = \sym w$, $V = \asym w$.
\end{definition}

\begin{proposition}
	Если взять произвольную $\sigma \in S_p$ и форму $w \in \Omega_p^0$, то имеют место 2 равенства:
	\[
		\asym(\sigma w) = \sigma \asym(w);\ \ \sym(\sigma w) = \sym(w)
	\]
\end{proposition}

\begin{proof}
	Докажем лишь первое равенство, ибо второе очевидно. Достаточно объяснить следующее:
	\[
		\sum_{\delta \in S_p} \sgn(\delta) \delta \sigma w = \sum_{\delta \in S_p} \sgn(\delta) \sigma \delta w
	\]
	Имеет место биекция в равенстве $\delta \sigma = \sigma \delta'$, то есть для каждой $\delta$ существует и единственна $\delta'$, выражающаяся как сопряжённый элемент $\delta' = \sigma^{-1} \delta \sigma$. При этом $\sgn(\delta') = \sgn^2(\sigma) \sgn(\delta) = \sgn(\delta)$. Отсюда уже и справедливость равенства.
\end{proof}

\begin{corollary}
	\(\asym(\sigma w) = \sgn(\sigma) \asym(w)\)
\end{corollary}

\begin{lemma}
	Если $U, V \in \Omega_p^0$ --- произвольные полилинейные формы, то имеет место равенство:
	\[
		\asym((\asym U) \otimes V) = \asym (U \otimes (\asym V)) = \asym(U \otimes V)
	\]
\end{lemma}

\begin{proof}
	По сути нужно воспользоваться уже упомянутыми свойствами ассиметризации и её определением:
	\begin{multline*}
		\asym((\asym U) \otimes V) = \frac{1}{p!} \sum_{\sigma \in S_p} \sgn(\sigma) \cdot \asym((\sigma U) \otimes V) =
		\\
		\frac{1}{p!} \sum_{\tau \in S'_{2p}} \sgn(\tau) \cdot \asym(\tau(U \otimes V)) = \frac{1}{p!} \sum_{\tau \in S'_{2p}} \sgn(\tau) \cdot \sgn(\tau) \cdot \asym(U \otimes V) = \asym(U \otimes V)
	\end{multline*}
	где $S'_{2p}$ --- это множество всех перестановок, которые отражают числа $p + 1, \ldots, 2p$ в самих себя, то есть $S'_{2p} \simeq S_p$
\end{proof}

\begin{designation}
	Подпростраство кососимметрических форм в $\Omega_p^0$ будет обозначаться как $\Lambda_p$
\end{designation}

\begin{definition}
	\textit{Внешним произведением форм} $U \in \Lambda_p$ и $V \in \Lambda_r$ называется форма $U \wedge V \in \Lambda_{p + r}$, определяемая следующим равенством:
	\[
		U \wedge V = \frac{(p + r)!}{p! \cdot r!} \asym(U \otimes V)
	\]
\end{definition}

\begin{note}
	Сам оператор $\wedge$ обычно называют <<(внешней) скобкой>>.
\end{note}

\begin{theorem} (Свойства внешнего произведения форм)
	Для любых $U \in \Lambda_p$, $V \in \Lambda_r$ и $W \in \Lambda_s$ верны следующие свойства:
	\begin{enumerate}
		\item $U \wedge 0 = 0 \wedge U = 0$
		
		\item $\forall \alpha, \beta \in \R\ \ (\alpha U + \beta V) \wedge W = \alpha(U \wedge W) + \beta(V \wedge W)$
		
		\item $(U \wedge V) \wedge W = U \wedge (V \wedge W)$
		
		\item $V \wedge U = (-1)^{pr} U \wedge V$
	\end{enumerate}
\end{theorem}

\begin{proof}~
	\begin{enumerate}
		\item Тривиально
		
		\item Очевидно
		
		\item Обычно такое равенство означает наличие некоторой общей симметричной формы записи. Мы хотим добиться такого результата:
		\[
			(U \wedge V) \wedge W = U \wedge (V \wedge W) = \frac{(p + r + s)!}{p! \cdot r! \cdot s!} \asym(U \otimes V \otimes W)
		\]
		Теперь проделаем шаги для левых круглых скобок:
		\begin{multline*}
			(U \wedge V) \wedge W = \frac{(p + r + s)!}{(p + r)! \cdot s!} \asym((U \wedge V) \otimes W) =
			\\
			\frac{(p + r + s)!}{(p + r)! \cdot s!} \asym\ps{\frac{(p + r)!}{p! \cdot r!} \asym((U \otimes V) \otimes W)}
		\end{multline*}
		Остаётся сослаться на вышедоказанную лемму.
		
		\item Если исходить из определения внешнего произведения, то доказательство этого факта сводится к доказательству аналогичного про $\asym(V \otimes U)$. Вспомним, как выглядит этот объект:
		\[
			\asym(V \otimes U) = \frac{1}{p!} \sum_{\sigma \in S_{p + r}} \sgn(\sigma) \cdot \sigma (V \otimes U)
		\]
		Так как мы живём в множестве пространств $\Omega_l^0$, то тензорное произведение работает почти как произведение функций, а ассимметризация заставляет просмотреть всевозможные перестановки аргументов. Не умаляя общности, посмотрим на пару слагаемых из двух ассимметризаций, когда верно равенство:
		\begin{multline*}
			\sigma(V \otimes U)(x_1, \ldots, x_{p + r}) = (V \otimes U)(x_{p + 1}, \ldots, x_{p + r}, x_1, \ldots, x_p) =
			\\
			(U \otimes V)(x_1, \ldots, x_p, x_{p + 1}, \ldots, x_{p + r}) = \delta(U \otimes V)(x_1, \ldots, x_{p + r})
		\end{multline*}
		Что можно сказать про соотношение между знаками $\sigma$ и $\delta$? Чтобы из $\sigma$ получить $\delta$ перестановку, достаточно сделать циклический сдвиг $(1, \ldots, p + r)$ ровно $r$ раз. При этом его знак равен $(-1)^{p + r - 1}$, что в композиции с оставшимися циклами даст $(-1)^{pr + r^2 - r}$. Заметим, что часть после $pr$ всегда даёт единицу (или же ноль по модулю двух), а стало быть $\sgn(\sigma) = (-1)^{pr} \sgn(\delta)$. Это верно для любой перестановки $\sigma$ и соответствующей ей $\delta$, а поэтому исходное равенство доказано.
	\end{enumerate}
\end{proof}

\begin{lemma}
	Для любых линейных функционалов $f^1, \ldots, f^p$ и любых векторов $x_1, \ldots, x_p$ верно следующее:
	\[
		f^1 \wedge \ldots \wedge f^p(x_1, \ldots, x_p) = \det(f^i(x_j))
	\]
	Пояснение: в $i$-й строке записаны значения $f^i$ на каждом векторе.
\end{lemma}

\begin{proof}
	Распишем скобки по определению:
	\begin{multline*}
		f^1 \wedge \ldots \wedge f^p(x_1, \ldots, x_p) = \asym(f^1 \otimes \ldots \otimes f^p)(x_1, \ldots, x_p) =
		\\
		\frac{1}{p!} \sum_{\sigma \in S_p} \sgn(\sigma) \sigma (f^1 \otimes \ldots \otimes f^p)(x_1, \ldots, x_p) = \frac{1}{p!} \sum_{\sigma \in S_p} \sgn(\sigma) f^1(x_{\sigma(1)}) \cdot \ldots \cdot f^p(x_{\sigma(p)})
	\end{multline*}
	Записанное выражение является ничем иным, кроме как детерминантом матрицы $(f^i(x_j))$
\end{proof}