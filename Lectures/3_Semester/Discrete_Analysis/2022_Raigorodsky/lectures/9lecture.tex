\begin{proof} (доказательство теоремы 1.11 \textcolor{red}{поправить ссылку на теорему})
	Мы уже знаем, что если $p = 1 / 2$, то асимптотически почти наверное $\alpha(G) \le 2\log_2 n$, а для хроматического числа это даёт оценку $\chi(G) \ge n / (2\log_2 n)$ (этим мы уже доказали часть теоремы, оценка снизу).
	
	Введём параметры \textcolor{red}{Переписать. Сделать выбор интуитивным.} $m = \floor{n / \ln^2 n}$ и $f_m(k) = \E X_k$, где $X_k$ - случайная величина, означающая число независимых множеств на $k$ вершинах в графе $G(m; 1 / 2)$. Тогда $f_m(k) = \E X_k = C_m^k \cdot 2^{-C_k^2}$. Посмотрим, что происходит при разных $k$:
	\begin{align*}
		&{f_m(1) = C_m^1 \cdot 2^0 = m \xrightarrow[n \to \infty]{} +\infty}
		\\
		&{f_m(2) = C_m^2 \cdot 2^{-1} \xrightarrow[n \to \infty]{} +\infty}
		\\
		&{\vdots}
		\\
		&{f_m(2\log_2 m) \xrightarrow[n \to \infty]{} 0}
	\end{align*}
	\textcolor{red}{Прокомментировать последний предел и существование некоторой границы, до которой стремимся в $+\infty$, а после в 0.} Введём $k_0(m)$ таким образом:
	\[
		k_0(m) = t \such f_m(t) < 1,\ f_m(t - 1) \ge 1
	\]
	Тогда имеет место факт, что $k_0(m) \sim 2\log_2 m,\ m \to \infty$.
\end{proof}

