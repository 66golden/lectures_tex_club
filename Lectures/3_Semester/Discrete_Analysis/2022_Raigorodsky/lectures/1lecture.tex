\section{Продолжение теории графов}

\subsection{Асимптотика числа унициклических графов}

\begin{proposition} \label{asympprop}
	Для числа унициклических графов $U(n)$ имеет место следующая асимптотика:
	\[
		U(n) \sim \sqrt{\frac{\pi}{8}} \cdot n^{n - \frac{1}{2}}
	\]
\end{proposition}

\begin{proof}
	Перепишем сумму в более удобном виде:
	\begin{multline*}
		U(n) = \frac{1}{2} \sum_{r = 3}^n C_n^r \cdot r! \cdot n^{n - 1 - r} = \frac{1}{2} n^{n - 1} \sum_{r = 3}^n \ps{n^{-r} \cdot \prod_{t = 0}^ {r - 1} (n - t)} =
		\\
		\frac{1}{2} n^{n - 1} \sum_{r = 3}^n \prod_{t = 0}^{r - 1} \ps{1 - \frac{t}{n}} = \frac{1}{2} n^{n - 1} \sum_{r = 3}^n \exp\ps{\sum_{t = 0}^{r - 1} \ln\ps{1 - \frac{t}{n}}}
	\end{multline*}
	Отдельно преобразуем внутреннюю сумму (при помощи формулы Тейлора с остаточным членом в форме О-большого):
	\[
		\sum_{t = 0}^{r - 1} \ln\ps{1 - \frac{t}{n}} = \sum_{t = 0}^{r - 1} \ps{-\frac{t}{n} + O\ps{\frac{t^2}{n^2}}} = -\frac{r(r - 1)}{2n} + O\ps{\frac{r^3}{n^2}}
	\]
	Отметим 1 интересный факт о полученном выражении: если $r = r(n)$ и $r^3 = o(n^2)$ (то есть $r = o(n^{2/3})$), то
	\[
		-\frac{r(r - 1)}{2n} + O\ps{\frac{r^3}{n^2}} = -\frac{r(r - 1)}{2n} + o(1) \sim -\frac{r(r - 1)}{2n}
	\]
	Теперь, мы применим классический трюк: разделим сумму на 2 и попробуем оценить каждую часть по отдельности
	\begin{multline*}
		\sum_{r = 3}^n \exp\ps{-\frac{r(r - 1)}{2n} + O\ps{\frac{r^3}{n^2}}} = \underbrace{\sum_{r = 3}^{l(n)} \exp\ps{-\frac{r(r - 1)}{2n} + O\ps{\frac{r^3}{n^2}}}}_{S_1} +
		\\
		\underbrace{\sum_{r = l(n) + 1}^n \exp\ps{-\frac{r(r - 1)}{2n} + O\ps{\frac{r^3}{n^2}}}}_{S_2}
	\end{multline*}
	\begin{itemize}
		\item Попробуем выяснить, при каких $l(n)$ $S_2$ будет зануляться:
		\begin{multline*}
			S_2 = \sum_{r = l(n) + 1}^n \exp\ps{\sum_{t = 0}^{r - 1} \ln\ps{1 - \frac{t}{n}}} \le
			\\
			\sum_{r = l(n) + 1}^n \exp\ps{-\sum_{t = 0}^{r - 1} \frac{t}{n}} = \sum_{r = l(n) + 1}^n \exp\ps{-\frac{r(r - 1)}{2n}} \le
			\\
			n \cdot \exp\ps{-\frac{(l(n) + 1)l(n)}{2n}}
		\end{multline*}
		В переходе от первой строки ко второй мы используем тот факт, что для $-1 < x < 0 \Ra \ln(1 + x) < x$. Если $l(n) > n^{1/2}$, то экспонента будет стремиться к нулю, утаскивая $S_2$ за собой.
		
		\item Теперь посмотрим, что мы могли бы сделать с $S_1$, если учитывать условие из $S_2$. Всё, что мы можем сделать - потребовать $l(n) < n^{2/3}$ и применить уже известный факт, ибо $r \le l(n) < n^{2/3} \Ra r = o(n^{2/3})$:
		\[
			S_1 = \sum_{r = 3}^{l(n)} \exp\ps{-\frac{r(r - 1)}{2n} + O\ps{\frac{r^3}{n^2}}} \sim \sum_{r = 3}^{l(n)} \exp\ps{-\frac{r(r - 1)}{2n}} \sim
			\\
			\sum_{r = 3}^{l(n)} \exp\ps{-\frac{r^2}{2n}}
		\]
		Для определенности, мы можем взять $l(n) = \floor{n^{0.6}}$, или, например, $l(n) = \floor{n^{\pi / 5}}$. Получившуюся сумму очень хотелось бы посчитать так:
		\[
			\sum_{r = 3}^{l(n)} \exp\ps{-\frac{r^2}{2n}} = \sum_{r = 0}^\infty \exp\ps{-\frac{r^2}{2n}} - \sum_{r = 0}^2 \exp\ps{-\frac{r^2}{2n}} - \sum_{r = l(n) + 1}^\infty \exp\ps{-\frac{r^2}{2n}}
		\]
		Но возникает вопрос: а почему эта запись корректна? Почему ряд сходится? Мы воспользуемся интегральным признаком Коши и его следствием:
		\[
			\sum_{r = 0}^\infty \exp\ps{-\frac{r^2}{2n}} \sim \int_0^\infty e^{-\frac{r^2}{2n}}dr
		\]
		Мы свели задачу к известному интегралу Гаусса, обладающуего конкретным значением (значит, ряды сходятся):
		\[
			\int_0^\infty e^{-\frac{r^2}{2n}}dr = \frac{\sqrt{2\pi n}}{2} = \sqrt{\frac{\pi n}{2}}
		\]
		Со второй суммой всё совсем просто:
		\[
			\lim_{n \to \infty} \sum_{r = 0}^2 e^{-\frac{r^2}{2n}} = 1 + 1 + 1 = 3
		\]
		Для последней придётся поисследовать ряд. Например, во сколько раз уменьшается следующее слагаемое по сравнению с предыдущим?
		\[
			e^{-\frac{(r + 1)^2}{2n}} / e^{-\frac{r^2}{2n}} = e^{-\frac{2r + 1}{2n}} = e^{-\frac{r}{n} - \frac{1}{2n}} < e^{-\frac{r}{n}}
		\]
		Если $r > n^2$, то $e^{-r/n} < e^{-n}$ - зависимость от $r$ пропадает, и мы можем получить оценку сверху на сумму. Значит, повторим старую стратегию:
		\[
			\sum_{r = l(n) + 1}^\infty \exp\ps{-\frac{r^2}{2n}} = \underbrace{\sum_{r = l(n) + 1}^{n^2} \exp\ps{-\frac{r^2}{2n}}}_{S'_1} + \underbrace{\sum_{r = n^2 + 1}^\infty \exp\ps{-\frac{r^2}{2n}}}_{S'_2}
		\]
		\begin{itemize}
			\item Для $S'_1$ делаем действия, аналогичные для $S_1$ (напоминаю, есть ограничения $n^{1/2} < l(n) < n^{2/3}$):
			\[
				S'_1 \le \sum_{r = l(n) + 1}^{n^2} \exp\ps{-\frac{(l(n) + 1)^2}{2n}} \le n^2 \cdot \exp\ps{-\frac{l^2(n)}{2n}} \xrightarrow[n \to \infty]{} 0
			\]
			
			\item Для $S'_2$ теперь пишем оценку исходя из упомянутого выше факта:
			\begin{multline*}
				S'_2 \le \sum_{r = n^2 + 1}^\infty \exp\ps{-\frac{(n^2 + 1)^2}{2n}} \cdot (e^{-n})^{r - n^2 - 1} = \exp\ps{-\frac{(n^2 + 1)^2}{2n}} \cdot (1 + e^{-n} + e^{-2n} + \ldots) =
				\\
				\exp\ps{-\frac{(n^2 + 1)^2}{2n}} \cdot \frac{1}{1 - e^{-n}} \xrightarrow[n \to \infty]{} 0
			\end{multline*}
		\end{itemize}
		Итого, мы нашли эквивалентность для $S_1$:
		\[
			S_1 \sim \sqrt{\frac{\pi n}{2}} - 3 \sim \sqrt{\frac{\pi n}{2}}
		\]
		В общей формуле это выливается в следующее:
		\[
			U(n) = \frac{1}{2} n^{n - 1} (S_1 + S_2) \sim \frac{1}{2} n^{n - 1} \sqrt{\frac{\pi}{2}} n^{\frac{1}{2}} = \sqrt{\frac{\pi}{8}} n^{n - \frac{1}{2}}
		\]
	\end{itemize}
\end{proof}