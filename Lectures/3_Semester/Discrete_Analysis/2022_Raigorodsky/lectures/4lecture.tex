\begin{theorem}
	Пусть $G$ --- дистанционный граф на плоскости, $|V| = 4n,\ \alpha(G) \le n$. Тогда $|E| \ge 7n$.
\end{theorem}

\begin{proof}
	Для начала, повторим рассуждения из теоремы Турана. Выделим $A \subseteq V$, реализующее число независимости. Из него мы получим $\ge 3n$ рёбер в $G$. Посмотрим на индуцированный при помощи $V \bs A$ граф и проделаем то же самое. Это даст нам $\ge 2n$ ребёр, ну и последний раз даст $\ge n$ рёбер. Суммарно это $|E| \ge 6n$, что не дотягивает до утверждения теоремы. Мы сможем доказать теорему, если мы из каких-то соображений улучшим оценку рёбер на первом шаге до $\ge 4n$.
	
	Для этого разобьём $V \bs A$ на 2 непересекающихся множества:
	\[
		V \bs A = V_1 \sqcup V_2
	\]
	где $V_1 = \{x \in V \bs A \colon \exists! y \in A,\ (x, y) \in E\}$
	\begin{proposition}
		Если $|V_2| \ge n$, то на первом шаге мы найдём $\ge 4n$ рёбер
	\end{proposition}

	\begin{proof}
		Действительно, $V_2$ даст $\ge 2|V_2|$ рёбер, тогда как от $V_1$ будет ровно $4n - \alpha(G) - |V_2| \ge 3n - |V_2|$ рёбер. В сумме это $3n + |V_2|$ рёбер и утверждение стало тривиальным.
	\end{proof}

	\begin{lemma}
		При всех условиях теоремы, $|V_2| \ge n$
	\end{lemma}

	\begin{proof}
		Предположим противное: $|V_2| \le n - 1$. В таком случае, $|V_1| \ge 2n + 1$. Более того, тогда должна существовать вершина $y \in A$, соединенная ребрами с тремя из $V_1$ (это следует по принципу Дирихле):
		\[
			\Ra \exists y \in A \such \exists x_1, x_2, x_3 \in V_1\ \ \forall i \in \{1, 2, 3\}\ (x_i, y) \in E
		\]
		Ну и катарсис доказательства: если между $x_1, x_2$ нет ребра, то $(A \bs \{y\}) \cup \{x_1, x_2\}$ будет б\'{о}льшим независимым множеством, чем $A$. Аналогично для остальных пар иксов, а стало быть мы получаем клику на вершинах $\{y, x_1, x_2, x_3\}$ в дистанционном графе на плоскости. Несложно убедиться, что такой существовать в принципе не может.
	\end{proof}
\end{proof}

\begin{note}
	Теорема не является лучшим результатом. Примерно теми же методами доказывается оценка $|E| \ge \frac{26}{3}n$.
\end{note}

\begin{corollary} (теоремы Турана)
	Если $G_n = (V_n, E_n)$, $|V_n| = n$, а также $\alpha(G_n) = \alpha_n = o(n)$, то имеет место следующее неравенство:
	\[
		|E_n| \ge (1 + o(1))\frac{n^2}{2\alpha}
	\]
\end{corollary}

\begin{proof}
	Несложно увидеть следующую эквивалентность:
	\[
		\floor{\frac{n}{\alpha}} \sim \frac{n}{\alpha}
	\]
	Если подставить эту эквивалентность для формулы из теоремы Турана, то получим данное следствие.
\end{proof}

\begin{theorem}
	Пусть $G_s = (V_s, E_s)$ --- последовательность дистанционных графов таких, что $|V_s| = n(s)$, $\alpha(G_s) = \alpha(s)$, $V_s \subset \R^{d(s)}$. Пусть также $d(s) \cdot \alpha(s) = o(n(s))$. В таком случае
	\[
		|E_s| \ge (1 + o(1)) \frac{n^2(s)}{\alpha(s)}
	\]
\end{theorem}

\begin{note}
	Пафос теоремы состоит в том, что для дистанционных графов мы улучшаем следствие теоремы Турана аж в 2 раза.
\end{note}

\begin{lemma} (без доказательства)
	Если $K_l$ --- это клика на $l$ вершинах, то в $\R^n$ всегда существует $K_{n + 1}$ и не существует $K_{n + 2}$.
\end{lemma}

\begin{proof}
	Будем повторять идею доказательства предыдущей теоремы, но уже ориентируясь на другую лемму:
	\begin{lemma}
		При условиях теоремы, $|V_1| \le d(s) \cdot \alpha(s)$
	\end{lemma}

	\begin{proof}
		Предположим, что $|V_1| \ge d(s) \cdot \alpha(s) + 1$. Тогда
		\[
			\exists y \in A, x_1, \ldots, x_{d + 1} \in V_1 \such \forall i \in \range{d + 1}\ \ (x_i, y) \in E
		\]
		Ровно тем же методом показываем, что вершины $\{y, x_1, \ldots, x_{d + 1}\}$ образуют клику. По лемме о кликах, такой в нашем пространстве не существует.
	\end{proof}
	
	Следовательно, мы найдём $\ge d(s) \cdot \alpha(s) + 2(n(s) - \alpha(s) - d(s) \cdot \alpha(s))$ рёбер. Если преобразовать данное выражение, то получим $\ge 2n(s) - d(s)\alpha(s) - 2\alpha(s)$ рёбер. Что поменяется в этой формуле, когда мы перейдём к $V \bs A$? Вместо $n(s)$ будет $n(s) - \alpha(s)$, и после раскрытия скобок получится то же выражение, только $4\alpha(s)$ в конце. Аналогично происходит на остальных итерациях, а потому всего их будет $k$ штук:
	\[
		k = \floor{\frac{2n(s) - d(s)\alpha(s)}{2\alpha(s)}} \sim \frac{2n(s)}{2\alpha(s)} = \frac{n(s)}{\alpha(s)}
	\]
	если обозначить $a(s) = 2n(s) - d(s)\alpha(s)$, то
	\begin{multline*}
		|E_s| \ge \sum_{t = 1}^k \ps{a(s) - t \cdot 2\alpha(s)} = ka(s) - 2\alpha(s) \cdot \frac{k(k - 1)}{2} \sim ka(s) - k^2\alpha(s) \sim
		\\
		a(s) \cdot \frac{n(s)}{\alpha(s)} - \alpha(s) \cdot \ps{\frac{a(s)}{2\alpha(s)}}^2 \sim \frac{2n^2(s)}{\alpha(s)} - \frac{4n^2(s)}{4\alpha(s)} = \frac{n^2(s)}{\alpha(s)}
	\end{multline*}
\end{proof}

\begin{example}
	Вернёмся к графу, который появлялся в теореме Эрдёша-Хватала:
	\begin{itemize}
		\item \(V = \{\vec{x} = (x_1, \ldots, x_d) \colon x_i \in \{0, 1\},\ \sum_{i = 1}^d x_i = 3\}\)
		
		\item \(E = \{\{\vec{x}, \vec{y}\} \colon |\vec{x} - \vec{y}| = 2\}\)
	\end{itemize}
	Напомним 3 факта об этом графе:
	\begin{enumerate}
		\item $|V| = n = C_d^3$
		
		\item $\alpha(G) = \System{&{d,\ d \equiv 0 \pmod 4} \\ &{d - 1,\ d \equiv 1 \pmod 4} \\ &{d - 2,\ \text{иначе}}}$
	\end{enumerate}
	Положим за номер графа в последовательности размерность пространства, в котором он находится. Тогда $n \sim d^3 / 6$,\ $\alpha(G) \sim d$, а также $\alpha \cdot d \sim d^2 = o(n)$. Стало быть, доказанная теорема верна для этой последовательности, и тут мы улучшаем оценку в 2 раза по сравнению со следствием теоремы Турана.
\end{example}