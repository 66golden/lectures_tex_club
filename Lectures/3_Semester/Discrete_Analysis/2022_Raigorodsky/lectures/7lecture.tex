\begin{definition}
	\textit{Обхватом} (на англ. \textit{girth}) графа $G$ называется величина $g(G)$, равная длине минимального цикла в этом графе.
\end{definition}

\begin{proposition}
	Чтобы лучше прочувствовать определение, заметим такое интересное свойство:
	\[
		g(G) > 3 \Lra \text{в $G$ нет треугольников}
	\]
\end{proposition}

\begin{proof}
	\textcolor{red}{Дописать}
\end{proof}

\begin{theorem} (1957, Эрдеш)
	Имеет место следующее утверждение:
	\[
		\forall k, l \in \N\ \exists G = (V, E) \such \chi(G) > k,\ g(G) > l
	\]
\end{theorem}

\begin{proof}
	$G(n, p)$, $p = p(n) = n^{\Theta - 1}$, $\Theta = \frac{1}{2l}$.
	
	$X_l := X_l(G)$ --- количество простых циклов длины $\le l$ в $G$. Посмотрим матожидание:
	\[
		\E X_l = \sum_{r = 3}^l C_n^r \frac{(r - 1)!}{2} p^r < \sum_{r = 3}^l \frac{n^r}{r!} \cdot r! \cdot p^r = \sum_{r = 3}^l (np)^r < l \cdot n^{\Theta l} = l \cdot n^{1/2}
	\]
	По неравенству Маркова:
	\[
		P\ps{X_l \ge \frac{n}{2}} \le \frac{\E X_l}{n / 2} < \frac{l\sqrt{n}}{n / 2} \xrightarrow[n \to \infty]{} 0
	\]
	Отсюда возникает следствие:
	\[
		\exists n_1 \such \forall n \ge n_1\ P\ps{X_l < \frac{n}{2}} > \frac{1}{2}
	\]
	\textcolor{red}{Тут был замысел, около 16й минуты}
	
	$Y_x := Y_x(G)$ --- количество независимых множеств размера $x$ в $G$. Положим $x = \ceil{\frac{3\ln n}{p}} \sim \frac{3\ln n}{p}$. Оценим матожидание:
	\[
		\E Y_x = C_n^x (1 - p)^{C_x^2} \le n^x \cdot e^{-pC_x^2} = \exp\ps{x\ln n - (1 + o(1))p \frac{x^2}{2}}
	\]
	Посмотрим асимптотику штуки в скобочках:
	\[
		x\ps{\ln n - (1 + o(1))\frac{px}{2}} \sim x\ps{\ln n - (1 + o(1))\frac{3\ln n}{2}} \xrightarrow[n \to \infty]{} -\infty
	\]
	Снова применим неравенство Маркова:
	\[
		P(\alpha(G) \ge x) = P(Y_x \ge 1) \le \frac{\E Y_x}{1} \to 0
	\]
	Стало быть, $\exists n_2 \such \forall n \ge n_2\ P(\alpha(G) < x) > 1 / 2$. Теперь можно рассмотреть $n > \max \{n_1, n_2\}$, а потому
	\[
		\exists G \such X_l(G) < \frac{n}{2} \wedge \alpha(G) < x
	\]
	Получим индуцированный граф $G'$ из $G$ путём удаления вершин и инциндентных им рёбер. Добьёмся того, что $X_l(G') = 0$, то есть $g(G') > l$. Чтобы убрать все простые циклы (по свойству $G$), нам потребуется удалить не более $n / 2$ вершин. Иначе говоря, $|V(G')| > n - n / 2 = n / 2$. При этом, мы не могли улучшить число независимости \textcolor{red}{почему?}. Отсюда
	\[
		\chi(G') > \frac{n}{2x} \sim \frac{np}{2 \cdot 3\ln n} = \frac{n^\Theta}{6\ln n} \to +\infty
	\]
	По итогу $\exists n_3 \such \forall n \ge n_3\ \chi(G') > k$.
\end{proof}

\begin{note}
	Теорема указывает на существование и количество таких графов (их большинство), а вот уже конструкцию никакую не даёт. Первая конструкция была придумана Ловасом Ласло примерно через 10 лет после публикации Эрдёша.
\end{note}

\begin{proposition}
	Имеют место следующие утверждения:
	\begin{enumerate}
		\item Если $p(n) = o(1 / n^2)$ и мы обозначим за $X(G)$ --- число рёбер в $G$, то асимптотически почти наверное $X = 0$ и $\chi(G) = 1$
		
		\item Пусть $p(n) = w(1 / n^2)$ (по определению это значит, что $pn^2 \to +\infty$), а также $p = o(1 / n)$. Тогда асимптотически почти наверное $\chi(G) = 2$.
		
		\item Пусть $p = c / n$, $c < 1$. Тогда асимптотически почти наверное $\chi(G) = 3$. Более того, существует $\Psi = \Psi(n) = o(n)$ и асимптотически почти наверное $n - \Psi(n)$ вершин графа $G(n, p)$ принадлежат древесным комптонентам (то есть компонентам-деревьям), а $\Psi(n)$ вершин принадлежат унициклическим компонентам.
	\end{enumerate}
\end{proposition}

\begin{proof}~
	\begin{enumerate}
		\item Посмотрим на матожидание $X$:
		\[
			\E X = C_n^2 \cdot p \sim \frac{pn^2}{2} = o(1)
		\]
		Тогда, мы можем оценить вероятность наличия рёбер:
		\[
			P(X \ge 1) \le \E X \to 0
		\]
		\textcolor{red}{Почему $\chi(G) = 1$ асимптотически почти наверное?}
		
		\item Оцени наличие рёбер в графе через известный трюк:
		\[
			P(X \ge 1) \ge 1 - \frac{DX}{(\E X)^2}
		\]
		где $DX = C_n^2 p(1 - p)$. Стало быть
		\[
			\frac{DX}{(\E X)^2} = \frac{C_n^2 p(1 - p)}{(C_n^2 p)^2} = \frac{1 - p}{p \cdot C_n^2} \xrightarrow[n \to \infty]{} 0
		\]
	\end{enumerate}
\end{proof}

\begin{theorem} (80-е годы прошлого века, Бела Боллобаш)
	Пусть $p = n^{-\alpha}$, $\alpha \in (5 / 6; 1)$. Тогда существует функция $u(n, p)$ такое, что для $G(n, p)$ асимптотически почти наверное $\chi(G) \in \{u, u + 1, u + 2, u + 3\}$.
\end{theorem}

\begin{exercise}
	$u(n, p) \to +\infty$ всегда.
\end{exercise}

\begin{note}
	Сейчас теорема улучшена до ограничений $\alpha \in (1 / 2; 1)$ и результата $\chi(G) \in \{u, u + 1\}$.
\end{note}

\begin{theorem} (80-е годы прошлого века, Бела Боллобаш)
	Пусть $p = 1 / 2$. Тогда существует функция $\phi(n) = o(n / \ln n)$ такая, что асимтотически почти наверное имеет место неравенство:
	\[
		\frac{n}{2\log_2 n} - \phi(n) \le \chi(G) \le \frac{n}{2\log_2 n} + \phi(n)
	\]
\end{theorem}

\begin{note}
	Разница с предыдущей теоремой в том, что конкретно для $p = 1 / 2$ отрезок значений $\chi(G)$ будет сужаться. При этом заменить $\phi(n)$ на константу нельзя, недавно было доказано, что $\phi(n) = \Omega(\sqrt[4]{n})$.
\end{note}