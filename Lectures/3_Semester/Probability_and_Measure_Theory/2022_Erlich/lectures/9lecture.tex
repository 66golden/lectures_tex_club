\begin{theorem}
	$\mu^*$ --- аддитивная мера на $M$.
\end{theorem}

\begin{proof}
	Так как мы уже доказали, что $M$ является алгеброй, то нам достаточно показать следующий факт:
	\[
		\forall A, B, C \in M,\ A = B \sqcup C \Ra \mu^*(A) = \mu^*(B) + \mu^*(C)
	\]
	Соответствующее неравенство в одну сторону уже есть, остаётся в другую.
	
	Зафиксируем $\eps > 0$. Поскольку $B, C \in M$, то $\exists B_\eps, C_\eps \in R(S)$ - приближения по определению лебеговой измеримости. Увидим следующее включение:
	\[
		A \tr (B_\eps \cup C_\eps) \subseteq (B \tr B_\eps) \cup (C \tr C_\eps) \Ra \mu^*(A \tr (B_\eps \cup C_\eps)) \le 2\eps
	\]
	Дополнительно воспользуемся следствием из полуаддитивности для оценки снизу $\mu^*(A)$:
	\[
		\mu^*(A) \ge \mu^*(B_\eps \cup C_\eps) - \mu^*(A \tr (B_\eps \cup C_\eps)) \ge \mu^*(B_\eps \cup C_\eps) - 2\eps
	\]
	Тут заметим, что $B_\eps, C_\eps \in R(S)$, стало быть и $B_\eps \cup C_\eps \in R(S)$. Иначе говоря, для этих множеств $\mu^* = \nu$ --- индуцированная мера (реальная!). Значит, мы можем воспользоваться формулой включений и исключений:
	\[
		\mu^*(A) \ge \underbrace{\mu^*(B_\eps) + \mu^*(C_\eps) - \mu^*(B_\eps \cap C_\eps)}_{\mu^*(B_\eps \cup C_\eps)} - 2\eps
	\]
	Аналогичной оценкой через следствие полуаддитивности мы пользуемся для $\mu^*(B_\eps), \mu^*(C_\eps)$:
	\begin{multline*}
		\mu^*(A) \ge \mu^*(B) - \mu^*(B \tr B_\eps) + \mu^*(C) - \mu^*(C \tr C_\eps) - \mu^*(B_\eps \cap C_\eps) - 2\eps \ge
		\\
		\mu^*(B) + \mu^*(C) - \mu^*(B_\eps \cap C_\eps) - 4\eps
	\end{multline*}
	Осталось показать, что $\mu^*(B_\eps \cap C_\eps)$ достаточно малая величина. Для этого снова заметим вложение:
	\[
		B_\eps \cap C_\eps \subseteq (B_\eps \bs B) \cup (C_\eps \bs C) \subseteq (B \tr B_\eps) \cup (C \tr C_\eps) \Ra \mu^*(B_\eps \cap C_\eps) \le 2\eps
	\]
	Итого $\mu^*(A) \ge \mu^*(B) + \mu^*(C) - 6\eps$ для $\forall \eps > 0$, что и требовалось.
\end{proof}

\begin{definition}
	$\mu^*$ на $M$ называется \textit{мерой Лебега} и обозначается $\mu$.
\end{definition}

\begin{theorem}
	$M$ является $\sigma$-алгеброй.
\end{theorem}

\begin{proof}
	Пусть $\{\cA_i\}_{i = 1}^\infty \subseteq M$ и $\cA := \bigcup_{i = 1}^\infty \cA_i$. В силу того, что $M$ уже алгебра, мы можем заменить $\cA_i$ на $B_i \in M$ так, что $\cA = \bscup_{i = 1}^\infty B_i$.
	\begin{enumerate}
		\item $\forall n \in \N\ \bscup_{i = 1}^n B_i \subseteq \cA$. Тогда между мерами есть соотношение следующего вида:
		\[
			\sum_{i = 1}^n \mu(B_i) = \mu\ps{\bscup_{i = 1}^n B_i} = \mu^*\ps{\bscup_{i = 1}^n B_i} \le \mu^*(A)
		\]
		Отсюда следует, что $\sum_{i = 1}^\infty \mu(B_i) \le \mu^*(A) < \infty$. Благодаря этому ряд сходится.
		
		\item В частности, нас интересует следующее свойство сходимости ряда:
		\[
			\forall \eps > 0\ \exists N \in \N \such \sum_{i = N + 1}^\infty \mu(B_i) < \frac{\eps}{2}
		\]
		Для основной части мы можем воспользоваться определением измеримости (ибо конечное дизъюнктное объединение тоже измеримо по Лебегу):
		\[
			\forall \eps > 0\ \exists C_{\eps / 2} \in M \such \mu\ps{C_{\eps / 2} \tr \bscup_{i = 1}^N B_i} < \eps / 2
		\]
		Дело остаётся за малым --- нашим кандидатом на приближение является $C_{\eps / 2}$, нужно проверить меру симметрической разности с $\cA$:
		\begin{multline*}
			\cA \tr C_{\eps / 2} \subseteq \ps{C_{\eps / 2} \tr \bscup_{i = 1}^N B_i} \cup \ps{\bscup_{i = N + 1}^\infty B_i} \Longrightarrow
			\\
			\mu^*(\cA \tr C_{\eps / 2}) \le \mu^*\ps{C_{\eps / 2} \tr \bscup_{i = 1}^N B_i} + \underbrace{\mu^*\ps{\bscup_{i = N + 1}^\infty B_i}}_{\le \sum_{i = N + 1}^\infty \mu^*(B_i)} < \frac{\eps}{2} + \frac{\eps}{2} = \eps
		\end{multline*}
	\end{enumerate}
\end{proof}

\begin{theorem}
	$\mu$ --- $\sigma$-аддитивная мера на $M$.
\end{theorem}

\begin{proof}
	Пусть $\cA \in M, \{\cA_i\}_{i = 1}^\infty \subseteq M$, причём $\cA = \bscup_{i = 1}^\infty \cA_i$. Тогда нам нужно доказать лишь $\mu(\cA) \ge \sum_{i = 1}^\infty \mu(\cA_i)$. Заметим такую вещь: так как $M$ теперь $\sigma$-алгебра, то
	\[
		\forall n \in \N \quad \cA = A_1 \sqcup \ldots \sqcup A_n \sqcup \ps{\bscup_{i = n + 1}^\infty A_i} \wedge \bscup_{i = n + 1}^\infty A_i \in M
	\]
	Отсюда по аддитивности:
	\[
		\forall n \in \N \quad \mu(\cA) = \sum_{i = 1}^n \mu(\cA_i) + \mu\ps{\bscup_{i = n + 1}^\infty A_i} \ge \sum_{i = 1}^n \mu(\cA_i)
	\]
	устремляя $n$ в бесконечность, получаем требуемое неравенство.
\end{proof}

\begin{note}
	Подведём итоги проделанной работы:
	\begin{align*}
		&{\text{$S$ с единицей $E$} \to R(S) \to M \text{--- $\sigma$-алгебра}}
		\\
		&{\text{$m$ --- $\sigma$-аддитивная мера} \to \nu \to \mu \text{--- $\sigma$-аддитивная мера}}
	\end{align*}
	Мы смогли продолжить $m$ на $\sigma(S)$. Почему? Ну просто потому что $\sigma(S) \subseteq M$ по определению. Возникает вопрос: <<А единственна ли мера, определенная на $\sigma(S)$?>> Окажется, что да.
\end{note}

\begin{example} (мера Бореля)
	Борелевской $\sigma$-алгеброй на отрезке $[a; b]$ будет наименьшая $\sigma$-алгебра, содержащая все открытые множества из $[a; b]$:
	\[
		\B_{[a; b]} = \sigma(\{A \subset [a;b] \such A \text{--- открытое}\}) = \sigma(\{\tbr{c; d} \such a \le c \le d \le b\})
	\]
	Мы уже показывали, что последняя система множеств являеся полукольцом с единицей $E = [a; b]$, $m(\tbr{c; d}) = d - c$ задаёт $\sigma$-аддитивную меру на этом полукольце, а потому построенная нами $\mu$ будет $\sigma$-аддитивной мерой на $\B_{[a; b]}$ --- \textit{мерой Бореля}.
\end{example}

\begin{example} (мера Лебега-Стилтьеса)
	Рассмотрим $\R = (-\infty; +\infty)$ и функцию $\phi(x)$, удовлетворяющую таким свойствам:
	\begin{enumerate}
		\item $\phi(x)$ неубывающая
		
		\item $\phi(x)$ непрерывна справа (в любой точке и в $+\infty$)
		
		\item $\phi(x)$ ограничена
	\end{enumerate}
	Положим за полукольцо $S$ следующую систему множеств (про которую, конечно же, читатель должен доказать свойства полукольца):
	\[
		S = \{\emptyset\} \cup \Big\{(a; b] \such a \in \R \cup \{-\infty\}, b \in \R\Big\} \cup \Big\{(a; +\infty) \such a \in \R \cup \{-\infty\}\Big\}
	\]
	Зададим меру на $S$ таким образом:
	\begin{align*}
		&{m(\emptyset) = 0}
		\\
		&{m((a; b]) = \phi(b) - \phi(a)}
		\\
		&{m((a; +\infty)) = \phi(+\infty) - \phi(a)}
	\end{align*}
	То, что этим задана хотя бы мера, должно быть очевидно (ну или домашнее задание читателю), а вот $\sigma$-аддитивность неясна. Для этого, мы отдельно разберём конечный и бесконечный случаи:
	\begin{enumerate}
		\item $(a; b] = \bscup_{i = 1}^\infty (a_i; b_i]$. У нас уже есть неравенство в одну сторону:
		\[
			m(\rsi{a; b}) \ge \sum_{i = 1}^\infty \rsi{a_i; b_i}
		\]
		Остаётся показать неравенство в обратную. Будем отталкиваться от желания воспользоваться компактностью --- надо взять некоторое замкнутое подмножество $\rsi{a; b}$ и найти его покрытие. Возьмём пока произвольные $a < d \le b$ и $c_i > b_i$. Это даст включение $[d; b] \subset \bigcup_{i = 1}^\infty (a_i; c_i)$. Чтобы выбрать уже конкретные значения, воспользуемся свойствами $\phi$:
		\begin{align*}
			&{\forall \eps > 0\ \exists a < d \le b \such \phi(d) - \phi(a) < \frac{\eps}{2}}
			\\
			&{\forall i \in \N\ \forall \eps > 0\ \exists c_i > b_i \such \phi(c_i) - \phi(b_i) < \frac{\eps}{2^{i + 1}}}
		\end{align*}
		В силу компактности отрезка, $\exists N \in \N \such [d; b] \subset \bigcup_{i = 1}^N \rsi{a_i; c_i}$. Однако, отрезок в нашей системе неизмерим, а поэтому выполним такой трюк:
		\[
			\rsi{d; b} \subset [d; b] \subset \bigcup_{i = 1}^N \rsi{a_i; c_i}
		\]
		А теперь будем писать цепочку неравенств:
		\begin{multline*}
			m(\rsi{a; b}) = \phi(b) - \phi(a) < \phi(b) - \phi(d) + \frac{\eps}{2} = m(\rsi{d; b}) + \frac{\eps}{2} \le \sum_{i = 1}^N m(\rsi{a_i; c_i}) + \frac{\eps}{2} <
			\\
			\sum_{i = 1}^\infty (\phi(c_i) - \phi(a_i)) + \frac{\eps}{2} < \sum_{i = 1}^\infty \ps{\phi(b_i) - \phi(a_i) + \frac{\eps}{2^{i + 1}}} + \frac{\eps}{2} = \sum_{i = 1}^\infty m(\rsi{a_i; b_i}) + \eps
		\end{multline*}
		Устремляя $\eps$ в ноль, получим требуемое неравенство.
		
		\item $(a; +\infty) = \bscup_{i = 1}^\infty A_i$, $A_i \in S$. Для начала, заметим такой факт:
		\[
			m((a; +\infty)) = \phi(+\infty) - \phi(a) = \lim_{n \to \infty} \phi(n) - \phi(a) = \lim_{n \to \infty} (\phi(n) - \phi(a)) = \lim_{n \to \infty} m(\rsi{a; n})
		\]
		Далее есть ещё один нереальный факт: $\rsi{a; n} = \bscup_{i = 1}^\infty \rsi{a; n} \cap A_i$. Так как у нас полукольцо, то эти пересечения тоже лежат в $S$, причём будут полуинтервалами. По первому пункту это даёт $\sigma$-аддитивность мер, ну а в частности $\sigma$-полуаддитивность:
		\[
			m(\rsi{a; n}) \le \sum_{i = 1}^\infty m(\rsi{a; n} \cap A_i)
		\]
		Коль скоро это работает при всех $n \ge a$, то верно и предельное неравенство. Значит
		\[
			m((a; +\infty)) = \lim_{n \to \infty} m(\rsi{a; n}) \le \lim_{n \to \infty} \sum_{i = 1}^\infty m(\rsi{a; n} \cap A_i) \le \sum_{i = 1}^\infty m(A_i)
		\]
		
		\item $\rsi{-\infty; b} = \bscup_{i = 1}^\infty A_i$. Аналогично предыдущему, просто надо рассматривать полуинтервалы $\rsi{-n; b}$ для $n \in \N$.
	\end{enumerate}
\end{example}