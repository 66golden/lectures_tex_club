\begin{definition}
	Мера $\mu$ на кольце $R$ называется \textit{полной}, если для любого $A \in R \colon \mu(A) = 0$ верно, что
	\[
		\forall B \subset A \Ra B \in R \wedge \mu(B) = 0
	\]
\end{definition}

\begin{note}
	Если в определении выше $B \in R$, то уже автоматически $0 \le \mu(B) \le \mu(A) \le 0$, поэтому требование на меру излишне.
\end{note}

\begin{proposition}
	Имеет место 2 факта:
	\begin{enumerate}
		\item Мера Лебега полна
		
		\item Мера Бореля неполна
	\end{enumerate}
\end{proposition}

\begin{proof}
	Доказательство проведём для случая, когда меры Лебега и Бореля строятся по $\sigma$-конечной мере, как более сложный случай. Поэтому пусть $E$ - это объемлющее множество, $S$ - полукольцо на $E$ такое, что $E \notin S$, а также $m$ --- это $\sigma$-конечная мера на этом полукольце.
	\begin{enumerate}
		\item По уже сказанному в этом параграфе
		\[
			E = \bscup_{k = 1}^\infty B_k,\ B_k \in R(S),\ \nu(B_k) < \infty
		\]
		По этим данным мы строим меру Лебега $\mu$ --- $\sigma$-конечная мера на $M$. Возьмём $A \in M$, $\mu(A) = 0$ и произвольное $B \subset A$. Тогда, зафиксировав произвольный $\eps > 0$, мы всегда можем взять $B_\eps = \emptyset \in R(S)$. По определению измеримости
		\[
			\forall j \in \N\ \ A \cap B_j \in M_j
		\]
		Точно такое же утверждение надо доказать для $B$. Для этого заметим вложение $(B \cap B_j) \subset (A \cap B_j)$, а так как $B_j$ является единицей осоответствующего кольца с мерой Лебега $\mu_j$, то
		\[
			\forall j \in \N \quad 0 \le \mu_j^*(B \cap B_j) \le \mu_j^*(A \cap B_j) = \mu_j(A \cap B_j) \le \mu(A) = 0
		\]
		Стало быть, $B \cap B_j \in M_j$ для любого $j \in \N$, ну а тогда по определению $B \in M$.
		
		\item \textcolor{red}{Потенциально появится ниже, здесь лектор решил пока доказательство не давать.}
	\end{enumerate}
\end{proof}

\begin{reminder}
	Пусть $\lambda$ --- это классическая мера Лебега на $\R$. Тогда $\forall A \subseteq [0; 1]$ существует неизмеримое подмножество $F \subseteq A$.
\end{reminder}

\begin{note}
	По сути это построение множества Витали. Подробное доказательство можно найти в конспекте по математическому анализу А. Л. Лукашова
\end{note}

\begin{theorem} (Структура измеримых множеств)
	Пусть $\mu$ --- это $\sigma$-конечная мера Лебега на $\sigma$-алгебре $M$, полученная продолжением $\sigma$-конечной меры $m$ полукольца $S$. Тогда, любое множество $A \in M$, $\mu(A) < \infty$ можно записать в таком виде:
	\[
		A = \bigcap_{i = 1}^\infty \bigcup_{j = 1}^\infty A_{i, j} \bs A_0
	\]
	где
	\begin{itemize}
		\item $A_{i, j} \in R(S)$
		
		\item $\forall i \in \N \quad A_{i, 1} \subseteq A_{i, 2} \subseteq \ldots$
		
		\item $B_i := \bigcup_{j = 1}^\infty A_{i, j}$, тогда $B_1 \supseteq B_2 \supseteq \ldots$
		
		\item $A_0 \in M$, $\mu(A_0) = 0$
	\end{itemize}
\end{theorem}

\begin{proof}
	В силу того, что $A \in M$ и $S \subseteq M$, мы можем потребовать из определения верхней меры Лебега такое свойство:
	\[
		\forall i \in \N\ \exists C_i = \bigcup_{j = 1}^\infty D_{i, j}, D_{i, j} \in S \such \underbrace{\mu(C_i) < \mu(A) + \frac{1}{i}}_{\mu(C_i \bs A) < 1 / i}
	\]
	За $B_i$ положим $B_i = \bigcap_{r = 1}^i C_r$. Эту невозрастающую последовательность тоже можно записать через счётное объединение элементов полукольца:
	\[
		B_i = \bigcup_{j_1 = 1}^\infty \cdots \bigcup_{j_i = 1}^\infty D_{i, j_1} \cap \ldots \cap D_{i, j_i} = \bigcup_{j = 1}^\infty E_{i, j}, E_{i, j} \in S
	\]
	Здесь мы явно записали пересечение всех $C_r$, образующих $B_i$, как всевозможные пересечения их кусочков. Конечное декартово счётных множеств счётно, а потому после перенумерации $E_{i, j}$ (в $j$ закодированы конкретные значения $j_1, \ldots, j_r$) будут тоже элементами полукольца. Заметим, что $\mu(B_1) = \mu(C_1) < \infty$, а поэтому можно применить непрерывность меры:
	\[
		\mu\ps{\bigcap_{i = 1}^\infty B_i \bs A} = \lim_{n \to \infty} \mu\ps{\bigcap_{i = 1}^n B_i \bs A}
	\]
	Если мы докажем, что мера множества слева равна нулю, то мы его смело кладём за $A_0 \in M$ из условия теоремы и побеждаем. Для этого произведём оценку того, что у нас в пределе:
	\[
		\mu\ps{\bigcap_{i = 1}^n B_i \bs A} \le \mu(B_n \bs A) \le \mu(C_n \bs A) < \frac{1}{n}
	\]
	Ну и всё, значит, предел равен нулю. Однако, $E_{i, j}$ при фиксированном $i$ не удовлетворяют условиям теоремы. Чтобы обойти этот момент, надо положить $A_{i, j} = \bigcup_{r = 1}^j E_{i, r}$. Очевидно, что объединение таких множеств по $j$ всё ещё будет $B_i$, они точно лежат в $R(S)$ и образуют неубывающую последовательность.
\end{proof}

\begin{theorem} (Каратеодори)
	Пусть $m$ --- это $\sigma$-конечная мера  на полукольце $S$ подмножеств $E$. Тогда утверждается, что существует и единственна $\mu$ --- $\sigma$-конечная мера на $\sigma(S)$, согласованная с $m$.
\end{theorem}

\begin{note}
	Согласованность означает, что $\forall A \in S\ \mu(A) = m(A)$. Отдельно отметим, что теорема Каратеодори в такой формулировке является слабой. Это сделано для того, чтобы не переусложнять материал.
\end{note}

\begin{proof}~
	\begin{itemize}
		\item Существование. Мы уже построили $\mu$ --- мера Лебега, согласованная с $m$, на $\sigma$-алгебре $M$ --- в данном случае множество измеримых подмножеств $E$. Так как $\sigma(S) \subseteq M$, то ограничение $\mu$ на $\sigma(S)$ даёт требуемое.
		
		\item Единственность. Мы работаем с $\sigma$-конечной мере в $\sigma$-алгебре, а это даёт такой факт:
		\[
			E = \bscup_{i = 1}^\infty B_i,\ B_i \in R(S) \wedge \mu(B_i) < \infty
		\]
		Предположим, что единственность неверна. Тогда $\exists A \in \sigma(S) \such \mu(A) \neq \mu'(A)$. Рассмотрим случай, когда $\mu(A) < \infty$ (в противном случае мы пользуемся рассуждениями ниже для всех $A \cap B_i \in R(S) \subseteq \sigma(S)$ и этого достаточно). Это позволяет применить теорему о структуре измеримого множества:
		\[
			A = \bigcap_{i = 1}^\infty \bigcup_{j = 1}^\infty A_{i, j} \bs A_0
		\]
		где $A_0 \in \sigma(S)$ тоже. Это уже ставит вопрос: <<Почему $\mu'(A_0) = 0$?>> Воспользуемся свойством определения меры Лебега:
		\[
			\forall \eps > 0\ \exists \underbrace{\bigcup_{i = 1}^\infty C_i}_{\in \sigma(S)} \supseteq A_0 \such \sum_{i = 1}^\infty m(C_i) < \eps
		\]
		При этом $\mu'$ обладает $\sigma$-полуаддитивностью. Стало быть, есть неравенство
		\[
			\mu'(A_0) \le \sum_{i = 1}^\infty \mu'(C_0) = \sum_{i = 1}^\infty m(C_0) < \eps \Lora \mu'(A_0) = 0
		\]
		Теперь, вернёмся к мере $\mu'(A)$:
		\[
			\mu'(A) = \mu'\ps{\bigcap_{i = 1}^\infty \bigcup_{j = 1}^\infty A_{i, j} \bs A_0} = \mu'\ps{\bigcap_{i = 1}^\infty \bigcup_{j = 1}^\infty A_{i, j}} = \lim_{i \to \infty} \lim_{j \to \infty} \mu'(A_{i, j})
		\]
		В последнем выражении $\mu'(A_{i, j}) = \mu(A_{i, j})$, так как $A_{i, j} \in R(S)$, а для минимального кольца мы уже установили единственность согласованной меры ранее. Повторяя все равенство в обратную сторону для другой меры, установим равенство $\mu(A) = \mu'(A)$, противоречие.
	\end{itemize}
\end{proof}