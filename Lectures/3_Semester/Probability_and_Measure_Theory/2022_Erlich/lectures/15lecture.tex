\subsubsection*{Построение интеграла Лебега для измеримой функции}

\begin{note}
	Неотрицательность функции в этом контексте также подразумевает её измеримость.
\end{note}

\begin{definition}
	Пусть $f \colon E \to \R$ --- неотрицательная функция на $E \in M$. Определим множество $Q_f$:
	\[
		Q_f = \{h \colon h(x) \le f(x) \text{ на } E \wedge h(x) \text{ --- простая функция}\}
	\]
	Тогда \textit{интегралом Лебега от неотрицательной функции на $E$} будет называться такое число:
	\[
		(L)\int_E f(x)d\mu = \sup_{h \in Q_f} (L)\int_E h(x)d\mu
	\]
\end{definition}

\begin{definition}
	Если для неотрицательной функции $(L)\int_E f(x)d\mu < +\infty$, то $f(x)$ называется \textit{интегрируемой по Лебегу на $E$} и это обозначают как $f \in L(E)$.
\end{definition}

\begin{proposition}
	Любая измеримая функция представима в виде разности двух неотрицательных.
\end{proposition}

\begin{proof}
	Пусть $f \colon E \to \R$ --- произвольная измеримая функция. Тогда определим такие неотрицательные функции:
	\begin{align*}
		&{f^+(x) = \max(f(x), 0)}
		\\
		&{f^-(x) = \max(-f(x), 0)}
	\end{align*}
	Тогда $f = f^+ - f^-$, что и требовалось.
\end{proof}

\begin{definition}
	Произвольная измеримая функция $f \colon E \to \R$ называется \textit{интегрируемой по Лебегу}, если $f^+, f^- \in L(E)$. Интеграл Лебега в таком случае по определению равен следующему:
	\[
		(L)\int_E f(x)d\mu = (L)\int_E f^+(x)d\mu - (L)\int_E f^-(x)d\mu
	\]
\end{definition}

\begin{proposition}
	Если $\{g_n\}_{n = 1}^\infty$ --- последовательность неубывающих неотрицательных простых функций на $E$ и имеет место утверждение:
	\[
		\forall x \in E\ \ \lim_{n \to \infty} g_n(x) = g(x)
	\]
	Тогда $g \in L(E)$ и имеет место интегральный предел:
	\[
		\lim_{n \to \infty} \int_E g_n(x)d\mu = \int_E g(x)d\mu
	\]
\end{proposition}

\begin{proof}
	Нужно разобраться с двумя вещами:
	\begin{enumerate}
		\item Измеримость $g$. Это просто следствие теоремы \ref{limtheorema}. Более того, $g$ будет неотрицательной в силу наличия этого свойства у всех $g_n$.
		
		\item Интегральное равенство. Его нужно показать в 2 стороны:
		\begin{enumerate}
			\item По сути $\forall n \in \N\ g_n \in Q_g$, а потому у нас будет как минимум такое соотношение из определения:
			\[
				\lim_{n \to \infty} \int_E g_n(x)d\mu \le \int_E g(x)d\mu
			\]
			
			\item Заметим такой несложный факт:
			\[
				\forall h \in Q_g\ \ g(x) = \lim_{n \to \infty} g_n(x) \ge h(x)
			\]
			Вместе с полученным неравенством мы удовлетворяем все требования последней теоремы про простые функции, откуда получаем неравенство:
			\begin{multline*}
				\forall h \in Q_g\ \ \lim_{n \to \infty} \int_E g_n(x)d\mu \ge \int_E h(x)d\mu \Lora
				\\
				\lim_{n \to \infty} \int_E g_n(x)d\mu \ge \sup_{h \in Q_g} \int_E h(x)d\mu = \int_E g(x)d\mu
			\end{multline*}
		\end{enumerate}
	\end{enumerate}
\end{proof}

\begin{lemma}
	Пусть $f \colon E \to \R$ --- неотрицательная функция. Тогда существует неубывающая последовательность простых функций $\{f_m\}_{m = 1}^\infty$ такая, что $\forall x \in E\ \ \lim_{m \to \infty} f_m(x) = f(x)$
\end{lemma}

\begin{proof}
	В силу $\sigma$-конечности пространства
	\[
		E = \bscup_{n = 1}^\infty E_n,\ \mu(E_n) < \infty
	\]
	Определим такую последовательность $f_m$:
	\[
		f_m(x) = \System{
			&{0,\ x \notin \bscup_{n = 1}^m E_n}
			\\
			&{2^m,\ x \in \bscup_{n = 1}^m E_n \wedge f(x) \ge 2^m}
			\\
			&{\frac{k - 1}{2^m},\ f(x) \in \lsi{\frac{k - 1}{2^m}; \frac{k}{2^m}},\ x \in \bscup_{n = 1}^m E_n \wedge 0 \le f(x) < 2^m}
		}
	\]
	Каждая $f_m$ проста, ибо мера носителя конечна и число значений равно $2^{2m}$ ($1 \le k \le 2^{2m}$ и ещё одно значение $2^m$). А сейчас мы будем очень грустно показывать неубывание этой последовательности $\forall x \in E$:
	\begin{itemize}
		\item Если $f_m(x) = 0$, то $f_{m + 1}(x) \ge 0$
		
		\item Если $f_m(x) = 2^m$, то $f(x) \ge 2^m$, а тогда либо $f_{m + 1}(x) \ge 2^m$, либо $f_{m + 1}(x) = 2^{m + 1} \ge 2^m$
		
		\item Если $f_m(x) = \frac{k}{2^m}$, то $f(x) \in \lsi{\frac{k}{2^m}; \frac{k + 1}{2^m}} = \lsi{\frac{2k}{2^{m + 1}}; \frac{2k + 2}{2^{m + 1}}}$. Следовательно, либо $f_{m + 1}(x) = \frac{2k}{2^{m + 1}}$, либо $f_{m + 1}(x) = \frac{2k + 1}{2^{m + 1}}$
	\end{itemize}
	Остаётся разобраться со сходимостью. Есть 2 варианта:
	\begin{itemize}
		\item $f(x) < +\infty$. В таком случае, $\exists m_0 \in \N \such x \in \bscup_{k = 1}^{m_0} E_k$, то есть $f_{m_0}(x) = k / 2^{m_0}$. Для $t > m_0$ имеет место неравенство:
		\[
			|f_t(x) - f(x)| < \frac{k}{2^t} - \frac{k - 1}{2^{t}} = \frac{1}{2^t} \xrightarrow[t \to \infty]{} 0
		\]
		
		\item $f(x) = +\infty$. Тогда снова $\exists m_0 \in \N \such x \in \bscup_{k = 1}^{m_0} E_k$ и для $\forall t > m_0$ будет верно равенство
		\[
			f_t(x) = 2^t \xrightarrow[t \to \infty]{} +\infty
		\]
	\end{itemize}
\end{proof}

\begin{theorem}
	Если $f, g \colon E \to \R$ --- неотрицательные функции, то есть 2 свойства:
	\begin{enumerate}
		\item (Линейность) \(\int_E (f + g)d\mu = \int_E fd\mu + \int_E gd\mu\)
		
		\item (Аддитивность) Если $E = A \sqcup B$, где $A, B \in M$, то \(\int_E fd\mu = \int_A fd\mu + \int_B fd\mu\)
	\end{enumerate}
\end{theorem}

\begin{proof}
	По сути мы просто ссылаемся на определение, лемму и уже доказанные свойства для простых функций:
	\begin{enumerate}
		\item Возьмём по лемме последовательности $\{f_n\}_{n = 1}^\infty$ и $\{g_n\}_{n = 1}^\infty$ соответственно. Тогда последовательность $f_n + g_n$ будет обладать всеми общими свойствами исходных, но при этом поточечно сходится к $f + g$. По утверждению о пределе интегралов получаем нужное равенство:
		\[
			\int_E (f + g)d\mu = \lim_{n \to \infty} \int_E (f_n + g_n)d\mu = \lim_{n \to \infty} \int_E f_nd\mu + \lim_{n \to \infty} \int_E g_nd\mu = \int_E fd\mu + \int_E gd\mu
		\]
		
		\item Аналогично первому пункту.
	\end{enumerate}
\end{proof}

\begin{theorem} (Абсолютная непрерывность интеграла Лебега, без доказательства)
	Пусть есть некоторая $f \colon E \to \R$. Имеет место несколько утверждений:
	\begin{enumerate}
		\item Если $\mu(E) = 0$ и $f$ измерима на $E$, то $f \in L(E)$ и $\int_E fd\mu = 0$
		
		\item Если $\mu$ полна, $f \in L(E)$ и $g \sim f$, то $g \in L(E)$ и $\int_E fd\mu = \int_E gd\mu$
		
		\item Если $f \in L(E)$, то $\mu\{x \in E \colon f(x) = \pm \infty\} = 0$
	\end{enumerate}
\end{theorem}

\begin{proof}~
	\begin{enumerate}
		\item Тривиально следует из определения.
		
		\item Положим $E_1 = \{x \in E \colon f(x) = g(x)\}$, тогда $\mu(E \bs E_1) = 0$. Без ограничения общности будем считать, что $f(x) \ge 0$ (иначе переходим к отдельному рассмотрению $f^+$ и $f^-$). Из этого факта уже
		\[
			\int_E fd\mu = \int_{E_1} fd\mu + \int_{E \bs E_1} fd\mu = \int_{E_1} gd\mu = \int_E gd\mu
		\]
		
		\item Без ограничения общности $f(x) \ge 0$ на $E$. Предположим противное:
		\[
			\mu(A) = \mu\{x \in E \colon f(x) = +\infty\} > 0
		\]
		Возможно 2 ситуации:
		\begin{enumerate}
			\item $\mu(A) < \infty$. Тогда рассмотрим последовательность $h_n(x) = n \cdot \chi_A(x)$. Понятно, что это неубывающие простые неотрицательные функции, причём $h_n(x) \le f(x)$. Стало быть, $h_n \in Q_f$. Но тогда по определению интеграла Лебега от неотрицательной функции
			\[
				(L)\int_E fd\mu = +\infty
			\]
			что противоречит с интегрируемостью $f$.
			
			\item $\mu(A) = \infty$. Выделим в $A$ подмножество конечной меры и проведём для него рассуждения выше.
		\end{enumerate}
	\end{enumerate}
\end{proof}

\begin{theorem} (Основные свойства интеграла Лебега)
	Если $f, g \in L(E)$, то имеют место следующие утверждения:
	\begin{enumerate}
		\item Для $\forall \alpha, \beta \in \R$ будет верно, что $\alpha f + \beta g \in L(E)$ и, более того, есть равенство:
		\[
			\int_E (\alpha f + \beta g)d\mu = \alpha \int_E fd\mu + \beta \int_E gd\mu
		\]
		
		\item $f \in L(E) \Lra |f| \in L(E)$
		
		\item \(\md{\int_E f(x)d\mu} \le \int_E |f(x)|d\mu\)
		
		\item Если $|g(x)| \le |f(x)|$ на $E$, то \(\int_E |g(x)|d\mu \le \int_E |f(x)|d\mu\)
		
		\item Если $f(x) \le g(x)$ на $E$, то \(\int_E fd\mu \le \int_E gd\mu\)
		
		\item Если $E = A \sqcup B$, то $\int_E fd\mu = \int_A fd\mu + \int_B fd\mu$
	\end{enumerate}
\end{theorem}

\begin{proof}~
	\begin{enumerate}
		\item Идёт без доказательства
		
		\item В одну сторону очевидно, в обратную тривиально: если $f = f^+ - f^-$, то $|f| = f^+ + f^-$
		
		\item Воспользуемся разложением $f = f^+ - f^-$:
		\begin{multline*}
			\md{\int_E f(x)d\mu} = \md{\int_E f^+(x)d\mu - \int_E f^-(x)d\mu} \le \md{\int_E f^+(x)d\mu} + \md{\int_E f^-(x)d\mu} =
			\\
			\int_E f^+(x)d\mu + \int_E f^-(x)d\mu = \int_E |f(x)|d\mu
		\end{multline*}
		
		\item \textcolor{red}{Пока не придумал}
		
		\item Заведём новую функцию $h(x) = g(x) - f(x)$. Тогда $h(x) = |h(x)|$ и $|h(x)| \ge 0$ на $E$. Следовательно
		\[
			\int_E 0d\mu = 0 \le \int_E |h(x)|d\mu = \int_E h(x)d\mu = \int_E g(x)d\mu - \int_E f(x)d\mu
		\]
		
		\item Просто пользуемся уже доказанным для $f^+$ и $f^-$
	\end{enumerate}
\end{proof}

\subsection{Сходимость интегралов Лебега}

\begin{note}
	Мы сохраняем условия на рассматриваемое пространство $(X, M, \mu)$ и функции в нём из предыдущей части.
\end{note}

\begin{note}
	Основной вопрос, который нас волнует: <<Если мы знаем, что некоторая последовательность $f_n$ как-то сходится к $f$ на $E$, то можно ли что-то утверждать про сходимость интегралов Лебега>>?
\end{note}

\begin{example}
	Рассмотрим $f_n = n^2 \chi_{\rsi{0; 1 / n}}$. Совершенно понятно, что и поточечно, и по мере эта последовательность сходится к нулю. Однако:
	\[
		\int_{[0; 1]} f_n(x)d\mu = n
	\]
	То есть сходимости интегралов какой-либо к нулю не наблюдается вообще.
\end{example}

\begin{theorem} (Беппо Леви)
	Пусть есть последовательность $\{f_n\}_{n = 1}^\infty$ и $f$ такие, что они определены на $E$ и выполнены условия:
	\begin{enumerate}
		\item $E \in M$
		
		\item $f_n$ неотрицательны на $E$
		
		\item $f_n$ образуют неубывающую последовательность при любом $x \in E$
		
		\item $f = \lim_{n \to \infty} f_n(x)$
	\end{enumerate}
	Тогда имеет место интегральный предел:
	\[
		\int_E f(x)d\mu = \lim_{n \to \infty} \int_E f_n(x)d\mu
	\]
\end{theorem}

\begin{proof}
	Индуктивно построим последовательность неотрицательных функций $\{g_n\}_{n = 1}^\infty$:
	\begin{align*}
		&{g_1(x) = f_1(x)}
		\\
		&{g_n(x) = f_n(x) - f_{n - 1}(x)}
	\end{align*}
	Дополнительно для каждой $g_n$ найдём её последовательность подходящих снизу на $E$ простых функций, обозначим их соответственно $\psi_{m, n}$. Теперь, перейдём к такой последовательности простых функций:
	\[
		F_m(x) = \sum_{n = 1}^m \psi_{m, n}(x)
	\]
	Исследуем свойства данной последовательности:
	\begin{enumerate}
		\item $F_m$ неубывает. Действительно:
		\[
			F_{m + 1} - F_m = \sum_{n = 1}^m (\psi_{m + 1, n}(x) - \psi_{m, n}(x)) + \psi_{m + 1, m + 1}(x) \ge 0
		\]
		
		\item $F_m$ мажорируется $f_m$. В самом деле:
		\[
			F_m(x) \le \sum_{n = 1}^m g_n(x) \le f_m(x) \le f(x)
		\]
		
		\item В силу определения есть такое неравенство:
		\[
			\forall N \in \N\ \ \lim_{m \to \infty} F_m(x) \ge \sum_{n = 1}^N \psi_{m, n}(x) = \sum_{n = 1}^N g_n(x) = f_N(x)
		\]
	\end{enumerate}
	Из последнего свойства также следует, что
	\[
		\lim_{m \to \infty} F_m(x) \ge f(x)
	\]
	Коль скоро это так, то $\lim_{m \to \infty} F_m(x) = f(x)$. Так как эта последовательность состоит из простых функций, то
	\[
		\int_E f(x)d\mu = \lim_{m \to \infty} \int_E F_m(x)d\mu
	\]
	При этом $0 \le F_m(x) \le f_m(x) \le f(x)$. Из монотонности интеграла Лебега получаем
	\[
		\int_E F_m(x)d\mu \le \int_E f_m(x)d\mu \le \int_E f(x)d\mu \Lora \lim_{m \to \infty} \int_E f_m(x)d\mu = \int_E f(x)d\mu
	\]
\end{proof}

\begin{corollary}
	Если имеется последовательность определённых на $E$ функций $\{f_n\}_{n = 1}^\infty$ и выполнены следующие свойства:
	\begin{enumerate}
		\item $E \in M$
		
		\item $\forall n \in \N\ f_n \in L(E)$
		
		\item $f_n$ обращуют неубывающую на $E$ последовательность
		
		\item $\exists c \in \R \such \sup_{n \in \N} \int_E f_n(x)d\mu \le c$
	\end{enumerate}
	Тогда $f = \lim_{n \to \infty} f_n \in L(E)$ и имеет место равенство:
	\[
		\int_E f(x)d\mu = \lim_{n \to \infty} \int_E f_n(x)d\mu
	\]
\end{corollary}

\begin{proof}
	Сведём следствие к уже доказанному. Рассмотрим последовательность $\psi_n(x) = f_n(x) - f_1(x) \ge 0$ на $E$. Тогда $\psi(x) = \lim_{n \to \infty} \psi_n(x) = f(x) - f_1(x)$ и эта последовательность удовлетворяет теореме Леви, отсюда
	\[
		\int_E \psi(x)d\mu = \lim_{n \to \infty} \int_E \psi_n(x)d\mu = \lim_{n \to \infty} \int_E f_n(x)d\mu - \int_E f_1(x)d\mu
	\]
	Коль скоро $\psi \in L(E)$ и $f_1 \in L(E)$, то и $f \in L(E)$. Так как мы обосновали существования интеграла от $f$, законно записать следующее разложение:
	\[
		\lim_{n \to \infty} \int_E f_n(x)d\mu - \int_E f_1(x)d\mu = \int_E \psi(x)d\mu = \int_E f(x)d\mu - \int_E f_1(x)d\mu
	\]
\end{proof}

\textcolor{red}{Осталась лемма Фату и теорема Лебега. Я очень постараюсь их дописать завтра утром}