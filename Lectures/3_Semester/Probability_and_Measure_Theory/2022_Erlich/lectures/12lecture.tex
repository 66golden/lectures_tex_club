\subsection{Измеримые функции}

\begin{note}
	Измеримые функции нужны нам для того, чтобы разрешить проблемы, возникающие со случайными величинами в теории вероятности.
\end{note}

\begin{definition}
	Тройка $(X, M, \mu)$, где
	\begin{itemize}
		\item $X$ --- произвольное множество
		
		\item $M$ --- $\sigma$-алгебра на $X$
		
		\item $\mu$ --- $\sigma$-аддитивная мера на $M$
	\end{itemize}
	называется \textit{измеримым пространством}.
\end{definition}

\begin{note}
	Стоит отметить, что некоторые свойства мы не гарантируем. Например, полноту - совсем необязательно любое подмножество множества нулевой меры будет тоже иметь меру 0 (или даже измеримо).
\end{note}

\begin{definition}
	Пусть $(X, M, \mu)$ --- измеримое пространство, $A \in M$. Функция $f \colon A \to \overline{\R}$ называется \textit{измеримой}, когда выполняется условие:
	\[
		\forall c \in \R \quad f^{-1}(\rsi{c; +\infty}) \in M
	\]
\end{definition}

\begin{lemma} (О прообразах измеримой функции)
	Пусть $f$ --- измеримая функция на измеримом пространстве $(X, M, \mu)$. Тогда $f^{-1}(\tbr{a; b}) \in M$, в том числе $f^{-1}(\{\pm\infty\})$.
\end{lemma}

\begin{proof}
	Идея состоит в том, чтобы выразить всё через уже имеющееся не более чем счётным числом операций:
	\begin{itemize}
		\item $\forall a < b \in \overline{\R} \quad f^{-1}(\rsi{a; b}) = f^{-1}(\rsi{a; +\infty}) \bs f^{-1}(\rsi{b; +\infty}) \in M$
		
		\item $\forall a < b, a \in \overline{R}, b \in \R \quad  f^{-1}((a; b)) = \bigcap_{n = 1}^\infty f^{-1}\ps{\rsi{a; b + \frac{1}{n}}} \in M$
		
		\item $\forall c \in \R \quad f^{-1}([-\infty; c]) \sqcup f^{-1}(\rsi{c; +\infty}) = X \in M \Lora f^{-1}([-\infty; c]) \in M$
		
		\item $\forall c \in \overline{\R} \quad f^{-1}(\lsi{-\infty; c}) = \bigcap_{n = 1}^\infty f^{-1}\ps{\sbr{-\infty; c + \frac{1}{n}}} \in M$
		
		\item $\forall a < b \in \overline{\R} \quad f^{-1}(\lsi{a; b}) = f^{-1}(\lsi{-\infty; b}) \bs f^{-1}(\lsi{-\infty; a}) \in M$
		
		\item \textcolor{red}{И так далее. Может быть допишу остальные случаи}
	\end{itemize}
\end{proof}

\begin{theorem}
	Пусть $(X, M, \mu)$ --- измеримое пространство, а $f$ --- измеримая функция на нём. Тогда
	\[
		\forall B \in \B(\R) \quad f^{-1}(B) \in M
	\]
\end{theorem}

\begin{note}
	Обычно именно это свойство измеримой функции берут за определение, но мы взяли усиленное. Особенность в том, что старый способ доказательства измеримости не пройдёт --- мы просто не знаем, как устроены множества борелевской сигма-алгебры. Потребуется новая техника: \textit{принцип подходящих множеств}.
\end{note}

\begin{definition}
	Зафиксируем измеримое пространство $(X, M, \mu)$ и какую-то функцию $f$ на нём. \textit{Системой подходящих множеств} $S$ называется следующее:
	\[
		S = \{A \subseteq \R \such f^{-1}(A) \in M\}
	\]
\end{definition}

\begin{proposition}
	Если задано измеримое пространство $(X, M, \mu)$ и измеримая функция $f$ на нём, то система подходящих множеств обладает следующими свойствами:
	\begin{itemize}
		\item $\R \in S$
		
		\item $S$ является $\sigma$-алгеброй
	\end{itemize}
\end{proposition}

\begin{proof}~
	\begin{itemize}
		\item Уже доказано выше, ибо $(-\infty; +\infty) = \R$
		
		\item Верны следующие равенства:
		\begin{align*}
			&{f^{-1}\ps{\bigcap_{i = 1}^\infty A_i} = \bigcap_{i = 1}^\infty f^{-1}(A_i)}
			\\
			&{f^{-1}\ps{\bigtriangleup_{i = 1}^\infty A_i} = \bigtriangleup_{i = 1}^\infty f^{-1}(A_i)}
		\end{align*}
	\end{itemize}
\end{proof}

\begin{proof} (теоремы)
	Что мы ещё можем сказать про $S$? Оно содержит по уже доказанному факту все интервалы из $\R$. То же самое делает $\B(\R)$, при этом борелевская $\sigma$-алгебра является минимальной. Стало быть, $\B(\R) \subseteq S$ и всё доказано автоматически.
\end{proof}

\begin{reminder}
	Если $A \subseteq \R$, то $A = \bscup_i (a_i; b_i)$, $a_i < b_i \in \overline{\R}$.
\end{reminder}

\begin{reminder}
	Если взять измеримое пространство $(\R, M_\lambda, \lambda)$, где $\lambda$ --- это классическая мера Лебега, то любая непрерывная функция $f \colon A \to \R$, определенная на открытом множестве $A \subseteq \R$ будет измеримой.
\end{reminder}

\begin{definition}
	Функция $f \colon \R \to \R$ называется \textit{борелевской}, если выполнено следующее условие:
	\[
		\forall B \in \B(\R) \quad f^{-1}(B) \in \B(\R)
	\]
\end{definition}

\begin{proposition}
	Любая непрерывная функция $f \colon A \to \R$, определённая на открытом множестве $A \subseteq \R$, является борелевской
\end{proposition}

\begin{proof}
	\textcolor{red}{Вроде очевидно, но может быть потом добавлю.}
\end{proof}

\begin{theorem}
	Пусть $f \colon A \to \R$ --- это измеримая и конечная функция на измеримом пространстве $(X, M, \mu)$, $\Ran f \subseteq G \subseteq \R$, где $G$ --- открытое множество, и ещё есть $g$ --- борелевская функция, определённая на $G$. Тогда $g \circ f$ является измеримой функцией на $A$.
\end{theorem}

\begin{proof}
	Достаточно зафиксировать произвольное $c \in \R$ и посмотреть прообраз $g \circ f$ на луче $(c; +\infty)$:
	\[
		(g \circ f)^{-1}(c; +\infty) = f^{-1}(\underbrace{g^{-1}(c; +\infty)}_{\in \B(\R)}) \in \B(\R)
	\]
\end{proof}

\begin{theorem} (Арифметические свойства измеримых функций)
	Если $f, g$ --- измеримые и конечные функции на измеримом пространстве $(X, M, \mu)$. Тогда имеют место следующие свойства:
	\begin{itemize}
		\item $\forall \alpha, \beta \in \R \quad \alpha \cdot f + \beta \cdot g$ --- измеримая и конечная функция
		
		\item $f \cdot g$ --- измеримая и конечная функция
		
		\item Если дополнительно $g(X) \centernot\ni 0$, то $f / g$ --- измеримая и конечная функция
	\end{itemize}
\end{theorem}

\begin{proof}
	\begin{itemize}
		\item Проведём доказательство в несколько пунктов:
		\begin{enumerate}
			\item $\forall a, c \in \R \quad a \cdot f + c$ --- измеримая и конечная функция. Тривиально
			
			\item Множество $A = \{x \in X \such f(x) > g(x)\}$ измеримо. Действительно, его можно записать таким образом:
			\[
				A = \{x \in X \such f(x) > g(x)\} = \bigcup_{n = 1}^\infty \left(\{x \in X \such f(x) > r_n\} \cup \{x \in X \such g(x) < r_n\}\right) \in M
			\]
			где $r_n \in \Q$ и мы обходим все рациональные числа $\R$.
			
			\item Достаточно рассмотреть случай $\alpha = \beta = 1$ в силу первого пункта. Распишем $\forall c \in \R$ измеримость по определению:
			\[
				\{x \in X \such f(x) + g(x) > c\} = \{x \in X \such f(x) > \underbrace{c - g(x)}_{g'(x)}\} \in M
			\]
		\end{enumerate}
	
		\item Воспользуемся трюком:
		\[
			f \cdot g = \frac{(f + g)^2 - (f - g)^2}{4}
		\]
		Это измеримая функция, потому что композицию и сумму функций мы уже обосновали ($f - g = f + (-g)$).
		
		\item То же самое, что и в предыдущем пункте:
		\[
			f / g = f \cdot (1 / g)
		\]
		А функция $1 / g = (1 / x) \circ g$
	\end{itemize}
\end{proof}

\begin{theorem} \label{limtheorema}
	Пусть $\{f_n\}_{n = 1}^\infty$ --- измеримая последовательность функций (с возможно бесконечными значениями) на измеримом пространстве $(X, M, \mu)$. Утверждается, что обе функции
	\begin{align*}
		&{\phi(x) = \sup_{n \in \N} f_n(x)}
		\\
		&{\psi(x) = \varlimsup_{n \to \infty} f_n(x)}
	\end{align*}
	являются измеримыми.
\end{theorem}

\begin{note}
	Естественно, то же самое верно для $\inf$ и $\varliminf$.
\end{note}

\begin{proof}
	Для $\phi(x)$ можно записать прообраз луча таким образом:
	\[
		\forall c \in \R \quad \{x \in X \such \phi(x) > c\} = \bigcup_{n = 1}^\infty \{x \in X \such f_n(x) > c\} \in M
	\]
	Для $\psi(x)$ надо воспользоваться эквивалентным определением верхнего предела. Для этого введём $\psi_k(x) = \sup_{n \ge k} f_n(x)$. Тогда $\psi(x) = \lim_{k \to \infty} \psi_k(x)$ и можно записать прообраз луча так:
	\[
		\{x \in X \such \psi(x) > c\} = \bigcup_{r = 1}^\infty \bigcap_{k = 1}^\infty \set{x \in X \such \psi_k(x) > c + \frac{1}{r}}
	\]
\end{proof}

\begin{corollary}
	В рамках условий теоремы, $F = \lim_{n \to \infty} f_n(x)$ будет измеримой функцией на том множестве, на котором существует предел (необязательно конечный).
\end{corollary}

\begin{proof}
	Мы уже доказали, что функции $\psi(x) = \varlimsup_{n \to \infty} f_n(x)$ и $\xi(x) = \varliminf_{n \to \infty} f_n(x)$ измеримы на $X$. Чтобы установить измеримость $F$, надо сначала показать измеримость множества её определения $A$, то есть
	\[
		A = \{x \in X \such \psi(x) = \xi(x)\} = X \bs \Big(\{x \in X \such \psi(x) > \xi(x)\} \sqcup \{x \in X \such \psi(x) < \xi(x)\}\Big) \in M
	\]
	Теперь заметим, что $(A, A \cap M, \mu)$ --- это тоже измеримое пространство, на котором $\psi$ и $\xi$ остаются измеримыми. Так как $F$ совпадает на $A$ с этими функциями, то она тоже измерима.
\end{proof}