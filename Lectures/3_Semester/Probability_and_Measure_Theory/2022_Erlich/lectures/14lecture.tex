\textcolor{red}{Тут начало 14й лекции. Покуда этот текст тут, далее я не гарантирую качество материала.}

\textcolor{red}{Есть понятие $\sigma$-конечного простанства. Надо дописать}

\begin{theorem} (<<Теорема Эрлиха>>, критерий сходимости по мере)
	Пусть задано измеримое пространство $(X, M, \mu)$, причём $\mu(X) < \infty$. Если $f_n, f$ --- это измеримые функции, то имеет место эквивалентность:
	\[
		f_n \Ra^\mu f \Longleftrightarrow \forall \{n_k\}_{k = 1}^\infty\ \exists \{n_{k_m}\}_{m = 1}^\infty \such f_{n_{k_m}} \xrightarrow[m \to \infty]{} f \text{ почти всюду}
	\]
\end{theorem}

\textcolor{red}{Дописать, откуда куда действуют функции}

\begin{proof}
	Проведём доказательство в каждую из сторон по отдельности:
	\begin{itemize}
		\item $\Ra$ Так как $f_n \Ra f$, то по теореме Рисса уже следует требуемое.
		
		\item $\La$ Предположим противное, то есть $f_n \centernot\Ra f$. По определению сходимости по мере это означает
		\[
			\exists \eps_0 > 0\ \such \delta_0 > 0\ \such \exists \{n_k\}_{k = 1}^\infty
		\]
		\textcolor{red}{Ничего не понял по кванторам}
	\end{itemize}
\end{proof}

\begin{definition}
	Измеримые функции $f, g$ на измеримом пространстве $(X, M, \mu)$ называются \textit{эквивалентными}, если выполнено равенство:
	\[
		\mu\{x \in X \colon f(x) \neq g(x)\} = 0
	\]
\end{definition}

\begin{note}
	Вполне справедливо ждать от сходимости как \textit{явления} следующих свойств:
	\begin{enumerate}
		\item Единственность предела с точностью до эквивалентности измеримых функций
		
		\item Линейность сходимости
		
		\item Непрерывность композиции
		
		\item Непрерывность произведения и деления (второе с оговоркой про нули)
	\end{enumerate}
\end{note}

\begin{theorem}
	Для поточечной сходимости почти всюду верны все свойства
\end{theorem}

\begin{proof}
	\textcolor{red}{В одну сторону очевидно, в обратную тривиально.}
\end{proof}

\begin{note}
	Верны ли все свойства для сходимости по мере? Оказывается, что не всё так просто, это показывает следующий пример.	
\end{note}

\begin{example}
	Рассмотрим измеримое пространство $(\R, M, \mu)$ ($M$ --- измеримые множества, $\mu$ --- классическая мера Лебега). Проверим выполнение непрерывности композиции на последовательности $f_n = x + 1 / n$. Она, идейно, должна сходится к $f(x) = x$, что и наблюдается:
	\[
		\forall \eps > 0\ \ \mu\{x \in X \colon |f_n(x) - f(x)| > \eps\} \xrightarrow[n \to \infty]{} 0
	\]
	Посмотрим композицию с непрерывной функцией $h(x) = x^2$. Тогда
	\[
		h(f_n(x)) = x^2 + 2x\frac{1}{n} + \frac{1}{n^2}
	\]
\end{example}

\begin{theorem}
	Если задано измеримое пространство $(X, M, \mu)$ и верно, что $\mu(X) < \infty$, то для сходимости по мере верны все свойства из замечания.
\end{theorem}

\begin{theorem}
	Если задано измеримое пространство $(X, M, \mu)$, то для сходимости по мере верны первые 2 свойства из замечания.
\end{theorem}

\begin{proof}
	\begin{enumerate}
		\item Пусть $f_n \Ra f$ и $f_n \Ra g$. Заметим такое вложение:
		\begin{multline*}
			\forall \eps > 0\ \forall n \in \N\ \ \{x \in X \colon |f(x) - g(x)| > \eps\} \subseteq
			\\
			\set{x \in X \colon |f(x) - f_n(x)| > \frac{\eps}{2}} \cup \set{x \in X \colon |f_n(x) - g(x)| > \frac{\eps}{2}}
		\end{multline*}
	\end{enumerate}
\end{proof}

\subsection{Интеграл Лебега}

\begin{note}
	Интеграл Лебега можно вводить разными методами. Конкретно у нас, зафиксируем $(X, M, \mu)$ --- $\sigma$-конечное пространство. Более того, любая рассматриваемая функция --- измеримая.
\end{note}

\begin{definition}
	\textit{Носителем функции} $f \colon D \to \R$ называется множество точек, где она принимает ненулевые значения.
\end{definition}

\begin{definition}
	Функция $f$ называется \textit{простой}, если она представима в следующем виде:
	\[
		f(x) = \sum_{k = 1}^n c_k \chi_{E_k} (x)
	\]
	при этом $c_k \neq 0$, $\mu(E_k) < \infty$ и $E_i \cap E_j = \emptyset$
\end{definition}

\begin{note}
	Если $f$ проста, то её носитель имеет конечную меру.
\end{note}

\begin{definition}
	\textit{Интегралом Лебега} для простой функции $f$ на множестве $X$ называется такая величина:
	\[
		(L)\int_X fd\mu := \sum_{k = 1}^n c_k \mu(E_k)
	\]
\end{definition}

\begin{note}
	Буковку $(L)$ обычно опускают.
\end{note}

\begin{note}
	Наше определение самую малость безосновательно: почему разные суммы дают одинаковый результат? Это необходимо обосновать.
\end{note}

\begin{lemma}
	Для любой простой функции $f$ существует \textit{каноническое представление}, при котором
	\[
		f = \sum_{k = 1}^n c_k \chi_{E_k}(x)
	\]
	где $c_i \neq c_j$ и $c_1 < \ldots < c_n$.
\end{lemma}

\begin{proof}
	\textcolor{red}{Дописать}
\end{proof}

\begin{proposition}
	Определение интеграла Лебега для простой функции корректно.
\end{proposition}

\begin{proof}
	\textcolor{red}{Дописать}
\end{proof}

\begin{theorem}
	Если $f$ и $g$ --- простые функции, то для интеграла Лебега имеет место \textit{линейность}:
	\[
		\forall \alpha, \beta \in \R\ \ \int_X (af(x) + bg(x))d\mu = a \int_X fd\mu + b \int_X gd\mu
	\]
\end{theorem}

\begin{proof}
	Есть 2 факта, которые нужно объяснить:
	\begin{enumerate}
		\item Простота функции-линейной комбинации. 
		
		\item Равенство интегралов.
	\end{enumerate}
\end{proof}

\begin{theorem} (Свойства интеграла Лебега от простой функции)
	Пусть $f$ и $g$ --- простые функции. Тогда имеют место такие свойства:
	\begin{enumerate}
		\item Если $f \ge 0$, то $\int_X f(x)d\mu \ge 0$
		
		\item Если $f \ge g$, то $\int_X f(x)d\mu \ge \int_X g(x)d\mu$
		
		\item $\left|\int_X f(x)d\mu\right| \le \int_X |f(x)|d\mu$
		
		\item Если $X = A \sqcup B$, где $A, B \in M$, тогда
		\[
			\int_X fd\mu = \int_A fd\mu + \int_B fd\mu
		\]
	\end{enumerate}
\end{theorem}

\begin{theorem}
	Пусть есть такие условия:
	\begin{enumerate}
		\item $E \in M$
		
		\item $g$ --- простая функция
		
		\item $\{g_n\}_{n = 1}^\infty$ --- неубывающая на $E$ последовательность простых функций, то есть
		\[
			\forall n \in \N\ \forall x \in E\ \ g_n(x) \le g_{n + 1}(x)
		\]
		
		\item $\forall x \in E\ \ \lim_{n \to \infty} g_n(x) \ge g(x)$ (допускается бесконечное значение предела)
	\end{enumerate}
	Тогда имеется неравенство:
	\[
		\lim_{n \to \infty} \int_E g_n(x)d\mu \ge \int_E g(x)d\mu
	\]
\end{theorem}

\begin{proof}
	Если предел интегралов бесконечен, то теорема очевидна. Рассмотрим конечный случай. Запишем каноническое разложение $g$:
	\[
		g(x) = \sum_{k = 1}^N a_k \chi_{E_k}
	\]
	где, из определения и условия, $0 < a_1 < \ldots < a_N$. Обозначим $F = \bscup_{k = 1}^N E_k$. Так как мы объединяем конечное число непересекающихся множеств, 
\end{proof}