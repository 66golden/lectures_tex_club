\subsection{Предельные теоремы}

\begin{theorem} (Неравенство Маркова)
	Если $\xi \ge 0$ --- случайная величина и существует её математическое ожидание, то
	\[
		\forall \eps > 0 \quad P\{\xi \ge \eps\} \le \frac{\E\xi}{\eps}
	\]
\end{theorem}

\begin{proof}
	Зафиксируем $\eps > 0$ и распишем математическое ожидание:
	\[
		\E\xi = \sum_{w \in \Omega} P(w)\xi(w) = \sum_{w \colon \xi(w) \ge \eps} P(w)\xi(w) + \sum_{w \colon \xi(w) < \eps} P(w)\xi(w) \ge \sum_{w \colon \xi(w) \ge \eps} \xi(w)\eps = P\{\xi \ge \eps\} \cdot \eps
	\]
\end{proof}

\begin{corollary} (Неравенство Чебышёва)
	Пусть дана случайная величина $\xi$, для которой существует дисперсия $D\xi$. Тогда
	\[
		\forall \eps > 0 \quad P\{|\xi - \E\xi| \ge \eps\} \le \frac{D\xi}{\eps^2}
	\]
\end{corollary}

\begin{proof}
	Введём случайную величину $\eta = (\xi - \E\xi)^2 \ge 0$. Тогда $\E\eta = D\xi$ и, соответственно, это матожидание существует. По неравенству Маркова
	\[
		\forall \eps > 0 \quad P\{\eta \ge \eps^2\} \le \frac{\E\eta}{\eps^2}
	\]
	При этом $\{\eta \ge \eps^2\} = \{|\xi - \E\xi| \ge \eps\}$, а $\E\eta / \eps^2 = D\xi / \eps^2$.
\end{proof}

\begin{theorem} (Закон Больших Чисел, сокращается как ЗБЧ)
	Пусть $\xi_1, \ldots, \xi_n$ --- независимые одинаково распределенные случайные величины, а также $\exists D\xi_1,\ \E\xi_1 = a$ (этого достаточно, чтобы те же характеристики с теми же значениями существовали у остальных случайных величин, в силу одинакового распределения). Тогда
	\[
		\forall \eps > 0 \quad P\left\{\md{\frac{\xi_1 \plusdots \xi_n}{n} - a} \ge \eps\right\} \xrightarrow[n \to \infty]{} 0
	\]
\end{theorem}

\begin{note}
	Де-факто закон утверждает следующее: если мы проведём достаточно большую серию экспериментов, зная, что матожидание ошибки измеряемой величины равно нулю, то среднее от измеренных значений будет близко к истинному.
\end{note}

\begin{proof}
	Уже должна быть ясна связь между матожиданиями:
	\[
		\E \frac{\xi_1 \plusdots \xi_n}{n} = \frac{\E\xi_1 \plusdots \E\xi_n}{n} = a
	\]
	Обозначим среднее случайных величин за $\eta$. Тогда, по неравенству Чебышёва:
	\[
		\forall \eps > 0 \quad P\left\{\md{\eta - a} \ge \eps\right\} \le \frac{D\eta}{\eps^2} = \frac{\sum_{i = 1}^n D\xi_i}{n^2 \eps^2} = \frac{n\sigma^2}{n^2 \eps^2} \xrightarrow[n \to \infty]{} 0
	\]
\end{proof}

\begin{note}
	Теорему можно ослабить по следующим пунктам:
	\begin{enumerate}
		\item Нам не нужна независимость, достаточно нулевой ковариации
		
		\item Вместо общего для всех $a$ можно писать среднее из математических ожиданий
		
		\item Для верности предела нам не нужны конкретные значения суммы дисперсий. Достаточно потребовать, что $\sum_{i = 1}^n D\xi_i = o(n^2)$. 
	\end{enumerate}
\end{note}

\begin{theorem} (Центральная Предельная Теорема, обозначается как ЦПТ)
	Пусть $\xi_1, \ldots, \xi_n$ --- независимые одинаково распределенные случайные величины, а также $\exists D\xi_1 \neq 0, \E\xi_1 = a$. Тогда, если обозначить $S_n = \xi_1 \plusdots \xi_n$, то
	\[
		P\left\{a \le \frac{S_n - \E S_n}{\sqrt{DS_n}} \le b\right\} \xrightarrow[n \to \infty]{} \int_a^b \frac{1}{\sqrt{2\pi}} e^{-\frac{x^2}{2}}dx
	\]
\end{theorem}

\begin{note}
	Забегая вперёд, справа написана вероятность $P\{a \le \eta \le b\}$, где $\eta \sim N(0, 1)$ --- случайная величина с \textit{нормальным распределением}. Теорема утверждает, что \textbf{вне зависимости от распределений} $\xi_i$, мы всегда сойдёмся к нормальному распределению.
\end{note}

\begin{proof}
	\textcolor{red}{Будет в конце года, сейчас без доказательства.}
\end{proof}

\subsection{Системы множеств}

\begin{definition}
	\textit{Системой множеств} называется какой-то набор подмножеств другого множества.
\end{definition}

\begin{example}
	$F = 2^\Omega$ или $F = \{\emptyset\}$
\end{example}

\begin{definition}
	Система множеств $S$ называется \textit{полукольцом}, если выполнены следующие требования:
	\begin{enumerate}
		\item \(\emptyset \in S\)
		
		\item \(\forall A, B \in S \Ra A \cap B \in S\)
		
		\item Если $A_1, A \in S,\ A_1 \subseteq A$, то \(\exists A_2, \ldots, A_n \in S \such A = \bscup_{i = 1}^n A_i\)
	\end{enumerate}
\end{definition}

\begin{example}
	Рассмотрим систему полуинтервалов какого-то фиксированного интервала $[A; B)$:
	\[
		T = \{[a; b) \subseteq [A; B)\},\ \ A \le a \le b \le B
	\]
	Докажем, что $T$ является полукольцом:
	\begin{enumerate}
		\item $\emptyset = [A; A) \in T$
		
		\item Пересечение интервалов либо пусто, либо полуинтервал. Формально проверяется через разбор случаев
		
		\item Снова разбор случаев
	\end{enumerate}
\end{example}

\begin{definition}
	\textit{Кольцом} называется система множеств $R$, обладающая следующими свойствами:
	\begin{enumerate}
		\item \(R \neq \emptyset\)
		
		\item \(\forall A, B \in R \quad A \cap B \in R,\ A \tr B \in R\)
	\end{enumerate}
\end{definition}

\begin{theorem}
	Если $R$ --- кольцо, то
	\begin{enumerate}
		\item $R$ --- полукольцо
		
		\item \(\forall A, B \in R \quad A \cup B \in R,\ A \bs B \in R\)
	\end{enumerate}
\end{theorem}

\begin{proof}
	\begin{enumerate}
		\item Проверим все свойства полукольца:
		\begin{enumerate}
			\item $R \neq \emptyset \Ra \exists A \in R$. Тогда $\emptyset = A \tr A \in R$
			
			\item По определению кольца
			
			\item Если $A_1 \subseteq A$, то $A = A_1 \sqcup (A \bs A_1)$
		\end{enumerate}
	
		\item $A \bs B = A \tr (A \cap B)$
		
		\item $A \cup B = (A \bs B) \tr B$
	\end{enumerate}
\end{proof}

\begin{proposition}
	Пересечение любого количества колец --- кольцо.
\end{proposition}

\begin{note}
	\textit{Любого} подразумевает абсолютно любую мощность множества индексов колец.
\end{note}

\begin{proof}
	Пусть $\{R_\alpha\}_{\alpha \in \Delta}$ --- множество пересекаемых колец. Тогда
	\[
		R = \bigcap_{\alpha \in \Delta} R_\alpha
	\]
	Проверим свойства кольца:
	\begin{enumerate}
		\item $\forall \alpha \in \Delta\ \ \emptyset \in R_\alpha \Ra \emptyset \in R$
		
		\item $\forall A, B \in R \Ra \forall \alpha \in \Delta\ \ A, B \in R_\alpha$. Отсюда $A \cap B \in R_\alpha,\ A \tr B \in R_\alpha$. Стало быть, $A \cap B,\ A \tr B \in R$
	\end{enumerate}
\end{proof}

\begin{definition}
	Система множеств $\cA$ называется \textit{алгеброй}, если
	\begin{enumerate}
		\item $\cA$ --- кольцо
		
		\item $\exists E \in \cA \colon\ \forall A \in \cA\ \ A \subseteq E$ --- наличие единицы
	\end{enumerate}
	Иначе говоря, алгебра --- это кольцо с единицей.
\end{definition}

\begin{proposition}
	Пересечение любого числа алгебр с общей единицей является тоже алгеброй.
\end{proposition}

\begin{proof}
	Единица уже будет в пересечении, а про то, что пересечение будет кольцом, мы уже знаем из утверждения выше.
\end{proof}

\begin{definition}
	\textit{Минимальным кольцом, содержащим систему множеств $X$}, называется кольцо $R(X)$ со следующими свойствами:
	\begin{enumerate}
		\item $X \subseteq R(X)$
		
		\item $\forall R_1$ --- кольца такого, что $X \subseteq R_1$, то $R(X) \subseteq R_1$
	\end{enumerate}
\end{definition}

\begin{theorem}
	Для любой системы множеств $X$ существует $R(X)$.
\end{theorem}

\begin{proof}
	Положим за $\{R_\beta\}_{\beta \in \Gamma}$ --- множество всех колец, содержащих $X$. Что мы знаем про $R = \bigcap_{\beta \in \Gamma} R_\beta$?
	\begin{enumerate}
		\item $R$ --- кольцо
		
		\item $X \subseteq R$
		
		\item $\forall R' \colon X \subseteq R'$ и $R'$ --- кольцо, то $\exists \beta_1 \in \Gamma \colon R' = R_{\beta_1} \supseteq R$
	\end{enumerate}
\end{proof}

\begin{note}
	В чём проблема теоремы? Она не даёт описания множества полученного кольца, только его существование.
\end{note}

\begin{anote}
	На самом деле, мы данной теореме мы лихо опустили 1 очень важный факт: а почему существует множество всех колец, содержащих $X$?
\end{anote}