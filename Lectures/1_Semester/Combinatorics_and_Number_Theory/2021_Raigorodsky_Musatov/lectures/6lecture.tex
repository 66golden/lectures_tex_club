\subsection{Плотный порядок}

\begin{definition}
	Порядок \textit{плотен}, если $\forall x, y\ |\ x < y \Ra \exists z\ |\ x < z < y$.
\end{definition}

\begin{example}
	$\Q, \R$ обладают плотным порядком, а $\N, \Z$ не обладают.
\end{example}

\begin{theorem}
	Любые 2 счётных плотно линейно упорядоченных множества без наименьшего и наибольшего элементов изоморфны.
\end{theorem}

\begin{proof}
	По индукции построим инъекцию $f \colon A \to B$ и докажем, что она также сюръективна.
	
	Оба множества счётны. Формально это означает существование биекций $g \colon A \to \N$ и $h \colon B \to \N$, но фактически для нас важно, что мы можем занумеровать элементы. Будем последовательно определять $f(a_0), f(a_1), \ldots$, тем самым построив $f$:
	\begin{enumerate}
		\item За базу положим $f(a_0) = b_0$.
		
		\item Теперь мы на шаге определения $f(a_n),\ n > 0$ и все предыдущие значения уже определены. Определим положение $a_n$ среди $a_0, \ldots, a_{n - 1}$ и рассмотрим 3 случая:
		\begin{enumerate}
			\item $a_n > \max \{a_0, \ldots, a_{n - 1}\}$. В этом случае нам нужно такое свободное $b_j$, что
			\[
				b_j > \max \{f(a_0), \ldots, f(a_{n - 1})\}
			\]
			Оно точно есть, так как $B$ не имеет наибольшего элемента.
			
			\item $a_n < \min \{a_0, \ldots, a_{n - 1}\}$. Поступаем аналогично предыдущему случаю:
			\[
				b_j < \min \{f(a_0), \ldots, f(a_{n - 1})\}
			\]
			
			\item $\min \{a_0, \ldots, a_{n - 1}\} < a_n < \max \{a_0, \ldots, a_{n - 1}\}$. В таком случае найдём $a_l$ и $a_r$, заданные следующим образом:
			\begin{align*}
				&{a_l = \max \{a_q \such q \in [0; n - 1],\ a_q < a_n\}}
				\\
				&{a_r = \min \{a_q \such q \in [0; n - 1],\ a_q > a_n\}}
			\end{align*}
			Осталось выбрать $b_j$ из свободных такое, что оно удовлетворяет условию:
			\[
				f(a_l) < b_j < f(a_r)
			\]
			Оно точно есть, так как множество $B$ плотное.
		\end{enumerate}
	
		Инъективность и гомоморфность $f$ очевидна в силу построения. Если потребовать, что во всех случаях $j$ - это минимальный подходящий номер, то мы получаем возможность доказать сюръективность от противного. Пусть есть $b_t$ без прообраза, при этом $t$ - минимимальный такой номер, то есть для $b_0, \ldots, b_{t - 1}$ прообразы нашлись. Тогда, выберем такой номер $s$, что если $f(a_n) \in \{b_0, \ldots, b_{t - 1}\}$, то $n < s$. Это можно сделать, например, так: взять максимальный номер из прообразов $\{b_0, \ldots b_{t - 1}\}$ и прибавить к нему единицу. Снова разберём 3 случая:
		\begin{enumerate}
			\item $b_t > \max \{f(a_0), \ldots, f(a_s)\}$. В таком случае, так как в $A$ нету наибольшего элемента, найдётся $a_p > \max \{a_0, \ldots, a_s\}$. Из всех таких $a_p$ выберем элемент с минимальным номером. По построению, должно было оказаться
			\[
				f(a_p) = b_t
			\]
			
			\item $b_t < \min \{f(a_0), \ldots, f(a_s)\}$. Аналогично предыдущему случаю, выберем $a_p$ с минимальным номером и $a_p < \min \{f(a_0), \ldots, f(a_s)\}$. Опять получаем, что
			\[
				f(a_p) = b_t
			\]
			
			\item $\min \{f(a_0), \ldots, f(a_s)\} < b_t < \max \{f(a_0), \ldots, f(a_s)\}$. Здесь $b_t$ лежит между какими-то элементами $a_i, a_j \in \{a_0, \ldots, a_s\}$. Так как $A$ - плотное множество, то есть свободные элементы между ними, и среди таких мы снова должны взять с минимальным номером и положить
			\[
				f(a_p) = b_t
			\]
			по построению
		\end{enumerate}
		Во всех трёх случаях было достигнуто противоречие. Стало быть, $f$ - изоморфизм.
	\end{enumerate}
\end{proof}

\begin{example}
	$\Q$, $\Q \cap (0;1)$, $\Q_2 = \{\frac{k}{2^n}\ |\ k \in \Z,\ n \in \N\}$, $\mathbb{A}$ - алгебраические числа.
\end{example}

\subsection{Предпорядки}

\begin{definition}
	Отношение предпорядка $\precsim$~---~это рефлексивное и транзитивное отношение.
\end{definition}

\begin{definition}
	Отношение \textit{полного} предпорядка~---~это такой предпорядок, что
	$$
		\forall a, b \Ra (a \precsim b) \vee (b \precsim a).
	$$
	Из полноты следует рефлексиность. В экономике отношение полного предпорядка называется \textit{рациональным предпочтением}.
\end{definition}

\begin{theorem} (Структурная теорема)
	Назовём \textit{отношением безразличия} следующее отношение: 
	$a \sim b := (a \precsim b) \wedge (b \precsim a)$, тогда:
	
	Для любого отношения предпорядка отношение безразличия $\sim$~---~это отношение эквивалентности. При этом $\precsim$ задаёт отношение порядка на фактормножестве.
\end{theorem}

\begin{proof}
	Проверим $\sim$ на отношение эквивалентности:
	\begin{enumerate}
		\item $a \sim a$, так как $a \precsim a$ (рефлексивность).
		\item $a \sim b = b \sim a$, так как конъюнкция симметрична.
		\item $(a \sim b) \wedge (b \sim c) \Ra \left\{\begin{aligned}
			a \precsim b, & &b \precsim c \\ b \precsim a, & &c \precsim b
		\end{aligned}\right\} \Ra \System{a \precsim c, \\ c \precsim a,} \Ra a \sim c.$
	\end{enumerate}
\end{proof}

\subsubsection*{Агрегирование}

\begin{definition}
	Пусть $\precsim_1, \dots, \precsim_n$ - предпорядки на одном и том же множестве.
	
	Агрегирование по большинству: $x \precsim y$, если $\#\{i\ |\ x \precsim_i y\} \ge \#\{i\ |\ x \succsim_i y\}$, где $\#$ означает количество.
\end{definition}

\begin{note}
	Может получиться нетранзитивное отношение. Таким примером служит цикл Кондорсе:
	
	\begin{align*}
	a &\prec_1 b \prec_1 c,
	\\
	b &\prec_2 c \prec_2 a,
	\\
	c &\prec_3 a \prec_3 b.
	\end{align*}
	Отсюда получим $a \prec b \prec c \prec a$.
\end{note}

\begin{theorem}[об агрегировании по большинству]
	Агрегированием по большинству \textbf{на конечном множестве} можно получить любое рефлексивное полное отношение.
\end{theorem}

\begin{proof}
	Пусть мы хотим $x \prec y$. Добавим 2 порядка: $x <_1 y <_1 a_1 <_1 \dots <_1 a_{n - 2}$, а другое $a_{n - 2} <_2 a_{n - 3} <_2 \dots <_2 a_1 <_2 x <_2 y$.
\end{proof}

\begin{definition}
	Пусть $\preceq_1, \ldots, \preceq_n$ - предпорядки на одном и том же множестве.
	
	Тогда их \textit{агрегированием по большинству} назовём отношение, в котором
	\[
		x \preceq y \lra (\forall i \in [1; n]\ x \preceq_i y)
	\]
\end{definition}

\begin{theorem}[об агрегировании консенсусом]
	Агрегирование порядков консенсусом~---~порядок. 
	Агрегирование предпорядков консенсусом~---~тоже порядок.
\end{theorem}

\begin{theorem}
	Любой предпорядок может быть получен агрегированием консенсусом полных предпорядков.
\end{theorem}

\subsection{Решётки (как упорядоченное множество)}

\begin{definition}
	Пусть задан некоторое частично упорядоченное множество $(A, \le)$. Тогда, \textit{верхняя грань} элементов $x$ и $y$~---~любой $z$ такой, что $z \ge x$ и $z \ge y$.
\end{definition}

\begin{definition}
	\textit{Точная верхняя грань (супремум)}~---~такая верхняя грань, что она $\le$ любой другой верхней грани. 
\end{definition}

\begin{definition}
	\textit{Точная нижняя грань (инфинум)}~---~такая нижняя грань, что она $\ge$ любой другой нижней грани.
\end{definition}

\begin{definition}
	\textit{Решётка}~---~это частично упорядоченное множество, в котором у любых $x$ и $y$, лежащих в нём, есть $\sup$ и $\inf$.
\end{definition}

\begin{note}
	Необходимо и достаточно существования такой грани, что она сравнима со всеми остальными из того же типа (то есть верхними или нижними).
\end{note}