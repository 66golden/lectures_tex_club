\subsection{Степень соответствия}

\begin{definition}
	Пусть задано соответствие $F \colon A \rra A$. Тогда
	$$
		F^n = \underbrace{F \circ \dots \circ F}_{n \text{ раз}},
		\quad n \in \N.
	$$
\end{definition}

\subsubsection*{Свойства степени}

\begin{itemize}
	\item $F^{n + m} = F^n \circ F^m$.
	\item $(F^n)^m = F^{n \cdot m}$.
\end{itemize}

\subsection{Возведение множества в степень другого множества}

Пусть есть $A$ и $B$ такие, что $|A| = n$, $|B| = k$.

Вопрос: сколько существует \textit{различных} отображений из $A$ в $B$?

Ответ: $k^n$.

\begin{definition}
	Степенью $A$ множества $B$ называется множество всевозможных отображений из множества $A$ в множество $B$:
	$$
		|B^A| = |B|^{|A|}.
	$$
\end{definition}

\subsubsection*{Случай с $k = 2$}

Заметим, что для полного определения $F : A \ra B$ нам необходимо и достаточно задать $F^{-1}(b_0)$ или $F^{-1}(b_1)$. Тогда $F(x)$ можно определить очень просто:
$$
	F(x) = \System{&{b_0, \text{ если } x \in F^{-1}(b_0),} \\ &{b_1, \text{ иначе}.}}
$$

Таким образом, получается вывод: любое подмножество \textit{однозначно} сопоставляется функции с двумя значениями.

\begin{definition}
	\textit{Булеаном} называют множество всех подмножеств множества $A$. Обозначается как $2^A$.
\end{definition}

\subsubsection*{Случай с $n = 0$}

Пусть $A = \emptyset$. Тогда посмотрим на произвольное отображение $F \colon A \ra B$. Как известно, $F$ представляет собой подмножество $A \times B = \emptyset \times B = \emptyset$. То есть $F = \emptyset$, при этом данное соответствие является отображением, так как любому элементу множества $A$ соответствует ровно один элемент множества $B$. Отсюда следствие:
$$
	B^{\emptyset} = \{\emptyset\}.
$$
Причём это верно для любого $B$, даже для $B = \emptyset$.

\subsubsection*{Случай с $n \neq 0, k = 0$}

$F \subset A \times \emptyset = \emptyset$, но если $A \neq \emptyset$, то такое соответствие не является отображением, так как любому элементу из $A$ ничего не соответствует (частично определенная функция).

\subsubsection*{Свойства множества в степени множества}

\begin{enumerate}
	\item $(A \times B)^C \cong A^C \times B^C$.
	\item $B \cap C = \emptyset \Ra A^{B \cup C} \cong A^B \times A^C$.
	\item $A^{B \times C} \cong (A^B)^C$.
\end{enumerate}

\begin{note}
	Писать знак $=$ вместо $\cong$~---~слишком сильное утверждение. Выражение справа и слева не являются эквивалентными, но между ними существует <<естественное>> отображение (равномощность).
\end{note}

\subsection{Мощность множества}

\begin{definition}
	Наивно определим мощность через рекурсию:
	\begin{enumerate}
		\item $|\emptyset| = 0$,
		\item $|A| = |A \bs \{a_1\}| + 1$.
	\end{enumerate}
\end{definition}

\begin{note}
	Так, вообще говоря, нельзя делать. Из-за такого определения мы зацикливаем определение натуральных чисел, если задавать их через аксиомы Пеано.
\end{note}

\begin{theorem}
	Мощность множества не зависит от того, какой конкретно элемент из него исключили, то есть $\forall a, b \in A \Ra |A \bs \{a\}| = |A \bs \{b\}|$.
\end{theorem}

\begin{proof}
	$$
		|A \bs \{a\}| = |A \bs \{a\} \bs \{b\}| + 1 = |A \bs \{b\}|.
	$$
\end{proof}

\begin{theorem}
	Если множества $A$ и $B$ конечны, то в них поровну элементов тогда и только тогда, когда между ними есть биекция.
\end{theorem}

\begin{proof}
	Докажем, что если $|A| = |B|$, то между множествами существует биекция. Сделаем это по индукции:
	
	База: $A = B = \emptyset$. Тогда, отображение $F \colon A \ra B$~---~биекция (несложно проверить).
	
	Ход индукции: теперь $A \neq \emptyset$ и $B \neq \emptyset$. Выберем $a \in A$ и $b \in B$. Согласно предположению индукции, то существует биекция $F \colon A \bs \{a\} \ra B \bs \{b\}$. Если мы добавим к данной биекции новую пару $b = F(a)$, то она всё ещё будет биекцией. Что и требовалось доказать.
	
	Теперь докажем в обратную сторону: если есть биекция, то в множествах поровну элементов.
	
	Пусть $A = \emptyset$. Тогда, $B = \emptyset$, иначе мы нарушим сюръективность $\Ra |A| = |B| = 0$.
	
	Рассмотрим $A \neq \emptyset$. Выберем $a \in A$. Тогда, так как у нас есть биекция, то найдётся $b = F(a)$. Заметим, что $F|_{A \bs \{a\}}$~---~тоже биекция, а значит $|A \bs \{a\}| = |B \bs \{b\}|$. Следовательно, $|A| = |B| = |A \bs \{a\}| + 1 = |B \bs \{b\}| + 1$.
\end{proof}

\subsection{Равномощность}

\begin{definition}
	Множества $A$ и $B$ называются \textit{равномощными}, если существует биекция $F \colon  A \ra B$.
	Обозначается как $A \cong B$.
\end{definition}

\begin{note}
	Отношение равномощности обладает всеми свойствами отношения эквивалентности, \textbf{но не является им из-за несуществования всеобъемлющего множества}.
\end{note}

\subsubsection*{Порядок на равномощности}

\begin{definition}
	Говорят, что $A \leqsim B$, если $\exists B' \subset B\ |\ A \cong B$.
\end{definition}

\begin{definition}
	$A \lesssim B$, если $A \leqsim B$ и при этом $A \not\cong B$.
\end{definition}

\begin{theorem} (Кантора-Бернштейна)
	Отношение порядка на равномощности антисимметрично, то есть
	\[
		(A \leqsim B) \wedge (B \leqsim A) \Ra A \cong B.
	\]
\end{theorem}