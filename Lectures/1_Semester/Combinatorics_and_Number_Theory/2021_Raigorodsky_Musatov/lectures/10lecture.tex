\subsection{Основная теорема арифметики}

\begin{theorem}
	Любое натуральное число единственным образом (с точностью до перестановки) представляется как
	\[
	n = p_1^{\alpha_1} \cdot \ldots \cdot p_s^{\alpha_s},
	\]
	где $p_i$ --- простое число (1 не является таковым), а также $\forall i\ \alpha_i \ge 1$. Такой вид числа называется каноническим.
\end{theorem}

\begin{note}
	Считается, что 1 не обладает каноническим видом.
\end{note}

\begin{proof}~
	\begin{itemize}
		\item Существование.
		
		Доказывается по индукции $n$:
		\begin{enumerate}
			\item База $n = p$ - простое число. Верность очевидна.
			
			\item Переход $n$ - составное число. Это значит, что
			\[
				\exists a, b \in (1; n) \such n = a \cdot b
			\]
			Так как $a, b \in (1; n)$, то для них верно предположение индукции. Разложив эти числа, получим разложение и для $n$.
		\end{enumerate}
	
		\item Единственность.
		
		Предположим обратное, и рассмотрим минимальное из чисел, у которого есть как минимум 2 разложения на простые множители. Тогда, все множители в разложениях различны (иначе сократим и получим меньшее число с двумя разложениями, что противоречит предположению). Пусть они имеют вид:
		\[
			n = p_1 \cdot \ldots \cdot p_k = q_1 \cdot \ldots \cdot q_l
		\]
		где $p_1$ и $q_1$ - минимальные из всех сомножителей в своих разложениях. Не умаляя общности, положим $p_1 < q_1$. Тогда, рассмотрим число $m$:
		\[
			m = (q_1 - p_1) \cdot q_2 \cdot \ldots \cdot q_l
		\]
		Докажем, что оно тоже имеет 2 разложения, вопреки предположению индукции. С одной стороны, $m$ делится на $p_1$, так как выражение выше можно переписать в следующем виде:
		\[
			m = q_1 \cdot \ldots \cdot q_l - p_1 \cdot q_2 \cdot \ldots \cdot q_l = p_1 (p_2 \cdot \ldots \cdot p_k - q_2 \cdot \ldots \cdot q_l)
		\]
		Отсюда следует, что у $m$ есть разложение, содержащее $p_1$. С другой стороны, в изначальной записи $m$ ни $q_1 - p_1$ не делится на $p_1$, ни любое $q_i \neq p_1$. Значит, разложив $q_1 - p_1$ на простые множители, мы придём к новому разложению, не содержащему $p_1$. Противоречие.
	\end{itemize}
\end{proof}

\subsection{Функция Мёбиуса}

\begin{definition}
	\textit{Функцией Мёбиуса} $\mu \colon \N \ra \{-1, 0, 1\}$ называется функция
	\[
		\mu(n) = \System{
			&{1,\ n = 1,}
			\\
			&{0,\ \exists \alpha_i \ge 2,\ 1 \le i \le s,}
			\\
			&{(-1)^s,\ n = p_1^1 \cdot \ldots \cdot p_s^1,}
		}
	\]
	где $n$ в каноническом виде выглядит как
	\[
		n = p_1^{\alpha_1} \cdot \ldots \cdot p_s^{\alpha_s}.
	\]
\end{definition}

\begin{lemma}
	Сумма значений функции Мёбиуса от натурального числа равна 1 для единицы и 0 для всех остальных чисел.
	\[
		\suml_{d \mid n} \mu(d) = \System{
			&{1,\ n = 1,}
			\\
			&{0, n > 1.}
		}
	\]
\end{lemma}

\begin{proof}
	Пусть $n$ имеет канонический вид
	\[
		n = p_1^{\alpha_1} \cdot \ldots \cdot p_s^{\alpha_s}.
	\]
	Тогда $d$ --- делитель $n$, если
	\[
		d = p_1^{\beta_1} \cdot \ldots \cdot p_s^{\beta_s},
	\]
	где $0 \le \beta_i \le \alpha_i$.
	
	Это значит, что нашу сумму по делителям можно переписать как сумму по всем возможным наборам $\beta_1, \ldots, \beta_s \such 0 \le \beta_i \le \alpha_i$:
	\[
		\suml_{d \mid n} \mu(d) = \suml_{\beta_1, \ldots, \beta_s \over 0 \le \beta_i \le \alpha_i} \mu(d) = \suml_{\beta_1, \ldots, \beta_s \over 0 \le \beta_i \le 1} \mu(d).
	\]
	Заметим, что теперь у нас только $2^s$ делителей, которые влияют на сумму в зависимости от того, сколько простых чисел они содержат. Следовательно,
	\[
		\suml_{\beta_1, \ldots, \beta_s \over 0 \le \beta_i \le 1} \mu(d) = 1 - C_s^1 + C_s^2 - \ldots + (-1)^s C_s^s = \\
		C_s^0 - C_s^1 + C_s^2 - \ldots + (-1)^s C_s^s = (1 - 1)^s = 0.
	\]
\end{proof}

\begin{theorem} (Обращение Мёбиуса)
	Пусть задана $f \colon \N \ra \R$. Определим $g$:
	\[
		g(n) := \suml_{d \mid n} f(d).
	\]
	Тогда
	\[
		f(n) = \suml_{d \mid n} \mu(d) \cdot g\left(\frac{n}{d}\right).
	\]
\end{theorem}

\begin{proof}
	Распишем сумму:
	\[
		\suml_{d \mid n} \mu(d) \cdot g(n / d) = \suml_{d \mid n} \mu(d) \cdot \suml_{d' \mid \frac{n}{d}} f(d').
	\]
	Выбрать $d$ такое, что $d \mid n$ и потом выбрать $d' \mid \frac{n}{d}$ --- это то же самое, что выбрать пару $(d, d'): d \cdot d' \mid n$. Отсюда
	\begin{multline*}
		\suml_{d \mid n} \mu(d) \cdot \suml_{d' \mid \frac{n}{d}} f(d') = \suml_{dd' \mid n} \mu(d) \cdot f(d') = \suml_{d'd \mid n} \mu(d') \cdot f(d) = \suml_{d \mid n} f(d) \cdot \suml_{d' \mid \frac{n}{d}} \mu(d') =
		\\ =
		f(n) + \suml_{{d: d \mid n} \over d < n} f(d) \cdot \underbrace{\suml_{d' \mid \frac{n}{d}} \mu(d')}_{0} = f(n).
	\end{multline*}
	Последний переход следует из доказанной выше леммы.
\end{proof}

\subsection{Циклические слова}

\begin{definition}
	Пусть задан алфавит $X = \{b_1, \ldots, b_r\}$ и число $n \in \N$. Тогда \textit{линейным словом длины} $n$ называется последовательность букв из алфавита
	\[
		a_1, \ldots, a_n.
	\]
\end{definition}

Пусть $V$ --- это множество всех линейных последовательностей длины $n \in \N$ над алфавитом $X,\ |X| = r$. Очевидно, что
\[
	|V| = r^n.
\]

\begin{definition}
	\textit{Сдвиг} --- это операция, которая переводит линейное слово $a_1, \ldots, a_n$ в слово $a_2, \ldots, a_n, a_1$.
\end{definition}

\begin{definition}
	Объединим слова, получающиеся из данного при помощи сдвига, в один класс эквивалентности. Тогда этот класс называется \textit{циклическим словом}.
\end{definition}

\begin{definition}
	\textit{Период линейной последовательности} --- это минимальное число сдвигов $d$, которое переводит линейное слово само в себя.
\end{definition}

\begin{lemma}
	Если $d$ --- период линейного слова длины $n$, то $d \mid n$.
\end{lemma}

\begin{proof}
	Предположим обратное. Тогда $n$ можно представить в виде
	\[
		n = kd + b,\ 0 < b < d.
	\]
	Тогда, раз $d$ --- период, то после $kd$ сдвигов слово перейдёт в себя. То же самое верно и для $n = kd + b$ сдвигов. Но это значит, что слово переходит в себя и за $(kd + b) - kd = b < d$ сдвигов, противоречие.
\end{proof}

\begin{lemma}
	Любая последовательность длины $n$ и периода $d$ имеет вид
	\[
		a_1, \ldots, a_d, a_1, \ldots, a_d, \ldots, a_1, \ldots, a_d
	\]
\end{lemma}

\begin{proof}
	Пусть мы сделали $kd < n$ сдвигов некоторого слова. Тогда, на его первых d позициях стоит слово
	\[
		a_{kd + 1}, \ldots, a_{kd + d}.
	\]
	Раз полученное и текущее слова равны, то
	\[
		a_i = a_{kd + i},\ 1 \le i \le d.
	\]
\end{proof}

\begin{theorem}
	Если $T_r(n)$ --- это количество циклических слов длины $n$ над алфавитом $X,\ |X| = r$. Тогда
	\[
		T_r(n) = \suml_{d \mid n} \frac{1}{d} \left(\suml_{d' \mid d} \mu(d') r^{d / d'}\right).
	\]
\end{theorem}