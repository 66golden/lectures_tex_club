\section{Теория множеств}

\begin{comment}
\subsection{О приближениях}

    Математический анализ "--- это наука о приближениях.

    \begin{definition}
        Определение производной по Коши:
        Пусть $f = f(x)$ -- функция, число $A$ называется производной функции $f$ в
        точке $x$, если при неограниченным уменьшении $|h|$ \[
        P(\varepsilon, \delta) :=   0 < |h| < \delta \longrightarrow \left(\frac{f(x + h) - f(x)}{h} - A\right) < \varepsilon,
        \text{ где } P(x) = \text{истина или ложь} \]
        \[\forall \varepsilon > 0 \
        \exists \delta > 0 \ P(\varepsilon, \delta)
        \]
    \end{definition}
    \begin{example}
        $f(x) = x^2$:
        \begin{equation*}
            \frac{f(x + h) - f(x)}{h} = \frac{x^2 + 2xh + h^2 - x^2}{h} = 2x + h 
        \end{equation*}
        Проверим, что $A  = 2x$ является производной $f$ в точке $x$:
        \begin{equation*}
            \forall \varepsilon \ \exists \delta > 0 \ (0 < |h| < \delta) \rightarrow \left|\frac{(f(x + h )
            - f(x))}{h} - 2x\right| < \varepsilon
        \end{equation*}
        Дано $\varepsilon > 0$. Какое $\delta$ взять, чтобы $P(\varepsilon, \delta) = 1$?
        Если взять $\delta = \dfrac{\varepsilon}{2}$, то $0 < |h| < \delta \longleftrightarrow  |h|\hm{<} \dfrac{\varepsilon}{2}
        \longrightarrow P(\varepsilon, \delta) = 1$
    \end{example}
    \begin{exercise}
        Проверить, что утверждения имеют одинаковые значения: (кто не с нами, тот 
        против нас) $\longleftrightarrow$ (кто не против нас, тот с нами)
    \end{exercise}
    В этом семестре:
    \begin{enumerate}
        \item приближение функций многочленами
        \item евклидова топология вещественной прямой
        \item приложения к исследованию функций
    \end{enumerate}
\end{comment}

\subsection{Кванторы}

\df{$\forall x \; P(x)$}{для любого $x$ выполнено $P(x)$.}
\df{$\exists x \; P(x)$}{существует $x$, для которого выполнено $P(x)$.}
\df{$\exists! x \; P(x)$}{существует и единственно $x$, для которого  $P(x)$.
\[ \exists! x \; P(x) \defev \exists x \; P(x) \wedge (\forall y \; P(y) \Rightarrow y = x) \]
\[ \overline{\forall x \; P(x)} \Leftrightarrow \exists x \; \overline{P(x)} \]
\[ \overline{\exists x \; P(x)} \Leftrightarrow \forall x \; \overline{P(x)} \]
}

\begin{gather}
    \forall x \in A \;   P(x) \defev \forall x \left( x \in A \Rightarrow P(x) \right) \\
    \forall x (\forall y \ P(x, y)) \defev \forall x, y \ P(x, y)
\end{gather}

\begin{note}
    \[ \forall x \in \mathbb{R} \; \exists n \in \mathbb{N} \; n > x \text{ -- истина} \]
    \[ \exists n \in \mathbb{N} \; \forall x \in \mathbb{R} \; n > x \text{ -- ложь} \]
\end{note}



\subsection{Базовые определения}
\df{Декартовым произведением множеств $X$ и $Y$}{
    множество, состоящее из всех возможных пар $(x, y)$, где $x \in X,\ y \in Y$.
    \[ X \times Y \defeq \{(x, y) \; \vert \; x \in X, y \in Y\} \]
}
\begin{gather}
    X^2 := X \times X \\
    Y \subseteq X \defev \forall y \; (y \in Y \Rightarrow x \in X) \\
    2^{X} \defeq \{ Y \sconstr Y \subseteq X \}
\end{gather}

\begin{proposition}
    Если $ X $ состоит из $ n $ элементов, то $ 2^X $ состоит из $ 2^n $ 
    элементов.
\end{proposition}

\df{Отношение на множестве $X$}{
    любое подмножество $ R\subset X^2 $, \guillemotleft $x$ и $y$ находятся в отношении $R$\guillemotright $\;\defev (x, y) \in R \defev x \; R \; y$.
}

\begin{definition}
    Пусть $ X, Y $ -- множества, $ \Gamma\subseteq X  \times  Y$, причём $ \forall x \in X\ \exists! y \in Y: (x, y) \in \Gamma $, тогда тройка $ f \defeq \left(X, Y, \Gamma\right) $ называется
    функцией, а $ \Gamma $ --- её графиком.
    %
    \begin{gather*}
        (y = f(x)) \defev (x \in X \lor y \in Y \lor (x, y) \in \Gamma) \\
        \fnis{f} X \mapsto Y; \quad \fnis{f} X\mapsto f(x); \quad \fnis{f} X \ni x \mapsto f(x) \in Y
    \end{gather*}
\end{definition}

\begin{figure}
    \centering
    \begin{tikzpicture}
        \begin{axis}[
          width=0.5\textwidth,
          height=0.4\textwidth,
          xmin=-0.4,
          xmax=4,
          ymin=-0.4,
          ymax=4,
          axis lines = middle,
          yticklabels={,,},
          x tick label style={major tick length=0pt},
          y tick label style={major tick length=0pt},
          xticklabels={,,}
        ]
            \addplot[mark=none, black, samples=150, domain=1.5:3.75] {(x-3)^3-x+6} node[pos=1] (endofplotsquare) {};
            \node [below] at (endofplotsquare) {$\Gamma$};

            \draw (2, 3) node[circle,fill=black,inner sep=1] {};
            \draw[dashed] (2, 3) -- (0, 3)
                          (2, 3) -- (2, 0);
            \draw (0, 3) node[left] {$y$};
            \draw (2, 0) node[below] {$x$};
            \draw (0, 4) node[below left] {$Y$};
            \draw (4, 0) node[below left] {$X$};
        \end{axis}
    \end{tikzpicture}
\end{figure}

Для $\fnis{f} X \times Y \mapsto Z$
\begin{equation}
    \forall x \in X, y \in Y f(x, y) \defeq f((x, y))
\end{equation}



\subsection{Аксиомы вещественных чисел}
\df{Вещественные числа}{
    четвёрка из множества $\R$, операций $\fnis{+} \R^2 \mapsto \R$, $\fnis{\cdot} \R^2 \mapsto \R$ и отношения на этом множестве $\le \; \subseteq \R^2$ такая, что
\begin{enumerate}
    \setcounter{enumi}{-1}
    \item $a + b \defev +(a, b)$; $a \cdot b \defev \cdot(a, b)$; $x < y \defev (x \le y) \land (x \ne y)$; \ldots
    \item $\exists 0 \in \R : \forall x \in \R \; 0 + x = x + 0 = x$ 
    \item $\forall x \in \R \; \exists (-x) \in \R : x + (-x) = (-x) + x = 0$ 
    \item $\forall x, y, z \in \R \; x + (y + z) = (x + y) = z$ % ассоциативность 
    \item $\forall x, y \in \R \; x + y = y + x$ % коммутативность 
    \item $\exists 1 \in \R \setminus \{0\} : \forall x \in \R \; 1 \cdot x = x \cdot 1 = 1$
    \item $\forall x \in \R \setminus \{0\} \; \exists x^{-1} \in \R : x \cdot x^{-1} = x^{-1} x = 1$
    \item $\forall x, y, z \in \R \; x \cdot (y \cdot z) = (x \cdot y) \cdot z$
    \item $\forall x, y \in \R \; x \cdot y = y \cdot x$
    \item $\forall x, y, z \in \R \; (x + y) \cdot z = x \cdot z + y \cdot z$ % дистрибутивность
    \item $\forall x \in \R \; x \le x$
    \item $\forall x, y \in \R \; (x \le y) \land (y \le x) \Rightarrow (x = y)$
    \item $\forall x, y, z \in \R \; (x \le y) \land (y \le z) \Rightarrow (x \le z)$
    \item $\forall x, y \in \R \; (x \le y) \lor (y \le x)$
    \item $\forall x, y, z \in \R \; (x \le y) \Rightarrow (x + z \le y + z)$
    \item $\forall x, y \in \R \; (0 \le x) \land (0 \le y) \Rightarrow (0 \le x \cdot y)$
    \item $\forall X, Y \subseteq \R : (\forall x \in X \; \forall y \in Y \; x \le y) \; \exists c \in \R : (\forall x \in X \; \forall y \in Y \; x \le c \le y)$
\end{enumerate}
}

\begin{theorem}[Единственность нуля]
    $\exists!0$.    
\end{theorem}
\begin{proof}
    Пусть $ 0_1 $ и $ 0_2 $ -- нейтральные элементы, тогда
    \begin{equation*}
        \begin{rcases}
            0_1 + 0_2 = 0_2, \text{ так как } 0_1 \text{ -- нейтральное } \\
            0_1 + 0_2 = 0_1, \text{ так как } 0_2 \text{ -- нейтральное }
        \end{rcases} \thus 0_1 = 0_2.
    \end{equation*}
\end{proof}

\begin{theorem}[Единственность противоположного элемента]
    $\exists! (-x): x + (-x) = 0$.
\end{theorem}
\begin{proof}
    Пусть $ a $ и $ b $ $ \in \R $ и являются противоположными к
$ x \in \R $:
\begin{gather*}
    a + x =  x + a = 0 \\ b + x =  x + b = 0 \\
   x + a =  x + b = 0 \\ a + x + a =  a + x + b \\ 0 + a =  0 + b \\
   a = b
\end{gather*}
\end{proof}

\begin{theorem}
    $
        \forall a, b \in \R\ \exists! x : a + x = b \text{, причём } x = (-a) + b
    $
\end{theorem}
\begin{proof}
    Предыдущие две теоремы задают достаточно много ограничений на множество вещественных чисел, что можно решить это уравнение с помощью материала 2-го класса.
\end{proof}

    \begin{theorem}
        $ \forall x = \R \ 0 \cdot  x = 0 $ 
    \end{theorem} \begin{proof}
        \begin{gather*}
            0 \cdot x = (0 + 0) \cdot x = 0 \cdot x + 0 \cdot x \\
            ( -(0 \cdot x)) + 0 \cdot x = ( -(0 \cdot x) + 0 \cdot x) + 0 \cdot x \\
            0 = 0 + 0 \cdot x \\
            0 = 0 \cdot x
        \end{gather*}
    \end{proof} 

    \begin{theorem}
        $\forall x \in \R \; (-x) = (-1) \cdot x$
    \end{theorem} \begin{proof}
        \begin{gather*}
            0 = 0 \\
        x + (-x) = x \cdot 0 \\
        x + (-x) = x \cdot (1 + (-1)) \\
        x + (-x) = x + (-1) \cdot x \\
        (-x) = (-1) \cdot x
        \end{gather*}
    \end{proof}
    \begin{theorem}
        $ 0 < 1 $
    \end{theorem} \begin{proof}
        Предположим $ 1 \leq 0 $, тогда из аксиомы 14 следует $ 1 + (-1) \leq 0 + (-1) $, то есть $ 0 \leq - 1 $.
        Тогда $ 0 \leq ( - 1)^2 = 1 $, получили $ 0 \leq  1 $ --- противоречие.
    \end{proof}



\subsection{Натуральные числа}

Для произвольного семейства множеств $\mathcal{F}$
\begin{gather}
    \cap \mathcal{F} : = \{x \sconstr \forall A \in \mathcal{F}\ x \in A\} \\
    \cup \mathcal{F} : = \{x \sconstr \exists A \in \mathcal{F}\ x \in A\}
\end{gather}

\df{Индуктивное подмножество $\R$}{подмножество $\R$ такое, что
    \begin{enumerate}
        \item $ 1 \in  X $ 
        \item $ \forall  x \in X $ выполнено $ x + 1 \in X $
    \end{enumerate}
}

 \begin{theorem}
     \label{definition_inductive}
    Пусть $ \mathcal{G} $ -- семейство подмножеств $ \R $ и $ \forall A \in \mathcal{G} \ A $ -- индуктивно, тогда
    $ \cap \mathcal{G} $ тоже индуктивно.
 \end{theorem}
 \begin{proof} $ M : = \cap \mathcal{G}$
    \begin{enumerate}
        \item $ \forall A \in  \mathcal{G} \ 1 \in A \thus 1 \in M $
        \item $x \in M \thus \forall A \in \mathcal{G} \; x \in A \thus \forall X \in \mathcal{G} \; (x + 1) \in A \thus (x + 1) \in M$
    \end{enumerate}
 \end{proof}

 \begin{equation}
     \begin{gathered}
        \label{definition_F}
        \mathcal{F} \defeq \{A \subset \R \sconstr A \text{ -- индуктивно}\} \\
        \N \defeq \cup \mathcal{F}
     \end{gathered}
 \end{equation}

\begin{theorem}
    $\N$ -- индуктивное подмножество $\R$; $\nexists A \subsetneq \N : A \text{ -- индуктивно}$ 
\end{theorem}
\begin{proof}  
    \phantom \\
    \begin{enumerate}
        \item $\N$ индуктивно по теореме \ref{definition_inductive}
        \item По определению пересечения множеств, $\forall A \in \mathcal{F}\ \N \subseteq A$. Не существует такого множества, что одновременно выполнено $X \subsetneq Y$ и $Y \subseteq X$.
    \end{enumerate}
\end{proof}

\begin{theorem}[Математическая индукция] \label{math_induction}
    Если $ M \subset \N, 1 \in M $\! и $ \forall n \in M \; n + 1 $, то $ M = \N $
\end{theorem} \begin{proof}
    $ M $ удовлетворяет определению индуктивого подмножества $ \R $, значит $ N \sub M $, но $ M \sub N \thus M = \N$.
\end{proof}
\begin{theorem}
    $ \forall x \in \N \; x \ge 1 $
\end{theorem}
\begin{proof}
    Пусть $\exists x \in N : x < 1$. Выберем произвольный набор $A \in \mathcal{F}$ (из определения \ref{definition_F}) и построим $B \defeq \{b \sconstr b \in A, b > x\}$. $1 \in B$, потому что $1 \in A$ и $x < 1$, $\forall c \in B \; (1 + c) \in A$ и так как $(1 + c) > (1 + x) > x$, $(1 + c) \in B$ $\thus B \in \mathcal{F}$. $x \notin B \thus x \notin \cap \mathcal{F} \thus  x \notin \N$ -- противоречие.
\end{proof}
\begin{theorem}
    $ \forall m, n \in \N \; m + n \in \N,\ m \cdot n \in \N$
\end{theorem}
\begin{proof}
    Построим $M_1 \defeq \{x \sconstr x \in \N,\; (m + x) \in \N\}$. $M_1$ удовлетворяет критериям теоремы \ref{math_induction} $\thus M_1 = \N \thus n \in M_1 \thus m + n \in \N$.

    Построим $M_2 \defeq \{x \sconstr x \in \N,\; m \cdot x \in \N\}$. Если $x \in M_2$, то $m \cdot (x + 1) = m \cdot x + m \in \N$ (по 1 части утвержения, так как $x \in M_2 \Rightarrow m \cdot x \in \N$, $m \in \N$). Поэтому, $M_2$ удовлетворяет критериям теоремы \ref{math_induction} $\thus M_2 = \N \thus n \in M_2 \thus m \cdot n \in \N$.
\end{proof}


\subsection{Расширенная числовая прямая}
\textbf{Внимание}: синтетический подраздел. Это означает, что если следовать хронологическому порядку изложения материала на лекциях, текст, написанный далее до конца подраздела, должен располагаться в другом месте, однако автор конспекта посчитал логичным расположить его именно тут.

\df{Расширенная числовая прямая $\RR$}{множество вещественных чисел, дополненное двумя бесконечностями: $\pm\infty$. Для $\forall x \in \R$ доопределены следующие операции:
    \begin{enumerate}
        \item $- \infty < x < + \infty$
        \item $ x + ( - \infty) = x - ( - \infty) = + \infty $
        \item $ x - ( + \infty) = x + ( - \infty) = - \infty $
        \item $ \dfrac{x}{ + \infty} = \dfrac{x}{ + \infty} = 0 $
        \item $ x \cdot \pm \infty = \pm \infty,\ x > 0 $
        \item $ x \cdot \pm \infty = \mp\, \infty,\ x < 0 $
        \item $ + \infty + ( + \infty) = + \infty $
        \item $ - \infty + ( - \infty) = - \infty $
        \item $ + \infty \cdot ( + \infty) = - \infty \cdot ( - \infty) = + \infty $
        \item $ + \infty \cdot ( - \infty) = - \infty \cdot ( + \infty) = - \infty $
   \end{enumerate}
}

\begin{note}
    Остаётся недопустимым выполнять
    \begin{gather*}
        + \infty - ( + \infty), \ - \infty - ( - \infty) \\
        - \infty + ( + \infty), \  + \infty + ( - \infty)
    \end{gather*}
\end{note}

\begin{equation}
    \begin{gathered}
        \forall a \in \R\ U_\epsilon(a) \defeq (a - \epsilon;\ a + \epsilon) \\
        U_\epsilon( + \infty) \defeq \left(\frac{1}{\epsilon}; + \infty\right) \\
        U_\epsilon( -\infty) = \left( - \infty; - \frac{1}{\epsilon}\right) \\
        U_\epsilon(\infty) = U_\varepsilon(+\infty) \cup U_\varepsilon(-\infty)
    \end{gathered}
\end{equation}

$U_\varepsilon(a)$ называют $\varepsilon$-окрестностью числа $a$.

\begin{equation}
    \begin{gathered}
        \forall a, b \in \R\ \sep(a, b) = \frac{\abs{a - b}}{2} \\
        \forall a \in \R\ \sep(-\infty, a) = \sep(a, +\infty) = \frac{-\abs{a} + \sqrt{a^2+4}}{2} \\
        \sep(-\infty, +\infty) = 1
    \end{gathered}
\end{equation}
\textbf{Внимание}: такая функция на лекциях не была определена.

\begin{theorem}[Очевидный разбор случаев]
    $ \forall a, b \in \RR\ U_{\sep(a, b)}(a) \cap U_{\sep(a, b)}(b) = \varnothing$
\end{theorem}