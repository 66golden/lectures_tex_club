\subsection{Вложенные отрезки}

\df{Промежуток $I$ числовой прямой}{подмножество числовой прямой такое, что
    \[ \forall x \in I, y \in I, z \in \R (x < z < y) \Rightarrow (z \in I) \]
}
\df{Конечный промежуток}{ограниченный промежуток.}

\begin{gather}
    \forall a \in \RR, b \in \RR\ (a;\ b) \defeq \{x \sconstr x \in \R \land a < x < b \} \\
    \forall a \in \R, b \in \RR\ [a;\ b) \defeq \{x \sconstr x \in \R \land a \le x < b \} \\
    \forall a \in \RR, b \in \R\ (a;\ b] \defeq \{x \sconstr x \in \R \land a < x \le b \} \\
    \forall a \in \R, b \in \R\ [a;\ b] \defeq \{x \sconstr x \in \R \land a \le x \le b \}
\end{gather}

\begin{theorem}
    Для промежутка числовой прямой $I \subseteq \R$ выполнено $(\inf{I};\ \sup{I})$.
\end{theorem}
\begin{proof}
    TODO
\end{proof}

Из теоремы следует, что любой промежуток будет либо отрезком, либо интервалом, либо полуинтервалом.

\begin{theorem}[Теорема Кантора о вложенных отрезках]
    Для последовательности отрезков таких, что $[a_1;\ b_1] \subseteq [a_2;\ b_2] \subseteq \ldots$ и $\forall n \in \N\ a_n < b_n$ $\exists c \in \bigcap_{i=1}^{+\infty} [a_i;\ b_i]$.

    Более того, если $\limpinf b_n - a_n =0$, то такая точка единственна, а последовательность называется стягивающейся.
\end{theorem}
\begin{proof}
    Sketch. $\{a_n\}$ лежит левее $\{b_n\}$. Тогда по аксиоме непрерывности существует точка $c$.

    Пусть $c'$ также принадлежит пересечению, тогда $\abs{c' - c} \le b_n - a_n$, тогда предел меньше $\limpinf b_n - a_n = 0$.
\end{proof}

\begin{proposition}
    Множество $ \R $ несчетно.
\end{proposition} \begin{proof}
    От противного. Пусть $ \exists  $ биекция $ f: \N \to \R $
    По теореме Кантора о вложенных отрезках \begin{gather}
        \exists c \in \R: \ c \in \cap_{n = 1}^ \infty[a_n, b_n] \implies \exists\, k \in\, \N: c = f(k)
    \end{gather}
    Но $ f(k) \nexists [a_n, b_k] \implies c \ni \cap_{n  = 1}^ \infty [a_n, b_n] $
    Пусть $ a_1 < b_1 $ и $ f(1) \ni [a_1, b_1] $
    Пусть $ [a_1, b_1] > [a_2, b_2] > \dots > [a_n, b_n] $ и \[\begin{cases}
        f(k) \ni [a_k, b_k] \\
        a_k < b_k
    \end{cases} \implies \exists a_{n + 1} \]
\end{proof}
