\subsection{Экспонента}
\begin{theorem}[Неравенство Бернулли]
\label{BernoulliInequality}
    $ \forall n \in \N $ и $ x \geq  -1 $ верно \begin{equation*}
        (1 + x)^ n \geq\, 1 + nx
    \end{equation*}
\end{theorem}
\begin{proof}
    \begin{equation*}
    n = 1: 1 + x \geq 1 + x \text{ -- верно }
   \end{equation*}
   Пусть неравенство верно для $ n $. Тогда \begin{multline*}
    (1 + x)^{n + 1} = (1 + x)(1 + x)^ n \geq (1 + x)(1 + nx) = 1 + (n + 1)x + \underset{ \geq 0}{nx^2} \geq 1 + (n + 1)x
   \end{multline*}
\end{proof}


% Вероятно в новую секцию надо
\begin{theorem}
    Для любого $ x \in \R $ существует конечный $ \lim_{n \to \infty}\left(1 + \dfrac{x}{n}\right)^n = : \exp x $.
    Более того, $ \exp(x + y) = \exp x \cdot \exp y \: \forall x, y \in \R $
\end{theorem}
\begin{proof}
    Покажем, что $ a_n = \left(1 + \dfrac{x}{n}\right)^ n $ сходится. Выберем натуральное $ m > |x| $. Тогда при $ n \geq  m:\: a_n(x) > 0 $: \begin{multline*}
        \frac{a_{n + 1}(x)}{a_n(x)} = \dfrac{\left(1 + \dfrac{x}{n + 1}\right)^{n + 1}}{\left(1 + \dfrac{x}{n}\right)^n} = \left(1 + \frac{x}{n}\right)\left(\dfrac{1 + \dfrac{x}{n + 1}}{1 + \dfrac{x}{n}}\right)^{n + 1} =\\=\left(1 + \frac{x}{n}\right)\left(\dfrac{1 + \dfrac{x}{n} - \dfrac{x}{n(n + 1)}}{1 + \dfrac{x}{n}}\right)^{n + 1} = \left(1 + \dfrac{x}{n}\right)\left(1 - \dfrac{\dfrac{x}{n(n + 1)}}{1 + \dfrac{x}{n}}\right)^{n + 1}
    \end{multline*}
    Выражение $ \dfrac{ - \dfrac{x}{n(n + 1)}}{1 + \dfrac{x}{n}} > 0$, если $ x < 0$ и $ \dfrac{ - \dfrac{x}{n(n + 1)}}{1 + \dfrac{x}{n}}> - 1 \implies  \dfrac{n + 1 + x}{n + x} > 0 $, если $ x \geq 0 $, и значит по неравенству Бернулли: \begin{multline*}
        \frac{a_{n + 1}(x)}{a_n(x)} \geq\, \left(1 + \frac{x}{n}\right)\left(1 - \dfrac{\dfrac{x}{n}}{1 + \dfrac{x}{n}}\right) = \left(1 + \frac{x}{n} - \frac{x}{n}\right) = 1 \implies\\ \implies  \{a_n(x)\} \text{ -- нестрого возрастающая последовательность, при } n \geq m 
    \end{multline*}
    По доказанному $ a_n( - x) \geq  a_m( - x) $. Поскольку \begin{equation*}
        a_n(x)a_n( - x) = \left(1 - \frac{x^2}{n^2}\right)^n \leq 1
    \end{equation*}
    \begin{equation*}
    a_n(x) \leq \frac{1}{a_n( - x)} \leq \frac{1}{a_m( - x)} \forall n \geq m
   \end{equation*}
   Тогда по теореме \ref{Weierstrass_theorem} $ \{a_n(x)\}_{n = m}^ \infty $ сходится.
\end{proof} \begin{proposition}
     $ \exp(x + y) = \exp x \cdot \exp y $.
\end{proposition} \begin{proof}
    При всех $ n > m$ \begin{equation*}
        \left(1 + \dfrac{x}{n}\right)^n\left(1 + \dfrac{y}{n}\right)^n = \left( 1 + \dfrac{x + y}{n} + \dfrac{xy}{n^2}  \right)^n = \left(1 + \dfrac{x + y}{n} \right)^n\left(1 + \dfrac{\dfrac{xy}{n^2} }{1 + \dfrac{x + y}{n} }\right)^ n
    \end{equation*}
    Пусть $ \alpha_n = \dfrac{xy}{n + x + y}  $. Достаточно показать, что \begin{equation*}
        \lim_{n \to \infty}\left(1 + \frac{\alpha_n}{n} \right)^n = 1
    \end{equation*}
    Выберем номер $ N $ так, что \begin{equation*}
        |\alpha_n| < 1 \:\forall n \geq\, N
    \end{equation*}
    Поскольку при $ n \geq N $ \begin{equation*}
        \left(1 + \frac{\alpha_n}{n} \right)^n\left(1 - \frac{\alpha_n}{n} \right)^n = \left(1 - \dfrac{\alpha_n^2}{n^2} \right)^n \leq\, 1
    \end{equation*}
    По неравенству Бернулли \begin{equation*}
        1 + \alpha_n  \leq \left(1 + \dfrac{\alpha_n}{n} \right)^n \leq \dfrac{1}{\left(1 - \dfrac{\alpha_n}{n}\right)^n} \underset{\left(\frac{\alpha_n}{n} > - 1\right)}{ \leq }\dfrac{1}{1 - \alpha_n}
    \end{equation*}
    Так как $ |\alpha_n| < 1 $:
    \begin{equation*}
    \begin{cases}
        \lim_{n \to \infty}1 - \alpha_n = 1 \\
        \lim_{n \to \infty}1 + \alpha_n = 1 \\
    \end{cases} \implies 1 + \alpha_n \leq \left(1 + \dfrac{\alpha_n}{n}\right)^n \implies \dfrac{1}{1 - \alpha_n} \implies\,\lim_{n \to \infty}\left(1 + \dfrac{\alpha_n}{n}\right)^n = 1 
   \end{equation*}
\end{proof}

\begin{equation}
    e \defeq \exp(1)
\end{equation}

$e$ -- экспонента или число Эйлера

\begin{theorem}
    Последовательность $\left(1 + \frac{1}{n}\right)^{n + 1}$ убывает.    
\end{theorem}

\begin{theorem}
    $2 < e < 3$
\end{theorem}

\begin{theorem}
    $\forall x\ exp(x) \ne 0$
\end{theorem}

\begin{theorem}
    \label{ExpInequality}
    $\forall x\ 1 + x \le \exp(x)$, $\forall x < 1\ \exp(x) \le \frac{1}{1 - x}$
\end{theorem}
\begin{proof}\phantom\\
\begin{enumerate}
    \item $ \exp(x) = \limpinf \left(1 + \frac{x}{n}\right)^n$
    При $n \gg 1$ $\frac{x}{n} > -1 \thus \text{(по теореме \ref{BernoulliInequality}) } (1 + \frac{x}{n})^n \ge 1 + x \thus \text{(по теореме \ref{limit_change_ineq}) } \exp(x) \ge 1 + x$.
\end{enumerate}

    Пусть $x < 1$, тогда $\exp(-x) \ge 1 - x \thus \frac{1}{\exp(-x)} \le \frac{1}{1 - x} \thus \exp(x) \le \frac{1}{1 - x}$.
\end{proof}

\begin{theorem}
    $\forall x < y\ \exp(x) < \exp(y)$
\end{theorem}
\begin{proof}
    \[
        \exp(y) = \exp(x)\exp(y - x) \ge \text{(по теореме \ref{ExpInequality}) } \exp(x)(y - x + 1) > \exp(x)
    \]
\end{proof}

\begin{theorem}
    $\limpinf x_n = a \thus \limpinf \exp(x_n) = \exp(a)$
\end{theorem}
\begin{proof}
    \begin{multline*}
        \limpinf \alpha_n = 0 \thus
        \exists N\ \forall n > N\ \abs{\alpha_n} < 1 \thus \\
        \begin{rcases}
            \thus 1 + \alpha_n \le \exp(\alpha_n) \le \frac{1}{1 - \alpha_n} \\
            \phantom{\thus{ }} \limpinf 1 + \alpha_n = 1 \\
            \phantom{\thus{ }} \limpinf \frac{1}{1 - x} = 1
        \end{rcases} \thus \limpinf \exp(\alpha_n) = 1
    \end{multline*}
\end{proof}

\begin{theorem}
    $\displaystyle \limpinf \frac{\exp(\alpha_n) - 1}{\alpha_n} = 1$
\end{theorem}
\begin{proof}
    \begin{multline*}
        \limpinf \alpha_n = 0 \thus
        \exists N\ \forall n > N\ \abs{\alpha_n} < 1 \thus 1 + \alpha_n \le \exp(\alpha_n) \le \frac{1}{1 - x} \\
        \thus \alpha_n \le \exp(\alpha_n) - 1 \le \frac{\alpha_n}{1 - \alpha_n} \thus \\
        \thus
        \begin{cases}
            \dfrac{1}{1 + \abs{\alpha_n}} \le 1 \le \dfrac{\exp(\alpha_n) - 1}{n} \le \dfrac{1}{1 - \alpha_n} = \dfrac{1}{1 - \abs{\alpha_n}}\ (\alpha_n > 0) \\
            \dfrac{1}{1 + \abs{\alpha_n}} = \dfrac{1}{1 - \alpha_n} \le \dfrac{\exp(\alpha_n) - 1}{n} \le 1 \le \dfrac{1}{1 - \abs{\alpha_n}}\ (\alpha_n < 0)
        \end{cases}
        \thus \\
        \begin{rcases}
            \thus \dfrac{1}{1 + \abs{\alpha_n}} \le \dfrac{\exp(\alpha_n) - 1}{n} \le \dfrac{1}{1 - \abs{\alpha_n}} \\
            \phantom{\thus{ }} \displaystyle \limpinf \frac{1}{1 - \abs{\alpha_n}} = 1 \\
            \phantom{\thus{ }} \displaystyle \limpinf \frac{1}{1 + \abs{\alpha_n}} = 1
        \end{rcases} \thus \limpinf \frac{\exp(\alpha_n) - 1}{n} = 1
    \end{multline*}
\end{proof}


Для непустого множества $A$ ограничено сверху, $\exists (x_n) : \limpinf x_n = \sup{A}, \forall n \in \N\ x_n \in A, x_n \le x_{n + 1}$, называемая максимизирующей.

\begin{itemize}
    \item $\sup{A} \in A \thus x_n = \sup{A}$
    \item $\sup{A} \notin A$, тогда $x_1$ возьмём произвольным, остальные элементы будем набирать последовательно. $x_n > \max(x_1, x_2, \ldots, x_{n - 1})$ и $x_n > \sup{A} - n$. Такое $x_n$ почти очевидно (TODO) существует.
\end{itemize}

\df{Промежуток $I$ числовой прямой}{подмножество числовой прямой такое, что
    \[ \forall x \in I, y \in I, z \in \R (x < z < y) \Rightarrow (z \in I) \]
}
\df{Конечный промежуток}{ограниченный промежуток.}

\begin{gather}
    \forall a \in \RR, b \in \RR\ (a;\ b) \defeq \{x \sconstr x \in \R \land a < x < b \} \\
    \forall a \in \R, b \in \RR\ [a;\ b) \defeq \{x \sconstr x \in \R \land a \le x < b \} \\
    \forall a \in \RR, b \in \R\ (a;\ b] \defeq \{x \sconstr x \in \R \land a < x \le b \} \\
    \forall a \in \R, b \in \R\ [a;\ b] \defeq \{x \sconstr x \in \R \land a \le x \le b \}
\end{gather}

\begin{theorem}
    Для промежутка числовой прямой $I \subseteq \R$ выполнено $(\inf{I};\ \sup{I})$.
\end{theorem}
\begin{proof}
    TODO
\end{proof}

Из теоремы следует, что любой промежуток будет либо отрезком, либо интервалом, либо полуинтервалом.

\begin{theorem}[Теорема Кантора о вложенных отрезках]
    Для последовательности отрезков таких, что $[a_1;\ b_1] \subseteq [a_2;\ b_2] \subseteq \ldots$ и $\forall n \in \N\ a_n < b_n$ $\exists c \in \bigcap_{i=1}^{+\infty} [a_i;\ b_i]$.

    Более того, если $\limpinf b_n - a_n =0$, то такая точка единственна, а последовательность называется стягивающейся.
\end{theorem}
\begin{proof}
    Sketch. $\{a_n\}$ лежит левее $\{b_n\}$. Тогда по аксиоме непрерывности существует точка $c$.

    Пусть $c'$ также принадлежит пересечению, тогда $\abs{c' - c} \le b_n - a_n$, тогда предел меньше $\limpinf b_n - a_n = 0$.
\end{proof}