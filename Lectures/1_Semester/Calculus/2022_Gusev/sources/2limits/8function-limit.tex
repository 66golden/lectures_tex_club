\subsection{Предел и непрерывность множеств}

% \begin{gather}
	% \U_\varepsilon(a) \defeq U_\varepsilon(a) \setminus \{a\} \\
	% \lim_{x \to x_0} f(x) = a \defev x_0 \in \RR, a \in \RR \forall \varepsilon > 0\ \exists \delta > 0 : \forall x \in \U_\varepsilon(x_0)\ f(x) \in U_\delta(a)
% \end{gather}

Определение предела функции по Коши

Проколотая окрестность
+ Замечание
+ Про функцию
Предел функции по множеству

\df{Последовательность Гейне, сходящая к $x_0$}{последовательность, сходящаяся к $x_0$, элементы которой не равны $x_0$ и принадлежат $D(f)$ для некоторой функции $f$.}

% \begin{theorem}[Определение предела функции по Гейне]
% 	\[ \lim_{x \to x_0} f(x) = a \;\Leftrightarrow\; \forall (a_n) \text{ -- последовательность Гейне, сходящаяся к $x_0$ } \limpinf f(a_n) = a \]
% \end{theorem}
% \begin{proof}
% \iffproofR{
% 	Рассмотрим последовательность Гейне $(a_n)$, которая сходится к $x_0$.
% 	Пусть $\varepsilon > 0$, тогда $\exists \delta > 0 : f(\U_\delta(x_0)) \subseteq U_\varepsilon(a)$. 
% 	$\forall \delta > 0\ \exists N : \forall n > N a_n \in U_\delta(x_0) \thus f(a_n) \in U_\varepsilon(a) \thus \limpinf f(a_n) = a$.
% }{
% 	От противного. $\exists \varepsilon > 0\ \forall \delta > 0\ \exists x \in \U_\delta(x_0) \cap D\ f(x) \notin U_\varepsilon(a) \thus \forall n \in \N\ \exists x_n \in \U_\frac{1}{n}(x_0) \cap D\ f(x) \notin U_\varepsilon(a)$.
% 	Последовательность $(x_n)$ -- последовательность Гейне, которая сходится к $x_0$, но $\limpinf x_n \ne a$.
% }
% \end{proof}


\begin{theorem}
	Если $\fnis{f, g, h} D \mapsto \R$, $x_0 \in \R$, $\forall \delta > 0\ \UU_\delta(x_0) \cap D \ne \varnothing$, $f(x) \le h(x) \le g(x)$ и $\lim_{x \to x_0} f(x) = \lim_{x \to x_0} g(x) = a$, то $\lim_{x \to x_0} h(x) = a$.
\end{theorem}

Пусть $\lim_{y \to A} g(y) = B \in \R$ и $\lim_{x \to x_0} f(x) = A$, 


\subsection{Пределы и непрерывность функций}
\begin{theorem}[Критерий Коши $ \exists  $ конечного предела функции в точке]
	Пусть $ D \sub \R, f:D\to \R $ и $ \forall r > 0\ \UU_r(x_0) \cap D \neq  \emptyset(x_0 \in \overline{ \R}) $
	Тогда $ \exists \lim_{x\to x_0}f(x) = A \in \R \longleftrightarrow \forall  \epsilon > 0 \exists \delta > 0 \forall x', x'' \in \UU_\delta(x_0)\cap D $
	 \begin{gather}
		|f(x') - f(x'')| < \epsilon
	\end{gather}
\end{theorem}
\begin{proof}
	 $ \Longrightarrow $
	Пусть $ \lim_{x \to  x_0}f(x) = A \in  \R $ Тогда \begin{gather}
		\forall \epsilon > 0 \exists \delta > 0: \forall x \in \UU \\
		|f(x) - A| < \dfrac{\epsilon}{2} \implies \forall x', x'' \in \UU_\delta (x_0)\cap D \\
		|f(x') - f(x'')| \leq |f(x') - A| + |A - f(x'')| < \epsilon
	\end{gather}
	 $ \Longleftarrow $ Пусть $ x_n \to x_0, n \to \infty $ -- последовательность Гейне.
	 $ \dots  $ \begin{lemma}
		
	 
	 \begin{gather}
		\exists r > 0 \exists C > 0 \forall x \in \UU_r(x_0)\cap D |f(x)| \leq\, D
	\end{gather}
\end{lemma}
	\begin{proof}
		
	\end{proof}
	 $ x_n \to  x_0 \implies$ при достаточно больших $ n: x_n \in \UU_r(x_0) \cap D \implies  |f(x_n) \leq  C$
	 По теореме Больцано-Вейерштрасса $ \exists  $ строго возрастающая последовательность $ n_k \in \N $ и $ \exists  A \in \R $ $ f(x_{n_k}) \to A $ при $ k \to \infty $.
	 $ x'' = x_{n_k} $
	 \begin{gather}
		 - \epsilon < f(x')\, - f(x'') < \epsilon
	\end{gather}
	По теореме о предельном переходе в неравенствах при $ k \to  \infty $: \begin{gather}
		 -\, \epsilon \leq\, f(x') - A \leq \epsilon
	\end{gather}
	Вывод $ \forall \epsilon > 0 \exists \delta > 0: \forall x' \in \UU_\delta(x_0) \cap D $ выполнено $ \dots  $, по определению предела \[ \lim_{x \to  x_0}f(x) = A \] 
\end{proof}
\begin{definition}
	Пусть $ f: D \to \R $ и $ x_0 \in D $. Если $ \exists \lim_{x \to x_0} f(x) = f(x_0)$, то $ f $ называется непрерывной в точке $ x_0 $(по множеству $ D $). Если $ f(x) $ не непрерывна в точке $ x_0 $,  то говорят, что $ f $ разрывна d $ x_0 $( и называют $ x_0 $ точкой разрыва).
	Если $ f $ непрерывна в $ x_0 \forall  x_0 \in  D $, то говорят, что $ f $ непрерывна на $ D $.
\end{definition}
\begin{proposition}
	 $ x \mapsto \sin x, x \mapsto \cos x, x \mapsto \exp x $ непрерывны.
\end{proposition} \begin{proof}
	TODO
\end{proof}
\begin{lemma}
	 $ \forall t \in \R \ |\sin t| \leq  t $.
\end{lemma} \begin{proof}
	TODO
\end{proof}