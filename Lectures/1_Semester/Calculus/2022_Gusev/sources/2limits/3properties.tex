\subsection{Свойства пределов}
\begin{theorem}
    Для числовых последовательностей $\{ a_n \}$ и $\{ b_n \}$ таких, что $ \exists m \in \N : b_n = a_{n + m}$
    \[ \lim_{n \to \infty} a_n = \lim_{n \to \infty} b_n \]
\end{theorem}
\begin{proof}
    \begin{multline*}
        \lim_{n \to \infty} a_n = p \thus
        \forall \epsilon > 0 \ \exists N \in \N: (\forall n \ge N\ a_n \in U_\epsilon(p)) \thus \\
        \thus (\forall n \ge N\ b_n \in U_\epsilon(p)) \thus
        \lim_{n \to \infty} b_n = p
    \end{multline*}
    \begin{multline*}
        \limpinf b_n = p \thus
        \forall \epsilon > 0 \ \exists N \in \N: (\forall n \ge N\ b_n \in U_\epsilon(p)) \thus \\
        \thus (\forall n \ge N\ a_{n + m} \in U_\epsilon(p)) \thus
        (\forall n \ge N + m \ a_{n} \in U_\epsilon(p)) \thus
        \limpinf a_n = p
    \end{multline*}
\end{proof}

\begin{theorem}[Предельный переход в неравенствах]
    \label{limit_change_ineq}
    Пусть $\limpinf a_n = a$, $\limpinf b_n = b$, тогда
    \begin{gather*}
        a < b \Rightarrow \exists N \in \N\ \forall n > N\ a_n < b_n \\
        \exists N \in \N\ \forall n > N\ a_n \le b_n \Rightarrow a \le b
    \end{gather*}
\end{theorem}
\begin{proof}\phantom\\
    \begin{itemize}
        \item Пусть $\varepsilon \defeq \sep(a, b)$, тогда
        \begin{multline*}
            \begin{rcases}
                \exists N_1 \in \N\ \forall n > N_1\ a_n \in U_\varepsilon(a) \\
                \exists N_2 \in \N\ \forall n > N_2\ b_n \in U_\varepsilon(b)
            \end{rcases} \thus \\
            \begin{rcases}
                \thus \forall n > \max(N_1, N_2)\ a_n \in U_\varepsilon(a), b_n \in U_\varepsilon(b) \\
                \phantom{\thus{ }} U_\varepsilon(a) \cap U_\varepsilon(b) = \varnothing \text{ (по лемме \ref{SepUIntersect})}
            \end{rcases} \thus \\
            \thus \text{(по лемме \ref{URelationWhenNotIntersecting}) } (a_n < b_n) \Leftrightarrow (a < b) \thus
            a_n < b_n
        \end{multline*}
        
        \item От противного. Пусть $a > b$, тогда по первой части $\exists N \in \N\ \forall n > N\ b_n < a_n \thus$ на бесконечном суффиксе $b$ меньше $a$, но формулировка говорит о существовании бесконечного суффикса, где $a$ не больше $b$ -- противоречие.
    \end{itemize}
\end{proof}

\begin{theorem}
    (Теорема о двух миллиционерах) Для последовательностей $a_n$, $b_n$, $c_n$
    \[ \exists N \in \N: \forall n > N\ a_n \le c_n \le b_n, \limpinf a_n = \limpinf b_n = a \thus \limpinf c_n = a \]
\end{theorem}
\begin{proof}
    TODO for RR
    \begin{multline*}
        \forall \varepsilon > 0\ 
        \begin{rcases}
            \exists N_1 \in \N\ \forall n > N_1\ a_n > a - \varepsilon \\
            \exists N_2 \in \N\ \forall n > N_2\ b_n < a + \varepsilon
        \end{rcases} \thus \\
        \thus \forall n > \max(N_1, N_2, n)\ a_n > a - \varepsilon, b_n < a + \varepsilon, a_n < c_n < b_n \thus \\
        \thus c_n \in U_\varepsilon(a) \thus
        \limpinf c_n = a
    \end{multline*}
\end{proof}

\begin{theorem}
    Для любой последовательности $c_n$ и бесконечно малой $\alpha_n$ 
    \[ \exists N \in \N: \forall n > N\ \abs{c_n} \le \alpha_n \thus \limpinf c_n = 0 \]
\end{theorem}
\begin{proof}
    TODO
\end{proof}

\begin{theorem}
    $\forall q : \abs{q} \le 1 \thus \limpinf{q^n} = 0$
\end{theorem}
\begin{proof}
    TODO
\end{proof}


\begin{theorem}
    Если для последовательностей $a_n$ и $b_n$ $\limpinf a_n = a$, $\limpinf b_n = b$ и соответствующая операция для $a$ и $b$ определена,
    \[ \limpinf a_n \pm b_n = a \pm b, \limpinf a_n \cdot b_n = a \cdot b \]
    \[ \forall n \in \N\ b_n \ne 0 \thus \limpinf \frac{a_n}{b_n} = \frac{a}{b} \]
\end{theorem}
\begin{proof}
    \begin{multline*}
        \forall \varepsilon > 0\ 
        \begin{rcases}
            \exists N_1 \in \N\ \forall n > N_1\ \abs{a_n - a} < \frac{\varepsilon}{2} \\
            \exists N_2 \in \N\ \forall n > N_2\ \abs{b_n - b} < \frac{\varepsilon}{2}
        \end{rcases} \thus \\
        \thus \forall n > \max(N_1, N_2)\ \abs{(a_n + b_n) - (a - b)} <
        \abs{a_n - a} + \abs{b_n - b} < \varepsilon \thus \\
        \thus \limpinf{a_n + b_n} = a + b
    \end{multline*}
    По теореме \ref{limit_means_bounds} $\exists C : \forall n\in\N\ \abs{a_n} < C, \abs{b_n} < C$.
    \begin{multline*}
        \forall \varepsilon > 0\ 
        \begin{rcases}
            \exists N_1 \in \N\ \forall n > N_1\ \abs{a_n - a} < \frac{\varepsilon}{2C} \\
            \exists N_2 \in \N\ \forall n > N_2\ \abs{b_n - b} < \frac{\varepsilon}{2C}
        \end{rcases} \thus \\
        \thus \forall n > \max(N_1, N_2)\ \abs{a_n b_n - ab} = \abs{a_n b_n - a_n b + a_n b - ab} \le \\
        \le \abs{a_n} \abs{b_n - b} + \abs{b} \abs{a_n - a} <
        C \frac{\varepsilon}{2C} + C \frac{\varepsilon}{2C} = \varepsilon \thus
        \limpinf{a_n b_n} = a b
    \end{multline*}

    TODO % Доказать, что модуль больше b/2 => расписываем формулу разности и всё сокращается
    \begin{multline*}
        \forall \varepsilon > 0 \exists N_1 \in \N\ \forall n > N_1\ \abs{b_n - b} < \frac{\varepsilon}{2C} \\
    \end{multline*}
     
\end{proof}

\begin{theorem}

    бесконечно малая * ограниченная = бесконечно малая
\end{theorem}
\begin{proof}
    TODO
\end{proof}

\begin{theorem}
    Пусть $ a_n \leq  b_n \ \forall n \geq n_0$. Тогда \begin{enumerate}
        \item $ \lim_{n \to  \infty} a_n = +\infty \implies \lim_{n \to \infty} b_n = + \infty $
        \item $ \lim_{n \to  \infty} b_n = -\infty \implies  \lim_{n \to \infty}a_n = - \infty$
    \end{enumerate}
\end{theorem}
\begin{proof}
    Очевидно следствие из свойства \ref{limit_change_ineq}.
\end{proof}