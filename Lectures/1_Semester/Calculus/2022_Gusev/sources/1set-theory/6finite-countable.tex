\subsection{Конечные и счётные множества}
\begin{proposition}
   Отображение $ f: X \to Y $ является биекцией $ \Leftrightarrow f $ является одновременно инъекцией и сюрьекцией 
\end{proposition}
\begin{definition}
    Множества $ X \text{ и } Y $ равномощны $ \Leftrightarrow \exists \text{ биекция } f: X \to Y $ 
\end{definition}
\begin{example}
    $ \N \text{ и } \Z $ равномощны 
\end{example} \begin{proof}
    
\end{proof}
\begin{definition}
    Множество $ X $ называется конечным, если $ \exists n \in \N\, X $ равномощно множеству $ \overline{1,n}\defeq \{1, 2, \dots , n\} $
    В противном случае $ X $ называется бесконечным.
\end{definition}
\begin{proposition}
    Пусть $ m, n \in \N $ тогда $ \overline{1,n} \text{ и } \overline{1,m} $ равномощны $ \Leftrightarrow  m = n$ (доказывается по индукции по $ \min(m, n) $) 
\end{proposition}
\begin{proposition}
    Если $ X $ конечно, то $ \exists ! n \in \N$:
    $ X $ равномощно $ \overline{1,n} $. Пишут $ |X| = n $
\end{proposition}
\begin{proposition}
    Если $ X $ -- конечное множество и $ Y \sub X $, то $ Y $ -- тоже конечное множество (Достаточно доказать, что если $ Y \sub \overline{1,n} $, то $ Y $ конечно)
\end{proposition}
\begin{lemma}
    Если $ A \sub \N $, то в $ A $ имеется минимальный элемент. Если $ A $ -- ограничено, то есть и максимальный элемент
\end{lemma}
\begin{proof}
    
\end{proof}
\begin{definition}
    Если $ X \text{ равномощно } \N $, то оно называется счётным.
\end{definition}
\begin{proposition}
    Если $ X \sub \N $ бесконечно, то $ X $счётно.
\end{proposition}\begin{proof}
    Построим биекциию $ f: \N \to X $ \begin{gather}
        f(1)\defeq \min\left(X\right) \\
        f(2) \defeq \min\left(X\backslash f(1)\right) \\
        \dots 
    \end{gather}
    Тогда $ f $ -- инъекция
\end{proof}
\begin{proposition}
    Если $ n < m $, то $ f(n) \neq f(m) $, то $ f(m) $ по построению принадлежит множеству, в которое не входит $ f(n) $
\end{proposition}
Кроме того, $ f $ -- сюрьекция.
Пусть $ a \in X $, $ \overline{1,a} $ -- конечно. $ X \cap \overline{1, a} $ -- тоже конечное в силу (утв. 7) TODO $ \implies \exists k \in  \N: |X \cap \overline{1, a}| = k \implies  f(k) = a $
\begin{proposition}
    $ \forall \text{ бесконечные подмножества счётного Множества счётны } $
\end{proposition}
\begin{proposition}
    $ \N^2 $ счётно $ \N^2 = \{(m, n)|\ m, n \in \N\} $
\end{proposition}
Точка $ (m, n) $ находится на диагонали $ d = m + n - 1 $ \begin{gather}
    1 + \dots\ + d - 1 = \dfrac{(1 + d - 1)(d - 1)}{2} = \dfrac{d(d - 1)}{2} \\
    f(m, n)\defeq \dfrac{(m + n - 1)(m + n - 2)}{2} + m
\end{gather}
По построению $ f $ является биекцией.

\begin{proposition}
    Пусть $ f : X \to Y $ -- инъекция. Обозначим $ \Z\defeq \{f(x) | x \in X\}$. Тогда $ f: X \to \Z $ -- биекция
\end{proposition}

\begin{proposition}
    $ \Q $ -- счётно.
\end{proposition}
\begin{proof}
    \begin{gather*}
        \forall r \in \Q_+ \exists\, \text{ несократимая дробь } \frac{p}{q} \\ 
        f: r \to (p, q) \text{ -- инъекция } \implies \Q_ + \text{ равномощно некоторому множеству  } \N^2, \\
        Q_ + \text{ -- бесконечно } \Rightarrow \Q_ +  \text{счётно} \text{ (как бесконечное подмножество счётного множества) }
    \end{gather*}
\end{proof}