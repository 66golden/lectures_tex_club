\subsection{Расширенная числовая прямая}
\textbf{Внимание}: синтетический подраздел. Это означает, что если следовать хронологическому порядку изложения материала на лекциях, текст, написанный далее до конца подраздела, должен располагаться в другом месте, однако автор конспекта посчитал логичным расположить его именно тут.

\df{Расширенная числовая прямая $\RR$}{множество вещественных чисел, дополненное двумя бесконечностями: $\pm\infty$. Для $\forall x \in \R$ доопределены следующие операции:
    \begin{enumerate}
        \item $- \infty < x < + \infty$
        \item $ x + ( - \infty) = x - ( - \infty) = + \infty $
        \item $ x - ( + \infty) = x + ( - \infty) = - \infty $
        \item $ \dfrac{x}{ + \infty} = \dfrac{x}{ - \infty} = 0 $
        \item $ x \cdot \pm \infty = \pm \infty,\ x > 0 $
        \item $ x \cdot \pm \infty = \mp\, \infty,\ x < 0 $
        \item $ + \infty + ( + \infty) = + \infty $
        \item $ - \infty + ( - \infty) = - \infty $
        \item $ + \infty \cdot ( + \infty) = - \infty \cdot ( - \infty) = + \infty $
        \item $ + \infty \cdot ( - \infty) = - \infty \cdot ( + \infty) = - \infty $
   \end{enumerate}
}

\begin{note}
    Остаётся недопустимым выполнять
    \begin{gather*}
        + \infty - ( + \infty), \ - \infty - ( - \infty) \\
        - \infty + ( + \infty), \  + \infty + ( - \infty)
    \end{gather*}
\end{note}

\begin{equation}
    \begin{gathered}
        \forall a \in \R\ U_\epsilon(a) \defeq (a - \epsilon;\ a + \epsilon) \\
        U_\epsilon( + \infty) \defeq \left(\frac{1}{\epsilon}; + \infty\right) \\
        U_\epsilon( -\infty) = \left( - \infty; - \frac{1}{\epsilon}\right) \\
        U_\epsilon(\infty) = U_\varepsilon(+\infty) \cup U_\varepsilon(-\infty)
    \end{gathered}
\end{equation}

$U_\varepsilon(a)$ называют $\varepsilon$-окрестностью числа $a$.

\textbf{Внимание}: материал далее является ``Original research'' в терминах Википедии. Будьте готовы явно ввести такое определение на экзамене. Политика данного конспекта -- иметь доказательства всех утверждений, данных на лекции в качестве упражнений. Доказательство некоторых из них заметно упрощается (в контексте разбора случаев), если, соответственно, ввести функцию $\sep$. Логика, по которой построено определение, следующая: пусть есть $a \ne b \in \RR$, тогда мы часто будем хотеть знать $\varepsilon$, при котором $\varepsilon$-окрестности точек $a$ и $b$ не пересекаются. В случае $a, b \in \R$ таким $\varepsilon$ является $\frac{\abs{a - b}}{2}$, однако такое значение не работает, если среди $a$ и $b$ есть бесконечности.

\begin{equation}
    \begin{gathered}
        \forall a, b \in \R\ \sep(a, b) \defeq \frac{\abs{a - b}}{2} \\
        \forall a \in \R\ \sep(\pm\infty, a) = \sep(a, \pm\infty) \defeq \frac{-\abs{a} + \sqrt{a^2+4}}{2} \\
        \sep(\pm\infty, \mp\infty) \defeq 1
    \end{gathered}
\end{equation}

\begin{lemma}
    $ \forall a \in \RR, \varepsilon_1 \ge \varepsilon_2 > 0\ U_{\varepsilon_2}(a) \subseteq U_{\varepsilon_1}(a)$
\end{lemma}
\begin{proof}\phantom\\
    \begin{itemize}
        \item $\forall a \in \R$:
        \begin{multline*}
            \varepsilon_1 \ge \varepsilon_2 \thus
            \begin{Bmatrix}
                a - \varepsilon_1 \le a - \varepsilon_2 \\
                a + \varepsilon_1 \ge a + \varepsilon_2
            \end{Bmatrix} \thus 
            (a - \varepsilon_2;\ a + \varepsilon_2) \subseteq (a - \varepsilon_1;\ a + \varepsilon_1) \thus \\
            \thus U_{\varepsilon_2}(a) \subseteq U_{\varepsilon_1}(a)
        \end{multline*}

        \item $a = \pm\infty$:
        \begin{multline*}
            \varepsilon_1 \ge \varepsilon_2 \thus
            \frac{1}{\varepsilon_1} \le \frac{1}{\varepsilon_2} \thus
            \begin{Bmatrix}
                (-\infty;\ \frac{1}{\varepsilon_2}) \subseteq (-\infty;\ \frac{1}{\varepsilon_1}) \\
                (\frac{1}{\varepsilon_2};\ +\infty) \subseteq (\frac{1}{\varepsilon_1};\ +\infty)
            \end{Bmatrix} \thus
            \begin{cases}
                U_{\varepsilon_2}(-\infty) \subseteq U_{\varepsilon_1}(-\infty) \\
                U_{\varepsilon_2}(+\infty) \subseteq U_{\varepsilon_1}(+\infty)
            \end{cases}
        \end{multline*}
    \end{itemize}
\end{proof}

\begin{lemma}
    \label{SepUIntersect}
    $ \forall a \ne b \in \RR\ U_{\sep(a, b)}(a) \cap U_{\sep(a, b)}(b) = \varnothing$
\end{lemma}
\begin{proof}
    Заметим, что $\sep(a, b) = \sep(b, a)$. Поэтому, не теряя общности, будем считать, что $a < b$.
    \begin{itemize}
        \item $a, b \in \R$: $\sep(a, b) = \frac{\abs{a - b}}{2} = \frac{b - a}{2}$
        \begin{multline*}
            \forall x \in U_{\sep(a, b)}(a) < a + \frac{b - a}{2} = b - \frac{b - a}{2} < \forall y \in U_{\sep(a, b)}(b) \thus \\
            \thus \forall x \in U_{\sep(a, b)}(a), y \in U_{\sep(a, b)}(b)\ (x < y \thus x \ne y) \thus \\
            \thus U_{\sep(a, b)}(a) \cap U_{\sep(a, b)}(b) = \varnothing
        \end{multline*}
        
        \item $a = -\infty, b = +\infty$: $\sep(a, b) = 1$
        \begin{equation*}
            \begin{rcases}
                U_1(-\infty) = (-\infty;\ -1) \\
                U_1(+\infty) = (1;\ +\infty)
            \end{rcases} \thus U_{\sep(a, b)}(-\infty) \cap U_{\sep(a, b)}(+\infty) = \varnothing
        \end{equation*}
        
        \item $a \in \R, b = +\infty$: $\sep(a, b) = \dfrac{-\abs{a} + \sqrt{a^2+4}}{2}$
        \begin{align*}
            \abs{a} + \sep(a, b) &\vee \frac{1}{\sep(a, b)} \\
            \abs{a} + \frac{-\abs{a} + \sqrt{a^2 + 4}}{2} &\vee \frac{2}{-\abs{a} + \sqrt{a^2 + 4}} \\
            \frac{\abs{a} + \sqrt{a^2 + 4}}{2} &\vee \frac{2\left(\abs{a} + \sqrt{a^2 + 4}\right)}{a^2 + 4 - a^2} \\
            \frac{\abs{a} + \sqrt{a^2 + 4}}{2} &\vee \frac{\abs{a} + \sqrt{a^2 + 4}}{2} \\
            \abs{a} + \sep(a, b) &= \frac{1}{\sep(a, b)}
        \end{align*}
        \begin{multline*}
             \forall x \in U_{\sep(a, b)}(a) < a + \sep(a, b) \le \abs{a} + \sep(a, b) = \frac{1}{\sep(a, b)} < \forall y \in U_{\sep(a, b)}(b) \thus \\
            \thus \forall x \in U_{\sep(a, b)}(a), y \in U_{\sep(a, b)}(b)\ (x < y \thus x \ne y) \thus \\
            \thus U_{\sep(a, b)}(a) \cap U_{\sep(a, b)}(b) = \varnothing
        \end{multline*}

        \item $a = -\infty, b \in \R$ аналогично предыдущему случаю, достаточно заменить знаки.
    \end{itemize}
\end{proof}

\begin{lemma}
    \label{URelationWhenNotIntersecting}
    $\forall a, b \in \RR\ U_\varepsilon(a) \cap U_\varepsilon(b) = \varnothing \thus (\forall x \in U_\varepsilon(a), y \in U_\varepsilon(b)\ (x < y) \Leftrightarrow (a < b))$
\end{lemma}
\begin{proof}
    Человеческими словами: если $\varepsilon$-окрестности не пересекаются, то все элементы окрестности $a$ лежат левее всех элементов $b$ согда $a < b$ и аналогично для $a > b$.

    Доказательство -- это очевидный разбор случаев (опять), но TODO
\end{proof}