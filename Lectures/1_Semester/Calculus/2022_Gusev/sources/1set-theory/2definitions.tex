\subsection{Базовые определения}
\df{Декартовым произведением множеств $X$ и $Y$}{
    множество, состоящее из всех возможных пар $(x, y)$, где $x \in X,\ y \in Y$.
    \[ X \times Y \defeq \{(x, y) \; \vert \; x \in X, y \in Y\} \]
}
\begin{gather}
    X^2 := X \times X \\
    Y \subseteq X \defev \forall y \; (y \in Y \Rightarrow x \in X) \\
    2^{X} \defeq \{ Y \sconstr Y \subseteq X \}
\end{gather}

\begin{proposition}
    Если $ X $ состоит из $ n $ элементов, то $ 2^X $ состоит из $ 2^n $ 
    элементов.
\end{proposition}

\df{Отношение на множестве $X$}{
    любое подмножество $ R\subset X^2 $, \guillemotleft $x$ и $y$ находятся в отношении $R$\guillemotright $\;\defev (x, y) \in R \defev x \; R \; y$.
}

\begin{definition}
    Пусть $ X, Y $ -- множества, $ \Gamma\subseteq X  \times  Y$, причём $ \forall x \in X\ \exists! y \in Y: (x, y) \in \Gamma $, тогда тройка $ f \defeq \left(X, Y, \Gamma\right) $ называется
    функцией, а $ \Gamma $ --- её графиком.
    %
    \begin{gather*}
        (y = f(x)) \defev (x \in X \lor y \in Y \lor (x, y) \in \Gamma) \\
        \fnis{f} X \mapsto Y; \quad \fnis{f} X\mapsto f(x); \quad \fnis{f} X \ni x \mapsto f(x) \in Y
    \end{gather*}
\end{definition}

\begin{figure}
    \centering
    \begin{tikzpicture}
        \begin{axis}[
          width=0.5\textwidth,
          height=0.4\textwidth,
          xmin=-0.4,
          xmax=4,
          ymin=-0.4,
          ymax=4,
          axis lines = middle,
          yticklabels={,,},
          x tick label style={major tick length=0pt},
          y tick label style={major tick length=0pt},
          xticklabels={,,}
        ]
            \addplot[mark=none, black, samples=150, domain=1.5:3.75] {(x-3)^3-x+6} node[pos=1] (endofplotsquare) {};
            \node [below] at (endofplotsquare) {$\Gamma$};

            \draw (2, 3) node[circle,fill=black,inner sep=1] {};
            \draw[dashed] (2, 3) -- (0, 3)
                          (2, 3) -- (2, 0);
            \draw (0, 3) node[left] {$y$};
            \draw (2, 0) node[below] {$x$};
            \draw (0, 4) node[below left] {$Y$};
            \draw (4, 0) node[below left] {$X$};
        \end{axis}
    \end{tikzpicture}
\end{figure}

Для $\fnis{f} X \times Y \mapsto Z$
\begin{equation}
    \forall x \in X, y \in Y f(x, y) \defeq f((x, y))
\end{equation}