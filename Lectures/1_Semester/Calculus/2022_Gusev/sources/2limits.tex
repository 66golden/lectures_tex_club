
% TODO Переформатировать и исправить орфографию

\subsection{Монотонные последовательности}
\begin{definition}
    Последовательность $ \{a_n\} $ называется не строго возрастающей(строго возрастающей), если \begin{equation}
        \forall n \in \N \ (a_n \underset{ <}{ \leq} a_{n + 1})
    \end{equation}
    Последовательность $ \{a_n\} $ называется не строго убывающей (строго убывающей), если $ \{- a_n\} $ не строго возрастающая(строго возрастающая).
\end{definition}
\begin{note}
    Если $ \forall  n \in \N\ (a_n \leq  a_{n + 1}) $, то по индуции доказывается, что $ \forall  n, m \in \N\ (n < m \implies  a_n \leq  a_m) $.
\end{note}
\begin{theorem}[О пределе монотонной последовательности --- теорема Вейштрасса] \label{Weierstrass_theorem}
    Если $ \{a_n\} $ не строго возрастает, то существует \[ \lim_{n \to \infty}a_n = \sup\{a_n\} \]
    Для не строго убывающей последовательности \[ \lim_{n \to \infty}a_n = \inf\{a_n\} \]
\end{theorem}
\begin{proof}
    Пусть $ \{a_n\} $ не строго возрастает. Если $ \{a_n\} $ ограничена, то $ c = \sup{a_n} \in  \R $. Тогда по определению $ \sup $: \begin{gather} \begin{cases}
        \forall n \in \N\:  (a_n \leq\, c) \\
        \exists N \ (a_N > c - \varepsilon)
    \end{cases} \implies a_n \geq\, a_N, \forall n \geq N
    \end{gather}
    Тогда при $ n \geq  N $ имеем $ \forall \varepsilon > 0 $: \begin{gather}
        c - \epsilon < a_N \leq\, a_n \leq x \leq\, c + \varepsilon \\
        |a_n - c | < \varepsilon \implies c = \lim_{n \to \infty} a_n
    \end{gather}
    Пусть $ a_n $ неограниченна сверху, $ \sup\{a_n\} = + \infty $. Зафиксируем $ \varepsilon > 0 $ \begin{equation}
        \exists N \ \left(a_N > \frac{1}{\varepsilon}\right)
    \end{equation} В силу возрастания $ a_n \geq a_N $ при всех $ n \geq  N $ и значит, $ a_n > \dfrac{1}{\varepsilon}  \implies  \lim a_n = + \infty$
    Аналогично доказывается для не строго убывающей.
\end{proof}
\subsection{Неравенство Бернулли}
\begin{lemma}[Неравенство Бернулли]
    $ \forall n \in \N $ и $ x \geq  -1 $ верно \begin{equation}
        (1 + x)^ n \geq\, 1 + nx
    \end{equation}
\end{lemma} \begin{proof}
    \begin{equation}
    n = 1: 1 + x \geq 1 + x \text{ -- верно }
   \end{equation}
   Пусть неравенство верно для $ n $. Тогда \begin{multline}
    (1 + x)^{n + 1} = (1 + x)(1 + x)^ n \geq (1 + x)(1 + nx) = 1 + (n + 1)x + \underset{ \geq 0}{nx^2} \geq 1 + (n + 1)x
   \end{multline}
\end{proof}


% Вероятно в новую секцию надо
\subsection{Экспонента}
\begin{theorem}
    Для любого $ x \in \R $ существует конечный $ \lim_{n \to \infty}\left(1 + \dfrac{x}{n}\right)^n = : \exp x $.
    Более того, $ \exp(x + y) = \exp x \cdot \exp y \: \forall x, y \in \R $
\end{theorem} \begin{proof}
    Покажем, что $ a_n = \left(1 + \dfrac{x}{n}\right)^ n $ сходится. Выберем натуральное $ m > |x| $. Тогда при $ n \geq  m:\: a_n(x) > 0 $: \begin{multline}
        \frac{a_{n + 1}(x)}{a_n(x)} = \dfrac{\left(1 + \dfrac{x}{n + 1}\right)^{n + 1}}{\left(1 + \dfrac{x}{n}\right)^n} = \left(1 + \frac{x}{n}\right)\left(\dfrac{1 + \dfrac{x}{n + 1}}{1 + \dfrac{x}{n}}\right)^{n + 1} =\\=\left(1 + \frac{x}{n}\right)\left(\dfrac{1 + \dfrac{x}{n} - \dfrac{x}{n(n + 1)}}{1 + \dfrac{x}{n}}\right)^{n + 1} = \left(1 + \dfrac{x}{n}\right)\left(1 - \dfrac{\dfrac{x}{n(n + 1)}}{1 + \dfrac{x}{n}}\right)^{n + 1}
    \end{multline}
    Выражение $ \dfrac{ - \dfrac{x}{n(n + 1)}}{1 + \dfrac{x}{n}} > 0$, если $ x < 0$ и $ \dfrac{ - \dfrac{x}{n(n + 1)}}{1 + \dfrac{x}{n}}> - 1 \implies  \dfrac{n + 1 + x}{n + x} > 0 $, если $ x \geq 0 $, и значит по неравенству Бернулли: \begin{multline}
        \frac{a_{n + 1}(x)}{a_n(x)} \geq\, \left(1 + \frac{x}{n}\right)\left(1 - \dfrac{\dfrac{x}{n}}{1 + \dfrac{x}{n}}\right) = \left(1 + \frac{x}{n} - \frac{x}{n}\right) = 1 \implies\\ \implies  \{a_n(x)\} \text{ -- нестрого возрастающая последовательность, при } n \geq m 
    \end{multline}
    По доказанному $ a_n( - x) \geq  a_m( - x) $. Поскольку \begin{equation}
        a_n(x)a_n( - x) = \left(1 - \frac{x^2}{n^2}\right)^n \leq 1
    \end{equation}
    \begin{equation}
    a_n(x) \leq \frac{1}{a_n( - x)} \leq \frac{1}{a_m( - x)} \forall n \geq m
   \end{equation}
   Тогда по теореме \ref{Weierstrass_theorem} $ \{a_n(x)\}_{n = m}^ \infty $ сходится.
\end{proof} \begin{proposition}
     $ \exp(x + y) = \exp x \cdot \exp y $.
\end{proposition} \begin{proof}
    При всех $ n > m$ \begin{equation}
        \left(1 + \dfrac{x}{n}\right)^n\left(1 + \dfrac{y}{n}\right)^n = \left( 1 + \dfrac{x + y}{n} + \dfrac{xy}{n^2}  \right)^n = \left(1 + \dfrac{x + y}{n} \right)^n\left(1 + \dfrac{\dfrac{xy}{n^2} }{1 + \dfrac{x + y}{n} }\right)^ n
    \end{equation}
    Пусть $ \alpha_n = \dfrac{xy}{n + x + y}  $. Достаточно показать, что \begin{equation}
        \lim_{n \to \infty}\left(1 + \frac{\alpha_n}{n} \right)^n = 1
    \end{equation}
    Выберем номер $ N $ так, что \begin{equation}
        |\alpha_n| < 1 \:\forall n \geq\, N
    \end{equation}
    Поскольку при $ n \geq N $ \begin{equation}
        \left(1 + \frac{\alpha_n}{n} \right)^n\left(1 - \frac{\alpha_n}{n} \right)^n = \left(1 - \dfrac{\alpha_n^2}{n^2} \right)^n \leq\, 1
    \end{equation}
    По неравенству Бернулли \begin{equation}
        1 + \alpha_n  \leq \left(1 + \dfrac{\alpha_n}{n} \right)^n \leq \dfrac{1}{\left(1 - \dfrac{\alpha_n}{n}\right)^n} \underset{\left(\frac{\alpha_n}{n} > - 1\right)}{ \leq }\dfrac{1}{1 - \alpha_n}
    \end{equation}
    Так как $ |\alpha_n| < 1 $:
    \begin{equation}
    \begin{cases}
        \lim_{n \to \infty}1 - \alpha_n = 1 \\
        \lim_{n \to \infty}1 + \alpha_n = 1 \\
    \end{cases} \implies 1 + \alpha_n \leq \left(1 + \dfrac{\alpha_n}{n}\right)^n \implies \dfrac{1}{1 - \alpha_n} \implies\,\lim_{n \to \infty}\left(1 + \dfrac{\alpha_n}{n}\right)^n = 1 
   \end{equation}
\end{proof}
