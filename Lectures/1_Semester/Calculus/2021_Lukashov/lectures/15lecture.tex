\subsubsection*{Производные высших порядков некоторых функций}

\begin{enumerate}
	\item $(\sin x)^{(n)} = \sin (x + \frac{\pi n}{2})$
	\item $(\cos x)^{(n)} = \cos (x + \frac{\pi n}{2})$
	\item $(x^\alpha)^{(n)} = \alpha \cdot (\alpha - 1) \cdot \ldots \cdot (\alpha - n + 1) x^{\alpha - n} = n! \cdot C_{\alpha}^n \cdot x^{\alpha - n}$, где
	\[
		C_{\alpha}^n = \frac{\alpha!}{n! \cdot (\alpha - n)!} = \frac{\alpha \cdot (\alpha - 1) \cdot \ldots \cdot (\alpha - n + 1)}{n!}
	\]
\end{enumerate}

\begin{note}
	Биномиальный коэффициент таким образом определён для \textit{любого вещественного $\alpha$}.
\end{note}

\begin{theorem}
	Для вещественного биномиального коэффициента справедливо свойство треугольника Паскаля:
	\[
		C_\alpha^n + C_\alpha^{n - 1} = C_{\alpha + 1}^n
	\]
\end{theorem}

\begin{proof}
	Распишем сумму по определению. Дальше уже всё сойдётся:
	\begin{multline*}
		C_\alpha^n + C_\alpha^{n - 1} = \frac{\alpha \cdot (\alpha - 1) \cdot \ldots \cdot (\alpha - (n - 1))}{n!} + \frac{\alpha \cdot (\alpha - 1) \cdot \ldots \cdot (\alpha - (n - 2))}{(n - 1)!} =
		\\
		\frac{\alpha \cdot (\alpha - 1) \cdot \ldots \cdot (\alpha - (n - 2)) \cdot (\alpha - (n - 1) + n)}{n!} = C_{\alpha + 1}^n
	\end{multline*}
\end{proof}

\begin{theorem} (Формула Лейбница)
	Если существуют конечные производные порядка $n$ функций $f$ и $g$ в некоторой точке, то для их произведения в этой точке справедлива формула
	\[
		(f \cdot g)^{(n)} = \suml_{k = 0}^{n} C_n^k f^{(k)} \cdot g^{(n - k)}
	\]
\end{theorem}

\begin{proof}
	Доказательство же формулы проведём по индукции:
	\begin{itemize}
		\item База $n = 1$:
		\[
			(fg)' = \suml_{k = 0}^1 C_1^k f^{(k)} g^{(1 - k)} = fg' + f'g \text{ - верно}
		\]
		
		\item Утверждением примем нашу формулу из теоремы. Покажем, что она верна для $n + 1$:
		\begin{multline*}
			(fg)^{(n + 1)} = \left((fg)^{(n)}\right)' = \suml_{k = 0}^n C_n^k \left(f^{(k)} g^{(n - k)}\right)' = \suml_{k = 0}^n C_n^k f^{(k + 1)} g^{(n - k)} + \suml_{k = 0}^n C_n^k f^{(k)} g^{(n + 1 - k)} = \\
			C_n^n f^{(n + 1)} g^{(0)} + \suml_{k = 0}^{n - 1} C_n^k f^{(k + 1)} g^{(n - k)} + C_n^0 f^{(0)} g^{(n + 1)} + \suml_{m = 0}^{n - 1} C_n^{m + 1} f^{(m + 1)} g^{(n - m)} = \\
			C_n^n f^{(n + 1)} g^{(0)} + C_n^0 f^{(0)} g^{(n + 1)} + \suml_{j = 0}^{n - 1} \left(C_n^j + C_n^{j + 1}\right) f^{(j + 1)} g^{(n - j)} = \\
			C_{n + 1}^0 f^{(0)} g^{(n + 1)} + \suml_{t = 1}^{n} C_n^t f^{(t)} g^{(n + 1 - t)} + C_{n + 1}^{n + 1} f^{(n + 1)} g^{(0)} = \suml_{t = 0}^{n + 1} C_{n + 1}^t f^{(t)} g^{(n + 1 - t)}
		\end{multline*}
	\end{itemize}
\end{proof}

\begin{enumerate}
\setcounter{enumi}{3}
	\item $(a^x)^{(n)} = a^x \cdot \ln^n a$
	\item $(\ln x)^{(n)} = (x^{-1})^{(n - 1)} = (n - 1)! \cdot C_{-1}^{n - 1} \cdot x^{-1 - (n - 1)} = (-1)^{(n - 1)} (n - 1)! \cdot x^{-n}$
\end{enumerate}

\begin{corollary} (Бином Ньютона)
	\[
		(a + b)^n = \suml_{k = 0}^n C_n^k a^k b^{n - k}
	\]
\end{corollary}

\begin{proof}
	С одной стороны,
	\[
		\left((\alpha\beta)^x\right)^{(n)} = (\alpha\beta)^x \cdot \ln^n (\alpha\beta) = (\alpha\beta)^x(\ln \alpha + \ln \beta)^n
	\]
	С другой стороны, по формуле Лейбница
	\begin{multline*}
		\left((\alpha\beta)^x\right) = \left(\alpha^x \cdot \beta^x\right)^{(n)} = \suml_{k = 0}^n C_n^k \left(\alpha^x\right)^{(k)}\left(\beta^x\right)^{(n - k)} = \suml_{k = 0}^n C_n^k (\alpha^x \ln^k \alpha)(\beta^x \ln^{(n - k)} \beta) = \\
		(\alpha\beta)^x \suml_{k = 0}^n C_n^k \ln^k \alpha \cdot \ln^{(n - k)} \beta
	\end{multline*}
	Положив $a := \ln \alpha$ и $b := \ln \beta$, получим необходимое утверждение.
\end{proof}

\begin{theorem} (Формула Фаа-ди-Бруно)
	Если существует производная $n$-го порядка функции $g$ в точке $a$ и $f$ в точке $g(a)$, то
	\[
		(f \circ g)^{(n)} = \suml_{\pi \in \Pi_n} \left(f^{(|\pi|)} \circ g\right) \cdot \prodl_{B \in \pi} g^{(|B|)}
	\]
	где 
	\begin{itemize}
		\item $\Pi_n$ обозначает множество всех разбиений $\{1, 2, \dots, n\}$ на непустые подмножества
		
		\item $\pi$ - разбиение множества $\{1, 2, \ldots, n\}$ на непустые подмножества. То есть элемент $B \in \pi$ - одно из множеств, на которые разбили исходное множество
		
		\item $|A|$ - число элементов множества $A$
	\end{itemize}
\end{theorem}

\begin{proof}
	По индукции
	\begin{itemize}
		\item База $n = 1$: $\Ra \Pi_1 = \{\{\{1\}\}\}$
		\[
			(f \circ g)' = (f' \circ g) \cdot g' \text{ - верно}
		\]
		
		\item Утверждением является наша формула. Докажем, что если она верна для $n$, то верна и для $n + 1$:
		\[
			(f \circ g)^{(n + 1)} = \left((f \circ g)^{(n)}\right)' = \suml_{\pi \in \Pi_n} \left((f^{(|\pi|)} \circ g) \cdot \prodl_{B \in \pi} g^{(|B|)}\right)'
		\]`
		Распишем производную одного слагаемого, принадлежащего разбиению $\pi^* \in \Pi_n$:
		\[
			\left((f^{(|\pi^*|)} \circ g) \cdot \prodl_{B \in \pi^*} g^{(|B|)}\right)' = (f^{(|\pi^*|)} \circ g)' \cdot \prodl_{B \in \pi^*} g^{(|B|)} + (f^{(|\pi^*|)} \circ g) \cdot \left(\prodl_{B \in \pi^*} g^{(|B|)}\right)'
		\]
		Теперь, у нас добавился к множеству $\{1, 2, \dots, n\}$ элемент $n + 1$. Если взять любое разбиение $\pi \in \Pi_{n + 1}$, то есть несколько вариантов того, какой оно имеет вид:
		\begin{enumerate}
			\item Элемент $n + 1$ обособлен в отдельную часть. То есть $\pi = \pi^* \cup \{n + 1\},\ \pi^* \in \Pi_n$. Каждое слагаемое таких разбиений по нашей формуле должно иметь вид:
			\[
				(f^{(|\pi^*| + 1)} \circ g) \cdot \prodl_{B \in \pi^*} g^{(|B|)} \cdot g'
			\]
			что в точности равно левой части суммы из производной слагаемого:
			\[
				(f^{(|\pi^*|)} \circ g)' \cdot \prodl_{B \in \pi^*} g^{(|B|)} = (f^{(|\pi^*| + 1)} \circ g) \cdot g' \cdot \prodl_{B \in \pi^*} g^{(|B|)}
			\]
			
			\item Элемент $n + 1$ помещён в некоторое $B_i \in \pi^*$. Это значит, что $|\pi| = |\pi^*|$. Каждое слагаемое таких разбиений по нашей формуле должно иметь вид:
			\[
				(f^{(|\pi^*|)} \circ g) \cdot g^{(|B_i| + 1)} \cdot \prodl_{B \in \pi^* \bs B_i} g^{(|B|)}
			\]
			Поймём, что $\pi^*$ соответствует не одно разбиение из $\Pi_{n + 1}$, а целое множество (действительно, мы можем выбирать разные $B_i$). Тогда, сумма слагаемых по всем $\pi$, полученным из $\pi^*$ записывается как
			\[
				\suml_{B_i \in \pi^*} (f^{(|\pi^*|)} \circ g) \cdot g^{(|B_i| + 1)} \cdot \prodl_{B \in \pi^* \bs B_i} g^{(|B|)} = (f^{(|\pi^*|)} \circ g) \suml_{B_i \in \pi^*} g^{(|B_i + 1|)} \cdot \prodl_{B \in \pi \bs B_i} g^{(|B|)}
			\]
			что в точности совпадает с правой частью суммы из производной слагаемого:
			\begin{multline*}
				(f^{(|\pi^*|)} \circ g) \cdot \left(\prodl_{B \in \pi^*} g^{(|B|)}\right)' = (f^{(|\pi^*|)} \circ g) \cdot \left(g^{(|B_1|)} \cdot \prodl_{B \in \pi^* \bs B_1} g^{(|B|)}\right)' = \\
				(f^{(|\pi^*|)} \circ g) \cdot \left(g^{(|B_1| + 1)} \cdot \prodl_{B \in \pi^* \bs B_1} g^{(|B|)} + g^{(|B_1|)} \cdot \left(\prodl_{B \in \pi^* \bs B_1} g^{(|B|)}\right)'\right) = \dots = \\
				(f^{(|\pi^*|)} \circ g) \suml_{B_i \in \pi^*} g^{(|B_i + 1|)} \cdot \prodl_{B \in \pi \bs B_i} g^{(|B|)}
			\end{multline*}
		\end{enumerate}
	\end{itemize}
\end{proof}

\begin{example}
	Рассмотрим $(f \circ g)^{(4)}$. Все разбиения тогда можно представить как способы разложить 4 на слагаемые(каждое слагаемое символизирует $|B|$):
	\begin{itemize}
		\item $4 = 4 \Ra$ есть только 1 разбиение длины 4 - это всё множество. То есть слагаемое имеет вид
		\[
			\left(f' \circ g\right) \cdot g''''
		\]
		\item $4 = 3 + 1$ - есть $C_4^1 = 4$ способов сделать разбиение с такими мощностями $B$. Нетрудно убедиться, что каждое разбиение имеет одинаковое слагаемое
		\[
			\left(f'' \circ g\right) \cdot g' \cdot g'''
		\]
		Их сумма будет соответственно
		\[
			4 \cdot \left(f'' \circ g\right) \cdot g' \cdot g'''
		\]
		\item $4 = 2 + 1 + 1$ - это $C_4^2 = 6$ способов сделать разбиение
		\[
			\Ra 6 \cdot \left(f''' \circ g\right) \cdot g'' \cdot g' \cdot g'
		\]
		\item $4 = 2 + 2$ - это $\frac{C_4^2}{2} = 3$, потому что мы учитываем каждое разбиение дважды
		\[
			3 \cdot \left(f'' \circ g\right) \cdot g'' \cdot g''
		\]
		\item $4 = 1 + 1 + 1 + 1$ - это 1 разбиение
		\[
			\left(f'''' \circ g\right) \cdot g' \cdot g' \cdot g' \cdot g'
		\]
	\end{itemize}
	Собственно, $(f \circ g)^{(4)}$ - это сумма всех полученных слагаемых
\end{example}

\subsubsection*{Дифференциалы высших порядков}

\begin{definition}
	\textit{Дифференциалом $n$-го порядка} от $f(x)$, где $x$ - независимая переменная, в точке $a$ называется дифференциал от дифференциала $n - 1$-го порядка, причём в качестве $dx$ в каждом из дифференциалов перётся одно и то же число ($\Delta x$)
	\[
		d^{n}f := d(d^{n - 1}f)
	\]
\end{definition}

\begin{corollary}
	\[
		d^nf(a) = f^{(n)}(a) dx^n
	\]
	если $x$ - независимая переменная
\end{corollary}

\begin{example}
	Пусть $f(x)$ - функция переменной $x = \varphi(t)$, где $t$ - независимая переменная.
	\[
		f(x) = f(\varphi(t)) = h(t)
	\]
	Отсюда
	\begin{multline*}
		d^2f = d^2h = h'' dt^2 = \left((f' \circ \varphi) \cdot \varphi'\right)dt^2 = \left((f'' \circ \varphi)(\varphi')^2 + (f' \circ \varphi)\varphi''\right)dt^2 = \\
		(f'' \circ \varphi)(\varphi'dt)^2 + (f' \circ \varphi)\varphi'' dt^2 = f''dx^2 + f'd^2x
	\end{multline*}
\end{example}

\subsection{Свойства производных}

\begin{definition}
	Точка $x_0$ называется \textit{точкой (строго) локального максимума} функции $f(x)$, если $\forall x$ из некоторой $\mc{U}_{\delta}(x_0)$ выполняется $f(x) \le f(x_0)\ (f(x) < f(x_0))$
\end{definition}

\begin{definition}
	Точка $x_0$ называется \textit{точкой (строго) локального минимума} функции $f(x)$, если $\forall x$ из некоторой $\mc{U}_{\delta}(x_0)$ выполняется $f(x) \ge f(x_0)\ (f(x) > f(x_0))$
\end{definition}

\begin{definition}
	Точки локального минимума и максимума в общем случае называются \textit{точками локального экстремума}
\end{definition}

\begin{theorem} (Теорема Ферма)
	Если $x_0$ - точка локального экстремума функции $f$, дифференцируемой в $x_0$, то
	\[
		f'(x_0) = 0
	\]
\end{theorem}

\begin{proof}
	Пусть $x_0$ - точка локального максимума. Это значит, что в некоторой окрестности точки $x_0$ имеет место неравенство
	\[
		f(x_0 + \Delta x) \le f(x_0)
	\]
	Рассмотрим односторонние пределы производной:
	\begin{align*}
		f'_+(x_0) = \liml_{\Delta x \to 0+} \frac{f(x_0 + \Delta x) - f(x_0)}{\Delta x} \le 0
		\\
		f'_-(x_0) = \liml_{\Delta x \to 0-} \frac{f(x_0 + \Delta x) - f(x_0)}{\Delta x} \ge 0
	\end{align*}
	В силу дифференцируемости $f$ в точке $x_0$:
	\[
		0 \le f'_-(x_0) = f'(x_0) = f'_+(x_0) \le 0
	\]
	Что верно тогда и только тогда, когда
	\[
		f'(x_0) = 0
	\]
\end{proof}

\begin{theorem} (Ролля)
	Если $f$ непрерывна на $[a; b]$, дифференцируема на $(a; b)$ и $f(a) = f(b)$, то
	\[
		\exists c \in (a; b) \such f'(c) = 0
	\]
\end{theorem}

\begin{proof}
	$f$ непрерывна на $[a; b]$. Пусть $f$ не постоянна (иначе тривиально). Тогда, по теореме Вейерштрасса
	\[
		\exists \max\limits_{x \in [a; b]} f(x) > \min\limits_{x \in [a; b]} f(x)
	\]
	Хотя бы одно из этих чисел (пусть $\max$) не совпадает с $f(a) = f(b)$. Следовательно,
	\[
		\max\limits_{x \in [a; b]} f(x) = f(c),\ c \in (a; b)
	\]
	То есть $c$ - точка локального максимума. По теореме Ферма $f'(c) = 0$
\end{proof}

\begin{theorem} (Обобщенная теорема о среднем) \label{common_mid}
	Если $f, g$ непрерывны на $[a; b]$, дифференцируемы на $(a; b)$, то
	\[
		\exists c \in (a; b) \such (f(b) - f(a))g'(c) = (g(b) - g(a))f'(c)
	\]
\end{theorem}

\begin{proof}
	Рассмотрим функцию $h(x)$:
	\[
		h(x) = (f(b) - f(a))g(x) - (g(b) - g(a))f(x)
	\]
	Посчитаем $h(a)$ и $h(b)$:
	\begin{align*}
		h(a) = (f(b) - f(a))g(a) - (g(b) - g(a))f(a) = f(b)g(a) - g(b)f(a)
		\\
		h(b) = (f(b) - f(a))g(b) - (g(b) - g(a))f(b) = g(a)f(b) - f(a)g(b)
	\end{align*}
	Отсюда по теореме Ролля
	\[
		\exists c \in (a; b) \such h'(c) = 0
	\]
	А это в свою очередь значит
	\begin{align*}
		(f(b) - f(a))g'(c) - (g(b) - g(a))f'(c) = 0
		\\
		(f(b) - f(a))g'(c) = (g(b) - g(a))f'(c)
	\end{align*}
\end{proof}

\begin{corollary} (Теорема Лагранжа о среднем)
	Если $f$ непрерывна на $[a; b]$, дифференцируема на $(a; b)$, то $\exists c \in (a; b)$ такое, что
	\[
		\frac{f(b) - f(a)}{b - a} = f'(c)
	\]
\end{corollary}

\begin{proof}
	По теореме \ref{common_mid} возьмём $g(x) = x$
\end{proof}

\begin{corollary} (Теорема Коши о среднем)
	Если $f, g$ непрерывна на $[a; b]$, дифференцируема на $(a; b)$, $g'$ не обращается в нуль на $(a; b)$, то
	\[
		\exists c \in (a; b) \such \frac{f(b) - f(a)}{g(b) - g(a)} = \frac{f'(c)}{g'(c)}
	\]
\end{corollary}

\begin{proof}
	Всё, что нам надо обосновать, так это то, что мы можем поделить обе части уравнения в теореме \ref{common_mid} за счёт данных нам условий.
	
	Предположим, что $g(b) = g(a)$. Но тогда $g$ удовлетворяет требованиям теоремы Ролля. Следовательно,
	\[
		\exists t \in (a; b) \such g'(t) = 0
	\]
	что противоречит условию. Отсюда следует, что мы можем смело поделить обе части на $(g(b) - g(a)) \cdot g'(c)$
\end{proof}

\begin{note}
	Смысл теоремы Коши тот же самый, что и у теоремы Лагранжа, но в предположении, что некоторая функция задана параметрически на осях: $x = g(t)$ и $y = f(t)$ соответственно.
\end{note}

\begin{example}
	\[
		f(x) = \System{
			&{x^2 \sin \frac{1}{x},\ x \neq 0}
			\\
			&{0,\ x = 0}
		}
	\]
	Функция $f$ - непрерывная и дифференцируемая на $\R$. Но пример примечателен тем, что производная всё же разрывная в 0.
	\begin{itemize}
		\item $x \neq 0$. Тогда, просто посчитаем производную по свойствам:
		\[
			f'(x) = 2x \sin \frac{1}{x} - \cos \frac{1}{x}
		\]
		
		\item $x = 0$. Посчитаем эту производную по определению:
		\[
			f'(0) = \liml_{\Delta x \to 0} \frac{f(\Delta x) - 0}{\Delta x} = \liml_{\Delta x \to 0} \frac{\Delta x^2 \sin \frac{1}{\Delta x}}{\Delta x} = \liml_{\Delta x \to 0} \Delta x \sin \frac{1}{\Delta x} = 0
		\]
		Но при этом $\not\exists \liml_{x \to 0} f'(x)$. Проверить это можно, взяв 2 последовательности Гейне:
		\begin{align*}
			&{x_n = \frac{1}{\frac{\pi}{2} + 2\pi n} \Ra f'(x_n) = \frac{2}{\frac{\pi}{2} + 2\pi n} \Ra \liml_{n \to \infty} f'(x_n) = 0}
			\\
			&{x_n = \frac{1}{2\pi n} \Ra f'(x_n) = -1 \Ra \liml_{n \to \infty} f'(x_n) = -1}
		\end{align*}
		То есть $f'(x)$ разрывная в нуле.
	\end{itemize}
\end{example}

\subsubsection*{Геометрический смысл теорем Ролля и Лагранжа}

\begin{tabular}{cc}
	\begin{tikzpicture}[scale = 1]
		% Axis
		\coordinate (y) at (0,5);
		\coordinate (x) at (6,0);
		\draw[<->, style={thick}] (y) node(yaxis)[above]{$y$} -- (0,0) --  (x) node(xaxis)[right]{$x$};
		\draw (-0.4,0) -- (0,0) --  (0,-0.4);
		
		% Points for the curve
		\path
		coordinate (start) at (1, 2)
		coordinate (d1) at (2, 2.5)
		coordinate (top) at (3, 4)
		coordinate (d2) at (4, 2.5)
		coordinate (end) at (5, 2);
		
		% Draw the curve
		\draw[style={thick}] plot [smooth, tension=0.9] coordinates {(start) (d1) (top) (d2) (end)};
		
		% Dashed lines for the points on x axis
		\draw[style={dashed}] let \p1=(start) in (\p1) -- (\x1, 0) node[circle, fill, inner sep=1pt, label={below:$a$}]{};
		\draw[style={dashed}] let \p1=(top) in (\p1) node[circle, fill, inner sep=1pt, label={above:$f'(c) = 0$}]{} -- (\x1, 0) node[circle, fill, inner sep=1pt, label={below:$c$}]{};
		\draw[style={dashed}] let \p1=(end) in (\p1) -- (\x1, 0) node[circle, fill, inner sep=1pt, label={below:$b$}]{};
		
		% Dashed lines for the points on y axis
		\draw[style={dashed}] let \p1=(end) in (\p1) -- (0, \y1) node[left, black]{$f(a) = f(b)$};
		
		% Tangent line for top point
		\draw let \p1=(top) in (1.5, \y1) -- (4.5, \y1);
	\end{tikzpicture}
	&
	\begin{tikzpicture}[scale = 1]
		% Axis
		\coordinate (y) at (0,5);
		\coordinate (x) at (6,0);
		\draw[<->] (y) node[above]{$y$} -- (0,0) --  (x) node[right]{$x$};
		\draw (-0.4,0) -- (0,0) --  (0,-0.4);
		
		% Points for the curve
		\path
		coordinate (start) at (1, 1)
		coordinate (d1) at (2, 1)
		coordinate (d2) at (4, 4)
		coordinate (end) at (5, 4);
		
		% The curve
		\draw[style={thick}] plot [smooth, tension=0.9] coordinates {(start) (d1) (d2) (end)};
		
		% Tangent line and point
		\coordinate (t) at (1.85, 0.85);
		\filldraw[black] (t) node[below right]{\scalebox{0.8}{$f'(c) = \frac{f(b) - f(a)}{b - a}$}} circle (1pt);
		\draw (1.5 - 0.3, 0.7 - 0.3) -- (1.5 + 1.5, 0.7 + 1);
		
		% Dashed lines for the points on x axis
		\draw[style={dashed}] let \p1=(start) in (\p1) -- (\x1, 0) node[circle, fill, inner sep=1pt, label={below:$a$}]{};
		\draw[style={dashed}] let \p1=(t) in (\p1) -- (\x1, 0) node[circle, fill, inner sep=1pt, label={below:$c$}]{};
		\draw[style={dashed}] let \p1=(end) in (\p1) -- (\x1, 0) node[circle, fill, inner sep=1pt, label={below:$b$}]{};
		
		% Dashed lines for the points on y axis
		\draw[style={dashed}] let \p1=(start) in (\p1) -- (0, \y1) node[circle, fill, inner sep=1pt, label={left:$f(a)$}]{};
		\draw[style={dashed}] let \p1=(end) in (\p1) -- (0, \y1) node[circle, fill, inner sep=1pt, label={left:$f(b)$}]{};
		
		% Line from the start to the end point
		\draw[style={dashed}] (start) node[circle, fill, inner sep=1pt]{} -- (end) node[circle, fill, inner sep=1pt]{};
	\end{tikzpicture}
\end{tabular}