\section{Предварительные сведения}

\subsection{Элементы математической логики}

\begin{definition}
    Высказывание - это выражение, принимающее либо значение истины $(1)$, либо ложности $(0)$.
\end{definition}

\begin{definition}
    Предикат - это высказывание, зависящее от переменных.
\end{definition}

\subsubsection*{Обозначения}

\begin{enumerate}
    \item Высказывания обозначаюся заглавными латинскими буквами: $A, B, \dots$
    \item $A(x_1, \dots, x_n)$ - предикат
    \item $:=$ - является по определению
    \item $\neg A$ - отрицание высказывания $A$. Логическое "не".
    \item $A \wedge B$ - конъюнкция. Логическое "и".
    \item $A \vee B$ - дизъюнкция. Логическое "или".
    \item $A \ra B$ - импликация. Логическое "если $A$, то $B$".
    \item $A \lra B$ - эквивалентность
\end{enumerate}

\begin{example}[предиката]
    $B(a) = ((\forall b \in \R)\ (\exists c \in \R)\ |\ (\forall x \in \R)\ ax^2 + bx + c \ge 0)$
\end{example}


\subsubsection*{Эквивалентность и равносильность}

Эквивалентность $\lra$ нужно не путать с логической равносильностью $\equiv$. Первое является логической операцией, тогда как второе точно гарантирует, что высказывание $B$ имеет ровно то же значение, что и высказывание $A$.


\subsection{"Наивная"\ теория множеств}

\subsubsection*{Обозначения}

\begin{enumerate}
    \item $A, B, \dots$ - множества. Обозначаются заглавными латинскими буквами (как правило).
    \item $a \in A$ - элемент $a$ принадлежит множеству $A$. То же самое, что и $A \ni a$.
    \item $\neg (a \in A) \lra a \notin A$
\end{enumerate}


\subsubsection*{Операции над множествами}

\begin{definition}
    \textit{Объединением множеств} $A$ и $B$ называется множество $A \cup B := \{x\ |\ (x \in A) \vee (x \in B)\}$
\end{definition}

\begin{definition}
    \textit{Пересечением множеств} $A$ и $B$ называется множество $A \cap B := \{x\ |\ (x \in A) \wedge (x \in B)\}$
\end{definition}

\begin{definition}
    \textit{Разностью множеств} $A$ и $B$ называется множество $A \bs B := \{x\ |\ (x \in A) \wedge (x \notin B)\}$.
    
    Также используется и второе обозначение $A \bs B \lra C_A B$. $C$ - это сокращение от французского \textit{complément}.
\end{definition}

\begin{definition}
    \textit{Симметрической разностью множеств} $A$ и $B$ называется множество $A \triangle B := \{x\ |\ (x \in (A \bs B)) \vee (x \in (B \bs A))\} \lra (A \bs B) \cup (B \bs A)$
\end{definition}

\begin{definition}
    \textit{Универсальным множеством} $U$ называется такое множество, которое включает в себя все множества рассматриваемой системы, кроме самого себя. Обычно обозначается как $U$ от слова \textit{universal}.
\end{definition}

\begin{definition}
    \textit{Дополнением множества} $A$ называется $C_U A$. Другим обозначением служит $A^C := C_U A$
\end{definition}

\begin{definition}
    \textit{Пустым множеством} $\emptyset$ называется такое множество, которое не содержит в себе элементов. Оно существует и единственно.
\end{definition}


\subsubsection*{Свойства операций над множествами}

\begin{itemize}
    \item Коммутативность
    \begin{align*}
        A \cup B = B \cup A \\
        A \cap B = B \cap A
    \end{align*}
    \item Ассоциативность
    \begin{align*}
        (A \cup B) \cup C = A \cup (B \cup C) \\
        (A \cap B) \cap C = A \cap (B \cap C)
    \end{align*}
    \item Дистрибутивность
    \begin{align*}
        A \cup (B \cap C) = (A \cup B) \cap (A \cup C) \\
        A \cap (B \cup C) = (A \cap B) \cup (A \cap C)
    \end{align*}
    \item Идемпотентность
    \begin{align*}
        A \cup A = A \\
        A \cap A = A
    \end{align*}
    \item Двойственность (правила де Моргана)
    \begin{align*}
        (A \cup B)^C = A^C \cap B^C \\
        (A \cap B)^C = A^C \cup B^C
    \end{align*}
    \item Универсальное множество
    \begin{align*}
        U \cap A = A \\
        U \cup A = U
    \end{align*}
    \item Пустое множество
    \begin{align*}
        \emptyset \cap A = \emptyset \\
        \emptyset \cup A = A
    \end{align*}
\end{itemize}


\subsection{Отображения и функции}

\begin{definition}
    $(f \colon X \ra Y) \lra \left(\forall x \in X\ \exists! y \in Y\right)\ y = f(x)$ - отображение (функция) из $X$ в $Y$
\end{definition}

\begin{definition}
    Множество $X$ называется \textit{областью определения} $f$
\end{definition}

\begin{definition}
    $f(X)$ - множество значений $f$. Более точная формулировка выглядит так:
    $$
    f(X) = \{y \in Y \such \exists x \in X\ f(x) = y\}
    $$
    Ещё множество значений $f$ называют \textit{образом множества} $X$
\end{definition}


\subsubsection*{Свойства отображений}

\begin{itemize}
    \item Инъекция \\
    $(f(x_1) = f(x_2)) \lra (x_1 = x_2)$, то есть функция однозначна
    \item Сюръекция \\
    $f(X) = Y$, то есть $f(x)$ принимает все возможные значения из множества $Y$
    \item Биекция \\
    Когда $f$ одновременно и инъективна, и сюръективна
\end{itemize}

\begin{definition}
    $f^{-1}(Y)$ называется \textit{прообразом множества} $Y$ и определяется как
    $$
        f^{-1}(Y) := \{x \in X \such \exists y \in Y\ y = f(x)\}
    $$
\end{definition}


\subsection{Декартово произведение и отношения}

\begin{definition}
    \textit{Декартовым произведением} называют множество
    $$
        A \times B := \{(a, b)\ |\ (a \in A) \wedge (b \in B)\}
    $$
    При этом $(a, b)$ называется \textit{упорядоченной парой}, то есть, в отличие от множеств, верно $(a, b) \neq (b, a)$
\end{definition}

\begin{definition}
    Декартово произведение множества $X$ на само себя называется \textit{декартовым квадратом} (или куб, если мы говорим о третьей степени). Обозначается как $X \times X = X^2$
\end{definition}

\begin{definition}
    Подмножество $R \subset X^2$ называется \textit{бинарным отношением на множестве $X$} \\
    $(x, y) \in R := xRy$ - введём краткую запись того, что упорядоченная пара принадлежит отношению.
\end{definition}

\begin{definition}
    Бинарное отношение $R$ называется \textit{отношением эквивалентности}, если выполнены условия:
    \begin{enumerate}
        \item Рефлексивность $\forall x \in X \Ra xRx$
        \item Симметричность $\forall x, y \in X (xRy) \Ra (yRx)$
        \item Транзитивность $\forall x, y, z \in X (xRy) \wedge (yRz) \Ra (xRz)$
    \end{enumerate}
\end{definition}

\begin{definition}
    Бинарное отношение $R$ называется \textit{отношением порядка}, если выполнены условия:
    \begin{enumerate}
        \item Рефлексивность $\forall x \in X \Ra xRx$
        \item Антисимметричность $\forall x, y \in X\ (xRy) \wedge (yRx) \Ra (x = y)$
        \item Транзитивность
        $\forall x, y, z \in X\ (xRy) \wedge (yRz) \Ra (xRz)$
    \end{enumerate}
\end{definition}

\begin{adefinition}
    Бинарное отношение $R$ называется \textit{отношением строгого порядка}, если выполнены условия:
    \begin{enumerate}
        \item Антирефлексивность $\forall x \in X \Ra \neg xRx$
        \item Антисимметричность $\forall x, y \in X\ (xRy) \wedge (yRx) \Ra (x = y)$
        \item Транзитивность
        $\forall x, y, z \in X\ (xRy) \wedge (yRz) \Ra (xRz)$
    \end{enumerate}
\end{adefinition}

\begin{example}
    Отношение $\subset$ между множествами является отношением порядка.
\end{example}
\begin{example}
    Отношение $\le$ между действительными числами является тоже отношением порядка.
\end{example}

\begin{definition}
    Говорят, что множество $X$ \textit{линейно упорядоченно}, если на нём задано отношение порядка $\prec$ такое, что $\forall x, y \in X (x \prec y) \vee (y \prec x)$ - всегда истинное высказывание.
\end{definition}

Исходя из этого, нетрудно заметить, что множества сами по себе не являются линейно упорядоченными.

\begin{anote}
    Дополнительно стоит заметить, что отношения на множествах как бы не являются отношениями. Мы не можем говорить о множестве всех множеств, иначе мы получим парадокс Рассела.
\end{anote}

\begin{definition}
    Если на множестве $X$ задано отношение порядка, но оно не является линейно упорядоченным, то его называют \textit{частично упорядоченным}.
\end{definition}

\begin{definition}
    Если на множестве $X$ определено отношение эквивалентности $R$, то множество $X$ называется \textit{классом эквивалентности}, если
    \[
    	\forall x, y \in X\ xRy
    \]
\end{definition}

\begin{theorem}
    Если на множестве $X$ задано отношение эквивалентности $R$, то $X$ может быть разбито на классы эквивалентности:
    $$
        X = \bigcup\limits_{\alpha \in A} X_{\alpha}
    $$
    При этом выполнены свойства:
    \begin{enumerate}
        \item $\forall \alpha_1 \neq \alpha_2\ X_{\alpha_1} \cap X_{\alpha_2} = \emptyset$
        \item $(\forall \alpha)(\forall x, y \in X_{\alpha})\ xRy$
        \item $(\forall \alpha_1 \neq \alpha_2)(\forall x \in X_{\alpha_1})(\forall y \in X_{\alpha_2})\ \neg (xRy)$
    \end{enumerate}
\end{theorem}