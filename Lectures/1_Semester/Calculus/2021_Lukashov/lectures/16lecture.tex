\begin{theorem} (Дарбу)
	Пусть $f$ дифференцируема на $(a; b)$ и $\exists f'_+(a), f'_-(b) \in \R$. Тогда для любого $c$ в интервале между $f'_+(a)$ и $f'_-(b)$ существует $\xi \in (a; b)$ такое, что $f'(\xi) = c$
\end{theorem}

\begin{proof}~
	\begin{enumerate}
		\item $c = 0 \Ra f'_+(a) \cdot f'_-(b) < 0$
		
		Так как $f$ дифференцируема на $(a; b)$ и односторонние производные конечны, то $f$ непрерывна на $[a; b]$. А значит по теореме Вейерштрасса
		\[
			\exists \xi \in [a; b] \such f(\xi) = \max\limits_{x \in [a; b]} f(x)
		\]
		При этом $\xi \neq a$, так как по определению односторонней производной
		\[
			f'_+(a) = \liml_{\Delta x \to 0+} \frac{f(a + \Delta x) - f(a)}{\Delta x}
		\]
		Если предположить, что $f'_+(a) > 0$, то из равенства выше следует
		\[
			f(a + \Delta x) - f(a) > 0 \Ra f(a + \Delta x) > f(a)
		\]
		для достаточного малого $\Delta x$. Поэтому точка $a$ точно не максимум на отрезке. Аналогично получим, что $\xi \neq b$, и в данном предположении $f'_-(b) < 0$. Таким же образом рассматривается случай, когда $f'_+(a) < 0 \Ra f'_-(b) > 0$. Отсюда
		\[
			\exists \xi \in (a; b) \such f(\xi) = \max\limits_{x \in [a; b]} f(x)
		\]
		и по теореме Ферма получаем, что
		\[
			f'(\xi) = 0 = c
		\]
		
		\item $c \neq 0$
		
		Сведём случай к уже доказанному. Достаточно рассмотреть вспомогательную функцию
		\[
			F(x) = f(x) - cx
		\]
		Так как прямая определена и дифференцируема на всей числовой прямой, то
		\[
			F'(x) = f'(x) - c
		\]
		для интервала $(a; b)$. По уже доказанному, 
		\[
			\exists \xi \in (a; b) \such F'(\xi) = f'(\xi) - c = 0 \Ra f'(\xi) = c
		\]
	\end{enumerate}
\end{proof}

\begin{corollary}
	Если $f$ дифференцируема на $(a; b)$, то $f'$ не может иметь точек разрыва первого рода.
\end{corollary}

\begin{proof}
	От противного. Пусть точка $x_0 \in (a; b)$ - разрыв первого рода. Тогда, как минимум 2 числа из $f'(x_0 - 0), f'(x_0), f'(x_0 + 0)$ не равны друг другу.
	
	Доказательство сводится к разбору 6 случаев. Для примера докажем случай, когда $f'(x_0) < f'(x_0 + 0)$. Тогда, раз $f'(x_0 + 0)$ существует и конечен, то
	\[
		\forall \eps > 0\ \exists \delta > 0 \such \forall x \in (x_0; x_0 + \delta)\ |f'(x) - f'(x_0 + 0)| < \eps
	\]
	Положим $\eps := \frac{f'(x_0 + 0) - f'(x_0)}{2}$. Отсюда следует, что
	\[
		\exists \delta > 0 \such \forall x \in (x_0; x_0 + \delta)\ f'(x) > f'(x_0 + 0) - \frac{f'(x_0 + 0) - f'(x_0)}{2} = f'(x_0) + \frac{f'(x_0 + 0) - f'(x_0)}{2}
	\]
	Теперь посмотрим на интервал $(x_0; x_0 + \delta / 2)$. Так как функция дифференцируема на $(a; b) \supset (x_0; x_0 + \delta / 2)$, то $f$ на этом интервале удовлетворяет условиям теоремы Дарбу. Значит
	\[
		\forall c \in (f'_+(x_0); f'(x_0 + \delta / 2))\ \exists \xi \in (x_0; x_0 + \delta / 2) \such f'(\xi) = c
	\]
	при этом $f'_+(x_0) = f'(x_0)$ и  $f'(x_0 + \delta / 2) > f'(x_0) + \frac{f'(x_0 + 0) - f'(x_0)}{2}$. То есть для $c = f'(x_0) + \frac{f'(x_0 + 0) - f'(x_0)}{2}$ у нас найдётся $\xi$ такое, что $f'(\xi) = c$. Но при этом $\xi \in (x_0; x_0 + \delta / 2) \subset (x_0; x_0 + \delta)$, а значит $f'(\xi) > f'(x_0) + \frac{f'(x_0 + 0) - f'(x_0)}{2}$. Получили противоречие.
\end{proof}

\begin{theorem} (Правило Лопиталя для случая $\frac{0}{0}$)
	Пусть $f$, $g$ дифференцируемы на $(a; b)$, при этом $\exists \liml_{x \to a+0} f(x) = \liml_{x \to a+0} g(x) = 0$ и $\exists \liml_{x \to a+0} \frac{f'(x)}{g'(x)} = C \in \bar{\R}$. Тогда
	\[
		\exists \liml_{x \to a+0} \frac{f(x)}{g(x)} = C
	\]
\end{theorem}

\begin{note}
	Аналогичное утверждение верно для предела $x \to b-0$, а также для любого предела $x \to x_0,\ x_0 \in (a; b)$.
\end{note}

\begin{proof}
	Доопределим $f(a) = g(a) = 0$. Из существование предела отношения производных следует, что
	\[
		\exists \delta > 0 \such \forall x \in (a; a + \delta)\ g'(x) \neq 0
	\]
	Следовательно, на отрезке $\left[a; a + \frac{\delta}{2}\right]$ для функции $g$ выполнены все условия теоремы Коши о среднем. Значит
	\[
		\forall x \in \left(a; a + \frac{\delta}{2}\right)\ \exists \xi \in (a; x) \such \frac{f(x) - f(a)}{g(x) - g(a)} = \frac{f'(\xi)}{g'(\xi)} = \frac{f(x)}{g(x)}
	\]
	при этом понятно, что $\xi = \xi(x)$. Так как $a < \xi(x) < x$, то если устремить $x$ к $a + 0$, то $a < \xi(x) \le a + 0 \Ra \xi(x) \to a + 0$. Отсюда получаем
	\[
		\liml_{x \to a + 0} \frac{f'(\xi(x))}{g'(\xi(x))} = \liml_{x \to a + 0} \frac{f'(x)}{g'(x)} = C = \liml_{x \to a + 0} \frac{f(x)}{g(x)}
	\]
\end{proof}

\begin{corollary} (Признак дифференцируемости)
	Если $f$ дифференцируема в $\mc{U}_\delta(x_0)$, непрерывна в $x_0$ и $\exists \liml_{x \to x_0} f'(x) \in \bar{\R}$, то
	\[
	\exists f'(x_0) = \liml_{x \to x_0} f'(x)
	\]
\end{corollary}

\begin{proof}
	Пусть $F(x) = f(x) - f(x_0)$, $g(x) = x - x_0$. Тогда, $\forall x \in \mc{U}_\delta(x_0)$
	\begin{align*}
	&F'(x) = f'(x)
	\\
	&g'(x) = 1
	\end{align*}
	Следовательно,
	\[
	\exists \liml_{x \to x_0} \frac{F'(x)}{g'(x)} = \liml_{x \to x_0} \frac{f'(x)}{g'(x)} \in \bar{\R}
	\]
	при этом $\liml_{x \to x_0} F(x) = \liml_{x \to x_0} g(x) = 0$. А значит
	\[
	\exists \liml_{x \to x_0} \frac{F(x)}{g(x)} = \liml_{x \to x_0} \frac{f(x) - f(x_0)}{x - x_0} = f'(x_0) = \liml_{x \to x_0} \frac{F'(x)}{g'(x)} \in \bar{\R}
	\]
\end{proof}

\begin{theorem} (Правило Лопиталя для случая $\frac{\infty}{\infty}$)
	Пусть $f$, $g$ дифференцируемы на $(a; b)$, $\exists \liml_{x \to a + 0} g(x) = \pm \infty$ и $\exists \liml_{x \to a + 0} \frac{f'(x)}{g'(x)} = C \in \bar{\R}$. Тогда
	\[
		\exists \liml_{x \to a + 0} \frac{f(x)}{g(x)} = C
	\]
\end{theorem}

\begin{note}
	Аналогичное утверждение верно для предела $x \to b-0$, а также для любого предела $x \to x_0,\ x_0 \in (a; b)$.
\end{note}

\begin{proof}
	Докажем случай для $\liml_{x \to a + 0} g(x) = +\infty$:
	\begin{itemize}
		\item $C = -\infty$. Тогда, выберем $\forall p > q > C$. Из существования предела отношения производных следует, что
		\begin{align*}
			&\exists \delta_1 > 0 \such \forall x \in (a; a + \delta_1)\ \ \frac{f'(x)}{g'(x)} < q
			\\
			&\exists \delta_2 \in (0; \delta_1) \such \forall x \in (a; a + \delta_2)\ \ g(x) > 0
		\end{align*}
		Зафиксируем $y > x > a, y \in (a; a + \delta_2)$. Тогда
		\[
			\exists \delta_3 > 0 \such \forall x \in (a; a + \delta_3) \subset (a; y)\ \ g(y) - g(x) < 0
		\]
		Заметим, что $f$, $g$ удовлетворяют условиям теоремы Коши о среднем на отрезке $[x; y]$. То есть
		\[
			\exists \xi \in (x; y) \subset (a; a + \delta_1) \such \frac{f(y) - f(x)}{g(y) - g(x)} = \frac{f'(\xi)}{g'(\xi)} < q
		\]
		Если убрать равенство с $\xi$, то получим
		\[
			\frac{f(y) - f(x)}{g(y) - g(x)} < q
		\]
		Умножим обе части на $\frac{g(x) - g(y)}{g(x)} > 0$ при $x \in (a; a + \delta_3)$. Отсюда имеем
		\begin{align*}
			&{\frac{f(x) - f(y)}{g(x)} < q \cdot \frac{g(x) - g(y)}{g(x)}}
			\\
			&{\Ra \frac{f(x)}{g(x)} < q - q \cdot \frac{g(y)}{g(x)} + \frac{f(y)}{g(x)}}
		\end{align*}
		Из последнего неравенства следует, что
		\[
			\forall p > C\ \exists \delta_4 > 0 \such \forall x \in (a; a + \delta_4)\ \ \frac{f(x)}{g(x)} < p
		\]
		Откуда согласно $C = -\infty$ имеем
		\[
			\forall \eps > 0\ \exists \delta_4 > 0 \such \forall x \in (a; a + \delta_4)\ \ \frac{f(x)}{g(x)} < -\frac{1}{\eps} \lra \liml_{x \to a + 0} \frac{f(x)}{g(x)} = -\infty = \liml_{x \to a + 0} \frac{f'(x)}{g'(x)}
		\]
		
		\item $C = +\infty$. Тогда аналогично случаю выше, получается утверждение
		\[
			\forall r < C\ \exists \delta_5 > 0 \such \forall x \in (a; a + \delta_5)\ \ \frac{f(x)}{g(x)} > r
		\]
		
		\item $-\infty < C < +\infty$. Ключевые утверждения, полученные выше, будут верны и в конечном случае, потому что $\exists p > C > r$. А значит
		\[
			\forall \eps > 0 \exists \delta := \min(\delta_4, \delta_5) \such \forall x \in (a; a + \delta)\ \ r < \frac{f(x)}{g(x)} < p
		\]
		где $r := C - \eps,\ p := C + \eps$. То есть
		\[
			\forall \eps > 0 \exists \delta \such \forall x \in (a; a + \delta)\ \ \left|\frac{f(x)}{g(x)} - C\right| < \eps \lra \liml_{x \to a + 0} \frac{f(x)}{g(x)} = C = \liml_{x \to a + 0} \frac{f'(x)}{g'(x)}
		\]
	\end{itemize}
\end{proof}

\begin{note}
	Правило Лопиталя работает и в случаях, когда $x \to \pm \infty$
\end{note}

\subsection{Равномерная непрерывность}

\begin{definition}
	$f$ \textit{равномерно непрерывна} на множестве $X \subset \R$, если
	\[
		\forall \eps > 0\ \exists \delta > 0 \such \forall x, y \in X, |x - y| < \delta\ \left|f(x) - f(y)\right| < \eps
	\]
\end{definition}

\begin{note}
	Отличие от обычного определения заключается в том, что выбор $\delta$ не зависит от рассматриваемой точки $x$.
\end{note}

\begin{example}
	\[
		X = (0; 1),\ f(x) = \frac{1}{x} \text{ - неравномерно непрерывна}
	\]
	То есть нужно доказать утверждение
	\[
		\exists \eps > 0 \such \forall \delta > 0\ \exists x, y \in X, |x - y| < \delta\ \left|\frac{1}{x} - \frac{1}{y}\right| \ge \eps
	\]
	Для любого $\delta > 0$ найдётся $n \in \N$ такое, что верно неравенство
	\[
		\frac{1}{n} \le \delta < \frac{1}{n - 1}
	\]
	Положим $x_n = \frac{1}{n}$, а $y_n = \frac{1}{3n}$. Тогда
	\begin{align*}
		&{|x_n - y_n| = \frac{2}{3n} < \frac{1}{n} \le \delta}
		\\
		&{\left|\frac{1}{x_n} - \frac{1}{y_n}\right| = 2n \ge 2}
	\end{align*}
	Отсюда наше утверждение выполнено $\forall \eps \le 2$, что и требовалось доказать.
\end{example}