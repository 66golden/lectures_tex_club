\subsection{Предел последовательности}

\begin{definition}
    Число $l \in \R$ называется \textit{пределом последовательности} $\{x_n\}_{n = 1}^\infty \subset R$, если
    $$
        \forall \eps > 0\ \exists N \in \N\ |\ \forall n > N\ |x_n - l| < \eps
    $$
\end{definition}

\begin{definition}
    Говорят, что последовательность $\{x_n\}_{n = 1}^\infty$ \textit{сходящаяся}, или \textit{сходится} к $l$, если $\exists l \in \R\ |\ \liml_{n \to \infty} x_n = l$
\end{definition}

\begin{example}
    $$
        \liml_{n \to \infty} \frac{1}{n} = 0 \lra \forall \eps > 0\ \exists N \in \N\ |\ n > N\ |\frac{1}{n} - 0| < \eps
    $$
    Положим $N := \left\lceil\frac{1}{\eps}\right\rceil + 1$. Тогда $\forall n > N \Ra n > \frac{1}{\eps} \Ra |\frac{1}{n} - 0| < \eps \Ra$ предел доказан.
\end{example}

\begin{theorem}
    Числовая последовательность может иметь не более чем один предел.
\end{theorem}

\begin{proof}
    Предположим, что $\exists l_1, l_2 \in \R\ |\ \liml_{n \to \infty} x_n = l_1, \liml_{n \to \infty} x_n = l_2$. Тогда:
    $$
        \System{
        \liml_{n \to \infty} x_n = l_1 \lra \forall \eps > 0\ \exists N_1 \in \N\ |\ \forall n > N_1\ |x_n - l_1| < \eps \lra l_1 - \eps < x_n < l_1 + \eps
        \\
        \liml_{n \to \infty} x_n = l_2 \lra \forall \eps > 0\ \exists N_2 \in \N\ |\ \forall n > N_2\ |x_n - l_2| < \eps \lra l_2 - \eps < x_n < l_2 + \eps
        }
    $$
    Рассмотрим $\eps = \frac{l_2 - l_1}{2} > 0$, $\forall n > max(N_1, N_2)$:
    $$
        \System{
        l_1 + \eps = l_1 + \frac{l_2 - l_1}{2} = \frac{l_1 + l_2}{2}
        \\
        l_2 - \eps = l_2 - \frac{l_2 - l_1}{2} = \frac{l_1 + l_2}{2}
        }
    $$
\end{proof}

\subsubsection*{Свойства предела, связанные с неравенствами}

\begin{theorem}
	\begin{enumerate}
		\item (Ограниченность сходящейся последовательности) Если последовательность сходится, то она ограничена.
		
		\item (Отделенность от нуля и сохранение знака) Если последовательность $\{x_n\}_{n = 1}^\infty$ сходится к $l \neq 0$, то $\exists N \in \N \such \forall n > N\ \sgn x_n = \sgn l$ и $|x_n| > \frac{|l|}{2}$
		
		\item (Переход к пределу в неравенстве) Если $\liml_{n \to \infty} x = x_0$, $\liml_{n \to \infty} y = y_0$ и $\exists N \in \N\ |\ n \ge N\ \ x_n \le y_n$, то $x_0 \le y_0$
		
		\item (О промежуточной последовательности) Если $\liml_{n \to \infty} x_n = \liml_{n \to \infty} z_n = l$ и $\exists N \in \N\ |\ \forall n > N\ \ x_n \le y_n \le z_n$, то $\liml_{n \to \infty} y_n = l$
	\end{enumerate}
\end{theorem}

\begin{proof}
	\begin{enumerate}
		\item По условию, $\exists l \in \R\ |\ \forall \eps > 0\ \exists N \in \N\ |\ \forall n > N\ |x_n - l| < \eps$.
		
		Положим $\eps := 1 > 0$. Тогда $\forall n > N\ l - 1 < x_n < l + 1$. Отсюда следует, что
		\begin{align*}
		x_n \le \max(x_1, x_2, \dots, x_N, l + 1) \Ra \{x_n\}_{n = 1}^\infty - \text{ ограничена сверху}
		\\
		x_n \ge \min(x_1, x_2, \dots, x_N, l - 1) \Ra \{x_n\}_{n = 1}^\infty - \text{ ограничена снизу}
		\end{align*}
		
		\item По условию, $\forall \eps > 0\ \exists N \in \N\ |\ \forall n > N\ |x_n - l| < \eps \lra l - \eps < x_n < l + \eps$.
		
		Тогда, рассмотрим $\eps := \frac{|l|}{2} > 0$.
		\begin{align*}
		l > 0 \Ra x_n > l - \eps = \frac{l}{2} > 0
		\\
		l < 0 \Ra x_n < l + \eps = \frac{l}{2} < 0
		\end{align*}
		
		\item От противного. Пусть $x_0 > y_0$. Тогда, по условию:
		\begin{align*}
		\liml_{n \to \infty} x_n = x_0 \lra \forall \eps > 0\ \exists N_1 \in \N\ |\ \forall n > N_1\ x_0 - \eps < x_n < x_0 + \eps
		\\
		\liml_{n \to \infty} y_n = y_0 \lra \forall \eps > 0\ \exists N_2 \in \N\ |\ \forall n > N_2\ y_0 - \eps < y_n < y_0 + \eps
		\end{align*}
		Рассмотрим $\eps := \frac{x_0 - y_0}{2} > 0$, $\forall n > \max(N_1, N_2)$:
		\[
			y_n < y_0 + \eps = \frac{x_0 + y_0}{2} = x_0 - \eps < x_n
		\]
		Получили противоречие.
		
		\item По условию,
		\[
		\System{
			\liml_{n \to \infty} x_n = l \lra \forall \eps > 0\ \exists N_1 \in \N\ |\ \forall n > N_1\ |x_n - l| < \eps
			\\
			\liml_{n \to \infty} z_n = l \lra \forall \eps > 0\ \exists N_2 \in \N\ |\ \forall n > N_2\ |z_n - l| < \eps
		}
		\]
		Отсюда следует: $l - \eps < x_n \le y_n \le z_n < l + \eps \Ra |y_n - l| < \eps$, то есть $\liml_{n \to \infty} y_n = l$
	\end{enumerate}
\end{proof}

\subsubsection*{Арифметические операции со сходящимися последовательностями}

Пусть $\liml_{n \to \infty} x_n = x_0$, $\liml_{n \to \infty} y_n = y_0$. Тогда

\begin{enumerate}
	\item $\liml_{n \to \infty} (x_n + y_n) = x_0 + y_0$
	\item $\liml_{n \to \infty} (x_n - y_n) = x_0 - y_0$
	\item $\liml_{n \to \infty} (x_n \cdot y_n) = x_0 \cdot y_0$
	\item Если $\forall n \in \N\ y_n \neq 0$ и $y_0 \neq 0$, то $\liml_{n \to \infty} \frac{x_n}{y_n} = \frac{x_0}{y_0}$
\end{enumerate}

\begin{proof}
\begin{enumerate}
    \item[1-2.] 
    По определению
    \begin{align*}
        \forall \eps > 0\ \exists N_1 \in \N\ |\ \forall n > N_1\ |x_n - x_0| < \frac{\eps}{2}
        \\
        \forall \eps > 0\ \exists N_2 \in \N\ |\ \forall n > N_2\ |y_n - y_0| < \frac{\eps}{2}
    \end{align*}
    Рассмотрим $\forall n > \max(N_1, N_2)$, тогда
    $$
        |(x_n \pm y_n) - (x_0 \pm y_0)| \le |x_n - x_0| + |y_n - y_0| < \frac{\eps}{2} + \frac{\eps}{2} = \eps
    $$
    \item[3.] 
    $\forall \eps > 0\ \exists N_1 \in \N\ |\ \forall n > N_1\ |x_n - x_0| < \frac{\eps}{2}$
    
    Из теоремы выше, $\exists C > 0\ |\ \forall n \in \N\ |x_n| \le C$
    
    $\liml_{n \to \infty} y_n = y_0 \lra \forall \eps > 0\ \exists N_2 \in \N\ |\ \forall n > N_2\ |y_n - y_0| < \frac{\eps}{2C}$
    
    Рассмотрим $\forall n > \max(N_1, N_2)$:
    $$
        |x_n y_n - x_0 y_0| \le |x_n y_n - x_n y_0| + |x_n y_0 - x_0 y_0| = |x_n| \cdot |y_n - y_0| + |y_0| \cdot |x_n - x_0| < C \cdot \frac{\eps}{2C} + \frac{\eps}{2} = \eps
    $$
    \item[4.]
    По условию,
    \begin{align*}
    	&\forall \eps > 0\ \exists N_1 \in \N\ |\ \forall n > N_1\ |x_n - x_0| < \frac{|y_0|}{2} \cdot \frac{\eps}{2}
    	\\
    	&\forall \eps > 0\ \exists N_2 \in \N\ |\ \forall n > N_2\ |y_n - y_0| < \frac{|y_0|^2}{2(|x_0| + 1)} \cdot \frac{\eps}{2}
    \end{align*}
    Так как $\liml_{n \to \infty} y_n = y_0 \neq 0$, то начиная с некоторого номера $|y_n| > \frac{|y_0|}{2}$. Будем считать, что это верно $\forall n > N_2$ (иначе можно \textit{подвинуть} наше значение $N_2$ вправо настолько, что это станет верно).
   	
    Рассмотрим $\forall n > \max(N_1, N_2)$
    \begin{multline*}
        \left|\frac{x_n}{y_n} - \frac{x_0}{y_0}\right| = \left|\frac{x_n y_0 - y_n x_0}{y_n y_0}\right| \le \frac{|x_n y_0 - x_0 y_0|}{|y_n| \cdot |y_0|} + \frac{|x_0 y_0 - y_n x_0|}{|y_n| \cdot |y_0|} =
        \\
        = \frac{|x_n - x_0|}{|y_n|} + \frac{|x_0| \cdot |y_0 - y_n|}{|y_n| \cdot |y_0|} < |x_n - x_0|\cdot \frac{2}{|y_0|} + |y_0 - y_n| \cdot \frac{2|x_0|}{|y_0|^2} <
        \\
        < \frac{\eps}{2} + \frac{\eps}{2} \cdot \frac{|x_0|}{|x_0| + 1} < \eps
    \end{multline*}
\end{enumerate}
\end{proof}