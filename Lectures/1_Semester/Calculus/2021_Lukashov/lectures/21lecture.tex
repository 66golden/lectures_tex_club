\begin{definition}
	\textit{Метрическим} пространством называется множество $X$ такое, что для любых $x, y \in X$ определено действительное число $\rho(x, y)$ (метрическое расстояние) и верны следующие утверждения:
	\begin{enumerate}
		\item $\forall x, y \in X\ \ \rho(x, y) \ge 0$, причём $\rho(x, y) = 0 \lra x = y$
		
		\item $\forall x, y \in X\ \ \rho(x, y) = \rho(y, x)$
		
		\item $\forall x, y, z \in X\ \ \rho(x, y) \le \rho(x, z) + \rho(z, y)$
	\end{enumerate}
\end{definition}

\begin{theorem}
	Каждое линейное нормированное множество (над $\R(\Cm)$) является метрическим пространством с метрикой, индуцированной нормой по формуле:
	\[
		\rho(x, y) = ||x - y||
	\]
\end{theorem}

\begin{proof}
	Доказательство сводится к проверке свойств:
	\begin{enumerate}
		\item Очевидно
		
		\item
		\[
			\rho(y, x) = ||y - x|| = ||(-1) \cdot (x - y)|| = |-1| \cdot ||x - y|| = ||x - y|| = \rho(x, y)
		\]
		
		\item 
		\[
			\rho(x, y) = ||x - y|| = ||(x - z) + (z - y)|| \le ||x - z|| + ||z - y|| = \rho(x, z) + \rho(z, y)
		\]
	\end{enumerate}
\end{proof}

\begin{corollary}
	$\R^n$ и $\Cm^n$ - метрические пространства с метрикой
	\[
	\rho(\vec{x}, \vec{y}) = ||\vec{x} - \vec{y}||
	\]
\end{corollary}

\begin{proposition}
	Любое множество является метрическим пространством
\end{proposition}

\begin{proof}
	Пусть $X$ - произвольное множество. Тогда можно определить метрику как
	\[
		\rho(x, y) = \System{
			&{1,\ x \neq y}
			\\
			&{0,\ x = y}
		}
	\]
\end{proof}

\begin{note}
	Дальнейшие определения даны для метрического пространства $X$. В качестве примера удобно брать $X = \R^2$.
\end{note}

\begin{definition}
	\textit{Открытым шаром} с центром в точке $x_0 \in X$ радиусом $\eps > 0$ (или же $\eps$-окрестностью точки $x_0$) называется множество
	\[
		U_\eps(x_0) = \{x \in X \such \rho(x, x_0) < \eps\}
	\]
\end{definition}

\begin{definition}
	Точка $x_0$ называется \textit{внутренней точкой} множества $A \subset X$, если она принадлежит $A$ вместе с некоторой своей $\eps$-окрестностью:
	\[
		U_\eps(x_0) \subset A
	\]
\end{definition}

\begin{definition}
	\textit{Внутренностью} множества $A$ называется множество всех внутренних точек множества $A$. Обозначается как
	\[
		\Int A,\ \mc{A}
	\]
	От слова interior.
\end{definition}

\begin{definition}
	Множество $A \subset X$ называется \textit{открытым}, если все его точки - внутренние, то есть $A \subset \Int A$. Естественно, $\emptyset$ - открытое множество.
\end{definition}

\begin{definition}
	Точка $x_0$ называется \textit{точкой прикосновения} множества $A \subset X$, если
	\[
		\forall \eps > 0\ U_\eps(x_0) \cap A \neq \emptyset
	\]
\end{definition}

\begin{definition}
	Множество всех точек прикосновения множества $A \subset X$ называется его \textit{замыканием} и обозначается как
	\[
		\cl A,\ \bar{A}
	\]
\end{definition}

\begin{definition}
	Множество $A \subset X$ называется \textit{замкнутым}, если оно содержит все свои точки прикосновения, то есть $A \supset \cl A$. Естественно, $\emptyset$ - замкнутое множество
\end{definition}

\begin{definition}
	\textit{Замкнутым шаром} с центром в точке $x_0 \in X$ и радиусом $\eps > 0$ называется
	\[
		\bar{B}_\eps(x_0) = \{x \in X \such \rho(x_0, x) \le \eps\}
	\] 
\end{definition}

\begin{lemma}
	Для любого $A \subset X$ верно, что
	\[
		\Int A \subset A \subset \cl A
	\]
\end{lemma}

\begin{proof}
	Первое включение очевидно, потому что любая внутренняя точка принадлежит $A$.
	
	Второе включение следует из того, что если $x_0 \in A$, то
	\[
		\forall \eps > 0\ \ U_\eps(x_0) \cap A \supset \{x_0\}
	\]
	Значит, любая точка $A$ также лежит и в замыкании $A$.
\end{proof}

\begin{corollary}
	$A$ - открытое множество $\lra \Int A = A$
	
	$A$ - замкнутое множество $\lra \cl A = A$ 
\end{corollary}

\begin{lemma} \label{includeLemma}
	$\forall A_1 \subset A_2 \subset X$ верно, что
	\begin{align*}
		&\Int A_1 \subset \Int A_2
		\\
		&\cl A_1 \subset \cl A_2
	\end{align*}
\end{lemma}

\begin{proof}~
	\begin{itemize}
		\item Вначале разберёмся с внутренностями множеств. Пусть $x_0 \in \Int A_1$. Тогда
		\[
			\exists \eps > 0 \such U_\eps(x_0) \subset A_1 \subset A_2
		\]
		Значит, $x_0 \in \Int A_2$ тоже.
		
		\item Теперь посмотрим на замыкание. Пусть $x_0 \in \cl A_1$. Тогда
		\[
			\forall \eps > 0\ \ U_\eps(x_0) \cap A_1 \neq \emptyset
		\]
		Коль скоро $A_1 \subset A_2$, то отсюда
		\[
			\forall \eps > 0\ \ U_\eps(x_0) \cap A_2 \neq \emptyset
		\]
		Следовательно, $x_0 \in \cl A_2$.
	\end{itemize}
\end{proof}

\begin{lemma}
	Открытый шар является открытым множеством.
\end{lemma}

\begin{proof}
	Рассмотрим $\forall x_1 \in U_\eps(x_0)$. Тогда
	\[
		\rho(x_0, x_1) < \eps
	\]
	Рассмотрим шар с центром в точке $x_1$ и радиусом $\eps - \rho(x_0, x_1)$. Выберем $\forall x_2 \in U_{\eps - \rho(x_0, x_1)}(x_1)$. Тогда
	\[
		\rho(x_1, x_2) < \eps - \rho(x_0, x_1)
	\]
	Отсюда следует
	\[
		\rho(x_0, x_2) \le \rho(x_0, x_1) + \rho(x_1, x_2) < \eps
	\]
	То есть $U_{\eps - \rho(x_0, x_1)}(x_1) \subset U_\eps(x_0)$. Значит $x_1$ - внутренняя точка. А раз $x_1$ было ещё и произвольной точкой открытого шара, то открытый шар является открытым множеством.
\end{proof}

\begin{lemma}
	Замкнутый шар является замкнутым множеством.
\end{lemma}

\begin{proof}
	Пусть $x$ - точка прикосновения для $\bar{B}_\eps(x_0)$. Это означает, что
	\[
		\forall \eta > 0\ U_\eta(x) \cap \bar{B}_\eps(x_0) \neq \emptyset \Ra \exists x_1 \in U_\eta(x) \cap \bar{B}_\eps(x_0)
	\]
	Оценим расстояние от $x_0$ до $x$:
	\[
		\rho(x_0, x) \le \rho(x_0, x_1) + \rho(x_1, x) < \eps + \eta
	\]
	То есть получили утверждение
	\[
		\forall \eta > 0\ \rho(x_0, x) < \eps + \eta
	\]
	Значит
	\[
		\rho(x_0, x) \le \eps
	\]
	Следовательно, $x$ - точка замкнутого шара.
\end{proof}

\begin{theorem}~
	\begin{enumerate}
		\item Внутренность любого множества $A \subset X$ открыта.
		
		\item Замыкание любого множества $A \subset X$ замкнуто.
	\end{enumerate}
\end{theorem}

\begin{proof}~
	\begin{enumerate}
		\item Положим $G := \Int A$. Выберем $\forall x_0 \in G$. Раз точка лежит в данном множестве, то
		\[
			\exists \eps > 0 \such U_\eps(x_0) \subset A
		\]
		По лемме \ref{includeLemma} из этого следует
		\[
			\Int U_\eps(x_0) \subset \Int A
		\]
		Так как открытый шар является открытым множеством, получаем вложение
		\[
			U_\eps(x_0) \subset G
		\]
		То есть $x_0$ - внутренняя точка $G$. Значит $G$ - открытое множество
	
		\item Положим $K := \cl A$. Пусть $x_0$ - точка прикосновения множества $K$. Это означает
		\[
			\forall \eps > 0\ U_{\eps / 2}(x_0) \cap K \neq 0
		\]
		Обозначим за $x_1 \in U_{\eps / 2}(x_0) \cap K$. Тогда сразу $x_1$ - точка прикосновения множества $A$. То есть
		\[
			U_{\eps / 2}(x_1) \cap A \neq \emptyset
		\]
		Теперь выберем $x_2 \in U_{\eps / 2}(x_1) \cap A$ и оценим расстояние между ней и $x_0$:
		\[
			\rho(x_0, x_2) \le \rho(x_0, x_1) + \rho(x_1, x_2) < \eps
		\]
		Следовательно
		\[
			\forall \eps > 0\ U_{\eps}(x_0) \cap A \supset \{x_2\}
		\]
		Значит $x_0$ - точка прикосновения множества $A$ $\lra x_0 \in K$.
	\end{enumerate}
\end{proof}

\begin{example}
	Порой геометрическая интерпретация данной модели обманывает, ибо здесь есть возможность шара с большим радиусом оказаться вложенным в шар с меньшим радиусом. Рассмотрим метрическое пространство $X = (-1; 1)$:
	\begin{align*}
		&{U_{1}(0) = (-1; 1)}
		\\
		&{U_{5/4}(1/2) = \left(-\frac{3}{4}; 1\right) \Ra U_{5/4}(1/2) \subsetneq U_{1}(0)}
	\end{align*}
\end{example}

\begin{lemma}
	Для любого $A \subset X$ верны равенства
	\begin{itemize}
		\item $X \bs \Int A = \cl(X \bs A)$
		
		\item $X \bs \cl A = \Int(X \bs A)$
	\end{itemize}
\end{lemma}

\begin{proof}~
	\begin{itemize}
		\item
		\[
			x \in X \bs \Int A \lra \left(\forall \eps > 0\ U_\eps(x) \cap (X \bs A) \neq \emptyset\right) \lra x \in \cl (X \bs A)
		\]
		
		\item
		\[
			x \in X \bs \cl A \lra \left(\exists \eps > 0 \such U_\eps(x) \cap A = \emptyset\right) \lra \left(\exists \eps > 0 \such U_\eps(x) \subset X \bs A\right) \lra x \in \Int (X \bs A)
		\]
	\end{itemize}
\end{proof}

\begin{corollary}
	$A \subset X$ - открытое множество тогда и только тогда, когда $X \bs A$ - замкнутое.
\end{corollary}

\begin{definition}
	Внутренняя точка дополнения множества $A \subset X$ называется \textit{внешней точкой}.
\end{definition}

\begin{definition}
	\textit{Границей} множества $A \subset X$ называется множество
	\[
		\vdelta A = \cl A \bs \Int A
	\]
	Все точки $\vdelta A$ называются \textit{граничными} точками множества $A$.
\end{definition}

\begin{lemma}
	$x_0 \in \vdelta A \lra \left(\forall \eps > 0\ \ U_\eps(x_0) \cap A \neq \emptyset,\ U_\eps(x_0) \cap (X \bs A) \neq \emptyset\right)$
\end{lemma}

\begin{proof}
	$x_0 \in \vdelta A \lra x_0 \in \cl A \bs \Int A$. Значит, $x_0 \in \cl A$:
	\[
		\forall \eps > 0\ U_\eps(x_0) \cap A \neq \emptyset
	\]
	при этом $x_0 \notin \Int A$:
	\[
		\forall \eps > 0\ U_\eps(x_0) \cap (X \bs A) \neq \emptyset
	\]
\end{proof}

\begin{theorem} (Основное свойство совокупности открытых множеств) \label{mainProp}
	Пусть $X$ - метрическое пространство. Тогда совокупность $\Tau$ открытых подмножеств $X$ обладает следующими свойствами:
	\begin{enumerate}
		\item $\emptyset \in \Tau$, $X \in \Tau$
		
		\item $\forall G_1, G_2 \in \Tau \Ra G_1 \cap G_2 \in \Tau$
		
		\item $\forall \{G_\alpha\}_{\alpha \in A} \subset \Tau \Ra \bigcup\limits_{\alpha \in A} G_\alpha \in \Tau \text{, где } A \text{ является некоторым множеством индексов}$
	\end{enumerate}
\end{theorem}

\begin{note}
	Под \textit{совокупностью} множеств подразумевается множество всех множеств, обладающих указанным свойством (в данном случае - множество всех открытых множеств).
\end{note}

\begin{proof}~
	\begin{enumerate}
		\item $\emptyset$ - замкнутое и открытое множество. Следовательно $X \bs \emptyset = X$ - открытое множество.
		
		\item Рассмотрим $\forall x \in G_1 \cap G_2$. Из выбора следует
		\begin{align*}
			&\exists \eps_1 \such U_{\eps_1}(x) \subset G_1
			\\
			&\exists \eps_2 \such U_{\eps_2}(x) \subset G_2 
		\end{align*}
		Следовательно, $U_{\min(\eps_1, \eps_2)}(x) \subset G_1 \cap G_2$.
		
		\item $x \in \bigcup\limits_{\alpha \in A} G_\alpha \Ra \left(\exists \alpha_0 \in A \such x \in G_{\alpha_0}\right)$. Значит
		\[
			\exists \eps > 0 \such U_\eps(x) \subset G_{\alpha_0} \subset \bigcup\limits_{\alpha \in A} G_{\alpha}
		\]
	\end{enumerate}
\end{proof}

\begin{definition}
	Множество $X$ называется \textit{топологическим пространством}, если в нём выделена система подмножеств $\Tau$, называемых \textit{открытыми}, которая удовлетворяет свойствам из теоремы \ref{mainProp}.
	
	Множество $\Tau$ называется \textit{топологией} множества $X$.
\end{definition}

\begin{definition}
	Пусть $X$ - топологическое пространство с топологией $\Tau$. Тогда $x_0$ называется \textit{пределом последовательности} $\{x_n\}_{n = 1}^\infty \subset X$, если
	\[
		(\forall G \in \Tau, x_0 \in G)\ \exists N \in \N \such \forall n > N\ x_n \in G
	\]
	Предел обозначается как
	\[
		\liml_{n \to \infty} x_n = x_0
	\]
\end{definition}

\begin{example}
	Если не накладывать на топологию никаких дополнительных ограничений, то предел не обязан быть даже единственным. Рассмотрим $X = \{a, b\}$ с топологией $\Tau = \{\emptyset, \{a\}, \{a, b\}\}$. Рассмотрим последовательность
	\[
		x_n = a,\ n \in \N
	\]
	Тогда понятно, что $\liml_{n \to \infty} x_n = a,\ \liml_{n \to \infty} x_n = b$.
\end{example}

\begin{anote}
	Чтобы обеспечить единственность предела, достаточно добавить свойство \textit{хаусдорфовости}:
	\[
		\forall x, y \in X\ \exists G_x, G_y \in \Tau \such (x \in G_x) \wedge (y \in G_y) \wedge (G_x \cap G_y = \emptyset)
	\]
	Такое топологическое пространство называется \textit{хаусдорфовым}.
\end{anote}