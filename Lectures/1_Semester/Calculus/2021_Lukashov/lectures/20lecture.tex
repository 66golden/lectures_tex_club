\section{Вектор-функции и топология пространства $\R^n$}

\subsection{Пространство $\R^n$}

\subsubsection*{Обозначения}

\[
	\vec{x} = (x_1, \ldots, x_n) \text{ - вектор};\ \ \Matrix{&\xi^1 \\ &\vdots \\ &\xi^n} \text{ - координатный столбец}
\]

\subsubsection*{Алгебраические структуры}

\begin{definition}
	\textit{Вещественным линейным пространством} называется множество $X$, на котором определены операции $+: X \times X \ra X$ и $\cdot: \R \times X \ra X$, удовлетворяющие аксиомам линейного пространства:
	\begin{enumerate}
		\item $\forall \vec{x}, \vec{y}\ \ \vec{x} + \vec{y} = \vec{y} + \vec{x}$
		
		\item $\forall \vec{x}, \vec{y}, \vec{z}\ \ (\vec{x} + \vec{y}) + \vec{z} = \vec{x} + (\vec{y} + \vec{z})$
		
		\item $\exists \vec{0} \such \forall \vec{x} \in X\ \ \vec{x} + \vec{0} = \vec{x}$
		
		\item $\forall \vec{x} \in X\ \exists (-\vec{x}) \in X \such \vec{x} + (-\vec{x}) = \vec{0}$
		
		\item $\forall \alpha, \beta \in \R\ \forall \vec{x} \in X\ \ \alpha(\beta \vec{x}) = (\alpha \beta) \vec{x}$
		
		\item $\forall \alpha, \beta \in \R\ \forall \vec{x} \in X\ \ (\alpha + \beta) \vec{x} = \alpha \vec{x} + \beta \vec{x}$
		
		\item $\forall \alpha \in \R\ \forall \vec{x}, \vec{y} \in X\ \ \alpha(\vec{x} + \vec{y}) = \alpha\vec{x} + \alpha\vec{y}$
		
		\item $\forall \vec{x} \in X\ \ 1 \cdot \vec{x} = \vec{x}$
	\end{enumerate}
\end{definition}

\begin{proposition}
	Ноль и обратный элемент единственны
\end{proposition}

\begin{proof}
	Единственность обратного элемента:
	\[
		(\vec{x} + (-\vec{x})_1) + (-\vec{x})_2 = (-\vec{x})_2 = (-\vec{x})_1 = (\vec{x} + (-\vec{x})_2) + (-\vec{x})_1
	\]
	Единственность нуля:
	\[
		\vec{0}_1 + \vec{0}_2 = \vec{0}_1 = \vec{0}_2 = \vec{0}_2 + \vec{0}_1
	\]
\end{proof}

\begin{proposition}
	\[
		0 \cdot \vec{x} = \vec{0}
	\]
\end{proposition}

\begin{proof}
	\[
		(0 + 0)\vec{x} = 0\vec{x} = 0\vec{x} + 0\vec{x}
	\]
	Из обеих частей равенства вычтем обратный элемент к $0\vec{x}$ и получим:
	\[
		\vec{0} = 0\vec{x}
	\]
\end{proof}

\begin{proposition}
	\[
		(-1)\vec{x} = -\vec{x}
	\]
\end{proposition}

\begin{proof}
	Рассмотрим выражение
	\[
		\vec{x} + (-1)\vec{x} = 1\vec{x} + (-1)\vec{x} = (1 - 1)\vec{x} = 0\vec{x} = \vec{0}
	\]
	То есть $(-1)\vec{x}$ является обратным к $\vec{x}$ по сложению. Отсюда по единственности
	\[
		(-1)\vec{x} = -\vec{x}
	\]
\end{proof}

\begin{definition}
	\textit{Комплексным} линейным пространством называется линейное пространство над $\Cm$. Определяется аналогично вещественному.
\end{definition}

\begin{lemma}
	$\R^n$ является вещественным линейным пространством, а $\Cm^n$ - комплексным линейным пространством.
\end{lemma}

\begin{proof}
	Следует напрямую из определения. Все операции определяются поэлементно.
\end{proof}

\begin{definition}
	Отображение линейного пространства $X_1$ над $\R(\Cm)$ на линейное пространство $X_2$ над $\R(\Cm)$ называется \textit{линейным отображением (оператором)}, если
	\begin{itemize}
		\item $\forall \vec{x}, \vec{y} \in X_1\ \ L(\vec{x} + \vec{y}) = L(\vec{x}) + L(\vec{y})$
		
		\item $\forall \alpha \in \R(\Cm)\ \forall \vec{x} \in X_1\ \ L(\alpha \vec{x}) = \alpha L(\vec{x})$
	\end{itemize}
\end{definition}

\begin{definition}
	Если существует биекция линейного пространства $X_1$ на $X_2$, являющаяся линейным оператором вместе со своим обратным, то $X_1 \cong X_2$ (изоморфны)
\end{definition}

\begin{definition}
	Говорят, что на действительном линейном пространстве $X$ задана \textit{комплексная структура}, если существует линейный оператор $\goth{j}: X \to X$ такой, что
	\[
		\goth{j}^2  = -\id_X
	\]
\end{definition}

\begin{example}
	На $\R^2$ комплексная структура задаётся оператором
	\[
		\goth{j}: (x, y) \ra (-y, x)
	\]
\end{example}

\begin{example}
	\[
		\{(x_1, x_2) \in \R^2 \such x_2 = 0\} \cong \R
	\]
\end{example}

\begin{example}
	\[
		\{(z_1, z_2) \in \Cm^2 \such z_2 = 0\} \cong \Cm
	\]
\end{example}

\begin{lemma}
	Комплексная структура на $\R^{2n}$ задаётся оператором с матрицей
	\[
		\Matrix{
		0& & -1& & \cdots& & & & 0 \\
		1& & 0& & -1& & \cdots& & \vdots \\
		\vdots& & 1& & 0& & \ddots& & \\
		& & \vdots& & \ddots& & \ddots& & -1 \\
		0& & \cdots& & & & 1& & 0
		}
	\]
\end{lemma}

\begin{definition}
	Вещественным \textit{евклидовым} пространством называется вещественное линейное пространство $X$, для любых элементов $\vec{x}, \vec{y}$ которого опредлено число $\trbr{\vec{x}, \vec{y}} \in \R$ так, чтобы
	\begin{itemize}
		\item $\forall \vec{x} \in X\ \ \trbr{\vec{x}, \vec{x}} \ge 0$, причём $\trbr{\vec{x}, \vec{x}} = 0 \lra \vec{x} = \vec{0}$
		
		\item $\forall \vec{x}, \vec{y} \in X\ \ \trbr{\vec{x}, \vec{y}} = \trbr{\vec{y}, \vec{x}}$
		
		\item $\forall \alpha, \beta \in \R\ \ \trbr{\alpha\vec{x} + \beta\vec{y}, \vec{z}} = \alpha\trbr{\vec{x}, \vec{z}} + \beta\trbr{\vec{y}, \vec{z}}$
	\end{itemize}
	То есть $\trbr{\vec{x}, \vec{y}}$ - скалярное произведение
\end{definition}

\begin{addition}
	Если в определении вещественного евклидового пространства заменить первое свойство на
	\[
	\forall \vec{y} \in X\ \ \left(\trbr{\vec{x}, \vec{y}} = 0 \lra \vec{x} = \vec{0}\right)
	\]
	то получим определение \textit{псевдоевклидового} пространства.
\end{addition}

\begin{lemma}
	$\R^n$ является вещественным евклидовым пространством со скалярным произведением, определённым как
	\[
		\trbr{\vec{x}, \vec{y}} = \suml_{i = 0}^n x_iy_i
	\]
	где $\vec{x} = (x_1, \ldots, x_n);\ \vec{y} = (y_1, \ldots, y_n)$
\end{lemma}

\begin{theorem} (Неравенство Коши-Буняковского-Шварца)
	Если $X$ - вещественное евклидовое пространство, то
	\[
		\forall \vec{x}, \vec{y} \in X\ \ \trbr{\vec{x}, \vec{y}}^2 \le \trbr{\vec{x}, \vec{x}} \cdot \trbr{\vec{y}, \vec{y}}
	\]
	причём равенство имеет место тогда и только тогда, когда $\exists \lambda \in \R \such \vec{x} = \lambda \vec{y}$.
\end{theorem}

\begin{proof}
	Рассмотрим скалярное произведение $\trbr{\vec{x} + \lambda \vec{y}, \vec{x} + \lambda \vec{y}},\ \lambda \in \R$:
	\begin{multline*}
		\trbr{\vec{x} + \lambda \vec{y}, \vec{x} + \lambda \vec{y}} = \trbr{\vec{x}, \vec{x} + \lambda\vec{y}} + \lambda\trbr{\vec{y}, \vec{x} + \lambda\vec{y}} = \trbr{\vec{x} + \lambda\vec{y}, \vec{x}} + \lambda\trbr{\vec{x} + \lambda\vec{y}, \vec{y}} =
		\\
		\trbr{\vec{x}, \vec{x}} + \lambda\trbr{\vec{y}, \vec{x}} + \lambda^2\trbr{\vec{y}, \vec{y}} + \lambda\trbr{\vec{x}, \vec{y}} = \trbr{\vec{x}, \vec{x}} + 2\lambda\trbr{\vec{x}, \vec{y}} + \lambda^2\trbr{\vec{y}, \vec{y}} \ge 0
	\end{multline*}
	Раз у квадратного трёхчлена относительно $\lambda$ коэффициент при старшей степени положителен и весь он неотрицателен, то дискриминант должен быть неположителен:
	\[
		\frac{D}{4} = \trbr{\vec{x}, \vec{y}}^2 - \trbr{\vec{x}, \vec{x}} \cdot \trbr{\vec{y}, \vec{y}} \le 0
	\]
	При этом если $D = 0$ и выражение выше обращается в равенство, то $\vec{x} + \lambda\vec{y} = \vec{0} \lra \vec{x} = (-\lambda)\vec{y}$
\end{proof}

\begin{corollary}
	\[
		\suml_{i = 1}^n x_i y_i \le \left(\suml_{i = 1}^n x_i^2\right) \cdot \suml_{i = 1}^n y_i^2,\ x_i, y_i \in \R
	\]
\end{corollary}

\begin{definition}
	Комплексным евклидовым (унитарным) пространством называется комплексное линейное пространство $X$, для любых двух элементов которого $\vec{x}, \vec{y} \in X$ определено число $\trbr{\vec{x}, \vec{y}} \in \Cm$ так, что
	\begin{itemize}
		\item $\forall \vec{x}, \vec{y} \in X\ \ \trbr{\vec{x}, \vec{x}} \ge 0$, причём $\trbr{\vec{x}, \vec{x}} = 0 \lra \vec{x} = \vec{0}$
		
		\item $\forall \vec{x}, \vec{y} \in X\ \ \trbr{\vec{x}, \vec{y}} = \overline{\trbr{\vec{y}, \vec{x}}}$
		
		\item $\forall \vec{x}, \vec{y}, \vec{z} \in X\ \forall \alpha, \beta \in \Cm\ \ \trbr{\alpha\vec{x} + \beta\vec{y}, \vec{z}} = \alpha\trbr{\vec{x}, \vec{z}} + \beta\trbr{\vec{y}, \vec{z}}$
	\end{itemize}
	$\trbr{\vec{x}, \vec{y}}$ называется \textit{эрмитовым} скалярным произведением.
\end{definition}

\begin{lemma}
	$\Cm^n$ является унитарным с $\trbr{\vec{z}, \vec{w}} = \suml_{i = 0}^n z_i \bar{w_i}$, для $\vec{z} = (z_1, \ldots, z_n)$ и $\vec{w} = (w_1, \ldots, w_n)$.
\end{lemma}

\begin{theorem} (Неравенство Коши-Буняковского-Шварца для унитарных пространств)
	Если $X$ - унитарное пространство, то для любых $\vec{z}, \vec{w} \in X$ верно неравенство
	\[
		|\trbr{\vec{z}, \vec{w}}| \le \sqrt{\trbr{\vec{z}, \vec{z}}} \cdot \sqrt{\trbr{\vec{w}, \vec{w}}}
	\]
	причём равенство имеет место тогда и только тогда, когда $\exists \lambda \in \Cm \such \vec{z} = \lambda\vec{w}$ или $\vec{w} = \vec{0}$
\end{theorem}

\begin{proof}
	Обозначим $\trbr{\vec{z}, \vec{w}} = |\trbr{\vec{z}, \vec{w}}|e^{i\varphi}$. Рассмотрим $\trbr{\vec{z} + \lambda e^{i\varphi} \vec{w}, \vec{z} + \lambda e^{i\varphi} \vec{w}},\ \lambda \in \R$:
	\begin{multline*}
		\trbr{\vec{z} + \lambda e^{i\varphi} \vec{w}, \vec{z} + \lambda e^{i\varphi} \vec{w}} = \trbr{\vec{z}, \vec{z} + \lambda e^{i\varphi} \vec{w}} + \lambda e^{i\varphi} \trbr{\vec{w}, \vec{z} + \lambda e^{i\varphi} \vec{w}} =
		\\
		\overline{\trbr{\vec{z} + \lambda e^{i\varphi} \vec{w}, \vec{z}}} + \lambda e^{i\varphi} \overline{\trbr{\vec{z} + \lambda e^{i\varphi} \vec{w}, \vec{w}}} = \overline{\trbr{\vec{z}, \vec{z}}} + \overline{\lambda e^{i \varphi}} \cdot \overline{\trbr{\vec{w}, \vec{z}}} + \lambda e^{i\varphi} \overline{\trbr{\vec{z}, \vec{w}}} + \lambda e^{i\varphi} \cdot \overline{\lambda e^{i\varphi}} \cdot \overline{\trbr{\vec{w}, \vec{w}}} =
		\\
		\trbr{\vec{z}, \vec{z}} + 2\lambda |\trbr{\vec{z}, \vec{w}}| + \lambda^2 \trbr{\vec{w}, \vec{w}} \ge 0
	\end{multline*}
	И снова получили квадратный трёхчлен относительно $\lambda$:
	\[
		\frac{D}{4} = |\trbr{\vec{z}, \vec{w}}|^2 - \trbr{\vec{w}, \vec{w}} \cdot \trbr{\vec{z}, \vec{z}} \le 0
	\]
	Отсюда
	\[
		|\trbr{\vec{z}, \vec{w}}| \le \sqrt{\trbr{\vec{w}, \vec{w}}} \cdot \sqrt{\trbr{\vec{z}, \vec{z}}}
	\]
\end{proof}

\begin{corollary}
	Для любых комплексных чисел
	\[
		\left|\suml_{i = 1}^n z_i w_i\right| \le \sqrt{\suml_{i = 1}^n |z_i|^2} \cdot \sqrt{\suml_{i = 1}^n |w_i|^2}
	\]
\end{corollary}

\begin{example}
	$R^4$ - псевдоевклидово пространство, где для любых векторов $\vec{x} = (x_0, x_1, x_2, x_3)$ и $\vec{y} = (y_0, y_1, y_2, y_3)$ скалярное произведение имеет вид
	\[
		\trbr{\vec{x}, \vec{y}} = x_0 y_0 - x_1 y_1 - x_2 y_2 - x_3 y_3
	\]
	Это пространство носит имя \textit{пространства Минковского} и играет большую роль в Специальной Теории Относительности.
\end{example}

\subsubsection*{Топологические структуры}

\begin{definition}
	Линейное пространство над $\R(\Cm)$ называется \textit{нормированным}, если $\forall \vec{x} \in X$ определено число $\|\vec{x}\| \in \R$ - \textit{норма}, так, что
	\begin{enumerate}
		\item $\forall \vec{x} \in X\ \ \|\vec{x}\| \ge 0$, причём $\|\vec{x}\| = 0 \lra \vec{x} = \vec{0}$
		
		\item $\forall \alpha \in \R(\Cm),\ \forall \vec{x} \in X\ \ \|\alpha \vec{x}\| = |\alpha| \cdot \|\vec{x}\|$
		
		\item $\forall \vec{x}, \vec{y} \in X\ \ \|\vec{x} + \vec{y}\| \le \|\vec{x}\| + \|\vec{y}\|$
	\end{enumerate}
\end{definition}

\begin{theorem} 
	Евклидово пространство над $\R(\Cm)$ является линейным нормированным пространством (ЛНП) с $\|\vec{x}\| = \sqrt{\trbr{\vec{x}, \vec{x}}}$.
\end{theorem}

\begin{proof}~
	\begin{enumerate}
		\item В обоих случаях следует из определения:
		\begin{align*}
			\vec{x} = \vec{0} \lra \trbr{\vec{x}, \vec{x}} = 0 \lra \|\vec{x}\| = 0
			\\
			\vec{x} \neq \vec{0} \lra \trbr{\vec{x}, \vec{x}} > 0 \lra \|\vec{x}\| > 0
		\end{align*}
		
		\item Докажем комплексный случай как более сложный:
		\[
			\|\alpha\vec{x}\| = \sqrt{\trbr{\alpha\vec{x}, \alpha\vec{x}}} = \sqrt{\alpha\trbr{\vec{x}, \alpha\vec{x}}} = \sqrt{\alpha\overline{\trbr{\alpha\vec{x}, \vec{x}}}} = \sqrt{\alpha \cdot \overline{\alpha}\overline{\trbr{\vec{x}, \vec{x}}}} = |\alpha|\sqrt{\trbr{\vec{x}, \vec{x}}} = |\alpha| \cdot \|\vec{x}\|
		\]
		
		\item
		\begin{multline*}
			\|\vec{x} + \vec{y}\|^2 = \trbr{\vec{x} + \vec{y}, \vec{x} + \vec{y}} = \trbr{\vec{x}, \vec{x} + \vec{y}} + \trbr{\vec{y}, \vec{x} + \vec{y}} = \overline{\trbr{\vec{x} + \vec{y}, \vec{x}}} + \overline{\trbr{\vec{x} + \vec{y}, \vec{y}}} =
			\\
			\trbr{\vec{x}, \vec{x}} + \overline{\trbr{\vec{y}, \vec{x}}} + \overline{\trbr{\vec{x}, \vec{y}}} + \trbr{\vec{y}, \vec{y}} = \trbr{\vec{x}, \vec{x}} + 2\re(\trbr{\vec{x}, \vec{y}}) + \trbr{\vec{y}, \vec{y}} \le
			\\
			\trbr{\vec{x}, \vec{x}} + 2 |\trbr{\vec{x}, \vec{y}}| + \trbr{\vec{y}, \vec{y}} \le \trbr{\vec{x}, \vec{x}} + 2 \sqrt{\trbr{\vec{x}, \vec{x}}} \sqrt{\trbr{\vec{y}, \vec{y}}} + \trbr{\vec{y}, \vec{y}} = \left(\sqrt{\trbr{\vec{x}, \vec{x}}} + \sqrt{\trbr{\vec{y}, \vec{y}}}\right)^2 =
			\\
			\left(\|\vec{x}\| + \|\vec{y}\|\right)^2
		\end{multline*}
	\end{enumerate}
\end{proof}

\begin{corollary} (Неравенство Минковского)
	\[
		\sqrt{\suml_{i = 1}^n (x_i + y_i)^2} \le \sqrt{\suml_{i = 1}^n x_i^2} + \sqrt{\suml_{i = 1}^n y_i^2},\ \ x_i, y_i \in \R 
	\]
\end{corollary}

\begin{lemma}
	$\R^n$ - ЛНП с $\|\vec{x}\| = \sqrt{\suml_{i = 1}^n x_i^2}$. При этом в $\R^2$ и $\R^3$ норма совпадает с длиной вектора.
\end{lemma}

\begin{note}
	Для удобства, в $\R^n$ будем обозначать норму просто как $|\vec{x}|$.
\end{note}

\begin{lemma}
	$\Cm^n$ - ЛНП с $\|\vec{z}\| = \sqrt{\suml_{i = 1}^n |z_i|^2}$
\end{lemma}