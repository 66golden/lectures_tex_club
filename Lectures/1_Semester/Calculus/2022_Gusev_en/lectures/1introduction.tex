\section{Introduction}

\subsection{Quantifiers}
\df{$\forall x \; P(x)$}{for all $x$ $P(x)$ is true.}
\df{$\exists x \; P(x)$}{exists $x$ such that $P(x)$ is true.}
\df{$\exists! x \; P(x)$}{exists unique $x$ such that $P(x)$ is true.
\[ \exists! x \; P(x) \defev \exists x \; P(x) \wedge (\forall y \; P(y) \Rightarrow y = x) \]%
}

\begin{equation}
    \forall x \in A \;   P(x) \defev \forall x \left( x \in A \Rightarrow P(x) \right)
\end{equation}

\begin{example}
    \[ \forall x \in \mathbb{R} \; \exists n \in \mathbb{N} \; n > x \text{ -- true} \]
    \[ \exists n \in \mathbb{N} \; \forall x \in \mathbb{R} \; n > x \text{ -- false} \]
\end{example}

\subsection{Simple derivative definition}
\begin{definition}
$f'(x)$ is a derivative of function $f(x)$ if
\[ P(\varepsilon, \delta) = (0 < \abs{h} < \delta) \Rightarrow \abs{\frac{f(x + h) - f(x)}{h} - f'(x)} < \varepsilon \]
\[ \forall \varepsilon > 0 \; \exists \delta > 0: P(\varepsilon, \delta) \]
\end{definition}

\begin{example} $ f(x) = x^2 \thus f'(x) = 2x $ \end{example}

\begin{proof}
    \[ \frac{f(x + h) - f(x)}{h} = \frac{x ^ 2 + 2h + h^2 - x^2}{h} = 2x + h \]
    Check if $f'(x) = 2x$:
    \[ \forall \varepsilon > 0 \; \exists \delta > 0: (0 < \abs{h} < \delta) \Rightarrow \abs{2x + h - 2x} < \varepsilon \]

    Suppose $\delta = \frac{\varepsilon}{2}$ then $\abs{2x + h - 2x} < \frac{\varepsilon}{2} < \varepsilon$.
\end{proof}

\subsection{Set theory}
\df{Cartesian product of sets $X$ and $Y$}{set which consists of all pairs of elements $(x, y)$ such that $x \in X$\!, $y \in Y$\!.
\[ X \times Y \defeq \{(x, y) \; \vert \; x \in X, y \in Y\} \]
}
\begin{equation}
    X^2 \defeq X \times X
\end{equation}

\begin{equation}
    Y \subseteq X \defev \forall y \; (y \in Y \Rightarrow x \in X)
\end{equation}

\begin{equation}
    2^{X} \defeq \{ Y \sconstr Y \subseteq X \}
\end{equation}

\begin{example} $ X = \{1, 2\} \thus 2^X = \{\emptyset, \{1\}, \{2\}, \{1, 2\}\} $ \end{example}

\df{Homogeneous relation over set $X$}{any set $R \subseteq X^2$. \guillemotleft $x$ and $y$ are in relation $R$\guillemotright $\;\defev (x, y) \in R \defev x \; R \; y$.}

\begin{definition}
If $X$, $Y$ -- sets and $\Gamma \subseteq X \times Y$ such that $\forall x \in X \; \exists! y \in Y : (x, y) \in \Gamma$, then $f \defeq (X, Y, \Gamma)$, $f{:}\; X \rightarrow Y$, $f{:}\; x \rightarrow f(x)$, $f{:}\; X \ni x \rightarrow f(x) \in Y$.

\[ (y = f(x)) \defev (x \in X \lor y \in Y \lor (x, y) \in \Gamma) \]
\end{definition}

If $f{:}\; X \times Y \rightarrow Z$, then $\forall x \in X, y \in Y f(x, y) \defeq f((x, y))$.


\subsection{Axiomatization of the reals}
\df{Set of rational numbers $\RR$, operations $\fnis{+} \RR^2 \rightarrow \RR$, $\fnis{\cdot} \RR^2 \rightarrow \RR$ and homogeneous relation $\le \; \subseteq \RR^2$}{a set, operations $+$ and $\cdot$ and a relation $\le$ such that the following axioms are true:
\begin{enumerate}
    \setcounter{enumi}{-1}
    \item $a + b \defev +(a, b)$; $a \cdot b \defev \cdot(a, b)$; $x < y \defev (x \le y) \land (x \ne y)$; \ldots
    \item $\exists 0 \in \RR : \forall x \in \RR \; 0 + x = x + 0 = x$ 
    \item $\forall x \in \RR \; \exists (-x) \in \RR : x + (-x) = (-x) + x = 0$ 
    \item $\forall x, y, z \in \RR \; x + (y + z) = (x + y) = z$ % ассоциативность 
    \item $\forall x, y \in \RR \; x + y = y + x$ % коммутативность 
    \item $\exists 1 \in \RR \setminus \{0\} : \forall x \in \RR \; 1 \cdot x = x \cdot 1 = 1$
    \item $\forall x \in \RR \setminus \{0\} \; \exists x^{-1} \in \RR : x \cdot x^{-1} = x^{-1} x = 1$
    \item $\forall x, y, z \in \RR \; x \cdot (y \cdot z) = (x \cdot y) \cdot z$
    \item $\forall x, y \in \RR \; x \cdot y = y \cdot x$
    \item $\forall x, y, z \in \RR \; (x + y) \cdot z = x \cdot z + y \cdot z$ % дистрибутивность
    \item $\forall x \in \RR \; x \le x$
    \item $\forall x, y \in \RR \; (x \le y) \land (y \le x) \Rightarrow (x = y)$
    \item $\forall x, y, z \in \RR \; (x \le y) \land (y \le z) \Rightarrow (x \le z)$
    \item $\forall x, y \in \RR \; (x \le y) \lor (y \le x)$
    \item $\forall x, y, z \in \RR \; (x \le y) \Rightarrow (x + z \le y + z)$
    \item $\forall x, y \in \RR \; (0 \le x) \land (0 \le y) \Rightarrow (0 \le x \cdot y)$
    \item $\forall X, Y \subseteq \RR : (\forall x \in X \; \forall y \in Y \; x \le y) \; \exists c \in \RR : (\forall x \in X \; \forall y \in Y \; x \le c \le y)$
\end{enumerate}
}

\begin{theorem}
    $\exists!0$.
\end{theorem}
\begin{proof}
    Suppose $0_1$ and $0_2$ are different neutral elements. $0_1 + 0_2 = 0_2$ by the first axiom as $0_1$ is neutral and $0_1 + 0_2 = 0_1$ as $0_2$ is neutral. Then, $0_1 + 0_2 = 0_2 = 0_1$ -- contradiction.
\end{proof}

\begin{theorem}
    $\exists! (-x): x + (-x) = 0$.
\end{theorem}
\begin{proof}
    Let $a \ne b \in \RR$ and $a + x = x + a = 0$, $b + x = x + b = 0$.
    \begin{align*}
        0 = x + a &= x + b = 0 \\
        \underline{a + x} + a &= \underline{a + x} + b \\
        0 + a &= 0 + b \\
        a &= b
    \end{align*}
\end{proof}

\begin{theorem}
    $\forall a, b \in \RR$ the equation $a + x = b$ has an unique solution $x \in \RR$, $x = (-a) + b$.
\end{theorem}
\begin{proof}
    From two previous theorems we have enough uniqueness constraints to just solve the equation via rules from the 2$^{\rm nd}$ grade.
\end{proof}

\begin{theorem} $ \forall x \in \RR \; 0 \cdot x = 0 $ \end{theorem}
\begin{proof}
    \begin{gather*}
        0 \cdot x = (0 + 0) \cdot x = 0 \cdot x + 0 \cdot x \\
        -(0 \cdot x) + 0 \cdot x = -(0 \cdot x) + 0 \cdot x + 0 \cdot x \\
        0 = 0 + 0 \cdot x \\
        0 = 0 \cdot x
    \end{gather*}
\end{proof}

\begin{theorem} $\forall x \in \RR \; (-x) = (-1) \cdot x$ \end{theorem}
\begin{proof}
    \begin{align*}
        0 &= 0 \\
        x + (-x) &= x \cdot 0 \\
        x + (-x) &= x \cdot (1 + (-1)) \\
        x + (-x) &= x + (-1) \cdot x \\
        (-x) &= (-1) \cdot x
    \end{align*}
\end{proof}

\begin{theorem} $0 < 1$ \end{theorem}
\begin{proof}
    Suppose $1 \le 0$, then (axiom 14) $1 + (-1) \le 0 + (-1)$ or $0 \le -1$. Via axiom 15 $0 \le (-1) \cdot (-1)$ or (by the previous theorem) $0 \le 1$ -- contradiction as $0 \ne 1$.
\end{proof}

For $\mathcal{F}$ -- an arbitrary family of sets
\begin{equation}
    \cap \mathcal{F} \defeq \{ x \sconstr \forall A \in \mathcal{F} : x \in A\}
\end{equation}
\begin{equation}
    \cup \mathcal{F} \defeq \{ x \sconstr \exists A \in \mathcal{F} : x \in A\}
\end{equation}

\df{Inductive set $X$}{$X \in \RR$ satisfying the following properties:
\begin{enumerate}
    \item $1 \in X$
    \item $\forall x \in X \; 1 + x \in X$
\end{enumerate}
}

\begin{equation}
\begin{gathered}
    \mathcal{F} \defeq \{ A \subseteq \RR \sconstr A \text{ is inductive} \} \\    
    \NN \defeq \cup \mathcal{F}
\end{gathered}
\end{equation}

\begin{theorem}
    \label{inductivesubset}
    Let $\mathcal{G}$ be a family of subsets of $\RR$ and $\forall A \in \mathcal{G}$ $A$ is inductive, then $\cap \mathcal{G}$ is also inductive.
\end{theorem}
\begin{proof}
    $M \defeq \cap \mathcal{G}$
    \begin{enumerate}
        \item $\forall X \in \mathcal{G} \; 1 \in X \thus 1 \in M$
        \item $x \in M \thus \forall X \in \mathcal{G} \; x \in A \thus \forall X \in \mathcal{G} \; (x + 1) \in A \thus (x + 1) \in M$
    \end{enumerate}
\end{proof}

\begin{theorem}
    $ \NN\text{ is an inductive subset of } \RR;\ \nexists A \subsetneq \RR, A \text{ is inductive}$.
\end{theorem}
\begin{proof}\phantom\\
    \begin{enumerate}
        \item $\NN$ is inductive by (\ref{inductivesubset})
        \item By set union definition we get $\forall A \in \mathcal{F} \; \NN \subseteq A$. No set can satisfy both $X \subsetneq Y$ and $Y \subseteq X$.
    \end{enumerate}
\end{proof}

\begin{theorem} (Mathematical induction)
    If $M \subseteq \NN$, $1 \in M$ and $\forall n \in M \; n + 1 \in M$, then $M = \NN$.
    \label{minduction}
\end{theorem}
\begin{proof}
    $M$ satisfies the definition of an inductive subset of $\RR \thus \NN \subseteq M$ but ${M \subseteq \NN} \thus {\NN = M}$.
\end{proof}

\begin{theorem}
    $ \forall x \in \NN \; x \ge 1 $
\end{theorem}
\begin{proof}
    Suppose $\exists x \in N : x < 1$. Pick an arbitrary set $A \in \mathcal{F}$ and construct $B \defeq \{b \sconstr b \in A, b > x\}$. $1 \in B$ because $1 \in A$ and $x < 1$, $\forall c \in B \; (1 + c) \in A$ and, as $(1 + c) > (1 + x) > x$, $(1 + c) \in B$ $\thus B \in \mathcal{F}$. $x \notin B \thus x \notin \NN$ -- contradiction.
\end{proof}

\begin{theorem}
    $ \forall m, n \in \NN \; m + n \in \NN,\ m \cdot n \in \NN$
\end{theorem}
\begin{proof}
    Construct $M_1 \defeq \{x \sconstr x \in \NN,\; (m + x) \in \NN\}$. Obviously, $M_1$ satisfies requirements of \ref{minduction} $\thus M_1 = \NN \thus n \in M_1 \thus m + n \in \NN$.

    Construct $M_2 \defeq \{x \sconstr x \in \NN,\; m \cdot x \in \NN\}$. If $x \in M_2$, then $m \cdot (x + 1) = m \cdot x + m \in \NN$ (by part 1 as $x \in M_2 \Rightarrow m \cdot x \in \NN$, $m \in \NN$). Thus, $M_2$ satisfies requirements of \ref{minduction} $\thus M_2 = \NN \thus n \in M_2 \thus m \cdot n \in \NN$.
\end{proof}