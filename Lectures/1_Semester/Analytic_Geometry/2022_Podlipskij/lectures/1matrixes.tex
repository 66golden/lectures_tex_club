\section{Матрицы}
\subsection{Базовые операции}

\df{Матрица}{упорядоченный набор из $n \times m$ чисел, записанный в виде набора из $n$ строк $m$ столбцов.}
\[ \bm{A} = \begin{pmatrix}
    a_{11} & a_{12} & \ldots & a_{1m} \\
    a_{21} & a_{22} & \ldots & a_{2m} \\
    \vdots & \vdots & \ddots & \vdots \\
    a_{n1} & a_{n2} & \ldots & a_{nm}
\end{pmatrix} \]
\df{Нулевая матрица $\bm{0}$}{матрица, состоящая из нулевых элементов.}
\df{Матрица-строка}{матрица, состоящая из одной строки}
\df{Матрица-столбец}{матрица, состоящая из одного столбца.}
\df{Квадратная матрица}{матрица, у которой количество столбцов равно количеству строк.}
\df{Главная диагональ}{элементы квадратной матрицы, у которых $i = j$.}
\df{Единичная матрица $\bm{E}$}{квадратная матрица, элементы главной диагонали которой равны 1, остальные -- 0.}
\df{Диагональная матрица}{квадратная матрица, элементы не на главной диагонали которой равны 0.}
\df{Симметрическая/симметричная матрица}{квадратная матрица, у которой $\forall i, j \; a_{ij} = a_{ji}$.}
\df{Кососимметрическая матрица}{\sloppy квадратная матрица, у которой ${\forall i, j \; a_{ij} = -a_{ji}}$.}
Диагональные элементы кососимметрической матрицы равны 0.
\df{Верхняя/нижняя треугольная матрица}{матрица, у которого элементы стоящие выше/ниже главной диагонали равны 0.}
\begin{equation} \bm{A}_{nm} = \bm{B}_{nm} \defev \forall i, j \; a_{ij} = b_{ij} \end{equation}
\begin{equation} \bm{C}_{nm} = \bm{A}_{nm} + \bm{B}_{nm} \defev \forall i, j \; c_{ij} = a_{ij} + b_{ij} \end{equation}
\begin{equation} \bm{B}_{nm} = k \cdot \bm{A}_{nm} \defev \forall i, j \; b_{ij} = k \cdot a_{ij} \end{equation}
\begin{equation} \bm{C}_{nm} = \bm{A}_{nm} + (-1) \cdot \bm{B}_{nm} \defev \forall i, j \; b_{ij} = k \cdot a_{ij} \end{equation}
\begin{equation} \bm{B}_{mn} = \bm{A}_{nm}^T \defev \forall i, j \; b_{ij} = a_{ji} \end{equation}

\begin{equation} \bm{A} = \bm{A}^T \implies \bm{A} \text{ -- симметрическая} \end{equation}
\begin{equation} \bm{A} = -\bm{A}^T \implies \bm{A} \text{ -- кососимметрическая} \end{equation}

\begin{equation} (\bm{A} + \bm{B})^T = \bm{A}^T + \bm{B}^T \end{equation}
\begin{equation} (k \bm{A})^T = k \bm{A}^T \end{equation}
\begin{equation} (\bm{A}^T)^T = \bm{A} \end{equation}

\subsection{Определитель матрицы второго и третьего порядка}
\df{Определитель (детерминант)}{функция, отображающая множество квадратных матриц на рациональные числа.
\[ \abs{\bm{A}}; \quad \det{\bm{A}}; \quad \det\!\begin{pmatrix}
    \ldots \\
    \ldots \\
\end{pmatrix} \]
\[ \det(a) = a; \quad \det\!\begin{pmatrix}
    a & b \\
    c & d
\end{pmatrix} = ad - bc \]
\[ \det\!\begin{pmatrix}
    a_{11} & a_{12} & a_{13} \\
    a_{21} & a_{22} & a_{23} \\
    a_{31} & a_{32} & a_{33} \\
\end{pmatrix} = a_{11} \cdot (-1)^{1+1} \cdot \abs{\begin{matrix}
    a_{22} & a_{23} \\
    a_{32} & a_{33}
\end{matrix}} + a_{12} \cdot (-1)^{1+2} \cdot \abs{\begin{matrix}
    a_{21} & a_{23} \\
    a_{31} & a_{33}
\end{matrix}} + a_{13} \cdot (-1)^{1+3} \cdot \abs{\begin{matrix}
    a_{21} & a_{22} \\
    a_{31} & a_{32}
\end{matrix}} = \]
\[ = a_{11} a_{22} a_{33} + a_{12} a_{23} a_{31} + a_{13} a_{21} a_{32} - a_{11} a_{23} a_{32} - a_{12} a_{21} a_{33} - a_{13} a_{22} a_{31} \]
}

\begin{equation} \det \bm{A}^T = \det \bm{A} \end{equation}
\begin{equation} \det \bm{A} = \sum_{s=1}^{n} a_{ks} \cdot (-1)^{k+s} \cdot M_{ks} \end{equation}
\begin{equation} \det(t\bm{A_{nn}}) = t^n \det\bm{A} \end{equation}
\begin{equation} \det(\bm{E}) = 1 \end{equation}
\begin{equation} \det(\diag({a_{11},\; \ldots,\; a_{nn}})) = \prod_{i=1}^{n} a_{ii} \end{equation}
\begin{equation} \det(\bm{A}) = \prod_{i=1}^{n} a_{ii}, \text{ где $\bm{A}$ -- треугольная матрица} \end{equation}
Если в матрице поменять две строки/столбца местами, то определитель поменяет знак. \\
Если к строке прибавить линейную комбинацию других строк, то определитель не изменится. \\
Если в матрице есть нулевая строка/столбец или две одинаковых строки/стоблца, то определитель -- 0.


\subsection{Системы линейных уравнений}
\begin{equation*}
    \begin{cases}
        a_{11}x_1 + a_{12}x_2 + \ldots + a_{1n} x_n = b_1 \\
        \ldots \\
        a_{m1}x_1 + a_{m2}x_2 + \ldots + a_{mn} x_n = b_m
    \end{cases}
\end{equation*}

\begin{equation}
    \begin{pmatrix}
        a_{11} & \ldots & a_{1n} \\
        \vdots & \ddots & \vdots \\
        a_{m1} & \ldots & a_{mn}
    \end{pmatrix}\text{ -- матрица системы}
\end{equation}

\begin{equation}
    \begin{pmatrix}
        x_1 \\
        \vdots \\
        x_n
    \end{pmatrix}\text{ -- cтолбец неизвестных}
\end{equation}

\begin{equation}
    \begin{pmatrix}
        b_1 \\
        \vdots \\
        b_m
    \end{pmatrix}\text{ -- cтолбец свободных членов}
\end{equation}

\begin{equation}
    \begin{pmatrix}
        a_{11} & \ldots & a_{1n} & \vrule & b_1 \\
        \vdots & \ddots & \vdots & \vrule & \vdots \\
        a_{m1} & \ldots & a_{mn} & \vrule & b_m
    \end{pmatrix}\text{ -- расширеная матрица системы}
\end{equation}

\df{Совместная система}{система, имеющая хотя бы одно решение.}
\df{Однородная система}{стоблец свободных членов равен $\bm{0}$.}
\begin{theorem} (Очевидно) Однородная система совместна. \end{theorem}

\begin{theorem} (Теорема Крамора)
    Пусть в системе линейных уравнений $\Delta = \abs{\begin{matrix}
        a_{11} & a_{12} \\
        a_{21} & a_{22}
    \end{matrix}}$, тогда система имеет единственное решение согда $\Delta \ne 0$.
\end{theorem}

\begin{proof}
    Пусть $\Delta \ne 0$, тогда определим $\Delta_1 = \abs{\begin{matrix}
        b_1 & a_{12} \\
        b_2 & a_{22}
    \end{matrix}}$, $\Delta_2 = \abs{\begin{matrix}
        a_{11} & b_1 \\
        a_{21} & b_2
    \end{matrix}}$.

    \[ \begin{cases}
        a_{11}x_1 + a_{12}x_2 = b_1 \\
        a_{21}x_1 + a_{22}x_2 = b_2
    \end{cases} \]
    \[
        \begin{cases}
            x_1 = \frac{b_1 a_{22} - b_2 a_{12}}{a_{11} a_{22} - a_{21} a_{12}} = \frac{\Delta_1}{\Delta} \\
            x_2 = \frac{b_2 a_{11} - b_1 a_{21}}{a_{11} a_{22} - a_{21} a_{12}} = \frac{\Delta_2}{\Delta} \\
        \end{cases}
    \]

    Доказательство неединственности решения при $\Delta = 0$ с текущими знаниями сводится к перебору случаев. Доказательство в общем случае будет позднее.
\end{proof}


\subsection{Умножение матриц}
\begin{equation}
    \begin{pmatrix} a_1 & \ldots & a_n \end{pmatrix}
    \begin{pmatrix} b_1 \\ \vdots \\ b_n \end{pmatrix} =
    \begin{pmatrix} a_1 \cdot b_1 + \ldots + a_n \cdots b_n \end{pmatrix}
\end{equation}

Для матриц $\bm{A}_{n \times m}$ и $\bm{B}_{m \times k}$:
\begin{equation}
    \bm{C} = \bm{A} \cdot \bm{B} \defev \bm{C}_{ij} = \sum_{s=1}^{m} \bm{A}_{is} \cdot \bm{B}_{sj}
\end{equation}

\df{Перестановочные матрицы $\bm{A}$ и $\bm{B}$}{
    матрицы, для которых выполняется
    \begin{equation}
        \bm{A} \cdot \bm{B} = \bm{B} \cdot \bm{A}
    \end{equation}
}

\begin{gather}
    \bm{AE} = \bm{EA} = \bm{A} \\
    (\bm{A} + \bm{B}) \cdot \bm{C} = \bm{AC} + \bm{BC} \\
    (\bm{AB})^T = \bm{B}^T \bm{A}^T
\end{gather}

\begin{theorem}
    \[ (\bm{A} \bm{B}) \bm{C} = \bm{A} (\bm{B} \bm{C}) \]
\end{theorem}

\begin{proof}
    $\exists (\bm{A}_{n \times m} \bm{B}_{m \times k}) \bm{C}_{k \times p} \thus \exists (\bm{B} \bm{C})_{m \times p} \thus \exists (\bm{A} (\bm{B} \bm{C}))_{n \times p}$.

    \[ (\bm{AB})_{ij} = \sum_{x=1}^{m} \bm{A}_{ix} \bm{B}_{xj} \]
    \[
        ((\bm{AB})\bm{C})_{ij} =
        \sum_{y=1}^{k} (\bm{AB})_{iy} \bm{C}_{yj} =
        \sum_{y = 1}^k (\sum_{x=1}^m (\bm{A}_{ix} \bm{B}_{xy}) \bm{C}_{yj}) =
        \sum_{y = 1}^k \sum_{x=1}^m (\bm{A}_{ix} \bm{B}_{xy} \bm{C}_{yj}) =
    \]
    \[ 
        = \sum_{x = 1}^m \sum_{y=1}^k (\bm{A}_{ix} \bm{B}_{xy} \bm{C}_{yj}) =
        \sum_{x = 1}^m \bm{A}_{ix} (\bm{BC})_{xj} =
        (\bm{A}(\bm{BC}))_{ij}
    \]
\end{proof}
