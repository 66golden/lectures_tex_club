\section{Векторы}
\subsection{Свойства векторов}
\begin{enumerate}
    \item $\forall \vv{a}, \vv{b}\ \vv{a} + \vv{b} = \vv{b} + \vv{a}$
    \item $\forall \vv{a}, \vv{b}, \vv{c}\ (\vv{a} + \vv{b}) + \vv{c} = \vv{a} + (\vv{b} + \vv{c})$
    \item $\exists \vv{0} : \forall \vv{a}\ \vv{a} + \vv{0} = \vv{a}$
    \item $\forall \vv{a}\ \exists\ {-\vv{a}} : -\vv{a} + \vv{a} = \vv{0}$
    \item $\forall \vv{a}, \alpha, \beta\ (\alpha \beta) \vv{a} = \alpha (\beta \vv{a})$
    \item $\forall \vv{a}\ 1 \cdot \vv{a} = \vv{a}$
    \item $\forall \vv{a}, \alpha, \beta\ (\alpha + \beta) \vv{a} = \alpha \vv{a} + \beta \vv{a}$
    \item $\forall \vv{a}, \vv{b}, \alpha\ \alpha \vv{a} + \alpha \vv{b} = \alpha (\vv{a} + \vv{b})$
\end{enumerate}

\subsection{Линейные комбинации векторов}
\df{Множество, замкнутое относительно операции}{множество и операция такие, что результат операции, применённой к элементам множества, также принадлежит множеству.}
\df{Векторное пространство}{непустое множество векторов, замкнутое относительно линейных операций.}
\df{Векторное подпространство}{векторное пространство, являющееся подмножеством другого векторого пространства.}

\df{Линейная комбинация векторов}{для упорядоченного множества коэффициентов $\alpha_i$ и набора векторов $\vv{a}_i$ 
    \begin{equation*}
        \alpha_1 \vv{a}_1 + \alpha_2 \vv{a}_2 + \ldots + \alpha_n \vv{a}_n
    \end{equation*}
}

\df{Тривиальная линейная комбинация}{линейная комбинация, все коэффициенты которой равны 0.}
\df{Нетривиальная линейная комбинация}{линейная комбинация, не являющаяся тривиальной.}

\df{Разложение вектора $\vv{b}$ по векторам $\vv{a}_1 \ldots \vv{a}_n$}{линейная комбинация векторов $\vv{a}_1 \ldots \vv{a}_n$, равная $\vv{b}$.}

\df{Линейно зависимый набор векторов}{набор векторов, в котором нулевой вектор раскладывается нетривиальной линейной комбинацией векторов из набора.}
\df{Линейно независимый набор векторов}{набор векторов, в котором нулевой вектор раскладывается линейной комбинацией векторов из набора единственным способом.}

\begin{theorem}
    (Очевидно) Если в набор входит нулевой вектор, то он линейно зависим.
\end{theorem}

\begin{theorem}
    (Очевидно) Если у набора существует линейно зависимое подмножество, то он также линейно зависим.
\end{theorem}

\begin{theorem}
    (Очевидно) Любой непустой поднабор линейно независимых векторов будет линейно независимым.
\end{theorem}

\begin{theorem}
    Если $\vv{x}$ раскладывается по векторам $\vv{a}_1 \ldots \vv{a}_k$, то это разложение единственно согда набор $\vv{a}_1 \ldots \vv{a}_k$ -- линейно независим.
    \label{UniqueBasis}
\end{theorem}
\begin{proof}\iffproofL{
    Пусть разложение $\vv{x}$ не единственно:
    \[ \vv{x} = \alpha_1 \vv{a}_1 + \ldots + \alpha_n \vv{a}_n \]
    \[ \vv{x} = \beta_1 \vv{a}_1 + \ldots + \beta_n \vv{a}_n \]
    \[ \vv{0} = (\alpha_1 - \beta_1) \vv{a}_1 + \ldots + (\alpha_n - \beta_n) \vv{a}_n \]
    Если набор линейно независим, то $\alpha_i - \beta_i = 0 \thus \alpha_i = \beta_i \thus$ разложения совпадают.
}{
    От противного. Если набор линейно зависим, то \sloppy $\exists m_1, \ldots, m_n : \prod m_i \ne 0, m_1 \vv{a}_1 + \ldots \hm{+} m_n \vv{a}_n = \vv{0}$.
    \[ \vv{x} = \alpha_1 \vv{a}_1 + \ldots + \alpha_n \vv{a}_n \]
    \[ \vv{x} + \vv{0} = \vv{x} = (\alpha_1 + m_1) \vv{a}_1 + \ldots + (\alpha_n + m_n) \vv{a}_n \]
}\end{proof}

\begin{theorem}
    (Критерий линейной зависимости) Набор линейно зависим согда существует вектор равный линейной комбинации других векторов из набора.
\end{theorem}

\begin{proof}\iffproofR{
     $ \exists m_1, \ldots, m_n : \prod m_i \ne 0, m_1 \vv{a}_1 + \ldots + m_n \vv{a}_n = \vv{0} $.
     Не теряя общности, будем считать, что $m_1 \ne 0 \thus \vv{a}_1 = \frac{m_2}{m_1} \vv{a}_2 + \ldots + \frac{m_n}{m_1} \vv{a}_n $.
}{
     Пусть $\vv{a}_1 = \beta_2 \vv{a}_2 + \ldots + \beta_n \vv{a}_n \thus \vv{0} = (-1) \cdot + \vv{a}_1 \beta_2 \vv{a}_2 + \ldots + \beta_n \vv{a}_n$
}\end{proof}

\begin{proposition}
    Система из одного вектора линейно зависима согда этот вектор равен $\vv{0}$.
\end{proposition}

\begin{proposition}
    Система из двух векторов линейно зависима согда эти вектора коллинеарны.
\end{proposition}

\begin{theorem}
    Система из трёх векторов линейно зависима согда эти вектора компланарны.
\end{theorem}
\begin{proof}\iffproofL{
    Пусть $\vv{a}$, $\vv{b}$, $\vv{c}$ компланарны.
    \begin{enumerate}
        \item $\vv{a}$ и $\vv{b}$ -- коллинеарны $\thus$ $(\vv{a}, \vv{b})$ -- линейно зависимы $\thus$ система линейно зависима.
        \item $\vv{a}$ и $\vv{b}$ -- не линейно зависимы, $\vv{c}$ раскладывается на плоскости. 
    \end{enumerate}
}{
    Пусть $\vv{a} = \alpha \vv{b} + \beta \vv{c}$. Существует плоскость, параллельная в $b$ и $c$, в которой лежит $a$ $\thus$ они коллинеарны.
}
\end{proof}

\begin{theorem}
    Система из четырёх векторов трёхмерного пространства всегда линейно зависима.
\end{theorem}
\begin{proof}
    Пусть $\vv{a}$, $\vv{b}$, $\vv{c}$ не линейно зависимы (иначе утверждение очевидно), иначе проводим плоскость $\parallel$ $\vv{a}$ и $\vv{b}$, есть точка пересечения, вектор от точки пересечения раскладывается в плоскости, и есть ещё кусок по вектору $\vv{c}$ $\thus$ нашли линейную комбинацию.
\end{proof}


\subsection{Базис}
\df{Базис в векторном пространстве}{упорядоченная, линейно независимая система векторов такая, что любой вектор векторного пространства может быть по ней разложен.}

\begin{proposition}
    В 0-мерном пространстве базиса нет.
\end{proposition}

\begin{proposition}
    В 1-мерном пространстве базисом является любой вектор.
\end{proposition}

\begin{proposition}
    На плоскости базисом является любая упорядоченная пара неколлинеарных векторов.
\end{proposition}

\begin{proposition}
    В пространстве базисом является любая упорядоченная тройка некомпланарных векторов.
\end{proposition}

\df{Координаты вектора в базисе}{коэффициенты разложения вектора в базисе.}

\begin{theorem}
    Вектора $\vv{a}$, $\vv{b}$, $\vv{c}$ с координатами в некотором базисе, соответственно, $(\vv{a_x}, \vv{a_y}, \vv{a_z})$, $(\vv{b_x}, \vv{b_y}, \vv{b_z})$ и $(\vv{c_x}, \vv{c_y}, \vv{c_z})$ линейно зависим согда
    \[ 
        \abs{\begin{matrix}
            a_x & a_y & a_z \\
            b_x & b_y & b_z \\
            c_x & c_y & c_z
        \end{matrix}} = 0
    \]
\end{theorem}
\begin{proof}
    \begin{gather*}
        k_1 \vv{a} + k_2 \vv{b} + k_3 \vv{c} = \vv{0} \\
        k_1 \vc{a_x}{a_y}{a_z} + k_2 \vc{b_x}{b_y}{b_z} + k_3 \vc{c_x}{c_y}{c_z} = \vv{0}
    \end{gather*}
    Последнее уравнение представляет собой систему из трёх линейных уравнений. Она всегда имеет решение $k_1 = 0$, $k_2 = 0$, $k_3 = 0$. Соответственно, по теореме \ref{CramerTheorem}, это решение будет единственным согда $\abs{\begin{matrix}
        a_x & a_y & a_z \\
        b_x & b_y & b_z \\
        c_x & c_y & c_z
    \end{matrix}} \ne 0$
\end{proof}

\df{Декартова система координат}{совокупность точки и базиса. Точка -- начало координат.}
\df{Ортогональный базис}{базис, углы между векторами которого равны $\frac{\pi}{2}$.}
\df{Ортонормированный базис}{ортогональный базис, длины базисных векторов которого равны 1.}
\df{Прямоугольная система координат}{декартова система координат с ортонормированным базисом.}
\df{Координата точки в системе координат}{координата вектора с началом в начале координат и концом в данной точке.}

Координаты вектора $\vv{a}$ в прямоугольной системе координат с базисом $\vv{e_1}, \vv{e_2}, \vv{e_3}$ равны $(\abs{\vv{d}} \cdot \cos{\varphi_1}, \abs{\vv{d}} \cdot \cos{\varphi_2}, \abs{\vv{d}} \cdot \cos{\varphi_3})$, где $\varphi_1 = \angle(\vv{a}, \vv{e_1}), \varphi_2 = \angle(\vv{a}, \vv{e_2}), \varphi_3 = \angle(\vv{a}, \vv{e_3})$. Косинусы этих углов называются направляющими косинусами.


\subsection{Замена базиса}
Пусть есть два базиса $\vv{e_1}, \vv{e_2}, \vv{e_3}$ -- старый базис, $\vv{e'_1}, \vv{e'_2}, \vv{e'_3}$ -- новый базис и вектор с координатами $x, y, z$ в старом базисе и $x', y', z'$ -- в новом.

\begin{gather*}
    \vv{e'_1} = a_{11} \vv{e_1} + a_{21} \vv{e_2} + a_{31} \vv{e_3} \\    
    \vv{e'_2} = a_{12} \vv{e_1} + a_{22} \vv{e_2} + a_{32} \vv{e_3} \\    
    \vv{e'_3} = a_{13} \vv{e_1} + a_{23} \vv{e_2} + a_{33} \vv{e_3}    
\end{gather*}

\begin{multline*}
    \vv{d} = x' \vv{e'_1} + y' \vv{e'_2} + z' \vv{e'_3} =
        (a_{11} x' + a_{12} y' + a_{13} z') \vv{e_1} + 
        (a_{21} x' + a_{22} y' + a_{33} z') \vv{e_2} + 
        (a_{31} x' + a_{22} y' + a_{33} z') \vv{e_3} \\
\end{multline*}

\begin{gather*}
    x = a_{11} x' + a_{12} y' + a_{13} z' \\
    y = a_{21} x' + a_{22} y' + a_{33} z' \\
    z = a_{31} x' + a_{22} y' + a_{33} z'
\end{gather*}

Детерминант системы не равен 0, так как новый базис линейно независим, значит решение системы существует и единственно.

\df{Матрица перехода от базиса $e$ к $e'$}{матрица, столбцами которой являются координаты старых базисных векторов в новом базисе.}

К базисам добавили $O$ и $O'$, получили системы координат. Пусть дополнительно известны координаты вектора $\vv{OO'}$, $(b_1, b_2, b_3)$, тогда

\begin{equation}
    \vc{x}{y}{z} = S \cdot \vc{x'}{y'}{z'} + \vc{b_1}{b_2}{b_3}
\end{equation}

\subsection{Матрица поворота на плоскости}
Два случая при ориентированных одинакого и по-разному систем координат.
Детерминант равный 1


\subsection{Скалярное произведение}
\df{Скалярное произведение векторов $\vv{a}$ и $\vv{b}$}{скаляр, равный 0, если один из векторов равен 0, иначе -- произведению длин векторов, умноженный на угол между векторами.}

\begin{enumerate}
    \item $(\vv{a}; \vv{b}) = 0 \Leftrightarrow \left[
        \begin{gathered}
            \vv{a} = \vv{0}\hfill\\
            \vv{b} = \vv{0}\hfill\\
            \cos\angle(\vv{a}; \vv{b}) = \frac{\pi}{2}\hfill
        \end{gathered}
    \right.$
    \item $(\vv{a}; \vv{b}) = (\vv{b}; \vv{a})$
    \item $(\vv{a}; \vv{a}) \ge 0$
    \item $(\vv{a}; \vv{a}) = 0 \Leftrightarrow \vv{a} = \vv{0}$
    \item $\abs{\vv{a}} = \sqrt{(\vv{a}; \vv{a})}$
    \item $\forall \vv{a} \ne 0, \vv{b} \ne 0\ \cos\angle(\vv{a}; \vv{b}) = \dfrac{(\vv{a}; \vv{b})}{\abs{\vv{a}}\abs{\vv{b}}}$
    \item $\forall (\vv{a}; \vv{x}) = 0 \thus \vv{a} = \vv{0}$
    \item $\forall \vv{e_1}, \vv{e_2}, \vv{e_3} \text{ -- ортонормированный базис}\; (\vv{e_i}; \vv{e_j}) = \begin{cases}
        0\text{, $i \ne j$}\\
        1\text{, $i = j$}
    \end{cases}$
\end{enumerate}

\begin{theorem}
    $(\alpha \vv{a} + \beta \vv{b}; \vv{c}) = \alpha(\vv{a}; \vv{c}) + \beta(\vv{b}; \vv{c})$
\end{theorem}
\begin{proof}
    Если $\vv{c} = \vv{0}$, то обе части обращаются в 0. Пусть $\vv{c} \ne \vv{0}$, тогда
    \[ \frac{(\alpha \vv{a} + \beta \vv{b}; \vv{c})}{\abs{\vv{c}}^2} = \frac{\alpha(\vv{a}; \vv{c}) + \beta(\vv{b}; \vv{c})}{\abs{\vv{c}}^2} \]
    \[ \frac{(\alpha \vv{a} + \beta \vv{b}; \vv{c})}{\abs{\vv{c}}^2} = \frac{\alpha(\vv{a}; \vv{c})}{\abs{\vv{c}}^2} + \frac{\beta(\vv{b}; \vv{c})}{\abs{\vv{c}}^2} \]
    Значит в базисе $\vv{c}$, $\vv{e_2}$, $\vv{e_3}$:
    \[ (\alpha \vv{a} + \beta \vv{b})_x = \alpha\vv{a}_x + \beta\vv{b}_x \]
    Но последнее равенство справдливо в силу линейности координат.
\end{proof}

Для векторов, разложенных в базисе $\vv{e_1}$, $\vv{e}_2$:
\[
    (\vv{a}; \vv{b}) =
    (\vv{a}_x \vv{e_1} + \vv{a}_y \vv{e_2}; \vv{b}_x \vv{e_1} + \vv{b}_y \vv{e_2}) =
    \vv{a}_x \vv{b}_x (\vv{e_1}; \vv{e_1}) + 
    (\vv{a}_x \vv{b}_y + \vv{a}_y \vv{b}_x) (\vv{e_1}; \vv{e_2}) + 
    \vv{a}_y \vv{b}_y (\vv{e_2}; \vv{e_2})
\]
Или в ортонормированном базисе:
\[ (\vv{a}; \vv{b}) = \vv{a}_x \vv{b}_x + \vv{a}_y \vv{b}_y \]

Аналогично формулы выводятся для пространства.

\df{Взаимный/биортогональный базис для базиса $\vv{e}_1$, $\vv{e}_2$, $\vv{e}_3$}{набор векторов $\vv{e^*}_1$, $\vv{e^*}_2$, $\vv{e^*}_3$ такой, что
\[ (\vv{e}_i; \vv{e^*}_j) = \begin{cases}
    0, i \ne j \\
    1, i = j
\end{cases} \]}

\begin{proposition}
    $\vv{e^*}_i$ определены однозначно.
\end{proposition}
\begin{proof}
    Не теряя общности, будем доказывать для $\vv{e^*}_1$. Из равенства скалярного произведения нулю, $\vv{e^*}_1$ перпендикулярен плоскости, содержащей $\vv{e^*}_2$ и $\vv{e^*}_3$ $\thus$ направление вектора задаётся однозначно. Длина и ориентация относительно плоскости задаётся скалярным произведением с $\vv{e}_1$.
\end{proof}

\begin{proposition}
    $\vv{e^*}_1$, $\vv{e^*}_2$, $\vv{e^*}_3$ образуют базис.
\end{proposition}
\begin{proof}
    Возьмём произвольный вектор $\vv{x}$, линейно раскладываемый в $\vv{e^*}_1$, $\vv{e^*}_2$, $\vv{e^*}_3$:
    \[ \vv{x} = \alpha_1 \vv{e^*}_1 + \alpha_2 \vv{e^*}_2 + \alpha_3 \vv{e^*}_3 \]
    Скалярно домножим на вектора старого базиса:
    \[ \begin{cases}
        (\vv{x}; \vv{e}_1) = \alpha_1 + 0 + 0
        (\vv{x}; \vv{e}_2) = 0 + \alpha_2 + 0
        (\vv{x}; \vv{e}_3) = 0 + 0 + \alpha_3
    \end{cases} \]
    Значит разложение $\vv{x}$ -- единственно, $\thus$ по теореме \ref{UniqueBasis} $\vv{e^*}_1$, $\vv{e^*}_2$, $\vv{e^*}_3$ -- базис.
\end{proof}

% \df{Проекция $\vv{a}$ на $\vv{b}$ $\vv{\text{Пр}}_\vv{b} \vv{a}$}{определение проекции}


\df{Векторное произведение $\vv{a}$ и $\vv{b}$ $[\vv{a}; \vv{b}]$}{вектор, равный $\vv{0}$, если один или оба из векторов нулевые или вектора коллинеарны, иначе его длина равна $\abs{\vv{a}} \cdot \abs{\vv{b}} \cdot \sin \angle (\vv{a}; \vv{b})$, он перпендикулярен плоскости, содержащей $\vv{a}$ и $\vv{b}$, и тройка векторов $\vv{a}$, $\vv{b}$, $[\vv{a}; \vv{b}]$ -- правая.}

\begin{enumerate}
    \item $[\vv{a}; \vv{b}] = \vv{0} \Leftrightarrow \vv{a}, \vv{b} \text{ -- коллинеарны } (\text{включая } \vv{0})$
    \item $\abs{[\vv{a}; \vv{b}]} = S_{\text{параллел. }(\vv{a}, \vv{b})}$
    \item $[\vv{a}; \vv{b}] = -[\vv{b}; \vv{a}]$
\end{enumerate}

\[
    [\vv{a}; \vv{b}] =
    [\vv{a}_x \vv{e_1} + \vv{a}_y \vv{e_2} + \vv{a}_z \vv{e_3}; \vv{b}_x \vv{e_1} + \vv{b}_y \vv{e_2} + \vv{b}_z \vv{e_3}] =
    \abs{\begin{matrix}
        [\vv{e}_2; \vv{e}_3] & [\vv{e}_3; \vv{e}_1] & [\vv{e}_1; \vv{e}_2] \\
        \vv{a}_x & \vv{a}_y & \vv{a}_z \\
        \vv{b}_x & \vv{b}_y & \vv{b}_z
    \end{matrix}}
\]

\begin{equation}
    (\vv{a}; \vv{b}; \vv{c}) \defeq ([\vv{a}; \vv{b}]; \vv{c})
    \text{ -- смешанное произведение}
\end{equation}

\begin{theorem}
    Смешанное произведение -- объём ориентированного параллелепипеда, построенного на этих векторах.
\end{theorem}
\begin{proof}
    $[\vv{a}; \vv{b}]$ -- площадь основания параллелепипеда, потом 
\end{proof}