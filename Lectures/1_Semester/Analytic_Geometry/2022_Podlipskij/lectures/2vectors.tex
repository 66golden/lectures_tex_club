\section{Векторы}
\subsection{Свойства векторов}
\begin{enumerate}
    \item $\forall \vv{a}, \vv{b}\ \vv{a} + \vv{b} = \vv{b} + \vv{a}$
    \item $\forall \vv{a}, \vv{b}, \vv{c}\ (\vv{a} + \vv{b}) + \vv{c} = \vv{a} + (\vv{b} + \vv{c})$
    \item $\exists \vv{0} : \forall \vv{a}\ \vv{a} + \vv{0} = \vv{a}$
    \item $\forall \vv{a}\ \exists\ {-\vv{a}} : -\vv{a} + \vv{a} = \vv{0}$
    \item $\forall \vv{a}, \alpha, \beta\ (\alpha \beta) \vv{a} = \alpha (\beta \vv{a})$
    \item $\forall \vv{a}\ 1 \cdot \vv{a} = \vv{a}$
    \item $\forall \vv{a}, \alpha, \beta\ (\alpha + \beta) \vv{a} = \alpha \vv{a} + \beta \vv{a}$
    \item $\forall \vv{a}, \vv{b}, \alpha\ \alpha \vv{a} + \alpha \vv{b} = \alpha (\vv{a} + \vv{b})$
\end{enumerate}

\df{Множество, замкнутое относительно операции}{множество и операция такие, что результат операции, применённой к элементам множества, также принадлежит множеству.}
\df{Векторное пространство}{непустое множество векторов, замкнутое относительно линейных операций.}
\df{Векторное подпространство}{векторное пространство, являющееся подмножеством другого векторого пространства.}

\df{Линейная комбинация векторов}{для упорядоченного множества коэффициентов $\alpha_i$ и набора векторов $\vv{a}_i$ 
    \begin{equation}
        \alpha_1 \vv{a}_1 + \alpha_2 \vv{a}_2 + \ldots + \alpha_n \vv{a}_n
    \end{equation}
}

\df{Тривиальная линейная комбинация}{линейная комбинация, все коэффициенты которой равны 0.}
\df{Нетривиальная линейная комбинация}{линейная комбинация, не являющаяся тривиальной.}

\df{Разложение вектора $\vv{b}$ по векторам $\vv{a}_1 \ldots \vv{a}_n$}{линейная комбинация векторов $\vv{a}_1 \ldots \vv{a}_n$, равная $\vv{b}$.}

\df{Линейно зависимый набор векторов}{набор векторов, в котором нулевой вектор раскладывается нетривиальной линейной комбинацией векторов из набора.}
\df{Линейно независимый набор векторов}{набор векторов, в котором нулевой вектор раскладывается линейной комбинацией векторов из набора единственным способом.}

\begin{theorem}
    (Очевидно) Если в набор входит нулевой вектор, то он линейно зависим.
\end{theorem}

\begin{theorem}
    (Очевидно) Если у набора существует линейно зависимое подмножество, то он также линейно зависим.
\end{theorem}

\begin{theorem}
    (Очевидно) Любой непустой поднабор линейно независимых векторов будет линейно независимым.
\end{theorem}

\begin{theorem}
    Если $\vv{x}$ раскладывается по векторам $\vv{a}_1 \ldots \vv{a}_k$, то это разложение единственно согда $\vv{a}_1 \ldots \vv{a}_k$ -- линейно независим.
\end{theorem}
\begin{proof}\iffproofL{
    Пусть разложение $\vv{x}$ не единственно:
    \[ \vv{x} = \alpha_1 \vv{a}_1 + \ldots + \alpha_n \vv{a}_n \]
    \[ \vv{x} = \beta_1 \vv{a}_1 + \ldots + \beta_n \vv{a}_n \]
    \[ \vv{0} = (\alpha_1 - \beta_1) \vv{a}_1 + \ldots + (\alpha_n - \beta_n) \vv{a}_n \]
    Если набор линейно независим, то $\alpha_i - \beta_i = 0 \thus \alpha_i = \beta_i \thus$ разложения совпадают.
}{
    От противного. Если набор линейно зависим, то \sloppy $\exists m_1, \ldots, m_n : \prod m_i \ne 0, m_1 \vv{a}_1 + \ldots \hm{+} m_n \vv{a}_n = \vv{0}$.
    \[ \vv{x} = \alpha_1 \vv{a}_1 + \ldots + \alpha_n \vv{a}_n \]
    \[ \vv{x} + \vv{0} = \vv{x} = (\alpha_1 + m_1) \vv{a}_1 + \ldots + (\alpha_n + m_n) \vv{a}_n \]
}\end{proof}

\begin{theorem}
    (Критерий линейной зависимости) Набор линейно зависим согда существует вектор равный линейной комбинации других векторов из набора.
\end{theorem}

\begin{proof}\iffproofR{
     $ \exists m_1, \ldots, m_n : \prod m_i \ne 0, m_1 \vv{a}_1 + \ldots + m_n \vv{a}_n = \vv{0} $.
     Не теряя общности, будем считать, что $m_1 \ne 0 \thus \vv{a}_1 = \frac{m_2}{m_1} \vv{a}_2 + \ldots + \frac{m_n}{m_1} \vv{a}_n $.
}{
     Пусть $\vv{a}_1 = \beta_2 \vv{a}_2 + \ldots + \beta_n \vv{a}_n \thus \vv{0} = (-1) \cdot + \vv{a}_1 \beta_2 \vv{a}_2 + \ldots + \beta_n \vv{a}_n$
}\end{proof}

\begin{theorem}
    (Очевидно) Система из одного вектора линейно зависима согда этот вектор равен $\vv{0}$.
\end{theorem}

\begin{theorem}
    (Не интересная) Система из двух векторов линейно зависима согда эти вектора коллинеарны.
\end{theorem}

\begin{theorem}
    Система из трёх векторов линейно зависима согда эти вектора компланарны.
\end{theorem}

\begin{theorem}
    Система из четырёх векторов трёхмерного пространства всегда линейно зависима.
\end{theorem}