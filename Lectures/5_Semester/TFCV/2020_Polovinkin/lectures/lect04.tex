\begin{flushright}
    \textit{Лекция 4 (от 15.09)}
\end{flushright}

\textbf{Кривая} $\gamma$~--- класс эквивалентных параметризаций $z(t) = x(t) +
iy(t), \ t \in \left[ t_0; t_1 \right]$
\\
\textbf{Параметризации эквивалентны}, если существует монотонная возрастающая
функция $\psi(\tau) = t$, сохраняющая направление.
\\
\textbf{Гладкая кривая (ГК)}~--- такая кривая $\gamma$, что существует параметризация
$z(t) = x(t) + iy(t)$, $x \in C^1 \left( \left[ t_0; t_1 \right] \right)$, $y
\in C^1 \left( \left[ t_0; t_1 \right] \right)$, $\forall t \in \left[ t_0; t_1
\right] \ z'(t) \neq 0$.
\\
\textbf{Замкнутая ГК (ЗГК)}~--- такая ГК, что $z(t_0) = z(t_1)$, $z'(t_0+0) =
z'(t_1-0)$.
\\
\textbf{Кусочно гладкая кривая (КГК)}~--- такая непрерывная кривая $\gamma$, что
$\exists t_0 = \theta_0 < \theta_1 < \dots < \theta_{n-1} < \theta_n = t_1$, что
$\forall k \ \gamma_k: \ z(t), \ t \in [\theta_{k-1}, \theta_k]$ есть ГК.
\\
\textbf{Разбиение} отрезка $\left[ t_0, t_1 \right]$: $\lambda = \{t_0 = \tau_0
< \tau_1 < \dots < \tau_{m_\lambda} = t_1\}$.
\\
\textbf{Мелкость (диаметр)} разбиения $\lambda$ $\left| \lambda \right| = \dst
\max_{k \{1, \dots, m_\lambda\}}\left| \tau_k - \tau_{k-1} \right|$.
\Def Пусть $f: \gamma \mapsto \CC$ непрерывна на КГК $\gamma$. Пусть
$\lambda$~--- разбиение $\left[ t_0, t_1 \right]$ для $\gamma: \ z(t)$~---
кусочно гладкой. Пусть $z(t_k) = z_k, \ k \in \{0, \dots, m_\lambda\}$, $\zeta_k
\in \gamma_k = \gamma_{z_{k-1}z_k}$
\begin{align*}
  & \sigma(\lambda) = \sum_{k = 1}^{m_{\lambda}} f(\zeta_k) \Delta z_k
\end{align*}
$\Delta z_k = z_k - z_{k-1}$~--- \textbf{интегральная сумма};
\\
если $\exists \lim \sigma(\lambda) \ \forall \zeta_k, \ \left| \lambda \right|
\to 0$, то этот предел называется \textbf{интегралом Римана от $f$ по $\gamma$}
и записывается как
\begin{equation} \label{(6.1)}
    \int_{\gamma}f(z) dz
\end{equation}
\theorem
Если условия определения интеграла выполняются, то он существует и справедливо:
\begin{equation} \label{(6.2)}
    \int_{\gamma}f(z)dz = \int_{\gamma}\left( u dx - v dy \right) + i \int_\gamma \left( v dx + u dy \right)
\end{equation}
\pr
\begin{align*}
  & \sigma(\lambda) = \sum_{k = 1}^{m_\lambda}\left( u(\xi_k, \eta_k) + iv(\xi_k, \eta_k) \right)\left( \Delta x_k + i \Delta y_k \right) = \sum_{k = 1}^{m_\lambda}\left( u(\xi_k, \eta_k) \Delta x_k - v(\xi_k, \eta_k) \Delta y_k\right) + \\
  & + \left( v(\xi_k, \eta_k) \Delta x_k + u(x_k, y_k) \Delta y_k\right) = \sigma_1(\lambda) + \sigma_2(\lambda) \to \int_{\gamma}\left( u dx - v dy \right) + i \int_\gamma \left( v dx + u dy \right)
\end{align*}
\corollary
Если условия определения интеграла выполняются, то верно:
\begin{equation} \label{(6.3)}
    \int_\gamma f(z) dz = \int_{t_0}^{t_1}f(z(t))z'(t)dt
\end{equation}
\begin{equation} \label{(6.4)}
    \gamma: z(t), \ \int_{t_0}^{t_1}\left( P(t) + iQ(t) \right) dt = \int_{t_0}^{t_1}P(t)dt + i \int_{t_0}^{t_1}Q(t)dt
\end{equation}
\pr
Из \eqref{(6.2)} и свойств криволинейных интегралов II рода
\begin{align*}
  &\int_{\gamma}P(x,y) dx + Q(x,y)dy = \int_{t_0}^{t_1} \left( P(x(t), y(t))x'(t) + Q(x(t), y(t))y'(t) \right) dt
\end{align*}
\underline{\textbf{Общие свойства интеграла}}
\begin{enumerate}
    \item Линейность:
    \begin{align*}
      &\int_{\gamma}\left(\lambda f(z) + \mu g(z)\right) dz = \lambda\int_{\gamma}f(z) dz + \mu \int_\gamma g(z) dz
    \end{align*}
    \item Аддитивность вдоль кривой: если $\gamma = \gamma_1 \cup \gamma_2$
    (общая точка~--- только конец), то
    \begin{align*}
      &\int_{\gamma}f(z) dz = \int_{\gamma_1}f(z) dz + \int_{\gamma_2} f(z) dz
    \end{align*}
    \item Изменение знака при смене направления: если $\gamma^{-1}$~---
    кривая, совпадающая во всех точках с $\gamma$, но противоположно
    направленная, то
    \begin{align*}
      &\int_{\gamma^{-1}}f(z) dz = - \int_{\gamma}f(z) dz
    \end{align*}
    \item Инвариантность относительно замены параметра.
    \item Выполняется неравенство:
    \begin{equation} \label{(6.5)}
        \left| \int_\gamma f(z) dz \right| \leq \int_{\gamma}\left| f(z) \right| \cdot \left| dt \right| = \int_{t_0}^{t_1} \left| f(z(t)) \right| \sqrt{(x'(t))^2 + (y'(t))^2}dt
    \end{equation}
    \pr
    Из \eqref{(6.3)}:
    \begin{equation} \label{(6.6)}
        \left| \sigma(\lambda) \right| \leq \sum_{k = 1}^{m_\lambda}\left| f(\zeta) \right| \cdot \left| \Delta z_k \right|
    \end{equation}
\end{enumerate}
\example
\begin{align*}
  &J_k = \int_{\gamma: \left| z-a \right| = r} (z-a)^k dz, \ k \in \ZZ
\end{align*}
Параметр: $z(t)= a+re^{it}, \ t \in [0;2\pi]$
\begin{align*}
  & dz(t) = rie^{it}dt
\end{align*}
\begin{align*}
  &J_k = \int_{0}^{2\pi} r^ke^{ikt}rie^{it}dt = r^{k+1}i\int_{0}^{2\pi}e^{i(k+1)t} dt
\end{align*}
\begin{align*}
  &J_{-1} = 2 \pi i
\end{align*}
\begin{align*}
  &J_k = r^{k+1}i\int_{0}^{2\pi} \left( \cos(k+1)t + i \sin(k+1)t \right) dt = 0, \ k \neq -1
\end{align*}
\example
\begin{align*}
  &J_1 = \int_{\gamma} dz, \ J_2 = \int_{\gamma}z dz
\end{align*}
$\gamma: z(t)$, и согласно \eqref{(6.3)}
\begin{align*}
  &J_1 = \int_{t_0}^{t_1} z'(t)dt = z(t_1) - z(t_0)
\end{align*}
\begin{align*}
  &J_2 = \int_{t_0}^{t_1} z(t)z'(t)dt = \frac{1}{2}\int_{t_0}^{t_1}\frac{d}{dt}\left( z^2(t) \right)dt = \frac{1}{2}\left( z^2(t_1) - z^2(t_0) \right)
\end{align*}
\theorem
Пусть $f_n: G \mapsto \CC$ непрерывна на $G$ $\forall n \in \NN$, $\gamma$~---
КГК, $\gamma \subseteq G$.
\\
Пусть $\dst \sum_{n=1}^{\infty}f_n(z) \underset{\gamma}{\rightrightarrows} S(z)
\ \forall z \in \gamma$.
Тогда $S(z)$ непрерывна на $\gamma$ и выполняется
\begin{equation} \label{(6.7)}
    \int_{\gamma}S(z) dz = \sum_{n=1}^{\infty}\int_{\gamma}f_n(z)dz
\end{equation}
\pr
Ряд сходится равномерно на $\gamma$
\begin{align*}
  & \Updownarrow
\end{align*}
\begin{equation} \label{(6.8)}
    \forall \varepsilon > 0 \ \exists N(\varepsilon): \ \forall N\geq N(\varepsilon) \ \underset{z \in \gamma}{\sup} \left| S(z) - \sum_{n = 1}^{N}f_n(z) \right| = \underset{z \in \gamma}{\sup} \abs{S - S_N(z)} \leq \varepsilon
\end{equation}
\begin{align*}
  &\exists \delta > 0: \ \forall z_0 \in \gamma, \ z \in \gamma, \ \left| z - z_0 \right| < \delta: \ \left| S(z) - S(z_0) \right| \leq \left| S(z) - S_N(z) \right| +  \left| S_N(z) - \right. \\
  &\left. - S_N(z_0) \right| + \left| S_N(z_0) - S(z_0) \right| \leq \left| S_N(z) - S_N(z_0) \right| + 2 \varepsilon \leq 3 \varepsilon
\end{align*}
\begin{align*}
  & \Downarrow
\end{align*}
\begin{align*}
  & S(z) \rightarrow S(z_0)
\end{align*}
\begin{align*}
  & \Downarrow
\end{align*}
\begin{align*}
  & \exists \int_{\gamma}S(z)dz
\end{align*}
\begin{align*}
  &\forall N \geq N(\varepsilon) \ \left| \int_{\gamma}S(z)dz - \int_{\gamma}S_N(z)dz\right| = \left| \int_\gamma \left( S(z) - S_N(z) \right)dz\right| \leq \int_\gamma \left| S(z) - S_N(z) \right| \left| dz \right| \leq \\
  & \leq \varepsilon \int_{\gamma} \left| dz \right| = \varepsilon l(\gamma) = \varepsilon \cdot const
\end{align*}
Теорема доказана.
\Def
Пусть $g: G \mapsto \CC$ на области $G$; назовем это \textbf{первообразной}
непрерывной функции $f: G \mapsto \CC$, если $g$ регулярна на $G$ и $g'(z) =
f(z) \ \forall z \in G$.
\Def
Выражение $f(z)dz$ называется \textbf{полным дифференциалом в области $G$}, если
существует первообразная $g$ для $f$ на $G$, т.~е. $f(z)dz = g'(z)dz$.
\theorem
Пусть $f:G \mapsto \CC$ непрерывна на области $G$. Тогда:
\begin{enumerate}
    \item если $f dz$~--- полный дифференциал на $G$, то для любой замкнутой КГК
    $\overset{\circ}{\gamma} \subseteq G$ выполняется
    \begin{equation} \label{(6.9)}
        \int_{\overset{\circ}{\gamma}} f(z) dz = 0
    \end{equation}
    \item если для любой замкнутой ломаной кривой $\gamma$ выполняется
    \eqref{(6.9)}, то $f dz$~--- полный дифференциал.
\end{enumerate}
\pr ~
\\
\begin{enumerate}
    \item $\exists g:G \mapsto \CC$, регулярная, такая, что $g'(z) = f(z)$.
    Тогда
    \begin{align*}
      & \int_{\overset{\circ}{\gamma}} f(z) dz \underset{\overset{\circ}{\gamma}: z = z(t)}{=} \int_{t_0}^{t_1} g'(z(t))z'(t)dt = \int_{t_0}^{t_1}\frac{d}{dt}(g(z(t)))dt = g(z(t_1)) - g(z(t_0)) = \\
      & = g(z(t_1)) - g(z(t_1)) = 0
    \end{align*}
    \item Фиксируем $a \in G$ как начальную точку ломаной $\gamma$. $\forall z
    \in G$ $\exists \gamma_{az}$~--- ломаная с началом в $a$ и концом в $z$.
    \begin{align*}
      & g(z) = \int_{\gamma_{az}} f(z) dz
    \end{align*}
    не зависит от $\gamma_{az}$, а лишь зависит от $z$. Действительно, если
    $\exists \gamma_{az}, \ \exists \tilde{\gamma_{az}}$, то пусть
    $\overset{\circ}{\gamma} = \gamma_{az} \cup \tilde{\gamma_{az}}^{-1}$,
    тогда из \eqref{(6.8)}
    \begin{align*}
      & \int_{\overset{\circ}{\gamma}} f(z) dz = 0 = \int_{\gamma_{az}}f dz - \int_{\tilde{\gamma_{az}}}f(z)dz
    \end{align*}
    Докажем, что $\forall z \ g'(z) = f(z)$. Рассмотрим $z_0$; $\exists
    \varepsilon > 0: B_{\varepsilon}(z_0) \subseteq G$, $\Delta z: 0 <
    \left| \Delta z \right| < \varepsilon$. Тогда $z_0 + \Delta z \in G$.
    Рассмотрим
    \begin{align*}
      & \frac{g(z_0+\Delta z) - g(z_0)}{\Delta z}
    \end{align*}
    Тогда
    \begin{align*}
      & g(z+ \Delta z) = \int_{\gamma_{az_0} \cap [z_0; \Delta z + z_0]}
    \end{align*}
    \begin{align*}
      & \Downarrow
    \end{align*}
    \begin{align*}
      & \frac{g(z_0+ \Delta z) - g(z_0)}{\Delta z} = \frac{1}{\Delta z} \int_{[z_0; \Delta z + z_0]} f(z)dz
    \end{align*}
    \begin{align*}
      & \left| \frac{\Delta g}{\Delta z} - f(z_0) \right| = \left| \frac{1}{\Delta z} \int_{[z_0; \Delta z + z_0]}(f(z) - f(z_0))dz \right|
    \end{align*}
    В силу непрерывности $f(z)$, полагая $r(\varepsilon)$~--- радиус шара, где
    $\left| f(z) - f(z_0) \right| < \varepsilon$,
    \begin{align*}
      & \forall z \in B_{r(\varepsilon)}(z_0)\cap B_\varepsilon(z_0) \ \left| \frac{\Delta g}{\Delta z} - f(z_0) \right| \leq \left| \frac{\varepsilon\min\left\{ r(\varepsilon), \varepsilon \right\}}{\min\left\{ r(\varepsilon), \varepsilon \right\}} \right| = \varepsilon
    \end{align*}
\end{enumerate}
