\begin{flushright}
    \textit{Лекция 23 (от 23.11)}
\end{flushright}
\theorem
Пусть $a, b \in \CC \setminus \left\{ 0 \right\}$. Пусть $\left( B_{\left| a
      \right|}(a), h_a \right)$~--- элемент, порожденный $\Ln z$. Тогда для
любой $\gamma_{ab}$, не содержащей нуля, существует аналитическое продолжение в
некоторый элемент $\left( B_{\left| b \right|}(b), h_b \right)$~--- элемент,
порожденный $\Ln z$, причем
\begin{equation}\label{(28.1)}
    h_b(b) = h_a(a) + \int_{\gamma_{ab}}\frac{d\zeta}{\zeta}
\end{equation}
\begin{equation}\label{(28.2)}
    \forall z \in B_{\left| b \right|}(b) \ h_b(z) = h_b(b) + \int_{b}^z\frac{d\zeta}{\zeta}
\end{equation}
$\forall c \in \CC \setminus\left\{ 0 \right\}$, $\forall \left( B_{\left| c
      \right|}(c), h_c\right)$, порожденного $\Ln z$, существует
$\tilde{\gamma}_{ac}$ такая, что этот элемент есть аналитическое продолжение
исходного вдоль этой кривой.
\pr
Заметим, что в силу того, что кривая не содержит нуля, $d = \min \left\{ \left|
        \zeta \right|: \zeta \in \gamma_{ab} \right\} > 0$. Рассмотрим набор
точек $a = z_0, z_1, \dots, z_{K-1}, z_K = b$ на кривой, причем
$l_{\gamma_{z_{k-1}z_k}} \leq \dst \frac{d}{2}$. Зададим для каждой из точек шар
$B_{\left| z_k \right|}z_k$. Заметим, что $\left| z_k- z_{k-1} \right| \leq \dst
\frac{d}{2} < \left| z_{k-1} \right|$. Тогда, как видим, следующая точка лежит в
окрестности предыдущей.
\\
Допустим, доказано существование продолжения на $\gamma_{az_{k-1}}$; докажем
существование $\gamma_{az_k}$. На $B_{\left| z_k \right|}(z_k)$
\begin{equation}\label{(28.3)}
    h_k(z_k) = h_{k-1}(z_{k-1}) + \ln \left| z_k \right| - \ln \left| z_{k-1} \right| + i\Delta_{\gamma_{z_{k-1}z_k}} \argt z = h_{k-1}(z_{k-1}) +\int_{\gamma_{z_{k-1}z_k}} \frac{d\zeta}{\zeta}
\end{equation}
\begin{equation}\label{(28.4)}
    \forall z \in B_{\left| z_k \right|}(z_k) h_k(z) = h_k(z_k) + \int_{z_k}^z \frac{d\zeta}{\zeta}
\end{equation}
Значит, получили искомое продолжение.
\\
Для точки $c$ фиксируем некоторую не содержащую нуля кривую $\gamma_{ac}$. Для
нее существует аналитическое продолжение $\left( B_{\left| c \right|}(c),
    \tilde{h}_c \right)$; и $\exists k \in \ZZ: h_c(z) = \tilde{h}_c(z) + 2 i
\pi k$. Тогда искомая $\tilde{\gamma}_{ac}$ будет равна $\gamma_{ac}$,
объединенной с кругом с центром в нуле и радиусом $c$, обойденным $k$ раз.
\corollary
Полная аналитическая функция $\Ln z$ состоит из элементов $\left( B_{\left| a
      \right|}(a), h_a(z) + 2 i \pi k \right)$, $a \neq 0$, $k \in \ZZ$,
$h_a$~--- регулярная ветвь логарифма.
\begin{align*}
  & \left\{ \sqrt[n]{z} \right\} = \exp \left( \frac{1}{n}\Ln z \right)
\end{align*}
\corollary
Полная аналитическая функция $\left\{ \sqrt[n]{z} \right\}$ состоит из элементов
$\left( B_{\left| a \right|}(a), g_a(z)\exp \left( \frac{i}{n} 2 \pi k
    \right)\right)$, $a \neq 0$, $k \in \left\{ 0, \dots, n-1 \right\}$,
$g_a$~--- регулярная ветвь корня.
\begin{center}
    \textbf{Риманова поверхность $\Ln z$}
\end{center}
\begin{align*}
  & G_0 = \CC \setminus (-\infty; 0]: \ h_0(z) = \ln \left| z \right| + i \argm z, \ \argm z \in (-pi; \pi); \ h_k(z) = h_0(z) + 2 i \pi k, \ k \in \ZZ
\end{align*}
\begin{align*}
  & G_k = \CC \setminus (-\infty; 0]: \\
  & \forall x < 0 \ h_{k-1}(x+i0) = \ln \left| x \right| + i \pi + 2\pi (k-1)i, \  h_{k}(x+i0) = \ln \left| x \right| + i (-\pi) + 2\pi k i
\end{align*}
Значения на верхнем и нижнем краях разреза равны. Можем склеить верхнюю часть
$G_k$ с нижней $G_{k+1}$ для всех $k$. На этой поверхности функция будет
регулярна в любой точке.
\begin{center}
    \textbf{Риманова поверхность $\sqrt{z}$}
\end{center}
\begin{align*}
  & G_0 = \CC \setminus (-\infty; 0]: \ g_0(z) = \sqrt{\left| z \right|} \exp \left( \frac{i}{2} \argm z \right)
\end{align*}
\begin{align*}
  & G_1 = \CC \setminus (-\infty; 0]: \ g_1(z) = -g_0(z)
\end{align*}
\begin{align*}
  & \forall x < 0 \ g_0(x+i0) = i\sqrt{\left| x \right|}, \ g_0(x-i0) = -i\sqrt{\left| x \right|} \\
  & g_1(x+i0) = g_0(x-i0), \ g_1(x-i0) = -g_0(x+i0)
\end{align*}
Значения на верхнем и нижнем краях разреза равны. Можем склеить верхние части с
нижними. На этой поверхности функция будет регулярна в любой точке.
\section{$\S 29.$ Особые точки аналитических функций.}
\Def
Пусть аналитическая функция $F(z)$ такова, что $\exists \gamma_{ab}$: $\forall z
\in \gamma_{ab}\setminus \left\{ b \right\}$ есть аналитическое продолжение
вдоль $\gamma_{az}$, а вдоль $\gamma_{ab}$ его нет. Тогда $b$~--- \textbf{особая
  точка.}
\Def
Пусть $b = \infty$, при замене аргумента $\tilde{F}\left( \
dst \frac{1}{z}\right) = F(z)$ имеет особую точку в нуле, тогда $\infty$~---
\textbf{особая точка} функции $F$.
\Note
Полюса и СОТ регулярной функции~--- особые точки аналитической, а УОТ~--- нет.
\Def
Точка $a$ называется \textbf{точкой ветвления аналитической функции}, если
существует проколотая окрестность, где аналитическая функция определена, но не
однозначно, т.~е. можем в некоторой проколотой окрестности точки $a$ продолжить
функцию, обходя вокруг этой точки, до элемента с той же окрестностью, но другой
функцией.
\Example
$0, \infty$~--- точки ветвления логарифма (никогда не повторяются функции,
логарифмический порядок) и корня (могут повториться, алгебраический порядок).
\theorem (Коши-Адамара)
Пусть степенной ряд
\begin{align*}
  & \sum_{n=0}^\infty c_n(z-a)^n = S(z)
\end{align*}
сходится в круге $B_R(a), \ 0 < a < \infty$. Тогда на границе области (круга)
сходимости существуют особые точки аналитической функции.
\pr
От противного. Пусть $\gamma_R = \left\{ \zeta: \left| \zeta - a \right| = R
\right\}$,
\begin{align*}
  &\forall \zeta \in \gamma_R \ \exists \left( B_{r_\zeta}, f_\zeta \right), \ r_\zeta > 0
\end{align*}
аналитическое продолжение $\left( B_R(a), S(z) \right)$ вдоль радиуса
$\gamma_R$. Всю окружность можно покрыть кругами с центрами в каждой точке, и
существует конечное подпокрытие (лемма Гейне-Бореля), т.~е. конечный набор точек
$\left\{ \zeta_k \right\}_{k=1}^K$ таких, что круги с центрами в них содержат в
себе $\gamma_R$. Получили функцию (аналитическую):
\begin{align*}
  & F(z) = \begin{cases}
      \left( B_R(a), S(z) \right) \\
      \left( B_{r_k}(\zeta_k), f_k \right)
  \end{cases}
\end{align*}
Пусть $G$~--- объединение $B_R(a)$ и всех перечисленных кругов. Докажем
однозначность $F$ на $G$. Рассмотрим $m, n: \ B_{r_n}(\zeta_n)\cap
B_{r_m}(\zeta_m)\neq \varnothing$; тгда в пересечении этих двух кругов с
$B_R(a)$ $f_m(z) = S(z) = f_n(z)$, и по теореме единственности $f_n(z) = f_m(z)$
на пересечении этих двух кругов. Рассмотрим теперь $r = \inf \left\{ \left|
        z-\zeta \right| : z \in B_R(a), \ \zeta \in \CC \setminus G\right\} >
0$; тогда $B_{R+r} \subseteq G$ и $F(z)$ регулярна в этом круге. Ряд
\begin{align*}
  & \sum_{n=0}^\infty \frac{F^{(n)}(z)}{n!}(z-a)^n
\end{align*}
сходится к сумме при $\left| z-a \right|< R$, но тогда этот ряд и в круге
$B_{R+r}$ сходится, противоречие.
\Example
\begin{align*}
  & f(z) = \frac{1}{(z+3)(z^2+2)}
\end{align*}
Радиус сходимости равен $\sqrt{2}$.
\Example
\begin{align*}
  & \frac{1}{1-z} \sum_{n=0}^{\infty} z^n
\end{align*}
Одна особая точка на границе, расходится в них всех.