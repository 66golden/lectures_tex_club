\section{Линейные нормированные пространства}

\begin{definition}
	\textit{Линейным нормированным пространством} над полем $\Kk$, где $\Kk = \R$ или $\Kk = \Cm$, называется линейное пространство $E$ над $\Kk$ с функцией $\|\cdot\| : E \to \R$, обладающей следующими свойствами:
	\begin{enumerate}
		\item $\forall x \in E: \|x\| \ge 0$, причем $\|x\| = 0 \lra x = 0$
		\item $\forall x \in E: \forall \alpha \in \Kk: \|\alpha x\| = |\alpha|\|x\|$
		\item $\forall x, y \in E: \|x + y\| = \|x\| + \|y\|$ (\textit{неравенство треугольника})
	\end{enumerate}

	Функция $\|\cdot\|$ называется \textit{нормой} на пространстве $E$.
\end{definition}

\begin{example}
    Легко видеть, что любое линейное нормированное пространство $E$ является метрическим с метрикой, заданной для произвольных $x, y \in E$ как $\rho(x, y) := \|x - y\|$. Приведенные ранее в качестве примеров метрические пространства являются линейными нормированными, и метрика в них имеет вид нормы разности элементов.
\end{example}

\begin{note}
	Норма непрерывна как функция $E \to \R$. Если для последовательности $\{x_n\} \subset E$ выполнено $x_n \to_E x$, то $\rho(x_n, x) = \|x_n - x\| \to 0$. Дважды воспользуемся неравенством треугольника:
	\begin{itemize}
		\item $\|x_n\| \le \|x_n - x\| + \|x\|$
		\item $\|x\| \le \|x_n - x\| + \|x_n\|$
	\end{itemize}

	Таким образом, $\big|\|x_n\| - \|x\|\big| \to 0$, то есть $\|x_n\| \to \|x\|$, что и требовалось.
\end{note}

\begin{definition}
	Пусть $E$ "--- линейное нормированное пространство. Множество $L \subset E$ называется:
	\begin{itemize}
		\item \textit{Линейным многообразием} в $E$, если $L$ замкнуто относительно сложения и умножения на скаляры из $\Kk$
  		\item \textit{Подпространством} в $E$, если $L$ является линейным многообразием в $E$ и при этом замкнуто 
	\end{itemize}
\end{definition}

\begin{definition}
	Пусть $E$ "--- линейное нормированное пространство, $M \subset E$. \textit{Линейной оболочкой} множества $M$ называется множсетво следующего вида:
	\[[M] := \left\{\sum_{k = 1}^n \alpha_im_i : \alpha_1, \dotsc \alpha_n \in \Kk, m_1, \dotsc, m_n \in M\right\}\]
\end{definition}

\begin{note}
	Линейная оболочка множества всегда является линейным многообразием в $E$, но не всегда является подпространством.
\end{note}

\begin{definition}
	Пусть $E$ "--- линейное пространство. Нормы $\|\cdot\|$ и $\|\cdot\|'$ на $E$ называются \textit{эквивалентными}, если существуют числа $C_1, C_2 > 0$ такие, что для любого $x \in E$ выполнены неравенства:
	\[C_1\|x\|' < \|x\| < C_2\|x\|'\]
\end{definition}

\begin{note}
	Если нормы $\|\cdot\|$ и $\|cdot\|'$ эквивалентны, то последовательность $\{x_n\} \subset E$ сходится или расходится относительно обеих норм одновременно.
\end{note}

\begin{definition}
	Пусть $E$ "--- линейное пространство. \textit{Размерностью} пространства $E$ называется наибольший размер линейно независимой системы в $E$. Обозначение "--- $\dim E$.
\end{definition}

\begin{proposition}\label{normequiv}
	Пусть $E$ "--- линейное пространство, и $\dim E < +\infty$. Тогда любые две нормы на $E$ эквивалентны.
\end{proposition}

\begin{proof}[Доказательство для случая, когда {$\Kk = \R$}]
	Поскольку $E$ конечномерно, то в нем можно выбрать максимальную по включению линейно независимую систему $e = \{e_1, \dotsc, e_n\} \subset E$, тогда $e$ будет являться базисом в $E$. Для произвольного элемента $x \in E$, имеющего в базисе $e$ координатный столбец $\alpha \in \R^n$, зададим его \textit{евклидову норму} следующим образом:
	\[\|x\|_{euc} := \sqrt{\sum_{k = 1}^n \alpha_k^2}\]

	Зафиксируем произвольную норму $\|\cdot\|$ и докажем, что она эквивалентна норме $\|\cdot\|_{euc}$
	\begin{enumerate}
		\item Покажем, что $\|\cdot\| < C \|\cdot\|_{euc}$ для некоторого $C > 0$. Для произвольного $x \in E$, имеющего в базисе $e$ координатный столбец $\alpha \in \R^n$, выполнено следующее:
		\[\|x\| = \left\|\sum_{k = 1}^n\alpha_k e_k \right\| \le \max_{1 \le k \le n}\|e_k\|\left(\sum_{k = 1}^n|\alpha_k|\right)\]
  
		Поскольку для любого $k \in \{1, \dotsc, n\}$ выполнено $|\alpha_k| < \|x\|_{euc}$, то достаточно взять число $C := n(\max_{1 \le k \le n}\|e_k\|)$.

		\item Покажем теперь, что $\|\cdot\|_{euc} < \widetilde C \|\cdot\|$ для некоторого $\widetilde C > 0$. Предположим противное, тогда для любого $n \in \N$ существует $x_n \in E$ такое, что $\|x\|_{euc} > n\|x\|$. Можно без ограничения общности считать, что $\|x_n\|_{euc} = 1$ для любого $n \in \N$, откуда $\|x\| < \frac 1n$.
		
		Поскольку последовательность $\{x_n\}$ содержится в единичной сфере $S_{euc}(0, 1)$ и относительно евклидовой нормы сфера компактна, можно выделить из $\{x_n\}$ подпоследовательность $\{x_{n_k}\}$, сходящуюся относительно евклидовой нормы. Тогда $\|x_{n_k} - x\|_{euc} \to 0$ для некоторого $x \in S_{euc}(0, 1)$, тогда, в силу пункта $(1)$, $\|x_{n_k} - x\| \to 0$. Но по построению $\|x_{n_k}\| < \frac 1{n_k}\to 0$, поэтому $x = 0$ --- противоречие с тем, что $x \in S_{euc}(0, 1)$.\qedhere
	\end{enumerate}
\end{proof}

\begin{corollary}
	Пусит $E$ "--- линейное нормированное пространство, $x_1, \dotsc, x_n$. Тогда линейная оболочка $L := [x_1, \dotsc, x_n]$ образует подпространство в $E$.
\end{corollary}

\begin{proof}
	Заметим, что $\dim L < +\infty$, и по утверждению \ref{normequiv} сужение нормы из $E$ на $L$ эквивалентно евклидовой норме. Относительно евклидовой нормы конечномерное пространство полно, поэтому и $L$ полно относительно нормы из $E$. Следовательно, $L$ замкнуто как подмножество в $E$.
\end{proof}

\begin{proposition}[лемма о <<почти перпендикуляре>>]\label{almostperp}
	Пусть $E$ "--- линейное нормированное пространство, $L \subsetneq E$ "--- подпространство. Тогда для любого $\epsilon > 0$ существует $y \in E$ такой, что $\|y\| = 1$ и $\rho(y, L) = \sup_{z \in L}\|y - z\| \ge 1 - \epsilon$.
\end{proposition}

\begin{proof}
		Зафиксируем $y_0 \in E \bs L$ положим $d := \rho(y, L) > 0$. Выберем вектор $z_0 \in L$ такой, что $d \le \|y_0 - z_0\| \le d(1 + \epsilon)$ и покажем, что подходит вектор $y := \frac{y_0 - z_0}{\|y_0 - z_0\|}$. Действительно, $\|y\| = 1$, и для любого $z \in L$ выполнены неравенства:
		\[\|y - z\| = \frac1{\|y_0 - z_0\|}\big\|y_0 - (z_0 + \|y_0 - z_0\|z)\big\| \ge \frac{d}{\|y_0 - z_0\|} \ge \frac1{1-\epsilon} \ge 1 + \epsilon\]

		Получено требуемое.
\end{proof}

\begin{theorem}[Рисса]\label{thm4.1}
	Пусть $E$ "--- линейное нормированное пространство. Тогда единичная сфера $S(0, 1)$ компактна в $E$ $\lra$ $\dim E < +\infty$.
\end{theorem}

\begin{proof}~
	\begin{itemize}
		\item[$\la$] По утверждению \ref{normequiv}, все нормы на конечномерном линейном пространстве эквивалентны, а относительно евклидовой нормы сфера $S(0, 1)$ компактна.
  
  		\item[$\ra$] Зафиксируем $\epsilon > 0$ и построим последовательность $\{x_n\} \subset S(0, 1)$ с попарными расстояниями не меньше $1 - \epsilon$, из чего будет следовать, что сфера $S(0, 1)$ не вполне ограниченна и потому не компактна:
		\begin{itemize}
			\item[$\bullet$] Выберем $x_1 \in S(0, 1)$ произвольным образом
			\item[$\bullet$] По утверждению \ref{almostperp} выберем $x_2 \in S(0, 1) \bs [x_1]$ такое, что $\rho(x_2, [x_1]) \ge 1- \epsilon$
			\item[$\bullet$] По утверждению \ref{almostperp} выберем $x_3 \in S(0, 1) \bs [x_1, x_2]$ такое, что $\rho(x_3, [x_1, x_2]) \ge 1- \epsilon$
		\end{itemize}

		Поскольку $\dim E = + \infty$, то процесс не закончится, и будет получена искомая последовательность $\{x_n\}$.\qedhere
	\end{itemize}
\end{proof}

\begin{definition}
	\textit{Евклидовым пространством} над полем $\Kk$, где $\Kk = \R$ или $\Kk = \Cm$, называется линейное пространство $E$ над $\Kk$ с функцией $(\cdot, \cdot) : E^2 \to \Kk$, обладающей следующими свойствами:
	\begin{enumerate}
		\item $\forall x \in E: (x, x) \ge 0$, причем $(x, x) = 0 \lra x = 0$
		\item $\forall x, y \in E: (x, y) = \overline{(y, x)}$
		\item $\forall x, y, z \in E: \forall \alpha, \beta \in \Kk (\alpha x + \beta y, z) = \alpha(x, z) + \beta(y, z)$
	\end{enumerate}

	Функция $(\cdot, \cdot)$ называется \textit{скалярным произведением} на пространстве $E$.
\end{definition}

\begin{example}
	Легко видеть, что любое евклидово пространство $E$ является линейным нормированным с нормой, заданной для произвольного $x \in E$ как $\|x\| := \sqrt{(x, x)}$. Некоторые из приведенных ранее в качестве примеров метрических пространств являются не только линейными нормированными, но и евклидовыми, и норма в них порождается скалярным произведением:
	\begin{enumerate}
		\item Скалярное произведение на $\R^n_2$ задается для произвольных $x, y \in \R^n$ следующим образом:
  		\[(x, y) := \sum_{k = 1}^n x_ky_k\]

		\item Скалярное произведение на $l_2$ задается для произвольных $x, y \in l_2$ следующим образом:
		\[(x, y) := \sum_{n = 1}^\infty x_ny_n\]

		\item Скалярное произведение на $L_2[a, b]$ задается для произвольных $f, g \in L_2[a, b]$ следующим образом:
		\[(f, g) := \int_a^b f(x)g(x)dx\]
	\end{enumerate}
\end{example}

\begin{theorem}[\textit{без доказательства}]
	Пусть $E$ "--- линейное нормированное пространство. Тогда норма в $E$ порождается скалярным произведением $\lra$ для любых $x, y \in E$ выполнено равенство параллелограмма:
	\[\|x+y\|^2 + \|x-y\|^2 = 2\|x\|^2 + 2\|y\|^2\]
\end{theorem}

\begin{definition}
	Пусть $E$ "--- линейное нормированное пространство, $L \subset E$ "--- линейное многообразие, $h \in E$. \textit{Элементом наилучшего приближения} для $h$ называется $x \in L$ такой, что $\|h - x\| = \rho(h, L)$
\end{definition}

\begin{note}
	Можно построить примеры, в которых элемент наилучшего приближения не существует или не единственен.
\end{note}

\begin{definition}~
	\begin{itemize}
		\item Полное линейное нормированное пространство $E$ называется \textit{банаховым}.
		\item Полное евклидово пространство $H$ называется \textit{гильбертовым}.
	\end{itemize}
\end{definition}

\begin{proposition}\label{bestelt}
	Пусть $H$ "--- гильбертово пространство, $M \subset H$ "--- подпространство в $H$. Тогда для любого $h \in H$ существует единственный элемент наилучшего приближения $x \in M$.
\end{proposition}

\begin{proof}
	Зафиксируем $h \in H$. Сначала докажем, что элемент наилучшего приближения для $h$ существует. Положим $d := \rho(h, M)$. Если $d = 0$, то, в силу замкнутости множества $M$, выполнено $h \in M$, и в качестве элемента наилучшего приближения для $h$ подходит сам $h$. Иначе --- выберем $\{x_n\} \subset M$ такую, что $\|h - x_n\| \to d$. По равенству параллелограмма, для любых $n, m \in \N$ выполнено следующее:
	\[\|x_n - x_m\|^2 = 2\|h - x_n\|^2 + 2\|h - x_m\|^2 - 4\left\|h - \frac{x_n + x_m}{2}\right\|^2\]

	Поскольку $\|h - x_n\|^2 \to d^2$ при $n \to \infty$, $\|h - x_m\|^2 \to d^2$ при $m \to \infty$ и выполнено неравенство $\|h - \frac{x_n + x_m}{2}\|^2 \le d^2$, последовательность $\{x_n\}$ фундаментальна. Поскольку пространство $H$ полно, а $M$ замкнуто, существует $x \in M$ такой, что $x_n \to_H x$, причем, в силу непрерывности нормы, $\|h - x\| = d$.

	Покажем теперь, что элемент наилучшего приближения для $h$ единственен. Пусть для некоторого $y \in M$ тоже выполнено равенство $\|h - y\| = d$, тогда, по равенству параллелограмма, выполнено следующее:
	\[4d^2 = 2\|h - x\|^2 + 2\|h - y\|^2 = \|x - y\|^2 + 4\left\|h - \frac{x+y}{2}\right\|^2 \ge \|x - y\|^2 + 4d^2\]

	Значит, $\|x - y\| = 0$, то есть $x = y$, и элемент наилучшего приближения единственен.
\end{proof}

\begin{definition}
	Пусть $E$ "--- евклидово пространство, $S \subset E$. \textit{Аннулятором} множества $S$ называется следующее множество:
	\[S^\perp := \{y \in E: \forall x \in S: (x, y) = 0\}\]
\end{definition}

\begin{note}
	Легко проверить, что $S^\perp$ является подпространством в $E$. Кроме того, выполнены равенства $S^\perp = [S]^\perp = \overline{[S]}^\perp$.
\end{note}

\begin{theorem}[Рисса, о проекции]\label{thm4.3}
	Пусть $H$ "--- гильбертово пространство, {$M \subset H$ "--- под-} пространство в $H$. Тогда $H = M \oplus M^\perp$.
\end{theorem}

\begin{proof}
	Покажем сначала, что $H = M + M^\perp$. Зафиксируем $h \in H$, и по утверждению \ref{bestelt} выберем его элемент наилучшего приближения $x \in M$. Положим $y := h - x$ и $d := \|y\|$, тогда для произвольного $m \in M$ и произвольного $\alpha \in \R \bs \{0\}$ выполнено следующее:
	\[d^2 = \|h - x\|^2 \le \|h - (x + \alpha m)\|^2 = d^2 - 2\alpha(h, m) + \alpha^2\|m\|^2\]

	Следовательно, $(h, m) \le \frac\alpha2\|m\|^2$ для любого $\alpha \in \R \bs \{0\}$, откуда $(h, m) = 0$. В силу произвольности выбора $m \in M$, получаем, что $y \in M^\perp$, причем $h = x + y$. Значит, выполнено равенство $H = M + M^\perp$. Проверим теперь, что сумма $M + M^\perp$ --- прямая. Действительно, если $z \in M \cap M^\perp$, то $(z, z) = 0$, откуда $z = 0$, что и означает требуемое.
\end{proof}

\begin{note}
	Теорема выше отражает одно из важных свойств гильбертовых пространств, которое для банаховых пространств верно не всегда. Далее мы обсудим еще одно такое свойство --- существование базиса в бесконечномерном сепарабельном пространстве.
\end{note}

\begin{definition}
	Пусть $E$ "--- линейное нормированное пространство. Система $\{e_n\} \subset E$ называется \textit{базисом} в $E$, если для любого $x \in E$ существует единственный набор чисел $\{\alpha_n\} \subset \Kk$ такой, что $x = \sum_{n} \alpha_n e_n$.
\end{definition}

\begin{proposition}[неравенство Бесселя]
	Пусть $E$ "--- евклидово пространство, элементы $e_1, \dotsc, e_k \in E$ образуют ортонормированную систему. Тогда для любого $x \in E$ выполнено следующее неравенство:
	\[\sum_{k = 1}^n|(x, e_k)|^2 \le \|x\|^2\]
\end{proposition}

\begin{proof}
	Достаточно заметить, что в силу ортнормированности системы $\{e_k\}$ выполнено следующее равенство:
	\[\left\|x - \sum_{k = 1}^n(x, e_k)e_k\right\|^2 = \|x\|^2 - \sum_{k = 1}|(x, e_k)|^2\]

	Поскольку левая часть равенства выше неотрицательна, то неотрицательна и правая часть, из чего следует требуемое.
\end{proof}

\begin{corollary}[неравенство Коши---Буняковского]
	Пусть $E$ "--- евклидово пространство. Тогда для любых $x, y \in E$ выполнено следующее неравенство:
	\[|(x, y)| \le \|x\|\|y\|\]
\end{corollary}

\begin{proof}
	Случай, когда $y = 0$, тривиален, поэтому можно считать, что $y \ne 0$. Тогда система $\big\{\frac{y}{\|y\|}\big\}$ является ортонормированной, и по неравенству Бесселя выполнено следующее:
	\[\left|\left(x, \frac y{\|y\|}\right)\right|^2 \le \|x\|^2 \ra |(x, y)| \le \|x\|\|y\|\qedhere\]
\end{proof}

\begin{note}
	Из неравенства Коши-Буняковского, в частности, следует, что скалярное произведение непрерывно на $E$ относительно порождаемой им нормы.
\end{note}

\begin{proposition}\label{minproperty}
	Пусть $E$ "--- евклидово пространство, элементы $e_1, \dotsc, e_k \in E$ образуют ортонормированную систему. Тогда для любого $x \in E$ и любого набора $\{\alpha_k\}_{k = 1}^n \subset \Kk$ выполнено следующее неравенство:
	\[\left\|x - \sum_{k = 1}^n \alpha_ke_k\right\| \ge \left\|x - \sum_{k = 1}^n (x, e_k)e_k\right\|\]

	Более того, равенство в неравенстве выше достигается тогда и только тогда, когда $\alpha_k = (x, e_k)$ для всех $k \in \{1, \dotsc, n\}$.
\end{proposition}

\begin{proof}[Доказательство для случая, когда {$\Kk = \R$}]
	В силу ортонормированности системы $\{e_k\}_{k = 1}^n$, выполнены следующие равенства:
	\begin{multline*}
		\left\|x - \sum_{k = 1}^n \alpha_ke_k\right\|^2
		=
		\left(x - \sum_{k = 1}^n \alpha_ke_k, x - \sum_{k = 1}^n \alpha_ke_k\right)
		= 
		\\
		=
		\|x\|^2 - 2\sum_{k = 1}^n\alpha_k(x, e_k) + \sum_{k = 1}^n\alpha_k^2 = \left\|x - \sum_{k = 1}^n(x, e_k)e_k\right\|^2 + \sum_{k = 1}^n\big((x, e_k) - \alpha_k\big)^2
	\end{multline*}

	Из равенства между самой левой и самой правой частями в цепочке выше получаем требуемое.
\end{proof}

\begin{theorem}\label{thm4.4}
	Пусть $H$ "--- сепарабельное гильбертово пространство, $\dim{H} = +\infty$, и ${e = \{e_n\} \subset E}$ "--- ортонормированная система. Тогда следующие условия эквивалентны:
	\begin{enumerate}
		\item $e$ "--- ортонормированный базис в $E$
  		\item $\overline{[e]} = E$
    	\item Для любого $h \in H$ выполнено равенство Парсеваля: $\|h\|^2 = \sum_{n = 1}^\infty |(h, e_n)|^2$
    	\item $e^\perp = \{0\}$
	\end{enumerate}
\end{theorem}

\begin{proof}~
	\begin{itemize}
		\item\imp{1}{2}Очевидно из определения базиса.
  
		\item\imp{2}{1}Зафиксируем произвольный $h \in H$ и произвольное $\epsilon > 0$. По условию, существует конечный набор $\alpha_1, \dotsc, \alpha_n$ такой, что $\|h - \sum_{k = 1}^n\alpha_k e_k\| < \epsilon$. Тогда, по утверждению \ref{minproperty}, выполнено также неравенство $\|h - \sum_{k = 1}^n(h, e_k) e_k\| < \epsilon$. Тогда, в силу произвольности числа $\epsilon$, выполнено равенство $h = \sum_{n = 1}^\infty (h, e_n)e_n$.
		
		Проверим теперь, что разложение элемента $h$ единственно. Пусть для некоторого набора $\{\beta_n\} \subset \Kk$ выполнено равенство $h = \sum_{n = 1}^\infty \beta_n e_n$. Тогда для любого $k \in \N$, скалярно умножая частичную сумму ряда $\sum_{n = 1}^\infty\beta_n e_n$ на $e_k$ и переходя к пределу, получаем, что $\beta_k = (h, e_k)$, что и означает требуемое.
  
		\item\eqv{1}{3}Уже было замечено, что для любого $h \in H$ и любого $k \in \N$ выполнено следующее равенство:
		\[\left\|h - \sum_{k = 1}^n(h, e_k)e_k\right\|^2 = \|h\|^2 - \sum_{k = 1}^n|(h, e_k)|^2\]
		
		Значит, $h = \sum_{n = 1}^\infty (h, e_n)e_n \lra \|h\|^2 = \sum_{n = 1}^\infty |(h, e_n)|^2$, и единственность разложения элемента $h$ доказывается так же, как в импликации выше.
  
		\item\eqv{2}{4}Из теоремы Рисса \ref{thm4.3} и равенства $e^\perp = \overline{[e]}^\perp$ получаем требуемое.\qedhere
	\end{itemize}
\end{proof}

\begin{corollary}
	Пусть $H$ "--- сепарабельное гильбертово пространство. Тогда в $H$ существует не более чем счетный ортонормированный базис.
\end{corollary}

\begin{proof}
	Случай, когда $\dim H <+\infty$, уже рассматривался в курсе алгебры, поэтому будем считать, что $\dim H <+\infty$. Рассмотрим счетное множество $F \subset H$ такое, что $\overline F = H$, тогда $F = \{f_n\}$. Проведем процесс ортогонализации Грама---Шмидта и получим ортонормированную систему $e$, для которой выполнено $[e] = [F]$, откуда $\overline{[e]} = H$, тогда, по теореме \ref{thm4.4}, $e$ "--- ортонормированный базис в $H$.  
\end{proof}

\begin{note}
	Результат выше позволяет построить изоморфизм между любыми двумя сепарабельными гильбертовыми пространствами бесконечной размерности: достаточно отобразить ортонормированный базис в одном пространстве в ортонормированный базис в другом пространстве.
\end{note}

\begin{definition}
	\textit{Линейным топологическим пространством} называется топологическое пространство $X$ с определенными на нем операциями сложения и умножения на числа из поля $\Kk$, непрерывными на $X$.
\end{definition}

\begin{example}
	Любое линейное нормированное пространство $E$ является линейным топологическим, как и пространство основных функций $D$, которое при этом даже не является метризуемым.
\end{example}