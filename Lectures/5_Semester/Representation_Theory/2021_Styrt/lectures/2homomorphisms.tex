\section{Гомоморфизмы представлений}

\begin{theorem}
	Пусть $\mathcal A \subset \mathcal{L}(V)$, пространство $V$ вполне приводимо относительно $\mc A$, $U \le V$ "--- $\mc A$-инвариантное подпространство. Тогда $U$ и $V / U$ тоже вполне приводимы.
\end{theorem}

\begin{proof}~
	\begin{enumerate}
		\item Пусть $U' \le U$ "--- $\mc A$-инвариантное подпространство. Выберем $\mc A$-инвариантное $W \le V$ такое, что $V = U' \oplus W$, тогда $U = U' \oplus (W \cap U)$, и подпространство $(W \cap U) \le U$ тоже $\mc A$-инвариантно. Значит, $U$ вполне приводимо.
		\item Выберем $\mc A$-инвариантное $W \le V$ такое, что $V = U \oplus W$. Применяя теорему о гомоморфизме для оператора проекции на $W$, получим, что $V / U \cong W$. Но подпространство $W$ вполне приводимо в силу пункта $(1)$, а действие операторов на $V / U$ можно естественным образом отождествить с действием операторов на $W$, из чего следует требуемое.\qedhere
	\end{enumerate}
\end{proof}

\begin{definition}
	\textit{Гомоморфизмом представлений} группы $G$ в пространствах $V$ и $W$ называется линейное отображение $\phi : V \to W$ такое, что для любого $g \in G$ выполнено $g\circ\phi = \phi\circ g$, то есть для любого $x \in V$ выполнено $g(\phi(v)) = \phi(gv)$.
\end{definition}

\begin{proposition}
	Пусть $\phi : V \to W$, $\psi : W \to U$ "--- гомоморфизмы представлений группы $G$. Тогда $\psi\circ\phi: V \to U$ "--- тоже гомоморфизм представлений.
\end{proposition}

\begin{proof}
	Достаточно заметить, что для любых $g \in G$ и $x \in V$ выполнены равенства $g(\psi(\phi(v))) = \psi(g(\phi(v))) = \psi(\phi(gv))$.
\end{proof}

\begin{note}
	Рассуждение выше может быть изображено на следующей коммутативной диаграмме:
	\[
	\begin{tikzcd}[row sep = large]
		V \arrow{r}{\phi} \arrow[swap]{d}{g} & W \arrow{r}{\phi} \arrow[swap]{d}{g} & U \arrow[swap]{d}{g}\\
		V \arrow{r}{\phi} & W \arrow{r}{\phi} & U
	\end{tikzcd}
	\]
\end{note}

\begin{definition}
	Гомоморфизм $\phi$ представлений группы $G$ в пространствах $V$ и $W$ называется:
	\begin{itemize}
		\item \textit{Эндоморфизмом}, если $W = V$ и рассматриваемые представления совпадают
		\item \textit{Изоморфизмом}, если $\phi$ биективен
		\item \textit{Автоморфизмом}, если $\phi$ "--- эндоморфизм и изоморфизм
	\end{itemize}
\end{definition}

\begin{definition}
	Представления группы $G$ в пространствах $V$ и $W$ называются \textit{изоморфными}, если существует изоморфизм $\phi: V \to W$. Обозначение "--- $V \gcong W$.
\end{definition}

\begin{example}
	Отображение $\id : V \to V$ является автоморфизмом любого представления группы $G$ в пространстве $V$.
\end{example}

\begin{note}
	Пусть $\phi : V \to W$ "--- изоморфизм представлений группы $G$ в пространствах $V$ и $W$. Легко видеть, что тогда $\phi^{-1} : W \to V$ "--- тоже является изоморфизмом представлений, поскольку для любого $g \in G$ выполнено $g\circ\phi = \phi \circ g \lra g^{-1}\circ\phi^{-1} = \phi^{-1}\circ g^{-1}$.
\end{note}

\begin{definition}
	\textit{Алгеброй} над полем $K$ называется линейное пространство $A$, на котором определена билинейная операция умножения $\cdot : A^2 \to A$. Алгебра называется:
	\begin{itemize}
		\item \textit{Ассоциативной}, если для любых $a, b, c \in A$ выполнено $(ab)c = a(bc)$
		\item \textit{Алгеброй с единицей}, если существует элемент $1 \in A$ такой, что для любого $a \in A$ выполнено $a1 = 1a = a$
		\item \textit{Коммутативной}, если для любых $a, b \in A$ выполнено $ab = ba$
	\end{itemize}
\end{definition}

\begin{definition}
	\textit{Алгеброй Ли} над полем $K$ называется линейное пространство $\mathfrak g$, на котором определена билинейная операция \textit{коммутирования} $[\cdot,\cdot] : \mathfrak g^2 \to \mathfrak g$, удовлетворяющая следующим условиям:
	\begin{enumerate}
		\item Для любых $x, y \in \mathfrak g$ выполнено $[x, y] = -[y, x]$
		\item Для любых $x, y, z \in \mathfrak g$ выполнено $[x, [y, z]] + [y, [z, x]] + [z, [x, y]] = 0$
	\end{enumerate}
\end{definition}

\begin{note}
	Зафиксируем базис $(e_1, \dotsc, e_n)$ в алгебре Ли $\mathfrak g$. Если $x, y \in \mathfrak g$ имеют координаты $(x_1, \dotsc, x_n)^T$ и $(y_1, \dotsc, y_n)^T$, то их коммутатор $[x, y]$ можно вычислить по следующей формуле:
	\[[x, y] = \left[\sum_{i=1}^nx_ie_i, \sum_{j=1}^ny_je_j\right] = \sum_{i=1}^n\sum_{j=1}^nx_iy_j[e_i, e_j]\]
\end{note}

\begin{proposition}
	Пусть $A$ "--- ассоциативная алгебра. Определим операцию коммутирования следующим образом:
	\[[x, y] := xy - yx,~x, y \in A\]
	
	Тогда с данной операцией алгебра $A$ образует алгебру Ли.
\end{proposition}

\begin{proof}
	Первое свойство коммутирования тривиально, докажем второе. Зафиксируем $x, y, z \in A$, тогда:
	\[[x, [y, z]] + [y, [z, x]] + [z, [x, y]] = [x, yz - zy] + [y, zx - xz] + [z, xy - yx] = 0\]
	
	Последнее равенство выше получается непосредственной проверкой при раскрытии всех коммутаторов по определению.
\end{proof}

\begin{definition}
	\textit{Представлением ассоциативной алгебры $A$ в пространстве $V$} называется гомоморфизм алгебр $\phi : A \to \mc L(V)$. \textit{Представлением алгебры Ли $\mathfrak g$ в пространстве $V$} называется гомоморфизм алгебр Ли $\phi : \mathfrak{g} \to \mathfrak{gl}(V)$, где $\mathfrak{gl}(V)$ "--- алгебра Ли, построенная по алгебре $\mathcal L(V)$.
\end{definition}

\begin{proposition}
	Пусть поле $K$ алгебраически замкнуто, группа $G$ "--- абелева, и представление группы $G$ в пространстве $V$ неприводимо. Тогда $\dim {V} = 1$.
\end{proposition}

\begin{proof}
	Рассмотрим произвольный элемент $g \in G$. В силу алгебраической замкнутости, у $g$ как линейного оператора на $V$ есть собственное значение $\lambda \in K^*$. Тогда для любого элемента $h \in G$ и собственного вектора $x \in V_{\lambda}$ выполнено следующее:
	\[g(hx) = h(gx) = h(\lambda x) = \lambda hx\]
	
	Значит, $hx \in V_{\lambda}$, то есть собственное подпространство $V_{\lambda} \le V$ инвариантно относительно представления. Но тогда, в силу неприводимости представления, $V = V_{\lambda}$, то есть $g = \lambda E$. Следовательно, все элементы группы $G$ действуют на $V$ как скалярные операторы, тогда любое подпространство в $V$ инвариантно относительно представления, поэтому $\dim{V} = 1$.
\end{proof}

\begin{note}
	Утверждение выше имеет обобщение, для доказательства которого потребуется один вспомогательный факт.
\end{note}

\begin{proposition}
	Пусть оператор $A \in \mc L(V)$ диагонализуем, подпространство $U \le V$ инвариантно относительно $A$. Тогда оператор $A|_U \in \mathcal L(U)$ тоже диагонализуем.
\end{proposition}

\begin{proof}
	По условию, пространство $V$ представимо в виде $V = V_1 \oplus \dotsb \oplus V_k$, где $V_1, \dotsc, V_k \le V$ "--- собственные подпространства оператора $A$. Индукцией по $m \in \{1, \dotsc, k\}$ легко убедиться, что если для собственных векторов $v_1 \in V_1, \dotsc, v_m \in V_m$ выполнено $v_1 + \dotsb + v_m \in U$, то $v_1, \dotsc, v_m \in U$. Тогда:
	\[\bigoplus_{i = 1}^k(V_i \cap U) = \left(\bigoplus_{i = 1}^kV_i\right) \cap U = V \cap U = U\]
	
	Равенство выше в точности означает, что оператор $A|_U \in \mathcal L(U)$ диагонализуем.
\end{proof}

\begin{theorem}
	Пусть $\mc A \subset \mc L(V)$ "--- семейство коммутирующих операторов. Тогда:
	\begin{enumerate}
		\item Если поле $K$ алгебраически замкнуто, то все операторы из $\mc A$ имеют общий собственный вектор
		\item Если все операторы из $\mc A$ диагонализуемы, то они имеют общий базис из собственных векторов
	\end{enumerate}
\end{theorem}

\begin{proof}
	Проведем индукцию по размерности пространства. Если все операторы из $\mc A$ скалярны, то оба утверждения тривиальны. Пусть теперь это не так. Выберем нескалярный оператор $A \in \mc A$, тогда:
	\begin{enumerate}
		\item В силу алгебраической заменутости, можно рассмотреть $\lambda \in K$ "--- собственное значение оператора $A$. По уже доказанному, его собственное подпространство $V_\lambda \le V$ является $\mc A$-инвариантным, причем $\dim{V_\lambda} < \dim{V}$, поэтому к нему применимо предположение индукции.
		\item По условию, пространство $V$ можно представить в виде $V = V_1 \oplus \dotsb \oplus V_k$, где $V_1, \dotsc, V_k \le V$ "--- собственные подпространства оператора $A$. Все эти подпространства являются $\mc A$-инвариантными и имеют размерность меньшую, чем $\dim{V}$, поэтому к ним применимо предположение индукции.\qedhere
	\end{enumerate}
\end{proof}