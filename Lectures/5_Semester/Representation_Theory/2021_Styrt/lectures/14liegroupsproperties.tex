\section{Свойства групп Ли}

\begin{note}
	В данном разделе зафиксируем группу Ли $G$, имеющую размерность $n$ как многообразие, и ее локальную карту $(U, \phi)$ такую, что $e \in U$. Гомеоморфизм $\phi$ позволяет отождествить множество $U$ с $\R^n$. Будем также считать, что $e = \phi(0)$, этого всегда можно добиться сдвигом системы координат. Наконец, для любых элементов $x, y \in U \equiv \R^n$ положим $z := (x, y)^T \in \R^{2n}$.
\end{note}

\begin{proposition}
	Для любых $x, y \in \R^n$ выполнено $\mu(x, y) = x + y + o(\|z\|)$, $z \to 0$.
\end{proposition}

\begin{proof}
	Функция $\mu$ дифференцируема, тогда по определению существуют матрицы $A, B \in M_n(\R)$ такие, что для любых $x, y \in \R^n$ выполнено $\mu(x, y) = Ax + By + o(\|z\|)$, $z \to 0$. При этом $A = B = E$, поскольку $\mu(x, 0) = \mu (0, x) = x$ для любого $x \in \R^n$.
\end{proof}

\begin{note}
	В дальнейшем мы часто будет производить рассуждения, аналогичные доказательству выше. Пусть $(E - A)x = o(\|x\|)$, $\|x\| \to 0$, для любого $x \in \R^n$. Линейный оператор $E - A$ однозначно задается своими значениями на векторах единичной нормы. Зафиксируем произвольный $x \in \R^n$ такой, что $\|x\| = 1$, тогда для любого $t > 0$ имеем:
	\[t(E - A)x = o(t) \ra (E - A)x = o(1),~t \to 0\]
	
	Значит, $(E - A)x = 0$. Тогда, в силу произвольности выбора вектора $x$ единичной нормы, получим, что $E - A = 0$, то есть $A = E$.
\end{note}

\begin{definition}
	Для любых $x, y \in \R^n$ определим следующие функции:
	\begin{itemize}
		\item $\fr(x, y) := xy^{-1}$ "--- \textit{частное}
		\item $\cm(x, y) := xyx^{-1}y^{-1}$ "--- \textit{коммутатор}
	\end{itemize}
\end{definition}

\begin{proposition}
	Для любых $x, y \in \R^n$ выполнено $\fr(x, y) = x - y + o(\|z\|)$, $z \to 0$.
\end{proposition}

\begin{proof}
	Функция $\fr$ дифференцируема, тогда по определению существуют матрицы $A, B \in M_n(\R)$ такие, что для любых $x, y \in \R^n$ выполнено $\fr(x, y) = Ax + By + o(\|z\|)$, $z \to 0$. Аналогично предыдущему утверждению, $A = E$ и $B = -E$, поскольку $\fr(x, 0) = x$ и $\fr(x, x) = 0$ для любого $x \in \R^n$.
\end{proof}

\begin{corollary}
	Для любого $x \in \R^n$ выполнено $\tau(x) = -x + o(\|x\|)$, $x \to 0$.
\end{corollary}

\begin{theorem}
	Существует такая кососимметрическая билинейная форма $\gamma$ на $\R^n$, что для любых $x, y \in \R^n$ выполнено $\cm(x, y) = \gamma(x, y) + o(\|z\|^2)$, $z \to 0$
\end{theorem}

\begin{proof}
	Заметим, что $\cm(x, 0) = \cm(0, x) = 0$ для любого $x \in \R^n$, поэтому для любых $i_1, \dotsc, i_k \in \{1, \dotsc, n\}$ выполнено следующее:
	\[\frac{\partial^k \cm}{\partial x_{i_1} \!\dotsb \partial x_{i_k}}(0, 0) = \frac{\partial^k \cm}{\partial y_{i_1} \!\dotsb \partial y_{i_k}}(0, 0) = 0\]
	
	Тогда для любых $x, y \in \R^n$ выполнено $\cm(x, y) = \gamma(x, y) + o(\|z\|^2)$, $z \to 0$, где $\gamma$ "--- некоторая билинейная форма на $\R^n$. Остается проверить ее кососимметричность. Действительно, для любого $x \in \R^n$ выполнено $\cm(x, x) = 0$, то есть $\gamma(x, x) + o(\|x\|^2) = 0$, откуда $\gamma(x, x) = 0$. Это и означает, что форма $\gamma$ кососимметрична.
\end{proof}

\begin{definition}
	\textit{Касательной алгеброй группы Ли} $G$ в точке $e$ называется алгебра, образованная пространством $\R^n$ с умножением, определенным как $[x, y] := \gamma(x, y)$ для любых $x, y \in \R^n$, где $\gamma$ "--- кососимметричная билинейная функция, сосответствующая коммутатору $\cm$. Обозначение "--- $T{G}$.
\end{definition}

\begin{note}
	Мы пока не проверили, что умножение на $T{G}$ определено корректно, то есть не зависит от выбора локальной карты, содержащей $e$. Пусть $(U_\alpha, \phi_\alpha)$, $(U_\beta, \phi_\beta)$ "--- две такие карты. \pagebreak На первой карте имеет место следующее равенство:
	\[\cm_\alpha(x, y) = \gamma_\alpha(x, y) + o(\|x\|^2 + \|y\|^2),~x, y \in \R^n\]
	
	Положим $g := \phi_\beta^{-1} \circ \phi_\alpha$, $u := g(x)$, $v := g(y)$. Тогда:
	\[\cm_\beta(u, v) = \cm_\alpha(g^{-1}(u), g^{-1}(v)) = \gamma_\alpha(g^{-1}(u), g^{-1}(v)) + o(\|u\|^2 + \|v\|^2),~u, v \in \R^n\]
	
	В равенстве выше $o(\|u\|^2 + \|v\|^2) = o(\|x\|^2 + \|y\|^2)$, поскольку отображение $g$ "--- гладкое и $g(0) = 0$. Положим теперь $A := d_0g$, тогда $d_0(g^{-1}) = A^{-1}$, и, следовательно:
	\begin{gather*}
		g^{-1}(u) = A^{-1}u + o(\|u\|^2 + \|v\|^2) \\
		g^{-1}(v) = A^{-1}v + o(\|u\|^2 + \|v\|^2)
	\end{gather*}
	
	Тогда, в силу билинейности функции $\gamma$, получим:
	\[\cm_\beta(u, v) = \gamma_\alpha(g^{-1}(u), g^{-1}(v)) + o(\|u\|^2 + \|v\|^2) = \gamma_\alpha(A^{-1}u, A^{-1}v) + o(\|u\|^2 + \|v\|^2)\]
	
	Функция $\gamma_\beta(u, v) := \gamma_\alpha(A^{-1}u, A^{-1}v)$ тоже билинейна и кососимметрична. Таким образом, касательные пространства $TG_\alpha$ и $TG_\beta$, полученные по разным локальным картам, изоморфны, причем этот изоморфизм осуществляется отображением $x \mapsto Ax$.
\end{note}

\begin{proposition}
	Касательная алгебра группы Ли $G$ антикоммутативна, то есть для любых $x, y \in TG$ выполнено $[x, y] = -[y, x]$.
\end{proposition}

\begin{proof}
	Требуемое верно в силу кососимметричности формы $\gamma$.
\end{proof}

\begin{definition}
	\textit{Гомоморфизмом групп Ли} $G$ и $H$ называется гомоморфизм групп $\phi: G \to H$, являющийся гладким отображением этих множеств как гладких многообразий.
\end{definition}

\begin{theorem}
	Пусть $\phi : G \to H$ "--- гомоморфизм групп Ли. Тогда дифференциал гомоморфизма $d_e\phi: TG \to TH$ является гомоморфизмом касательных алгебр групп Ли.
\end{theorem}

\begin{proof}
	Обозначим через $\cm_G$, $\cm_H$ коммутаторы в $G$ и $H$ соответственно. Заметим, что для любых $x, y \in G$ выполнено $\phi(\cm_G(x, y)) = \cm_H(\phi(x), \phi(y))$. Далее мы снова будем отождествлять окрестности нейтральных элементов из $G$ и $H$ с пространствами $\R^{\dim G}$ и $\R^{\dim H}$. Требуется проверить, что для отображения $A := d_e\phi$ при любых $x, y \in TG$ выполнено равенство $A[x, y]_G = [Ax, Ay]_H$. Нам потребуются следующие соотношения:
	\begin{gather*}
		\cm_G(x, y) = [x, y]_G + o(\|z\|^2),~z \to 0
		\\
		\cm_H(u, v) = [u, v]_H + o(\|u\|^2 + \|v\|^2),~(u, v) \to 0
	\end{gather*}
	
	Кроме того, поскольку отображение $A$ линейно, то справедливо следующее:
	\begin{gather*}
		\|z\|^2 = O(\|Ax\|^2 + \|Ay\|^2),~z \to 0
		\\
		\|Ax\|^2 + \|Ay\|^2 = O(\|z\|^2),~z \to 0
	\end{gather*}

	Преобразуем выражение $\phi(\cm_G(x, y))$:
	\[\phi(\cm_G(x, y)) = A(\cm_G(x, y)) + o(\|\cm_G(x, y)\|) = A[x, y]_G + o(\|z\|^2)\]
	
	Теперь преобразуем выражение $\cm_H(\phi(x), \phi(y))$:
	\[\cm_H(\phi(x), \phi(y)) = [Ax + o(\|z\|), Ay + o(\|z\|)]_H + o(\|Ax\|^2 + \|Ay\|^2) = [Ax, Ay]_H + o(\|z\|^2)\]
	
	Таким образом, получено равенство:
	\[[Ax, Ay]_H = A[x, y]_G + o(\|z\|^2)\]
	
	В силу билинейности обеих частей равенства, от $o(\|z\|^2)$ можно избавиться аналогично уже доказанному.
\end{proof}