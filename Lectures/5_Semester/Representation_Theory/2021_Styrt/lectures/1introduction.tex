\section{Введение в теорию представлений}

\textbf{Здесь и далее} под пространствами будут пониматься конечномерные линейные пространства над некоторым основным полем $K$. Иногда на поле $K$ будут налагаться ограничения, что будет оговорено отдельно.

\begin{definition}
	\textit{Представлением группы $G$ в пространстве $V$} называется действие группы $G$ на множестве $V$ такое, что каждый ее элемент действует на $V$ как линейный оператор, то есть для любых $g \in G$,  $x, y \in V$ и $\alpha, \beta \in K$ выполнено $g(\alpha x + \beta y) = \alpha gx + \beta gy$.
\end{definition}

\begin{note}
	Напомним, что, по определению, \textit{действием} группы на линейном пространстве называется сопоставление каждой паре $(g, x)$, $g \in G$, $x \in V$, элемента $gx \in V$, удовлетворяющее двум условиям:
	\begin{enumerate}
		\item Для любых $g, h \in G$ и $x \in V$ выполнено $(gh)x = g(hx)$
		\item Для любого $x \in V$ выполнено $ex = x$
	\end{enumerate}
\end{note}

В случае представления к условиям выше добавляется требование линейности сопоставления по элементам из $V$. Кроме того, имеет место эквивалентное определение, согласно которому представлением называется гомоморфизм $R : G \to \GL(V)$.

\begin{example}
	Рассмотрим несколько примеров представлений:
	\begin{enumerate}
		\item \textit{Скалярным оператором} на $V$ называется оператор, в любом базисе имеющий матрицу вида $\lambda E$, где $\lambda \in K$. Имеет место естественное вложение $K^* \emb \GL(V)$, при котором каждому скаляру $\lambda \in K^*$ сопоставляется скалярный оператор с матрицей $\lambda E$. Пусть задан гомоморфизм $\phi: G \to K^* \emb \GL(V)$. Тогда по нему можно построить представление следующего вида:
		\[gx := \phi(g)x,~g \in G, x \in V\]
		
		\textit{Тождественным представлением} называется представление, построенное описанным выше образом по тривиальному гомоморфизму $\phi : G \to \{e\}$ и имеющее следующий вид:
		\[gx := x,~g \in G, x \in V\]
		
		\item Пусть $G \le \GL(V)$, тогда $G \xhookrightarrow[]{\id} \GL(V)$. \textit{Тавтологическим представлением} называется представление следующего вида:
		\[gx := g(x),~g \in G, x \in V\]
		
		\item Пусть $G$ "--- конечная группа, $\dim{V} = |G|$ и $V = \gl e_g\gr_{g \in G}$, тогда система $(e_g)_{g \in G}$ образует базис в $V$. \textit{Регулярным представлением} называется представление, заданное на базисных векторах пространства $V$ следующим образом:
		\[ge_h := e_{gh},~g, h \in G\]
		
		Линейность сопоставления достигается по построению, а условия из определения действия группы, очевидно, выполнены. Обозначим через $x_g$ координату вектора $x \in V$, соответствующую базисному вектору $e_g$, и заметим следующее:
		\[x = \sum_{h \in G}x_he_h \stackrel{g}{\mapsto} gx = \sum_{h \in G}x_he_{gh} = \sum_{\widetilde h \in G}x_{g^{-1}\widetilde h}e_{\widetilde h}\]
		
		Значит, координатные записи векторов при таком представлении преобразуются следующим образом:
		\[(gx)_h = x_{g^{-1}h},~x \in V, g, h \in G\]
		
		\item Пусть $G = S_n$ для некоторого $n \in \N$, $\dim{V} = n$ и $V = \gl e_1, \dotsc, e_n\gr$, тогда система $(e_1, \dotsc, e_n)$ образует базис в $V$. Построим представление, заданное на базисных векторах пространства $V$ следующим образом:
		\[ge_k := e_{g(k)},~g \in G, k \in \{1, \dotsc, n\}\]
		
		\item Пусть $G$ действует на некотором конечном множестве $M$, $\dim{V} = |M|$ и $V = \gl e_x \gr_{x \in M}$. Построим представление, заданное на базисных векторах пространства $V$ следующим образом:
		\[ge_x := e_{gx},~g \in G, x \in M\]
	\end{enumerate}
\end{example}

\begin{definition}
	Пусть $R$ "--- представление группы $G$ в $V$. Подпространство $U \le V$ называется \textit{инвариантным относительно представления $R$}, если для каждого $g \in G$ выполнено $gU \le U$.
\end{definition}

\begin{note}
	Поскольку каждый элемент группы $G$ действует на $V$ как невырожденный линейный оператор, то включение $gU \subset U$ в случае конечномерного пространства $V$ можно заменить на равенство $gU = U$. Отметим также, что если $U_1, U_2 \le V$ инвариантны относительно $R$, то $U_1 \cap U_2$ и $U_1 + U_2$ --- тоже.
\end{note}

\begin{definition}
	Пусть $R$ "--- представление группы $G$ в $V$, $U \le V$ "--- инвариантное относительно $R$ подпространство.
	\begin{itemize}
		\item \textit{Подпредставлением} представления $R$ называется сужение представления $R$ на $U$.
		\item \textit{Факторпредставлением} представления $R$ по подпространству $U$ называется представление $R / U$ такое, что для любых $g \in G$ и $x \in V$ выполнено $g(v + U) := gv + U$.
	\end{itemize}
\end{definition}

\begin{note}
	В определении факторпредставления, вообще говоря, требуется проверить корректность, но она выполнена в силу инвариантности пространства $U$. Действительно, если $x_1 + U = x_2 + U$ для некоторых $x_1, x_2 \in V$, то $x_1 - x_2 \in U$, тогда:
	\[g(x_1 + U) = gx_1 + U = gx_2 + g(x_1 - x_2) + U = gx_2 + U = g(x_2 + U)\]
	
	Отметим также, что представление $R : G \to \GL(V)$ естественным образом порождает тавтологическое представление группы $R(G) \le \GL(V)$ в $V$.
\end{note}

\begin{definition}
	Пусть $\mathcal A \subset \mathcal{L}(V)$, причем необязательно $\mc A \subset \GL{(V)}$. Подпространство $U \le V$ называется \textit{$\mc A$-инвариантным}, если для каждого $A \in \mc A$ выполнено $AU \subset U$. $\mc A$-инвариантное подпространство $U \le V$ называется \textit{неприводимым}, если оно нетривиально и не существует $\mc A$-инвариантного подпространства $W$ такого, что $\{0\} < W < U$, и \textit{приводимым} в противном случае.
\end{definition}

\begin{definition}
	Представление $R$ группы $G$ в $V$ называется \textit{неприводимым}, если все пространство $V$ неприводимо относительно $R(G)$, то есть в $V$ нет нетривиальных инвариантных относительно $R$ подпространств.
\end{definition}

\begin{definition}
	Пусть $\mathcal A \subset \mathcal{L}(V)$. Пространство $V$ называется \textit{вполне приводимым} относительно $\mc A$, если выполнено одно из следующих условий:
	\begin{enumerate}
		\item Существует система неприводимых относительно $\mc A$ подпространств $\{U_i\}_{i \in I}$ такая, что $A = \bigoplus_{i = I}U_i$
		\item Для любого $\mc A$-инвариантного подпространства $U \le V$ существует $\mc A$-инвариантное подпространство $W \le V$ такое, что $V = U \oplus W$
	\end{enumerate}
\end{definition}

\begin{theorem}
	Условия в определении полной приводимости эквивалентны.
\end{theorem}

\begin{proof}~
	\begin{itemize}
		\item\imp{2}{1}Рассмотрим всевозможные линейно независимые наборы неприводимых относительно $\mathcal A$ подпространств в $V$. Таким набором, например, является пустой набор подпространств, и в каждом таком наборе не более $\dim{V}$ подпространств. Выберем набор $\{U_i\}_{i \in I}$ максимального размера, и рассмотрим $\mc A$-инвариантное $W \le V$ такое, что:
		\[\left(\bigoplus_{i \in I}U_i\right) \oplus W = V\]
		
		Если $W = \{0\}$, то получено требуемое. Если $\dim W \ge 1$, то выберем в $W$ $\mc A$-инвариан-тное подпространство $W'$ минимальной размерности. Тогда набор $\{U_i\}_{i \in I} \cap \{W'\}$ тоже линейно независим и имеет размер $|I| + 1  > |I|$ --- противоречие.
		
		\item\imp{1}{2}Пусть $V = \bigoplus_{i \in I}U_i$, где все подпространства из $\{U_i\}_{i \in I}$ "--- $\mc A$-инвариантные. Зафиксируем произвольное $\mc A$-инвариантное подпространство $U \le V$. Рассмотрим всевозможные множества индексов $I' \subset I$ такие, что:
		\[\left(\bigoplus_{i \in I'}U_i\right) \cap U = \{0\}\]
		
		Таким множеством индексов, например, является пустое множество. Выберем множество $I'' \subset I$ максимального размера и положим $W := \bigoplus_{i \in I''}U_i$. Остается проверить, что $U \oplus W = V$, то есть для любого $i \in I$ выполнено $U_i \subset (U \oplus W)$. Действительно, в силу неприводимости подпространства $U_i$, либо $U_i \cap (U \oplus W) = U_i$, что и требовалось, либо $U_i \cap (U \oplus W) = \{0\}$. Но если для некоторого $i \in I$ выполнено $U_i \cap (U \oplus W) = \{0\}$, то индекс $i$ можно добавить к набору $I''$ --- противоречие.\qedhere
	\end{itemize}
\end{proof}

\begin{problem}
	Найдите все инвариантные подпространства стандартного представления группы $S_n$ в пространстве $K^n$, где $\cha{K} \nmid n$.
\end{problem}

\begin{solution}
	Относительно представления выше инвариантны следующие подпространства:
	\begin{align*}
		U &:= \{(x_1, \dotsc, x_n)^T \in K^n : x_1 = \dotsb = x_n\} \le K^n
		\\
		W &:= \{(x_1, \dotsc, x_n)^T \in K^n : x_1 + \dotsb + x_n = 0\} \le K^n
	\end{align*}
	
	При $\cha{K} \nmid n$ выполнено $U \cap W = \{0\}$, откуда $K^n = U \oplus W$ из соображений размерности. Покажем, что других нетривиальных инвариантных подпространств в $K^n$ нет. Рассмотрим $\{0\} < U' \le K^n$ "--- инвариантное подпространство такое, что $U' \cap U = \{0\}$, и выберем вектор $x = (x_1, \dotsc, x_n)^T \in U'$, хотя бы две координаты которого различны. Пусть без ограничения общности $x_1 \ne x_n$, тогда, в силу инвариантности подпространства $U'$, $x - (1n)x \in U'$, откуда $y := (1, 0, \dotsc, 0, -1)^T \in U'$. Переставляя координаты вектора $y$, получим, что $(1, 0, \dotsc, 0, -1)^T, \dotsc, (0, \dotsc, 0, 1, -1)^T \in U'$, поэтому $W \le U'$. Следовательно, $W \le U'$. Значит, $U' = W$, поскольку иначе $U' = K^n \ge U$.
	
	Таким образом, подпространства $U, W \le V$ инвариантны, причем других нетривиальных инвариантных подпространств в $V$ нет, поэтому $U, W$ "--- неприводимые. Но поскольку $V = U \oplus W$, то $V$ вполне приводимо.
\end{solution}