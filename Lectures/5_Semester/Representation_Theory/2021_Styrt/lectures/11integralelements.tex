\section{Целые элементы над кольцами}

\begin{definition}
	Пусть $A$ и $B \supset A$ "--- коммутативные ассоциативные кольца с единицей, причем кольцо $A$ "--- целостное. Элемент $x \in B$ называется \textit{целым} над $A$, если существует многочлен $f \in A[t]$ со старшим коэффициентом $1$ такой, что $f(x) = 0$.
\end{definition}

\begin{theorem}
	Элемент $x \in B$ является целым над $A$ $\lra$ расширение $A[x]$ образует конечнопорожденный модуль над $A$.
\end{theorem}

\begin{proof}~
	\begin{enumerate}
		\item[$\ra$] По условию, для некоторого $n \in \N$ и элементов $a_0, \dotsc, a_{n - 1} \in A$ выполнено равенство $x^n + a_{n-1}x^{n-1} + \dotsb + a_0 = 0$. Тогда, по индукции, любой элемент вида $x^{n + m}$ при $m \in \N \cup \{0\}$ линейно выражается через $1, x, \dotsc, x^{n- 1}$, поэтому $A[x] = \gl1, x, \dotsc, x^{n- 1}\gr$.
		
		\item[$\la$] Пусть $A[x] = \gl p_1(x), \dotsc, p_m(x)\gr$ для некоторых многочленов $p_1, \dotsc, p_m \in A[t]$. Положим $N := \max\{\deg p_1, \dotsc, \deg p_m\} + 1$, тогда для некоторых элементов $a_1, \dotsc, a_m \in A$ выполнено равенство $x^N = a_1p_1(x) + \dotsb + a_mp_m(x)$. Это и дает многочлен со старшим коэффициентом $1$, корнем которого является $x$. \qedhere
	\end{enumerate}
\end{proof}

\begin{corollary}
	Элемент $x \in B$ является целым над $A$ $\lra$ существует конечнопорожденный модуль $A \subset C \subset B$, содержащий $x$.
\end{corollary}

\begin{proof}~
	\begin{itemize}
		\item[$\ra$] Таким конечнопорожденным модулем является расширение $A[x]$.
		\item[$\la$] Рассмотрим линейное отображение $\phi \in \mc L(C)$ вида $a \mapsto xa$. К его характеристическому многочлену $f$ над полем частных $\quot{A}$ применима теорема Гамильтона-Кэли, тогда $f(\phi) = 0$. В частности, $(f(\phi))(1) = f(x) = 0$, и многочлен $f$ имеет старший коэффициент $(-1)^n$.\qedhere
	\end{itemize}
\end{proof}

\begin{corollary}
	Целые над $A$ элементы из $B$ образуют подкольцо в $B$.
\end{corollary}

\begin{proof}
	Пусть элементы $x, y \in B$ "--- целые над $A$, тогда расширения $A[x]$ и $A[y]$ образуют конечнопорожденные модули над $A$. Но тогда и расширение $A[x, y]$ образует конечнопорожденный модуль над $A$.
\end{proof}

\begin{proposition}
	Пусть $A$ "--- факториальное кольцо. Тогда элемент $x \in \quot{A}$ является целым над $A$ $\lra$ $x \in A$.
\end{proposition}

\begin{proof}
	Нетривиально только доказательство $(\ra)$. Пусть элемент $x \in \quot{A}$ является целым над $A$, тогда существуют $a_0, \dotsc, a_{n-1} \in A$ такие, что выполнено равенство $x^n + a_{n-1}x^{n-1} + \dotsb + a_0 = 0$. В силу факториальности кольца, $x$ можно представить в виде $x = \frac pq$, где $p, q \in A$ "--- взаимно простые. Поскольку $q \ne 0$, то $p^n + a_{n-1}{p^{n- 1}}q + \dotsb + a_0q^n = 0$. Значит, $q \mid p$, откуда $x \in A$.
\end{proof}

\begin{proposition}
	Пусть $G$ "--- конечная группа, поле $K$ алгебраически замкнуто, причем $\cha{K} = 0$, $C \subset G$ "--- класс сопряженности в $G$, и представление группы $G$ в пространстве $V$ неприводимо. Тогда элемент $\frac1{\dim{V}}\sum_{g \in C}\Chi_V(g)$ "--- целый над $\Z$.
\end{proposition}

\begin{proof}
	Рассмотрим элемент $a := \sum_{g \in C}e_g \in K[G]$ и покажем, что он является центральным. Для этого достаточно проверить, что он коммутирует с базисными элементами вида $e_h$, где $h \in G$:
	\[ae_h = \sum_{g \in C}e_ge_h = \sum_{g \in C}e_{gh} = \sum_{\widetilde g \in C}e_{h\widetilde g} = \sum_{\widetilde g \in C}e_{h}e_{\widetilde g} = e_ha\]
	
	Поскольку имеет место естественный гомоморфизм алгебр $K[G]$ и $\mc L(V)$, заданный на базисе в $K[G]$ как $e_g \mapsto g$, то оператор $A := \sum_{g \in C}{g}$ является $G$-эквивариантным. Тогда, по лемме Шура, $A = \lambda E$ для некоторого $\lambda \in K$, откуда:
	\[\lambda\dim{V} = \tr{A} = \sum_{g \in C}\Chi_V(g) \ra \lambda = \frac1{\dim{V}}\sum_{g \in C}\Chi_V(g)\]
	
	Поскольку $\cha{K} = 0$, то $K[G]$ содержит $\Z[G]$, причем $\Z[G]$ "--- конечнопорожденный модуль над $\Z$. Можно показать, что тогда его подмодуль $\Z[a] \subset \Z[G]$ тоже является конечнопорожденным над $\Z$. Но кольцо $\Z[a]$ коммутативно, поэтому элемент $a$ является целым над $\Z$, то есть для некоторого многочлена $f \in \Z[t]$ со старшим коэффициентом $1$ выполнено $f(a) = 1$. Тогда $f(A) = f(\lambda)E = 0$, откуда $f(\lambda) = 0$, то есть $\lambda$ "--- тоже целый элемент.
\end{proof}

\begin{note}
	Если в утверждении выше $C = h^G$ для некоторого $h \in G$, то элемент из условия можно переписать в следующем виде:
	\[\frac1{\dim{V}}\sum_{h^g \in h^G}\Chi_V(h^g) = \frac{|h^G|\Chi_V(h)}{\dim V}\]
\end{note}

\begin{theorem}
	Пусть $G$ "--- конечная группа, поле $K$ алгебраически замкнуто, $\cha{K} = 0$, и представление группы $G$ в пространстве $V$ неприводимо. Тогда $\dim{V} \mid |G|$.
\end{theorem}

\begin{proof}
	Как уже было доказано, $(\Chi_V, \Chi_V) = 1$, то есть:
	\[1 = \frac1{|G|}\sum_{g \in G}\Chi_V(g)\Chi_V(g^{-1}) \ra |G| = \sum_{g \in G}\Chi_V(g)\Chi_V(g^{-1})\]
	
	Сгруппируем слагаемые в сумме выше по классам сопряженности:
	\[\sum_{h^G \subset G}|h^G|\Chi_V(h)\Chi_V(h^{-1}) = |G| \ra \frac{|G|}{\dim{V}} = \sum_{h^G\subset G}\Chi_V(h^{-1})\frac{|h^G|\Chi_V(h)}{\dim{V}}\]
	
	Элемент $\Chi_V(h^{-1}) = \tr{(h^{-1})}$ "--- целый над $\Z$, тогда и дробь $\frac{|G|}{\dim{V}}$ является целым элементом над $\Z$, поэтому она является целым числом, то есть $\dim{V} \mid |G|$, что и требовалось.
\end{proof}