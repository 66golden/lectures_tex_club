\section{Следствия из теоремы Колмогорова}

Перепишем условия симметрии и согласованности в терминах характеристических функций.
\begin{theorem}
(Условия симметрии и согласованности для характеристических функций, б/д). Пусть $\displaystyle T$ -- непустое множество, $\displaystyle \forall n\in \mathbb{N} \ \forall t_{1} ,\ \dotsc ,\ t_{n}$ задана вероятностная мера $\displaystyle P_{t_{1} ,\ \dotsc ,\ t_{n}}$ на $\displaystyle \left(\mathbb{R}^{n} ,\ \mathcal{B}\left(\mathbb{R}^{n}\right)\right)$. Пусть также $\displaystyle \phi _{t_{1} ,\ \dotsc ,\ t_{n}}$ -- характеристическая функция распределения $\displaystyle P_{t_{1} ,\ \dotsc ,\ t_{n}}$. Тогда меры $\displaystyle P_{t_{1} ,\ \dotsc ,\ t_{n}}$ удовлетворяют условиям симметрии и согласованности тогда и только тогда, когда характеристические функции $\displaystyle \phi _{t_{1} ,\ \dotsc ,\ t_{n}}$ удовлетворяют условиям симметрии и согласованности, т.е. выполняется

\begin{enumerate}
    \item $\displaystyle \phi _{t_{1} ,\ \dotsc ,\ t_{n}}( \lambda _{1} ,\ \dotsc ,\ \lambda _{n}) =\phi _{t_{\sigma ( 1)} ,\ \dotsc ,\ t_{\sigma ( 2)}}( \lambda _{\sigma ( 1)} ,\ \dotsc ,\ \lambda _{\sigma ( n)}) ,\ \forall \sigma \in S_{n}$.
    \item $\displaystyle \phi _{t_{1} ,\ \dotsc ,\ t_{n-1} ,\ t_{n}}( \lambda _{1} ,\ \dotsc ,\ \lambda _{n-1} ,\ 0) =\phi _{t_{1} ,\ \dotsc ,\ t_{n-1}}( \lambda _{1} ,\ \dotsc ,\ \lambda _{n-1})$.
\end{enumerate}
\end{theorem}
\begin{corollary}
Пусть $\displaystyle T\subset \mathbb{R} ,\ \forall n\in \mathbb{N} ,\ \forall t_{1} ,\ \dotsc ,\ t_{n} \in T:\ t_{1} < \dotsc < t_{n}$ задана вероятностная мера $\displaystyle P_{t_{1} ,\ \dotsc ,\ t_{n}}$ на $\displaystyle \left(\mathbb{R}^{n} ,\ \mathcal{B}\left(\mathbb{R}^{n}\right)\right)$ с характеристической функцией $\displaystyle \phi _{t_{1} ,\ \dotsc ,\ t_{n}}$. Если функции $\displaystyle \phi _{t_{1} ,\ \dotsc ,\ t_{n}}$ удовлетворяют условию

\begin{gather}
\forall m\in \{1,\ \dotsc ,\ n\} \hookrightarrow \phi _{t_{1} ,\ \dotsc ,\ t_{n}}( \lambda _{1} ,\ \dotsc ,\ \lambda _{n}) \big|_{\lambda _{m} =0} =\\
=\phi _{t_{1} ,\ \dotsc ,\ t_{m-1} ,\ t_{m+1} ,\ \dotsc ,\ t_{n}}( \lambda _{1} ,\ \dotsc ,\ \lambda _{m-1} ,\ \lambda _{m+1} ,\ \dotsc ,\ \lambda _{n}) , \notag
\end{gather}
то существует такое вероятностное пространство $\displaystyle ( \Omega ,\ \mathcal{F} ,\ P)$ и случайный процесс $\displaystyle ( X_{t} ,\ t\in T)$ на нем, что $\displaystyle P_{t_{1} ,\ \dotsc ,\ t_{n}}$ будет распределением вектора $\displaystyle ( X_{t_{1} ,\ \dotsc ,\ t_{n}})$.
\end{corollary}
\begin{proof}
TODO -- доказать, что выполняется первое условие.
\end{proof}
\section{Процессы с независимыми приращениями}
\begin{definition}
Случайный процесс $\displaystyle ( X_{t} ,\ t\geqslant 0)$ является процессом с независимыми приращениями, если $\displaystyle \forall n\in \mathbb{N} \ \forall t_{1} ,\ \dotsc ,\ t_{n}$ случайные величины $\displaystyle X_{t_{n}} -X_{t_{n-1}} ,\ \dotsc ,\ X_{t_{2}} -X_{t_{1}} ,\ X_{t_{1}}$ независимы в совокупности.
\end{definition}
\begin{note}
Случайный процесс с независимыми приращениями является непрерывным аналогом случайного блуждания.
\end{note}
\begin{theorem}
(О существовании процессов с независимыми приращениями). Пусть $\displaystyle \forall s,t:\ 0\leqslant s\leqslant t$ задана вероятностная мера $\displaystyle Q_{s,t}$ на $\displaystyle (\mathbb{R} ,\ \mathcal{B}(\mathbb{R}))$ с характеристической функцией $\displaystyle \phi _{s,t}$. Пусть также задана вероятностная мера $\displaystyle Q_{0}$ на $\displaystyle (\mathbb{R} ,\ \mathcal{B}(\mathbb{R}))$. Пусть также случайный процесс $\displaystyle ( X_{t} ,\ t\geqslant 0)$ является случайным процессом с независимыми приращениями и распределениями приращений
\begin{gather*}
X_{t} -X_{s}\xlongequal{d} Q_{s,t} ,\ 0\leqslant s< t,\\
X_{0}\xlongequal{d} Q_{0} .
\end{gather*}
Такой процесс существует тогда и только тогда, когда
\begin{equation}
\forall s,u,t:\ 0\leqslant s< u< t\hookrightarrow \phi _{s,t}( \tau ) =\phi _{s,u}( \tau ) \cdotp \phi _{u,s}( \tau ) .
\end{equation}
\end{theorem}
\begin{proof}
Пусть такой процесс существует. Тогда в силу независимости приращений $\displaystyle ( X_{t} ,\ t\geqslant 0)$ выполняется
\begin{gather*}
\phi _{s,t}( \tau ) =\phi _{X_{t} -X_{s}}( \tau ) =\phi _{X_{t} -X_{u} +X_{u} -X_{s}}( \tau ) =\phi _{X_{t} -X_{u}}( \tau ) \cdotp \phi _{X_{u} -X_{s}}( \tau ) =\\
=\phi _{s,u}( \tau ) \cdotp \phi _{u,t}( \tau ) .
\end{gather*}
Пусть выполняется условие (2), и предположим, что процесс существует. Пусть также$\displaystyle 0=t_{0} < t_{1} < \dotsc < t_{n}$. Рассмотрим вектор $\displaystyle \xi =( X_{t_{n}} ,\ X_{t_{n-1}} ,\ \dotsc ,\ X_{t_{1}} ,\ X_{t_{0}})$. Найдем его характеристическую функцию. Для этого рассмотрим вектор приращений $\displaystyle \xi '=( X_{t_{n}} -X_{t_{n-1}} ,\ \dotsc ,\ X_{t_{1}} -X_{t_{0}} ,\ X_{t_{0}})$. Компоненты $\displaystyle \xi '$ независимы, следовательно, характеристическая функция $\displaystyle \xi '$ имеет вид
\begin{gather*}
\phi _{\xi '}( \lambda _{n} ,\ \dotsc ,\ \lambda _{0}) =\phi _{X_{t_{n}} -X_{t_{n-1}}}( \lambda _{n}) \cdotp \ \dotsc \ \cdotp \phi _{X_{t_{1}} -X_{0}}( \lambda _{1}) \cdotp \phi _{X_{0}}( \lambda _{0}) =\\
=\phi _{t_{n-1} ,t_{n}}( \lambda _{n}) \cdotp \ \dotsc \ \cdotp \phi _{0,t_{1}}( \lambda _{1}) \cdotp \phi _{0}( \lambda _{0}) .
\end{gather*}
Далее, заметим, что
\begin{equation*}
\xi =A\xi ',\ A=\begin{pmatrix}
1 & 1 & \dotsc  & 1\\
0 & 1 & \dotsc  & 1\\
\dotsc  & \dotsc  & \dotsc  & \dotsc \\
0 & 0 & \dotsc  & 1
\end{pmatrix} .
\end{equation*}
Тогда
\begin{gather*}
\phi _{\xi }(\overline{\lambda }) =\mathbb{E} e^{i< \overline{\lambda } ,\ \xi > } =\mathbb{E} e^{i< \overline{\lambda } ,\ A\xi '> } =\mathbb{E} e^{i\left< A^{T}\overline{\lambda } ,\ \xi '\right> } =\phi _{\xi '}\left( A^{T}\overline{\lambda }\right) =\\
=\phi _{t_{n-1} ,t_{n}}( \lambda _{n}) \cdotp \phi _{t_{n-2} ,t_{n-1}}( \lambda _{n} +\lambda _{n-1}) \cdotp \ \dotsc \ \cdotp \phi _{0}( \lambda _{n} +\ \dotsc \ +\lambda _{0}) =:\phi _{0,t_{1} ,\ \dotsc ,\ t_{n}}( \lambda _{0} ,\ \dotsc ,\ \lambda _{n}) .
\end{gather*}
Положим $\displaystyle \phi _{t_{1} ,\ \dotsc ,\ t_{n}}( \lambda _{1} ,\ \dotsc ,\ \lambda _{n}) :=\phi _{0,t_{1} ,\ \dotsc ,\ t_{n}}( 0,\ \lambda _{1} ,\ \dotsc ,\ \lambda _{n})$.

Проверим, что выполняется условие (1) из следствия для характеристических функций $\displaystyle \phi _{t_{1} ,\ \dotsc ,\ t_{n}} ,\ 0\leqslant t_{1} < \dotsc < t_{n}$. Без ограничения общности проверим только ситуацию, когда есть нулевой момент времени $\displaystyle t_{0} =0$. Если $\displaystyle m=0$, то по определению $\displaystyle \phi _{t_{0} ,\ \dotsc ,\ t_{n}}( \lambda _{0} ,\ \dotsc ,\ \lambda _{n}) \big\rvert_{\lambda _{0} =0} =\phi _{t_{1} ,\ \dotsc ,\ t_{n}}( \lambda _{1} ,\ \dotsc ,\ \lambda _{n})$. Ели $\displaystyle m >0$, то по условию теоремы выполняется

\begin{gather*}
\phi _{0,\ t_{1} ,\ \dotsc ,\ t_{n}}( \lambda _{0} ,\ \lambda _{1} ,\ \dotsc ,\ \lambda _{n}) \big\rvert_{\lambda _{m} =0} =\\
=\phi _{t_{n-1} ,t_{n}}( \lambda _{n}) \cdotp \ \dotsc \ \cdotp \phi _{t_{m} ,t_{m+1}}( \lambda _{n} +\ \dotsc +\lambda _{m+1}) \cdotp \phi _{t_{m-1} ,t_{m}}( \lambda _{n} +\ \dotsc +\lambda _{m+1} +0) \cdotp \ \dotsc \ \cdotp \\
\cdotp \phi _{0,t_{1}}( \lambda _{n} +\ \dotsc \ +\lambda _{m+1} +\lambda _{m-1} +\ \dotsc \ +\lambda _{0}) =\\
=\phi _{t_{n-1} ,t_{n}}( \lambda _{n}) \cdotp \ \dotsc \ \cdotp \phi _{t_{m-1} ,t_{m+1}}( \lambda _{n} +\ \dotsc +\lambda _{m+1}) \cdotp \ \dotsc \ \cdotp \\
\cdotp \phi _{0,t_{1}}( \lambda _{n} +\ \dotsc \ +\lambda _{m+1} +\lambda _{m-1} +\ \dotsc \ +\lambda _{0}) =\\
=\phi _{0,t_{1} ,\ \dotsc ,t_{m-1} ,t_{m+1} ,\ \dotsc ,\ t_{n}}( \lambda _{0} ,\ \dotsc ,\ \lambda _{m-1} ,\ \lambda _{m+1} ,\ \dotsc ,\ \lambda _{n}) .
\end{gather*}
Таким образом, выполняется условие (1) следствия. Следовательно, по следствию существует вероятностное пространство и случайный процесс $\displaystyle ( X_{t} ,\ t\geqslant 0)$ на нем, что $\displaystyle \phi _{t_{1} ,\ \dotsc ,\ t_{n}}$ является характеристической функцией вектора $\displaystyle ( X_{t_{1}} ,\ \dotsc ,\ X_{t_{n}}) ,\ 0\leqslant t_{1} < \dotsc < t_{n}$. По построению такой процесс ялвяется процессом с независимыми приращениями, и выполняется $\displaystyle X_{t_{j}} -X_{t_{i}}\xlongequal{d} Q_{t_{j-1} ,\ t_{i}} .$
\end{proof}
\section{Пуассоновский процесс}
\begin{definition}
Процесс $\displaystyle ( N_{t} ,\ t\geqslant 0)$ называется \textit{пуассоновским процессом интенсивности }$\displaystyle \lambda $, если

\begin{enumerate}
    \item $\displaystyle N_{0} =0$ почти наверное,
    \item $\displaystyle N_{t}$ имеет независимые приращения,
    \item $\displaystyle N_{t} -N_{s} \ \sim \ Pois( \lambda ( t-s)) ,\ 0\leqslant s< t$.
\end{enumerate}

\end{definition}
\begin{proposition}
Пуассоновский процесс существует.
\end{proposition}
\begin{proof}
Пусть $\displaystyle \phi _{s,t}$ -- характеристическая функция $\displaystyle Pois( \lambda ( t-s))$. Тогда
\begin{equation*}
\phi _{s,t}( \tau ) =\sum _{k=0}^{\infty } e^{i\tau k}\dfrac{( \lambda ( t-s))^{k}}{k!} e^{-\lambda ( t-s)} =e^{-\lambda ( t-s)} e^{\lambda ( t-s) e^{i\tau }} =e^{\lambda ( t-s)\left( e^{i\tau } -1\right)} .
\end{equation*}
Следовательно, $\displaystyle \phi _{s,u}( \tau ) \cdotp \phi _{u,t}( \tau ) =e^{\lambda ( u-s)\left( e^{i\tau } -1\right)} e^{\lambda ( t-u)\left( e^{i\tau } -1\right)} =e^{\lambda ( t-s)\left( e^{i\tau } -1\right)}$, и по теореме о существовании найдется такой процесс $\displaystyle ( X_{t} ,\ t\geqslant 0)$ с независимыми приращениями, что $\displaystyle X_{t} -X_{s}$ имеет характеристическую функцию $\displaystyle \phi _{s,t}$ для $\displaystyle 0\leqslant s< t$.
\end{proof}
\begin{proposition}
(Свойства траекторий $\displaystyle N_{t}$).
\begin{enumerate}
    \item $\displaystyle N_{t} \ \sim \ Pois( \lambda t) \in \mathbb{Z}_{+}$ -- целочисленные траектории,
    \item $\displaystyle N_{t} -N_{s} \ \sim \ Pois( \lambda ( t-s)) \geqslant 0$ -- неубывают по $\displaystyle t$.
\end{enumerate}
\end{proposition}
\begin{theorem}
(Явная конструкция пуассоновского процесса). Пусть $\displaystyle ( X_{t} ,\ t\geqslant 0)$ -- процесс восстановления, построенный по случайным величинам $\displaystyle \{\xi _{n}\}_{n=1}^{\infty } ,\ \xi _{i} \ \sim \ Exp( \lambda ) \ \forall i\in \mathbb{N}$. Тогда $\displaystyle X_{t}$ -- пуассоновский процесс интенсивности $\displaystyle \lambda $.
\end{theorem}
\begin{proof}
Рассмотрим вектор $\displaystyle ( S_{1} ,\ S_{2} ,\ \dotsc ,\ S_{n})$. Тогда его плотность имеет вид
\begin{gather*}
p_{S_{1} ,\ \dotsc ,\ S_{n}}( x_{1} ,\ \dotsc ,\ x_{n}) =p_{\xi _{1}}( x_{1}) \cdotp p_{\xi _{2}}( x_{2} -x_{1}) \cdotp \ \dotsc \ \cdotp p_{\xi _{n}}( x_{n} -x_{n-1}) =\\
=\lambda e^{-\lambda x_{1}} \cdotp \lambda e^{-\lambda ( x_{2} -x_{1})} \cdotp \ \dotsc \ \cdotp \lambda e^{-\lambda ( x_{n} -x_{n-1})} =\lambda ^{n} e^{-\lambda x_{n}} \cdotp I( 0< x_{1} < \dotsc < x_{n}) .
\end{gather*}
 Пусть $\displaystyle 0< t_{1} < \dotsc < t_{n}$ и $\displaystyle 0\leqslant k_{1} \leqslant \dotsc \leqslant k_{n} ,\ k_{j} \in \mathbb{Z} \ \forall j=\overline{1,n}$. Тогда

\begin{gather*}
P( X_{t_{n}} -X_{t_{n-1}} =k_{n} -k_{n-1} ,\ \dotsc ,\ X_{t_{2}} -X_{t_{1}} =k_{2} -k_{1} ,\ X_{t_{1}} =k_{1}) =\\
=P( S_{1} ,\ \dotsc ,\ S_{k_{1}} \in ( 0,\ t_{1}] ,\ \dotsc ,\ S_{k_{n-1} +1} ,\ \dotsc ,\ S_{k_{n}} \in ( t_{n-1} ,\ t_{n}] ,\ S_{k_{n+1}}  >t_{n}) =\\
\underset{A}{\idotsint} p_{S_{1} ,\ \dotsc ,\ S_{k_{n+1}}}( x_{1} ,\ \dotsc ,\ x_{k_{n+1}}) dx_{1} \dotsc dx_{k_{n+1}} =:I,
\end{gather*}
где $\displaystyle A=\{( x_{1} ,\ \dotsc ,\ x_{k_{1}}) \in ( 0,\ t_{1}] ,\ \dotsc ,\ ( x_{k_{n-1} +1} ,\ \dotsc ,\ x_{k_{n}}) \in ( t_{n-1} ,\ t_{n}] ,\ x_{k_{n+1}}  >t_{n}\}$. Тогда

\begin{gather*}
I=\int _{t_{n}}^{+\infty } \lambda ^{k_{n+1}} e^{-\lambda x_{k_{n} +1}} dx_{k_{n} +1} \cdotp \\
\prod _{j=1}^{n}\underset{{x_{k_{j-1} +1} ,\ \dotsc ,\ x_{k_{j}} \in ( t_{j-1} ,\ t_{j}]}}{\idotsint}\mathbb{I}( x_{k_{j-1} +1} < \dotsc < x_{k_{j}}) dx_{k_{j-1} +1} \dotsc dx_{k_{j}} .
\end{gather*}
Каждый интеграл в произведении равен объему симплекса. Куб в $\displaystyle k$-мерном пространстве полностью покрывается $\displaystyle k!$ непересекающимися равными по объему симплексами, каждый из которых порождается соответствующей перестановкой переменных. Поэтому, объем одного симплекса в $\displaystyle k$-мерном пространстве внутри куба со стороной $\displaystyle m$ будет равняться $\displaystyle \dfrac{m^{k}}{k!}$. Из этого получаем, что
\begin{equation*}
I=\lambda ^{k_{n+1}} e^{-\lambda t_{n}} \cdotp \prod _{j=1}^{n}\dfrac{( t_{j} -t_{j-1})^{k_{j} -k_{j-1}}}{( k_{j} -k_{j-1}) !} =\prod _{j=1}^{n}\dfrac{( \lambda ( t_{j} -t_{j-1}))^{k_{j} -k_{j-1}}}{( k_{j} -k_{j-1}) !} e^{-\lambda ( t_{j} -t_{j-1})} .
\end{equation*}
Из этого получается, что приращения случайного процесса $\displaystyle ( X_{t} ,\ t\geqslant 0)$ независимы и приращения $\displaystyle X_{t} -X_{s} \ \sim \ Pois( \lambda ( t-s)) ,\ 0\leqslant s< t$.
\end{proof}