\section{Теоремы Тихонова}

\subsection{Топологическое произведение}

\begin{definition}
    Пусть $\lbrace X_{\alpha}\rbrace_{\alpha \in \mf A}$ "--- система топологических пространств. \textit{Элементарным открытым множеством} в прямом произведении $X := \prod_{\alpha \in \mf A}X_{\alpha}$ называется следующее множество для некоторых открытых множеств $U_{\alpha_1} \subset X_{\alpha_1}, \dotsc, U_{\alpha_s} \subset X_{\alpha_s}$:
    \[O(U_{\alpha_1}, \ldots, U_{\alpha_s}) := \lbrace x \in X: x_{\alpha_1} \in U_{\alpha_1}, \dotsc, x_{\alpha_s} \in U_{\alpha_s}\rbrace\]

    \textit{Топологией Тихонова} на пространстве $X$ называется топология, базой которой является совокупность всех элементарных открытых множеств.
\end{definition}

\begin{proposition}
    Пусть $\lbrace X_{\alpha}\rbrace_{\alpha \in \mf A}$ "--- система хаусдорфовых пространств. Тогда пространство $X := \prod_{\alpha \in \mf A}X_{\alpha}$ с топологией Тихонова тоже является хаусдорфовым.
\end{proposition}

\begin{proof}
    Пусть $x, y \in X$ "--- две различных точки, тогда существует $\alpha \in \mf A$ такое, что $x_\alpha \ne y_\alpha$. Поскольку пространство $X_\alpha$ хаусдорфово, то в нем существуют непересекающиеся окрестности $U(x_\alpha), U(y_\alpha)$. Соответствующие этим окрестностям элементарные открытые множества в $X$ являются непересекающимися окрестностями точек $x, y$.
\end{proof}

\begin{definition}
    Пусть $\lbrace X_{\alpha}\rbrace_{\alpha \in \mf A}$ и $\lbrace Y_{\alpha}\rbrace_{\alpha \in \mf A}$ "--- системы топологических пространств, $X := \prod_{\alpha \in \mf A}X_{\alpha}$, $Y := \prod_{\alpha \in \mf A}Y_{\alpha}$. \textit{Произведением}, или \textit{простым произведением}, системы отображений $\{f_\alpha: X_\alpha \to Y_\alpha\}_{\alpha \in \mf A}$ называется отображение $f : X \to Y$, заданное для произвольного $x \in X$ следующим образом:
    \[f(x) := (f_\alpha(x_\alpha))_{\alpha \in \mf A}\]

    Обозначение "--- $f = \prod_{\alpha}f_{\alpha}$.
\end{definition}

\begin{proposition}
    Пусть $\lbrace X_{\alpha}\rbrace_{\alpha \in \mf A}$ и $\lbrace Y_{\alpha}\rbrace_{\alpha \in \mf A}$ "--- системы топологических пространств, $X := \prod_{\alpha \in \mf A}X_{\alpha}$, $Y := \prod_{\alpha \in \mf A}Y_{\alpha}$, $\{f_\alpha: X_\alpha \to Y_\alpha\}_{\alpha \in \mf A}$ "--- система непрерывных отображений. Тогда простое произведение $f$ этой системы тоже непрерывно.
\end{proposition}

\begin{proof}
    Пусть $O := O(U_{\alpha_1}, \ldots, U_{\alpha_s})$ "--- элементарное открытое множество в $Y$. Достаточно проверить, что прообраз $f^{-1}(O)$ открыт в $X$. Заметим, что выполнены следующие равенства:
    \[f^{-1}(O) = f^{-1}(O(U_{\alpha_1}, \ldots, U_{\alpha_s})) = O(f_{\alpha_1}^{-1}(U_{\alpha_1}), \ldots, f_{\alpha_s}^{-1}(U_{\alpha_s}))\]

    Таким образом, прообраз элементарного открытого множества сам является элементарным открытым множеством. В силу произвольности выбора множества $O$, получено требуемое.
\end{proof}

\begin{definition}
    Пусть $X$ "--- топологическое пространство, $\lbrace Y_{\alpha}\rbrace_{\alpha \in \mf A}$ "--- система топологических пространств, $Y := \prod_{\alpha \in \mf A}Y_{\alpha}$. \textit{Диагональным произведением} системы отображений $\{f_\alpha: X \to Y_\alpha\}_{\alpha \in \mf A}$ называется отображение $f : X \to Y$, заданное для произвольного $x \in X$ следующим образом:
    \[f(x) := (f_\alpha(x))_{\alpha \in \mf A}\]
\end{definition}

\begin{proposition}
    Пусть $X$ "--- топологическое пространство, $\lbrace Y_{\alpha}\rbrace_{\alpha \in \mf A}$ "--- система топологических пространств, $Y := \prod_{\alpha \in \mf A}Y_{\alpha}$, $\{f_\alpha: X \to Y_\alpha\}_{\alpha \in \mf A}$ "--- система непрерывных отображений. Тогда диагональное произведение $f$ этой системы тоже непрерывно.
\end{proposition}

\begin{proof}
    Для каждого $\alpha \in \mf A$ положим $X_{\alpha} := X$ и рассмотрим \textit{диагональное вложение} $i: X \rightarrow \prod_{\alpha \in \mf A}X_{\alpha}$, заданное для произвольного $x \in X$ как $i(x) := (x)_{\alpha \in \mf A}$. Вложение $i$ непрерывно, и отображение $f$ представляется в виде композиции вложения $i$ и простого произведения системы $\{f_{\alpha}\}_{\alpha \in \mf A}$, поэтому оно тоже является непрерывным.
\end{proof}

\subsection{Первая теорема Тихонова}

\begin{definition}
    Система $\mathcal{F}$ подмножеств множества $M$ называется \textit{фильтром} на $M$, если выполнены следующие условия:
    \begin{enumerate}
        \item $M\in\mathcal{F}$, $\emptyset\notin \mathcal{F}$
        \item $\forall A, B \in \mathcal{F}: A\cap B \in \mathcal{F}$
        \item $\forall A\in\mathcal{F}: \forall B \in M,~A\subset B: B\in\mathcal{F}$
    \end{enumerate}
\end{definition}

\begin{example}
    Рассмотрим несколько примеров фильтров:
    \begin{itemize}
        \item Система $\mathcal{F} = \lbrace M\rbrace$, где $M$ "--- произвольное множество, образует \textit{тривиальный фильтр}

        \item Система $\mathcal{F} = \lbrace A \subset M: x\in A\rbrace$, где $M$ "--- произвольное множество $x \in M$ "--- фиксированная точка, образует \textit{главный фильтр}, порожденный точкой $x$
        
        \item Система всех коконечных подмножеств бесконечного множества $M$ образует фильтр
    \end{itemize}
\end{example}

\begin{definition}
    Фильтр $\mathcal{U}$ на множестве $M$ называется \textit{ультрафильтром}, если для любого подмножества $A \subset M$ выполнено либо $A \in \mathcal{U}$, либо $M\backslash A\in\mathcal{U}$.
\end{definition}

\begin{proposition}
    Пусть $\mathcal{F}$ "--- фильтр на множестве $M$. Тогда $\mathcal{F}$ является центрированной системой.
\end{proposition}

\begin{proof}
    Для произвольных $A_1, \ldots, A_s \in \mathcal{F}$ выполнено $A_1 \cap \dotsb \cap A_s\in\mathcal{F}$, и это пересечение не может быть пустым, поскольку $\emptyset \notin \mc F$.
\end{proof}

\begin{proposition}
    Пусть $S$ "--- центрированная система на множестве $M$. Тогда существует минимальный по включению фильтр, содержащий $S$.
\end{proposition}

\begin{proof}
    Добавим к $S$ все конечные пересечения множеств из $S$ и обозначим полученную систему через $S'$. Система $S'$ не содержит пустого множества в силу центрированности системы $S$. Тогда искомым минимальным фильтром является система $\{B \subset M: \exists A \in S': A \subset B\}$.
\end{proof}

\begin{proposition}[существование ультрафильтров]
    Пусть $\mathcal{F}$ "--- фильтр на множестве $M$. Тогда $\mathcal{F}$ содержится в некотором ультрафильтре.
\end{proposition}

\begin{proof}
    Рассмотрим множество всех фильтров, содержащих $\mathcal{F}$, частично упорядоченное по включению. Любая цепь, то есть линейно упорядоченное подмножество, имеет максимальный элемент, равный объединению элементов всех элементов цепи. Тогда, по лемме Цорна, множество всех фильтров, содержащих $\mathcal{F}$, имеет максимальный элемент $\mathcal{U}$. Предположим, что $\mf U$ не является ультрафильтром, тогда существует $A \subset M$ такое, что $A \notin \mathcal{U}$ и $M\backslash A \notin \mathcal{U}$. Проверим, что тогда система $\mathcal{U}\cup \lbrace A\rbrace$ является центрированной. Действительно, если для некоторых $A_1, \dotsc, A_s \in \mc U$ выполнено $A_1\cap\ldots\cap A_s\cap A = \emptyset$, то $A_1 \cap\ldots\cap A_s\subset M\backslash
    A$, откуда $M\backslash A\in\mathcal{U}$, что неверно. Значит, система $\mathcal{U}\cup \lbrace A\rbrace$ порождает фильтр, который строго больше фильтра $\mathcal{U}$ и при этом содержит $\mathcal{F}$, --- противоречие.
\end{proof}

\begin{proposition}\label{propositionfiniteunionsmall}
    Пусть $\mathcal{F}$ "--- фильтр на множестве $M$. Тогда $\mathcal{F}$ является ультрафильтром $\lra$ для любых множеств $A_1, \dotsc, A_s \subset M$ таких, что $A_1 \cup \dotsb \cup A_s \in \mc F$, существует $i \in \{1, \dotsc, s\}$ такое, что $A_i \in \mathcal{F}$.
\end{proposition}

\begin{proof}~
    \begin{itemize}
        \item[$\la$] Для любого $A \subset M$ выполнено равенство $M = A \cup (M \bs A)$, поэтому либо $A \in \mc F$, либо $M \bs A \in \mc F$.

        \item[$\ra$] Пусть $\mathcal{F}$ "--- ультрафильтр, и $A_1 \cup \dotsb \cup A_s \in\mathcal{F}$. Если ни одно из множеств $A_1, \dotsc, A_s$ не лежит в $\mc F$, то, по свойству ультрафильтра, $(M \bs A_1), \dotsc, (M \bs A_s) \in \mc F$. Тогда:
        \[(A_1 \cup \dotsb \cup A_s) \cap (M \bs A_1) \cap \dotsb \cap (M \bs A_s) = \emptyset \in \mc F\]

        Получено противоречие, значит, существует $i \in \{1, \dotsc, s\}$ такое, что $A_i \in \mathcal{F}$.\qedhere
    \end{itemize}
\end{proof}

\begin{proposition}\label{randomprop}
    Пусть $f: M \rightarrow N$ "--- отображение множеств, и $\mathcal{F}$ "--- фильтр на множестве $M$. Тогда фильтром на $N$ является следующая система:
    \[f^*(\mathcal{F}) := \lbrace B \subset N : f^{-1}(B)\in\mathcal{F}\rbrace\]

    Кроме того, если $\mathcal{F}$ "--- ультрафильтр, то и $f^*(\mathcal{F})$ "--- ультрафильтр.
\end{proposition}

\begin{proof}
    Очевидно, $N \in f^*(\mc F)$, поскольку $f^{-1}(N) = M \in \mc F$, и $\emptyset \notin f^*(\mc F)$, поскольку $f^{-1}(\emptyset) = \emptyset \notin \mc F$. Остальные свойства проверяются элементарно, поскольку взятие прообраза сохраняет теоретико-множественные операции.
\end{proof}

\begin{definition}
    Пусть $X$ "--- топологическое пространство, $\mc F$ "--- фильтр на $X$. Точка $x \in X$ является \textit{предельной точкой фильтра} $\mathcal{F}$, если любая окрестность $U(x)$ точки $x$ принадлежит $\mathcal{F}$.
\end{definition}

\begin{proposition}
    Топологическое пространство $X$ бикомпактно $\lra$ любой ультрафильтр $\mathcal{U}$ на $X$ имеет хотя бы одну предельную точку.
\end{proposition}

\begin{proof}~
    \begin{itemize}
        \item[$\ra$] Предположим, что на бикомпактном пространстве $X$ существует ультрафильтр $\mathcal{U}$, не имеющий ни  одной предельной точки. Тогда для любой точки $\forall x\in X$ существует окрестность $U(x)$ такая, что $U(x)\notin\mathcal{U}$. Система $\{U(x): {x \in X}\}$ покрывает $X$ и, в силу бикомпактности, имеет конечное подпокрытие $\{U(x_1), \dotsc, U(x_n)\}$. Тогда $X = \bigcup_{i = 1}^n U(x_i)$, причем ни одна из окрестностей $U(x_1), \dotsc, U(x_n)$ не принадлежит ультрафильтру. В силу утверждения \ref{propositionfiniteunionsmall}, получено противоречие.

        \item[$\la$] Пусть $X$ не бикомпактно, тогда существует $\lbrace U_{\alpha}\rbrace_{\alpha \in \mf A}$ "--- открытое покрытие, не имеющее конечного подпокрытия. Значит, система $\lbrace X\backslash U_{\alpha}\rbrace_{\alpha \in \mf A}$ является центрированной и потому содержится в некотором ультрафильтре $\mathcal{U}$. Пусть $x \in X$, тогда существует $\alpha \in \mf A$ такое, что $U_{\alpha}$ покрывает $x$. Так как $X\backslash U_{\alpha}\in\mathcal{U}$ и $\mathcal{U}$ "--- фильтр, то
        $U_{\alpha} \notin \mathcal{U}$. Значит, $x \in X$ не является предельной точкой для $\mc U$, поэтому, в силу произвольности выбора точки $x$, $\mc U$ не имеет предельных точек --- противоречие.\qedhere
    \end{itemize}
\end{proof}

\begin{proposition}
    Топологическое пространство $X$ хаусдорфово $\lra$ любой ультрафильтр $\mathcal{U}$ на $X$ имеет не более одной предельной точки.
\end{proposition}

\begin{proof}~
    \begin{itemize}
        \item[$\ra$] Предположим, что существует ультрафильтр $\mathcal{U}$ на $X$, имеющий две различных предельных точки $x, y \in X$. Пусть $U(x), U(y)$ "--- непересекающиеся окрестности этих точек, тогда $U(x), U(y) \in \mc F$, откуда $U(x) \cap U(y) = \emptyset \in \mc U$ --- противоречие.
        
        \item[$\la$] Пусть $X$ не хаусдорфово, тогда существуют две различные точки $x, y \in X$ такие, что любые их окрестности $U(x), U(y)$ имеют непустое пересечение. Тогда система, состоящая из всех окрестностей точек $x, y$, является центрированной, поэтому она содержится в некотором ультрафильтре, для которого $x$ и $y$ являются предельными точками --- противоречие.\qedhere
    \end{itemize}
\end{proof}

\begin{theorem}[первая теорема Тихонова]
    Пусть $\lbrace X_{\alpha}\rbrace_{\alpha \in \mf A}$ "--- система бикомпактных пространств. Тогда пространство $X := \prod_{\alpha \in \mf A}X_{\alpha}$ тоже является бикомпактным.
\end{theorem}

\begin{proof}
    Покажем, что каждый ультрафильтр $\mathcal{U}$ на $X$ имеет предельную точку. Пусть $\{\pi_{\alpha}: X \rightarrow X_{\alpha}\}_{\alpha \in \mf A}$ "--- система проекций. По утверждению \ref{randomprop}, для любого $\alpha \in \mf A$ система $(\pi_{\alpha})^*(\mathcal{U})$ образует ультрафильтр на $X_{\alpha}$, который, в силу бикомпактности пространства $X_{\alpha}$, имеет предельную точку $x_{\alpha} \in X_\alpha$. Покажем, что точка $x := (x_{\alpha})_{\alpha \in \mf A} \in X$ является предельной точкой ультрафильтра $\mathcal{U}$.
    
    Если для некоторого $\alpha \in \mf A$ и открытого множества $U \subset X_\alpha$ выполнено, что $X \in \pi_{\alpha}^{-1}(U)$, то $x_{\alpha} \in U$, откуда $U \in (\pi_{\alpha})^*(\mathcal{U})$, поскольку точка $x_\alpha$ "--- предельная для ультрафильтра $(\pi_{\alpha})^*(\mathcal{U})$. Значит, $\pi_{\alpha}^{-1}(U) \in \mathcal{U}$. Наконец, система множеств вида $\pi_{\alpha}^{-1}(U)$, где $\alpha \in \mf A$, а множество $U \subset X_\alpha$ "--- открытое, образует предбазу в $X$, поэтому всякая окрестность $U(x)$ точки $x$ содержит окрестность этой точки, являющуюся пересечением конечного числа множеств такого вида и потому лежащую в $\mc U$, откуда $U(x) \in \mc U$. Таким образом, $x$ "--- предельная точка для $\mathcal{U}$.
\end{proof}

\begin{corollary}
    Пусть $\lbrace X_{\alpha}\rbrace_{\alpha \in \mf A}$ "--- система бикомпактов. Тогда пространство $X := \prod_{\alpha \in \mf A}X_{\alpha}$ тоже является бикомпактом.
\end{corollary}

\subsection{Вторая теорема Тихонова}

\begin{proposition}
    Пусть $X$ "--- вполне регулярное пространство, $x_0 \in X$. Тогда для любой окрестности $G(x_0)$ точки $x_0$ существует окрестность $G'(x_0)$ этой же точки такая, что замкнутые множества $\overline{G'(x_0)}$ и $X \backslash G(x_0)$ функционально отделимы в $X$.
\end{proposition}

\begin{proof}
    Пусть $f : X \to [0, 1]$ "--- функция, отделяющая точку $x_0$ от множества $X \backslash G(x_0)$, тогда $f(x_0) = 0$ и $f(G(x_0)) = \{1\}$. Определим множество $G'(x_0)$ следующим образом:
    \[G'(x_0) := \lbrace x \in X: f(x)<1/2\rbrace\]
    
    Теперь определим функцию $f_1$ для произвольного $x \in X$ следующим образом:
    \[
        f_1(x) =\begin{cases}
            0, & \text{если }f(x) \le \frac12\\
            2f(x) - 1, & \text{если }f(x) \ge \frac12
        \end{cases}
    \]
    
    Легко проверить, что для функции $f_1$ прообраз замкнутого множества является замкнутым множеством , поэтому функция $f_1$ непрерывна, и она отделяет множество $\overline{G'(x_0)}$ от множества $X\backslash G(x_0)$.
\end{proof}

\begin{definition}
    Пусть $X$ "--- вполне регулярное пространство. Множество $\Xi$ непрерывных отображений $X \rightarrow [0, 1]$ \textit{расчленяет} пространство $X$, если для любой точки $x_0 \in X$ и любой окрестности $U(x_0)$ этой точки существует функция $f \in \Xi$ такая, что $f(x_0) = 0$ и $f(x) = 1$ для любой точки $x \in X \backslash U(x_0)$.
\end{definition}

\begin{proposition}\label{thm2prop1}
    Пусть $X$ "--- вполне регулярное пространство веса $\tau$. Тогда пространство $X$ имеет расчленяющее множество $\Xi$ мощности не больше $\tau$.
\end{proposition}

\begin{proof}
    Пусть $\lbrace U_i\rbrace$ "--- база пространства $X$ мощности $\tau$. Назовем пару $(U_i, U_j)$ элементов базы \textit{канонической}, если множества $\overline{U_i}, (X \backslash U_j)$ функционально отделимы. Обозначим множдество канонических пар через $\{\pi_m\}$. Оно имеет мощность не больше $\tau$. Выберем для каждой канонической пары $\pi_m = (U, V)$ функцию $f_m$, непрерывную на $X$, для которой выполнены равенства $f(\overline{U}) = \{0\}$ и $f(X\backslash V) = \{1\}$. Множество $\Xi$ всех таких функций имеет мощность не больше $\tau$. Проверим, что оно является расчленяющим множеством.
    
    Для любой точки $x_0 \in X$ и любой ее окрестности $G$ выберем множество $U_j$ из базы, содержащее $x$ и содержащееся в $G$. По предыдущему утверждению, можно выбрать окрестность $G'$ точки $x_0$ такую, что множества $\overline{G'}$ и $X \bs U_j$ функционально отделимы, тогда, выбирая множество $U_i$ из базы, содержащее $x$ и содержащееся в $G'$, получим каноническую пару $\pi_m := (U_i, V_j)$. Соответствующая этой паре функция $f_m$ отделяет точку $x$ от множества $X \bs G$.
\end{proof}

\begin{definition}
    Пусть $\tau$ "--- некоторое кардинальное число. \textit{Тихоновским кирпичом} веса $\tau$ называется бикомпакт, образованный произведением $\tau$ отрезков числовой прямой. Обозначение "--- $I^{\tau}$.
\end{definition}

\begin{proposition}\label{thm2prop2}
    Пусть $X$ "--- вполне регулярное пространство, $\Xi$ "--- множество фун\-кций, расчленяющее $X$, мощности $\tau'$. Тогда для некоторого $Y \subset I^{\tau'}$ существует гомеоморфизм $\varphi: X \rightarrow Y$.
\end{proposition}

\begin{proof}
    Пусть $\Xi = \lbrace f_\alpha\rbrace_{\alpha \in \mf A}$ функций, расчленяющих $X$, и $\mf A$ имеет мощность $\tau'$. Построим требуемый гомоморфизм следующий образом. Для каждого $\alpha \in \mf A$ выберем отрезок $I_\alpha = [0, 1]$, и каждой точке $x\in X$ поставим в соответствие точку следующего вида:
    \[\phi(x) := (f_\alpha(x))_{\alpha \in \mf A} \in \prod_{\alpha \in \mf A}I_\alpha = I^{\tau'}\]
    
    Полученное отображение является диагональным произведением непрерывных отображений и потому непрерывно. Положим $Y := \phi(X)$. Проверим, что $\phi$ инъективно. Действительно, если $x, x' \in X$ "--- две различных точки пусть, то, в силу регулярности, можно выбрать их непересекающиеся окрестности $U(x), U'(x)$, тогда, поскольку множество $\Xi$ расчленяет пространство $X$, существует $f_\alpha \in \Xi$ такое, что $f_\alpha(x) = 0$, но $f_\alpha(x') = 1$, поэтому $\phi(x) \ne \phi(x')$.

    Проверим теперь, что $\varphi^{-1}$ непрерывно. Зафиксируем произвольную точку $x \in X$ и ее окрестность $U(x)$. Выберем функцию $f_\alpha \in \Xi$ такую, что $f_\alpha(x) = 0$ и ${f_\alpha(X \bs U(x)) = \{1\}}$. Для точки $y := \varphi(x)$ выполенно равенство $y_\alpha= 0$. Рассмотрим следующую окрестность точки $y$:
    \[U(y) := \{y' \in I^\tau: (y')_\alpha < 1\}\]
    
    Для любой точки $y' \in U(y) \cap Y$ выполнено $\varphi^{-1}(y') \in U(x)$, поскольку иначе выполнены равенства $f_\alpha(\phi^{-1}(y')) = (y')_\alpha = 1$. Значит, $\phi^{-1}(U(y) \cap Y) \subset U(x)$. В силу произвольности выбора точки $x$, отображение $\phi^{-1} : Y \to X$ непрерывно.
\end{proof}

\begin{theorem}[вторая теорема Тихонова]
    Пусть $X$ "--- вполне регулярное пространство веса $\tau$. Тогда $X$ гомеоморфно некоторому подпространству тихоновского кирпича $I^\tau$.
\end{theorem}

\begin{proof}
    По утверждению \ref{thm2prop1}, существует множество функций, расчленяющее $X$, веса не больше $\tau$. Дополним его произвольным образом до множества функций веса $\tau$ и применим утверждение \ref{thm2prop2}.
\end{proof}

\begin{corollary}
    Пусть $X$ "--- топологическое пространство. Тогда $X$ вполне регулярно $\lra$ $X$ является подпространством некоторого бикомпакта.
\end{corollary}

\begin{proof}~
    \begin{itemize}
        \item[$\la$] Если $Y \supset X$ "--- бикомпакт, то он нормален и, по большой лемме Урысона, вполне регулярен. Свойство вполне регулярности является наследственным, поэтому пространство $X$ тоже является вполне регулярным пространством.
        
        \item[$\ra$] По первой теореме Тихонова, любой тихоновский кирпич является бикомпактом, и пространство $X$ гомеоморфно подпространству некоторого тихоновского кирпича по второй теореме Тихонова.\qedhere
    \end{itemize}
\end{proof}