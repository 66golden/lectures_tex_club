\section{База и вес топологического пространства}

\subsection{База открытой топологии}

\begin{definition}
    Пусть $X$ "--- топологическое пространство с открытой топологией $\tau$. Система $\mathfrak{B} \subset \tau$ такая, что любое открытое множество в $X$ представимо в виде объединения некоторого набора множеств из $\mf B$, называется \textit{открытой базой} пространства $X$.
\end{definition}

\begin{note}
    Сама система $\tau$ является открытой базой. Кроме того, если $(X, \rho)$ "--- метрическое пространство, то его открытой базой является следующая система:
    \[\mf B := \left\{U\left(x, \frac 1n\right): x\in X,~n\in \mathbb N\right\}\]
\end{note}

\begin{proposition}\label{propbase}
    Пусть $(X, \tau)$ "--- топологическое пространство. Тогда система $\mathfrak{B} \subset \tau$ является открытой базой $\lra$ для любой точки $x \in X$ и любой ее окрестности $U(x)$ существует окрестность $V(x) \in \mathfrak{B}$ этой точки такая, что $V(x) \subset U(x)$.
\end{proposition}

\begin{proof}~
    \begin{itemize}
        \item[$\ra$] Если $\mathfrak{B}$ "--- открытая база, то для любой точки $x \in X$ и ее окрестности $U(x)$ существует такой набор $\{U_\alpha\}_{\alpha \in \mf A} \subset \mf B$, что $U(x) = \cup_{\alpha \in \mathfrak{A}} U_{\alpha}$, поэтому существует $U_\alpha \in \mf B$ такое, что $x \in U_{\alpha}$.

        \item[$\la$] Пусть система $\mathfrak{B}$ удовлетворяет условию. Для любого открытого множества $U \subset X$ и любой точки $x \in U$ выберем множество $V(x) \in \mathfrak{B}$, тогда $U = \cup_{x\in U}V(x)$. Значит, $\mathfrak{B}$ "--- открытая база топологии на $X$.\qedhere
    \end{itemize}
\end{proof}

\begin{definition}
    Пусть $X$ "--- топологическое пространство с замкнутой топологией $\kappa$. Система $\mathfrak{F} \subset \kappa$ такая, что любое замкнутое множество в $X$ представимо в виде пересечения некоторого набора множеств из $\mf F$, называется \textit{замкнутой базой} пространства $X$.
\end{definition}

\begin{definition}
    Пусть $X$ "--- топологическое пространство, $\mathfrak{C}$ "--- его открытая или замкнутая база. Базой, \textit{сопряженной} к базе $\mf C$, называется система, полученная из $\mf C$ заменой каждого из множеств системы на его дополнение.
\end{definition}

\begin{definition}
    \textit{Весом} топологического пространства $X$ называется наименьшее кардинальное число $w(X)$, являющееся мощностью какой-либо базы пространства.
\end{definition}

\begin{note}
    Если $Y \subset X$, то $w(Y)\leq w(X)$.
\end{note}

\begin{definition}
    Пусть $X$ "--- топологическое пространство. Система $S$ открытых множеств, содержащих точку $x \in X$, такая, что в любой окрестности $U(x)$ точки $x$ содержится некоторое множество из $S$, называется \textit{локальной базой} пространства $X$ в точке $x$.
\end{definition}

\begin{definition}
    \textit{Локальным ве\-сом} топологического пространства $X$ в точке $x \in X$ наименьшее кардинальное число, являющееся мощностью какой-либо локальной базы пространства в точке $x$.
\end{definition}

\begin{note}
    Локальный вес пространства в любой его изолированной точке $x$ равен $1$, поскольку локальной базой пространства является система из одного множества $\{x\}$.
\end{note}

\subsection{Аксиомы счетности}

\begin{definition}
    Топологическое пространство $X$ удовлетворяет \textit{первой аксиоме счетности}, если его локальный вес в каждой точке не более чем счетен.
\end{definition}

\begin{note}
    Важнейшим классом пространств с первой аксиомой счетности являются метрические пространства. Если $X$ "--- метрическое пространство, то для каждой точки $x \in X$ система окрестностей $\{U(x, \frac 1n): n\in\mathbb{N}\}$ образует счетную локальную базу.
\end{note}

\begin{definition}
    Топологическое пространство $X$ удовлетворяет \textit{второй аксиоме счетности}, если оно обладает счетной базой.
\end{definition}

\begin{definition}
    Пусть $X$ "--- топологическое пространство, $\{x_n\} \subset X$. Последовательность $\{x_n\}$ \textit{сходится} к некоторой точке $x \in X$, если каждая окрестность точки $x$ содержит все точки последовательности $\{x_n\}$, начиная с некоторой.
\end{definition}

\begin{note}
    В определении выше достаточно ограничиться рассмотрением системы окре\-стностей точки $x$, образующей локальную базу в этой точке.
\end{note}

\begin{note}
    В общей топологии значение сходимости, аналогичное определению из математического анализа, несколько меньше, чем в анализе, что связано, как будет выяснено далее, с отсутствием, вообще говоря, свойств счетности открытой топологии. Существует обобщение сходимости, пригодное для нужд общей топологии, выходящее за рамки данного курса.
\end{note}

\begin{proposition}\label{propfrecheturysohn}
    Пусть $X$ "--- топологическое пространство, удовлетворяющее первой аксиоме счетности. Тогда точка $x \in X$ является точкой прикосновения множества $M \subset X$ $\lra$ существует последовательность $\{x_n\} \subset M$, сходящаяся к точке $x$.
\end{proposition}

\begin{proof}
    Нетривиально только доказательство $(\ra)$. Пусть $x \in X$ "--- точка прикосновения множества $M$. Возьмем какую-нибудь счетную локальную базу $\{U_n\}$ в точке $x$. Без ограничения общности можно считать, что множества локальной базы образуют убывающую цепочку, поскольку каждое множество $U_n$ в базе можно заменить на множество $U_1\cap U_2\cap\dotsb\cap U_n$, и полученная система тоже будет базой.

    Так как $x$ "--- точка прикосновения множества $M$, то в каждом множестве $U_n$ найдется точка $x_k \in M\cap U_n$. Докажем, что последовательность $\{x_n\}$ сходится к $x$. В самом деле, пусть $U(x)$ -- некоторая окрестность точки $x$, тогда по определению локальной базы найдется множество $U_k$ из базы такое, что $U_k\subset U(x)$. Поскольку последовательность $\{U_n\}$ образует убывающую цепочку, то все элементы последовательности $\{x_n\}$, начиная с $x_k$, принадлежат множеству $U_k\subset U(x)$, а это и означает, что $\{x_k\}$ сходится к $x$.
\end{proof}

\begin{note}
    В силу утверждения выше, в пространствах с первой аксиомой счетности топологию можно полностью определить с помощью сходимости.
\end{note}

\begin{definition}
    Пусть $X$ "--- топологическое пространство, $M \subset X$. \textit{Секвенциальным замыканием} множества $M$ называется множество $[M]_{seq}$ точек из $X$, являющихся пределами последовательностей точек из $M$.
\end{definition}

\begin{note}
    Оператор секвенциального замыкания удовлетворяет всем аксиомам Куратовского, кроме идемпотентности, и потому не является оператором замыкания. Операторы на множестве подмножеств множества, удовлетворящие набору аксиом Куратовского без аксиомы идемпотентности называют \emph{операторами предзамыкания} или \emph{операторами замыкания Чеха}. Нетрудно видеть также, что предзамыкание всегда монотонно.
\end{note}

\begin{definition}
    \textit{Пространством Фреше--Урысона} называется топологическое пространство $X$, оператор секвенциального замыкания на котором совпадает с оператором замыкания.
\end{definition}

\begin{note}
    В силу утверждения \ref{propfrecheturysohn}, в классе пространств Фреше--Урысона лежат все пространства с первой аксиомой счетности и, в частности, все метрические
    пространства.
\end{note}

\subsection{Сепарабельные пространства}

\begin{proposition}
    Если в топологическом пространстве $X$ имеется база мощности $\mathfrak{m}$, то в нем также имеется всюду плотное множество мощности $\mathfrak{m}$.
\end{proposition}

\begin{proof}
    Выберем по одной точке из каждого множества базы, имеющей мощность $\mf m$, и проверим, что полученное множество всюду плотно. Действительно, для любой точки $x \in X$ и любой ее окрестности $U(x)$ верно, что в этой окрестности содержится некоторое множество из базы, а значит, и некоторая точка из требуемого множества. 
\end{proof}

\begin{definition}
    \textit{Плотностью} топологического пространства $X$ называется наименьшее кардинальное число $\rho (X)$ такое, что в $X$ существует всюду плотное подмножество мощности $\rho (X)$.
\end{definition}

\begin{definition}
    Топологическое пространство $X$ называется \textit{сепарабельным}, если оно имеет не более чем счетную плотность.
\end{definition}

\begin{note}
    В силу уже доказанного, для любого пространства $X$ выполнено неравенство $\rho (X)\leq w(X)$. В частности, пространства, удовлетворяющие второй аксиоме счетности, являются сепарабельными. Однако обратное неверно, что будет показано позднее.
\end{note}

\begin{theorem}\label{thmmetricseparablecond}
    Метрическое пространство $X$ является сепарабельным $\lra$ $X$ имеет счетную базу.
\end{theorem}

\begin{proof}
    Нетривиально только доказательство $(\ra)$. Пусть $X$ "--- сепарабельное метрическое пространство, $M = \{x_m\} \subset X$ "--- всюду плотное множество. Рассмотрим счетное множество $\mathfrak{B} := \{U(x_m, r): x_m\in M,~r \in\mathbb{Q},~r>0\}$ и докажем, что оно является базой пространства $X$.
    
    Зафиксируем точку $x \in X$ и ее окрестность $U(x)$, тогда существует $\epsilon > 0$ такое, что $U(x, \epsilon) \subset U(x)$. Выберем точку $x_m$, лежащую в $U(x, \frac{\epsilon}{3})$, и число $r \in (\frac{\epsilon}{3}, \frac{2\epsilon}{3}) \cap \Q$. Тогда $U(x_m, r) \subset U(x, \epsilon) \subset U(x)$, и $x \in U(x_m, r)$. В силу произвольности выбора точки $x$ и окрестности $U(x)$, получено требуемое.
\end{proof}

\begin{note}
    Аналогичным образом доказывается и более общее утверждение, согласно которому для любого метрического пространства $X$ выполнено равенство $\rho(X) = w(X)$.
\end{note}

\begin{theorem}[Александрова--Урысона]\label{thmalexandrovurysohn}
    Пусть $X$ "--- топологическое пространство с весом $\mf m$. Тогда всякая база $\mathfrak{B}$ пространства $X$ содержит подмножество $\mathfrak{B}'$ мощности $\mathfrak{m}$, также являющееся базой пространства $X$.
\end{theorem}

\begin{proof}
Пусть $\mathfrak{B}_0$ "--- база пространства $X$ мощности $\mf m$, $\mathfrak{B}$ "--- произвольная база пространства $X$. Выделим из $\mathfrak{B}$ базу $\mathfrak{B}'$ мощности $\mathfrak{m}$.

Назовем пару $(U_a, U_b)$ элементов базы $\mathfrak{B}_0$ \textit{отмеченной}, если существует множество $V \in \mathfrak{B}$ такое, что $U_a\subset V\subset U_b$. Множество всех отмеченных пар имеет мощность, не превосходящую мощности множества $\mathfrak{B}_0^2$, равную $\mf m$. Для каждой отмеченной пары $(U_a, U_b)$ выберем по одному элементу $V \in \mathfrak{B}$, удовлетворяющему условию $U_a \subset V \subset U_b$. Множество $\mathfrak{B}'$ выбранных таким образом элементов базы $\mathfrak{B}$ также имеет мощность, не превосходящую $\mf{m}$. Покажем, что $\mathfrak{B}'$ "--- база пространства $X$.

Зафиксируем точку $x\in X$ и ее окрестность $U(x)$. Так как $\mathfrak{B}_0$ "--- база, то существует такое множество $U_b \in \mathfrak{B}_0$, что $x \in U_b \subset U(x)$. Но базой является также $\mathfrak{B}$, поэтому существует $V \in \mf B$ такое, что $x \in V \subset U_b$. Наконец, существует $U_a \in \mathfrak{B}_0$ такое, что $x \in U_a \subset V$. Пара $(U_a, U_b)$ "--- отмеченная, поэтому существует $V' \in \mathfrak{B}'$, удовлетворяющее условию $U_a \subset V' \subset U_b$, откуда $x \in V' \subset U(x)$. В силу произвольности выбора точки $x$ и окрестности $U(x)$, получено требуемое.
\end{proof}

\subsection{Способы задания топологии}

\begin{example}
    Пусть в множестве $X$ задана система $\{\Gamma_{\alpha}\}_{\alpha \in \mf A} \subset 2^X$, удовлетворяющая следующим условиям:
    \begin{enumerate}
        \item $\bigcup_{\alpha \in \mf A} \Gamma_\alpha = X$
        
        \item Если для некоторой точки $x \in X$ и множеств $\Gamma_\alpha, \Gamma_\beta$ выполнено, что $x \in \Gamma_{\alpha}\cap\Gamma_{\beta}$, то существует множество $\Gamma_{\gamma}$ такое, что $x\in\Gamma_{\gamma}\subset \Gamma_{\alpha}\cap\Gamma_{\beta}$
    \end{enumerate}

    Тогда система $\{\emptyset\} \cup \{\bigcup_{\beta \in \mf B} \Gamma_\beta: \mf B \subset \mf A\}$ образует открытую топологию на $X$, для которой система $\{\Gamma_{\alpha}\}_{\alpha \in \mf A}$ является открытой базой.
\end{example}

\begin{example}
    Пусть $X$ "--- множество, и для каждой точки $x \in X$ задана система $\{U(x)\}$ окрестностей этой точки так что выполнены следующие условия:
    \begin{enumerate}
        \item Для каждой точки задана хотя бы одна окрестность $U(x)$, причем каждая окрестность $U(x)$ точки $x$ содержит эту точку $x$
        
        \item Для любой точки $x \in X$ и ее окрестностей $U_1(x), U_2(x)$ существует окрестность $U_3(x)$ такая, что $U_3(x)\subset U_1(x)\cap U_2(x)$
        
        \item Для любой точки $x \in X$ и ее окрестности $U(x)$ выполнено, что если $y \in U(x)$, то существует окрестность $U(y)$ точки $y$ такая, что $U(y) \subset U(x)$
    \end{enumerate}

    Тогда система множеств, содержащих каждую свою точку вместе с некоторой ее окрестностью, задает открытую топологию на $X$, для которой система окрестностей всех точек является открытой базой.
\end{example}

\begin{note}
    Выше приведены способы задания топологии на произвольном множестве, основанные на построении некоторой меньшей системы множеств. Ниже будет рассмотрен еще один тип систем множеств, задающих топологию.
\end{note}

\begin{definition}
    Пусть $(X, \tau)$ "--- топологическое пространство. Система $\Sigma \subset \tau$ называется \textit{предбазой} пространства $X$, если множества, являющиеся пересечениями конечного набора множеств из $\Sigma$, образуют открытую базу пространства $X$.
\end{definition}

\begin{example}
    Очевидно, всякая база пространства является его предбазой. Кроме того, на числовой прямой $\mathbb R$ бесконечные интервалы вида $(-\infty, a)$, $a \in \R$, и $(b, +\infty)$, $b \in \R$, образуют предбазу, не образуя базы.
\end{example}

\begin{proposition}\label{propcontsubbase}
    Пусть $X, Y$ "--- топологические пространства, $\Sigma$ "--- предбаза прост\-ранства $Y$, $f : X \to Y$ "--- отображение такое, что прообразы множеств из $\Sigma$ являются открытыми множествами. Тогда $f$ непрерывно.
\end{proposition}

\begin{proof}
    Для любых $U_{\alpha_1}, \dotsc, U_{\alpha_s} \in \Sigma$ положим $U_{\alpha_1 \dotsc \alpha_s} := \bigcap_{i=1}^s U_{\alpha_i}$, тогда:
    \[f^{-1}(U_{\alpha_1 \dotsc \alpha_s}) = \bigcap_{i=1}^s f^{-1}(U_{\alpha_i})\]
    
    Значит, множество $f^{-1}(U_{\alpha_1 \dotsc \alpha_s})$ "--- открытое для любых $U_{\alpha_1}, \dotsc, U_{\alpha_s} \in \Sigma$. Но система $\mf B := \{U_{\alpha_1 \dotsc \alpha_s} : U_{\alpha_1}, \dotsc, U_{\alpha_s} \in \Sigma\}$ образует базу в пространстве $Y$, поэтому любое открытое множество $V \subset Y$ представимо в виде объединения некоторого набора множеств из $\mathfrak{B}$, откуда множество $f^{-1}(V)$ "--- тоже открытое. Значит, отображение $f$ непрерывно.
\end{proof}
    
\begin{proposition}\label{propisomsubbase}
    Пусть $X, Y$ "--- топологические пространства, $\Sigma$ "--- предбаза прост\-ранства $X$, $f : X \to Y$ "--- взаимно однозначное отображение. Тогда $f$ является гомеоморфизмом $\lra$ образы множеств из $\Sigma$ под действием $f$ являются открытыми множествами.
\end{proposition}

\begin{proof}
    Аналогично утверждению \ref{propcontsubbase}.
\end{proof}

\begin{example}
    Пусть $X := [0, 1)$ "--- пространство с топологией, базой которой является система $S$ всех полуинтервалов на $X$. Пространство $X$ удовлетворяет первой аксиоме счетности, поскольку для любой точки $x \in X$ система $\{[x, x+r): r \in \Q,~0 < r \le 1 - x\}$ является локальной базой, при этом пространство $X$ не имеет счетной базы по теореме \ref{thmalexandrovurysohn}. Действительно, любая база $S' \subset S$ должна содержать полуинтервал с началом в каждой из точек множества $X$ и потому не может быть счетной.
\end{example}