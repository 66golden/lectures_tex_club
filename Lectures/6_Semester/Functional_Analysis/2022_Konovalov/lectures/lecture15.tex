\begin{theorem}
    (Хан-Банах) Пусть $\displaystyle E$ -- линейное нормированное пространство, $\displaystyle M\subset E$ -- линейное многообразие, $\displaystyle f$ -- линейный ограниченный функционал на $\displaystyle M$. Тогда $\displaystyle \exists \tilde{f} \in E^{*} :$
    \begin{enumerate}
        \item $\displaystyle \left.\tilde{f} \right\vert_{M} =f$.
        \item $\displaystyle \left\Vert \tilde{f}\right\Vert =\Vert f\Vert $.
    \end{enumerate}
    \end{theorem}
    \begin{proof}
    Если $\displaystyle M=E$, то, взяв $\displaystyle \tilde{f} =f$, получим требуемое. Пусть теперь $\displaystyle M\neq E$. Предположим, что $\displaystyle E$ -- вещественное, сепарабельное пространство. Рассмотрим многообразие $\displaystyle M_{1} =M\oplus [ x_{0}]$, где $\displaystyle x_{0} \notin M$. Тогда $\displaystyle \forall y\in M_{1} \hookrightarrow y=x+\alpha x_{0}$, где $\displaystyle x\in M,\ \alpha \in \mathbb{K}$. Определим функционал $\displaystyle f_{1}$ на $\displaystyle M_{1}$ как $\displaystyle f_{1}( y) =f( x) +\alpha f( x_{0})$. Тогда $\displaystyle \Vert f_{1}\Vert \geqslant \Vert f\Vert $, т.к. $\displaystyle f_{1} |_{M} =f$. Покажем, что $\displaystyle \Vert f_{1}\Vert \leqslant \Vert f\Vert $. Если $\displaystyle \alpha =0$, то выполняется неравенство $\displaystyle | f_{1}( y)| =| f( x)| \leqslant \Vert f\Vert \cdotp \Vert x\Vert $. Пусть $\displaystyle \alpha \neq 0$. Достаточно доказать верность неравенства
    \begin{equation*}
    \forall y\in M_{1} \hookrightarrow | f_{1}( y)| =| f( x) +\alpha f( x_{0})| \leqslant \Vert f\Vert \cdotp \Vert x+\alpha x_{0}\Vert .
    \end{equation*}
    Обозначим $\displaystyle z:=\dfrac{x}{\alpha }$. Если неравенство верно, то
    \begin{gather*}
    \forall z\hookrightarrow -\Vert f\Vert \cdotp \Vert z+x_{0}\Vert \leqslant f( z) +f( x_{0}) \leqslant \Vert f\Vert \cdotp \Vert z+x_{0}\Vert \Leftrightarrow \\
    \Leftrightarrow -\Vert f\Vert \cdotp \Vert z+x_{0}\Vert -f( z) \leqslant f( x_{0}) \leqslant \Vert f\Vert \cdotp \Vert z+x_{0}\Vert \ -\ f( z) .
    \end{gather*}
    Существование такого $\displaystyle f( x_{0})$ равносильно
    \begin{gather*}
    \sup _{z} -\Vert f\Vert \cdotp \Vert z+x_{0}\Vert -f( z) \leqslant \inf_{z}\Vert f\Vert \cdotp \Vert z+x_{0}\Vert -f( z) \Leftrightarrow \\
    \Leftrightarrow \forall z_{1} ,z_{2} \in M\hookrightarrow -\Vert f\Vert \cdotp \Vert z_{1} +x_{0}\Vert -f( z_{1}) \leqslant \Vert f\Vert \cdotp \Vert z_{2} +x_{0}\Vert -f( z_{2}) \Leftrightarrow \\
    f( z_{2}) -f( z_{1}) \leqslant \Vert f\Vert (\Vert z_{1} +x_{0}\Vert +\Vert z_{2} +x_{0}\Vert ) .
    \end{gather*}
    Последнее неравенство следует из неравенства
    \begin{equation*}
    | f( z_{2}) -f( z_{1})| =f( z_{2} -z_{1}) \leqslant \Vert f\Vert \cdotp \Vert z_{2} -z_{1}\Vert \leqslant \Vert f\Vert (\Vert z_{1} +x_{0}\Vert +\Vert z_{2} +x_{0}\Vert ) ,
    \end{equation*}
    которое выполнено. Таким образом, существует такое число $\displaystyle f( x_{0})$, что выполняется неравенство $\displaystyle | f( x) +\alpha f( x_{0})| \leqslant \Vert f\Vert \cdotp \Vert x+\alpha x_{0}\Vert $, и, следовательно, $\displaystyle \Vert f_{1}\Vert \leqslant \Vert f\Vert $.
    
    Продолжим этот процесс, определив $\displaystyle M_{2} =M_{1} \oplus [ x_{1}]$, где $\displaystyle x_{1} \notin M_{1}$, если $\displaystyle M_{1} \neq E$, и т.д. Если за конечное количество шагов $\displaystyle k$ получилось, что $\displaystyle M_{k} =E$, то функционал $\displaystyle f_{k} |_{M} =f$, и $\displaystyle \Vert f_{k}\Vert =\Vert f\Vert $.
    
    Иначе, получим последовательность линейных многообразий $\displaystyle \{M_{n}\}_{n=1}^{\infty }$, что $\displaystyle \forall n\in \mathbb{N} \hookrightarrow M_{n} \neq E$. Из сепарабельности пространства $\displaystyle E$ следует, что $\displaystyle \exists X=\{x_{n}\}_{n=0}^{\infty }$, что $\displaystyle X$ -- всюду плотно. Тогда построим последовательность $\displaystyle \{M_{n}\}_{n=1}^{\infty }$ следующим образом: по очереди перебирая $\displaystyle x_{i} \in X$, проверяем, лежит ли $\displaystyle x_{i}$ в $\displaystyle M_{j}$. Если да, то переходим к $\displaystyle x_{i+1}$. Иначе, строим $\displaystyle M_{j+1} =M_{j} \oplus [ x_{i}]$. Обозначим $\displaystyle M_{\infty } :=\bigcup _{n=0}^{\infty } M_{n}$, где $\displaystyle M_{0} =M$. Так как $\displaystyle X\subset M_{\infty }$, то $\displaystyle M_{\infty }$ всюду плотно в $\displaystyle E$. Определим на $\displaystyle M_{\infty }$ функционал $\displaystyle f_{\infty } :\ f_{\infty } |_{M_{n}} =f_{n} \ \forall n\in \mathbb{N}$. Тогда $\displaystyle \Vert f_{\infty }\Vert =\Vert f\Vert $.
    
    Вспомним Теорему 5.3, в которой утверждается, что, если $\displaystyle E_{1}$ -- линейное нормированное пространство, $\displaystyle E_{2}$ -- банахово пространство, $\displaystyle D( A) \subset E_{1}$ -- линейное многообразие, являющееся всюду плотным в $\displaystyle E_{1}$, $\displaystyle A:D( A)\rightarrow E_{2}$ -- линейный ограниченный функционал, то $\displaystyle \exists !\tilde{A} \in \mathcal{L}( E_{1} ,\ E_{2})$, что $\displaystyle \tilde{A} |_{D( A)} =A$, $\displaystyle \left\Vert \tilde{A}\right\Vert =\Vert A\Vert $.
    
    Таким образом, $\displaystyle f_{\infty }$ удовлетворяет условиям теоремы, следовательно, $\displaystyle \exists !\tilde{f}$, являющееся продолжением $\displaystyle f_{\infty }$ на все $\displaystyle E_{1}$.
    \end{proof}
    \begin{note}
    Вообще говоря, продолжение на все пространство не единственно.
    \end{note}
    \begin{exercise}
    Доказать теорему Хана-Банаха для случая, когда $\displaystyle E$ не является сепарабельным.
    \end{exercise}
    \begin{corollary}
    Пусть $\displaystyle E$ -- линейное нормированное пространство.
    \begin{enumerate}
        \item Пусть $\displaystyle M\subset E$ -- линейное многообразие, $\displaystyle M\neq E$, $\displaystyle x_{0} \notin \overline{M}$. Тогда $\displaystyle \exists f\in E^{*}$, что $\displaystyle f|_{M} =0,\ f( x_{0}) =1,\ \Vert f\Vert =\dfrac{1}{\rho ( x_{0} ,\ M)}$.
        \item $\displaystyle \forall x\in E,\ x\neq 0\ \exists f\in E^{*}$, что $\displaystyle \Vert f\Vert =1,\ f( x) =\Vert x\Vert $.
        \item Если $\displaystyle f( x) =0\ \forall f\in E^{*}$, то $\displaystyle x=0$. Другими словами, если $\displaystyle f( x) =f( y) \ \forall f\in E^{*}$, то $\displaystyle x=y$.
        \item $\displaystyle \forall x\in E\hookrightarrow \Vert x\Vert =\sup _{\Vert f\Vert =1}| f( x)| $.
    \end{enumerate}
    \end{corollary}
    \begin{proof} ~
    \begin{enumerate}
        \item Пусть $\displaystyle M_{1} =M\oplus [ x_{0}]$. Тогда $\displaystyle \forall y\in M_{1} \hookrightarrow y=x+\alpha x_{0}$, где $\displaystyle x\in M,\ \alpha \in \mathbb{K}$. Определим $\displaystyle f( y) =f( x) +\alpha f( x_{0}) =0+\alpha \cdotp 1$. Тогда по теореме Хана-Банаха существует и единственно продолжение $\displaystyle \tilde{f} \in E^{*} :\ \tilde{f} |_{M_{1}} =f,\ \left\Vert \tilde{f}\right\Vert =\Vert f\Vert $.
    
        Докажем, что $\displaystyle \Vert f\Vert =\dfrac{1}{\rho ( x_{0} ,\ M)}$. Сперва, найдем верхнюю грань $\displaystyle \Vert f\Vert $:
        \begin{equation*}
        \dfrac{\Vert f( y)\Vert }{\Vert y\Vert } =\dfrac{| \alpha | }{\Vert y\Vert } =\begin{cases}
        \leqslant \dfrac{1}{\rho ( x_{0} ,\ M)} & ,\ \alpha =0\\
        \dfrac{1}{\left\Vert \dfrac{y}{\alpha }\right\Vert } & ,\ \alpha \neq 0
        \end{cases} .
        \end{equation*}
        Далее, $\displaystyle \left\Vert \dfrac{y}{\alpha }\right\Vert =\left\Vert \dfrac{x}{\alpha } +x_{0}\right\Vert \geqslant \rho ( x_{0} ,\ M) \Rightarrow \dfrac{1}{\left\Vert \dfrac{y}{\alpha }\right\Vert } \leqslant \dfrac{1}{\rho ( x_{0} ,\ M)}$.
        
        Так как $\displaystyle \rho ( x_{0} ,\ M) =\inf_{x\in M} \rho ( x_{0} ,\ x)$, то $\displaystyle \exists \{z_{n}\} \subset M:\ \rho ( x_{0} ,\ z_{n})\xrightarrow[n\rightarrow \infty ]{} \rho ( x_{0} ,\ M)$.
        \begin{equation*}
        \dfrac{| f( y)| }{\Vert y\Vert } =\dfrac{| \alpha | }{\Vert y\Vert } =\dfrac{1}{\left\Vert \dfrac{x}{\alpha } +x_{0}\right\Vert } ,\ \alpha \neq 0.
        \end{equation*}
        Так как $\displaystyle \dfrac{x}{\alpha }$ принимает всевозможные значения из $\displaystyle M$, то
        \begin{equation*}
        \sup _{x\in M}\dfrac{1}{\left\Vert \dfrac{x}{\alpha } +x_{0}\right\Vert } =\lim _{n\rightarrow \infty }\dfrac{1}{\Vert z_{n} +x_{0}\Vert } =\dfrac{1}{\rho ( x_{0} ,\ M)} .
    \end{equation*}
        \item Пусть $\displaystyle M=\{0\} ,\ x_{0} =\dfrac{1}{\Vert x\Vert } x$, где $\displaystyle x\neq 0$. Тогда из предыдущего пункта существует функционал $\displaystyle f$, что $\displaystyle f( x_{0}) =f\left(\dfrac{x}{\Vert x\Vert }\right) =1\Rightarrow f( x) =\Vert x\Vert $, а $\displaystyle \Vert f\Vert =\dfrac{1}{\rho ( x_{0} ,\ M)} =1$.
    \end{enumerate}
    \end{proof}