\subsection{Последовательности де Брёйна}

\begin{definition}
	Пусть дано $n \in \N$ и алфавит $\Sigma = \{0, 1\}$. Составим словарь из двоичных слов длины $n$. Тогда, суммарное число символов в словаре будет $2^n \cdot n$. \textit{Последовательностью де Брёйна} называется такая последовательность из нулей и единиц длины $N$, что \underline{любое} подслово длины $n$ будет уникальным и соответствующим какому-то слову из словаря выше.
\end{definition}

\begin{note}
	То есть, если мы возьмём <<рамку>> на $n$ элементов и пройдёмся по последовательности де Брёйна от начала и до конца, то в рамке будет получаться любое слово из словаря, причём один раз
\end{note}

\begin{proposition}
	Длина последовательности де Брёйна всегда $2^n + n - 1$.
\end{proposition}

\begin{proof}
	Рассмотрим более подробно идею с окном. У нас должны встретиться все $2^n$ слов и только они, а чтобы была возможность показать первое - нужно ещё $n - 1$ символ. Отсюда получаем длину.
\end{proof}

\subsubsection*{Построение последовательностей де Брёйна}

\begin{theorem}
	Последовательности де Брёйна существуют для любого $n \in \N$.
\end{theorem}

\begin{proof}
	Как доказательство этой теоремы, приведём 2 способа построения таких последовательностей:
	
	\begin{enumerate}
		\item Правило <<\textit{0 лучше 1}>>: начнём последовательность с $n$ единиц. Затем будем руководствоваться следующим соображением: ставим следующим символом $0$, если такое слово-суффикс ещё не встречалось в последовательности. Иначе $1$.
		
		\begin{example}
			\(n = 3 \Ra N = 10 \Ra 1110001011\)
		\end{example}
	
		\item Введём понятия \textit{графа де Брёйля} $G = (V, E)$. Это ориентированный граф, у которого вершины $V = \{0, 1\}^{n - 1}$, а ребро $(v, u) \in E$ тогда и только тогда, когда суффикс слова $v$ длины $n - 2$ является префиксом $u$.
		
		Для орграфов есть аналогичное условие эйлеровости:
		\[
			\forall v \in V\ \indeg v = \outdeg v > 0
		\]
		Понятно, что у любой вершины построенного нами графа входящая и исходящая степень равны 2. Значит, он эйлёров и есть эйлеров обход. Если мы возьмём за начало последовательности первую вершину обхода, а по мере прохождения будем брать только последние элементы следующих вершин, то получим в точности последовательность де Брёйна.
	\end{enumerate}
\end{proof}