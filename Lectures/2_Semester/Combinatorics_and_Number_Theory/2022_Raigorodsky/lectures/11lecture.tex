\subsection{Равномерное распределение последовательностей}

\begin{note}
	До конца данной темы, мы будем обозначать последовательность чисел как $(x_n)_{n = 0}^\infty$. При этом $x_n \in \R$, если не оговорено иного.
	
	Сделано это для того, чтобы за $\{x_n\}$ обозначать здесь дробную часть числа $x_n$.
\end{note}

\begin{definition}
	Последовательность $(x_n)_{n = 0}^\infty$ называется \textit{равномерно распределённой по модулю} 1, если выполнено следующее:
	\[
		\forall a, b \in [0; 1],\ a < b \quad \frac{|\{n \in \{0, \ldots, N - 1\} \colon \{x_n\} \in [a; b)\}|}{N} \xrightarrow[N \to \infty]{} b - a
	\]
\end{definition}

\begin{note}
	Интуитивное понятие такое: берём первые $N$ чисел, смотрим на дробную часть каждого. Если она попала в полуинтервал $[a; b)$, то учитываем это число, иначе нет.
\end{note}

\begin{proposition}
	Эквивалентным определением равномерной распределённости будет утверждение
	\[
		\forall \gamma \in [0; 1) \quad \frac{|\{n \in \{0, \ldots, N - 1\}\colon \{x_n\} < \gamma\}|}{N} \xrightarrow[N \to \infty]{} \gamma
	\]
\end{proposition}

\begin{proof}~
	\begin{itemize}
		\item Из изначального в новое определение очевидно: просто положим $a = 0, b = \gamma$
		
		\item Из нового определения тоже понятно, как получить изначальное: выберем $\gamma_1 = a, \gamma_2 = b$ и вычтем одно множество-индикатор из другого. Если расписать предел доли, то получим ровно $b - a$ в пределе.
	\end{itemize}
\end{proof}

\begin{example}
	Рассмотрим $x_n = \sqrt{n}$. Докажем, что она равномерно распределена, пользуясь эквивалентным определением:
	\[
		x_0, \ldots, x_N = \sqrt{0}, \ldots, \sqrt{N}
	\]
	Сколько среди чисел выше полных квадратов (то есть заведомо попадающих в множество-индикатор, потому что не имеют дробной части)? Их ровно $\floor{\sqrt{N}} + 1$. Какие ещё числа подойдут, если мы зафиксировали $\gamma$? Ну, например, те, которые отступили от целого корня вправо не более чем на $\gamma$. Значит, если целый корень имел значение $k$, то нам подойдут также $x_n$ с номерами в полуинтервале $[k^2; (k + \gamma)^2)$. Количество номеров внутри этого полуинтервала не превышает $2k\gamma + \gamma^2$. Более того, их не меньше $2k\gamma + \gamma^2 - 1$ и из-за этого можно сказать, что их $2k\gamma + O(1)$ (Интуитивно это можно понять, если начертить прямую с отметками $k^2$ и $(k + \gamma)^2$). Теперь, распишем нашу долю:
	\begin{multline*}
		\frac{|n \in \{0, \ldots, N - 1\} \colon \{x_n\} < \gamma|}{N} = \frac{\suml_{k = 0}^{\floor{\sqrt{N - 1}}} (2k\gamma + O(1))}{N} =
		\\
		\frac{2\gamma}{N} \cdot \frac{\floor{\sqrt{N - 1}} (\floor{\sqrt{N - 1}} + 1)}{2} + \frac{\floor{\sqrt{N - 1}} \cdot O(1)}{N} =
		\\
		\frac{\gamma \floor{\sqrt{N - 1}}(\floor{\sqrt{N - 1}} + 1)}{N} + O\left(\frac{\floor{\sqrt{N}}}{N}\right) \xrightarrow[N \to \infty]{} \gamma + 0 = \gamma
	\end{multline*}
\end{example}

\begin{example}
	Рассмотрим $x_n = \lambda^n,\ \lambda < 1$. В таком случае последовательность очевидно не равномерно распределена, так как
	\[
		\liml_{n \to \infty} x_n = 0
	\]
\end{example}

\begin{example}
	Рассмотрим последовательность из предыдущего примера, но с $\lambda > 1$.
	\begin{enumerate}
		\item Пусть нам так повезло, что $\lambda$ - один из корней квадратного уравнения $x^2 + px + q = 0$ с целыми коэффициентами, причём второй корень $\theta \in (0; 1)$. Рассмотрим последовательность $y_n = \lambda^n + \theta^n$. Она удовлетворяет линейному рекуррентному соотношению вида
		\[
			y_{n + 2} + py_{n + 1} + qy_n = 0
		\]
		При этом
		\[
			\System{
				&{y_0 = \lambda^0 + \theta^0 = 2 \in \Z}
				\\
				&{y_1 = \lambda^1 + \theta^1 = -p \in \Z}
			}
		\]
		То есть $y_n \in \Z$ для любого $n$. Стало быть
		\[
			\{\lambda^n + \theta^n\} = 0 \lra \{\lambda^n\} = 1 - \theta^n
		\]
		Значит, при таких условиях $x_n$ не будет равномерно распределённой последовательностью
		
		\item Открытой проблемой на сегодняшний день является случай при $\lambda = (3/2)$. Неизвестно ничего: ни плотность множества дробных частей, ни распределение.
	\end{enumerate}
\end{example}

\begin{theorem}
	Последовательность $(x_n)_{n = 0}^\infty$ равномерно распределена по модулю 1 тогда и только тогда, когда для любой непрерывной на $[0; 1]$ функции $f$ верно следующее:
	\[
		\frac{1}{N} \suml_{n = 0}^{N - 1} f(\{x_n\}) \xrightarrow[N \to \infty]{} \int_0^1 f(x)dx
	\]
\end{theorem}

\begin{definition}
	Функция $g(x) \colon [0; 1] \to \R$ называется \textit{ступенчатой}, если её можно описать как
	\[
		g(x) = c_1 \I_{[0; a_1)} + c_2 \I_{[a_1; a_2)} + \ldots + c_{n + 1} \I_{[a_n; 1]}
	\]
	где $P \colon 0 = a_0 < a_1 < \ldots < a_n < a_{n + 1} = 1$ - разбиение отрезка $[0; 1]$, а $\I_{[a; b]}$ --- функция-индикатор
\end{definition}

\begin{proof}
	\begin{enumerate}
		\item Для начала докажем утверждение не для непрерывных функций, а ступенчатых. Если мы как-то докажем, что
		\[
			\frac{1}{N} \suml_{n = 0}^{N - 1} \I_{[a; b)} (\{x_n\}) \xrightarrow[N \to \infty]{} \int_0^1 \I_{[a; b)} dx = b - a
		\]
		то и общее утверждение про ступенчатую функцию станет очевидным (в силу аддитивности интеграла).
		
		Что из себя представляет выражение слева? На самом деле, это то же самое выражение, что было в равномерном распределении:
		\[
			\frac{1}{N} \suml_{n = 0}^{N - 1} \I_{[a; b)} (\{x_n\}) = \frac{|\{n \in \{0, \ldots, N - 1\} \colon \{x_n\} \in [a; b)\}|}{N}
		\]
		Отсюда уже очевидно следует равносильность утверждений.
		
		\item Для доказательства исходной теоремы, воспользуемся одним утверждением из математического анализа:
		\begin{proposition}
			$\forall \eps > 0$ существуют 2 ступенчатые функции $f_1, f_2$ такие, что
			\[
				\forall x \ f_1(x) \le f(x) \le f_2(x); \ \ \int_0^1 (f_2(x) - f_1(x))dx \le \eps
			\]
		\end{proposition}
		Аналогично мы можем зажать ступенчатую функцию непрерывными. Теперь мы можем заняться непосредственно доказательством теоремы
		\begin{itemize}
			\item $\Ra$ Зафиксируем $\eps > 0$. Тогда, мы можем написать следующую цепочку неравенств:
			\begin{multline*}
				\int_0^1 f(x)dx - \eps \le \int_0^1 f_2(x)dx - \eps \le \left(\eps + \int_0^1 f_1(x)dx\right) - \eps = \liml_{N \to \infty} \frac{1}{N} \suml_{n = 0}^{N - 1} f_1(\{x_n\}) \le
				\\
				\varliminf\limits_{N \to \infty} \suml_{n = 0}^{N - 1} f(\{x_n\}) \le \varlimsup\limits_{N \to \infty} \suml_{n = 0}^{N - 1} f(\{x_n\}) \le
				\\
				\liml_{N \to \infty} \frac{1}{N} \suml_{n = 0}^{N - 1} f_2(\{x_n\}) = \int_0^1 f_2(x)dx \le \int_0^1 f_1(x)dx + \eps \le \int_0^1 f(x)dx + \eps
			\end{multline*}
			
			\item $\La$ Нам нужно показать, что
			\[
				\frac{1}{N} \suml_{n = 0}^{N - 1} \I_{[a; b)}(\{x_n\}) \xrightarrow[N \to \infty]{} b - a
			\]
			Положим $f = \I_{[a; b)}$. Для неё существуют непрерывные $g_1, g_2$ из утверждения выше. Зафиксируем $a, b \in [0; 1],\ a < b,\ \eps > 0$. Тогда
			\begin{multline*}
				b - a - \eps = \int_0^1 \I_{[a; b)}(x)dx - \eps \le \int_0^1 g_2(x)dx - \eps \le \int_0^1 g_1(x)dx =
				\\
				\liml_{N \to \infty} \suml_{n = 0}^{N - 1} \frac{1}{N} g_1(\{x_n\}) \le \varliminf\limits_{N \to \infty} \suml_{n = 0}^{N - 1} \I_{[a; b)}(\{x_n\}) \le \frac{1}{N} \suml_{n = 0}^{N - 1} \I_{[a; b)}(\{x_n\}) \le
				\\
				\varlimsup\limits_{N \to \infty} \suml_{n = 0}^{N - 1} \I_{[a; b)}(\{x_n\}) \le \liml_{N \to \infty} \suml_{n = 0}^{N - 1} g_2(\{x_n\}) = \int_0^1 g_2(x)dx \le \int_0^1 g_1(x)dx + \eps \le b - a + \eps
			\end{multline*}
		\end{itemize}
	\end{enumerate}
\end{proof}

\begin{corollary}
	Последовательность $(x_n)_{n = 0}^\infty$ равномерно распределена по модулю 1 тогда и только тогда, когда для любой интегрируемой по Риману на $[0; 1)$ функции $f$ верно следующее:
	\[
		\frac{1}{N} \suml_{n = 0}^{N - 1} f(\{x_n\}) \xrightarrow[N \to \infty]{} \int_0^1 f(x)dx
	\] 
\end{corollary}

\begin{corollary}
	Последовательность $(x_n)_{n = 0}^\infty$ равномерно распределена тогда и только тогда, когда для любой комплекснозначной функции $f$, которая периодична с периодом 1 верно, что
	\[
		\frac{1}{N} \suml_{n = 0}^{N - 1} f(x_n) \xrightarrow[N \to \infty]{} \int_0^1 f(x)dx
	\]
	где интеграл - это надо взять отдельно интеграл от реальной и мнимой части, интеграл мнимой части домножить на $i$ и сложить с реальной.
\end{corollary}

\begin{corollary} (из последнего следствия)
	Если $(x_n)_{n = 0}^\infty$ равномерно распределена, то
	\[
		\forall m \in \Z \bs \{0\} \quad \frac{1}{N} \suml_{n = 0}^{N - 1} e^{2\pi im x_n} \xrightarrow[N \to \infty]{} 0
	\]
\end{corollary}

\begin{proof}
	Понятно, что $f(x) = e^{2\pi im x_n}$ - периодическая комплекснозначная функция с периодом $1$. Тогда, нам просто надо посчитать интеграл:
	\[
		\int_0^1 e^{2\pi im x}dx = \frac{1}{2\pi im} e^{2\pi im x} \Big|_0^1 = \frac{1}{2\pi im} (e^{2\pi im} - 1) = 0
	\]
\end{proof}

\begin{theorem} (Критерий Вейля)
	Последовательность $(x_n)_{n = 0}^\infty$ равномерно распределена тогда и только тогда, когда верно следующее:
	\[
		\forall m \in \Z \bs \{0\} \quad \frac{1}{N} \suml_{n = 0}^{N - 1} e^{2\pi im x_n} \xrightarrow[N \to \infty]{} 0
	\]
\end{theorem}

\begin{example}
	Рассмотрим последовательность $x_n = \{\alpha n\}$. Если пытаться определить её распределение без критерия Вейля, то это будет крайне тяжело. Однако, применим теорему:
	\begin{itemize}
		\item $\alpha \in \Q$. В таком случае, $(x_n)_{n = 0}^\infty$ будет ходить по кругу конечного числа значений. Понятно, что это не может быть равномерно распределённой последовательностью.
		
		\item $\alpha \in \iR$, то она равномерно распределена. Действительно, по критерию Вейля
		\[
			\frac{1}{N} \suml_{n = 0}^{N - 1} e^{2 \pi im n\alpha} = \frac{1}{N} \frac{e^{2\pi im \alpha N} - 1}{e^{2\pi im \alpha} - 1}
		\]
		Посмотрим на модуль этой суммы:
		\[
			\left|\frac{1}{N} \suml_{n = 0}^{N - 1} e^{2 \pi im n\alpha}\right| = \left|\frac{1}{N} \frac{e^{2\pi im \alpha N} - 1}{e^{2\pi im \alpha} - 1}\right| \le \frac{1}{N} \frac{2}{C},\ C \neq 0
		\]
		Очевидно, что при стремлении $N$ к бесконечности, модуль стремится к нулю.
	\end{itemize}
\end{example}