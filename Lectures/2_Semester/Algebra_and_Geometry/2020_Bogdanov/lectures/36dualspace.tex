\subsection{Сопряженное пространство}

\textbf{В данном разделе} будем считать, что $V$ "--- евклидово пространство.

\begin{definition}
	\textit{Сопряженным к V пространством} называется пространство линейных функционалов на $V$. Обозначение "--- $V^*$.
\end{definition}

\begin{theorem}
	Для каждого $\overline{v} \in V$ положим $f_{\overline{v}}(\overline{u}) := (\overline{v}, \overline{u})$. Тогда сопоставление $\overline{v} \mapsto f_{\overline{v}}$ осуществляет изоморфизм между $V$ и $V^*$.
\end{theorem}

\begin{proof}
	Проверим, что заданное сопоставление линейно:
	\begin{gather*}
		f_{\overline{v_1} + \overline{v_2}}(\overline{u}) = (\overline{v_1} + \overline{v_2}, \overline{u}) = (\overline{v_1}, \overline{u}) + (\overline{v_2}, \overline{u}) = f_{\overline{v_1}}(\overline{u}) + f_{\overline{v_2}}(\overline{u})\\
		f_{\alpha\overline{v}}(\overline{u}) = (\alpha\overline{v}, \overline{u}) = \alpha(\overline{v}, \overline{u}) = \alpha f_{\overline{v}}(\overline{u})
	\end{gather*}
	
	Поскольку $\dim{V} = \dim{V^*}$ и отображение линейно, то нам достаточно проверить его инъективность, что эквивалентно условию $\forall \overline{v} \in V, \overline{v} \ne \overline{0}: f_{\overline{v}} \ne 0$. Но это условие выполнено в силу положительной определенности скалярного произведения: $\forall \overline{v} \in V, \overline{v} \ne \overline{0}: f_{\overline{v}}(\overline{v}) \hm= (\overline{v}, \overline{v}) > 0$.
\end{proof}

\begin{corollary}
	Пусть $U, W \le V$. Тогда:
	\begin{enumerate}
		\item $(U^\perp)^\perp = U$
		\item $(U + W)^\perp = U^\perp \cap W^\perp$
		\item $(U \cap W)^\perp = U^\perp + W^\perp$
	\end{enumerate}
\end{corollary}

\begin{proof}
	Рассмотрим изоморфизм $\Theta: V \rightarrow V^*$ из предыдущей теоремы и заметим, что $\Theta(U^\perp) = \{f_{\overline{v}} \in V^*~|~\overline{v} \in U^\perp\} = \{f \in V^*~|~\forall \overline{u} \in U: f(\overline{u}) = 0\} = U^0$ для любого подпространства $U \le V$.
	
	\begin{enumerate}
		\item Докажем данную формулу непосредственно. Так как $\dim{U^\perp} \hm{=} \dim{V} - \dim{U}$ и $\dim{(U^\perp)^\perp} = \dim{V} - \dim{U^\perp}$, то $\dim{U} \hm{=} \dim{(U^\perp)^\perp}$. При этом, с другой стороны, $U \le (U^\perp)^\perp$, поскольку $\forall \overline{u} \in U: \forall \overline{v} \in U^\perp: \overline{u} \perp \overline{v}$. Значит, $U = (U^\perp)^\perp$.
		
		\item Поскольку $(U + W)^0 = U^0 \cap W^0$, то, применяя $\Theta$ к обеим частям равенства, получим $(U + W)^\perp = U^\perp \cap W^\perp$.
		
		\item Поскольку $(U \cap W)^0 = U^0 + W^0$, то, применяя $\Theta$ к обеим частям равенства, получим $(U \cap W)^\perp = U^\perp + W^\perp$.\qedhere
	\end{enumerate}
\end{proof}

\begin{note}
	В эрмитовом пространстве ситуация почти аналогична: если рассмотреть сопоставление $\overline{v} \mapsto f_{\overline{v}}$, то функционал $f_{\overline{v}}$ уже не линеен, а сопряженно-линеен. Если же положить $g_{\overline{v}}(\overline{u}) := (\overline{u}, \overline{v})$, то функционал $g_{\overline{v}}$ линеен, но сопоставление $\overline{v} \mapsto g_{\overline{v}}$ не линейно, а сопряженно-линейно. Оно не является изоморфизмом, но является сопряженно-линейной биекцией, называемой \textit{антиизоморфизмом}.
	
	Легко проверить, что при антиизоморфизме $\Theta$ остается справедливым равенство $\Theta(U^\perp) = U^0$. Следовательно, в эрмитовом пространстве $V$ остается справедливым, что для любых подпространств $U, W \le V$ выполнены равенства $(U^\perp)^\perp = U$, $(U + W)^\perp = U^\perp \cap W^\perp$ и $(U \cap W)^\perp = U^\perp + W^\perp$.
\end{note}