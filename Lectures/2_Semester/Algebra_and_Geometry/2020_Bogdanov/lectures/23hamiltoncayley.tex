\subsection{Теорема Гамильтона-Кэли}

\textbf{До конца раздела} зафиксируем линейное пространство $V$ над полем $F$ и положим $n := \dim{V}$. Отметим также, что в следующих разделах на рассматриваемый оператор $\phi \hm{\in} \mathcal{L}(V)$ часто будет налагаться требование, что $\Chi_\phi$ раскладывается в произведение линейных сомножителей, то есть имеет следующий вид при некоторых $\lambda_1, \dotsc, \lambda_k \in F$ и $\alpha_1, \dotsc, \alpha_k \in \N$ таких, что $\sum_{i = 1}^k\alpha_i = n$:
\begin{equation}\tag{$\star$}\label{charpol}
	\Chi_\phi(\lambda) = \prod_{i = 1}^k(\lambda_i - \lambda)^{\alpha_i}
\end{equation}

\begin{definition}
	Введем следующие обозначения:
	\begin{itemize}
		\item Для произвольных $\phi \in \mathcal{L}(V)$ и $\mu \in F$ положим $\phi_\mu := \phi - \mu$
		\item Для произвольных $A \in M_n(F)$ и $\mu \in F$ положим $A_\mu := A - \mu E$
	\end{itemize}
\end{definition}

\begin{proposition}
	Пусть $\phi \in \mathcal{L}(V)$. Тогда $U \le V$ является инвариантным относительно $\phi$ $\Leftrightarrow$ $U$ является инвариантным относительно $\phi_\mu$.
\end{proposition}

\begin{proof}~
	\begin{itemize}
		\item[$\ra$] Если $U$ инвариантно относительно $\phi$, то $\phi_\mu(U) \le \phi(U) + \mu U \le U$
		\item[$\la$] Если $U$ инвариантно относительно $\phi_\mu$, то $\phi(U) = (\phi_\mu + \mu)(U) \le \phi_\mu(U) + \mu U \le U$\qedhere
	\end{itemize}
\end{proof}

\begin{proposition}
	Пусть оператор $\phi \in \mathcal{L}(V)$ имеет собственное значение. Тогда существует инвариантное относительно $\phi$ подпространство $U \le V$ размерности $n - 1$.
\end{proposition}

\begin{proof}
	Пусть $\mu \in F$ "--- собственное значение оператора $\phi$. Тогда $\phi_\mu$ "--- вырожденный оператор, то есть $\dim{\im{\phi_\mu}} \le n - 1$. Дополним базис в $\im{\phi_\mu}$ до базиса в некотором подпространстве $U \le V$ размерности $n - 1$, тогда $U$ инвариантно относительно $\phi_\mu$ и, следовательно, относительно $\phi$.
\end{proof}

\begin{theorem}
	Пусть $\phi \in \mathcal{L}(V)$, и $\Chi_\phi$ имеет вид $\eqref*{charpol}$. Тогда в $V$ существует такой базис $e = (\overline{e_1}, \dots, \overline{e_n})$, что для любого $i \in \{1, \dots, n\}$ подпространство $\langle\overline{e_1}, \dots, \overline{e_i}\rangle \le V$ инвариантно относительно $\phi$.
\end{theorem}

\begin{proof}
	Проведем индукцию по $n$. База, $n = 1$, тривиальна, докажем переход, $n > 1$. Многочлен $\Chi_\phi$ имеет корни, поэтому существует инвариантное относительно $\phi$ подпространство $U \le V$ размерности $n - 1$. Положим $\psi := \phi|_U \hm{\in} \mathcal{L}(U)$, тогда $\Chi_\psi\mid \Chi_\phi$, поэтому $\Chi_\psi$ также имеет вид $\eqref*{charpol}$, и к нему применимо предположение индукции. Выберем подходящий базис $e'$ в $U$ и дополним его до базиса в $V$, получим требуемое.
\end{proof}

\begin{corollary}
	Пусть $\phi \in \mathcal{L}(V)$, и $\Chi_\phi$ имеет вид $\eqref*{charpol}$. Тогда в $V$ существует базис $e$, в котором матрица преобразования $\phi$ имеет верхнетреугольный вид.
\end{corollary}

\begin{proof}
	В базисе из теоремы выше матрица оператора $\phi$ имеет верхнетреугольный вид, что и требовалось.
\end{proof}

\begin{note}
	В базисе $e$ из утверждения выше матрица оператора $\phi$ не только имеет верхнетреугольный вид, но и значения на ее диагонали в точности совпадают с набором корней многочлена $\Chi_\phi$ с учетом кратности.
\end{note}

\begin{theorem}[Гамильтона-Кэли]
	Для любого оператора $\phi \in \mathcal{L}(V)$ выполнено следующее равенство:
	\[\Chi_\phi(\phi) = 0\]
\end{theorem}

\begin{proof}[Доказательство для случая, когда $\Chi_\phi$ имеет вид $\eqref*{charpol}$]
	Выберем базис $e$ в $V$, в котором для любого $i \in \{1, \dots, n\}$ подпространство $V_i := \langle\overline{e_1}, \dots, \overline{e_i}\rangle \le V$ инвариантно относительно $\phi$, тогда матрица оператора $\phi$ в этом базисе имеет верхнетреугольный вид. Пусть $\phi \leftrightarrow_e \diag(\lambda_1, \dotsc, \lambda_n) \in M_n(F)$. Покажем, что тогда $\phi_{\lambda_i}(V_i) \le V_{i - 1}$ для любого $i \in \{1, \dots, n\}$. Действительно, так как $V_i$ инвариантно относительно $\phi$, то оно также инвариантно относительно $\phi_{\lambda_i}$, и матрица сужения $\psi := \phi_{\lambda_i}|_{V_i} \in \mathcal{L}(V_i)$ имеет следующий вид:
	\[\psi \xleftrightarrow[(\overline{e_1}, \dots, \overline{e_i})]{} \begin{pmatrix}
		\lambda_1 - \lambda_i & * & \dots & *\\
		0 & \lambda_2 - \lambda _i & \dots & *\\
		\vdots & \vdots & \ddots & \vdots\\
		0 & 0 & \dots & 0
	\end{pmatrix}
	\]
	
	Последняя координата у образов всех базисных векторов нулевая, поэтому $\psi(V_i) \le V_{i - 1}$. Следовательно, выполнены следующие включения:
	\[\Chi_\phi(\phi)(V) = (\phi_{\lambda_1}\dots\phi_{\lambda_n})(V) \le (\phi_{\lambda_1}\dots\phi_{\lambda_{n - 1}})(V_{n - 1}) \le \phi_{\lambda_1}(V_1) \le V_0 = \{\overline{0}\}\]
	
	Таким образом, $\Chi_\phi(\phi) = 0$, что и требовалось.
\end{proof}

\begin{note}
	Теорема Гамильтона-Кэли справедлива и в общем случае. Это можно доказать, воспользовавшись фактом, который будет доказан позднее: если $F$ "--- поле и $P \in F[x]$, то существует надполе $K \supset F$ такое, что $P$ раскладывается на линейные сомножители над $K$. Тогда $A \in M_n(K)$, и в $K$ теорема верна, но поскольку все действия в вычислении $\Chi_A(A)$ происходят в $F$, то и в $F$ теорема верна.
\end{note}