\subsection{Тензорное произведение пространств}

\begin{definition}
	Пусть $U, V$ "--- линейные пространства над полем $F$. \textit{Тензорным произведением пространств $U$ и $V$} называется пространство $T$ над $F$ вместе с билинейным отображением $f: U \times V \rightarrow T$ таким, что для любых базисов $e = (e_i)$ в $U$ и $\mathcal{G} = (g_j)$ в $V$ векторы $t_{ij} := f({e_i}, {g_j})$ образуют базис в $T$. Обозначение "--- $U \otimes V$.
\end{definition}

\begin{note}
	По определению, если $U \otimes V$ существует, то $\dim{T} = \dim{V}\dim{U}$.
\end{note}

\begin{proposition}
	Пусть $U, V, W$ "--- линейные пространства над $F$, $e = ({e_1}, \dots, {e_k})$ и $\mathcal{G} \hm{=} ({g_1}, \dots, {g_l})$ "--- базисы в $U$ и $V$ соответственно, и $f: U \times V \rightarrow W$ "--- билинейное отображение. Тогда эквивалентны следующие условия:
	\begin{enumerate}
		\item ${t_{ij}} := f({e_i}, {g_j})$ образуют базис в $W$
		\item $\forall \overline{w} \in W: \exists!\,\overline{v_1}, \dots, \overline{v_k} \in V: \overline{w} = \sum_{i = 1}^kf({e_i}, \overline{v_i})$
		\item $\forall \overline{w} \in W: \exists!\,\overline{u_1}, \dots, \overline{u_l} \in U: \overline{w} = \sum_{j = 1}^lf(\overline{u_j}, {g_j})$
	\end{enumerate}
\end{proposition}

\begin{proof}
	Докажем равносильность $1 \Leftrightarrow 2$, поскольку равносильность $1 \Leftrightarrow 3$ доказывается аналогично. Действительно, ${t_{ij}}$ образуют базис в $W$ $\Leftrightarrow$ $\forall \overline{w} \in W: \exists!\,\alpha_{ij} \in F: \overline{w} \hm{=} \sum_{i = 1}^k\sum_{j = 1}^l\alpha_{ij}f({e_i}, {g_j}) = \sum_{i = 1}^kf({e_i}, \sum_{j = 1}^l\alpha_{ij}{g_j})$. Но поскольку $\mathcal{G}$ "--- базис в $V$, это эквивалентно тому, что $\forall \overline{w} \in W: \exists!\,\overline{v_1}, \dots, \overline{v_k} \in V: \sum_{i = 1}^kf({e_i}, \overline{v_i})$. 
\end{proof}

\begin{note}
	Доказанное выше означает, что свойство $(1)$ не зависит от выбора базисов: если при фиксированном базисе в одном из пространств это свойство выполняется, то базис в другом пространстве можно выбирать произвольно. Следовательно, если свойство $(1)$ выполнено хотя бы для одной пары базисов, то оно выполнено и для всех пар базисов, тогда $W$ и $f$ задают тензорное произведение $U \otimes V$.
\end{note}

\begin{corollary}
	Для любых линейных пространств $U$ и $V$ над $F$ существует их тензорное произведение $U \otimes V$.
\end{corollary}

\begin{proof}
	В силу утверждения выше, достаточно взять пространство $T$ размерности $\dim{U}\dim{V}$, выбрать в нем базис $({t_{ij}})$, и для фиксированных базисов $({e_i})$ в $U$ и $({g_j})$ в $V$ положить $f({e_i}, {g_j}) := {t_{ij}}$.
\end{proof}

\begin{theorem}
	Пусть $U$ и $V$ "--- линейные пространства над $F$, $T$ и $f$ задают $U \otimes V$. Тогда для любого билинейного отображения $b: U \times V \rightarrow W$ существует единственное линейное отображение $\phi_b: T \rightarrow W$ такое, что $\phi_b \circ f = b$. Более того, соответствие между $b$ и $\phi_b$ осуществляет изоморфизм между $\mathcal{B}(U, V; W)$ и $\mathcal{L}(T, W)$.
\end{theorem}

\begin{proof}
	Пусть $e = ({e_i})$ и $\mathcal{G} = ({g_j})$ "--- базисы в $U$ и $V$ соответственно. Тогда на базисе $(t_{ij}) = f({e_i}, {g_j})$ искомое отображение $\phi_b$ однозначно задается как $\phi_b({t_{ij}}) := b({e_i}, {g_j})$. Сопоставление $\phi_b \mapsto b$ линейно, и, более того, оно обратимо: действительно, по любому отображению $\phi_b \in \mathcal{L}(T, W)$ можно восстановить $b \in \mathcal{B}(U, V; W)$, задав его на базисах $e, \mathcal{G}$ как $b({e_i}, {g_j}) := \phi_b({t_{ij}})$.
\end{proof}

\begin{corollary}
	Пусть $U$ и $V$ "--- линейные пространства над $F$, $(T_1, f_1)$ и $(T_2, f_2)$ "--- два их тензорных произведения. Тогда существует изоморфизм $\phi: T_2 \rightarrow T_1$ такой, что $\phi \circ f_2 = f_1$.
\end{corollary}

\begin{proof}
	Применим теорему выше, считая $(T_2, f_2)$ тензорным произведением, а $f_1$ "--- некоторым билинейным отображением, и получим отображение $\phi \in \mathcal{L}(T_2, T_1)$ такое, что $\phi \circ f_2 = f_1$. При этом $\im{\phi} \hm{\supset} \im{f_1}$, поэтому $\im{\phi}$ содержит базис пространства $T_1$, откуда $\im{\phi} = T_1$. Поскольку $\phi$ линеен и $\dim{T_1} = \dim{T_2}$, то он биективен и потому осуществляет изоморфизм между $T_1$ и $T_2$.
\end{proof}

\begin{note}
	С точностью до такого изоморфизма тензорное произведение единственно, поэтому можно зафиксировать произвольное тензорные произведение $(T, f)$, опуская при этом отображение $f$, и для любых $\overline{u} \in U$, $\overline{v} \hm{\in} V$ обозначать $f(\overline{u}, \overline{v})$ через $\overline{u} \otimes \overline{v} \in T$.
\end{note}

\begin{note}
	$\im{f}$ в доказательстве выше не обязан быть подпространством в $T$, поскольку $f$ "--- билинейное, а не линейное отображение. Не любой вектор из $T$ представляется в виде $\overline{u} \otimes \overline{v}$. Например, если оба пространства $U$, $V$ хотя бы двумерные, то ${e_1} \otimes {g_1} + {e_2} \otimes {g_2} \not\in \im f$.
\end{note}

\begin{example}
	Пусть $U, V$ "--- линейные пространства над $F$. Рассмотрим несколько тензорных произведений пространств:
	\begin{enumerate}
		\item $U^* \otimes V^* = \mathcal{B}(U, V; F)$. Для любых $c \in U^*$, $d \in V^*$ положим $f(c, d) := c \otimes d \in \mathcal{B}(U, V; F)$, где $\otimes$ означает тензорное произведение тензоров. Если выбрать пары взаимных базисов $e, e^*$ в $U$ и $\mathcal{G}, \mathcal G^*$ в $V$, то $f(e^i, g^j) = e^i \otimes g^j$ "--- это билинейная форма, принимающая значение $1$ на паре $(e_i, e_j)$ и $0$ "--- на других парах базисных векторов. Значит, $f$ задает тензорное произведение, потому что переводит пару базисов в $U^*$ и $V^*$ в базис в $\mathcal{B}(U, V; F)$.
		
		\item $U \otimes V = \mathcal{B}(U^*, V^*; F)$ в силу пункта $(1)$ и двойственности пространств $V$ и $V^*$.
		
		\item $U^* \otimes V = \mathcal{L}(U, V)$. Для любых $c \in U^*$, $\overline{v} \hm{\in} V$ положим $f(c, \overline{v})(\overline{u}) := c(\overline{u})\overline{v} \in \mathcal{L}(U, V)$. Аналогично выберем пары взаимных базисов $U$ и $V$, тогда матрица каждого оператора вида $f(e^i, g_j)$ будет состоять из одной единицы и остальных нулей, поэтому такие операторы образуют базис в $\mathcal{L}(U, V)$, и $f$ задает тензорное произведение.
	\end{enumerate}
\end{example}

\begin{proposition}
	Пусть $U, V, W$ "--- линейные пространства над полем $F$. Тогда выполнены следующие свойства тензорного произведения:
	\begin{enumerate}
		\item $\otimes$ ассоциативно, то есть $U \otimes (V \otimes W) \cong (U \otimes V) \otimes W$
		\item $\otimes$ коммутативно, то есть $U \otimes V \hm{\cong} V \otimes U$
	\end{enumerate}
\end{proposition}

\begin{proof}
	Имеет место естественные изоморфизмы $\overline{u} \otimes (\overline{v} \otimes \overline{w}) \mapsto (\overline{u} \otimes \overline{v}) \otimes \overline{w}$ и $\overline{u} \otimes \overline{v} \mapsto \overline{v} \otimes \overline{u}$.
\end{proof}

\begin{note}
	Примеры выше с учетом доказанных свойств тензорного произведения позволяют сделать следующий вывод:
	\[\mathbb{T}^p_q = \{t : (V^*)^p \times V^q \rightarrow F\} = \underbrace{V \otimes \dots \otimes V}_p \otimes \underbrace{V^* \otimes \dots \otimes V^*}_q =: V^{\otimes p} \otimes (V^*)^{\otimes q}\]
	
	Из этого, в частности, следует, что $\mathbb{T}^p_q \otimes \mathbb{T}^{p'}_{q'} \cong \mathbb{T}^{p + p'}_{q + q'}$.
\end{note}

\begin{example} 
	Пусть $U, V$ "--- линейные пространства над $F$. Тогда, по свойствам тензорного произведения, $\mathcal{L}(U, V) = U^* \otimes V \cong V \otimes U^* = \mathcal{L}(V^*, U^*)$. Можно проверить, что этот изоморфизм задает сопоставление $\phi \mapsto \phi^*$, где $\phi \in \mathcal{L}(U, V)$, а $\phi^* \in \mathcal{L}(V^*, U^*)$ "--- сопряженное к $\phi$ отображение.
\end{example}