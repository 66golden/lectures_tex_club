\subsection{Простейшие приложения жордановой нормальной формы}

\textbf{До конца раздела} зафиксируем линейное пространство $V$ над полем $F$ и положим $n := \dim{V}$.

\begin{note}
	Пусть $A, B \in M_n(F)$ "--- матрицы, характеристические многочлены которых имеют вид $\eqref*{charpol}$. Тогда матрицы $A, B$ подобны $\lra$ их жордановы формы $A'$ и $B'$ совпадают с точностью до перестановки клеток. Это верно потому, что жорданова нормальная форма оператора, а значит и задающей его матрицы, единственна с точностью до перестановки клеток.
\end{note}

\begin{proposition}
	Пусть $\phi \in \mathcal{L}(V)$, и $\Chi_\phi$ имеет вид $\eqref*{charpol}$. Тогда его минимальный многочлен имеет вид $\mu_\phi = \prod_{i = 1}^{k}(\lambda_i - \lambda)^{\beta_i}$,
	где $\beta_i$ "--- наибольший размер клетки c собственным значением $\lambda_i$ в жордановой нормальной форме оператора $\phi$ для каждого $i \in \{1, \dotsc, k\}$.
\end{proposition}

\begin{proof}
	Уже было доказано, что многочлен $\mu_\phi$ делит многочлен $\Chi_\phi$. Заметим, что наблюдение о независимом возведении жордановых клеток в степень можно обобщить на случай произвольного многочлена $p \in F[x]$:
	\[p\left(\begin{array}{@{}ccc@{}}
		\cline{1-1}
		\multicolumn{1}{|c|}{J_{k_1}} & \dots & 0\\
		\cline{1-1}
		\vdots & \ddots & \vdots\\
		\cline{3-3}
		0 & \dots & \multicolumn{1}{|c|}{J_{k_s}}\\
		\cline{3-3}
	\end{array}\right) = \left(\begin{array}{@{}cccc@{}}
		\cline{1-1}
		\multicolumn{1}{|c|}{p(J_{k_1})} & \dots & 0\\
		\cline{1-1}
		\vdots & \ddots & \vdots\\
		\cline{3-3}
		0 & \dots & \multicolumn{1}{|c|}{p(J_{k_s})}\\
		\cline{3-3}
	\end{array}\right)\]
	
	Значит, многочлен $p$ "--- аннулирующий $\Leftrightarrow$ $p$ обнуляет каждую клетку жордановой нормальной формы. Пусть $s_j$ "--- наибольший размер клетки, соответствующий значению $\lambda_j$ в жордановой нормальной форме оператора $\phi$, тогда:
	\[0 = \mu_\phi(J_{s_j}(\lambda_j)) = \prod_{i = 1}^k(\lambda_iE - J_{s_j}(\lambda_j))^{\beta_i} = (-J_{s_j})^{\beta_j}\prod_{i \ne j}(\lambda_iE - J_{s_j}(\lambda_j))^{\beta_i}\]
	
	Поскольку все матрицы в произведении кроме первой "--- невырожденные, то их произведение "--- тоже невырожденная матрица, поэтому выполнено равенство $(J_{s_j})^{\beta_j} = 0$, откуда $\beta_j \ge s_j$. С другой стороны, степени $\beta_j = s_j$ достаточно, чтобы обнулить все клетки, соответствующие значению $\lambda_j$.
\end{proof}

\begin{proposition}
	Если минимальный многочлен $\mu_\phi$ оператора $\phi \in \mathcal{L}(V)$ раскладывается на линейные сомножители, то и $\Chi_\phi$ раскладывается на линейные сомножители, то есть имеет вид $\eqref*{charpol}$.
\end{proposition}

\begin{proof}
	Минимальный многочлен, конечно, является аннулирующим, причем $\mu_\phi = \prod_{i = 1}^k(\lambda_i - \lambda)^{\gamma_i}$, поэтому, как уже было доказано, $V = V_1 \oplus \dots \oplus V_k$, где $V_i = \ke{(\phi_{\lambda_i})^{\gamma_i}}$ для каждого $i \in \{1, \dotsc, k\}$. Тогда справедливы дальнейшие рассуждения из доказательства существования жордановой нормальной формы оператора $\phi$. Но если оператор $\phi$ имеет жорданову нормальную форму, то многочлен $\Chi_\phi$ раскладывается на линейные сомножители.
\end{proof}

\begin{note}
	Жорданова нормальная форма позволяет быстро возводить в степень матрицы $A \in M_n(F)$ такие, что $\Chi_A$ имеет вид $\eqref*{charpol}$. Действительно, если матрица $A$ имеет жорданову нормальную форму $B$, то для некоторой матрицы $S \in \GL_n(F)$ выполнено $A \hm{=} S^{-1}BS$, откуда $A^n = S^{-1}B^nS$. Поскольку жордановы клетки возводятся в степень независимо, достаточно уметь находить степень каждой из них. Для этого заметим, что для любой клетки $J_k(\lambda_0)$ и любого $m \in \N$ выполнено следующее:
	\[(J_k(\lambda_0))^m = (\lambda_0E + J_k)^m = \sum_{i = 0}^mC_m^i\lambda_0^{m - i}J_k^i = \sum_{i = 0}^{k - 1}C_m^i\lambda_0^{m - i}J_k^i\]
	
	Формула бинома Ньютона для матриц $\lambda_0 E$ и $J_k$ справедлива потому, что эти матрицы коммутируют.
\end{note}

\begin{note}
	Можно показать, что верно более сильное утверждение: если $\cha{F} = 0$, то для любого многочлена $p \in F[x]$ выполнено равенство $p(A) \hm{=} S^{-1}p(B)S$, причем для каждой жордановой клетки $J_k(\lambda)$ в $B$ справедлива следующая формула:
	\[p(J_k(\lambda)) = p(\lambda)E + p'(\lambda)J_k + \frac{p''(\lambda)}{2}J_k^2 + \dots + \frac{p^{(k-1)}(\lambda)}{(k-1)!}J_k^{k-1}\]
\end{note}