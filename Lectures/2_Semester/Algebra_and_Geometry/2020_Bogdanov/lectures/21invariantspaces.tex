\section{Линейные операторы}

\subsection{Инвариантные подпространства}

\textbf{До конца раздела} зафиксируем линейное пространство $V$ над полем $F$ и положим $n := \dim{V}$.

\begin{definition}
	Пусть $\phi \in \mathcal{L}(V)$. Подпространство $U \le V$ называется \textit{инвариантным} относительно преобразования $\phi$, если $\phi(U) \hm{\le} U$.
\end{definition}

\begin{example}
	Рассмотрим несколько примеров инвариантных подпространств:
	Инвариантными подпространствами относительно соответствующих преобразований являются:
	\begin{itemize}
		\item Прямая $l$ в плоскости $V_2$ и прямая $n \perp l$ инвариантны относительно $\phi \in \mc L(V_2)$, где $\phi$ "--- симметрия относительно $l$
		\item Пусть $k \in \N$, тогда подпространство $P_k := \{P \in F[x]: \deg P \le k\} \le F[x]$ инвариантно относительно $\phi \in \mc F[x]$, где $\phi$ "--- формальное дифференцирование
	\end{itemize}
\end{example}

\begin{proposition}
	Пусть $\phi \in \mathcal{L}(V)$, $e = (\overline{e_1}, \dots, \overline{e_n})$ "--- базис в $V$, $\phi \leftrightarrow_e A \hm{\in} M_n(F)$, $k \in \N$, $k \le n$. Тогда $U := \langle\overline{e_1}, \dots, \overline{e_k}\rangle \le V$ инвариантно относительно $\phi$ $\Leftrightarrow$ $A$ имеет следующий вид для некоторых $B \in M_k(F)$, $D \in M_{n - k}(F)$:
	\[A = \left(\begin{array}{@{}c|c@{}}
	B & C\\
	\hline
	0 & D
	\end{array}\right)\]
\end{proposition}

\begin{proof}
	$U$ инвариантно относительно $\phi$ $\Leftrightarrow$ $\forall i \in \{1, \dots, k\}: \phi(\overline{e_i}) \in U$ $\Leftrightarrow$ $A$ имеет требуемый вид.
\end{proof}

\begin{note}
	В утверждении выше матрица $B$ является матрицей оператора $\phi|_U \in \mathcal{L}(U)$ в базисе $(\overline{e_1}, \dots, \overline{e_k})$.
\end{note}

\begin{proposition}
	Пусть $\phi \in \mathcal{L}(V)$, $U_1, U_2 \le V$ "--- инвариантные относительно $\phi$ подпространства. Тогда подпространства $U_1 + U_2$ и $U_1 \cap U_2$ тоже инвариантны относительно $\phi$.
\end{proposition}

\begin{proof}~
	\begin{itemize}
		\item $\phi(U_1 + U_2) = \phi(U_1) + \phi(U_2) \le U_1 + U_2$
		\item $\phi(U_1 \cap U_2) \le \phi(U_1) \cap \phi(U_2) \le U_1 \cap U_2$\qedhere
	\end{itemize}
\end{proof}

\begin{proposition}
	Пусть $\phi \in \mathcal{L}(V)$, подпространства $U, W \le V$ таковы, что $U \le \ke{\phi}$ и $\im{\phi} \le W \le V$. Тогда $U$ и $W$ инвариантны относительно $\phi$.
\end{proposition}

\begin{proof}~
	\begin{itemize}
		\item $\phi(U) \le \phi(\ke{\phi}) = \{\overline{0}\} \le U$
		\item $\phi(W) \le \im{\phi} \hm{\le} W$\qedhere
	\end{itemize}
\end{proof}

\begin{proposition}
	Пусть $\phi, \psi \in \mathcal{L}(V)$, причем $\phi \circ \psi = \psi \circ \phi$. Тогда $\ke{\psi}$ и $\im{\psi}$ инвариантны относительно $\phi$.
\end{proposition}

\begin{proof}~
	\begin{itemize}
		\item Пусть $\overline{u} \in \ke{\psi}$, тогда $\psi(\phi(\overline{u})) = \phi(\psi(\overline{u})) = \phi(\overline{0}) = \overline{0}$, откуда $\phi(\overline{u}) \in \ke{\psi}$
		\item Пусть $\overline{u} \in \im{\psi}$, тогда существует вектор $\overline{v} \in V$ такой, что выполнено равенство $\psi(\overline{v}) = \overline{u}$, откуда $\phi(\overline{u}) = \phi(\psi(\overline{v})) \hm{=} \psi(\phi(\overline{v})) \in \im{\psi}$\qedhere
	\end{itemize}
\end{proof}

\begin{note}
	Последнее утверждение полезно в случае, когда $\psi \hm{=} P(\phi)$, $P \in F[x]$, частности, когда $\psi = \phi - \lambda$, где $\lambda \in F$.
\end{note}

\begin{note}
	Пусть $V = U \oplus W$, причем подпространства $U, W$ инвариантны относительно $\phi \in \mathcal{L}(V)$ и $e' = (\overline{e_1}, \dots, \overline{e_k})$, $e'' = (\overline{e_{k + 1}}, \dots, \overline{e_n})$ "--- базисы в $U$ и $W$, $\phi|_U \leftrightarrow_{e'} B$, $\phi|_W \leftrightarrow_{e''} D$. Тогда, по свойству прямой суммы, $e = (\overline{e_1}, \dots, \overline{e_n})$ "--- базис в $V$, причем матрица оператора $\phi$ в этом базисе имеет следующий вид:
	\[\phi \leftrightarrow_e \left(\begin{array}{@{}c|c@{}}
	B & 0\\
	\hline
	0 & D
	\end{array}\right)\]
\end{note}