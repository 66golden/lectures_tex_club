\section{Билинейные и квадратичные формы}

\subsection{Билинейные формы}

\textbf{До конца раздела} зафиксируем линейное пространство $V$ над полем $F$ и положим $n := \dim{V}$.

\begin{definition}
	\textit{Билинейной формой} на $V$ называется функция $b: V \times V \rightarrow F$, линейная по обоим аргументам:
	\begin{itemize}
		\item $\forall \overline{x_1}, \overline{x_2}, \overline{y} \in V: b(\overline{x_1} + \overline{x_2}, \overline{y}) = b(\overline{x_1}, \overline{y}) + b(\overline{x_2}, \overline{y})$
		\item $\forall \overline{x}, \overline{y} \in V: \forall \alpha \in F: b(\alpha\overline{x}, \overline{y}) = \alpha b(\overline{x}, \overline{y})$
		\item $\forall \overline{x}, \overline{y_1}, \overline{y_2} \in V: b(\overline{x}, \overline{y_1} + \overline{y_2}) = b(\overline{x}, \overline{y_1}) + b(\overline{x}, \overline{y_2})$
		\item $\forall \overline{x}, \overline{y} \in V: \forall \alpha \in F: b(\overline{x}, \alpha\overline{y}) = \alpha b(\overline{x}, \overline{y})$
	\end{itemize}
	
	Множество всех билинейных форм на $V$ обозначается через $\mathcal{B}(V)$.
\end{definition}

\begin{example}
	Рассмотрим несколько примеров билинейных форм:
	\begin{itemize}
		\item Скалярное произведение в пространстве векторов $V_n$, $n \in \{1, 2, 3\}$, является билинейной формой на $V_n$
		\item Умножение в поле $F$ является билинейной формой на $F$
		\item Функция $b$, заданная для произвольных $x, y \in F^2$ как $b(\overline{x}, \overline{y}) = |\overline{x}~\overline{y}|$, является билинейной формой на $F^2$
		\item Функция $b$, заданная для произвольных $x, y \in F^3$ как $b(\overline{x}, \overline{y}) = |\overline{x}~\overline{y}~\overline{c}|$, где $\overline{c} \in F^3$ "--- фиксированный столбец, является билинейной формой на $F^3$
	\end{itemize}
\end{example} 

\begin{definition}
	 \textit{Матрицей формы} $b \in \mathcal{B}(V)$ в базисе $(\overline{e_1}, \dots, \overline{e_n}) \hm{=:} e$ называется следующая матрица $B$:
	\[B = (b(\overline{e_i}, \overline{e_j})) = \begin{pmatrix}b(\overline{e_1}, \overline{e_1}) & \dots & b(\overline{e_1}, \overline{e_n})\\
		\vdots & \ddots & \vdots\\
		b(\overline{e_n}, \overline{e_1}) & \dots & b(\overline{e_n}, \overline{e_n})
	\end{pmatrix} \in M_n(F)\]
	
	Обозначение "--- $b \leftrightarrow_e B$.
\end{definition}

\begin{proposition}
	Пусть $b \in \mathcal{B}(V)$, $e$ "--- базис в $V$, $b \leftrightarrow_e B$, $\overline{u}, \overline{v} \in V$, $\overline{u} \leftrightarrow_e x$, $\overline{v} \leftrightarrow_e y$. Тогда выполнено равенство $b(\overline{u}, \overline{v}) = x^TBy$.
\end{proposition}

\begin{proof}
	Выполнены следующие равенства:
	\[b(\overline{u}, \overline{v}) = b\left(\sum_{i = 1}^nx_i\overline{e_i}, \sum_{j = 1}^ny_j\overline{e_j}\right) = \sum_{i = 1}^nx_i\left(\sum_{j = 1}^ny_jb(\overline{e_i}, \overline{e_j})\right) = x^TBy\]

	Получено требуемое.
\end{proof}

\begin{note}
	Множество $\mathcal{B}(V)$ образует линейное пространство над $F$ с операциями сложения и умножения на скаляр, определенными следующим образом:
	\begin{itemize}
		\item $\forall b_1, b_2 \in \mathcal{B}(V): \forall \overline{u}, \overline{v} \in V: (b_1 + b_2)(\overline{u}, \overline{v}) := b_1(\overline{u}, \overline{v}) + b_2(\overline{u}, \overline{v})$
		\item $\forall b \in \mathcal{B}(V): \forall \alpha \in F: \forall \overline{u}, \overline{v} \in V: (\alpha b)(\overline{u}, \overline{v}) = \alpha b(\overline{u}, \overline{v})$
	\end{itemize}
\end{note}

\begin{theorem}
	Пусть $e$ "--- базис в $V$. Тогда соответствие между $\mathcal{B}(V)$ и $M_n(F)$ вида $b \leftrightarrow_e B$ осуществляет изоморфизм линейных пространств $\mathcal{B}(V)$ и $M_n(F)$.
\end{theorem}

\begin{proof}~
	\begin{enumerate}
		\item Если $b_1, b_2 \in \mathcal{B}(V)$, $b_1 \leftrightarrow_e B_1$, $b_2 \leftrightarrow_e B_2$, то $b_1 + b_2 \leftrightarrow_e B_1 \hm{+} B_2$, $\alpha b_1 \leftrightarrow_e \alpha B_1$. Значит, описанное в теореме отображение $\phi: \mathcal{B}(V) \rightarrow M_n(F)$ линейно.
		\item Отображение $\phi$ инъективно, поскольку если $\phi(b_1) = \phi(b_2) = B$, то $\forall \overline{u}, \overline{v} \in V$, $\overline{u} \leftrightarrow_e x$, $\overline{v} \leftrightarrow_e y$: $b_1(\overline{u}, \overline{v}) = b_2(\overline{u}, \overline{v})$, то есть $b_1 = b_2$.
		\item Отображение $\phi$ сюръективно, поскольку для каждой матрицы $B \in M_n(F)$ можно определить билинейную форму $b$ как $b(\overline{u}, \overline{v}) = x^TBy$, где $\overline{u} \leftrightarrow_e x, \overline{v} \leftrightarrow_e y$. Тогда если $B = (b_{ij})$, то $b(\overline{e_i}, \overline{e_j}) = b_{ij}$, то есть $b \leftrightarrow_e B$.
	\end{enumerate}
	
	Таким образом, $\phi$ "--- это изоморфизм линейных пространств.
\end{proof}

\begin{corollary}
	Выполнено равенство $\dim{\mathcal{B}(V)} = n^2$.
\end{corollary}

\begin{theorem}
	Пусть $b \in \mathcal{B}(V)$, $e$ и $e'$ "--- два базиса в $V$, $e' = eS$, и $b \leftrightarrow_e B$, $b \leftrightarrow_{e'} B'$. Тогда выполнено равенство $B' = S^TBS$.
\end{theorem}

\begin{proof}
	Перепишем $S$ в виде $(s_1|\dots|s_n)$. Тогда:
	\[B' = \begin{pmatrix}
		b(\overline{e_1'}, \overline{e_1'})&\dots&b(\overline{e_1'}, \overline{e_n'})\\
		\vdots&\ddots&\vdots\\
		b(\overline{e_n'}, \overline{e_1'})&\dots&b(\overline{e_n'}, \overline{e_n'})
	\end{pmatrix} = \begin{pmatrix}
		s_1^TBs_1&\dots&s_1^TBs_n\\
		\vdots&\ddots&\vdots\\
		s_n^TBs_1&\dots&s_n^TBs_n\\
	\end{pmatrix} = S^TBS\]
	
	Получено требуемое.
\end{proof}

\begin{corollary}
	Пусть $b \in \mathcal{B}(V)$, $e$ и $e'$ "--- два базиса в $V$, и $b \leftrightarrow_e B$, $b \leftrightarrow_{e'} B'$. Тогда выполнено равенство $\rk{B} = \rk{B'}$.
\end{corollary}

\begin{definition}
	\textit{Рангом билинейной формы} $b \in \mathcal{B}(V)$ называется ранг ее матрицы в произвольном базисе. Обозначение "--- $\rk{b}$.
\end{definition}

\begin{corollary}
	Пусть $F = \mathbb{R}$, $b \in \mathcal{B}(V)$, $e$ и $e'$ "--- два базиса в $V$, $e' = eS$, и $b \leftrightarrow_e B$, $b \leftrightarrow_{e'} B'$. Тогда $\det{B}$ и $\det{B'}$ имеют один и тот же знак.
\end{corollary}

\begin{proof}
	$\det{B'} = \det{S^TBS} = (\det{S})^2\det{B}$, что и означает требуемое в силу положительности числа $\det{S}$.
\end{proof}

\begin{definition}
	Пусть $b \in \mathcal{B}(V)$. Форма $b$ называется \textit{симметрической}, если для всех $\overline{u}, \overline{v} \in V$ выполнено $b(\overline{u}, \overline{v}) = b(\overline{v}, \overline{u})$. Пространство симметрических форм на $V$ обозначается через $\mathcal{B}^+(V)$.
\end{definition}

\begin{definition}
	Пусть $b \in \mathcal{B}(V)$. Форма $b$ называется \textit{кососимметрической}, если выполнены следующие условия:
	\begin{enumerate}
		\item $\forall \overline{u}, \overline{v} \in V: b(\overline{u}, \overline{v}) = -b(\overline{v}, \overline{u})$
		\item $\forall \overline{u} \in V: b(\overline{u}, \overline{u}) = 0$
	\end{enumerate}
	
	Пространство кососимметрических форм на $V$ обозначается через $\mathcal{B}^-(V)$.
\end{definition}

\begin{note}
	В определении выше из условия $(2)$ следует условие $(1)$. Действительно, пусть выполенно условие $(2)$, тогда:
	\[0 = b(\overline{u} + \overline{v}, \overline{u} + \overline{v}) = b(\overline{u}, \overline{u}) + b(\overline{u}, \overline{v}) + b(\overline{v}, \overline{u}) + b(\overline{v}, \overline{v}) = b(\overline{u}, \overline{v}) + b(\overline{v}, \overline{u})\]
	
	Значит, $b(\overline{u}, \overline{v}) = -b(\overline{v}, \overline{u})$. При этом из условия $(1)$ следует условие $(2)$ лишь в том случае, когда $\cha{F} \ne 2$.
\end{note}

\begin{proposition}
	Пусть $e$ "--- базис в $V$, $b \in \mathcal{B}(V)$, и $b \leftrightarrow_e B$. Тогда:
	\begin{enumerate}
		\item $b \in \mathcal{B}^+(V)$ $\Leftrightarrow$ $B^T = B$
		\item $b \in \mathcal{B}^-(V)$ $\Leftrightarrow$ $B^T = -B$ и на главной диагонали $B$ стоят нули
	\end{enumerate}
\end{proposition}

\begin{proof}~
	\begin{itemize}
		\item[$\Rightarrow$] Поскольку $B = (b_{ij}) = (b(\overline{e_i}, \overline{e_j}))$, то непосредственная подстановка базисных векторов позволяет убедиться, что в первом случае $B^T = B$, а во втором --- $B^T = -B$, и на главной диагонали матрицы $B$ стоят нули.
		
		\item[$\Leftarrow$] Пусть $\overline{u}, \overline{v} \in V$, $\overline{u} \leftrightarrow_e x$, $\overline{v} \leftrightarrow_e y$. Тогда выполнены следующие цепочки равенств:
		\begin{enumerate}
			\item $b(\overline{u}, \overline{v}) = x^TBy = (x^TBy)^T = y^TBx = b(\overline{v}, \overline{u})$
   
			\item $b(\overline{u}, \overline{u}) = x^TBx \hm{=} \sum\limits_{i = 1}^n\sum\limits_{j = 1}^nx_ib_{ij}x_j \hm= \sum\limits_{i = 1}^nx_ib_{ii}x_i + \sum\limits_{1 \le i < j \le n}x_i(b_{ij} + b_{ji})x_j = 0$
		\end{enumerate}

		В обоих случаях получено требуемое.\qedhere
	\end{itemize}
\end{proof}