\subsection{Положительная и отрицательная определенность}

\textbf{В данном разделе} будем считать, что $V$ "--- линейное пространство над $\mathbb{R}$.

\begin{definition}
	Пусть $h \hm{\in} \mathcal{Q}(V)$. Базис, в котором матрица $h$ диагональна с числами $0$ и $\pm1$ на главной диагонали, называется \textit{нормальным базисом}, а матрица $h$ в этом базисе "--- \textit{нормальной формой} $h$.
\end{definition}

\begin{definition}
	Пусть $h \in \mathcal{Q}(V)$. Тогда $h$ называется:
	\begin{itemize}
		\item \textit{положительно определенной}, если $\forall \overline{v} \hm{\in} V, \overline{v} \ne \overline{0}: h(\overline{v}) \hm{>} 0$.
		\item \textit{положительно полуопределенной}, если $\forall \overline{v} \in V: h(\overline{v}) \ge 0$.
		\item \textit{отрицательно определенной}, если $\forall \overline{v} \in V, \overline{v} \ne \overline{0}: h(\overline{v}) \hm{<} 0$.
		\item \textit{отрицательно полуопределенной}, если $\forall \overline{v} \in V: h(\overline{v}) \le 0$.
	\end{itemize}
	
	Форма $b \in \mathcal{B}^+(V)$, полярная к $h$, приобретает те же названия.
\end{definition}

\begin{proposition}
	Пусть $h \in \mathcal{Q}(V)$, $B$ "--- нормальная форма $h$ в нормальном базисе $e$. Тогда:
	\begin{enumerate}
		\item $h$ положительно определена $\Leftrightarrow$ $B = E$.
		\item $h$ положительно полуопределена $\Leftrightarrow$ на диагонали $B$ стоят только нули и единицы.
	\end{enumerate}
\end{proposition}

\begin{proof}~
	\begin{itemize}
		\item[$\Leftarrow$] Пусть $\overline{v} \in V, v \ne 0, \overline{v} \leftrightarrow_e x$. Если $B = E$, то $h(\overline{v}) = x^TBx = x_1^2 + \dots + x_n^2 > 0$. Если на диагонали $B$ расположены только нули и единицы, то $h(\overline{v})$ "--- это сумма квадратов некоторых координат вектора $\overline{v}$, откуда $h(\overline{v}) \ge 0$.
		
		\item[$\Rightarrow$] $i$-й диагональный элемент матрицы $B$ "--- это $h(\overline{e_i})$. Если $h$ положительно определена, то $\forall i \in \{1, \dots, n\}: h(\overline{e_i}) > 0$, то есть все диагональные элементы "--- единицы. Если же $h$ положительно полуопределена, то $\forall i \in \{1, \dots, n\}: h(\overline{e_i}) \ge 0$, то есть диагональные элементы "--- это нули и единицы.\qedhere
	\end{itemize}
\end{proof}

\begin{note}
	Для случая отрицательной определенности и полуопределенности утверждение аналогично при замене единиц на минус единицы. Более того, $h$ положительно определена (полуопределена) $\Leftrightarrow$ $-h$ отрицательно определена (полуопределена).
\end{note}

\begin{definition}
	Пусть $h \in \mathcal{Q}(V)$. Ее \textit{положительным индексом инерции} $\sigma_+(h)$ называется наибольшая размерность подпространства $U \le V$ такого, что $h|_U$ положительно определена, \textit{отрицательным индексом инерции} $\sigma_-(h)$ "--- наибольшая размерность подпространства $U \le V$ такого, что $h|_U$ отрицательно определена.
\end{definition}

\begin{note}
	Множества $\{\overline{v} \in V~|~h(\overline{v}) \hm{>} 0\}$, $\{\overline{v} \in V~|~h(\overline{v}) \ge 0\}$ не всегда является подпространствами в $V$. Например, это неверно для следующей квадратичной формы:
	\[h \leftrightarrow_e B \hm{=} \begin{pmatrix}1&0\\0&-1\end{pmatrix}\]
\end{note}

\begin{theorem}
	Пусть $h \in \mathcal{Q}(V)$, $B$ "--- нормальная форма $h$ в нормальном базисе $e$. Тогда на диагонали матрицы $B$ стоит ровно $\sigma_+(h)$ единиц и ровно $\sigma_-(h)$ минус единиц.
\end{theorem}

\begin{proof}
	Пусть $n := \dim{V}$. Без ограничения общности можно считать, что нормальная форма $h$ имеет следующий вид:
	\[B = \left(\begin{array}{@{}ccc@{}}
		\cline{1-1}
		\multicolumn{1}{|c|}{E_k} & 0 & 0\\
		\cline{1-2}
		0 & \multicolumn{1}{|c|}{0_l} & 0\\
		\cline{2-3}
		0 & 0 & \multicolumn{1}{|c|}{-E_m}\\
		\cline{3-3}
	\end{array}\right),~k + l + m = n\]
	
	Пусть $U := \langle\overline{e_1}, \dots, \overline{e_k}\rangle$, $W := \langle\overline{e_{k + 1}}, \dots, \overline{e_n}\rangle$. Сужение $h|_U$ положительно определено, откуда $\sigma_+(h) \ge k$. С другой стороны, сужение $h|_W$ отрицательно полуопределено. Пусть теперь $U' \le V$ "--- такое, что $h|_{U'}$ положительно определено, тогда $U' \cap W = \{\overline{0}\}$, поскольку $\forall \overline{w} \in W: h(\overline{w}) \le 0$. Значит, $\sigma_+(h) = k$. Аналогично, $\sigma_-(h) = m$.
\end{proof}

\begin{corollary}[закон инерции]
	Нормальный вид квадратичной формы $h \in \mathcal{Q}(V)$ определен однозначно с точностью до перестановки диагональных элементов.
\end{corollary}

\begin{definition}
	Пусть $B \in M_n(\mathbb{R})$ "--- симметричная матрица. $B$ называется \textit{положительно} или \textit{отрицательно определенной (полуопределенной)}, если она задает квадратичную форму, обладающую этим свойством.
\end{definition}

\begin{proposition}
	$B \in M_n(\mathbb{R})$ положительно определена $\Leftrightarrow$ $\exists A \hm{\in} \GL_n(\mathbb{R})$: $B = A^TA$.
\end{proposition}

\begin{proof}
	Квадратичная форма $h \leftrightarrow_e B$ положительно определена $\lra$ $h$ имеет нормальный вид $E$ $\lra$ для некоторой матрицы $S \in \GL_n(\R)$ выполнено $E = S^TBS$ $\lra$ для некоторой матрицы $A \hm{\in} \GL_n(\mathbb{R})$ выполнено $B = A^TA$.
\end{proof}

\begin{proposition}
	$B \in M_n(\mathbb{R})$ положительно полуопределена $\hm{\Leftrightarrow} \exists A \hm{\in} M_n(\mathbb{R})$: $B = A^TA$.
\end{proposition}

\begin{proof}
	Аналогично утверждению выше с заменой нормального вида $E$ на нормальный вид $E' = \diag(1, \dotsc, 1, 0, \dotsc, 0)$.
\end{proof}

\begin{definition}
	Пусть $B \in M_n(\mathbb{R})$ "--- симметричная матрица. Ее \textit{главным минором порядка $i$} называется $\Delta_i(B)$ "--- определитель подматрицы размера $i \times i$, расположенной в левом верхнем углу $B$.
\end{definition}

\begin{theorem}[Метод Якоби]
	Пусть $h \in \mathcal{Q}(V)$, $h \leftrightarrow_e B$, причем все главные миноры матрицы $B$ отличны от нуля. Тогда существует такой базис $e' = eS$, что матрица перехода $S$ "--- верхнетреугольная с единицами на главной диагонали, $h \leftrightarrow_{e'} B'$ и $B'$ диагональна. Более того, тогда $B' = \diag(\Delta_1(B), \frac{\Delta_2(B)}{\Delta_1(B)}, \dotsc, \frac{\Delta_{n}(B)}{\Delta_{n-1}(B)})$.
\end{theorem}

\begin{proof}
	Докажем индукцией по $n := \dim{V}$, что матрица формы $h$ приводится к диагональному виду в базисе с матрицей перехода из условия. База, $n = 1$, тривиальна: подходит исходный базис $e$. Пусть теперь $n > 1$, тогда $U := \langle\overline{e_1}, \dots, \overline{e_{n - 1}}\rangle$ невырожденно относительно формы $b$, полярной к $h$, так как $\Delta_{n - 1}(B) \ne 0$. Значит, $V = U \oplus U^\perp$. Представим $\overline{e_n}$ в виде $\overline{e_n} = \overline{u} + \overline{e_n'}$, где $\overline{u} \in U$, $\overline{e_n'} \in U^\perp$, причем $\overline{e_n'} \ne \overline{0}$. По предположению индукции, в $U$ можно выбрать подходящий базис $(\overline{e_1'}, \dots, \overline{e_{n-1}'})$, тогда его объединение с $\overline{e_n'}$ будет искомым. Матрица перехода $S$ действительно будет верхнетреугольной с единицами на главной диагонали: для первых $n - 1$ столбцов это верно по предположению индукции, для последнего столбца "--- в силу того, что $\overline{e_n'} = \overline{e_n} - u$.
	
	Вычислим теперь значения диагональных элементов $d_i$, $i \in \{1, \dotsc, n\}$. Заметим, что, поскольку базис $e'$ получен описанным выше образом, $\forall i \in \{1, \dots, n\}: \overline{e_i'} \hm{\in} \langle\overline{e_1}, \dots, \overline{e_i}\rangle$ и $\langle\overline{e_1}, \dots, \overline{e_i}\rangle = \langle\overline{e_1'}, \dots, \overline{e_i'}\rangle$. Пусть $B_i$ "--- подматрица $B$ в левом верхнем углу, а $B_i'$ "--- аналогичная подматрица $B'$. Тогда $B_i' = S_i^TB_iS_i$, где $S_i$ "--- соответствующая подматрица $S$, также являющаяся верхнетреугольной с единицами на диагонали, поэтому $\Delta_i(B') = |B_i'|\hm{=}|S_i^TB_iS_i| \hm{=} |B_i| = \Delta_i(B)$. Значит, $\forall i \in \{1, \dotsc, n\}: \Delta_i(B) = \Delta_i(B') = d_1\dots d_i$, откуда $B' = \diag(\Delta_1(B), \frac{\Delta_2(B)}{\Delta_1(B)}, \dotsc, \frac{\Delta_{n}(B)}{\Delta_{n-1}(B)})$.
\end{proof}

\begin{theorem}[Критерий Сильвестра]
	Пусть $h \in \mathcal{Q}(V)$, $h \leftrightarrow_e B$. Тогда $h$ положительно определена $\Leftrightarrow$ $\forall i \in \{1, \dots, n\}: \Delta_i(B) > 0$.
\end{theorem}

\begin{proof} Пусть $n := \dim{V}$.
	\begin{itemize}
		\item[$\Rightarrow$] Если $h$ положительно определена, то $B = A^TA$ для некоторой $A \in \GL_n(\R)$. Тогда $\Delta_n(B) = |B| = |A|^2 > 0$. Поскольку главному минору порядка $i \in \{1, \dots, n - 1\}$ соответствувет ограничение $h$ на $U \hm{:=} \langle\overline{e_1}, \dots, \overline{e_i}\rangle$, которое тоже положительно определено, то, аналогично, $\Delta_i(B) > 0$.
		\item[$\Leftarrow$] Согласно методу Якоби, существует базис $e'$ в $V$ такой, что матрица $h$ в нем диагональна, причем $h \leftrightarrow_{e'} \diag(\Delta_1(B), \frac{\Delta_2(B)}{\Delta_1(B)}, \dotsc, \frac{\Delta_{n}(B)}{\Delta_{n-1}(B)})$. Все элементы на главной диагонали положительны, поэтому $h$ положительно определена.\qedhere
	\end{itemize}
\end{proof}

\begin{note}
	Если $\forall i \in \{1, \dots, n\}: \Delta_i(B) \ne 0$, то, согласно методу Якоби, $\sigma_-(h)$ "--- это число перемен знака в последовательности $(1, \Delta_1(B), \dots, \Delta_n(B))$, а $\sigma_+(h)$ "--- число сохранений знака в этой же последовательности.
\end{note}

\begin{note}
	Прямой аналог критерия Сильвестра для положительной полуопределенности неверен. Например, пусть $B$ имеет следующий вид:
	\[B = \begin{pmatrix}
		1&0&0\\
		0&0&0\\
		0&0&-1
	\end{pmatrix}\]
	
	Тогда $\Delta_1(B) = 1, \Delta_2(B) = \Delta_3(B) = 0$, но при этом $B$ не является положительно полуопределенной. Тем не менее, можно показать, что $B \in M_n(\mathbb{R})$ положительно полуопределена $\Leftrightarrow$ все ее симметричные относительно главной диагонали миноры неотрицательны.
\end{note}

\begin{theorem}
	Пусть $b \in \mathcal{B}^-(V)$. Тогда в $V$ существует базис $e$, в котором матрица $b$ имеею следующий вид:
	\[b \leftrightarrow_e B = \left(\begin{array}{@{}cccc@{}}
		\cline{1-1}
		\multicolumn{1}{|c|}{B_1} & 0 & \dots & 0\\
		\cline{1-2}
		0 & \multicolumn{1}{|c|}{B_2} & \dots & 0\\
		\cline{2-2}
		\vdots & \vdots & \ddots & \vdots\\
		\cline{4-4}
		0 & 0 & \dots & \multicolumn{1}{|c|}{B_m}\\
		\cline{4-4}
	\end{array}\right),\]
	
	где $\forall i \in \{1, \dots, m\}: B_i = (0)$ или $B_i = \begin{pmatrix}0&1\\
		-1&0\end{pmatrix}$.
\end{theorem}

\begin{proof}
	Докажем данное утверждение индукцией по $n \hm{:=} \dim{V}$. База, $n = 1$, и случай, когда $b = 0$, тривиальны. Пусть теперь $\exists \overline{e_1}, \overline{e_2} \in V: b(\overline{e_1}, \overline{e_2}) \ne 0$. Тогда эти векторы линейно независимы в силу кососимметричности формы $b$, и без ограничения общности можно считать, что $b(\overline{e_1}, \overline{e_2}) = 1$. Рассмотрим ограничение $b$ на $U := \langle\overline{e_1}, \overline{e_2}\rangle$, тогда в базисе $e' := (\overline{e_1}, \overline{e_2})$ матрица ограничения имеет вид:
	\[b|_U \leftrightarrow_{e'} \begin{pmatrix}0&1\\-1&0\end{pmatrix}\]
	
	Заметим, что $U$ невырожденно относительно $b$, откуда $V = U \oplus U^{\perp}$, и к $U^\perp$ применимо предположение индукции.
\end{proof}

\begin{corollary}
	Если $b \in \mathcal{B}^-(V)$, то $\rk{b}$ "--- четное число.
\end{corollary}

\begin{note}
	Полученные матрица и базис называются \textit{нормальными} для кососимметричной формы $b$. Перестановкой базисных векторов нормальную форму можно также преобразовать к следующему виду:
	\[B = \left(\begin{array}{@{}ccc@{}}
		\cline{2-2}
		0 & \multicolumn{1}{|c|}{E_k} & 0\\
		\cline{1-2}
		\multicolumn{1}{|c|}{-E_k} & 0 & 0\\
		\cline{1-1}
		0 & 0 & 0\\
	\end{array}\right)\]
\end{note}