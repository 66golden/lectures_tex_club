\subsection{Симметрические билинейные и квадратичные формы}

\textbf{До конца раздела} зафиксируем линейное пространство $V$ над полем $F$ и положим $n := \dim{V}$.

\begin{theorem}
	Пусть $\cha{F} \ne 2$. Тогда $\mathcal{B}(V) = \mathcal{B}^+(V) \oplus \mathcal{B}^-(V)$.
\end{theorem}

\begin{proof}
	Рассмотрим произвольную форму $b \in \mc B(V)$ и зададим $b^+ \in \mathcal{B}^+$, $b^- \in \mathcal{B}^-$ на произвольных $\overline u, \overline v \in V$ следующим образом:
	\[b^+(\overline{u}, \overline{v}) := \frac{b(\overline{u}, \overline{v}) + b(\overline{v}, \overline{u})}{2},~b^-(\overline{u}, \overline{v}) := \frac{b(\overline{u}, \overline{v}) - b(\overline{v}, \overline{u})}{2}\]
	
	Тогда $b = b^+ + b^-$, и, в силу произвольности выбора формы $b \in \mathcal{B}(V)$, получено равенство $\mathcal{B}(V) \hm{=} \mathcal{B}^+(V) + \mathcal{B}^-(V)$. Проверим теперь, что $\mathcal{B}^+(V) \cap \mathcal{B}^-(V) = \{0\}$. Действительно, если $b \in \mathcal{B}^+(V) \cap \mathcal{B}^-(V)$, то для любых $\overline u, \overline v \in V$ выполнено следующее:
	\[b(\overline{u}, \overline{v}) = b(\overline{v}, \overline{u}) = -b(\overline{u}, \overline{v}) \Rightarrow b(\overline{u}, \overline{v}) = 0\]
	
	Значит, $b = 0$, поэтому сумма $\mathcal{B}^+(V) + \mathcal{B}^-(V)$ "--- прямая.
\end{proof}

\begin{note}
	Если $\cha{F} = 2$, то теорема выше уже неверна, поскольку выполнено включение $\mc B^-(V) \le \mc B^+(V)$.
\end{note}

\begin{definition}
	Пусть $b \in \mathcal{B}^\pm(V)$. \textit{Ядром} формы $b$ называется подпространство $\ke{b} := \{\overline{v} \in V: \forall \overline{u} \in V: b(\overline{u}, \overline{v}) = 0\} = \{\overline{u} \in V: \forall \overline{v} \in V: b(\overline{u}, \overline{v}) = 0\} \le V$.
\end{definition}

\begin{note}
	Для произвольной билинейной формы $b \in \mc B(V)$ два множества выше необязательно совпадают и называются \textit{правым} и \textit{левым ядром} соответственно.
\end{note}

\begin{theorem}
	Для любой формы $b \in \mc B^\pm(V)$ выполнено равенство $\dim{\ke{b}} = \dim{V} - \rk{b}$.
\end{theorem}

\begin{proof}
	Пусть $e$ "--- произвольный базис в $V$, $b \leftrightarrow_e B$. Тогда для произвольного вектора $\overline{v} \in V$, $\overline{v} \leftrightarrow_e x$, выполнены следующие равносильности:
	\[\overline{v} \in \ke{b} \Leftrightarrow \forall \overline{u} \in V: b(\overline{u}, \overline{v}) \hm{=} 0 \Leftrightarrow \forall i \in \{1, \dotsc, n\}: b(\overline{e_i}, \overline{v}) = 0\]
	
	Последнее из условий выше равносильно тому, что $Bx = 0$. Размерность пространства решений полученной однородной системы равна $\dim{V} - \rk{B} = \dim{V} - \rk{b}$.
\end{proof}

\begin{definition}
	Пусть $b \in \mathcal{B}^\pm(V)$.
	\begin{itemize}
		\item Векторы $\overline{u}, \overline{v} \in V$ называются \textit{ортогональными относительно $b$}, если $b(\overline{u}, \overline{v}) = 0$.
  
		\item \textit{Ортогональным дополнением подпространства $U \le V$ относительно $b$} называется подпространство $U^\perp := \{\overline{v} \in V: \forall \overline{u} \in U: b(\overline{u}, \overline{v}) = 0\} \le V$.
	\end{itemize}
\end{definition}

\begin{example}
	Справедливы следующие утверждения:
	\begin{enumerate}
		\item Если $b$ "--- скалярное произведение в пространстве $V_n$, $n \in \{1, 2, 3\}$, то ортогональность относительно $b$ соответствует ортогональности геометрических векторов. Тогда, например, ортогональное дополнение к плоскости в $V_3$ относительно $b$ "--- это прямая, перпендикулярная ей.
		\item Если $b = 0$, то для любого подпространства $U \le V$ выполнено равенство $U^\perp = V$.
	\end{enumerate}
\end{example}

\begin{definition}
	Форма $b \in \mathcal{B}^\pm(V)$ называется \textit{невырожденной}, если $\rk{b} = \dim{V}$.
\end{definition}

\begin{theorem}Пусть $b \in \mathcal{B}^\pm(V)$, $U \le V$. Тогда:
	\begin{enumerate}
		\item $\dim{U^\perp} \ge \dim{V} - \dim{U}$
		\item Если форма $b$ "--- невырожденная, то $\dim{U^\perp} = \dim{V} - \dim{U}$
	\end{enumerate}
\end{theorem}

\begin{proof}
	Пусть $n := \dim{V}$, $k := \dim{U}$. Дополним базис $(\overline{e_1}, \dots, \overline{e_k})$ в $U$ до базиса $e := (\overline{e_1}, \dots, \overline{e_n})$ в $V$. Тогда, если $\overline{v} \in V$, $\overline{v} \leftrightarrow_e x$, то $\overline{v} \in U^\perp \hm\Leftrightarrow \forall i \in \{1, \dots, k\}: b(\overline{e_i}, \overline{v}) \hm{=} 0 \Leftrightarrow B'x = 0$, где $B'$ "--- матрица из первых $k$ строк матрицы $B$. Тогда, так как $\rk{B'} \le k$, то $\dim{U^\perp} \ge \dim{V} - \dim{U}$. Более того, если форма $b$ невырожденна, то матрица $B$ тоже невырожденна, откуда $\rk{B'} = k$ и $\dim{U^\perp} = \dim{V} - \dim{U}$.
\end{proof}

\begin{definition}
	Подпространство $U \le V$ называется \textit{невырожденным относительно $b\in \mathcal{B}^{\pm}(V)$}, если ограничение $b|_U \in \mathcal{B}^{\pm}(U)$ невырожденно.
\end{definition}

\begin{theorem}
	Пусть $b \in \mathcal{B}^{\pm}(V)$. Тогда подпространство $U \le V$ невырожденно относительно $b$ $\Leftrightarrow$ $V = U \oplus U^{\perp}$.
\end{theorem}

\begin{proof} Пусть $n := \dim{V}$, $k := \dim{U}$.
	\begin{itemize}
		\item[$\Rightarrow$] Построим базис $e$ аналогично предыдущей теореме. Если $b \leftrightarrow_e B$, то матрица $b|_U$ в базисе $(\overline{e_1}, \dots, \overline{e_k})$ "--- это левый верхний угол $B'$ матрицы $B$. Поскольку $U$ невырожденно относительно $b$, то $\rk{B'} = k$, следовательно, первые $k$ строк $B$ линейно независимы. Значит, $\dim{U^{\perp}} \hm{=} n - k$. Кроме того, $\ke{b|_U} = \{\overline{0}\}$, так как $\dim{\ke{b|_U}} = 0$. Значит, $\forall \overline{v} \in U, \overline{v} \ne \overline{0}: \exists \overline{u} \in U: b(\overline{u}, \overline{v}) \ne 0$. Это значит, что $U \cap U^{\perp} = \{\overline{0}\}$, то есть сумма $U + U^{\perp}$ "--- прямая, тогда $\dim(U \oplus U^\perp) = n$ и потому $U \oplus U^{\perp} = V$.
		
		\item[$\Leftarrow$] Если сумма $U \oplus U^{\perp}$ прямая, то $U \cap U^{\perp} \hm{=} \{\overline{0}\}$, тогда $\ke{b|_U} \hm{=} \{\overline{0}\}$, откуда $\rk{B'} = k$ и $U$ невырожденно относительно $b$.\qedhere
	\end{itemize}
\end{proof}

\begin{example} Зафиксируем базис $e := (\overline{e_1}, \overline{e_2})$ в пространстве $F^2$, и рассмотрим следующие матрицы:
	\[B = \begin{pmatrix}1&0\\0&0\end{pmatrix},~C = \begin{pmatrix}0&1\\\pm1&0\end{pmatrix}\]
	
	Тогда верны следующие утверждения:
	\begin{enumerate}
		\item Форма $b \leftrightarrow_e B$ вырожденна, но $\langle\overline{e_1}\rangle$ невырожденно относительно $b$.
		\item Форма $b\leftrightarrow_e C$ невырожденна, но $\langle\overline{e_1}\rangle$ вырожденно относительно $b$.
	\end{enumerate}
\end{example}

\begin{proposition}
	Пусть $b \in \mathcal{B}^{\pm}(V)$, $U \le V$ и  $V = U \oplus U^{\perp}$. Выберем базисы $(\overline{e_1}, \dots, \overline{e_k})$ в $U$ и $(\overline{e_{k+1}}, \dots, \overline{e_n})$ в $U^{\perp}$. Тогда в базисе $(\overline{e_1}, \dots, \overline{e_n})$ в $V$ форма $b$ имеет матрицу:
	\[B = \left(\begin{array}{@{}c|c@{}}
		B_1 & 0\\
		\hline
		0 & B_2
	\end{array}\right),\]
	где $B_1$, $B_2$ "--- матрицы форм $b|_U$ и $b|_{U^{\perp}}$
\end{proposition}

\begin{proof}
	Если $i \in \{1, \dots, k\}$ и $j \in \{k + 1, \dots, n\}$, то по определению ортогонального дополнения $b(\overline{e_i}, \overline{e_j}) = 0$.
\end{proof}

\begin{definition}
	\textit{Квадратичной формой}, соответствующей форме $b \in \mathcal{B}(V)$, называется функция $h: V \rightarrow F$ такая, что $\forall \overline{v} \in V: h(\overline{v}) = b(\overline{v}, \overline{v})$. Квадратичные формы на $V$ образуют линейное пространство над $F$, обозначаемое через $\mathcal{Q}(V)$.
\end{definition}

\begin{note}
	Разным билинейным формам могут соответствовать одинаковые квадратичные. Например, любой кососимметрической форме соответствует нулевая квадратичная.
\end{note}

\begin{theorem}
	Пусть $\cha{F} \ne 2$. Тогда для любой квадратичной формы $h$ на $V$ существует единственная форма $b \in \mathcal{B}^+(V)$, соответствующая ей.
\end{theorem}

\begin{proof}
	Пусть $b \in \mathcal{B}(V)$, $h(\overline{v}) = b(\overline{v}, \overline{v})$. Как уже было доказано, $b$ представима в виде $b^+ + b^-$, где $b^\pm \in \mathcal{B}^\pm(V)$, тогда $h(\overline{v}) = b(\overline{v}, \overline{v}) = b^+(\overline{v}, \overline{v})$. Более того, по $h$ можно однозначно восстановить $b^+ \in \mathcal{B}^+(V)$ следующим образом: $b^+(\overline{u}, \overline{v}) = \frac{h(\overline{u} + \overline{v}) - h(\overline{u}) - h(\overline{v})}{2}$.
\end{proof}

\begin{note}
	Согласно теореме выше, если $\cha{F} \ne 2$, то $\mathcal{Q}(V) \cong \mathcal{B}^+(V)$, и изоморфизм осуществляется описанным выше образом. Если же $\cha{F} = 2$, то теорема неверна. Рассмотрим следующую билинейную форму:
	\[b \leftrightarrow_e B = \begin{pmatrix}0&1\\0&0\end{pmatrix}\]
	
	Соответствующая данной билинейной форме квадратичная форма не выражается никакой симметрической билинейной формой.
\end{note}

\begin{definition}
	Пусть $\cha{F} \ne 2$, $h \in \mathcal{Q}(V)$. Симметрическая билинейная форма $b \in \mathcal{B}^+(V)$ называется \textit{полярной к $h$}, если $\forall \overline{v} \hm{\in} V: h(\overline{v}) = b(\overline{v}, \overline{v})$. \textit{Матрицей квадратичной формы $h$} в базисе $e$ называется матрица $B$ полярной к ней формы $b$ в базисе $e$. Обозначение "--- $h \leftrightarrow_e B$.
\end{definition}

\begin{note}
	Матрица квадратичной формы всегда симметрична.
\end{note}

\begin{theorem}
	Пусть $\cha{F} \ne 2$, $h \in \mathcal{Q}(V)$. Тогда в пространстве $V$ существует такой базис $e$, что $h$ в этом базисе имеет диагональную матрицу.
\end{theorem}

\begin{proof}
	Проведем индукцию по $n := \dim{V}$. База, $n = 1$, тривиальна. Пусть теперь $n > 1$, $h \in \mathcal{Q}(V)$, $b \in \mathcal{B}^+(V)$ "--- полярная к $h$ форма. Если $h = 0$, то $b = 0$, поэтому у $h$ нулевая матрица. Если же $h \ne 0$, то $\exists \overline{e_1} \in V: h(\overline{e_1}) \ne 0$. Тогда $\langle\overline{e_1}\rangle$ невырожденно относительно $b$, откуда $V = \langle\overline{e_1}\rangle \oplus \langle\overline{e_1}\rangle^\perp$. По предположению индукции, в $U := \langle\overline{e_1}\rangle^\perp$ существует базис, в котором матрица $h|_U$ диагональна, и объединение этого базиса с $\overline{e_1}$ и дает искомый базис в $V$.
\end{proof}

\begin{note}
	Диагональные элементы разных диагональных матриц одной квадратичной формы могут быть различными. Например, в базисе $(2\overline{e_1}, \overline{e_2}, \dots, \overline{e_n})$ элемент в левом верхнем углу в 4 раза больше, чем в $(\overline{e_1}, \overline{e_2}, \dots, \overline{e_n})$, так как он равен $h(2\overline{e_1}) = b(2\overline{e_1}, 2\overline{e_1}) \hm{=} 4b(\overline{e_1}, \overline{e_1}) = 4h(\overline{e_1})$
\end{note}

\begin{corollary}
	Пусть $F = \R$, $h \in \mathcal{Q}(V)$. Тогда в пространстве $V$ существует такой базис $e$, что $h$ в этом базисе имеет диагональную матрицу с числами $0$ и $\pm1$ на главной диагонали.
\end{corollary}

\begin{proof}
	Пусть $f = (\overline{f_1}, \dots, \overline{f_n})$ "--- базис, в котором матрица $h$ диагональна. Тогда искомым базисом является базис $e \hm{=} (\overline{e_1}, \dots, \overline{e_n})$ такой, что для каждого $i \in \{1, \dotsc, n\}$ вектор $\overline{e_i}$ имеет вид:
	\[\overline{e_i} := \left\{\begin{aligned}
		&\overline{f_i}\text{, если }h(\overline{f_i}) = 0\\
		&\frac{1}{\sqrt{\left|h(\overline{f_i})\right|}}\overline{f_i}\text{, если }h(\overline{f_i}) \ne 0
	\end{aligned}\right.\]
	
	В базисе $e$ на главной диагонали матрицы будут только числа $0$ и $\pm1$, а элементы вне диагонали останутся нулевыми.
\end{proof}