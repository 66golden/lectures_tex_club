\subsection{Свертка тензора}

\textbf{В данном разделе} зафиксируем линейное пространство $V$ над полем $F$.

\begin{example}
	Рассмотрим тензор $\phi \in \mathbb{T}^1_1$. Пусть в некотором базисе он задается координатами $\phi^i_j$. При переходе к новому базису с матрицей перехода $(a^i_j)$ и обратной матрицей перехода $(b^i_j)$ координаты тензора меняются по закону $\phi^i_j = a^i_k\phi'^k_l b^l_j$. Тогда $\phi^i_i = a^i_k\phi'^k_lb^l_i \hm{=} \delta^l_k\phi'^k_l = \phi'^k_k$. Таким образом, мы доказали тензорным способом инвариантность следа $\phi^i_i$ относительно замены координат.
\end{example}

\begin{definition}
	\textit{Сверткой} тензора $t \in \mathbb{T}^p_q$ по индексам $i_p, j_q$ называется тензор $t' \in \mathbb T^{p-1}_{q-1}$ с координатами следующего вида:
	\[\widetilde t^{i_1, \dots, i_{p-1}}_{j_1, \dots, j_{q-1}} = t^{i_1, \dots, i_{p - 1}, i}_{j_1, \dots, j_{q-1}, i}\]
	
	Свертка по другим парам из верхнего и нижнего индексов определяется аналогично.
\end{definition}

\begin{note}
	Аналогично проверке для следа оператора, можно показать, что полученный объект действительно является тензором, поскольку величина, получаемая при суммировании по $i$ и фиксации всех остальных индексов, не зависит от выбора базиса.
\end{note}

\begin{note}
	Зафиксируем базис $e$ в $V$. Тогда свертка по индексам $i_p, j_q$ действует на базисных векторах пространства $\mathbb T^p_q$ следующим образом:
	\[
	e_{i_1} \otimes \dots \hm{\otimes} e_{i_p} \otimes e^{j_1} \otimes \dots \otimes e^{j_q} \mapsto e^{j_q}(e_{i_p}) e_{i_1} \otimes \dots \hm{\otimes} e_{i_{p - 1}} \otimes e^{j_1} \otimes \dots \otimes e^{j_{q - 1}}
	\]
	
	Значит, по полилинейности, для произвольных $\overline{v_1}, \dotsc, \overline{v_p} \in V$, $u^1, \dotsc, u^q \in V^*$:
	\[\overline{v_1} \otimes \dots \hm{\otimes} \overline{v_p} \otimes u^1 \otimes \dots \otimes u^q \mapsto u^q(\overline{v_p}) \overline{v_{1}} \otimes \dots \hm{\otimes} \overline{v_{p - 1}} \otimes u^1 \otimes \dots \otimes u^{q - 1}\]
\end{note}

\begin{example}
	Рассмотрим несколько примеров свертки:
	\begin{enumerate}
		\item Пусть $\overline{v} \in V$, $u \in V^*$. Тогда свертка тензора $u \otimes \overline{v}$ "--- это скаляр $u(\overline{v})$.
		\item Пусть $b \in \mathcal{B}(V)$ "--- тензор с координатами $b_{ij}$, $\overline{u}, \overline{v} \in V$. Тогда скаляр $b(\overline{u}, \overline{v}) = u^ib_{ij}v^j$ получается как \textit{двойная}, или \textit{полная}, свертка тензора $\overline{u} \otimes b \otimes \overline{v}$.
		\item Пусть $\phi \in \mathcal{L}(V)$ "--- тензор с координатами $\phi^i_j$, $\overline{v} \in V$. Тогда вектор $\phi(\overline{v})$ имеет координаты $\phi^i_jv^j$.
		\item Пусть $\phi, \psi \in \mathcal{L}(V)$ "--- тензоры с координатами $\phi^i_j, \psi^k_l$. Тогда тензор $\phi \circ \psi$ имеет координаты $\phi^i_j\psi^j_k$.
		\item Пусть $V$ "--- евклидово пространство, в нем введено скалярное произведение, или \textit{метрический тензор}, с координатами $g_{ij}$. Тогда канонический изоморфизм между $V$ и $V^*$ осуществляется сопоставлением $v^i \mapsto v^ig_{ij}$, называемым \textit{опусканием индекса}. На $V^*$ тоже можно задать скалярное произведение как тензор с координатами $g^{ij}$, тоже называемый метрическим тензором, позволяющий, наоборот, поднимать индексы. Можно также показать, что $g_{ij}g^{ik} = \delta^k_j$.
		\item Пусть $\phi \in \mathcal{L}(V)$ "--- тензор с координатами $\phi^i_j$. Если в пространстве $V$ задано скалярное произведение с координатами $g_{ij}$, то сопоставление $\phi^i_j \mapsto \phi^i_jg_{ik}$ осуществляет это изоморфизм между $\mathcal{L}(V)$ и $\mathcal{B}(V)$.
	\end{enumerate}
\end{example}