\subsection{Приведение к главным осям}

В евклидовом случае речь пойдет о квадратичных формах, в эрмитовом "--- об эрмитовых квадратичных формах. В обоих случаях множество квадратичных форм будем обозначать через $\mathcal{Q}(V)$.

\begin{theorem}[о приведении к главным осям]
	Пусть $V$ "--- евклидово (эрмитово) пространство,  $q \in \mathcal{Q}(V)$. Тогда в $V$ существует ортонормированный базис $e$, в котором $q$ имеет диагональный вид.
\end{theorem}

\begin{proof}
	Пусть $b \in \mathcal{B}^+(V)$ "--- $\theta$-линейная форма, полярная к $q$. Тогда $\exists \phi \in \mathcal{L}(V)$ такой, что $b(\overline{u}, \overline{v}) = (\phi(\overline{u}), \overline{v})$. При этом:
	\[(\phi(\overline{u}), \overline{v}) = b(\overline{u}, \overline{v}) = \overline{b(\overline{v}, \overline{u})} = \overline{(\phi(\overline{v}), \overline{u})} = (\overline{u}, \phi(\overline{v}))\]
	
	Значит, $\phi$ "--- самосопряженный, и в $V$ существует ортонормированный базис $e$, в котором $\phi$ диагонализуем. Тогда если $\phi \leftrightarrow_e A$, то $b \leftrightarrow_e A^T$ и $q \leftrightarrow_e A^T$, поэтому форма $q$ тоже имеет диагональную матрицу в базисе $e$.
\end{proof}

\begin{note}
	Напротив, если в ортонормированном базисе $e$ матрица формы $q$ диагональна, то и матрица оператора $\phi$ диагональна и, следовательно, задана однозначно собственными значениями $\phi$. Значит, диагональный вид $q$ в ортонормированном базисе определен однозначно.
\end{note}

\begin{theorem}
	Пусть $V$ "--- линейное пространство над $\mathbb{R}$ (над $\mathbb{C}$), $q_1, q_2 \hm{\in} \mathcal{Q}(V)$, и $q_2$ положительно определена. Тогда в $V$ существует такой базис $e$, в котором матрицы форм $q_1$ и $q_2$ диагональны.
\end{theorem}

\begin{proof}
	Пусть $b$ "--- $\theta$-линейная форма, полярная к $q_2$. Тогда $b$  можно объявить $b$ скалярным (эрмитовым скалярным) произведением на $V$. В полученном евклидовом (эрмитовом) пространстве форма $q_1$ приводится к главным осям в некотором ортонормированном базисе $e$. Поскольку базис $e$ "--- ортонормированный, то в этом же базисе $q_2$ имеет диагональный вид $E$.
\end{proof}

\begin{note}
	Требование положительной определенности в теореме существенно. Зафиксируем некоторый базис $e := (\overline{e_1}, \overline{e_2})$ в пространстве $V$ над $\R$ (над $\Cm$) и рассмотрим формы $q_1, q_2 \in \mathcal{Q}(V)$ следующего вида:
	\[q_1 \leftrightarrow_e \begin{pmatrix}1&0\\0&-1\end{pmatrix},~q_2 \leftrightarrow_e \begin{pmatrix}0&1\\1&0\end{pmatrix}\]
	
	Предположим, что в пространстве $V$ существует такой базис $\mathcal{F}$, в котором обе формы имеют диагональный вид, то есть для некоторых вещественных чисел $\lambda_1, \mu_1, \lambda_2, \mu_2$ выполнено следующее:
	\[q_1 \leftrightarrow_{\mathcal{F}} \begin{pmatrix}\lambda_1&0\\0&\lambda_2\end{pmatrix},~q_2 \leftrightarrow_{\mathcal{F}} \begin{pmatrix}\mu_1&0\\0&\mu_2\end{pmatrix}\]
	
	Тогда их линейная комбинация $\mu_1q_1 - \lambda_1q_2$ вырожденна, но при подсчете в базисе $e$ определитель матрицы формы $\mu_1q_1 - \lambda_1q_2$ равен $-\mu_1^2 - \lambda_1^2$. Но обе формы $q_1, q_2$ невырожденны, поэтому $\mu_1, \lambda_1$ отличны от нуля и $-\mu_1^2 - \lambda_1^2 \ne 0$ --- противоречие.
\end{note}

\begin{note}
	В первом семестре мы уже использовали приведение к главным осям в маломерном случае. Действительно, рассмотрим уравнение кривой второго порядка:
	\[Ax^2 + 2Bxy + Cy^2 + 2Dx + 2Ey + F = 0\]
	
	Здесь выражение $Ax^2 + 2Bxy + Cy^2$ в базисе $e$ задает следующую форму $q$:
	\[q \leftrightarrow_e \begin{pmatrix}A&B\\B&C\end{pmatrix}\]
	
	Для дальнейшего приведения уравнения к каноническому виду мы приводили форму $q$ к виду $A'x'^2 + C'y'^2$ в базисе $e'$. По только что доказанной теореме, тот же метод работает в многомерном случае. Рассмотрим уравнение следующего вида:
	\[\sum_{1 \le i, j \le n}b_{ij}x_ix_j + \sum_{1 \le i \le n}c_ix_i \hm{+} d = 0\]
	
	Зафиксируем базис $e$ в $n$-мерном евклидовом (эрмитовом) пространстве $V$. Пусть квадратичная форма $h$ в базисе $e$  имеет вид
	$h(\overline{x}) = \sum_{1 \le i, j \le n}b_{ij}x_ix_j$. Тогда $h$ можно привести к главным осям, и в некотором базисе $e'$ она примет вид $\sum_{1 \le i \le n}b_i'x_i'^2$. Тогда исходное уравнение примет следующий вид:
	\[\sum_{1 \le i \le n}b_i'x_i'^2 \hm{+} \sum_{1 \le i \le n}c_i'x_i' + d = 0\]
	
	Далее можно применить аналогичную двумерному случаю процедуру избавления от линейных членов $c_ix_i'$, если соответствующие квадратичные члены $b_i'x_i'^2$ отличны от нуля, и затем привести уравнение к окончательному виду с не более чем одним линейным членом.
\end{note}