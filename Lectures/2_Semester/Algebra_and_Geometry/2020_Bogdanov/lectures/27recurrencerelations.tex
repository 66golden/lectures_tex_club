\subsection{Линейные рекурренты}

\textbf{До конца раздела} зафиксируем поле $F$.

\begin{definition}
	Пусть $F$ "--- поле. Обозначим через $F^{\infty}$ \textit{линейное пространство последовательностей} вида $(a_0, a_1, \dots)$, с элементами из $F$, с покоординатными операциями сложения и умножения на скаляр из $F$.
\end{definition}

\begin{definition}
	Последовательность $(a_i) \in F^{\infty}$ называется \textit{линейной рекуррентой} с \textit{характеристическим многочленом} $p = x^n + p_{n - 1}x^{n - 1} \hm{+} \dots + p_0 \in F[x]$, если выполнено следующее условие:
	\begin{equation}\tag{$\dagger$}\label{linrec}
		\forall k \in \mathbb{N} \cup \{0\}: a_{k + n} + p_{n - 1}a_{k + n - 1} + \dots + p_0a_k = 0
	\end{equation}
	
	Обозначим множество рекуррент с характеристическим многочленом $p$ через $V_p$.
\end{definition}

\begin{note}
	Если $p_0 = 0$, то элемент $a_0$ никак не влияет на все остальные элементы последовательности и может принимать произвольные значения, а остальные элементы образуют линейную рекурренту с характеристическим многочленом $\frac{p}{x} \in F[x]$. Поэтому далее будем считать, что $p_0 \ne 0$.
\end{note}

\begin{proposition}
	$V_p$ "--- линейное пространство над $F$, причем $\dim{V_p} = \deg{p} = n$.
\end{proposition}

\begin{proof}
	Нетривиальна только вторая часть утверждения. Заметим, что любая последовательность $(a_i) \in V_p$ однозначно задается первыми своими $n$ членами $a_0, a_1, \dots, a_{n - 1}$, и рассмотрим в $V_p$ следующие последовательности:
	\begin{align*}
		\overline{e_0} &= (\overset{0}{1}, \overset{1}{0}, \dotsc, 0, 0, \overset{n}{-p_0}, \dots)\\
		&\dotsc\\
		\overline{e_{n - 1}} &= (0, 0\dotsc, 0, 1, -p_{n - 1}, \dots)
	\end{align*}
	
	Эти последовательности, очевидно, образуют линейно независимую систему, и любая последовательность $(a_i) \in V_p$ выражается через них как $\sum_{i = 0}^{n - 1}a_i\overline{e_i}$.
\end{proof}

\begin{definition}
	\textit{Оператором левого сдвига} на $F^\infty$ называется оператор ${\phi: F^{\infty} \rightarrow F^{\infty}}$, заданный для каждой последовательности $(a_i) \in F^{\infty}$ как $\phi((a_i)) := (a_{i + 1})$.
\end{definition}

\begin{proposition}
	Пусть $p \in F[x]$, $\phi$ "--- оператор левого сдвига. Тогда $V_p = \ke{p(\phi)}$.
\end{proposition}

\begin{proof}
	Для любой последовательности $A = (a_i) \in F^\infty$ выполнена равносильность $A \in \ke{p(\phi)} \Leftrightarrow p(\phi)(A) = (0)$. Заметим, что для любого $k \in \mathbb{N} \cup \{0\}$ выполнено равенство $[p(\phi)(A)]_k = a_{k+n} + a_{k + n - 1}p_{n - 1} + \dots + a_kp_0$, поэтому $A \in \ke{p(\phi)} \Leftrightarrow A$ удовлетворяет $\eqref*{linrec}$.
\end{proof}

\begin{note}
	В частности, из доказанного следует, что $V_p$ инвариантно относительно $\phi$. С этого момента рассматривать оператор $\psi := \phi|_{V_p} \hm{\in} \mathcal{L}(V_p)$ для произвольного $p \in F[x]$.
\end{note}

\begin{proposition}
	Минимальный многочлен $\mu_\psi$ оператора $\psi$ равен $p$ с точностью до ассоциированности.
\end{proposition}

\begin{proof}
	С одной стороны, $V_p = \ke{p(\phi)}$, поэтому $\im{p(\psi)} = p(\phi)(V_p) = \{\overline{0}\}$, значит, $\mu_\psi\mid p$. С другой стороны, $\forall A \in V_p: \mu_\psi(\psi)(A) = (0) \hm{\Leftrightarrow} \mu_\psi(\phi)(A) = (0)$, тогда $A \in \ke{\mu_\psi(\psi)} \Leftrightarrow A \in V_{\mu_{\psi}}$. Значит, $V_p \le V_{\mu_\psi}$, откуда $\deg{p} = \dim{V_p} \le \dim{V_{\mu_\psi}} = \deg{\mu_\psi}$. Таким образом, $p = \mu_\psi$ с точностью до ассоциированности.
\end{proof}

\begin{proposition}
	Характеристический многочлен $\chi_\psi$ оператора $\psi$ равен $(-1)^np$.
\end{proposition}

\begin{proof}
	В использовавшемся ранее базисе $e = (\overline{e_0}, \dots, \overline{e_{n - 1}})$ матрица преобразования $\psi$ имеет следующий вид:
	\[\psi \leftrightarrow_e A_p = \begin{pmatrix}
		0 & 1 & \dots & 0 \\ 
		0 & \ddots & \ddots & \vdots \\ 
		\vdots & \ddots & \ddots & 1 \\ 
		-p_0 & -p_1 & \dots & -p_{n-1}
	\end{pmatrix}\]
	
	Докажем индукцией по $n$, что $|A_p - \lambda E| = (-1)^np(\lambda)$. База, $n = 1$, тривиальна. Докажем переход, используя разложение определителя по первому столбцу:
	\begin{multline*}\begin{vmatrix}
			-\lambda& 1 & \dots & 0 \\ 
			0 & \ddots & \ddots & \vdots \\ 
			\vdots & \ddots & \ddots & 1 \\ 
			-p_0 & -p_1 & \dots & -p_{n-1}-\lambda
		\end{vmatrix} = -\lambda\begin{vmatrix}
			-\lambda& 1 & \dots & 0 \\ 
			0 & \ddots & \ddots & \vdots \\ 
			\vdots & \ddots & \ddots & 1 \\ 
			-p_1 & -p_2 & \dots & -p_{n-1}-\lambda
		\end{vmatrix} + (-1)^{n+1}(-p_0) =\\
		= (-\lambda)(-1)^{n-1}(p_1 + p_2\lambda + \dots + p_{n - 1}\lambda^{n-2}) + (-1)^np_0 = (-1)^np(\lambda)
	\end{multline*}
	
	Таким образом, $\chi_\psi(\lambda) = (-1)^np(\lambda)$.
\end{proof}

\begin{note}
	Матрица $A_p \in M_n(F)$ называется \textit{сопутствующей матрицей} многочлена $p \in F[x]$.
\end{note}

\begin{theorem}
	Пусть $p \in F[x]$ имеет вид $\eqref*{charpol}$. Тогда каждое решение линейной рекурренты с характеристическим многочленом $p$ имеет следующий вид для некоторого набора коэффициентов $\{\beta_{it}\} \subset F$:
	\[(a_m) = \left(\sum_{i = 1}^k\sum_{t = 1}^{\alpha_i}\beta_{it}C_m^{t - 1}\lambda_i^{m + 1 - t}\right)\]
\end{theorem}

\begin{proof}
	Поскольку $\mu_\psi = \chi_\psi = (-1)^np$, то для каждого собственного значения $\lambda$ в жордановой нормальной форме оператора $\psi$ есть ровно одна жорданова клетка с таким значением, и она имеет размер $\alpha$. Найдем последовательности $A_1, A_2, \dots, A_\alpha$ такие, что:
	\begin{align*}
		\phi_\lambda(A_1) &= (0)\\
		\phi_\lambda(A_2) &= A_1\\
		&\dotsc\\
		\phi_\lambda(A_{\alpha}) &=  A_{\alpha - 1}
	\end{align*}
	
	Все эти последовательности обнуляются оператором $(\phi_\lambda)^\alpha$, и, как следствие, оператором $p(\phi)$, поэтому они лежат в $V_p$. Именно они образуют жорданов базис в $V^\lambda$. Итак, пусть $A_1 = (1, \lambda, \lambda^2, \dots)$, тогда $\psi_\lambda(A) = (0)$. Теперь для каждого $t \in \{1, \dots, \alpha\}$ положим $A_t := \left(f_t(m)\lambda^{m + 1 - t}\right)$ для некоторой функции $f: \N \cup \{0\} \to F$, причем $f_1 \equiv 1$ по построению. Считая $A_t$ уже найденным, найдем $A_{t + 1}$:
	\begin{align*}
		\phi_\lambda(A_{t + 1}) &= A_t\\
		f_{t + 1}(m + 1)\lambda^{m + 1 - t} - \lambda f_{t + 1}(m)\lambda^{m - t} &= f_t(m)\lambda^{m + 1 - t}\\
		f_{t + 1}(m + 1) - f_{t + 1}(m) &= f_t(m)
	\end{align*}
	
	Базе $f_1 \equiv 1$ и рекуррентному соотношению выше удовлетворяет семейство функций $\{f_t\}_{t = 1}^\alpha$ такое, что $f_t(m) = C_{m}^{t - 1}$ для любых $t \in \{1, \dotsc, \alpha\}$ и $m \in \N \cup \{0\}$. Таким образом, $A_t = (C_m^{t - 1}\lambda^{m + 1 - t})$ для каждого $t \in \{1, \dotsc, \alpha\}$. Объединение жордановых базисов клеток и будет искомым базисом в $V_p$, который позволяет представить каждый элемент из $V_p$ в требуемом виде.
\end{proof}

\begin{note}
	Полученная формула справедлива в поле любой характеристики, поскольку число сочетаний "--- это целое число.
\end{note}