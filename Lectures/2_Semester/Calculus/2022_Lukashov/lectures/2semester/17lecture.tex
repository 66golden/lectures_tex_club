\subsection{Комплексные ряды Тейлора}

\begin{note}
	Формально говоря, говорить о комплексных рядах Тейлора мы не имеем права, ибо нет понятия дифференцируемости для комплекснозначных функций. Однако, мы должны прояснить тот факт, что равенство
	\[
		re^{i\phi} := r(\cos \phi + i\sin \phi)
	\]
	мы брали за определение (ну или же просто сказали, что докажем позже). Время настало.
\end{note}

\begin{definition}
	\textit{Комплексной экспонентой} называется комплекснозначная функция $e^z$:
	\[
		e^z := e^x (\cos y + i \sin y),\ z = x + iy
	\]
\end{definition}

\begin{note}
	Аналогично определяются \textit{комплексный синус и косинус}:
	\[
		\cos z := \frac{e^{iz} + e^{-iz}}{z}
	\]
	\[
		\sin z := \frac{e^{iz} - e^{-iz}}{zi}
	\]
\end{note}

\begin{proposition}
	Для комплексной экспоненты верны следующие свойства:
	\begin{enumerate}
		\item \(e^{z_1 + z_2} = e^{z_1} \cdot e^{z_2}\)
		
		\item \(e^{-z} = 1 / e^z\)
	\end{enumerate}
\end{proposition}

\begin{proof}
	Положим $\forall z_1 = x_1 + iy_1,\ z_2 = x_2 + iy_2$. Тогда
	\begin{itemize}
		\item \(e^{z_1 + z_2} = e^{x_1 + x_2} (\cos(y_1 + y_2) + i\sin(y_1 + y_2)) = \big(e^{x_1}(\cos y_1 + i\sin y_1)e^{x_2}(\cos y_2 + i\sin y_2)\big) = e^{z_1} \cdot e^{z_2}\)
		
		\item \(e^{-z} = e^{-x}(\cos(-y) + i\sin(-y)) = \frac{1}{e^x} \cdot \frac{1}{\cos(y) + i\sin(y)} = \frac{1}{e^z}\)
	\end{itemize}
\end{proof}

\begin{theorem}
	Для $\forall z \in \Cm$ верно, что
	\[
		e^z = \row{n = 0}{\frac{z^n}{n!}};\ \cos z = \row{n = 0}{(-1)^n \frac{z^{2n}}{(2n)!}};\ \sin z = \row{n = 0}{(-1)^n \frac{z^{2n + 1}}{(2n + 1)!}}
	\]
\end{theorem}

\begin{proof}
	Воспользуемся рядами вещественных функций:
	\[
		e^x = \row{n = 0}{\frac{x^n}{n!}};\ \cos y = \row{n = 0}{(-1)^n \frac{y^{2n}}{(2n)!}};\ \sin y = \row{n = 0}{(-1)^n \frac{y^{2n + 1}}{(2n + 1)!}}
	\]
	Распишем выражение $\cos y + i\sin y$:
	\[
		\cos y + i\sin y = \row{n = 0}{\frac{(iy)^{2n}}{(2n)!}} + \row{n = 0}{\frac{(iy)^{2n + 1}}{(2n + 1)!}} = \row{n = 0}{\frac{(iy)^n}{n!}}
	\]
	Отсюда получаем $e^z$ через произведение по Коши:
	\[
		e^z = e^x(\cos y + i\sin y) = \row{n = 0}{\suml_{k = 0}^n \frac{x^k}{k!} \cdot \frac{(iy)^{n - k}}{(n - k)!}} = \row{n = 0}{\frac{1}{n!}\suml_{k = 0}^n C_n^k x^k (iy)^{n - k}} = \row{n = 0}{\frac{(x + iy)^n}{n!}} = \row{n = 0}{\frac{z^n}{n!}}
	\]
	Для синуса и косинуса ряды уже по определению через доказанную экспоненту.
\end{proof}

\section{Меры Жордана и Лебега}

\begin{note}
	Забегая вперёд, мера Жордана, вообще говоря, мерой не является. В иностранных статьях она называется не <<Jordan measure>>, а <<Jordan content>>.
\end{note}

\subsection{Элементарные множества}

\begin{note}
	Через $\trbr{a; b}$ будем обозначать \textit{конечный промежуток}, то есть что-то из $[a; b], [a; b), (a; b], (a; b)$.
\end{note}

\begin{definition}
	\textit{Брусом в $\R^n$} назовём множество $P$:
	\[
		P = \prodl_{k = 1}^n \trbr{a_k; b_k}
	\]
\end{definition}

\begin{definition}
	\textit{Объёмом бруса $P$} назовём число $|P|$:
	\[
		|P| = \prodl_{k = 1}^n (b_k - a_k)
	\]
\end{definition}

\begin{proposition}~
	\begin{enumerate}
		\item Если $P \subset Q$ - брусья, то $|P| \le |Q|$
		
		\item Если брус $P = \bigsqcup\limits_{i = 1}^m P_i$, то $|P| = \suml_{n = 1}^m |P_i|$ \textit{(Конечная аддитивность объёма на множестве брусьев)}
	\end{enumerate}
\end{proposition}

\begin{proof}
	Тривиально.
\end{proof}

\begin{definition}
	$M$ называется \textit{элементарным множеством}, если оно представимо дизъюнктным объединением конечного числа брусьев.
\end{definition}

\begin{definition}
	\textit{Объёмом элементарного множества} $M = \bigsqcup\limits_{i = 1}^m P_i$ называется число $|M|$:
	\[
		|M| = \suml_{i = 1}^m |P_i|
	\]
\end{definition}

\begin{lemma}
	Объединение, пересечение, разность и симметрическая разность двух элементарных множеств является элементарным множеством.
\end{lemma}

\textcolor{red}{Сюда бы маленькую картиночку с $P$ и $Q$. 39я минута 17й лекции за весну 2022.}

\begin{proof}
	Будем обозначать за $P, Q$ - брусья, а $M, N$ - элементарные множества. Тогда
	\begin{enumerate}
		\item Пересечение. Для начала заметим, что $P \cap Q$ - брус. Значит, $M \cap Q$ - тоже элементарное множество. Ну и отсюда становится очевидным, что $M \cap N$ - элементарное множество.
		
		\item Разность. Для начала, $P \bs Q = P \cap Q^C$, где $Q^C$ - дополнение к $Q$. При этом
		\[
			Q^C = \bigsqcup\limits_{j = 1}^m Q_j
		\]
		где $Q_j$ могут иметь бесконечные рёбра. Допуская такое свойство, получаем, что $P \bs Q$ - брус. Отсюда следующее:
		\[
			M \bs Q = \bigsqcup\limits_{i = 1}^m (P_i \bs Q) \Ra \text{брус}
		\]
		Ну и тогда
		\[
			M \bs N = M \bs \bigcup\limits_{j = 1}^k Q_j = M \cap \left(\bigcap\limits_{j = 1}^k Q_j^C\right) = \bigcap\limits_{j = 1}^k (M \cap Q_j^C)
		\]
		
		\item Объединение. 
		\[
			M \cup Q = (M \bs Q) \sqcup Q
		\]
		Таким способом мы можем разложить одно из $M, N$ на брусья и проделать для всех его брусьев операцию выше.
		
		\item Симметрическая разность
		\[
			M \tr N = (M \bs N) \sqcup (N \bs M)
		\]
	\end{enumerate}
\end{proof}

\begin{note}
	Доказанная лемма означает, что совокупность элементарных множеств является \textit{кольцом множеств}. В качестве нуля - $\emptyset$, а в качестве единицы будем дальше брать $K_I$:
	\[
		K_I = \left[-\frac{1}{2}; \frac{1}{2}\right]^n
	\]
	Отметим, что совокупность элементарных подмножеств $K_I$ образует \textit{алгебру множеств}.
\end{note}

\begin{theorem} (Свойства объёма на кольце элементарных множеств)
	\begin{enumerate}
		\item Определение объёма корректно (то есть не зависит от того, как разбить элементарное множество)
		
		\item Если $M, N$ - элементарные множества, то
		\[
			|M \cup N| + |M \cap N| = |M| + |N|
		\]
		
		\item Если $M \subset N$, то $|M| \le |N|$
		
		\item Имеет место конечная аддитивность. То есть
		\[
			M = \bigsqcup\limits_{i = 1}^m M_i \Longrightarrow |M| = \suml_{i = 1}^m |M_i|
		\]
	\end{enumerate}
\end{theorem}

\begin{proof}~
	\begin{enumerate}
		\item Пусть \(M = \bsqcupl_{i = 1}^m P_i = \bsqcupl_{j = 1}^k Q_j = \bsqcupl_{i, j} P_i \cap Q_j\). \textcolor{red}{Далее лектор ссылается на существование <<канонического>> разбиения, которое можно получить из данных, если отсортировать границы брусьев по каждой из координат. На данный момент я пожалуй оставлю это так. Для желающих - минута 55 17й лекции весны 2022.}
	\end{enumerate}
	Остальные свойства тоже геометрически очевидны.
\end{proof}

\begin{definition}
	\textit{$\eps$-растяжением бруса} $P = \prodl_{k = 1}^n \trbr{a_k; b_k}$ называется брус $P^\eps$:
	\[
		P^\eps = \prodl_{k = 1}^n \left(\frac{a_k + b_k}{2} - \frac{b_k - a_k}{2} \cdot \sqrt[n]{1 + \eps}; \frac{a_k + b_k}{2} + \frac{b_k - a_k}{2} \cdot \sqrt[n]{1 + \eps}\right)
	\]
	Дополнительно отметим, что
	\[
		|P^\eps| = \prodl_{k = 1}^n \big((b_k - a_k)\sqrt[n]{1 + \eps}\big) = |P|(1 + \eps)
	\]
\end{definition}

\begin{definition}
	\textit{$\eps$-сжатием бруса} $P = \prodl_{k = 1}^n \trbr{a_k; b_k}$ называется брус $P^{-\eps}$:
	\[
		P^{-\eps} = \prodl_{k = 1}^n \left[\frac{a_k + b_k}{2} - \frac{b_k - a_k}{2} \cdot \sqrt[n]{1 - \eps}; \frac{a_k + b_k}{2} + \frac{b_k - a_k}{2} \cdot \sqrt[n]{1 - \eps}\right]
	\]
	Дополнительно отметим, что
	\[
		|p^{-\eps}| = |P| \cdot (1 - \eps)
	\]
\end{definition}

\begin{note}
	Для любого достаточно малого $\eps > 0$ верно, что $P^{-\eps} \subset P \subset P^\eps$
\end{note}

\begin{theorem} (Счётная, или $\sigma$-аддитивность объёма на кольце элементарных множеств)
	Если элементарное множество $M$ представлено дизъюнктным объединением счётного числа элементарных множеств $M_i$, то
	\[
		|M| = \row{n = 1}{|M_i|}
	\]
\end{theorem}

\begin{proof}
	Докажем равенство через неравенства в 2 стороны:
	\begin{enumerate}
		\item $\le$. Пусть $M \subset \bsqcupl_{i = 1}^\infty M_i$. Если сумма объёмов бескончна, то включение очевидно. Иначе
		\[
			M_i = \bsqcupl_j P_{ij} \Ra \bsqcupl_{i = 1}^\infty M_i = \bsqcupl_{k = 1}^\infty P_k
		\]
		При этом $M = \bsqcupl_{i = 1}^m Q_i$. Рассмотрим $\eps$-сжатие для $Q_i$ и $\alpha_k$-растяжения для $P_k$. Тогда
		\[
			\bsqcupl_{i = 1}^m Q_i^{-\eps} \subset \bcupl_{k = 1}^\infty P_k^{\alpha_k}
		\]
		Идея состоит в том, что слева стоит замкнутое ограниченное множество, то есть компактное. Значит, из его компактности:
		\[
			\exists \{k_1, \ldots, k_N\} \such \bsqcupl_{i = 1}^m Q_i^{-\eps} \subset \bcupl_{l = 1}^N P_{k_l}^{\alpha_{k_l}}
		\]
		Дополнительно $|\bigsqcup_{i = 1}^m Q_i^{-\eps}| = |M|(1 - \eps)$, а в правой части
		\[
			\left|\bigcup_{l = 1}^N P_{k_l}^{\alpha_{k_l}}\right| \le \sum_{l = 1}^N |P_{k_l}|(1 + \alpha_{k_l}) = \suml_{l = 1}^N |P_{k_l}| + \suml_{l = 1}^N \alpha_{k_l} \cdot |P_{k_l}|
		\]
		Если мы положим $\alpha_k := \eps / (|P_k| \cdot 2^k)$, то тогда будет следующая оценка:
		\[
			\left|\bigcup_{l = 1}^N P_{k_l}^{\alpha_{k_l}}\right| \le \row{n = 1}{|P_k|} + \eps = \suml_{i = 1}^\infty |M_i| + \eps
		\]
		Полученное неравенство верно для всех достаточно малых $\eps$.
		
		\item $\ge$. Теперь $M \supset \bigsqcup_{i = 1}^\infty M_i$. Более того
		\[
			\forall N \in \N \quad M \supset \bigsqcup_{i = 1}^N M_i \Ra |M| \ge \suml_{i = 1}^N |M_i|
		\]
		Устремляя $N \to \infty$, получим требуемое.
	\end{enumerate}
\end{proof}