\subsection{Площадь поверхности тела вращения}

\begin{note}
	Что есть площадь поверхности тела? Изначально математики определили это как точная верхняя грань площади вписанного многогранника, но следующий пример показывает, что так делать не стоит.
\end{note}

\textcolor{red}{Сюда бы картиночку с 24 лекции весны 2022, 4:30}

\begin{example} (Сапог Шварца)
	Рассмотрим цилиндр высоты 1 и такого же радиуса основания. Рассечём его по высоте на $n$ равных частей, а окружности каждого из сечений поделим на $m$ равных частей (при этом, если одна окружность уже разлинована, то на второй мы делаем деление на середине отрезка между соседними делениями первой). Если теперь соединить ближайшие точки на соседних, то получится очертание вписанной многогранной поверхности, вписанной в цилиндр
	
	Несложно убедиться, что площадь одного треугольника будет
	\[
		S_\tr = \sin \frac{\pi}{m} \sqrt{\frac{1}{n^2} + \ps{1 - \cos \frac{\pi}{m}}^2}
	\]
	Тогда вся площадь многогранника будет $S = 2mn S_\tr$. Методом пристального взгляда заметим, что если $n, m \to \infty$, при этом $n / m^2 \to p$ --- некоторое число, то
	\[
		S = 2m\sin \frac{\pi}{m} \sqrt{1 + n^2\ps{1 - \cos \frac{\pi}{m}}^2} \to 2\pi \sqrt{1 + \frac{\pi^4 p^2}{4}}
	\]
	То есть значение неоднозначно, определение некорректно.
\end{example}

\begin{definition}
	\textit{Площадью поверхности тела вращения} называется предел площадей поверхности усечённых конусов, полученных вращением ломаных, вписанных во вращаемую кривую, если максимальная из длин проекций звеньев ломаной на ось вращения стремится к нулю.
\end{definition}

\textcolor{red}{Сюда картиночку с 24й лекции весны 2022, 23:55}

\begin{theorem} (Площадь поверхности тела вращения)
	Если $f \in R[a; b]$ --- неотрицательная функция, причём $f' \in C[a; b]$, то тело, полученное вращением её подграфика $G_f$ вокруг оси $ox$, имеет площадь $S$, находимую по следующей формуле:
	\[
		S = 2\pi \int_a^b f(x) \sqrt{1 + (f'(x))^2}dx
	\]
\end{theorem}

\begin{proof}
	Вращение звена ломаной даёт один усечённый конус. Если у нас есть разбиение $P \colon a = x_0 < \ldots < x_n = b$, то положим $S_P$ --- сумма площадей поверхности этих конусов. Тогда
	\[
		S_P = \sum_{k = 1}^n \pi (f(x_{k - 1}) + f(x_k)) \sqrt{(x_{k - 1} - x_k)^2 + (f(x_k) - f(x_{k - 1}))^2}
	\]
	В силу наличия непрерывной производной, мы можем воспользоваться формулой Лагранжа в каждом отрезочке:
	\[
		S_P = \pi \sum_{k = 1}^n (f(x_{k - 1}) + f(x_k)) \sqrt{1 + (f'(\xi_k))^2} \cdot \Delta x_k
	\]
	Всё равно это не очень похоже на интегральную сумму из-за первого множителя в слагаемых. Посмотрим на $S(P, F, \{\xi_k\}_{k = 1}^n)$ для $F(x) = 2\pi f(x) \sqrt{1 + (f'(x))^2}$:
	\[
		S(P, F, \{\xi_k\}_{k = 1}^n) = \sum_{k = 1}^n F(\xi_k) \Delta x_k = 2\pi \sum_{k = 1}^n f(\xi_k) \sqrt{1 + (f'(\xi_k))^2} \Delta x_k \xrightarrow[\Delta(P) \to 0]{} \int_a^b F(x)dx
	\]
	Надо доказать, что $|S_P - S(P, F, \{\xi_k\}_{k = 1}^n)| \to 0$ при $\Delta(P) \to 0$. Давайте просто оценим модуль разности, но перед этим упомянем пару фактов:
	\begin{itemize}
		\item В силу непрерывности $f'$, она ограничена:
		\[
			\exists M > 0 \such \forall x \in [a; b]\ \ |f'(x)| \le M
		\]
		
		\item Так как $f'$ ограничена, то $f$ будет равномерно непрерывна по соответствующему признаку:
		\[
			\forall \eps > 0\ \exists \delta > 0 \such \forall x, t \in [a; b],\ |x - t| < \delta \quad |f(x) - f(t)| < \frac{\eps}{2\pi (b - a) \sqrt{1 + M^2}}
		\]
	\end{itemize}
	Теперь перейдём к модулю. Зафиксируем $\eps > 0$, по нему найдём $\delta > 0$ для равномерной непрерывности и посмотрим на $\forall P \colon \Delta(P) < \delta$:
	\begin{multline*}
		|S_P - S(P, F, \{\xi_k\}_{k = 1}^n)| = \pi \mo{\sum_{k = 1}^n (f(x_{k - 1}) + f(x_k) - 2f(\xi_k))\sqrt{1 + (f'(\xi_k))^2} \Delta x_k} \le
		\\
		\pi \sum_{k = 1}^n |f(x_{k - 1}) + f(x_k) - 2f(\xi_k)| \sqrt{1 + (f'(\xi_k))^2} \Delta x_k \le
		\\
		\pi \sqrt{1 + M^2}\sum_{k = 1}^n (|f(x_{k - 1}) - f(\xi_k)| + |f(x_k) - f(\xi_k)|) \Delta x_k < \eps
	\end{multline*}
\end{proof}

\section{Неявные функции}

\begin{example}
	Для примера простейшей неявной функции посмотрим на уравнение окружности $x^2 + y^2 = 1$
	
	\textcolor{red}{Сюда бы картиночку с 24 лекции весны 2022, 34:50}
	
	Понятно, что сама окружность не задаёт функции в силу многозначности для одного $x$. Однако, если мы возьмём окрестной у какой-то точки окружности, то мы можем явно выделить функцию. Например, для любой точки в верхней полуокружности это будет $y = \sqrt{1 - x^2}$, а для любой точки из нижней --- $y = -\sqrt{1 - x^2}$. При этом, для точек $(\pm 1, 0)$ мы ничего задать не сможем, там в любой окрестности будет многозначность.
\end{example}

\begin{theorem} (Теорема о неявной функции, заданной одним уравнением)
	Пусть $F(\vv{x}, y) = F(x_1, \ldots, x_n, y)$ - дифференцируемая в окрестности точки $(\vv{x}_0, y_0) = (x_{1, 0}, \ldots, x_{n, 0}, y_0)$ и выполнены следующие условия:
	\begin{itemize}
		\item $\pd{F}{y}$ непрерывна в той же окрестности точки $(\vv{x}_0, y_0)$
		
		\item $F(\vv{x}_0, y_0) = 0$
		
		\item $\pd{F}{y}(\vv{x}_0, y_0) \neq 0$
	\end{itemize}
	Тогда для достаточно малого $\eps > 0$ существует $\delta > 0$ такое, что $\forall \vv{x} \in K_{\delta, \vv{x}_0} = \prod_{k = 1}^n (x_{k, 0} - \delta; x_{k, 0} + \delta)$ существует единственная функция $y = \phi(\vv{x})$ такая, что
	\[
		\forall (\vv{x}, y) \in K_{\delta, \vv{x}_0} \times (y_0 - \eps; y_0 + \eps) \quad F(\vv{x}, y) = 0 \lra y = \phi(\vv{x})
	\]
	Более того, $\phi(\vv{x})$ дифференцируема в точке $\vv{x}_0$
\end{theorem}

\begin{proof}
	Для определенности будем считать $\pd{F}{y} (\vv{x}_0, y_0) > 0$. В силу непрерывности производной, существует окрестность точки $(\vv{x}_0, y_0)$, в которой $\pd{F}{y}(\vv{x}, y) > 0$.
	\textcolor{red}{Сюда бы картиночки с 24 лекции весны 2022, 55:50}
	
	Пристально вглядимся в значения $F(\vv{x}_0, y)$ при <<бегающем>> $y$ соответственно. В силу положительности производной
	\[
		\exists \eps_0 > 0 \such \forall 0 < \eps < \eps_0 \quad F(\vv{x}_0, y_0 - \eps) < 0 \wedge F(\vv{x}_0, y_0 + \eps) > 0
	\]
	Так как $F$ дифференцируема, то она же и непрерывна. Значит, в каждой <<гиперплоскости>> (если зафиксировать только $y$ и двигаться от точки $(\vv{x}_0, y)$) будет окрестность с положительными значениями $F$. Благодаря этому $\exists K_{\delta, \vv{x}_0}$ --- кубик внутри окрестности с положительной частной производной и у которого одна половина точек имеет положительное значение на $F$, а другая отрицательное. Дальше уже идея понятна: выбираем любой $\vv{x}$, у которого есть точки внутри куба, и тогда $\exists ! y$, что $F(\vv{x}, y) = 0$ (за счёт строгой монотонности по $y$ и разным знакам значений на концах).
	\[
		\exists \delta > 0 \such \forall \vv{x} \in K_{\delta, \vv{x}_0}\ \exists ! y = \phi(\vv{x}) \quad F(\vv{x}, \phi(\vv{x})) = 0
	\]
	По построению вышло так, что $\phi(\vv{x})$ уже непрерывна в точке $\vv{x}_0$. Действительно, мы строили $\phi$ через $\eps_0$, который можно делать бесконечно близким нулю.
	
	В силу дифференцируемости $F$ в точке $(\vv{x}_0, y_0)$, мы можем значение в любой точке из окрестности представить следующим образом:
	\begin{multline*}
		F(\vv{x}, y) = \sum_{k = 1}^n \pd{F}{x_k}(\vv{x}_0, y_0) (x_k - x_{k, 0}) + \pd{F}{y}(\vv{x}_0, y_0) (y - y_0) + \alpha(\vv{x}, y);
		\\
		\alpha(\vv{x}, y) = o(|(\vv{x}, y) - (\vv{x}_0, y_0)|),\ (\vv{x}, y) \to (\vv{x}_0, y_0)
	\end{multline*}
	Распишем $\alpha(\vv{x}, y)$ чуть более подробно:
	\[
		\alpha(\vv{x}, y) = \sum_{j = 1}^n \alpha(\vv{x}, y)\frac{(x_j - x_{j, 0})^2}{|(\vv{x}, y) - (\vv{x}_0, y_0)|^2} + \alpha(\vv{x}, y)\frac{(y - y_0)^2}{|(\vv{x}, y) - (\vv{x}_0, y_0)|^2}
	\]
	Введём такие функции:
	\[
		\forall j \in \range{n}\ \alpha_j(\vv{x}, y) = \frac{\alpha(\vv{x}, y)(x_j - x_{j, 0})}{|(\vv{x}, y) - (\vv{x}_0, y_0)|^2};\ \beta(\vv{x}, y) = \frac{\alpha(\vv{x}, y)(y - y_0)}{|(\vv{x}, y) - (\vv{x}_0, y_0)|^2}
	\]
	Все эти функции стремятся к нулю при $(\vv{x}, y) \to (\vv{x}_0, y_0)$. В самом деле, разность координат, поделенная на модуль, будет ограничена единицей сверху, а оставшиеся множители связаны свойством $\alpha(\vv{x}, y)$. Преобразуем выражение для $F$, исходя из разложения $\alpha$:
	\[
		F(\vv{x}, y) = \sum_{k = 1}^n \ps{\pd{F}{x_k} (\vv{x}_0, y_0) + \alpha_k(\vv{x}, y)}(x_k - x_{k, 0}) + \ps{\pd{F}{y}(\vv{x}_0, y_0) + \beta(\vv{x}, y)}(y - y_0)
	\]
	Если взять точку в $K_{\delta, \vv{x}_0}$ и $y = \phi(\vv{x})$, то будет выполнено зануление $F(\vv{x}, y)$. Отсюда мы достанем приращение $\phi$:
	\[
		0 = \sum_{k = 1}^n \ps{\pd{F}{x_k}(\vv{x}_0, y_0) + \hat{\alpha}_k(\vv{x})}(x_k - x_{k, 0}) + \ps{\pd{F}{y}(\vv{x}_0, y_0) + \hat{\beta}(\vv{x})}(\phi(\vv{x}) - \phi(\vv{x}_0))
	\]
	\[
		\phi(\vv{x}) - \phi(\vv{x}_0) = \sum_{k = 1}^n \ps{-\frac{\pd{F}{x_k}(\vv{x}_0, y_0)}{\pd{F}{y}(\vv{x}_0, y_0)} + \gamma_k(\vv{x})}(x_k - x_{k, 0})
	\]
	где $\gamma_k(\vv{x})$ берётся из равенства:
	\[
		-\frac{\pd{F}{x_k}(\vv{x}_0, y_0)}{\pd{F}{y}(\vv{x}_0, y_0)} + \gamma_k(\vv{x}) = -\frac{\pd{F}{x_k}(\vv{x}_0, y_0) + \hat{\alpha}_k(\vv{x})}{\pd{F}{y}(\vv{x}_0, y_0) + \hat{\beta}(\vv{x})}
	\]
	Отсюда, если $\vv{x} \to \vv{x}_0$, то $\gamma_k(\vv{x}) \to 0$. Стало быть
	\[
		\phi(\vv{x}) - \phi(\vv{x}_0) = \sum_{k = 1}^n A_k + \gamma(\vv{x}),\ \gamma(\vv{x}) = o(|\vv{x} - \vv{x}_0|),\ \vv{x} \to \vv{x}_0
	\]
	где $\gamma(\vv{x}) = \sum_{k = 1}^n \gamma_k(\vv{x})(x_k - x_{k, 0})$
\end{proof}

\begin{note}
	Факт дифференцируемости даёт метод нахождения частной производной $\pd{y}{x_k}$:
	\[
		F(x_1, \ldots, x_n, y) = 0 \Ra \pd{F}{x_k} + \pd{F}{y} \cdot \pd{y}{x_k} = 0
	\]
\end{note}