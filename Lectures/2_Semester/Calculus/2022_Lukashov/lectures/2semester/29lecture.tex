\begin{definition}
    Билинейное отображение $B\colon E_1 \times E_1 \to E_2$, где  $E_1, E_2$ - линейные нормированные пространства, называется \textit{ограниченным}, если 
    \[
        \exists M \in \R \such \forall x, y \in E_1 \quad \|B(x, y)\|_{E_2} \leq M \|x\|_{E_1} \cdot \|y\|_{E_1}
    \]
\end{definition}

\begin{definition}
    Наименьшая из констант $M$ называется \textit{нормой отображения} $B$, обозначается как $\|B\|$.
\end{definition}

\begin{note}
	Совокупность билинейный ограниченных отображений $E_1 \times E_1 \to E_2$ c введённой нормой образует линейно нормированное пространство $\Lin(E_1^2, E_2)$.
\end{note}

\begin{proposition}
    Линейные нормированные пространства $\Lin(E_1^2, E_2)$ и $\Lin(E_1, \Lin(E_1, E_2))$ изоморфны и \textit{изометричны} (то отображение, которое осуществляет изоморфизм, сохраняет норму).
\end{proposition}

\begin{proof}
    Пусть $A \in \Lin(E_1, \Lin(E_1, E_2))$. Покажем способ восстановления билинейной формы $B$ по $A$:
    \[
        B(x, y)=(Ax)y
    \]
    Линейность очевидна. Проверим, что имеет место ограниченность:
    \[
        \|B(x, y)\|_{E_2}=\|(Ax)y\|_{E_2} \leq \|Ax\|_{\Lin(E_1, E_2)} \cdot \|y\|_{E_1} \leq \|A\|_{\Lin(E_1, \Lin(E_1, E_2))} \cdot \|x\|_{E_1} \cdot \|y\|_{E_1}
    \]
    Отсюда следует, что $B$ --- билинейное ограниченное отображение и даже $\|B\| \leq \|A\|$.
    
    Докажем утверждение в другую сторону. Пусть у нас есть билинейное ограниченное отображение $B$. Как только мы в нем зафиксируем $x$, то получим некоторый линейный оператор $A_x$, который будет также линеен. Посмотрим на норму $A$ как на норму линейного ограниченного оператора:
    \begin{multline*}
        \|A\|_{\Lin(E_1, \Lin(E_1, E_2))} = \sup_{x \in E_1,\ \|x\|_{E_1} \leq 1} \|A x\|_{\Lin(E_1, E_2)} =
        \\
        \sup_{x \in E_1,\ \|x\|_{E_1} \leq 1} \ps{\sup_{y \in E_1,\ \|y\|_{E_1} \leq 1} \|(Ax)y \|_{E_2}} \leq \|B\|_{\Lin(E_1^2, E_2)}
    \end{multline*}
    Из вышедоказанных неравенств следует, что $\|B(x, y)\| = \|(Ax)y\|$. Значит, существует изоморфизм с изометрией.
\end{proof}

\begin{example}
    Рассмотрим $F \colon E \to R,\ F(x)=\|x\|_E^2$, где $E$ --- евклидово пространство. Продифференцируем данное отображение (через скалярное произведение в евклидовом пространстве). Для начала выпишем явно приращение функции:
    \[
        F(x + h) - F(x) = (x + h, x + h) - (x, x) = 2(x, h) + (h, h)
    \]
    Теперь нужно найти такой линейный неограниченный оператор, чтобы выполнялось условие Фреше:
    \[
        \|F(x + h) - F(x) - F'(x)h\|_R = o(\|h\|_{E_1}),\ h \to 0
    \]
    Видно, что нужно взять $F'(x)h := (2x, h)$. Тогда
    \[
        \|F(x + h) - F(x) - F'(x)h\|_R = \|(h, h)\|_R = \|h\|^2_{E_1} = o(\|h\|_{E_1}), h \to 0
    \]
    Также, очевидно, что $F'(x)h = (2x, h)$ --- сильный дифференциал, а в качестве производной мы рассматриваем $F'(x)=(2x, \cdot)$ --- линейный ограниченный оператор, который состоит в скалярном умножении на $2x$.
\end{example}

\begin{example}
    Рассмотрим функцию $\vv{F} \colon D \to \R^m$, $D \subset \R^n$ и все её координатные отображения $F_j$. Имеет место эквивалентность:
    \begin{multline*}
    	\forall j \in \range{m} \Big(F_j \text{ --- дифференцируема в точке } \vv{x}_0 \Lra
    	\\
    	F_j(\vv{x}_0 + \vv{h}) - F_j(\vv{h}) - (\grad F_j(\vv{x}_0), \vv{h}) = o(\|\vv{h}\|_{\R^n}),\ \vv{h} \to \vv{0}\Big)
    \end{multline*}
    Рассмотрим это же выражение в векторном виде:
    \[
        \|\vv{F}(\vv{x}_0 + \vv{h}) - \vv{F}(\vv{h}) - \vv{F}'(\vv{x}_0)\vv{h}\|_{R^m} = o(\|h\|_{R^n}),\ \vv{h} \to \vv{0}
    \]
\end{example}