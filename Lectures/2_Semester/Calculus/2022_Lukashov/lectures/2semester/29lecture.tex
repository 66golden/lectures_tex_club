\begin{definition}
    Билинейное отображение $B\colon E_1 \times E_1 \to E_2$, где  $E_1, E_2$ - линейные нормированные пространства, называется ограниченным, если 
    \[
        \exists M\quad \forall x, y \in E_1 \quad \|B(x, y)\|_{E_2} \leq M \|x\|_{E_1} \cdot \|y\|_{E_1}
    \]
\end{definition}

\begin{definition}
    Наименьшее из констант $M$ называется \textit{нормой} $B$, обозначается $\|B\|$. Совокупность билинейных ограниченных отображений из $E_1 \times E_1 \to E_2$ с введенной нормой образует линейное нормированное пространство $\mathcal{L} (E_1^2 \to E_2)$
\end{definition}

\begin{proposition}
    Линейные нормированные пространства $\mathcal{L} (E_1^2 \to E_2)$ и $\mathcal{L} (E_1 \to \mathcal{L}(E_1 \to E_2))$ изоморфны и изометричны (то отображение, которое осуществляет изоморфизм, сохраняет норму).
\end{proposition}
\begin{proof}
    Пусть $A \in \mathcal{L} (E_1 \to \mathcal{L}(E_1 \to E_2))$\\
    Способ восстановления формы, если дано отображение $A\colon$
    \[
        B(x, y)=(A x)y
    \]
    Теперь рассмотрим ограниченность билинейной формы$B\colon$
    \[
        \|B(x, y)\|_{E_2}=\|(A x)_{y}\|_{E_2} \leq \|A x\|_{\mathcal{L}(E_1 \to E_2)} \cdot \|y\|_{E_1} \leq \|A\|_{\mathcal{L} (E_1 \to \mathcal{L}(E_1 \to E_2))} \cdot \|x\|_{E_1} \cdot \|y\|_{E_1}
    \]
    Отсюда следует, что $B$ - билинейное ограниченное отображение и $\|B\| \leq \|A\|$.
    Докажем утверждение в другую сторону. Предположим, что у нас есть билинейное ограниченное отображение $B$. Как только мы в нем зафиксируем $x$, то получим некоторый линейный оператор $A_x$, а теперь заметим, что для любого $y$ $A_x$-линейный оператор.\\
    Посмотрим на норму $A$ как на норму линейного ограниченного оператора$\colon$
    \[
        \|A\|_{\mathcal{L} (E_1 \to \mathcal{L}(E_1 \to E_2))} = \sup_{x \in E_1, \|x\|_{E_1} \leq 1} \|A x\|_{\mathcal{L} (E_1) \to E_2} = \sup_{x \in E_1, \|x\|_{E_1} \leq 1} \sup_{y \in E_1, \|y\|_{E_1} \leq 1} \|(A x ) (y) \|_{E_2} \leq \|B\|_{\mathcal{L} (E_1^2 \to E_2)}
    \]
    Из вышедоказанных неравенств следует$\colon$
    \[
         \|B(x, y)\|=\|(A x)y\|
    \]
    А значит, существует изометрия.
\end{proof}

\begin{example}
    $F \colon E \to R, \quad F(x)=\|x\|^2_E$, $E$-евклидово пространстов.\\ Продифференцируем данное отображение (через скалярное произведение в евклидовом пространстве)$\colon$
    \[
        F(x+h)-F(x)=(x+h, x+h)-(x,x)=2(x, h)+(h, h)
    \]
    Теперь нужно найти такой линейный неограниченный оператор, чтобы выполнялось$\colon$
    \[
        \|F(x+h)-F(x)-F^\prime(x) h\|_R=o(\|h\|_{E_1}), h \to 0
    \]
    Видно, что$\colon$
    \[
        F^\prime(x) h = (2x, h)
    \]
    Тогда
    \[
        \|F(x+h)-F(x)-F^\prime(x) h\|_R = \|(h, h)\|_R = \|h\|^2_{E_1} = o(\|h\|_{E_1}), h \to 0
    \]
    Также, очевидно, что $F^\prime(x) h = (2x, h)$-сильный дифференциал, а в качестве производной мы рассматриваем $F^\prime(x)=(2x, \cdot)$-линейный ограниченный оператор, который состоит в скалярном умножении на $2x$
\end{example}

\begin{example}
    Рассмотрим функцию $\vv{F} \colon D \to \R^m$, $D \subset \R^n$ и её координатное отображение $F_j(x_1, \ldots, x_n)$, $j \in \range{m}$. Если все эти отображения дифференцируемы в точке $\vv{x}^0 \Lra F_j(\vv{x}^0 + \vv{h}) - F_j (\vv{h}) - (\grad F_j (\vv{x}^0), \vv{h}) = o(\|h\|_{R^n}), h \to 0$, $j \in \range{m}$
    
    Рассмотрим это же выражение в векторном виде:
    \[
        \|\vv{F}(\vv{x}^0+\vv{h})-\vv{F} (\vv{h}) - \vv{F}^\prime(\vv{x}^0)\vv{h}\|_{R^m} = o(\|h\|_{R^n}), h \to 0
    \]
\end{example}