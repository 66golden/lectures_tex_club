\textcolor{red}{Сюда бы картиночку с конца 21й лекции весны 2022}

\begin{proof}
	Разобьём полуинтервал $[0; +\infty)$ на малые полуинтервалы:
	\[
		\forall m \in \N\ \ [0; +\infty) = \bscup_{l = 0}^\infty \left[\frac{l}{2^m}; \frac{l + 1}{2^m}\right)
	\]
	Рассмотрим множества $E_{l, m} = \{x \in E \colon f(x) \in [l / 2^m; (l + 1) / 2^m)\}$. Они измеримы, ибо их можно представить в таком виде:
	\[
		E_{l, m} = E_{\frac{l + 1}{2^m}}(f) \bs E_{\frac{l}{2^m}}(f)
	\]
	Более того, область определения исходной функции выражается следующим образом:
	\[
		\forall m \in \N \quad E = \bscup_{l = 0}^\infty E_{l, m}
	\]
	Теперь, определим $h_m(x)$ так:
	\[
		\forall l, m\ \left(\forall x \in E_{l, m} \Ra h_m(x) = \frac{l}{2^m}\right)
	\]
	Полученная функция имеет не более чем счётное число значений и она измерима (например, потому что записывается в виде ступенчатой функции). Следовательно, она ступенчатая. При этом
	\[
		\forall m \in \N\ \forall x \in E \quad 0 \le f(x) - h_m(x) \le \frac{1}{2^m}
	\]
	Значит, последовательность равномерно сходится к $f(x)$. Остаётся показать, что $\forall x \in E$ она будет неубывающей. Заметим следующий факт:
	\[
		\left[\frac{l}{2^m}; \frac{l + 1}{2^m}\right) = \left[\frac{2l}{2^{m + 1}}; \frac{2l + 1}{2^{m + 1}}\right) \sqcup \left[\frac{2l + 1}{2^{m + 1}}; \frac{2l + 2}{2^{m + 1}}\right)
	\]
	Отсюда $\forall x \in E_{l, m}$ верно, что $h_m(x) \le h_{m + 1}(x)$.
\end{proof}

\begin{definition}
	Последовательность измеримых функций $\{f_m\}_{m = 1}^\infty$ сходится к измеримой функции $f$ на $E \subset \R^n$ по мере, если
	\[
		\forall \eps > 0 \quad \lim_{m \to \infty} \mu(\{x \in E \colon |f_m - f(x) \ge \eps|\}) = 0
	\]
\end{definition}