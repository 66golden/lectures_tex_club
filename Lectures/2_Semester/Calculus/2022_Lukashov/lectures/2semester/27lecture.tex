\begin{theorem} (Достаточное условие условного экстремума)
    Пусть функции $f_1, \phi_1, \dots, \phi_m$, $m < n$, дважды непрерывно дифференцируемы в окрестности точки $\vv{x}_0$, причём верны условия:
    \begin{enumerate}
    	\item Ранг следующей матрицы максимален:
    	\[
    		\rk \begin{pmatrix}
	    		\frac{\vdelta \phi_1}{\vdelta x_1}(\vv{x}_0) & \dots & \frac{\vdelta \phi_1}{\vdelta x_n}(\vv{x}_0)
	    		\\
	    		\vdots & \ddots & \vdots
	    		\\
	    		\frac{\vdelta \phi_m}{\vdelta x_1}(\vv{x}_0) & \dots & \frac{\vdelta \phi_m}{\vdelta x_n}(\vv{x}_0)
	    		\\
    		\end{pmatrix} = m
    	\]
    	
    	\item Точка $(\vv{x}_0, \vv{\lambda}_0)$ является стационарной точкой функции Лагранжа $L(\vv{x}, \vv{\lambda})$
    	
    	\item $d^2_{xx}L(\vv{x}_0, \vv{\lambda}_0)$ является положительно (отрицательно) определенной квадратичной формой переменных $dx_1, \ldots, dx_n$, продифференцированным уравнением связи $\forall j \in \range{m}\ d\phi_j(\vv{x}_0)$
    	
    	\item Имеются связи $\forall j \in \range{m}\ \phi_j(\vv{x}_0) = 0$
    \end{enumerate}
	Тогда $\vv{x}_0$ является точкой строгого условного минимума (максимума) функции $f$. Если же $d^2_{xx} L(\vv{x}_0, \vv{\lambda}_0)$ является неопределенной квадратичной формой (от тех же переменных), то $\vv{x}_0$ не является точкой условного экстремума той же функции.
\end{theorem}

\textcolor{red}{Далее и до начала главы о дифференциальном исчислении в ЛНП не было проделано правки основным автором.}

\begin{proof}
    Без ограничения общности считаем главный минор $\frac{\vdelta (\phi_1, \dots, \phi_m)}{\vdelta (x_1, \dots, x_m)} \neq 0$. Как было отмечено перед теоремой 1 задача на условный экстремум $\Leftrightarrow$ задаче на локальный (безусловный) экстремум функции 
    \[
        F (x_{m+1}, \dots, x_n) = f(\psi(x_{m+1}, \dots, x_n), \dots, \psi_m(x_{m+1}, \dots, x_n), x_{m+1}, \dots, x_n)
    \]
    Достаточное условие локального экстремума $F\colon d^2F(\vec{\tilde{x^{(0)}}})$ - положительно (отрицательно) определенная форма.\\
    \[
        d^2F = d(d F) = d \left ( \begin{aligned} \sum^n_{j=1} \frac{\vdelta f}{\vdelta x_j} d x_j \end{aligned}  \right ) = d \left ( \begin{aligned} \sum^n_{j = 1} \frac{\vdelta L}{\vdelta x_j} d x_j \end{aligned}  \right ) = \sum^n_{j=1} \sum^n_{k=1} \frac{\vdelta^2 L}{\vdelta x_j \vdelta x_k} d x_j d x_k + \sum^m_{j=1} \frac{\vdelta L}{\vdelta x_j} d^2 x_j.
    \]
    Воспользуемся тем, что точки $(\vv{x}^{(0)}, \vec{\lambda^{(0)}})$ - стационарные, значит $\frac{\vdelta L}{\vdelta x_j} = 0$. Мы получили в точности форму второго дифференциала. Причем переменные $x_j, x_k$ не являются независимыми, а являются связанными условиями дифференцирования. \\
    Аналогичное доказательство производится для неопределенной квадратичной формы.
\end{proof}

\begin{example} (Кудрявцев - 3 часть, $\S 5, 33(2)$) \\
    Найти максимум и минимум функции $u = x^2 + 2y^2+3z^2$ при ограничениях $x^2+y^2+z^2=1, x+2y+3z=0.$
\end{example}

\begin{solution}
    Заметим, что речь идет об абсолютном экстремуме (в условии нет слова локальный). 
    В данном случае ограничения на $x, y, z$ задают сферу, которая пересекается с плоскостью, проходящей через начало координат. Значит, мы имеем дело с окружностью. \\
    Решение с помощью функции Лагранжа:
    \[ 
        L(x, y, z, \lambda, \mu) = x^2+2y^2+3z^3+\lambda(x^2+y^2+z^2-1)+\mu(x+2y+3z)
    \]
    Найдем стационарные точки функции Лагранжа$\colon$
    \[
        \frac{\vdelta L}{\vdelta x} = 2x+2\lambda x + \mu =0
    \]
    \[
        \frac{\vdelta L}{\vdelta y} = 4y+2\lambda y + 2\mu=0\\
    \]
    \[
        \frac{\vdelta L}{\vdelta z} = 6z+2\lambda z + 3\mu=0\\
    \] 
    Мы получили систему из 5 уравнений с 5 неизвестными. Для ее решения воспользуемся приемом:
    \[
        (2x+2\lambda x + \mu) \cdot x =0 \cdot x\\
    \]
    \[
        (4y+2\lambda y + 2\mu) \cdot y=0 \cdot y\\
    \]
    \[
        (6z+2\lambda z + 3\mu) \cdot z=0 \cdot z\\
    \] 
    Сложим все уравнения системы:
    \[
        (2x+2\lambda x + \mu) \cdot x + \\
    \]
    \[
        + (4y+2\lambda y + 2\mu) \cdot y + \\
    \]
    \[
        + (6z+2\lambda z + 3\mu) \cdot z=\\
    \]
    \[
        = 2u + 2\lambda + 0\mu = 2u + 2\lambda = 0\\
    \]
    Из этого уравнения можно сделать следующий вывод: максимальное (минимальное) значение $u$ зависит от минимального (максимального) значения $\lambda$. Т.е. нам не нужно решать задачу в полном виде (ведь требуется найти минимум и максимум функции, а не какую-то конкретную точку).\\
    Еще раз сложим все уравнения системы (в этот раз без домножения):
    \[
        (2x+2\lambda x + \mu) + \\
        + (4y+2\lambda y + 2\mu) + \\
        + (6z+2\lambda z + 3\mu)=\\
        = 0 + 2\lambda (x+y+z) +6\mu=0\\
    \]
    \[
        \mu = -\frac{\lambda}{3} (x+y+z)
    \]
     Подставим найденное $\mu$ в исходную систему:
     \[
        x(2+\frac{5}{3}\lambda)-\frac{\lambda}{3}y-\frac{\lambda}{3}z=0\\
    \]
    \[
        -\frac{2}{3}\lambda x + y(4+\frac{4}{3}\lambda)-\frac{2}{3}\lambda z=0\\
    \]
    \[
        -\lambda x - \lambda y + z(6+\lambda)=0\\
     \]
     Мы получили неоднородную систему уравнений. Тривиальное решение $x=0, y=0, z=0$ не является ее решением из-за ограничения в условии: $x^2+y^2+z^2=1$. Поэтому для того, чтобы решение существовало, определитель матрицы этой системы однородных уравнений должен быть равен $0$.
     \[
        \begin{vmatrix}
            2+\frac{5}{3}\lambda & -\frac{\lambda}{3} & -\frac{\lambda}{3}\\
            -\frac{2}{3}\lambda & 4+\frac{4}{3}\lambda & -\frac{2}{3}\lambda\\
            -\lambda & -\lambda & 6+\lambda
        \end{vmatrix} = 0\\
    \]
    \[
        det (A - \lambda B)=0 \\
     \]
     Посчитаем определитель, раскроем скобки и приведем подобные слагаемые:
     \[
        \frac{56}{3}\lambda^2 + 64\lambda+48=0\\
    \]
    \[
        \frac{7}{3}\lambda^2+8\lambda+6=0\\
    \]
    \[
        \lambda_1 = \frac{-12+3\sqrt{2}}{7}\\
    \]
    \[
        \lambda_2 = \frac{-12-3\sqrt{2}}{7}
     \]
    Найденные значения $\lambda$ согласно вышеприведенным рассуждениям соответсвуют минимуму и максимуму исходной функции.
\end{solution}


\section{Дифференциальные исчисления в линейных нормированных пространствах}

\begin{note}
	Если $A \colon E_1 \to E_2$ --- линейные оператор между линейно нормированными пространствами, то принято опускать скобки у аргумента:
	\[
		\forall x \in E_1\ \ Ax := A(x)
	\]
\end{note}

\begin{note}
	Далее, если не оговорено обратного, мы считаем $E_1$ и $E_2$ всегда линейно нормированными пространствами.
\end{note}

\begin{definition}
    Линейный оператор $A \colon E_1 \to E_2$ называется \textit{ограниченным}, если 
    \[
        \exists M \in R \such \forall x \in E_1\ \ \|Ax\|_{E_2} \leq M
    \]
\end{definition}

\begin{proposition}
	Пусть $A \colon E_1 \to E_2$ --- линейный оператор между ЛНП. Тогда он ограничен $\Lra$ он непрерывен (в любой точке $E_1$).
\end{proposition}

\begin{proof}~
    \begin{itemize}
        \item Пусть $A$ --- ограниченный линейный оператор: $E_1 \to E_2$. Из ограниченности есть такая оценка:
        \[
            \exists M > 0 \such \|Ax_1 - Ax_2\|_{E_2} = \|A(x_1 - x_2)\|_{E_2} \leq M\|x_1 - x_2\|_{E_1}
        \]
        Отсюда сразу следует непрерывность:
        \[
            \forall \eps > 0\ \exists \delta := \frac{\eps}{M} > 0 \such \forall x_1, x_2 \in E_1,\  \|x_1 - x_2\|_{E_1} < \delta \quad \|Ax_1 - Ax_2\|_{E_2} < \eps
        \]
        
        \item Непрерывность во всех точках означает как минимум непрерывность в нуле:
        \[
            \forall \eps > 0\ \exists \delta > 0 \such \forall h \in E_1, \|h\|_{E_1} < \delta \quad \|Ah\|_{E_2} < \eps
        \] 
        Зафиксируем $\eps = 1$ и, следовательно, $\delta > 0$. Тогда, для любого ненулевого $x \in E_1$ можно выбрать $h = (\delta / 2) \cdot x / \|x\|_{E_1}$, удовлетворяющий условию непрерывности в нуле. Подставим $h$:
        \[
        	\|Ah\|_{E_2} = \nr{A \frac{\delta x}{2\|x\|_{E_1}}}_{E_2} = \nr{\frac{\delta}{2\|x\|_{E_1}}Ax}_{E_2} < \eps
        \]
        Так как $A$ --- линейный оператор, то положительную константу можно вытащить за знак нормы:
        \[
              \nr{\frac{\delta}{2\|x\|_{E_1}}Ax}_{E_2} = \frac{\delta}{2\|x\|_{E_1}}\|Ax\|_{E_2} < \eps = 1 \Lora \|Ax\|_{E_2} < \frac{2}{\delta}\|x\|_{E_1}
        \]
        Следовательно, искомая константа $M = 2 / \delta$.
    \end{itemize}
\end{proof}

\begin{definition}
    Пусть $A \colon E_1 \to E_2$ --- линейный ограниченный оператор между ЛНП. \textit{Нормой оператора} $A$, обозначаемой как $\|A\|$, называется следующая величина:
    \[
    	\|A\| := \inf \{M \in \R \colon \forall x \in E_1\ \|Ax\|_{E_2} \le M\|x\|_{E_1}\}
    \]
\end{definition}

\begin{proposition}
    Если $A \colon E_1 \to E_2$ --- линейный ограниченный оператор, то имеет место формула для нормы:
    \[
        \|A\| = \sup_{x \in E_1,\ \|x\| = 1} \|Ax\|_{E_2} = \sup_{x \in E,\ x \neq 0} \frac{\|Ax\|_{E_2}}{\|x\|_{E_1}} 
    \]
\end{proposition}

\begin{proof}
	\begin{enumerate}
		\item Установим равенство нормы с последним супремумом. Для этого поймём, почему оно вообще может быть верным. Если в определении нормы $x \neq 0$, то
		\[
			\|Ax\|_{E_2} \le M\|x\|_{E_1} \Lra \frac{\|Ax\|_{E_2}}{\|x\|_{E_1}} \le M
		\]
		Стало быть, имеет место общее неравенство
		\[
			\sup_{x \in E_1,\ x \neq 0} \frac{\|Ax\|_{E_2}}{\|x\|_{E_1}} \le \|A\|
		\]
		Предположим противное: неравенство выполнилось строго. Тогда $\exists M$ такое, что оно лежит в интервале между супремумом и нормой:
		\[
			\exists M \in \R \such M \in \ps{\sup_{x \in E_1,\ x \neq 0} \frac{\|Ax\|_{E_2}}{\|x\|_{E_1}}; \|A\|}
		\]
		Это сразу же приводит к противоречию, ибо из факта $M < \|A\|$ следует, что для всех $x \neq 0$ выполнено неравенство из определения, но и для нуля оно верно автоматически (ибо $A(0) = 0$).
		
		\item Установим равенство супремумов. Действительно:
		\[
			\forall x \in E,\ x \neq 0\ \exists h := \frac{x}{\|x\|},\ \|h\| = 1
		\]
		Более того, имеет место такое соотношение:
		\[
			\frac{\|Ax\|_{E_2}}{\|x\|_{E_1}} = \frac{\|x\|_{E_1} \cdot \|Ah\|_{E_2}}{\|x\|_{E_1}}
		\]
		Иначе говоря, каждому элементу множества $\|Ax\|_{E_2} / \|x\|_{E_1}$ соответствует какой-то элемент $\|Ah\|_{E_2}$ и наоборот. Значит, супремумы этих множеств будут совпадать.
	\end{enumerate}
\end{proof}

\begin{proposition}
	Если $\Lin(E_1, E_2)$ --- множество линейных ограниченных операторов из $E_1$ в $E_2$, то оно является линейно нормированным пространством с нормой $\|A\|$.
\end{proposition}

\begin{proof}
	Проверим 3 аксиомы нормы:
	\begin{enumerate}
		\item Если $\|A\| = 0$, то $\forall x \in E_1\ \|Ax\|_{E_2} = 0$. Следовательно, $A = 0$ --- нулевой оператор.
		
		\item Определим операцию умножения оператора $A$ на константу $\alpha \in \R$ поточечно:
		\[
			\forall x \in E_1 \quad (\alpha A)(x) := \alpha \cdot A(x)
		\]
		В силу доказанной формулы нормы оператора, вынесение константы за знак нормы оператора работает как надо.
		
		\item Проверим, что выполняется неравенство треугольника:
		\begin{multline*}
			\|A_1 + A_2\| = \sup_{x \in E_1,\ \|x\|_{E_1} = 1} \|(A_1 + A_2)x\|_{E_2} \leq
			\\
			\sup_{x \in E_1,\ \|x\|_{E_1} = 1} \|A_1x\|_{E_2} + \sup_{x \in E_1,\ \|x\|_{E_1} = 1} \|A_2x\|_{E_2} = \|A_1\| + \|A_2\|
		\end{multline*}
	\end{enumerate}
\end{proof}