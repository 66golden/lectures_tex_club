\begin{theorem} (Достаточное условие условного экстремума)
    Пусть функции $f_1, \phi_1, \dots, \phi_m$, $m < n$, дважды непрерывно дифференцируемы в окрестности точки $\vv{x}_0$, причём верны условия:
    \begin{enumerate}
    	\item Ранг матрицы \[
    		\begin{pmatrix}
	    		\frac{\vdelta \phi_1}{\vdelta x_1}(\vv{x}^{(0)}) & \dots & \frac{\vdelta \phi_1}{\vdelta x_n}(\vv{x}^{(0)})
	    		\\
	    		\vdots & \ddots & \vdots
	    		\\
	    		\frac{\vdelta \phi_m}{\vdelta x_1}(\vv{x}^{(0)}) & \dots & \frac{\vdelta \phi_m}{\vdelta x_n}(\vv{x}^{(0)})
	    		\\
    		\end{pmatrix}
    	\]
    	максимален
    	
    	\item 
    \end{enumerate}
	и $(\vv{x}^{(0)}, \vec{\lambda^{(0)}})$ - стационарная точка функции Лагранжа $L(\vec{x}, \vec{\lambda}).$ Если \\
    $d^2_{xx} L(\vv{x}^{(0)}, \vec{\lambda^{(0)}})$ является положительно (отрицательно) определенной квадратичной формой переменных $dx_1, \dots, dx_n$, продифференцированным уравнением связи $d\phi_j (\vv{x}^{(0)}) = 0, j = 1, \dots, m,$ то $\vv{x}^{(0)}$ является точкой строгого условного минимума (максимума) функции $f$ при наличии связей $\phi_j(\vv{x}^{(0)}) = 0, j = 1, \dots, m.$ Если же $d^2_{xx} L(\vv{x}^{(0)}, \vec{\lambda^{(0)}})$ является неопределенное квадратичной формой (от тех же переменных), то $\vv{x}^{(0)}$ не является точкой условного экстремума той же функции.
\end{theorem}

\begin{proof}
    Без ограничения общности считаем главный минор $\frac{\vdelta (\phi_1, \dots, \phi_m)}{\vdelta (x_1, \dots, x_m)} \neq 0$. Как было отмечено перед теоремой 1 задача на условный экстремум $\Leftrightarrow$ задаче на локальный (безусловный) экстремум функции 
    \[
        F (x_{m+1}, \dots, x_n) = f(\psi(x_{m+1}, \dots, x_n), \dots, \psi_m(x_{m+1}, \dots, x_n), x_{m+1}, \dots, x_n)
    \]
    Достаточное условие локального экстремума $F\colon d^2F(\vec{\tilde{x^{(0)}}})$ - положительно (отрицательно) определенная форма.\\
    \[
        d^2F = d(d F) = d \left ( \begin{aligned} \sum^n_{j=1} \frac{\vdelta f}{\vdelta x_j} d x_j \end{aligned}  \right ) = d \left ( \begin{aligned} \sum^n_{j = 1} \frac{\vdelta L}{\vdelta x_j} d x_j \end{aligned}  \right ) = \sum^n_{j=1} \sum^n_{k=1} \frac{\vdelta^2 L}{\vdelta x_j \vdelta x_k} d x_j d x_k + \sum^m_{j=1} \frac{\vdelta L}{\vdelta x_j} d^2 x_j.
    \]
    Воспользуемся тем, что точки $(\vv{x}^{(0)}, \vec{\lambda^{(0)}})$ - стационарные, значит $\frac{\vdelta L}{\vdelta x_j} = 0$. Мы получили в точности форму второго дифференциала. Причем переменные $x_j, x_k$ не являются независимыми, а являются связанными условиями дифференцирования. \\
    Аналогичное доказательство производится для неопределенной квадратичной формы.
\end{proof}

\begin{example} (Кудрявцев - 3 часть, $\S 5, 33(2)$) \\
    Найти максимум и минимум функции $u = x^2 + 2y^2+3z^2$ при ограничениях $x^2+y^2+z^2=1, x+2y+3z=0.$
\end{example}

\begin{solution}
    Заметим, что речь идет об абсолютном экстремуме (в условии нет слова локальный). 
    В данном случае ограничения на $x, y, z$ задают сферу, которая пересекается с плоскостью, проходящей через начало координат. Значит, мы имеем дело с окружностью. \\
    Решение с помощью функции Лагранжа:
    \[ 
        L(x, y, z, \lambda, \mu) = x^2+2y^2+3z^3+\lambda(x^2+y^2+z^2-1)+\mu(x+2y+3z)
    \]
    Найдем стационарные точки функции Лагранжа$\colon$
    \[
        \frac{\vdelta L}{\vdelta x} = 2x+2\lambda x + \mu =0
    \]
    \[
        \frac{\vdelta L}{\vdelta y} = 4y+2\lambda y + 2\mu=0\\
    \]
    \[
        \frac{\vdelta L}{\vdelta z} = 6z+2\lambda z + 3\mu=0\\
    \] 
    Мы получили систему из 5 уравнений с 5 неизвестными. Для ее решения воспользуемся приемом:
    \[
        (2x+2\lambda x + \mu) \cdot x =0 \cdot x\\
    \]
    \[
        (4y+2\lambda y + 2\mu) \cdot y=0 \cdot y\\
    \]
    \[
        (6z+2\lambda z + 3\mu) \cdot z=0 \cdot z\\
    \] 
    Сложим все уравнения системы:
    \[
        (2x+2\lambda x + \mu) \cdot x + \\
    \]
    \[
        + (4y+2\lambda y + 2\mu) \cdot y + \\
    \]
    \[
        + (6z+2\lambda z + 3\mu) \cdot z=\\
    \]
    \[
        = 2u + 2\lambda + 0\mu = 2u + 2\lambda = 0\\
    \]
    Из этого уравнения можно сделать следующий вывод: максимальное (минимальное) значение $u$ зависит от минимального (максимального) значения $\lambda$. Т.е. нам не нужно решать задачу в полном виде (ведь требуется найти минимум и максимум функции, а не какую-то конкретную точку).\\
    Еще раз сложим все уравнения системы (в этот раз без домножения):
    \[
        (2x+2\lambda x + \mu) + \\
        + (4y+2\lambda y + 2\mu) + \\
        + (6z+2\lambda z + 3\mu)=\\
        = 0 + 2\lambda (x+y+z) +6\mu=0\\
    \]
    \[
        \mu = -\frac{\lambda}{3} (x+y+z)
    \]
     Подставим найденное $\mu$ в исходную систему:
     \[
        x(2+\frac{5}{3}\lambda)-\frac{\lambda}{3}y-\frac{\lambda}{3}z=0\\
    \]
    \[
        -\frac{2}{3}\lambda x + y(4+\frac{4}{3}\lambda)-\frac{2}{3}\lambda z=0\\
    \]
    \[
        -\lambda x - \lambda y + z(6+\lambda)=0\\
     \]
     Мы получили неоднородную систему уравнений. Тривиальное решение $x=0, y=0, z=0$ не является ее решением из-за ограничения в условии: $x^2+y^2+z^2=1$. Поэтому для того, чтобы решение существовало, определитель матрицы этой системы однородных уравнений должен быть равен $0$.
     \[
        \begin{vmatrix}
            2+\frac{5}{3}\lambda & -\frac{\lambda}{3} & -\frac{\lambda}{3}\\
            -\frac{2}{3}\lambda & 4+\frac{4}{3}\lambda & -\frac{2}{3}\lambda\\
            -\lambda & -\lambda & 6+\lambda
        \end{vmatrix} = 0\\
    \]
    \[
        det (A - \lambda B)=0 \\
     \]
     Посчитаем определитель, раскроем скобки и приведем подобные слагаемые:
     \[
        \frac{56}{3}\lambda^2 + 64\lambda+48=0\\
    \]
    \[
        \frac{7}{3}\lambda^2+8\lambda+6=0\\
    \]
    \[
        \lambda_1 = \frac{-12+3\sqrt{2}}{7}\\
    \]
    \[
        \lambda_2 = \frac{-12-3\sqrt{2}}{7}
     \]
    Найденные значения $\lambda$ согласно вышеприведенным рассуждениям соответсвуют минимуму и максимуму исходной функции.
\end{solution}

\section{Дифференциальные исчисления в линейных нормированных пространствах}
\begin{definition}
    Линейный оператор $A: E_1 \to E_2, E_1, E_2$ - линейное нормированное пространство (л.н.п.) называется \textit{ограниченным}, если 
    \[
        \exists M \in R \quad \forall x \in E_1 \| Ax \|_{E_2} \leq M \quad \| x \|_{E_1}
    \]
\end{definition}

\begin{proposition}
    Линейный оператор ограничен $\Leftrightarrow$ он непрерывен (в любой точке) .
\end{proposition}

\begin{proof}
    \begin{itemize}
        \item Пусть $A$ - ограниченный линейный оператор: $E_1 \to E_2$. 
        \[
            \| A x_1 - A x_2 \|_{E_2} = \| A (x_1 - x_2) \|_{E_2} \leq M \| x_1 - x_2 \|_{E_1}
        \]
        \[
            \forall \varepsilon > 0 \quad \exists \delta > 0 \forall x_1, x_2 \in E_1 \quad \| x_1 - x_2 \|_{E_1} < \delta\colon
        \]
        \[
            \| A x_1 - A x_2 \|_{E_2} < \varepsilon
        \]
        \item Пусть $A$ - непрерывен в $0,$ т.е.  
        \[
            \forall \varepsilon > 0 \quad \exists \delta = \frac{\varepsilon}{2 M} > 0 \quad \forall h \in E_1  \| h \|_{E_1} < \delta \colon
            \| A h \|_{E_2} < \varepsilon
        \] 
        Зафиксируем $\varepsilon \colon=1 $
        \[
            \forall x \in E_1, x \neq 0 \quad h:=\frac{\delta x }{2 \| x \|_{E_1}} \to
        \]
        \[
            \| h \|_{E_1}  = \frac{\delta}{2 \| x \|_{E_1}} \cdot \| x \|_{E_1} = \frac{\delta}{2} < \delta 
        \]
        Подставим $h$ в неравенство $ \| A h \|_{E_2} < \varepsilon \colon$
        \[
            \| A h \|_{E_2} = \| A \cdot \frac{\delta x}{2 \| x \|_{E_1}} \|_{E_2} = \| A \cdot \frac{\delta x}{2 \| x \|_{E_1}} \|_{E_2} = \| \frac{\delta}{2 \| x \|_{E_1}} \cdot A x \|_{E_2}
        \]
        $A$ - линеный оператор, поэтому положительную константу можно вытащить за знак нормы$\colon$
        \[
              \| \frac{\delta}{2 \| x \|_{E_1}} \cdot A x \|_{E_2} = \frac{\delta}{2 \| x \|_{E_1}} \| A x \|_{E_2} < 1
        \]
        \[
            \| A x \|_{E_2} < \frac{2}{\delta} \| x \|_{E_1}
        \]
        Благодаря этому мы получили константу $M = \frac{2}{\delta}$
    \end{itemize}
    
\end{proof}

\begin{definition}
    Пусть $A$ - линейный ограниченный оператор из $E_1 \to E_2$. $\inf$ множества всех $M$ таких, что $\| A x \|_{E_2} \leq M \| x \|_{E_1} \quad \forall x \in E_1$ называется \textit{нормой} оператора $A$, обозначается $\| A \|$
\end{definition}

\begin{proposition}
    Если $A$ - линейный ограниченный оператор из $E_1 \to E_2, $ то 
    \[
        \| A \| = \sup_{x \in E_1, \| x \| = 1} \| A x \|_{E_2} = \sup_{x \in E, x \neq 0} \frac{\| A x \|_{E_2}}{\| x \|_{E_1}} 
    \]
\end{proposition}

\begin{proof}
    \[
        \sup_{x \in E, x \neq 0} \frac{\| A x \|_{E_2}}{\| x \|_{E_1}}  \leq \| A \|
    \]
    От противного$\colon$ пусть 
    \[
        \sup_{x \in E, x \neq 0} \frac{\| A x \|_{E_2}}{\| x \|_{E_1}}  < \| A \|
    \]
    \[
        \exists M \in (\sup_{x \in E, x \neq 0} \frac{\| A x \|_{E_2}}{\| x \|_{E_1}}, \| A \|)
    \]
    Противоречие, т.к. из этого следует что 
    \[
        M > \sup_{x \in E, x \neq 0} \frac{\| A x \|_{E_2}}{\| x \|_{E_1}}
    \]
    \[
        \forall x \in E, x \neq 0\quad h\colon= \frac{x}{\| x \|}, \| h \| = 1
    \]
    \[
        \frac{\| A x \|_{E_2}}{\| x \|_{E_1}} = \frac{\| x \|_{E_1} \| A h \|_{E_2}}{\| x \|_{E_1}} = \| A h \|_{E_2}
    \]
    Т.е. каждый элемент множества $\frac{\| A x \|_{E_2}}{\| x \|_{E_1}} $  совпадает с каким-то элементом множества \\ $\| A x \|_{E_2}$ и наоборот. Это значит что супремумы этих двух множеств совпадают.
\end{proof}

\begin{note}
    $\mathcal{L} (E_1 \to E_2$) - множество линейных ограниченных оператор из $E_1$ в $E_2$, а также линейное нормированное пространство с нормой $\| A \|$. \\
    Проверим 3 аксиомы нормы$\colon$
    \begin{itemize}
        \item Корректность вынесения константы следует из 
        \[
            \sup_{x \in E_1, \| x \| = 1} \| A x \|_{E_2} = \sup_{x \in E, x \neq 0} \frac{\| A x \|_{E_2}}{\| x \|_{E_1}}
        \]
        \item Eсли норма $=0 \to$ все $\| A x \|_{E_2} = 0 \to A - $ тождественно нулевой оператор.
        \item Проверим, что выполняется неравенство треугольника$\colon$
        \[  
            \| A_1 + A_2 \| = \sup_{x \in E_1, \| x \|_{E_1} = 1} \| (A_1 + A_2) x \|_{E_2} \leq \sup_{x \in E_1, \| x \|_{E_1} = 1} \| A_1 x \|_{E_2} +
        \]
        \[
            + \sup_{x \in E_1, \| x \|_{E_1} = 1} \| A_2 x \|_{E_2} = \| A_1 \| + \| A_2 \|
        \]
    \end{itemize}
\end{note}