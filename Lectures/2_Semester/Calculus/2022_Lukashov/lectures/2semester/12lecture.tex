\begin{theorem} (Критерий Коши сходимости числовых рядов)
	Ряд $\suml_{n = 1}^\infty a_n$ сходится тогда и только тогда, когда верно утверждение:
	\[
		\forall \eps > 0\ \exists N \in \N \such \forall n > N, p \in \N \quad \left|\suml_{k = n + 1}^{n + p} a_k\right| < \eps
	\]
\end{theorem}

\begin{proof}
	Сходимость ряда означает, что последовательность частичных сумм $\{S_n\}_{n = 0}^\infty$ сходится. Остаётся применить к ней критерий Коши о сходимости числовой последовательности:
	\[
		\forall \eps > 0\ \exists N \in \N \such \forall n > N, p \in \N \quad |S_{n + p} - S_n| = \left|\suml_{k = n + 1}^{n + p} a_k\right| < \eps
	\]
\end{proof}

\begin{definition}
	Ряд $\row{n = 1}{a_n}$ \textit{сходится абсолютно}, если сходится ряд $\row{n = 1}{|a_n|}$
\end{definition}

\begin{theorem}
	Если ряд сходится абсолютно, то он сходится.
\end{theorem}

\begin{proof}
	Если ряд сходится абсолютно, то можно расписать критерий сходимости ряда Коши для ряда модулей:
	\[
		\forall \eps > 0\ \exists N \in \N \such \forall n > N, p \in \N \quad \left|\suml_{k = n + 1}^{n + p} |a_n|\right| = \suml_{k = n + 1}^{n + p} |a_n| < \eps
	\]
	А для исходного ряда можно записать неравенство треугольника:
	\[
		\left|\suml_{k = n + 1}^{n + p} a_n\right| \le \suml_{k = n + 1}^{n + p} |a_n| < \eps
	\]
\end{proof}

\begin{definition}
	Если ряд сходится, но не абсолютно, то он \textit{сходится условно}.
\end{definition}

\begin{theorem}
	Если $\forall n \in \N\ \ a_n = b_n + c_n$, то из абсолютной сходимости рядов $\row{n = 1}{b_n}$ и $\row{n = 1}{c_n}$ следует абсолютная сходимость ряда $\row{n = 1}{a_n}$, а из абсолютной сходимости одного из рядов и условной сходимости другого следует условная сходимость $\row{n = 1}{a_n}$.
\end{theorem}

\begin{proof}
	Условие даёт нам неравенство треугольника для элементов:
	\[
		|a_n| = |b_n + c_n| \le |b_n| + |c_n|
	\]
	Если оба ряда сходятся абсолютно, то
	\[
		\suml_{k = n + 1}^{n + p} |a_n| \le \suml_{k = n + 1}^{n + p} |b_n| + \suml_{k = n + 1}^{n + p} |c_n| < \eps
	\]
	Если лишь один ряд сходится абсолютно, то можно предположить, что $\sum_{n = 1}^{\infty} a_n$ сходится абсолютно. Однако, в таком случае третий ряд будет автоматически тоже сходится абсолютно, что не так.
\end{proof}

\begin{note}
	Аналог теоремы выше справедлив и для несобственных интегралов Римана.
\end{note}

\begin{theorem} (Признак Лейбница)
	Если последовательность $\{b_n\}_{n = 1}^\infty$ - невозрастающая и бесконечно малая, то ряд $\row{n = 1}{(-1)^{n + 1} b_n}$ сходится, причём для его остатков $r_n$
	\[
		r_n := \suml_{k = n + 1}^\infty (-1)^{n + 1} b_n
	\]
	справедлива оценка $\forall n \in \N\ |r_n| \le b_{n + 1}$
\end{theorem}

\begin{proof}
	Рассмотрим частичную сумму $S_{2n}$:
	\[
		S_{2n} = b_1 - b_2 + b_3 - b_4 + \ldots + b_{2n - 1} - b_{2n}
	\]
	\begin{itemize}
		\item (Сходимость частичных сумм) Если мы расставим скобки следующим образом:
		\[
			S_{2n} = (b_1 - b_2) + (b_3 - b_4) + \ldots + (b_{2n - 1} - b_{2n})
		\]
		то очевидно, что $S_{2n} \le S_{2(n + 1)}$. Однако, если мы поставим их другим образом:
		\[
			S_{2n} = b_1 - (b_2 - b_3) - (b_4 - b_5) - \ldots - (b_{2n - 2} - b_{2n - 1}) - b_{2n} \le b_1
		\]
		Отсюда $\{S_n\}_{n = 1}^\infty$ - неубывающая, ограниченная сверху. Значит, по теореме Вейерштрасса $\exists \liml_{n \to \infty} S_{2n} = S$. При этом
		\[
			\liml_{n \to \infty} S_{2n + 1} = \liml_{n \to \infty} (S_{2n} + b_{2n + 1}) = S + 0 = S
		\]
		Значит, вся последовательность сходится к $S$.
		
		\item (Оценка остатков) Поступим аналогично предыдущему пункту. В частности, покажем рассуждения для $r_1$:
		\[
			r_1 = -b_2 + b_3 - b_4 + \ldots
		\]
		Если расставить скобки так:
		\[
			r_1 = -b_2 + (b_3 - b_4) + (b_5 - b_6) + \ldots
		\]
		то получим, что $r_1 \ge -b_2$. А если сделать так:
		\[
			r_1 = -(b_2 - b_3) - (b_4 - b_5) - \ldots
		\]
		то поймём, что $r_1 \le 0$. Значит, $|r_1| \le b_2$.
	\end{itemize}
\end{proof}

\begin{theorem} (Признак Дирихле сходимости числовых рядов)
	Если выполнены следующие условия:
	\begin{enumerate}
		\item Частичные суммы $A_n = \sum_{n = 1}^k a_k$ образуют ограниченную последовательность
		
		\item Последовательность $\{b_n\}_{n = 1}^\infty$ монотонна и бесконечно мала
	\end{enumerate}
	Тогда ряд $\row{n = 1}{a_n b_n}$ сходится.
\end{theorem}

\begin{note}
	Предыдущая теорема следует из данной, если положить $a_n = (-1)^{n + 1}$
\end{note}

\begin{proof}
	Рассмотрим сумму $\sum_{k = n + 1}^{n + p} a_k b_k$. Воспользуемся преобразованием Абеля:
	\begin{multline*}
		\suml_{k = n + 1}^{n + p} a_k b_k = \suml_{k = n + 1}^{n + p} (A_k - A_{k - 1}) b_k = A_{n + 1} b_{n + 1} - A_n b_{n + 1} + \ldots + A_{n + p} b_{n + p} - A_{n + p - 1} b_{n + p} =
		\\
		A_{n + p} b_{n + p} - A_n b_{n + 1} - \suml_{k = n + 1}^{n + p - 1} A_k (b_k - b_{k + 1})
	\end{multline*}
	Оценим эту сумму, исходя из того, что $\exists M \such \forall n \in \N\ |A_n| \le M$:
	\[
		\left|\suml_{k = n + 1}^{n + p} a_k b_k\right| \le M \left(|b_{n + p}| + |b_{n + 1}| + \suml_{k = n + 1}^{n + p - 1} |b_k - b_{k + 1}|\right) \le 2M (|b_{n + p}| + |b_{n + 1}|)
	\]
	Последнее неравенство справедливо из-за того, что $b_n$ монотонна и можно вынести модуль за сумму. Внутри этого модуля мы её телескопически сжимаем и снова применяем неравенство треугольника, которое и даст выражение в конце.
	
	Коль скоро $\{b_n\}_{n = 1}^\infty$ - бесконечно малая последовательность, то $\forall \eps > 0$ мы сможем найти $n, p \in \N$ такие, что дальше будет оценка $< \eps$. Таким образом, теорема доказана.
\end{proof}

\begin{corollary} \textit{(Признак Абеля сходимости числовых рядов)}
	Если верны следующие условия:
	\begin{enumerate}
		\item Ряд $\row{n = 1}{a_n}$ сходится
		
		\item $\{b_n\}_{n = 1}^\infty$ - монотонная ограниченная последовательность
	\end{enumerate}
	Тогда ряд $\row{n = 1}{a_n b_n}$ тоже сходится.
\end{corollary}

\begin{proof}
	Поскольку $\{b_n\}_{n = 1}^\infty$ - монотонная и ограниченная, то по теореме Вейерштрасса она сходится:
	\[
		\exists \liml_{n \to \infty} b_n = b_0
	\]
	Положим $\tilde{b}_n := b_n - b_0$. Тогда $\{\tilde{b}_n\}_{n = 1}^\infty$ - монотонная и бесконечно малая последовательность. Тогда, по признаку Дирихле, ряд $\row{n = 1}{a_n \tilde{b}_n}$ сходится. Рассмотрим частичную сумму этого ряда:
	\[
		\tilde{S}_n = \suml_{k = 1}^n a_k \tilde{b}_k = \suml_{k = 1}^n a_k (b_k - b_0) = \suml_{k = 1}^n a_k b_k - b_0\suml_{k = 1}^n a_k
	\]
	Так как ряд $\row{n = 1}{a_n}$ тоже сходится, то и исходный тоже сойдётся.
\end{proof}

\begin{example}
	Изучим ряд, чьи члены имеют следующий вид, через признак Дирихле:
	\[
		a_n = \frac{\sin (nx)}{n^\alpha},\ x \neq \pi k, k \in \Z
	\]
	Рассмотрим случаи:
	\begin{itemize}
		\item $\alpha > 0$. Положим $b_n = \sin nx$. Рассмотрим частичную сумму $B_n$ такого ряда, с коэффициентами $b_n$:
		\[
			B_n = \suml_{k = 1}^n \sin (kx) = \suml_{k = 1}^n \sin (kx) \frac{\sin (x / 2)}{\sin (x / 2)} = \frac{1}{2\sin (x / 2)} \suml_{k = 1}^n \left(\cos \frac{2k - 1}{2}x - \cos \frac{2k + 1}{2}x\right)
		\]
		Несложно заметить, что сумма является телескопической. Отсюда
		\[
			B_n = \frac{\cos \frac{x}{2} - \cos \frac{2n + 1}{2}}{2 \sin (x / 2)}
		\]
		Конечный результат заключается в том, что
		\[
			\left|\suml_{k = 1}^n \sin (kx)\right| \le \frac{1}{|\sin (x / 2)|}
		\]
		
		При всём при этом, $c_n = 1 / n^\alpha$ является монотонной и бесконечно малой последовательностью. Значит, по признаку Дирихле, исходный ряд $\row{n = 1}{a_n}$ сходится при $\alpha > 0$.
		
		\item $\alpha \le 0$. Докажем следующее утверждение:
		\begin{lemma}
			При данных условиях $b_n = \sin (nx)$ не является бесконечно малой последовательностью.
		\end{lemma}
	
		\begin{proof}
			Предположим обратное. Тогда, то же самое можно заявить для последовательности $u_n = \sin ((n + 2)x) - \sin (nx)$. При этом
			\[
				u_n = \sin ((n + 2)x) - \sin (nx) = 2 \sin (x) \cos ((n + 1)x)
			\]
			Коль скоро $\sin x \neq 0$, то $\cos ((n + 1)x) \to 0$ при $n \to \infty$, но это, очевидно, не так.
		\end{proof}
	
		Значит, существует подпоследовательность ${b_{n_k}}_{k = 1}^\infty$ такая, что
		\[
			\exists c > 0 \such \forall k \in \N \quad |\sin (n_k x)| \ge c > 0
		\]
		Но тогда очевидно, что
		\[
			\frac{\sin (n_k x)}{n_k^\alpha} \centernot{\xrightarrow[k \to \infty]{}} 0
		\]
	\end{itemize}

	Теперь узнаем, при каких $\alpha > 0$ этот ряд будет сходится абсолютно. Заметим, что
	\[
		\left|\frac{\sin (nx)}{n^\alpha}\right| \le \frac{1}{n^\alpha}
	\]
	Отсюда следует абсолютная сходимости при $\alpha > 1$. Теперь $0 < \alpha \le 1$. Тогда
	\[
		\left|\frac{\sin (nx)}{n^\alpha}\right| \ge \frac{\sin^2 (nx)}{n^\alpha} = \frac{1 - \cos (2nx)}{2n^\alpha}
	\]
	При этом $\row{n = 1}{1 / n^\alpha}$ расходится при $0 < \alpha \le 1$, а $\row{n = 1}{\cos(2nx) / n^\alpha}$, наоборот, сходится (тоже по признаку Дирихле). Отсюда следует условная сходимость ряда при $0 < \alpha \le 1$.
\end{example}

\begin{example}
	Для ряда $\row{n = 1}{a_n}$ применим признак Абеля, чтобы исследовать сходимость:
	\[
		a_n = \left(1 + \frac{1}{n}\right)^n \cdot \frac{\sin(nx)}{n^\alpha}
	\]
	Теперь уже положим $b_n = \frac{\sin(nx)}{n^\alpha}$. Про второй множитель помним, что он сходится к числу Эйлера, а также монотонен. Значит, при $\alpha > 0$ исходный ряд будет сходится.
\end{example}

\begin{theorem} (Перестановки абсолютно сходящегося ряда)
	Если ряд $\row{n = 1}{a_n}$ сходится абсолютно, то сходится (и притом абсолютно) ряд $\row{n = 1}{a^*_n}$, где $\forall k \in \N\ \ a^*_k = a_{n_k}$ и подпоследовательность $\{n_k\}_{k = 1}^\infty$ является биективным образом $\N$. Более того
	\[
		\row{n = 1}{a_n} = \row{n = 1}{a^*_n}
	\]
\end{theorem}

\begin{proof}
	Так как исходный ряд сходится абсолютно, то частичные суммы его модулей ограничены. Рассмотрим частичную сумму $\sum_{k = 1}^n |a^*_k|$. Тогда
	\[
		\exists N \such \suml_{k = 1}^n |a^*_k| \le \suml_{k = 1}^N |a_k|
	\]
	Это верно, ибо нужно посмотреть на индексы элементов исходного ряда, с которыми связаны $a^*_k$, и выбрать максимальный из них. Отсюда моментально следует абсолютная сходимость ряда со звёздочкой.
	
	Обозначим $S^*_n = \sum_{k = 1}^n a^*_k$. Тогда, пользуясь уже написанным выше, получим утверждение:
	\[
		\forall n \in \N\ \exists N \in \N \such \forall m \ge N \quad |S_m - S^*_n| \le |a^*_{n + 1}| + \ldots \suml_{k = n + 1}^\infty = |a^*_k|
	\]
	Как понимать правую часть? Все, что есть в $S^*_n$, есть и в $S_m$. При этом по условию, у нас есть биекция между рядами. Стало быть, для оставшихся в $S_m$ элементов есть их образы где-то в ряду со звёздочкой, причём явно дальше номера $n$. Отсюда и следует полученное неравенство.
	
	Коль скоро $\liml_{m \to \infty} S_m = S$, то имеет место неравенство
	\[
		|S - S^*_n| \le \suml_{k = n + 1}^\infty |a^*_k| \xrightarrow[n \to \infty]{} 0
	\]
	Итого, $\liml_{n \to \infty} S^*_n = S$
\end{proof}

\begin{theorem} (Римана)
	Если ряд $\row{n = 1}{a_n}$ сходится условно, то $\forall A \in \bar{\R} = \R \cup \{\pm \infty\}$ существует биективный образ $\N$ такой, что
	\[
		\suml_{k = 1}^K a_{n_k} \xrightarrow[K \to \infty]{} A
	\]
\end{theorem}

\begin{proof}
	Определим 2 последовательности:
	\[
		a^+_n = \System{
			&{a_n, a_n \ge 0}
			\\
			&{0, a_n < 0}
		}
	\]
	\[
		a^-_n = \System{
			&{0, a_n \ge 0}
			\\
			&{-a_n, a_n < 0}
		}
	\]
	Тогда $\forall n \in \N\ \ a_n = a^+_n - a^-_n$, а также $|a_n| = a^+_n + a^-_n$.
	
	Обозначим за $\{p_n\}_{n = 1}^\infty \subset \{a^+_n\}_{n = 1}^\infty$ - подпоследовательность, состоящая из всех ненулевых элементов в том же порядке. Аналогично $\{q_n\}_{n = 1}^\infty \subset \{a^-_n\}_{n = 1}^\infty$. Заметим, что ряды $\row{n = 1}{p_n}$ и $\row{n = 1}{q_n}$ расходятся (иначе должен сойтись ряд модуля, а это не так).
	
	Построим биективный образ $\N$ так, что частичная сумма нового ряда сразу же останавливается, как только перевалилась за значение $A$:
	\begin{align*}
		&{N_1 := \System{
			&{\min \{n \in \N \such p_1 + \ldots + p_n > A\}, A \in \R}
			\\
			&{\min \{n \in \N \such p_1 + \ldots + p_n > 1\}, A = +\infty}
			\\
			&{1, A = -\infty}
		}}
		\\
		&{M_1 := \System{
				&{\min \{n \in \N \such p_1 + \ldots + p_{N_1} - q_1 - \ldots - q_n < A\}, A \in \R}
				\\
				&{1, A = +\infty}
				\\
				&{\min \{n \in \N \such p_1 - q_1 - \ldots - q_n < -1\}, A = -\infty}
		}}
	\end{align*}
	Аналогично определяем $N_m$ и $M_m$:
	\begin{itemize}
		\item $A \in \R$
		\[
			N_m := \min \{n \in \N \such \suml_{k = 1}^{N_1} p_k - \suml_{k = 1}^{M_1} q_k + \ldots + \suml_{k = N_{m - 2} + 1}^{N_m - 1} p_k - \suml_{k = M_{m - 2} + 1}^{M_{m - 1}} q_k + p_{N_{m - 1} + 1} + \ldots + p_n > A\}
		\]
		\[
			M_m := \min \{n \in \N \such \suml_{k = 1}^{N_1} p_k - \ldots - \suml_{k = M_{m - 2} + 1}^{M_{m - 1}} q_k + \suml_{k = N_{m - 1} + 1}^{N_m} p_k - q_{M_{m - 1} + 1} - \ldots - q_n < A\}
		\]
		
		\item $A = +\infty$
		
		
		\item $A = -\infty$
		
	\end{itemize}

	\textcolor{red}{Я пока не готов морально описывать то, что тут происходит. Просто смотрите конец 12й лекции по матану за весну 2022.}
\end{proof}