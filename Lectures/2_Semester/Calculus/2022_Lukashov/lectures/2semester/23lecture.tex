\section{Приложения интеграла Римана в геометрии}

\subsection{Площадь плоских фигур}

\begin{definition}
	\textit{Площадью} называется мера Жордана в $\R^2$.
\end{definition}

\begin{definition}
	Фигуры, измеримые по Жордану в $\R^2$, называются \textit{квадрируемыми}.
\end{definition}

\begin{theorem}
	Пусть $f$ - неотрицательная интегрируемая по Риману функция на $[a; b]$. Тогда её подграфик (криволинейная трапеция) $G_f = \{(x, y) \colon a \le x \le b,\ 0 \le y \le f(x)\}$ квадрируем, причём его площадь равна $\int_a^b f(x)dx$ и наоборот.
\end{theorem}

\begin{proof}
	Заметим, что $L(P, f) \le \downjm(G_f) \le \upjm(G_f) \le U(P, f)$, ибо нижняя и верхняя суммы Дарбу соответствуют площадям элементарных множеств. Отсюда уже видно, что при интегрируемости $f$ будет совпадение площади с интегралом.
	
	Аналогично в обратную сторону: если верхняя и нижняя меры Жордана совпадают, то мы можем приблизить их с любой точностью, а отсюда уже можно взять элементарные множества для нижней и верхней сумм Дарбу.
\end{proof}

\begin{exercise}
	Докажите, что площадь сектора с центральным углом $0 \le \alpha \le 2\pi$ на окружности радиуса $R$ равна
	\[
		S_\alpha = \frac{\alpha R^2}{2}
	\]
\end{exercise}

\begin{theorem} (Площадь фигуры в полярных координатах)
	Если фигура $G$ задана как $G = \{(r, \phi) \colon \psi_1 \le \phi \le \psi_2,\ 0 \le r \le r(\phi)\}$, где $r, \phi$ - полярные координаты, причём $r(\phi) \in R[\psi_1; \psi_2]$ (предполагаем, что $\psi_2 - \psi_1 \le 2\pi$), то $G$ квадрируема, причём
	\[
		\jm(G) = \frac{1}{2} \int_{\psi_1}^{\psi_2} r^2(\phi)d\phi
	\]
\end{theorem}

\textcolor{red}{Сюда бы картиночку с 23 лекции весны 2022, 1:04:30}

\begin{proof}
	Положим $f(\phi) = \frac{1}{2}r^2(\phi)$. Запишем, например, нижнюю сумму Дарбу для этой функции:
	\[
		L(P, f) = \sum_{k = 1}^n m_k \Delta \phi_k
	\]
	Что есть слагаемое из этой суммы? Это площадь сектора с центральным углом $\Delta \phi_k$ на окружности радиуса $m_k$. Стало быть, можно применить аналогичные рассуждения, как и в предыдущей теореме (только элементарные множества немного искривляются).
\end{proof}

\subsection{Длина кривой}

\begin{proposition}
	Если кривая задана параметризацией $\vv{r}(t),\ t_0 \le t \le T$, где $\vv{r}$ - непрерывно дифференцируемая вектор-функция, то её длина равна
	\[
		\int_{t_0}^T |\vv{r'}(t)|dt
	\]
\end{proposition}

\begin{proof}
	Уже известен тот факт, что $s'(t) = |\vv{r'}(t)|$. Тогда
	\[
		\int_{t_0}^T s'(t)dt = \int_{t_0}^T |\vv{r'}(t)|dt = s(T) - s(t_0)
	\]
\end{proof}

\subsection{Объём тела вращения}

\begin{definition}
	\textit{Объём} --- это мера Жордана в $\R^3$.
\end{definition}

\begin{definition}
	Тело, измеримое по Жордану в $\R^3$, называется \textit{кубируемым}.
\end{definition}

\begin{lemma}
	Если $G$ --- квадрируемая фигура, то $\mathcal{G} = \{(x, y, z) \colon (x, y) \in G,\ z \in [a; b]\}$ --- кубируемая фигура, причём
	\[
		\jm(\mathcal{G}) = (b - a)\jm(G)
	\]
\end{lemma}

\begin{proof}
	По условию известно, что
	\[
		\forall \eps > 0\ \exists E_1 \subset G \subset E_2 \text{ -- элементарные} \colon |E_2 \bs E_1| < \frac{\eps}{b - a}
	\]
	Тогда $\mathcal{E}_1 = E_1 \times [a; b]$, $\mathcal{E}_2 = E_2 \times [a; b]$ --- тоже элементарные множества, причём $\mathcal{E}_1 \subset \mathcal{G} \subset \mathcal{E}_2$ и $|\mathcal{E}_2 \bs \mathcal{E}_1| = |E_2 \bs E_1| \cdot (b - a) < \eps$. Отсюда тривиально следует измеримость $\mathcal{G}$ и равенство в мере.
\end{proof}

\begin{theorem} (Объём тела вращения)
	Если $f$ - неотрицательная интегрируемая по Риману функция на $[a; b]$, то тело, полученное вращением её подграфика $G_f$ вокруг оси $ox$, кубируемо, причём его объём вычисляется следующим образом:
	\[
		\jm(\mathcal{G}_f) = \pi \int_a^b f^2(x)dx
	\]
\end{theorem}

\begin{proof}
	Обозначим $F(x) = \pi f^2(x)$ и заметим, что слагаемое в сумме Дарбу для $f$ - это объём цилиндра, который измерим по уже доказанной лемме. К этому слагаемому можно приблизиться сколь угодно близко при помощи элементарных множеств сверху/снизу, стандартная тактика.
\end{proof}