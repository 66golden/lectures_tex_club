\begin{note}
	Поведение степенного ряда на радиусе сходимости может быть каким угодно. Следующие 3 примера это иллюстрируют.
\end{note}

\begin{example}
	Рассмотрим ряд $\row{n = 1}{z^n/n}$. Очевидно, что радиус сходимости $R = 1$. Возможно 2 случая для $|z| = 1$:
	\begin{itemize}
		\item $z = 1$. Тогда ряд превращается в гармонический. Про него мы знаем, что он расходится.
		
		\item $z \neq 1$. Тогда
		\[
			z = \cos \phi + i\sin \phi = e^{i\phi},\ \phi \neq 2\pi k,\ k \in \Z
		\]
		Отсюда ряд получает вид
		\[
			\row{n = 1}{\frac{z^n}{n}} = \row{n = 1}{\frac{e^{i\phi}}{n}} = \row{n = 1}{\frac{\cos n\phi}{n}} + i\row{n = 1}{\frac{\sin n\phi}{n}}
		\]
		Про мнимую часть мы уже знаем, что она сходится для любых наших $\phi$. Для первого слагаемого покажем, что $\sum_{k = 1}^n \cos (k\phi)$ - ограниченная частичная сумма
		\begin{multline*}
			\suml_{k = 1}^n = \suml_{k = 0}^n \cos (k\phi) - 1 = \frac{1}{2}\suml_{k = 0}^n (e^{ik\phi} + e^{-ik\phi}) - 1 = \frac{1}{2}\suml_{k = 0}^n (e^{i\phi})^k + \frac{1}{2}\suml_{k = 0}^n(e^{-i\phi})^k - 1 =
			\\
			\frac{1}{2} \cdot \frac{1 - e^{i(n + 1)\phi}}{1 - e^{i\phi}} + \frac{1}{2} \cdot \frac{1 - e^{-i(n + 1)\phi}}{1 - e^{-i\phi}} - 1 = \frac{\sin^2 \frac{\phi}{2} + \sin \frac{\phi}{2} \sin \frac{2n + 1}{2}\phi}{2\sin^2 \frac{\phi}{2}} - 1
 		\end{multline*}
 		Отсюда возникает формула, правая часть которой называется \textit{ядром Дирихле}:
 		\[
 			\frac{1}{2} + \suml_{k = 1}^n \cos(k\phi) = \frac{\sin\left(\frac{2n + 1}{2}\phi\right)}{2\sin\left(\frac{\phi}{2}\right)}
 		\]
 		Несложно увидеть, что при наших $\phi$ правая часть ограничена. А значит, по признаку Дирихле, ряд будет сходится.
	\end{itemize}
\end{example}

\begin{example}
	Рассмотрим ряд $\row{n = 1}{\frac{z^n}{n^2}}$. Аналогично $R = 1$ и при $|z| = 1$ ряд сходится (по признаку Вейерштрасса).
\end{example}

\begin{example}
	Рассмотрим ряд $\row{n = 1}{z^n}$. Снова $R = 1$, но при $|z| = 1$ ряд расходится (общий член не стремится к нулю при таком модуле).
\end{example}

\begin{lemma}
	Степенные ряды $\row{n = 0}{a_n(x - x_0)^n}$ и $\row{n = 1}{na_n(x - x_0)^{n - 1}}$ имеют одинаковые радиусы сходимости.
\end{lemma}

\begin{proof}
	Рассмотрим радиус сходимости второго ряда по определению:
	\[
		\varlimsup\limits_{n \to \infty} \sqrt[n]{|a_n|} = \underbrace{\liml_{n \to \infty} \sqrt[n]{n}}_{1} \cdot \varlimsup\limits_{n \to \infty} \sqrt[n]{|a_n|} = \varlimsup\limits_{n \to \infty} \sqrt[n]{|na_n|}
	\]
\end{proof}

\begin{note}
	До конца параграфа мы полагаем все коэффициенты и аргументы действительными, если не оговорено иного.
\end{note}

\begin{theorem} (Почленное дифференцирование и интегрирование степенных рядов)
	Пусть $f(x) = \row{n = 0}{a_n(x - x_0)^n}$, где $|x - x_0| < R$, $R > 0$ - радиус сходимости этого степенного ряда. Тогда
	\begin{enumerate}
		\item $f(x)$ бесконечно дифференцируема (то есть имеет производные любого порядка) в $\forall x, |x - x_0| < R$, причём
		\[
			f^{(k)}(x) = \suml_{n = k}^\infty a_n \cdot n(n - 1) \cdot \ldots \cdot (n - k + 1) \cdot (x - x_0)^{n - k}
		\]
		
		\item Для $\forall x, |x - x_0| < R$ верно, что $f$ интегрируема по Риману на отрезке с концами $x_0$, $x$, причём
		\[
			\int_{x_0}^x f(t)dt = \suml_{n = 0}^\infty a_n \frac{(x - x_0)^{n + 1}}{n + 1}
		\]
		
		\item Все степенные ряды, упомянутые выше, имеют радиус сходимости $R$.
		
		\item Имеет место равенство
		\[
			a_n = \frac{f^{(n)}(x_0)}{n!},\ n \in \N \cup \{0\}
		\]
	\end{enumerate}
\end{theorem}

\begin{proof}~
	\begin{enumerate}
		\item Радиус сходимости ряда $\row{n = 1}{na_n(x - x_0)^{n - 1}}$ равен $R$ - радиусу изначального ряда по уже доказанной лемме. Значит, на $|x - x_0| \le r < R$ он сходится равномерно. Отсюда, по теореме о дифференцировании функциональных последовательностей (которую мы применяем для частичных сумм), имеет место равенство:
		\[
			f'(x) = \suml_{n = 1}^\infty na_n(x - x_0)^{n - 1}
		\]
		Данные рассуждения верны рекурсивно до производной любого порядка.
		
		\item Заметим, что для $\forall x, |x - x_0| = r < R$ ряд для $f(x)$ на отрезке между $x_0$ и $x$ сходится равномерно. Значит, его можно почленно интегрировать.
		
		\item Если взять формальную производную от ряда интеграла, то получим исходный ряд. По лемме это даёт равенство радиусов сходимости (а для дифференцирования уже всё доказано).
		
		\item Заметим, что
		\[
			f^{(k)}(x_0) = a_k \cdot k!
		\]
		Отсюда
		\[
			a_k = \frac{f^{(k)}(x_0)}{k!}
		\]
	\end{enumerate}
\end{proof}

\begin{definition}
	Если $f$ бесконечно дифференцируема в точке $x_0$, то ряд $\row{n = 0}{\frac{f^{(n)}(x_0)}{n!}(x - x_0)^n}$ называется её \textit{рядом Тейлора (с центром) в точке $x_0$}. Если $x_0 = 0$, то ряд Тейлора называется \textit{рядом Маклорена}.
\end{definition}

\begin{note}
	По доказанной теореме, если функция представлена каким-либо рядом в точке $x_0$, то этот ряд будет рядом Тейлора.
\end{note}

\begin{example}
	Рассмотрим функцию следующего вида:
	\[
		f(x) = \System{
			&{e^{-1/x^2}, x \neq 0}
			\\
			&{0, x = 0}
		}
	\]
	Прежде всего докажем, что $f^{(n)}(x) = P_{3n}(1/x)e^{-1/x^2}$, где $P_{3n}(t)$ - многочлен степени $3n$ со старшим коэффициентов $2^n,\ x \neq 0$. Доказательство индукцией по $n$:
	\begin{itemize}
		\item База $n = 1$:
		\[
			f'(x) = \frac{2}{x^3} e^{-1/x^2} \Longrightarrow P_3(t) = 2t^3
		\]
		
		\item Переход $n = k \Ra k + 1$:
		\[
			f^{(k + 1)}(x) = \left(P_{3n}\left(\frac{1}{x}\right)e^{-\frac{1}{x^2}}\right)' = -\frac{1}{x^2} P'_{3k}\left(\frac{1}{x}\right)e^{-\frac{1}{x^2}} + \frac{2}{x^3} P_{3k}\left(\frac{1}{x}\right)e^{-\frac{1}{x^2}}
		\]
		Значит $P_{3k + 3}(t) := -t^2 P'_{3k}(t) + 2t^3 P_{3k}(t)$
	\end{itemize}
	Теперь, докажем по индукции, что $f^{(n)}(0) = 0,\ n \in \N$:
	\begin{itemize}
		\item База $n = 1$:
		\[
			f'(0) = \liml_{x \to 0} \frac{e^{-1/x^2}}{x} = 0
		\]
		
		\item Переход $n = k \Ra k + 1$: распишем производную по определению:
		\[
			f^{(k + 1)}(0) = \liml_{x \to 0} \frac{f^{(k)}(x) - f^{(k)}(0)}{x} = \liml_{x \to 0} \frac{1}{x}P_{3n - 3}\left(\frac{1}{x}\right)e^{-\frac{1}{x^2}} = \liml_{t \to \infty} \frac{P_{3n - 2}(t)}{e^{t^2}}
		\]
		Осталось заметить следующее:
		\[
			\forall m \in \N\ \liml_{t \to \infty} \frac{t^m}{e^{t^2}} = \liml_{y \to +\infty} \frac{\pm y^{m/2}}{e^y} = \ldots = 0
		\]
		где троеточие символизирует многократное приминение правила Лопиталя. Значит
		\[
			f^{(k + 1)}(0) = \liml_{t \to \infty} \frac{P_{3n - 2}(t)}{e^{t^2}} = 0
		\]
	\end{itemize}
	Из всего вышесказанного следует, что ряд Тейлора для $f(x)$ будет нулевым для $x_0 = 0$. Значит
	\[
		\forall x \neq 0 \quad f(x) \neq \row{n = 0}{\frac{f^{(n)}(0)}{n!}x^n}
	\]
\end{example}

\begin{example}
	Рассмотрим $f(x) = e^x$. Тогда понятно, что
	\[
		f^{(n)}(x) = e^x,\ f^{(n)}(0) = 1
	\]
	Отсюда получаем ряд Маклорена:
	\[
		\row{n = 0}{\frac{x^n}{n!}}, \quad R = \liml_{n \to \infty} \frac{1/n!}{1/(n + 1)!} = +\infty
	\]
	Однако, мы не можем пока написать равенства с $e^x$, ибо у нас нету для этого признаков. Тем не менее, следствием сходимости ряда будет интересный предел:
	\[
		\liml_{n \to \infty} \frac{x^n}{n!} = 0
	\]
\end{example}

\begin{theorem} (Достаточное условие представимости функции рядом Тейлора)
	Если $f$ бесконечно дифференцируема на $(x_0 - h; x_0 + h)$, причём
	\[
		\exists M \ge 0 \such \forall n \in \N, x \in (x_0 - h; x_0 + h) \quad |f^{(n)}(x)| \le M
	\]
	Тогда $f(x)$ представима своим рядом Тейлора в точке $x_0$ при всех $x \in (x_0 - h; x_0 + h)$.
\end{theorem}

\begin{proof}
	По теореме о формуле Тейлора с остаточным членом в форме Лагранжа, для всех $x$ из требуемого интервала можно написать следующую формулу:
	\[
		f(x) = \suml_{k = 0}^n \frac{f^{(k)}(x_0)}{k!}(x - x_0)^k + \frac{f^{(n + 1)}(\xi)}{(n + 1)!}(x - x_0)^{n + 1},\ \xi \text{ между } x \text{ и } x_0
	\]
	Следовательно
	\[
		\left|f(x) - \suml_{k = 0}^n \frac{f^{(k)}(x_0)}{k!}(x - x_0)^k\right| \le M \cdot \frac{|x - x_0|^{n + 1}}{(n + 1)!} \xrightarrow[n \to \infty]{} 0
	\]
	Стремление к нулю следует из следствия в примере. В итоге. разность между функцией и частичной суммой стремится к нулю, а значит $f(x)$ разложима в ряд Тейлора.
\end{proof}

\subsubsection*{Основные разложения элементарных функций в ряды Маклорена}

\begin{enumerate}
	\item[I] \(e^x = \suml_{n = 0}^\infty \frac{x^n}{n!},\ x \in \R\)
	
	\item[II] \(\sin x = \suml_{n = 0}^\infty (-1)^n \frac{x^{2n + 1}}{(2n + 1)!},\ x \in \R\)
	
	\item[III] \(\cos x = \suml_{n = 0}^\infty (-1)^n \frac{x^{2n}}{(2n)!},\ x \in \R\)
	
	\item[IV] \((1 + x)^\alpha = 1 + \suml_{n = 1}^\infty \frac{\alpha \cdot \ldots \cdot (\alpha - n + 1)}{n!}x^n,\ -1 < x < 1\) (при отдельных $\alpha$ $\pm 1$ может включаться)
	
	\item[V] \(\ln(1 + x) = \suml_{n = 1}^\infty (-1)^{n + 1} \frac{x^n}{n},\ -1 < x \le 1\)
	
	\item \(\ch x = \suml_{n = 0}^\infty \frac{x^{2n}}{(2n)!},\ x \in \R\)
	
	\item \(\sh x = \suml_{n = 0}^\infty \frac{x^{2n + 1}}{(2n + 1)!},\ x \in \R\)
	
	\item \(\arctg x = \suml_{n = 0}^\infty (-1)^n \frac{x^{2n + 1}}{2n + 1},\ -1 < x \le 1\)
	
	\item \(\arcsin x = \row{n = 1}{\frac{(2n - 1)!!}{(2n)!!} \cdot \frac{x^{2n + 1}}{2n + 1}},\ -1 < x < 1\), где $n!!$ - двойной факториал.
	
	\item \(\ln(1 + \sqrt{1 - x^2}) = \ln 2 - \row{n = 1}{\frac{(2n - 1)!!}{(2n)!!} \cdot \frac{x^{2n}}{2n}},\ -1 < x < 1\)
\end{enumerate}

\begin{proof}~
	\begin{enumerate}
		\item[I.] Ряд Маклорена мы уже получили. Чтобы доказать, что он действительно совпадает с $e^x$, рассмотрим $\forall x \in (-h; h)$. Тогда
		\[
			|(e^x)^{(n)}| = |e^x| \le e^h
		\]
		Этим утверждением разложение доказано.
		
		\item[II.] Для формулы Тейлора мы уже получали формулу. Покажем ограниченность производной, как и в первом пункте:
		\[
			|(\sin x)^{(n)}| = \left|\sin\left(x + \frac{\pi n}{2}\right)\right| \le 1
		\]
		
		\item[III.]
		\[
			|(\cos x)^{(n)}| = \left|\cos\left(x + \frac{\pi n}{2}\right)\right| \le 1
		\]
		
		\item[IV.] Метод доказанного достаточного условия не сможет показать равенство функции с этим рядом. Но есть выход через формулу Тейлора с остаточным членом в интегральной форме:
		\[
			r_n(x) = \frac{1}{n!} \int_0^x (x - t)^n \left((1 + t)^\alpha\right)^{(n + 1)}dt \text{ - остаточный член}
		\]
		Тогда
		\begin{multline*}
			r_n(x) = \frac{\alpha \cdot \ldots \cdot (\alpha - n)}{n!} \int_0^x (x - t)^n (1 + t)^{(\alpha - n - 1)}dt =
			\\
			\frac{\alpha \cdot \ldots \cdot (\alpha - n)}{n!} \int_0^x \left(\frac{x - t}{1 + t}\right)^n (1 + t)^{\alpha - 1}dt
		\end{multline*}
		Рассмотрим функцию $\phi_x(t) = (x - t) / (1 + t)$. Её производная выглядит следующим образом:
		\[
			(\phi_x)'(t) = \frac{-(1 + t) - (x - t)}{(1 + t)^2} = -\frac{x + 1}{(1 + t)^2} < 0
		\]
		То есть функция убывает. Тогда $\phi_x(t)$ принимает самое большое своё значение в нуле и остаточный член оценивается следующим образом:
		\[
			|r_n(x)| \le \frac{|\alpha \cdot \ldots \cdot (\alpha - n)|}{n!}|x|^n \int_0^1 (1 + t)^{\alpha - 1}dt
		\]
		Выражение, стоящее слева от интеграла, можно рассмотреть как общий член степенного ряда. Тогда, радиус сходимости $R$ для него будет
		\[
			R = \liml_{n \to \infty} \left|\frac{\alpha \cdot \ldots \cdot (\alpha - n)}{n!} \cdot \frac{(n + 1)!}{\alpha \cdot \ldots \cdot (\alpha - n - 1)}\right| = \liml_{n \to \infty} \frac{n + 1}{n + 1 - \alpha} = 1
		\]
		Значит
		\[
			\forall x, |x| < 1 \quad \liml_{n \to \infty} \frac{\alpha \cdot \ldots \cdot (\alpha - n)}{n!}|x|^n = 0
		\]
		Ну а так как интеграл - константа, то остаточный член стремится к нулю при $n \to \infty$.
		
		\item[V.] Воспользуемся тем, что
		\[
			\big(\ln (1 + x)\big)' = (1 + x)^{-1} = \suml_{n = 0}^\infty (-1)^n x^n,\ -1 < x < 1
		\]
		Интегрируя почленно, получим ряд для $-1 < x < 1$. Для $x = 1$:
		\[
			\ln (2) = \suml_{n = 1}^\infty \frac{(-1)^{n + 1}}{n}
		\]
		Ряд сходится по признаку Лейбница. Значит, по второй теореме Абеля, функция суммы ряда является непрерывной. Отсюда, если мы посмотрим предел $x \to 1-$, получим требуемое равенство.
		
		\item[3.] Вспомним, что
		\[
			(\arctg x)' = (1 + x^2)^{-1} = \suml_{n = 0}^\infty (-1)^n x^{2n},\ -1 < x < 1
		\]
		Сходимость в $x = 1$ следует по признаку Лейбница. В частности, из этого факта следует следующее:
		\[
			\frac{\pi}{4} = \row{n = 0}{(-1)^n\frac{1}{2n + 1}}
		\]
		
		\item[4.] Снова посмотрим на производную:
		\begin{multline*}
			(\arcsin x)' = \frac{1}{\sqrt{1 - x^2}} = (1 - x^2)^{-1/2} = 1 + \row{n = 1}{C_{-1/2}^n (-x^2)^n} =
			\\
			1 + \row{n = 1}{\frac{(2n - 1)!!}{2^n \cdot n!}x^{2n}} = 1 + \row{n = 1}{\frac{(2n - 1)!!}{(2n)!!}x^{2n}},\ -1 < x < 1
		\end{multline*}
		
		\item[5.] Аналогично рассматриваем производную:
		\begin{multline*}
			(\ln(1 + \sqrt{1 - x^2}))' = \frac{1}{1 + \sqrt{1 - x^2}} \cdot \frac{-x}{\sqrt{1 - x^2}} = -\frac{1}{x}\left(\frac{1}{\sqrt{1 - x^2}} - 1\right) =
			\\
			-\frac{1}{x}\big((1 - x^2)^{-1/2} - 1\big) = -\frac{1}{x}\row{n = 1}{\frac{(2n - 1)!!}{(2n)!!}x^{2n}} =
			\\
			-\row{n = 1}{\frac{(2n - 1)!!}{(2n)!!} x^{2n - 1}},\ -1 < x < 1
		\end{multline*}
	\end{enumerate}
\end{proof}