\section{Неопределённый интеграл}

\subsection{Основные приёмы интегрирования}

\begin{definition}
	Функция $F$ называется \textit{первообразной} для функции $f$ на промежутке $I$, если верно равенство:
	\[
		\forall x \in I\ \ F'(x) = f(x)
	\]
\end{definition}

\begin{note}
	Если мы имеем дело с концевыми точками, то тогда нужно смотреть односторонние пределы производной.
\end{note}

\begin{lemma}
	Если $\forall x \in (a; b)\ \ \phi'(x) = 0$, то $\phi(x) = const$ на $(a; b)$.
\end{lemma}

\begin{proof}
	Зафиксируем произвольную точку $x_0 \in (a; b)$. Функция $\phi$ удовлетворяет условиям теоремы Лагранжа и для $\forall x \in (x_0; b)$ выполняется равенство
	\[
		\exists c \in (x_0; x) \such \phi(x) - \phi(x_0) = \phi'(c) \cdot (x - x_0)
	\]
	Поскольку $\phi'(c) = 0$, то $\phi(x) = \phi(x_0)$. Аналогично рассматривается случай $\forall x \in (a; x_0)$.
\end{proof}

\begin{theorem} (Об общем виде первообразных) \label{commonViewTheorem}
	Если $F$ - первообразная $f$ на $(a; b)$, то любая первообразная $\Phi$ для $f$ на $(a; b)$ имеет вид:
	\[
		\Phi(x) = F(x) + C
	\]
	где $C \in \R$.
	
	И наоборот: любая функция $\Phi(x) = F(x) + C$, где $C \in \R$ - первообразная $f$ на $(a; b)$.
\end{theorem}

\begin{proof}~
	\begin{enumerate}
		\item $\Ra$ Пусть $\Phi(x)$ и $F(x)$ - первообразные $f$ на $(a; b)$. Тогда они удовлетворяют равенствам:
		\[
			\System{
				&{\Phi'(x) = f(x)}
				\\
				&{F'(x) = f(x)}
			}
		\]
		Следовательно по лемме $h(x) = \Phi(x) - F(x) = const$, что и требовалось доказать.
		
		\item $\La$ Просто посчитаем производную $\Phi(x)$:
		\[
			\Phi'(x) = (F(x) + C)'(x) = f(x)
		\]
		То есть $\Phi(x)$ является первообразной $f(x)$, что и требовалось доказать.
	\end{enumerate}
\end{proof}