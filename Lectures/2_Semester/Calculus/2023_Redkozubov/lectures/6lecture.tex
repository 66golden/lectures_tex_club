%16.02.23

\begin{theorem}[признак Гаусса]
    Пусть $a_{n} > 0$ для всех $n \in \N$ и существуют такие $s > 1$ и ограниченная последовательность $\{\alpha_{n}\}$, что для всех $n$ выполнено
    \[\frac{a_{n+1}}{a_{n}} = 1 - \frac{A}{n} + \frac{\alpha_{n}}{n^{s}}.\]
    Тогда ряд $\sum_{n = 1}^{+\infty} a_{n}$ сходится при $A > 1$ и расходится иначе. 
\end{theorem}

\begin{proof}
    При $n > 1$ имеем
    \[a_{n} = a_{1}\cdot \frac{a_{2}}{a_{1}}\cdot \ldots \cdot \frac{a_{n}}{a_{n - 1}} = a_{1}\cdot\prod_{k = 1}^{n - 1}\left(1 - \frac{A}{k} + \frac{\alpha_{k}}{k^{s}}\right) = a_{1}\cdot \exp\left(\sum_{k = 1}^{n-1} \ln(1 - \frac{A}{k} + \frac{\alpha_{k}}{k^{s}})\right).\]
    
    Так как $\ln(1 + t) = t + O(t^{2})$, $t \to 0$, имеем
    \[a_{n} = a_{1}\cdot \exp\left(\sum_{k = 1}^{n - 1}\left(-\frac{A}{k} + \frac{\alpha_{k}}{k^{s}} + O\left(\frac{1}{k^{2}}\right)\right)\right).\]
    
    Воспользуемся равенством $\sum_{k = 1}^{n - 1} \frac{1}{k} = \ln n + \gamma + o(1)$ и сходимостью рядов $\sum_{k = 1}^{+\infty}\frac{1}{k^{p}}$ при $p > 1$. Тогда
    \[a_{n} = a_{1}\cdot \exp\left(-A \ln n + O(1)\right) = a_{1} \frac{e^{O(1)}}{n^{A}}.\]
    
    Теперь утверждение следует по признаку сравнения с рядом $\sum_{n = 1}^{+\infty}\frac{1}{n^{A}}$.
\end{proof}

\subsection{Ряды с произвольными членами}

Вернемся к изучению рядов с произвольными (в общем случае комплексными) членами.

\begin{definition}
    Ряд $\sum_{n = 1}^{+\infty}a_{n}$ называется \textit{абсолютно сходящимся}, если сходится ряд $\sum_{n = 1}^{+\infty}|a_{n}|$.

    Если ряд $\sum_{n = 1}^{+\infty}a_{n}$ сходится, но не сходится абсолютно, то он называется \textit{условно сходящимся}. 
\end{definition}

\begin{corollary}
    Абсолютно сходящийся ряд сходится.
\end{corollary}

\begin{proof}
    Для любых $m, n \in \N, m \leq n$,
    \[\left|\sum_{k = m}^{n} a_{k}\right| \leq \sum_{k = m}^{n}|a_{k}|.\]
    Поэтому, если ряд $\sum_{n = 1}^{+\infty} |a_{n}|$ удовлетворяет условию Коши, то условию Коши удовлетворяет ряд $\sum_{n = 1}^{+\infty}a_{n}$. 
\end{proof}

\begin{note}
    Если ряд $\sum_{n = 1}^{+\infty}a_{n}$ сходится абсолютно, то
    \[\left|\sum_{k = 1}^{+\infty} a_{k}\right| \leq \sum_{k = 1}^{+\infty}|a_{k}|.\]
\end{note}

\begin{lemma}
    \begin{enumerate}
        \item Если $\sum_{n = 1}^{+\infty} b_{n}$ сходится, то $\sum_{n = 1}^{+\infty} (a_{n} + b_{n})$ и $\sum_{n = 1}^{+\infty} a_{n}$ сходятся или расходятся одновременно.
        \item Если $\sum_{n = 1}^{+\infty} b_{n}$ абсолютно сходится, то $\sum_{n = 1}^{+\infty} (a_{n} + b_{n})$ и $\sum_{n = 1}^{+\infty} a_{n}$ либо одновременно расходятся, либо одновременно сходятся условно, либо одновременно сходятся абсолютно.
    \end{enumerate}
\end{lemma}

\begin{proof}~

    \begin{enumerate}
        \item Следует из свойства линейности. Для всех $n \in \N$ верно
        \[|a_{n} + b_{n}| \leq |a_{n}| + |b_{n}|, \ |a_{n}| \leq |a_{n} + b_{n}| + |b_{n}|.\]
        Следовательно, по признаку сравнения ряды $\sum_{n = 1}^{+\infty} (a_{n} + b_{n})$ и $\sum_{n = 1}^{+\infty} a_{n}$ одновременно абсолютно сходятся.
        
        \item Вытекает из пункта 1.
    \end{enumerate}
    
\end{proof}

\begin{theorem}[признак Дирихле]
    Пусть $\{a_{n}\}$ -- комплексная последовательность, $\{b_{n}\}$ -- действительная последовательность, причем
    \begin{enumerate}
        \item Последовательность $A_{N} = \sum_{n = 1}^{N} a_{n}$ ограничена,
        \item $\{b_{n}\}$ монотонна,
        \item $\lim_{n \to +\infty} b_{n} = 0$.
    \end{enumerate}
    Тогда ряд $\sum_{n = 1}^{+\infty} a_{n}b_{n}$ сходится.
\end{theorem}

\begin{problem}
    Доказать признак Дирихле.
\end{problem}

В следующих параграфах признак будет доказан в общности. Это относится и к следующему утверждению, которое можно сформулировать как следствие признака Дирихле.

\begin{theorem}[признак Абеля]
    Пусть $\{a_{n}\}$ -- комплексная последовательность, $\{b_{n}\}$ -- действительная последовательность, причем
    \begin{enumerate}
        \item Ряд $\sum_{n = 1}^{+\infty} a_{n}$ сходится,
        \item $\{b_{n}\}$ монотонна,
        \item $\{b_{n}\}$ ограничена.
    \end{enumerate}
    Тогда ряд $\sum_{n = 1}^{+\infty} a_{n}b_{n}$ сходится.
\end{theorem}

Полезно для практики выделить частные случаи признака Дирихле, в которых ограниченность частичных сумм (условие 1) выполняется автоматически.

\begin{corollary}[признак Лейбница]
    Пусть последовательность $\{\alpha_{n}\}$ монотонна и $\alpha_{n} \to 0$. Тогда ряд $\sum_{n = 1}^{+\infty} (-1)^{n - 1} \alpha_{n}$ сходится, причем
    \[|S - S_{n}| \leq |\alpha_{n + 1}|.\]
\end{corollary}

\begin{proof}
    Сходимость вытекает из признака Дирихле. Докажем ее прямо. Пусть для определенности $\{\alpha_{n}\}$ нестрого убывает, и, значит, все $\{\alpha_{n}\} \geq 0$.
    
    $S_{2n + 2} - S_{2n} = \alpha_{2n + 1} - \alpha_{2n + 2} \geq 0 \Rightarrow \{S_{2n}\}$ нестрого возрастает.
    
    $S_{2n + 1} - S_{2n - 1} = -\alpha_{2n} + \alpha_{2n + 1} \leq 0 \Rightarrow \{S_{2n - 1}\}$ нестрого убывает.
    
    Кроме того, $S_{2n} - S_{2n - 1} = - \alpha_{2n} \leq 0$. Поэтому для любых $m, n \in \N$ имеем
    \[S_{2n} \leq S_{2k} \leq S_{2k - 1} \leq S_{2m - 1},\]
    
    где $k = \max\{m, n\}$. Следовательно, последовательности $\{S_{2n}\}$ и $\{S_{2n - 1}\}$ сходятся, $S_{2n} \to S'$, $S_{2n - 1} \to S''$, и, в частности,
    \[S_{2n} \leq S' \leq S'' \leq S_{2n - 1}.\]
    
    Поскольку $S_{2n} - S_{2n - 1} = -\alpha_{2n} \to 0$, то $S' = S'' = S$.
\end{proof}

\begin{corollary}
    Пусть $\{\alpha_{n}\}$ монотонна и $\alpha_{n} \to 0, \ x \neq 2\pi m, \ m \in \Z$. Тогда ряды
    $\sum_{n = 1}^{+ \infty} \alpha_{n}\cos(nx)$ и $\sum_{n = 1}^{+ \infty} \alpha_{n}\sin(nx)$ сходятся.
\end{corollary}

\begin{proof}
    Положим $s_{N} = \sum_{n = 1}^{N} e^{inx}$. По формуле суммы геометрической прогрессии с $q = e^{ix}$ имеем 
    \[s_{N} = \frac{e^{ix}(1 - e^{iNx})}{1 - e^{ix}}.\]
    
    Поэтому, так как $|e^{ikx}| = 1$, $|s_{N}| \leq \frac{2}{\sqrt{(1 - \cos(x))^{2} + \sin^2(x)}} = \frac{2}{\sqrt{2 - 2\cos(x)}} = \frac{1}{|\sin(\frac{x}{2})|}$.
    
    Ограниченность сумм $C_{N} = \sum_{n = 1}^{N} \cos(nx)$ и $S_{N} = \sum_{n = 1}^{N} \sin(nx)$ следует из ограниченности $\{s_{N}\}$ и равенств $C_{N} = Re(s_{N})$, $S_{N} = Im(s_{N})$.
    
    Сходимость указанных рядов теперь следует из признака Дирихле.
\end{proof}

\begin{example}
    Найти сумму ряда $\sum_{n = 1}^{+\infty} \frac{(-1)^{n - 1}}{n}$.
\end{example}

\begin{solution}
    \[S_{2m} = 1 - \frac{1}{2} + \frac{1}{3} - \ldots + \frac{1}{2m - 1} - \frac{1}{2m} = 1 + \frac{1}{2} + \ldots + \frac{1}{2m - 1} + \frac{1}{2m} - 2\left(\frac{1}{2} + \frac{1}{4} + \ldots + \frac{1}{2m}\right) = \]
    \[= H_{2m} - H_{m} = (\ln 2m + \gamma + o(1)) - (\ln m + \gamma + o(1)) = \ln 2 + o(1), \ m \to +\infty.\]
    Значит искомая сумма равна $\ln 2$.
\end{solution}

\subsection{Перестановки рядов}

\begin{definition}
    Пусть дан ряд $\sum_{n = 1}^{+\infty} a_{n}$ и биекция $\phi: \N \to \N$. Тогда $\sum_{n = 1}^{+\infty}a_{\phi(n)}$ называется \textit{перестановкой} ряда $\sum_{n = 1}^{+\infty} a_{n}$.
\end{definition}

\begin{example}
    Рассмотрим следующую перестановку ряда $\sum_{n = 1}^{+\infty} \frac{(-1)^{n - 1}}{n}$:
    \[1, - \frac{1}{2}, -\frac{1}{4}, + \frac{1}{3}, - \frac{1}{6}, - \frac{1}{8}, + \frac{1}{5}, - \frac{1}{10}, \ldots\]
    Найдем сумму этой перестановки
    \[S_{p} = (1 - \frac{1}{2}) -\frac{1}{4} + (\frac{1}{3} - \frac{1}{6}) - \frac{1}{8} + (\frac{1}{5} - \frac{1}{10}) + \ldots = \frac{1}{2}(1 - \frac{1}{2} + \frac{1}{3} - \frac{1}{4} + \ldots) = \frac{1}{2} \ln 2.\]
    Заметим, что мы корректно нашли сумму, однако она отличается от ответа прошлой задачи. Это следует из условной сходимости ряда.
\end{example}

\begin{theorem}
    \label{convergence-9}
    Если ряд $\sum_{n = 1}^{+\infty} a_{n}$ сходится абсолютно, то любая его перестановка $\sum_{n = 1}^{+\infty} a_{\phi(n)}$ сходится абсолютно, причем к той же сумме.
\end{theorem}

\begin{proof}
    Абсолютная сходимость перестановки следует из оценки
    \[\sum_{n = 1}^{N}|a_{\phi(n)}| \leq \sum_{n = 1}^{\underset{1 \leq k \leq N}{\max\{\phi(k)\}}} |a_{n}| \leq \sum_{n = 1}^{+\infty}|a_{n}| < +\infty.\]
    Пусть $\epsilon > 0$. Выберем номер $m$ так, что $\sum_{n = m + 1}^{+\infty}|a_{n}| < \epsilon$. Выберем $M$ так, что $\{1, \ldots, m\} \subset \{\phi(1), \ldots, \phi(M)\}$ (достаточно положить $M = \max_{1 \leq j \leq m}\phi^{-1}(j)$). Тогда для любого $N \geq M$ имеем $\{1, \ldots, m\} \subset \{\phi(1), \ldots, \phi(N)\}$ и $\left|\sum_{n = 1}^{+\infty}a_{n} - \sum_{n = 1}^{N} a_{\phi(n)}\right| \leq \sum_{n = m + 1}^{+\infty}|a_{n}| < \epsilon$.
    Таким образом, частичные суммы перестановки сходятся у сумме исходного ряда.
\end{proof}

\begin{problem}[Теорема Римана]
    Если ряд с действительными членами $\sum_{n = 1}^{+\infty} a_{n}$ сходится условно, то для любого $L \in \overline{\R}$ существует такая перестановка $\sum_{n = 1}^{+\infty} a_{\phi(n)}$, что её сумма равна $L$.
\end{problem}