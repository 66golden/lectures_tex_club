\documentclass[../main.tex]{subfiles}

\begin{document}
 
 \begin{corollary}
   $ f_k(x) \cancel{\convergesuniformly{k \to \infty}{X}} f(x)$ \nas $ \exists \{x_{n}\}_{k=1}^{ \infty} : f_k(x_k ) - f(x_k ) \cancel{  \xrightarrow[k \to  \infty ]{} } 0$ 
 \end{corollary}

 \begin{proof}[\circled{\lhence}]
   $ \sup_{x \in X } \left| f_k (x) - f(x)   \right| \geq \left| f_k(x_k ) - f(x_k )  \right| \cancel{\xrightarrow[k \to  \infty ]{}} 0 $. По критерию равномерной сходимости нет. 
 \end{proof}

 \begin{proof}[\circled{\hence}]
   Пусть $ M_k = \sup_{x \in X} \left| f_k(x)  - f(x)  \right| \in \R \cup \{ + \infty \} $. Тогда рассмотрим $ x_k $ такую, что 
   \begin{itemize}
     \item если $ M_k = + \infty$ $ \left| f_k(x_k ) - f(x_k ) \right| > 1$;
     \item если $ M_k \in \R$ $ \left| f_k (x_k ) - f(x_k ) \right| > M_k - \frac{1}{k}$ (всегда такой найдётся).  
   \end{itemize}
   Предположим, что для заданной последовательности $ f_k (x_k ) - f(x_k )  \xrightarrow[k \to  \infty ]{} 0 $. Тогда переходя к пределу в 
   \begin{gather} 
      0 \leq M_k \leq \left| f_k (x_k) - f(x_k )\right| - \frac{1}{k}
   \end{gather} 
   получаем, что $ M_k  \xrightarrow[k \to  \infty ]{} 0 $, противоречие с $ f_k(x) \cancel{\convergesuniformly{k \to \infty}{X}} f(x)$. 
 \end{proof}

 
 \begin{definition} \label{def:funcseq:limited}
   Будем говорить, что $ \{ f_{k}(x) \}_{k = 1}^{\infty} $ \emph{равномерно ограничена} на $ X$, если 
   \begin{gather} 
     \exists C > 0: \forall k \in N \forall x \in X \hence \left| f_k(x)  \right| \leq C .
   \end{gather}
 \end{definition}
 
 
 \begin{proposition} \label{prop:funcseq:limdotconvtozero}
   Пусть $ \{ f_{k}(x) \}_{k = 1}^{\infty} $ равномерно ограничена на $ X$, $ g_k(x) \convergesuniformly{k \to \infty}{X} 0$ тогда $ f_k(x) g_k(x)  \convergesuniformly{k \to \infty}{X} 0$.
 \end{proposition}
 
 
 \begin{proof}
   Рассмотрим поточечную сходимость. При фиксированном $ x_0 \in X$ получаем, что $ f_k (x_0 ) g_k (x_0 )  \xrightarrow[k \to  0 ]{} 0 $ (как ограниченная на бесконечно малую). Значит поточечно сходится к $ 0$. Рассмотрим 
   \begin{gather} 
     \sup_{x \in X} \left| f_k(x)  g_k(x)  \right| \leq \sup_{x\in X} C \left| g_k(x)  \right|   = C \sup_{x \in X} \left| g_k(x)  \right|  \xrightarrow[k \to  \infty ]{} 0.
   \end{gather} 
   Значит, по критерию, она сходится равномерно к $ 0$. 
 \end{proof}

 
 \begin{proposition}[Критерий Коши]
   $ f_k(x)  \convergesuniformly{k \to \infty}{X} f(x)$ \nas 
   \begin{gather} 
    \forall \eps > 0 \exists N: \forall n \geq N \forall p \in \N \forall x \in X \hence \left| f_n(x) - f_{n+p}(x) \right| < \eps .
   \end{gather}
 \end{proposition}
 
 
 \begin{proof}[\circled{\hence}]
  Из равномерной сходимости 
  \begin{gather} 
    \forall \eps > 0 \exists N : \forall n \geq N \forall x \in X \hence \System{\left| f_n(x) - f(x) \right| < \frac{\eps}{2} \\ \forall p \in \N \left| f_{n+p}(x) - f(x) \right| < \frac{\eps}{2}} \hence \left| f_n(x) - f_{n+p}(x) \right| < \eps.
  \end{gather} 
 \end{proof}
 
 \begin{proof}[\circled{\lhence}]
   Зафиксируем $ x_0 \in X$. Из условия 
   \begin{gather} 
     \forall \eps > 0 \exists N \in N: \forall n \geq N \forall p \in N \hence \left| f_n(x_0 ) - f_{n+p}(x_0) \right| < \eps .
   \end{gather}
   Получили критерий Коши для числовых последовательностей. Тогда 
   \begin{gather} 
     \exists f(x_0 ) : \lim_{k \to \infty} f_k(x_0) = f(x_0 ) \in \R .
   \end{gather}
   Получили поточечную сходимость к $ f(x)$. \\
   Переходя к пределу в условии при $ p \to \infty$, а затем к $ \sup$ получаем 
   \begin{gather} 
     \forall \eps > 0 \exists N \in \N : \forall n \geq N \sup_{x\in X} \left| f_k(x) - f(x) \right| \leq \eps .
   \end{gather}  
   По критерию получаем, что $ f_k(x)  \convergesuniformly{k \to \infty}{X} f(x)$.
 \end{proof}
 
 \subsection{Свойства предельной функции}

 
 \begin{proposition} \label{prop:funcseq:cont}
   Пусть $ f_k: X \to \R$, $ \forall k \in N$ $ f_k $ непрерывна на $ X$ и $ f_k(x) \convergesuniformly{k \to \infty}{X} f(x) $. Тогда $ f(x)$ непрерывна на $ X$. 
 \end{proposition}

 
 \begin{proof}
   Зафиксируем $ x_0 \in X$. Рассмотрим 
   \begin{multline} \label{proof:funcser:cont}
     \left| f(x) - f(x_0 ) \right| = \left| f(x) - f_k(x)  + f_k(x) - f_k (x_0 ) + f_k (x_0 ) - f(x_0) \right| \leq \\ \leq \left| f(x) - f_k(x) \right| + \left| f_k(x) - f_k (x_0 ) \right| + \left| f_k (x_0 ) - f(x_0) \right|   .
   \end{multline}
   Первый и третий модуль оцениваются из условия равномерной непрерывности: 
   \begin{gather} 
     \forall \eps > 0 \exists N: \forall k \geq N \forall x \in X \hence \left| f_k(x)  - f(x)  \right| < \frac{\eps}{3}.
   \end{gather}
   Второй модуль оценим из условия непрерывности $ f_k $ (пусть $ k = N$, мы же сами выбираем что нам прибавлять и вычитать): 
   \begin{gather} 
     \forall \eps > 0 \exists \delta: \forall x \in U_{\delta}(x_0 ) \cap X \hence \left| f_k(x) - kf(x_0 ) \right| < \frac{\eps}{3} .
   \end{gather}
   С учётом \eqref{proof:funcser:cont} получаем условие непрерывности на множестве: 
   \begin{gather} 
    \forall x_0 \in X \forall \eps > 0 \exists \delta > 0 : \forall x \in U_{\delta}(x_0) \cap X \hence \left| f(x) - f(x_0) \right| < \eps .
   \end{gather}
   Напомним, что по $ \eps$ из условия равномерной непрерывности выбираем $ N$, и $ k = N$.
 \end{proof}
 
 
 \begin{proposition} \label{prop:funcseq:changeint}
   Пусть $ f_k : [a, b ] \to \R, \forall k\in \N$, $ f_k $~--- непрерывна на $ [a,b]$ и $ f_k(x) \convergesuniformly{k \to \infty}{[a,b]} f(x)$, тогда 
   \begin{gather} 
     \lim_{k \to \infty} \int_{a}^{b} f_k(x) dx = \int_{a}^{b} \left(  \lim_{k \to \infty} f_k(x)  \right) dx.
   \end{gather}    
 \end{proposition}
 
 
 \begin{proof}
   По предыдущему утверждению $ f(x)$ непрерывна. Пусть начиная с некоторого $ k = k_0$ $ M_k = \sup_{x\in X} \left| f_k(x)  - f(x)  \right| \in \R$. Рассмотрим 
   \begin{multline} 
     \left| \int_{a}^{b} f_k(x) dx - \int_{a}^{b} f(x) dx \right|  = \left| \int_{a}^{b} \left( f_k(x) - f(x) \right) \right| \leq \\ \leq \int_{a}^{b} \left| f_k(x) - f(x)  \right| dx \leq \int_{a}^{b} M_k dx  = M_k (b-a)   \xrightarrow[k \to  \infty ]{} 0 .
   \end{multline}
 \end{proof}

 \begin{example}
   Посмотрим, что может быть при наличии только поточечной сходимости (а не равномерной). Рассмотрим $ f_k(x)  $: 
   \begin{gather} 
     f_k(x) = \System{4k^{2}x, &x \in [0, \frac{1}{2k})\\ -4k^{2}(x-\frac{1}{k}), &x \in [\frac{1}{2k}, \frac{1}{k}] \\ 0, &x \in [\frac{1}{k}, 1]} .
   \end{gather} 
   Заметим, что $ \forall k \hence \int_{0}^{1} f_k(x) dx = 1$. \\
   Зафиксируем $ x_0 \in [0, 1]$, тогда $ f_k (x_0 )  \xrightarrow[k \to  \infty ]{} 0 $. Другими словами, $ f_k(x)  \xrightarrow[k \to  \infty ]{[0;1]} 0 $ . Тогда $ \lim_{k \to \infty} \left(\int_{0}^{1} f_k(x) dx\right) = 1$. Но $ \int_{0}^{1} \left(\lim_{k \to \infty} f_k(x) \right)dx = 0 \neq 1$.    
 \end{example}

 %здесь бы рисунок TODO
 
 
 \begin{proposition} \label{prop:funcseq:changeder}
   Пусть $ f_k : [a,b] \to \R$, $ \forall k \in \N$ $ f_k $ непрерывно дифференцируема на $ [a, b]$, $ \exists x_0 \in [a,b] : f_k (x_0 )  \xrightarrow[k \to  \infty ]{} A \in \R$ и $ f'_k (x) \convergesuniformly{k \to \infty}{[a,b]} \phi(x)$. Тогда 
   \begin{gather} 
     \left(\lim_{k \to \infty} f_k(x)  \right)' = \lim_{k \to \infty} f_k'(x) .
   \end{gather}    
 \end{proposition}
 
 
 \begin{proof}
   По одному из прошлых утверждений $ \phi(x)$ непрерывна на \segab. Пусть $ f_k(x)  = f_k (x_0 ) + \int_{x_0 }^{x} f_k ' (t) dt$, $ f(x) = A + \int_{x_0 }^{x} \phi (t) dt$. Рассмотрим 
   \begin{multline} 
     \left| f_k(x) - f(x)  \right| = \left| \left(f_k (x_0) - A\right) + \int_{x_0 }^{x} (f_k '(t) - \phi(t))dt\right|  \leq \\ \leq \left|f_k (x_0) - A\right|+ \left|\int_{x_0 }^{x} (f_k '(t) - \phi(t))dt\right| \leq \left| f_k (x_0) - A \right|  \leq \int_{x_0 }^{x} \left| f_k ' (t) - \phi (t) \right| dt \leq \\ \leq \left| f_k (x_0) \right|  + \int_{x_0 }^{x} \sup_{t \in [a, b]} \left| f_k '(t) - \phi (t) \right| dt = \left| f_k (x_0) - A \right| + \sup_{t \in [a, b]} \left| f_k'(t) - \phi(t) \right|  \left| x - x_0  \right| \leq \\ \leq \left| f_k (x_0) - A \right| + \sup_{t\in[a,b]} \left| f_k '(t) - \phi(t) \right| \left| b - a \right|  \xrightarrow[k \to  \infty ]{} 0 .
   \end{multline} 

   При рассмотрении супремума все рассуждения аналогичны, значит $ f_k(x)  \convergesuniformly{k \to \infty}{[a,b]} f(x) $. Тогда 
   \begin{gather} 
     \left(\lim_{k \to \infty} f_k(x) \right)' = (f(x))' = f'(x) = \phi(x) = \lim_{k \to \infty} f'_k (x) .
   \end{gather}  

 \end{proof}
 
\end{document}