\documentclass[../main.tex]{subfiles}
\begin{document}
\section{Теория меры}

\begin{definition}
    Рассмотрим множество $\Pi \subset \Rn$, $\{a_k\}_{k=1}^n$, $\{b_k\}_{k=1}^n$, $\forall k\ a+k \leq b_k$. Тогда $\Pi = \{ (x_1, \dots, x_n) \in \Rn \such a_k \leq x_k \leq b_k \}$ называется \emph{клеткой (блоком)}. 
\end{definition}

\begin{example}
    В одномерном случае клетка может представлять отрезок, в том числе и точку (если $a_k = b_k$). В двумерном случае~--- прямоугольник (отрезок, точку). В трёхмерном~--- в общем случае параллелепипед.
\end{example}

\begin{note}
    В некоторых книжках могут встречаться определения с другими неравенствами (каждое из них может быть как строгим, так и не строгим). Мы дальше будем пользоваться определение с нестрогими неравенствами.
\end{note}

\begin{definition}
    \emph{Мерой клетки} называется $\mu (\Pi) = \prod_{k=1}^n (b_k - a_k)$.
\end{definition}

\begin{definition}
    Подмножество $A \subset \Rn$ называется \emph{элементарным (клеточным, блочным)}, если $\exists \{\Pi_i\}_{i=1}^k: \forall i,j\ i \neq j \Int \Pi_i \cap \Int \Pi_j = \emptyset,\ A = \bigcup_{i=1}^k \Pi_i$.
    Другими словами, если множество может быть представлено как конечное объединение клеток без общих внутренних точек.
\end{definition}

\begin{definition}
    \emph{Мерой клеточного множества} называется $\mu(A) = \sum_{i=1}^k \mu(\Pi_i)$.
\end{definition}

\begin{proposition}
    Пускай $A, B \subset \Rn$~--- элементарные множества. Тогда выполняется
    \begin{enumerate}
        \item $\mu(A)$ определено корректно:
        \begin{gather*}
            A = \bigcup_{i = 1}^k \Pi_i = \bigcup_{j=1}^m \hence \sum_{i=1}^k \mu(\Pi_i) = \sum_{j=1}^m \mu(\Pi_j).
        \end{gather*}
        \begin{proof}
            Рассмотреть разбиение на клетки $\forall i, j\ \Pi_i \cap \Pi_j$.
        \end{proof}
        \item $A \cup B, A \cap B, A \setminus \Int B$~--- элементарные множества.
        \begin{note}
            $A \setminus B$ в общем случае не будет клеточным множеством, ведь клетка~--- это замкнутое множество, значит и элементарное множество замкнуто, а $A \setminus B$ может и не быть замкнутым.
        \end{note}
        \item $\Int A \cap \Int B = \emptyset \hence \mu(A \cup B) = \mu(A) + \mu(B)$, иначе $\mu(A\cup B) \leq \mu(A) + \mu(B)$. В общем случае $ \mu(A\cup B) = \mu(A) + \mu(B) - \mu (A\cap B)$.
        \item $A \subset B \hence \mu(A) \leq \mu(B)$.
    \end{enumerate}
\end{proposition}

\begin{definition}
    Пусть $X \subset \Rn$. \\ \emph{Верхней мерой Жордана} будем называть $\mu^*(X) = \Inf \{\mu(B) \such B~\text{--- элементарное}, X\subset B\}$. \emph{Нижней мерой Жордана} будем называть $\mu_*(X) = \Sup \{\mu(A) \such A~\text{--- элементарное}, A\subset X\}$.
\end{definition}

\begin{reminder}
    Клеточное множество~--- объединение \emph{конечного} числа клеток.
\end{reminder}

\begin{proposition}
    $\mu_*{X} \leq \mu^*{X}$.
\end{proposition}

\begin{definition}
    Если $\mu_*{X} \leq \mu^*{X}$, то $X$~--- \emph{измеримо по Жордану}.
    \emph{Мерой Жордана} будем называть $\mu(X) = \mu_*(X) = \mu^*(X)$.
\end{definition}

\begin{note}
    Меру мы вводим как функцию на каком-то семействе множеств. Поскольку это функция, мы хотим в итоге получать какие-то числа, поэтому на $X$ накладывается требование измеримости.
\end{note}

\begin{proposition}
    $X$~--- измеримо \hence $X$~--- ограничено.
\end{proposition}

\begin{note}
    Не все множества измеримы по Жордану.
\end{note}

\begin{example}
    $X = [0,1] \cap \Q$. Тогда, очевидно, $\mu^*(X) = 1$ (в любом случае мы обязаны покрыть весь отрезок). Однако $\mu_*(X) = 0$ (какой бы мы отрезок $\subset [0,1]$ не взяли, там всегда будут точки $\in \Q$). $\mu^*(X) \neq \nu_*(X)$ \hence \emph{$X$ не измеримо по Жордану}.
\end{example}

\begin{note}
    В дальнейшем будет удобный критерий измеримости множества, с помощью которого можно будет строго доказать предыдущий пример.
\end{note}

\begin{proposition}
    Пусть $X \subset \Rn$~--- измеримо по Жордану тогда и только тогда, когда $$\forall \eps > 0\ \exists A_\eps, B_\eps~\text{--- клеточные}: A_\eps \subset X \subset B_\eps,\ \mu(B_\eps) - \mu(A_\eps) < \eps.$$
\end{proposition}

\begin{proof}[\circled{$\hence$}] 
    По определению $\mu_*(X) = \Sup \{\mu(A) \such A~\text{--- клет.}, A\subset X\}$, из определения супремума:
    $$\forall \eps > 0\ \exists A_\eps\cell \subset X: \mu(X) - \mu(A_\eps) < \frac{\eps}{2}.$$
    Аналогично для $\mu^*(X) = \Sup \{\mu(B) \such B~\text{--- клет.}, B\supset X\}$, из определения инфинума:
    $$\forall \eps > 0\ \exists B_\eps\cell \supset X: \mu(B_\eps) - \mu(X) < \frac{\eps}{2}.$$
    Сложим получившиеся неравенства:
    $$\mu (B_\eps) - \mu(A_\eps) < \eps.$$
\end{proof}

\begin{proof}[\circled{$\lhence$}]
    $\forall \eps > 0$ как супремум $\mu_*(X) \geq \mu(A_\eps)$, как инфинум $\mu^*(X) \leq \mu(B_\eps)$, складывая второе с первым домноженным на $-1$:
    $$\mu^*(X) - \mu_*(X) \leq \mu(B_\eps) - \mu(A_\eps) < \eps.$$
    Теперь собираем:
    $$\forall \eps>0 \mu_*(X) \leq \mu^*(X) < \mu_*(X) + \eps,$$
    при $\eps \to 0$:
    $$ \mu_*(X) \leq \mu^*(X) \leq \mu_*(X) \hence \mu(X) = \mu^*(X) = \mu_*(X) \hence X~\text{--- измеримо}.$$
\end{proof}

\begin{definition}
    \emph{$\delta$-окрестность множества} $U_\delta(X) = \{x\in\Rn\such \Inf_{y\in X} |x-y| < \delta\}$.
\end{definition}

\begin{note}
    $\Inf_{y\in X} |x-y|$ можно определять как расстояние от точки $x$ до множества $X$.
\end{note}

\begin{proposition}
    $X$~--- измеримо \hence $\mu^*(U_\delta(X)) \to_{\delta \to +0} \mu(X)$.
\end{proposition}

\begin{proof}
    Пусть $X = \{x\in\Rn\such a_i \leq x \leq b_i\}$~--- клетка. Тогда $\mu(X) = \prod_{i=1}^n (b_i-a_i)$, $U_\delta (X) \subset \{x\in\Rn\such a_i - \delta \leq x \leq b_i + \delta\}$. Значит 
    $$\mu(X) \leq \mu^*(U_\delta(X)) \leq \prod_{i=1}^n(b_i-a_i - 2\delta) \underset{\delta \to +0}{\to} \prod_{i=1}^n(b_i-a_i) = \mu(X).$$
    Пусть теперь $X = \bigcup_{i=1}^k \Pi_i$~--- клеточное множество. Тогда $U_\delta(X) = \bigcup_{i=1}^k U_{\delta}(\Pi_i)$ и 
    $$\mu(X) \leq \mu^*(U_\delta(X)) = \mu^*\left(\bigcup_{i=1}^k U_\delta (\Pi_i)\right) \leq \sum_{i=1}^k \mu^*(U_\delta (\Pi_i)) \underset{\delta \to +0}{\to} \sum_{i=1}^k \mu(\Pi_i) = \mu(X).$$
    Как следствие:
    $$\forall \eps > 0\ \exists \delta_0:\ \forall \delta < \delta_0\ \left|\sum_{i=1}^k \mu^*(U_\delta (\Pi_i)) - \mu(x)\right| < \frac{\eps}{2}.$$
    Пусть теперь $X$~--- измеримое множество, $\forall \eps > 0\ X \subset B_\eps = \bigcup_{i=1}^k \Pi^\eps_i:\ \mu(B_\eps) - \mu(X) < \eps/2$:
    $$\forall \eps > 0\ \exists \delta_0:\ \forall \delta < \delta_0\ \mu(X) \leq \mu^*(U_\delta(X)) \leq \mu(U_\delta(B_\eps)) \leq \mu(B_\eps) + \frac{\eps}{2} < \mu(X) + \eps \underset{\eps \to 0}{\to} \mu(X).$$
\end{proof}

\begin{proposition}
    Пусть $X, \Pi \subset \Rn$, тогда из $X \cap \Pi \neq \emptyset$ и $(\Rn\setminus X)\cap\Pi \neq \emptyset$ следует, что $\Pi \cap \partial X \neq \emptyset$.
\end{proposition}

\begin{proof}
    Пусть $a_1 \in X \cap \Pi$, $b_1 \in (\Rn\setminus X)\cap\Pi$. Рассмотрим точку $c_1 = (a_1+b_1)/2$. 
    Если $c_1 \in X$, то $a_2 = c_1,\ b_2 = b_1$.
    Если $c_1 \not\in X$, то $a_2 = a_1,\ b_2 = c_1$. Таким образом строим систему вложенных стягивающихся отрезков $\{[a_k,b_k]\}$ \hence $!\exists c \in \bigcap_k [a_k,b_k]$. Значит:
    $$ \forall U_\eps(c) \cap X \neq \emptyset, U_\eps(c) \cap (\Rn \setminus X) \neq \emptyset \hence c \in \partial X.$$
\end{proof}

\begin{theorem}[Критерий измеримости]
    $X$~--- измеримо \nas $X$~--- ограничено и $\mu(\partial X) = 0$.
\end{theorem}

\begin{proof}[\circled{\hence}]
    Ограниченность сразу следует из измеримости. 
    $$\forall \eps > 0\ \exists A_\eps, B_\eps\cell:\ A_\eps \subset X \subset B_\eps, \mu(B_\eps) - \mu(A_\eps) < \eps.$$
    $\bar X \subset \bar B_\eps = B_\eps$, $\Int A_\eps \subset \Int X$ \hence $\partial X = \bar X \setminus \Int X \subset B_\eps \setminus \Int A_\eps$ \hence $$\mu^*(\partial X) \leq \mu^*(B_2 \setminus \Int A_\eps).$$ Также мы знаем, что $A_\eps \cup (B_\eps \setminus \Int A_\eps) = B_\eps$ \hence $\mu(B_\eps) = \mu(A_\eps) + \mu(B_\eps \setminus \Int A_\eps)$. В итоге:
    $$0 \leq \mu^*(\partial X) \leq \mu^*(B_2 \setminus \Int A_\eps) = \mu(B_\eps) - \mu(A_\eps) < \eps \to 0.$$
    Получаем, что $0\leq \mu_*(\partial X) \leq \mu^*(\partial X) \leq 0$, \hence $\mu(\partial X) = 0$.
\end{proof}

\begin{proof}[\circled{\lhence}]
    $X$~--- ограничено \hence $\exists \Pi$~--- клетка: $X \subset \Pi$. Из $\mu(\partial X) = 0$ и определения инфинума получаем, что $\forall \eps > 0\ \exists C_\eps\cell: \partial X \subset C_\eps,\ \mu(C_\eps) < \eps$. Пусть $A_\eps = X \setminus C_\eps\cell$, $B_\eps = A_\eps \cup C_\eps$. По построению $A_\eps \subset X \subset B_\eps$. Рассмотрим $\mu(B_\eps) \leq \mu(A_\eps) + \mu(C_\eps) < \mu(A_\eps) + \eps$. Получаем $\mu(B_\eps) - \mu(A_\eps) < \eps$ \hence $X$~--- измеримо.
\end{proof}

\begin{proposition}
    Пусть $X_1, X_2 \subset \Rn$, тогда
    \begin{enumerate}
        \item $\partial(X_1 \cup X_2) \subset \partial(X_1) \cup \partial(X_2)$,
        \item $\partial(X_1 \cap X_2) \subset \partial(X_1) \cap \partial(X_2)$,
        \item $\partial(X_1 \setminus X_2) \subset \partial(X_1) \setminus \partial(X_2)$.
    \end{enumerate}
\end{proposition}

\begin{proof} Докажем только первое, остальные по аналогии. $x \in \partial(X_1 \cup X_2) = \bar{X_1 \cup X_2} \setminus \Int(X_1 \cup X_2) \hence x \notin \Int X_1 \land x \notin \Int X_2$. Осталось доказать, что $x \in \bar X_1 \lor x \in \bar X_2$. Предположим противное, тогда $\exists \delta > 0:\ U_\delta(x) \cap \bar X_1 = \emptyset \land U_\delta(x) \cap \bar X_2 = \emptyset$. При этом мы знаем, что $x \in \bar{X_1 \cap X_2}$ \hence $\exists \{x_k\}_{k=1}^\infty:\ x_k \in X_1 \cup X_2,\ \lim_{k\to\infty} x_k = x$, другими словами $\forall \delta > 0\ \exists K:\ \forall k \geq K\ x_k \in U_\delta (x)$. Таким образом мы приходим к противоречию, то есть $x \in \bar X_1 \lor x \in \bar X_2$. С учётом $x \notin \Int X_1 \land x \notin \Int X_2$ получаем, что $x \in \bar X_1 \setminus \Int X_1 \lor x \in \bar X_2 \setminus \Int X_2$.
\end{proof}

\begin{proposition}
    Пусть $X_1, X_2 \subset \Rn$ измеримы $\hence$ $\bar X_1, \Int X_1, X_1 \cap X_2, X_1 \setminus X_2$ измеримы.
\end{proposition}

\begin{proof}
    Докажем только для объединения, остальные по аналогии. $\partial (X_1 \cup X_2) \subset \partial X_1 \cup \partial X_2$, значит 
    $$ 0 \leq \mu_*(\partial(X_1 \cup X_2)) \leq \mu^*(\partial(X_1 \cup X_2)) \leq \mu^*(\partial X_1 \cup \partial X_2) \leq \mu^*(\partial X_1) + \mu^*(\partial X_2) = 0.$$
    Значит $\mu(\partial(X_1 \cup X_2)) = 0$, по критерию измеримости $X_1 \cup X_2$ измеримо.
\end{proof}

\begin{note}
    Пусть $A, B\cell$. Тогда $A\setminus B$ может и не быть клеточным, но будет измеримым, причём $\mu(A\setminus B) = \mu(A \setminus \Int B)$.
\end{note}

\begin{example}
    $A \subset B \subset \Rn$, $\mu(B) = 0$ \hence $\mu(A) = 0 $. 
\end{example}

\begin{proposition}
    Пусть $\Gamma$~--- спрямляемая кривая из $\Rn, n\geq 2$ \hence $\mu(\Gamma) = 0$.
\end{proposition}

\begin{reminder}
    Спрямляемая кривая~--- кривая конечной длины.
\end{reminder}

\begin{proof}
    Пускай длина кривой $l = l(\Gamma)$. Рассмотрим набор точек $\{p_k\}_{k=0}^m$. Они делят кривую на кусочки равной длины $l/m$. Рассмотрим набор многомерных кубов $\{Q_k\}_{k=0}^m$ с центрами в точках $p_k$ соответственно и длинами рёбер $l/m$. Они полностью покрывают кривую $\Gamma$. Рассмотрим меру 
    $$\mu\left(\bigcup_k Q_k\right) \leq m \cdot \frac{l^n}{m^n} = \frac{l^n}{m^{n-1}} \underset{m \to \infty}{\to} 0.$$

    Под устремлением $m \to \infty$ имеется ввиду построение последовательности мер клеточных покрытий, полностью содержащих в себе $\Gamma$. Поскольку мы можем выделить из мер всех таких множеств сходящуюся к нулю последовательность, инфинум должен быть $\leq 0$, то есть $\mu^*(\Gamma) \leq 0$. . В силу $\mu_* (\Gamma)\geq 0$ получаем $\mu(\Gamma) = 0$.
\end{proof}

\begin{corollary}
    Граница круга~--- спрямляемая кривая, следовательно круг измерим.
\end{corollary}

\begin{proposition}
    Пусть $X \subset \Rn$~--- измеримо, $[a,b]\subset \R$, $G = X \times [a,b] \subset \R^{n+1}$ (где $\times$~--- это декартово произведение). Тогда $G$~--- измеримо, и $\mu(G) = \mu(X) \cdot (b-a)$. 
\end{proposition}

\begin{note}
    $G$~--- в некотором смысле <<цилиндр>>.
\end{note}

\begin{proof}
    $\forall \eps > 0\ \exists A_\eps, B_\eps \cell: A_\eps \subset X \subset B_\eps$ и
    \begin{gather*}
        \mu(A_\eps) \geq \mu(X) - \eps,\\
        \mu(B_\eps) \leq \mu(X) + \eps.
    \end{gather*}
     Рассмотрим $A_\eps \times [a,b]$, $B_\eps \times [a,b]$. Это~--- клеточные множества. При этом $A_\eps \times [a,b] \subset G \subset B_\eps \times [a,b]$. По определению 
     \begin{gather*}
        \mu(A_\eps \times [a,b]) = \mu(A_\eps) \cdot(b-a) \geq (\mu(X) - \eps) \cdot (b-a) ,\\
        \mu(B_\eps \times [a,b]) = \mu(B_\eps) \cdot(b-a) \geq (\mu(X) + \eps) \cdot (b-a) .
    \end{gather*}
    Как следствие, $\mu(A_\eps \times [a,b]) - \mu(B_\eps \times [a,b]) \leq 2 \eps (b-a)$. По критерию измеримости $G$ измеримо.
    \begin{gather*}
        (\mu(X) - \eps) \cdot (b-a) \leq \mu(A_\eps \times [a,b]) \leq \mu(G) \leq \mu(B_\eps \times [a,b]) \leq (\mu(X) + \eps) \cdot (b-a).  
    \end{gather*}
    При $\eps \to 0$ $\mu(G) \to \mu(X)(b-a)$.
\end{proof}

\begin{proposition}
    $X_1, X_2 \subset \Rn$~--- измеримы, $\Int X_1 \cap \Int X_2 = \emptyset$, тогда $\mu(X_1 \cup X_2) = \mu(X_1) + \mu(X_2)$.
\end{proposition}

\begin{proof}
    $\forall \eps>0, i \in \{1,2\}\ \exists A_i^\eps, B_i^\eps\cell:\ A_i^\eps \subset X_i \subset B_i^\eps$:
    \begin{gather*}
        \mu(A_i^\eps) \geq \mu(X) - \eps/2,\\
        \mu(B_i^\eps) \leq \mu(X) + \eps/2.
    \end{gather*}
    Заметим, что $\Int A_1^\eps \cap \Int A_2^\eps = \emptyset$ и
    \begin{gather*} 
        A_1^\eps \cup A_2^\eps \subset X_1 \cup X_2 \subset B_1^\eps \cup B_2^\eps,\\
        \mu(A_1^\eps \cup A_2^\eps) \leq \mu(X_1 \cup X_2) \leq \mu(B_1^\eps \cup B_2^\eps).
    \end{gather*}
    Так как $A_i^\eps$~--- клеточные множества, то $\mu(A_1^\eps \cup A_2^\eps) = \mu(A_1^\eps) + \mu(A_2^\eps) \geq \mu(X_1) + \mu(X_1) - \eps$. Оценим $\mu(B_1^\eps \cup B_2^\eps) \leq \mu(B_1^\eps) + \mu(B_2^\eps) \leq \mu(X_1) + \mu(X_1) + \eps$. При $\eps \to 0$ получаем $\mu(X_1 \cup X_2) = \mu(X_1) + \mu(X_2)$.
\end{proof}

\begin{note}
    Существенно, что мы уже доказали, что объединение измеримых множеств измеримо. Таким образом в это доказательство неявно зашит критерий измеримости множества, связанный с нулевой мерой границы.
\end{note}

\end{document}