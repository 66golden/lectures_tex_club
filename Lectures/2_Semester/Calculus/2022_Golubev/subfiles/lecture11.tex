\documentclass[../main.tex]{subfiles}

\begin{document}
\begin{proposition}
    Если $f, g$~--- интегрируемы на \segab, то $fg$~--- интегрируема на \segab.
\end{proposition}

\begin{proof}
    $f, g$~--- ограничены на \segab. 
    \begin{gather}
        \left| f(x_1)g(x_1)-f(x_2)g(x) \right| = \left| f(x_1)g(x_1)-f(x_1)g(x_2)+f(x_1)g(x_2)-f(x_2)g(x_2) \right| \leq \\ \leq \left| f(x_1) \right| \left| g(x_1) - g(x_2) \right| + \left| f(x_2) \right| \left| f(x_1)-f(x_2) \right| 
    \end{gather}
    так как $f, g$~--- ограничены, продолжим неравенство 
    \begin{gather} 
      \dots\leq M_f \left| g(x_1)-g(x_2) \right| +M_g \left| f(x_1) - f(x_2) \right|. 
    \end{gather}
    Рассмотрим
    \begin{multline} 
      \Delta(fg,T) = \sum_{k=1}^{m-1} (x_{k+1} - x_k) \omega_k(fg) \leq \\ \leq\sum_{k=1}^{m-1} (x_{k+1}-x_{k}) \left(M_f \sup_{x', x'' \in [x_{k+1},x_{k}]} \left| g(x') - g(x'') \right| + M_g \sup_{x', x'' \in [x_{k+1},x_{k}]} \left| f(x') - g(x'') \right|  \right) = \\ = \sum_{k=1}^{m-1} (x_{k+1}-x_{k}) \left(M_f \omega_k(g) + M_g \omega_k(f)\right) = \\ = M_f\sum_{k=1}^{m-1} \omega_k(g) (x_{k+1}-x_{k}) + M_g \sum_{k=1}^{m-1} \omega_k (f) (x_{k+1}-x_{k}) = M_f \Delta(f,T) + M_g \Delta(f, T)  \xrightarrow[l(T) \to  0 ]{} 0.
    \end{multline}
    Значит $f, g$~--- интегрируема.
\end{proof}

\begin{proposition}[Интегральная теорма о среднем]
    Пусть $f, g$~--- непрерывна на \segab, $g(x) \neq 0 \forall x \in \segab$. Тогда $\exists \xi$:
    \begin{gather} 
    \int_{a}^{b} f(x)g(x) dx = f(\xi) \int_{a}^{b} g(x)dx.
    \end{gather}
\end{proposition}

\begin{proof}
    Пусть 
    \begin{gather} 
      G(x) = \int_{a}^{x} g(t) dt, \\
      F(x) = \int_{a}^{x} f(t)g(t)dt 
    \end{gather}
    $F,G$~--- непрерывны на \segab, дифференцируемы на $(a,b)$, тогда по теореме Коши о среднем $\exists \xi$: 
    \begin{gather} 
      \frac{F(b) - F(a)}{G(b) - G(a)} = \frac{F'(\xi)}{G'(\xi)}. 
    \end{gather} 
    Переписывая эту формулу 
    \begin{gather} 
      \frac{\int_{a}^{b} f(x)g(x)dx}{\int_{a}^{b} g(x)dx} = \frac{f(\xi)g(\xi)}{g(\xi)}, \\
      \int_{a}^{b} f(x)g(x)dx = f(\xi) \int_{a}^{b} g(x) dx.
    \end{gather}
\end{proof}

\begin{proposition}
    Пусть $x(t)$ непрерывно дифференцируема на \segab, $f(x)$ непрерывна на $x\segab$, тогда 
    \begin{gather} 
      \int_{a}^{b} f(x(t)) d(x(t)) = \int_{x(a)}^{x(b)} f(x)dx.
    \end{gather}
\end{proposition}

\begin{proof}
    Рассмотрим 
    \begin{gather} 
      \int_{x(a)}^{x(b)} f(x)dx = F(x(b)) - F(x(a)).
    \end{gather}
    $F$ можно воспринимать как первообразную сложной функции. Тогда 
    \begin{gather} 
      F'_t = f(x(t)) x'(t)
    \end{gather} 
    ~--- непрерывна. По формуле Ньютона-Лейбница: 
    \begin{gather} 
      \int_{a}^{b} f(x(t)) x'(t) dt = \int_{a}^{b} d(x(t)) = F(x(b)) - F(x(a)).
    \end{gather}
\end{proof}

\begin{proposition}
    Пусть $f, g$ непрерывно дифференцируемы на \segab. Тогда 
    \begin{gather} 
      \int_{a}^{b} f(x) g'(x) dx = f(x)g(x) \at{a}{b} - \int_{a}^{b} f'(x) g(x) dx.
    \end{gather}
\end{proposition}

\begin{proof}
    Рассмотрим $(f(x)g(x))' = f'(x)g(x) + f(x)g'(x)$. Проинтегрируем 
    \begin{gather} 
     \int_{a}^{b} f'(x)g(x)dx + \int_{a}^{b} f(x)g'(x)dx= \int_{a}^{b} (f(x)g(x))'dx = f(x)g(x) \at{a}{b}.
    \end{gather}  
\end{proof}

\begin{note}
    Можно переписать в таком виде: 
    \begin{gather} 
      \int_{a}^{b} f(x) dg(x) = f(x)g(x) \at{a}{b} - \int_{a}^{b} g(x)df(x).
    \end{gather}
\end{note}

\subsection{Объём тел вращение}
Пусть \fabr, $\forall x \in \segab \hence f(x) \geq 0$. Тогда рассмотрим множество $$C = \left\{(x,y,z) \in \R^{3}: \sqrt[]{y^{2}+z^{2}} = f(x)\right\}.$$ Посчитаем объём фигуры ограниченной этим множеством. Пусть \tpab, тогда 
\begin{gather} 
  Q(f, T) = \pi \sum_{k=1}^{m-1}M_k^{2} (x_{k+1}-x_{k}), \\
  q(f, T) = \pi \sum_{k=1}^{m-1} m_k^{2} (x_{k+1}-x_{k}).
\end{gather} 

\begin{note}
    Мы сейчас ограничиваем цилиндрами, но это легко сводится к ограничению клеточными множествами.
\end{note}

\begin{definition}
    Будем считать, что $V$~--- \emph{объём тела вращения} (функцию $y=f(x)$ вращаем вокруг $Ox$), если 
    \begin{gather} 
      V = \lim_{l(T) \to 0} Q(f, T) = \lim_{l(T) \to 0} q(f,T).
    \end{gather}
\end{definition}

\begin{proposition}
    Пусть $f$~--- непрерывна на \segab\space и $f(x)\geq 0$, $\forall x \in \segab$, тогда 
    \begin{gather} 
      V = \pi \int_{a}^{b} f^{2}(x) dx
    \end{gather}  
\end{proposition}

\begin{proof}
    Заметим, что $f^{2}$~--- непрерывна, значит и интегрируема. Тогда 
    \begin{gather} 
       Q(f,T) - q(f, T) = \sum_{k=1}^{m-1} \pi(M^{2}_k - m^{2}_k) (x_{k+1}-x_{k})  = \Delta(\pi f^{2}, T) \xrightarrow[l(T) \to  0 ]{} 0 .
    \end{gather}
    В силу предыдущего равенства получаем, что 
    \begin{gather}
        V = \lim_{l(T) \to 0} Q(f, T) = \lim_{l(T) \to 0} q(f,T) = \intab \pi f^{2}(x)dx = \pi \intab f^{2}(x) dx.
    \end{gather}
\end{proof}

\begin{definition}
    Будем говорить, что $S$~--- площадь поверхности вращения (графика функции $y=f(x)$ вокруг оси $Ox$), если 
    \begin{gather} 
      \exists \lim_{l(T) \to 0} \left(\pi \sum_{k=1}^{m-1} (f(x_k) + f(x_{k+1})) \sqrt{(f(x_k)-f(x_{k+1}))^2 + (x_{k+1}-x_{k})^2}\right) =S.
    \end{gather}
\end{definition}

\begin{proposition}
    Рассмотрим $f$~--- непрерывна на \segab, $f$~--- непрерывно дифференцируема на $(a,b)$, тогда 
    \begin{gather} 
      S = 2\pi \intab f(x)\sqrt{1+(f'(x))^2} dx.
    \end{gather}
\end{proposition}

\begin{proof}
    Пусть \tpab. Рассмотрим 
    \begin{gather} 
      \pi (f(x_k)+f(x_{k+1})) \sqrt{(f(x_{k+1}) - f(x_k))^2 + (x_{k+1} - x_k)^2}.
    \end{gather}
    По теореме Лагранжа найдётся такая точка $\xi_k \in [x_{k+1},x_{k}]$: 
    \begin{gather} 
      \pi (f(x_k)+f(x_{k+1})) \sqrt{1+(f'(\xi_k))^2} (x_{k+1}-x_{k}).
    \end{gather}
    Пусть $h(x) = 2\pi f(x) \sqrt{1+(f'(x))^2}$, $\xi$--- выборка. Тогда 
    \begin{gather} 
      \s(h, T, \xi) = \sum_{k=1}^{m-1} 2\pi f(\xi_k) \sqrt{1+(f'(\xi_k))^2} (x_{k+1}-x_{k})  \xrightarrow[l(T) \to  0 ]{} 0 .
    \end{gather}
    Рассмотрим разность $k$-ых слагаемых из этих сумм: 
    \begin{gather} 
      \left| 2 \pi f(\xi_k) \sqrt{1+(f'(\xi_k))^2} - \pi (f(x_k) - f(x_{k+1})) \sqrt{1+(f'(\xi_k))^2}  \right| = \\ = \sqrt{1+(f'(\xi_k))^2} \left| (f(\xi k) - f(x_k)) + (f(\xi_k) - f(x_{k+1})) \right| \leq \\ \leq \pi C\left( \left| f(\xi_k) - f(x_k) \right|  + \left| f(\xi_k) - f(x_{k+1}) \right| \right) \leq 2\pi C \omega_k (f).
    \end{gather}
    Теперь для сумм: 
    \begin{gather} 
      \left| \s(h, T, \xi) - \sum_{k=1}^{m-1} \pi (f(x_k)+f(x_{k+1})) \sqrt{(f(x_{k+1}) - f(x_k))^2 + (x_{k+1} - x_k)^2}\right| \leq \\ \leq 2\pi C \sum_{k=1}^{m-1} \omega_k (f) (x_{k+1}-x_{k}) = 2\pi C \Delta(f, T)  \xrightarrow[l(T) \to  0 ]{} 0   .
    \end{gather}
    Тогда 
    \begin{gather} 
      S = 2\pi \intab f(x) \sqrt{1+(f'(x))^2} dx .
    \end{gather}
\end{proof}

Пусть $r(t)$ задаёт кривую $\Gamma$, $r(t)$~--- непрерывно дифференцируема на \segab, $s(t)$~--- натуральный параметр. Тогда $ \left| \Gamma \right| = s(b) - s(a)$, при этом $s'(t) = \left| r'(t) \right| $. В таком случае 
\begin{gather} 
  \int_{a}^{b} \left| r'(t) dt \right| = s(b) - s(a) = \left| \Gamma \right|  .
\end{gather}

\begin{proposition}
    Пусть $\Gamma = \{ r(t): t \in \segab \}$, $r$~--- непрерывно дифференцируема на \segab. Тогда  
    \begin{gather} 
        \left| \Gamma \right| = \int_{a}^{b} \left| r'(t) dt \right|.
    \end{gather}
\end{proposition}

\begin{proof}
    Очевидно.
\end{proof}

\begin{corollary}
    Пусть $r(t) = \begin{pmatrix}
    x \\ 
    f(x)
    \end{pmatrix}$, $r'(t) = \begin{pmatrix}
    1 \\ 
    f'(x)
    \end{pmatrix}$. Если $f$~--- непрерывно дифференцируема на \segab, тогда 
    \begin{gather} 
      l = \int_{a}^{b} \sqrt{1 + (f'(x))^2} dx,
    \end{gather}
    где $l$~--- длина куска графика функции $f(x)$ на \segab. 
\end{corollary}

\begin{corollary}
    Пусть $r(t) = \begin{pmatrix}
    x(t) \\ 
    y(t)
    \end{pmatrix}$, $r'(t) = \begin{pmatrix}
    x'(t) \\ 
    y'(t)
    \end{pmatrix}$. Если $f$~--- непрерывно дифференцируема на \segab, тогда 
    \begin{gather} 
      l = \int_{a}^{b} \sqrt{(x'(t))^2 + (y'(t))^2} dx,
    \end{gather}
    где $l$~--- длина куска графика функции $f(x)$ на \segab. 
\end{corollary}


\end{document}