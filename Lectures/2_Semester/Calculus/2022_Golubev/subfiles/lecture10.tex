\documentclass[../main.tex]{subfiles}

\begin{document}
\begin{proposition}
    Пусть \fabr, $f \in [a,c]$, \fintonab\space и $[b, c]$. Тогда $f$~--- интегрируема на $[a,c]$ и 
    \begin{gather*}
        \int_a^c f(x)dx =  \intfab + \int_b^c f(x) dx.
    \end{gather*}
\end{proposition}

\begin{proof}
    Пусть $T = \{x_k\}_{k=1}^m$~--- разбиение на $[a,c]$. \\ Рассмотрим отрезок $[x_j,x_{j+1}] \ni b$. Пусть $T_1$ = $\{x_k\}_{i=1}^j \cup \{b\}$, $\{x_k\}_{i=j+1}^m \cup \{b\}$, тогда $T_1$~--- разбиение $[a,b]$, $l(T_1) \leq l(T)$, $T_2$ разбиение $[b,c]$ и $l(T_2) \leq l(T)$. По определению
    \begin{gather*}
        \s(f, T, \xi) = \sumifromto{1}{m-1}f(\xi_i)(x_{i+1} - x_i), \\
        \s(f, T, \xi') = \sumifromto{1}{j}f(\xi'_i)(x_{i+1} - x_i), \\
        \s(f, T, \xi') = \sumifromto{j+1}{m-1}f(\xi''_i)(x_{i+1} - x_i).
    \end{gather*}
    Рассмотрим модуль разности
    \begin{multline*}
        |\s(f, T, \xi_i) - \s(f, T_1, \xi'_i) - \s(f, T_2, \xi''_i)| = |(f(\xi_j) - f(\xi'_j))(b-x_j) + (f(\xi_j) - f(\xi''_{j+1}))(x_{j+1} -b)| \leq \\ \leq |f(\xi_j) - f(\xi'_j)|l(T) + |f(\xi_j) - f(\xi''_{j+1})|l(T) \limto_{l(T) \to 0} 0.
    \end{multline*}
    Последний переход сделан в силу ограниченности функции. Переходя к пределу получаем
    \begin{gather*}
        \left| \s(f, T, \xi_i) - \intfab - \int_b^c f(x) dx\right| = 0.
    \end{gather*}
    Значит функция интегрируема, и его значение равно искомому выражению.
\end{proof}

\begin{definition}
    Договоримся, что $\int_a^a f(x) dx = 0$ и $\int_b^a f(x) dx = -\intfab$.
\end{definition}

\begin{corollary}
    Пусть $f$ определена и интегрируема на отрезке, содержащем $a, b, c$, тогда
    \begin{gather*}
        \intfab = \int_a^c f(x) dx + \int_c^b f(x) dx
    \end{gather*}
\end{corollary}
\begin{proof}
    Очевидно.
\end{proof}

\begin{proposition}
    Пусть \fabr, \fintonab, $g$ определена на \segab\space и отличается от $f$ в конечном числе точек. Тогда $g$ интегрируема на \segab.
\end{proposition}

\begin{proof}
    Рассмотрим $h(x) = f(x) - g(x)$. Пусть 
    \begin{gather*}
        \{x_1, \ldots, x_m\} \subset \segab :\ h(x_i) \neq 0 \forall \ito{m}, \forall x \in \segab \setminus \{x_1, \ldots, x_n\} \hence h(x) = 0.
    \end{gather*}
    Пусть \tpab, $C = \max_\ito{m} h(x_i)$. Рассмотрим 
    \begin{gather}
        |\s(h, T, \xi)| \leq C \cdot m l(T) \limto_{l(T) \to 0} 0.
    \end{gather}
    Тогда $h$~--- интегрируема по критерию интегрируемости. Тогда $f-h = g$~--- интегрируема.
\end{proof}

\subsection{Достаточные условия интегрируемости}

\begin{proposition}
    Пусть \fabr, $f$~--- непрерывна на \segab. Тогда \fintonab.
\end{proposition}

\begin{proof}
    $f$~--- непрерывна на \segab, значит $f$~--- равномерно непрерывна на \segab. Рассмотрим $\omega_f(\delta) = \sup_{|x_1-x_2| < \delta} |f(x_1) - f(x_2)|$ и $\omega_i(f) = \sup_{x',x'' \in [x_i,x_{i+1}]}(f(x') - f(x''))$.
    Пусть \tpab. Заметим, что $\forall x',x'' \in [x_i,x_{i+1}]\ |x'-x''| \leq l(T) < 2l(T)$. Запишем разность сумм Дарбу
    \begin{gather*}
        \Delta(f, T) = \sumifromto{1}{m-1} \omega_i(f)(x_{i+1} - x_i) \leq \sumifromto{1}{m-1} \omega_f(2l(T))(x_{i+1} - x_i) = \omega_f(2l(T)) (b-a) \limto_{l(T)\to0} 0,
    \end{gather*}
    значит \fintonab.
\end{proof}

\begin{definition}
    Пусть \fabr. Будем говорить, что $f$~--- \emph{кусочно непрерывна} на \segab, если $\exists\{x_i\}_{i=1}^{m}: a < x_1 < \ldots < x_m < b$, $f$~--- непрерывна на $[a,x_1)$, $(x_m,b]$, $(x_i, x_{i+1}),\ \forall \ito{m-1}$ и $\exists f(x_i \pm 0),\ \forall \ito{m}$.
\end{definition}

\begin{reminder}
    Последняя запись означает существование право- и левосторонних пределов у функции в точке.
\end{reminder}

\begin{proposition}
    Пусть $f$~--- кусочно непрерывна на отрезке \segab, тогда \fintonab.
\end{proposition}

\begin{proof}
    Рассмотрим $g:(x_k,x_{k+1}) \to \R$:
    \begin{gather*}
        \System{
            g(x) = f(x), \forall x \in (x_k,x_{k+1}) \\
            g(x_k) = f(x_k+0) \\
            g(x_{k+1}) = f(x_{k+1} - 0)
        } \hence
        \begin{gathered}
            g~\text{--- непрерывна на $[x_k,x_{k+1}]$} \hence \\ 
            g~\text{--- интегрируема на $[x_k,x_{k+1}]$},
        \end{gathered}
    \end{gather*}
    $g$ отличается от $f$ лишь на конечное число точек, значит $f$~--- интегрируема на $[x_k,x_{k+1}]$. В силу аддитивности \fintonab.
\end{proof}

\begin{proposition}
    Пусть $f$~--- монотонна на \segab, тогда \fintonab.
\end{proposition}

\begin{proof}
    Без ограничения общности предположим, что $f$ не убывает. Пусть \tpab. Заметим, что $m_k = f(x_k$) и $M_k = f(x_{k+1})$. Рассмотрим 
    \begin{multline*}
        \Delta(f, T) = S(f, T) - s(f, T) = \sumifromto{1}{m-1}f(x_{k+1})(x_{x+1} - x_k) - \sumifromto{1}{m-1}f(x_k)(x_{x+1} - x_k) = \\ = \sumifromto{1}{m-1}(f(x_{k+1}-x_{k}))(x_{x+1} - x_k) \leq l(T) \sumifromto{1}{m-1}(x_{x+1} - x_k) = l(T) \bigl(f(b) - f(a)\bigr) \limto_{l(T) \to 0} 0.
    \end{multline*}
    Значит \fintonab.
\end{proof}

\subsection{Связь определённого и неопределённого интегралов.}

\begin{proposition}
    Пусть \fintonab. Тогда $F(x) = \int_a^x f(t) dt$~--- непрерывна на \segab.
\end{proposition}

\begin{proof}
    Рассмотрим $x_1 \geq x_2 \in \segab$. Тогда
    \begin{gather*}
        |F(x_2) - F(x_1)| = \left|\int_{x_1}^{x_2}f(t)dt\right| \leq \int_{x_1}^{x_2}|f(t)|dt \leq \int_{x_1}^{x_2} M dt = M(x_2 - x_1).
    \end{gather*}
    В итоге:
    \begin{gather*}
        \forall \eps > 0 \exists \delta = \frac{\eps}{M}: \forall x_2 \in U_\delta(x_1) \hence |F(x_2) - F(x_1)| < \eps.
    \end{gather*}
\end{proof}

\begin{definition}
    Будем называть $F$~--- \emph{первообразной} $f$ на \segab, если $\forall x \in (a, b) \hence F'(x) = f(x)$, $f(a) = F'_+(a)$ и $f(b) = F'_-(b)$.
\end{definition}

\begin{proposition}
    Пусть $f$~--- непрерывна на \segab. Тогда $F(x) = \int_a^xf(t)dt$ является первообразной $f$ на \segab.
\end{proposition}

\begin{proof}
    Пусть $x_0 \in (a,b)$. 
    \begin{multline*}
    F(x) - F(x_0) = \int_{x_0}^x f(t) dt = (x-x_0)f(x_0) - \int_{x_0}^x f(x_0) dt + \int_{x_0}^x f(t) dt = \\ = (x-x_0)f(x_0) + \int_{x_0}^x (f(t)-f(x_0)) dt.
    \end{multline*}
    Для вычисления производной нужно разделить на $(x-x_0)$:
    \begin{gather*}
        \frac{F(x) - F(x_0)}{x-x_0} = f(x_0) + \frac{1}{x-x_0} \int_{x_0}^x (f(t)-f(x_0)) dt.
    \end{gather*}
    Воспользуемся определением непрерывности:
    \begin{gather*}
        \forall\eps>0 \exists\delta:\forall t\in U_\delta(x_0) \hence |f(t) - f(x_0)| < \eps.
    \end{gather*} 
    Теперь рассматривая $x \in U_\delta(x_0)$ получаем   
    \begin{gather*}
        \left|\frac{F(x) - F(x_0)}{x-x_0} - f(x_0)\right| \leq \frac{1}{x-x_0} \int_{x_0}^x (f(t)-f(x_0)) dt < \frac{1}{x-x_0} \int_{x_0}^x \eps dt = \eps.
    \end{gather*}
    Получили определение производной.
\end{proof}

\begin{corollary}
    Если $f$~--- непрерывна на \segab, то $\intfab + C$~--- семейство первообразных.
\end{corollary}

\begin{corollary}
    Если $f$~--- непрерывна на \segab\space и $F$~--- первообразная $f$ на \segab, то 
    \begin{gather*}
        \intfab = F(b) - F(a).
    \end{gather*}
    Это равенство называется \emph{формула Ньютона-Лейбница}. 
\end{corollary}

\begin{proof}
    Очевидно.
\end{proof}

\begin{note}
    Вообще говоря условие непрерывности слишком сильное, его можно ослабить.
\end{note}

\begin{proposition}
    Пусть $f$~--- интегрируема на \segab, $F$~--- первообразная $f$ на \segab. Тогда 
    \begin{gather*}
        \intfab = F(b) - F(a).
    \end{gather*}
\end{proposition}

\begin{note}
    Первое требование~--- про меру Жордана, второе требование~--- про производную.
\end{note}


\end{document}