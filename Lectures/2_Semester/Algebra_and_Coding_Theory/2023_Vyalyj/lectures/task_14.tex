\section{№14 Приложения теории конечных полей}

\subsection{Теория кодирования конечных машин}

\begin{problem}
  Необходимо передать сообщение с минимальными потерями связи.
\end{problem}

\begin{solution}
  Будем передавать дискретный сигнал
\end{solution}

\begin{example}
  Пусть есть битовый код:
  \begin{gather}
     \{0, 1\} ^k \ra \{0, 1\} ^n \\
     \overset{\{0, 1\}^k}{\ra} |\text{Кодер}| \overset{\{0, 1\}^n}{\ra} | \text{Канал} | \overset{y}{\ra} | \text{Декодер} | \overset{D(y)}{\ra}
  \end{gather}
  Хотим: $D(y) = x$.
  % \[\overset{asv}{123}\]
\end{example}

\begin{definition}
  Код: $C \subseteq  \{0, 1\}^n$, где $n$~--- длина кода.
\end{definition}

\begin{definition}
  Размерность $C$: $k = \log_2 |C|$. 
\end{definition}

\begin{definition}
  Кодовое расстояние: $n, k, d$.
\end{definition}

Модель ошибки: инвертируем не более $r$ битов.
\subsection{Расстояние Хэмминга}

\begin{definition}
  Расстояние Хэмминга:
  \[d = \min_{x \neq y \quad x, y \in C}(\rho(x, y))\]
  \[\rho(x, y) = \# (i: x_i \neq y_i)\]
\end{definition}

\begin{lemma}
  Для расстояния Хэмминга:
  \[\rho(x,z) \leq \rho(x, y) + \rho(y, z)\]
\end{lemma}

\begin{proof}
  Пусть: 
  \begin{gather}
    A = \{i: x_i \neq y_i\} \\
    B = \{j: x_j \neq y_j\} \\
    C = \{l: x_l \neq y_l\}
  \end{gather}
  Тогда:
  \[\rho(x,z) \leq \rho(x, y) + \rho(y, z) \Lra C \subseteq A \cup B\]
\end{proof}

\begin{note}
  Также это расстояние можно рассмотреть как расстояние кратчайшего пути из двух вершин в булевом кубе. 
\end{note}

\begin{proposition}
  $d \geq 2 r + 1 \Ra $ код исправляет $\leq r$ ошибок, где код $C$ -- $(n, k, d)$.
\end{proposition}

\begin{proof}
  \begin{gather}
    x \ra x' \in B_r(x) = \{z: \rho(x, z) \leq r \}
  \end{gather}
  Тогда:
  \[\rho(x, y) \leq \rho(x, y) - \rho(x, x) \leq 2r + 1 - r = r + 1\]
\end{proof}

\begin{example}~
  \begin{itemize}
    \item Код повторения: $\{0^n, 1^n\}, \quad k = 1, d = n$. (Очень медленный)
    \item Тривиальный код: $\{0, 1\}^n, \quad k = n, d = 1$.
    \item Код проверки на чётность: $\{x: \sum x_i \equiv 0 \mod 2\}, \quad k = n - 1, d = 2$. (Не исправляем ни одной ошшбки)
  \end{itemize}
\end{example}

\begin{example}[?]
  $n = 5, \quad k = 4$. Тогда: 
  \[1011 \ra 10111\]
\end{example}

\begin{definition}
  Метрика Хэммингового шара:
  \[B_{n, r} = \sum_{i = 0}^{r} \binom{n}{i} = |B_r(o^n)|\]
\end{definition}

\begin{definition}
  Граница Хэмминга. Если $C$ -- $(n, k, 2r + 1)$~--- код, то
  \[2^n = |C| \leq \frac{2^n}{B_{n, r}}\]
\end{definition}

\begin{definition}
  Граница Варшамова - Гилберта, $\exists (n, k, 2d + 1)$~--- код:
  \[2^k = |C| \geq \frac{2^n}{B_{n, 2r}}\]
\end{definition}

\begin{proof}
  Будем выбирать точки на шарах, так что:
  \[x_{i + 1} \in \{0, 1\}^n \setminus V_i \cd B(x_i, 2V) \]
  \[k \cd B_{n, 2r} < 2^n\]
\end{proof}

\subsection{Явная конструкция кода}

Построим двоичные сообщения $\{ 0, 1\}^n \lra \FF_x^n$

\begin{definition}
  Линейный код $C$~--- подпространство $\FF_2^n$.
\end{definition}

Тога кодовое расстояние $d = \min_{x \neq 0, \quad x \in C}(||xn||)$. Учтя, что $||x|| = x(i: \quad  x_i = 1)$, получаем:
\[\rho(x, y)=  ||x - y|| \]
% Учтя, что: $$
\subsection{Код исправляющий ошибку}

\begin{example}~
  \begin{itemize}
    \item Проверочная матрица: $C = \{ x: \quad Hx = 0\}$
    \item Порождающая матрица: $C = \{x \quad : yG = x\}$
  \end{itemize}
\end{example}

\begin{example}
  $d = 3, n = 2^5 - 1, k = 2^s - s - 1$~--- код Хэмминга.

  \[Gk 
  \begin{pmatrix}
    1 & 0 & 0 & \dots \\
  0 & 1 & 0 & \dots \\
  0 & 0 & 1 & \dots \\
  \end{pmatrix}
  \]

  % n = k + s. Строка $y \in {0, 1}^5$, $||y|| \leq 2$.
  % \begin{align*}
  %   &p=1, \quad q \leq 2, $\quad (p + q)$ \leq 3 \\
  %   &p=2, \quad q \leq 1, $\quad (p + q)$ \leq 3 \\
  %   &p \leq 3, $\quad (p + q)$ \leq 3
  % \end{align*}
\end{example}

\begin{definition}
  Совершенный код~---
\end{definition}

\subsection{Коды с одной ошибкой}
Пусть $\{0, 1\}^n \lra \FF_2^n \lra \RR_n = \FF_2[x] / (x^n-1)$

% \begin(definition)
%   Циклический код~---
% \end(definition)

\begin{definition}  
  Оператор циклического свдига:
  \[S: [x^i] \ra [x^{i + 1} \mod n]\]
\end{definition}

\begin{lemma}
  $I$ -- индексы числе в $b_n \Lra SI = I$
\end{lemma}

\begin{proof}~
  \begin{itemize}
    \item[$\Ra$] 
    \begin{gather}
      I - \text{индекс} \\
      S[f] = [xf(x)] \in I, \quad SI \subseteq I \\
      [f] = [\sum \alpha_i x^i ] = \sum \alpha_i [x^i] \Ra \\
      S^{-1} = S^{n-1} 
    \end{gather}
    \item[$\La$]
    \begin{gather}    
      [f] = [\sum \alpha_i x^i] = \sum \alpha _i [x^i] \\
      [g] = \sum \beta_i [x^i] \\
      [fg] = \sum \beta_i [x^i f] \in I 
  \end{gather}
\end{itemize}
\end{proof}

\begin{example}
  Зададим код: $(1100100000000000)$. Возьмём все циклические суммы и всевозможные суммы. На самом деле это задаёт код Хэмминга.
\end{example}

% \[\emptyset\]