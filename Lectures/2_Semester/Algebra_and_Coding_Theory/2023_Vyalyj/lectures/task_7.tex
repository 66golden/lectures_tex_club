\section{№7 Действия группы}

\begin{example}
  Пусть $x \in X$ --- точка, $G$ --- элементы
  \begin{gather}
    \phi: G \ra S(x) \\
    G: X \ra X \\
    \phi(g): X \ra X \\
    g.x = \phi(g)(x)
  \end{gather}
\end{example}

\begin{proposition}
  Действие $G: G \ra G$:
  \[g.h = gh\]
\end{proposition}

\begin{definition}
  Точным наззывается гомоморфизм, так что:
  \[\ke \phi = \{ e \}\]
\end{definition}

\begin{definition}[?]
  Действие называется транзитивным, если:
  \[\forall x, y \quad \exists g: \ g.x = y\]
  \[(g_1 g_2)x \ = g_1 (g_2 x)\]
\end{definition}

\subsection{Теорема Кэли}

\begin{theorem}[Кэли]
  Если $|G| < \infty$, тогда $G \cong G' < S_n$.
\end{theorem}

\begin{proof}
  \begin{gather}
    n = |G| \\
    \phi: G \ra S_n \\
    \im \phi \cong G / \ke \phi \Ra \im \phi \cong G, \quad \ke \phi = \{ e \}
  \end{gather}
\end{proof}

\subsection{Действие сопряжением}

\begin{definition}
  Действие сопряжением:
  \begin{gather}
    G: G \ra G \\
    g.h = gh g^{-1} \\
    (g_1 g_2).h = (g_1 g_2)h(g_1 g_2)^{-1} = (g_1 g_2)h(g_2^{-1} g_1^{-1}) = g_1 (g_2.h) g_1^{-1} = g_1 (g_2.h)
  \end{gather}
  Замечание:
  \[ h' = ghg^{-1} \Lra h = g^{-1} h' g\]
\end{definition}

\begin{proposition}
  $G$ --- абелева, тогда:
  \[\phi : G \ra S(G), \quad \ke \phi = G\]
\end{proposition}

\begin{definition}
  $\ke \phi = Z(G)$ --- центр, то есть:
  \begin{gather}
    \forall h: \ ghg^{-1} = h \Lra gh = hg
  \end{gather}
\end{definition}

\begin{corollary}
  Центр является нормальной подгруппой. То есть:
  \[Z(G) \vartriangleleft G \]
\end{corollary}

\begin{definition}
  Орбита действия:
  \[\orb (x) = \{ g: \exists g: \ g.x = y \}\]
\end{definition}

\begin{definition}
  Стабилизатор:
  \[\stab (x) = \{ g: g.x = x \}\]
\end{definition}

\begin{proposition}
  $\stab (x) < G$.
\end{proposition}

\begin{example}
  Пусть $x \in X, \ g \in G$. Тогда:
  \[ g.x = y, \ g'.x = y \Lra g' \in g \stab (x) \]
\end{example}

\begin{lemma}
  Орбиты действия разбивают точки на непересекающиеся множества.
  \[x \sim y \defev y = g.x \ g \in G\]
\end{lemma}

\begin{proof}~
  \begin{enumerate}
    \item $x \sim x x = e.x$ 
    \item $x \sim y \Ra y \sim x \quad g^{-1}.y = g^{-1}.x .(g.x) = (g^{-1} g).x = e.x = x$ 
    \item $x \sim y, y \sim z \Ra y = g1.x, \ z = g_2.y = (g_2 g_1).x$
  \end{enumerate}
\end{proof}

\begin{theorem}[Лагранжа]
  $|\orb (x)| = (G : \stab (x))$.
\end{theorem}

\begin{theorem}
  $G: x \ra x, \ x \in X$. Тогда:
  \[|G| = |\orb (x)| \cd |\stab (x)|\]
\end{theorem}

\subsection{Группы многогранников}

\begin{reminder}~
  \begin{itemize}
  \item Додекаэдр --- 20 вершин, 12 граней.
  \item Икосаэдр --- 12 вершин, 20 граней.
  \end{itemize}
\end{reminder}

\begin{proposition}
  Для каждого многогранника существует собственная группа движений, которая оставляет его на месте.
\end{proposition}



\begin{example}
  Рассмотрим мощности групп правильных многогранников:
  \begin{itemize}
    \item $|\text{Тетраэдр}| = 12$
    \item $|\text{Куб}| = 24$
    \item $|\text{Октаэдр}| = 24$
    \item $|\text{Додекаэдр}| = 20 \cd 3 = 60$
    \item $|\text{Икосаэдр}| = 5 \cd 12 = 60$
  \end{itemize}
\end{example}

\begin{proposition}
  Куб $\cong S_4$.
\end{proposition}
