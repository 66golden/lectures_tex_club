\section{№10 Конечные поля}

Пусть есть конечно поле $F$.

\begin{definition}
  Характеристика $F$: $p = \ord_{+} \langle 1 \rangle$, где $p$~--- простое число.
\end{definition}

\begin{corollary}
  $\FF_p[x] \subset F[x]$. Точка $\{a \in F \quad f \in F_p[x]\}$.
\end{corollary}

\begin{definition}
  Отображение оценки: 
  \[\eva: \FF_p[x] \ra F\]
\end{definition}

Свойства:
\begin{gather}
  \eva (f) = f(a) \\
  \eva(f + g) = f(a) + g(a) = \eva(f) + \eva(g) = (f + g)(a) \\
  \eva(fg) = fg(a) = f(g)(g(a)) = \eva(f)\eva(g)
\end{gather}

\subsection{Гомоморфизм колец}

\begin{definition}
  Гомоморфизм $\phi: R_1 \ra R_2$
  \begin{gather}
    \phi(x + y) = \phi(x) + \phi(y) \\
    \phi(xy) = \phi(x) \phi(y)
  \end{gather}
\end{definition}

\begin{definition}
  Биективный Гомоморфизм~--- изоморфизм.
\end{definition}

\begin{example}
  \[\ZZ \ra \ZZ/n\ZZ\]
\end{example}

\begin{theorem}
  Конечное поле~--- гомоморфизм образа $\FF_p[x]$.
\end{theorem}

\begin{proof}
  \begin{gather}
    \langle \alpha \rangle = F^{*} \\
    \eva : \FF_p[x] \ra F \text{~--- сюръекция} \\
    \eva(0) = 0 \\
    \eva(x^n) = \alpha^n, \forall x \neq \exists n, x = \alpha^n
  \end{gather}
\end{proof}

\begin{definition}
  Ядро гомоморфизма $\phi: R_1 \ra R_2$.
  \[\re \phi = \{ x \in R_1: \phi(x) = 0 \}\]
\end{definition}

\subsection{Подгруппа аддитивной группы}
\begin{proposition}
  Подгруппа аддитивной группы.
  \begin{gather}
    \forall r_1 \in R_1, \quad \forall k \in \ke \phi \quad (r_1 k \in \ke \phi \lor k r_1 \in \ke \phi) \\
    \phi(r_1 k) = \phi(r_1) \phi(k) = \phi (r_1) \cd 0 = 0 \\
    \phi(k r_1) = \phi(k) \phi(r_1) = 0 \cd \phi (r_1) = 0
  \end{gather}    
\end{proposition}

\subsection{Идеал}

\begin{definition}
  Идеал $I \subset R$:
  \begin{enumerate}
    \item Подгруппа аддитивной группы
    \item $\forall r \in R \quad \forall i \in I \Ra ri \in I$~--- Левый идеал
    \item[2'] $\forall r \in R \quad \forall i \in I \Ra ir \in I$~--- Правый идеал
  \end{enumerate}
  Двухсторонний идеал: левый и правый идеал одновременно.
\end{definition}

\subsection{Кольцо вычетов $R/I$}

\begin{definition}
  Пусть $R$~--- кольцо, а $I$~--- двухсторонний идеал $R$. Пусть $x \in R$:
  \begin{itemize}
    \item $x + I = [x]$
    \item $[x] + [y] = [x + y]$
    \item $[x][y] = [xy]$ (Так как $x \equiv y \mod I \Lra (x - y) \in I$)
  \end{itemize}
\end{definition}

\begin{proof}
  Корректность:
  \begin{gather}
    x' = x + i' \\
    y' = y + i'' \\
    x'y' - xy = xy + \underbrace{i'y}_{\in I} + \underbrace{xi''}_{\in I}  + \underbrace{i'i''}_{\in I} - xy \in I \\
    c: R \ra R/I \quad c: x \ra [x]
  \end{gather}
\end{proof}

\begin{theorem}[О Гомоморфизмах]
  $\phi: R_1 \ra R_2$~--- гомоморфизм, то $\im \phi \cong R_1 / \ke \phi$.
\end{theorem}

\begin{proof}~
  \begin{itemize}
    \item $\psi([x]) = \phi(x), \quad R_1/\ke \phi \ra \im \phi$
    \item $\psi([x][y]) = \psi([xy]) = \phi(xy) = \phi(x) + \phi(y) = \psi[x] \psi[y]$
  \end{itemize}
\end{proof}

\begin{theorem}[О максимальном идеале]
  $R$~--- коммутирует с 1, и $R/I$~--- поле $\Lra$ $I$~--- максимальный идеал.
\end{theorem}

\begin{proof}
  $I$~--- максимальный $\equiv$ $I \subset R \  \forall J (I \subseteq J) \Ra (J = I), \quad (J = R)$
\end{proof}

\begin{proposition}
  В полях идеалов нет.
\end{proposition}

\begin{definition}
  Главный идеал: $(a) = \{ra: r \in R\}$.
\end{definition}

\begin{example}
  \begin{gather}
    (1) = R \Ra (J \subset R \Ra 1 \notin J) \\
    R/I \text{~--- поле}, \quad [a][b] = [1], a \notin I \\
    I \subset J \subset R, a \in J/I \\
    \underbrace{ab}_{\in J} = \underbrace{1 + i}_{\in J}, i \in I
  \end{gather}
\end{example}

\begin{example}
  $I$~--- максимальный идеал $a \notin I$ 
  \begin{gather}
    I \subset \underbrace{(a, I)}_{=R} = \{ra + i : r \in R, i \in I\}
    1 = ra + i \\
    [a] \neq 0, [r][a] = [1]
  \end{gather}
\end{example}

\subsection{Евклидова кольца}

\begin{definition}
  Евклидово кольцо:
  \begin{itemize}
    \item[$(E_1)$] $R$~--- коммутирует с единицей, без делителей нуля.
    \item[$(E_2)$] Деление с остатком:
    \begin{gather}
      \forall a, b \in R, b \neq 0 \\
      \exists q, r: a = qb + r,(r = 0 \lor N(r) < N(b)) 
    \end{gather}
    \item[$(E_3)$] $N(ab) \geq \max {(N(a), N(b))}$
  \end{itemize}
\end{definition}

\begin{example}[(?)Гауссовы целые]
  \begin{gather}  
    \ZZ[i] = \{ a + bi, a, b \in \ZZ \} \\
    N(x) = |x| \\
    \exists N: R \setminus \{ 0\} \ra \NN \\
    i = i (1 + i) + 1 \\
    i = 1 (1 + i) - i
  \end{gather}
\end{example}

\begin{theorem}
  Евклидово кольцо~--- кольцо главных идеалов.
\end{theorem}

\begin{proof}
  Пусть $I \subseteq R$:
  \begin{gather}
    I \neq \{0\} = (0) \\
    a \in I \ N(a) \text{с наименьшей нормой} \\
    N(a) = \min (R: \exists r \in R, N(r) = k) \\
    x \in I, x = qa + r, (r = 0 \lor N(r) < (n(a)))
  \end{gather}
  Тогда имеем: $r = 0, \quad r = x - qa \in I$. То есть $I = (a)$.
\end{proof}
