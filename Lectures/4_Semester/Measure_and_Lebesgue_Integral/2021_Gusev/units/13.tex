\newpage
\lecture{13}{Интеграл Лебега для знакопеременных функций.}

\begin{definition}
    Тройка $(X,\, \CE,\, \mu)$, где $X$~--- некоторое множество,
    $\CE$~--- $\sigma$"=алгебра подмножества множества $X$,
    $\mu:\:\CE\to[0,\,+\infty]$~--- $\sigma$"=аддитивная мера,
    $(X,\, \CE)$~--- измеримое пространство, называется
    \mdef{пространство с мерой}.
\end{definition}

\begin{definition}
    Пусть функция $f:\: X\to\R$ измерима относительно $\CE$.
    Пусть \begin{align*}
        f^{+}(x)=\max(0,\, f(x))\geqslant 0, \\
        f^{-}(x)=\max(0,\, -f(x))\geqslant 0,
    \end{align*}
    тогда $\forall x\ f(x)=f^+(x)-f^{-}(x)$.

    Следовательно существуют конечные или бесконечные интегралы Лебега:
    \[
        \int f^{+}d\mu,\quad \int f^{-1}d\mu.
    \]

    Если интегралы конечные, то $f$ называется \mdef{интегрируемой по Лебегу} и \[
        \int fd\mu:=\int f^{+}d\mu-\int f^{-}d\mu.
    \]

    Пишут, что \begin{align*}
        f & \in L^{1}(X,\, \CE,\, \mu), \\
        f & \in L^{1}(X,\, \mu),        \\
        f & \in L^{1}(X),               \\
        f & \in L^{1}(\mu).
    \end{align*}

    \begin{remark}
        Часто под $L^1(X,\, \CE,\, \mu)$ понимается множество классов эквивалентности:
        $L^1(X,\, \CE,\, \mu)\setminus\sim$, где $f\sim g\Leftrightarrow f=g$ $\mu$"=почти всюду.
    \end{remark}
\end{definition}

\begin{remark}
    Функция $f$ интегрируема по Лебегу, тогда и только тогда, когда \[
        \int |f|d\mu<\infty.
    \]

    Следует непосредственно из определения, так как $|f|=f^{+}+f^{-}$.
\end{remark}

\begin{claim}
    Если $f,\, g\in L^{1}(\mu)$, то $\forall \alpha,\, \beta\in\R$ выполняется
    $\alpha f+\beta g\in L^1(\mu)$ и \[
        \int (\alpha f+\beta g)d\mu=\alpha\int fd\mu + \beta\int gd\mu.
    \]

    \begin{proof}

        Ограничимся случаем $\alpha=\beta=1$. Пусть $h=f+g$. Тогда \[
            h^+-h^-=f^+-f^-+g^+-g^-\Rightarrow
            h^++f^-+g^-=f^++g^++h^-.
        \]
        Ранее была доказана аддитивность интеграла Лебега, а значит, проинтегрировав полученное равенство,
        имеем \begin{align*}
            \int h^+d\mu+\int f^-d\mu+\int g^-d\mu & =
            \int \left(h^++f^-+g^-\right)d\mu=                                           \\
                                                   & =\int \left(f^++g^++h^-\right)d\mu=
            \int f^+d\mu+\int g^+d\mu+\int h^-d\mu.
        \end{align*}
        Перегруппировав слагаемые получаем то, что требовалось: \[
            \int h^+d\mu-\int h^-d\mu=
            \int f^+d\mu-\int f^-d\mu+\int g^+d\mu-\int g^-d\mu.
        \]

    \end{proof}
\end{claim}

Рассмотрим некоторые факты, которые впоследствии приведут к лемме Фату.
Пусть функции $f_n:\: X\to[0,\, +\infty]$ сходятся поточечно к $f$. Тогда \[
    \inf_{n}f_n(x)\leqslant f_n(x)\leqslant\sup_n f_n(x).
\]
Тогда, проинтегрировав левое неравенство, \begin{equation}
    \label{eq:lect13:ints}
    \int \inf_n f_n(x)d\mu\leqslant\int f_n(x)d\mu\Rightarrow
    \int \inf_n f_n(x)d\mu\leqslant\inf_n\int f_n(x)d\mu.
\end{equation}

\begin{lemma}[Фату]
    Пусть функции $f_n:\: X\to[0,\, +\infty]$ измеримы относительно $\CE$. Тогда
    \begin{equation}
        \label{eq:lect13:fatu}
        \int \lim_{\overline{n\to\infty}}f_n(x)d\mu(x)\leqslant\lim_{\overline{n\to\infty}}
        \int f_n(x)d\mu(x).
    \end{equation}

    \begin{remark}
        Ранее было доказано, что $f(x)=\lim\limits_{\overline{n\to\infty}}f_n(x)$ измерима
        относительно $\CE$.
    \end{remark}

    \begin{proof}

        Из курса математического анализа имеем \[
            \lim_{\overline{n\to\infty}} f_n(x)=\lim_{n\to\infty}\inf_{k\geqslant n}f_k(x).
        \]
        В силу \eqref{eq:lect13:ints} \begin{equation}
            \label{eq:lect13:2}
            \forall n\in\N\quad \int \inf_{k\geqslant n}f_kd\mu\leqslant\inf_{k\geqslant n}\int f_kd\mu.
        \end{equation}

        Рассмотрим $g_n(x):=\inf\limits_{k\geqslant n} f_k(x)$, тогда
        $g_{n+1}(x)\geqslant g_n(x)\ \forall n\in\N$. А также \[
            g_n(x)\xrightarrow[n\to\infty]{}\lim_{\overline{n\to\infty}}f_n(x)=g(x).
        \]
        Поэтому по теореме Леви: \[
            \int g_nd\mu\xrightarrow[n\to\infty]{}\int gd\mu.
        \]
        Итак, перейдя к пределу в неравенстве \eqref{eq:lect13:2} получим неравенство \eqref{eq:lect13:fatu}.

    \end{proof}
\end{lemma}

\begin{theorem}[Лебега об ограниченной сходимости]
    Пусть $\forall n\in\N:\: f_n\in L^1(X,\, \CE,\, \mu)$, а также
    $\forall x\in X:\: f_n(x)\xrightarrow[n\to\infty]{} f(x)$ (отсюда следует, что $f$ измерима относительно $\CE$).

    Пусть $g\in L^1(X,\, \CE,\, \mu)$ и $\forall x \in X$
    выполнено $f_n(x)\leqslant g(x)\ \forall n\in\N$.
    Тогда $f\in L^1(X,\, \CE,\, \mu)$ и \[
        \int |f_n-f|d\mu\xrightarrow[n\to\infty]{}0.
    \]

    \begin{remark}
        Данная сходимость называется \mdef{сходимость в среднем}.
    \end{remark}

    \begin{next0}
        В частности, \[
            \int f_nd\mu\xrightarrow[n\to\infty]{}\int fd\mu,
        \]
        то есть можно переставлять знаки предела и интеграла:\[
            \lim_{n\to\infty}\int f_nd\mu=\int\lim_{n\to\infty}f_nd\mu.
        \]
    \end{next0}

    \begin{proof}

        Вначале заметим, что \[
            |f_n(x)|\leqslant g(x)\quad\forall n\Rightarrow|f(x)|\leqslant g(x)\Rightarrow
            \int |f|d\mu\leqslant\int gd\mu<\infty\Rightarrow f\in L^1.
        \]

        Далее, \[
            |f_n(x)-f(x)|\leqslant 2\cdot g(x)\Rightarrow 2g(x)-|f_n(x)-f(x)|\geqslant 0.
        \]
        Тогда по лемме Фату, \[
            \int \lim_{\overline{n\toi}}\left(2g(x)-|f_n(x)-f(x)|\right)d\mu(x)
            \leqslant \lim_{\overline{n\toi}}\int \left(2g-|f_n-f|\right)d\mu.
        \]
        Откуда, \[
            \int 2gd\mu\leqslant\lim_{\overline{n\toi}}\int \left(2g-|f_n-f|\right)d\mu=
            \int 2gd\mu-\overline{\lim_{n\toi}}\int |f_n-f|d\mu.
        \]
        Окончательно, \[
            \overline{\lim_{n\toi}}\int |f_n-f|d\mu\leqslant 0\Rightarrow
            \int |f_n-f|d\mu\xrightarrow[n\toi]{}0.
        \]

    \end{proof}
\end{theorem}

\subsection{Связь с интегралом Римана.}

\begin{theorem}
    Пусть $f:\: [a,\, b]\to\R$ интегрируема по Риману (в собственном смысле) на $[a,\, b]$, $a<b$,
    $a,\, b\in\R$. Тогда \[
        f\in L^1\left([a,\, b],\, \mathfrak{m}_1,\, \lambda\right) \text{ и }
        \overbrace{\int_{[a,\, b]}fd\lambda}^{\text{Лебега}}=\underbrace{\int_{a}^{b}fdx}_{\text{Римана}},
    \]
    где $\mathfrak{m}_1$~--- $\sigma$"=алгебра $\R$, состоящая из подмножеств,
    измеримых по Лебегу, $\lambda$~--- мера Лебега.

    \begin{proof}

        Рассмотрим следующее разбиение: делим отрезок $[a,\, b]$ пополам и
        в каждой половине берем инфинум и супремум. Повторяем тоже самое
        для меньших отрезков (см. рис. красным цветом отмечены инфинумы, зеленым~--- супремумы).
        \begin{center}
            

\tikzset{every picture/.style={line width=0.75pt}} %set default line width to 0.75pt        

\begin{tikzpicture}[x=0.75pt,y=0.75pt,yscale=-1,xscale=1]
%uncomment if require: \path (0,300); %set diagram left start at 0, and has height of 300

%Straight Lines [id:da26592190121253423] 
\draw    (40,10) -- (40,290) ;
%Straight Lines [id:da17272901136308416] 
\draw    (10,260) -- (320,260) ;
%Straight Lines [id:da4118407802248629] 
\draw    (280,10) -- (280,290) ;
%Curve Lines [id:da5515064926669888] 
\draw    (40,120) .. controls (123.4,-70.33) and (223.4,293) .. (280,150) ;
%Straight Lines [id:da16722302908182263] 
\draw  [dash pattern={on 0.84pt off 2.51pt}]  (160,10) -- (160,290) ;
%Straight Lines [id:da6036644091583632] 
\draw [color={rgb, 255:red, 208; green, 2; blue, 27 }  ,draw opacity=1 ]   (160,184.67) -- (280,184.67) ;
%Straight Lines [id:da6721536932995469] 
\draw [color={rgb, 255:red, 208; green, 2; blue, 27 }  ,draw opacity=1 ]   (40,120) -- (160,120) ;
%Straight Lines [id:da2590934738397619] 
\draw [color={rgb, 255:red, 65; green, 117; blue, 5 }  ,draw opacity=1 ]   (40.67,63.33) -- (160.67,63.33) ;
%Straight Lines [id:da11447176814187543] 
\draw [color={rgb, 255:red, 65; green, 117; blue, 5 }  ,draw opacity=1 ]   (160,106.67) -- (280,106.67) ;
%Straight Lines [id:da6305486371019795] 
\draw  [dash pattern={on 0.84pt off 2.51pt}]  (100,10) -- (100,290) ;
%Straight Lines [id:da24393055003487474] 
\draw  [dash pattern={on 0.84pt off 2.51pt}]  (220,10) -- (220,290) ;
%Straight Lines [id:da8994049519539502] 
\draw [color={rgb, 255:red, 208; green, 2; blue, 27 }  ,draw opacity=1 ]   (100,106.67) -- (160,106.67) ;
%Straight Lines [id:da01959424270179566] 
\draw [color={rgb, 255:red, 208; green, 2; blue, 27 }  ,draw opacity=1 ][line width=1.5]    (40,120) -- (100,120) ;
%Straight Lines [id:da07578301019995393] 
\draw [color={rgb, 255:red, 208; green, 2; blue, 27 }  ,draw opacity=1 ]   (160,172) -- (220,172) ;
%Straight Lines [id:da5874498527299239] 
\draw [color={rgb, 255:red, 208; green, 2; blue, 27 }  ,draw opacity=1 ][line width=1.5]    (220,184.67) -- (280,184.67) ;
%Straight Lines [id:da07299724197268098] 
\draw [color={rgb, 255:red, 65; green, 117; blue, 5 }  ,draw opacity=1 ][line width=1.5]    (40.67,63.33) -- (100.67,63.33) ;
%Straight Lines [id:da4355604378945248] 
\draw [color={rgb, 255:red, 65; green, 117; blue, 5 }  ,draw opacity=1 ][line width=1.5]    (100.67,63.33) -- (160.67,63.33) ;
%Straight Lines [id:da5836238050619307] 
\draw [color={rgb, 255:red, 65; green, 117; blue, 5 }  ,draw opacity=1 ][line width=1.5]    (160,106.67) -- (220,106.67) ;
%Straight Lines [id:da7444590579842632] 
\draw [color={rgb, 255:red, 65; green, 117; blue, 5 }  ,draw opacity=1 ][line width=0.75]    (220,150) -- (280,150) ;

% Text Node
\draw (41,262) node [anchor=north west][inner sep=0.75pt]   [align=left] {$\displaystyle x_{0} =a$};
% Text Node
\draw (281,262) node [anchor=north west][inner sep=0.75pt]   [align=left] {$\displaystyle x_{2^{n}} =b$};


\end{tikzpicture}

        \end{center}
        Более формально, пусть $\{x_k\}_{k=0}^{2^n}$~--- разбиение отрезка $[a,\, b]$
        на отрезки длины $2^{-n}$.
        Пусть $I_k=[x_{k-1},\, x_k)$, тогда \[
            f(x)=\sum_{k=1}^{2^n}1_{I_k}(x)\cdot f(x)\quad \forall x\in[a,\, b].
        \]
        А также пусть \[
            \underline{f}_n(x):=\sum_{k=1}^{2^n}1_{I_k}(x)\cdot \inf_{x\in I_k} f(x),\quad
            \overline{f}_n(x):=\sum_{k=1}^{2^n}1_{I_k}(x)\cdot \sup_{x\in I_k} f(x).
        \]
        Тогда \[
            \underline{f}_n(x)\leqslant f(x)\leqslant \overline{f}_n(x)\quad\forall x\in[a,\, b].
        \]
        По построению, \[
            \forall x\in[a,\, b]\quad \underline{f}_n(x)\leqslant \underline{f}_{n+1}(x)
            \text{ и }
            \overline{f}_n(x)\geqslant \overline{f}_{n+1}(x).
        \]
        Поэтому \[
            \forall x\in[a,\, b]:\ \exists\text{ конечные }
            \underline{f}(x)=\lim_{n\toi}\underline{f}_n(x),\,
            \overline{f}(x)=\lim_{n\toi}\overline{f}_n(x).
        \]

        По критерию Дарбу $\forall \varepsilon>0\ \exists\delta>0$:
        если $2^{-n}<\delta$, то \[
            \int_{a}^b\left(\overline{f}_n(x)-\underline{f}_n(x)\right)dx<\varepsilon.
        \]
        Далее $f$ интегрируема по Риману, следовательно ограничена, то есть
        \[
            \exists C>0:\: |\underline{f}_n|\leqslant C,\,
            |\overline{f}_n|\leqslant C\ \forall n\Rightarrow
            |\underline{f}|\leqslant C,\, |\overline{f}|\leqslant C.
        \]
        Откуда \[
            \left|\overline{f}_n-\underline{f}_n\right|\leqslant 2C.
        \]
        Далее по теореме Лебега об ограниченной сходимости\[
            \int_{[a,\, b]}\left(\overline{f}-\underline{f}\right)d\lambda
            =\lim_{b\toi}\underbrace{\int\left(\overline{f}_n-\underline{f}_n\right)d\lambda}_{\leqslant\varepsilon\text{ при больших } n}
        \]

        Далее нужна следующая лемма
        \begin{lemma}
            Если $f\geqslant 0$ и $\int fd\mu=0$, то $f=0$ почти всюду.

            \begin{proof}

                Используя неравенство Чебышева (см. ниже), получаем \[
                    \mu\left\{x:\: f>\dfrac{1}{n}\right\}\leqslant
                    n\int fd\mu=0\Rightarrow
                    \mu\left\{x:\: f>0\right\}=0,
                \]
                так как \[
                    \mu\left\{x:\: f>0\right\}=\bigcup_{n=1}^{\infty}\left\{x:\: f(x)>\dfrac{1}{n}\right\}.
                \]

            \end{proof}
        \end{lemma}

        \begin{theorem}[Неравенство Чебышева]
            Если $f\geqslant 0$, то \[
                \mu\{x:\: f(x)>c\}\leqslant\dfrac{1}{c}\int fd\mu.
            \]

            \begin{proof}

                Пусть $E=\{x:\: f(x)>c\}$, тогда $c\cdot 1_E\leqslant f$ везде.
                Поэтому \[
                    \int fd\mu\geqslant c\int 1_{E}d\mu=c\mu(E).
                \]

            \end{proof}
        \end{theorem}

        Поэтому \[
            \int_{[a,\, b]}\left(\overline{f}-\underline{f}\right)d\lambda=0\Rightarrow
            \overline{f}-\underline{f}=0\text{ почти всюду.}
        \]

        С другой стороны\[
            \underline{f}(x)\leqslant f(x)\leqslant\overline{f}(x)\quad \forall x,
        \]
        Поэтому, $\forall x$ если $\underline{f}(x)=\overline{f}(x)$,
        то $\underline{f}(x)=\overline{f}(x)=f(x)$.

        Итак, для почти всех $x\in[a,\, b]$ $f(x)=\lim\limits_{n\toi}\overline{f}_n(x)$.
        Следовательно $f$ измерима по Лебегу.
        И по теореме Лебега об ограниченной сходимости \[
            \int fd\lambda=\lim_{n\toi}\int\underline{f}_nd\lambda\xrightarrow[n\toi]{}
            \int_{a}^b fdx.
        \]
        Так как,
        \[
            \int_{a}^b\underline{f}_nd\lambda=\underbrace{2^{-n}\sum_{k=1}^{2^n}f_n(\xi_k)}_
            {\text{нижняя сумма Дарбу}},\quad
            \xi_k\in I_k.
        \]

    \end{proof}
\end{theorem}

\begin{theorem}[критерий Лебега интегрируемости по Риману]
    Ограниченная функция $f:\: [a,\, b]\to\R$ интегрируема по Риману тогда и только тогда,
    когда $f$ непрерывна почти всюду.

    \begin{proof}

        \circled{$\Leftarrow$} Пусть $x\in [a,\, b]$ не является узлом разбиения
        ни для какого $n\in \N$. Тогда $f$ непрерывна $x$ тогда и только тогда, когда
        $\underline{f}(x)=\overline{f}(x)$. В самом деле, если
        $\underline{f}(x)=\overline{f}(x)$, то при достаточно больших $n$
        $\overline{f}_n(x)\leqslant\underline{f}_n(x)+\varepsilon$
        на отрезке разбиения. А значит на отрезке разбиения выполняется $|f(x)-f(y)|<2\varepsilon$.
        В обратную сторону аналогично.

        Итак, если $f$ непрерывна почти всюду, то для почти всех $x$ $\underline{f}(x)=\overline{f}(x)$.
        Следовательно, \[
            \int \underline{f}d\lambda=\int\overline{f}d\lambda\Rightarrow
            \lim_{n\toi}\int_{[a,\, b]}\left(\overline{f}_n-\underline{f}_n\right)d\lambda=
            \int_{[a,\, b]}\left(\overline{f}-\underline{f}\right)d\lambda=0.
        \]
        Поэтому по критерию Дарбу $f$ интегрируема по Риману.

        \circled{$\Rightarrow$} Упражнение.

    \end{proof}
\end{theorem}

