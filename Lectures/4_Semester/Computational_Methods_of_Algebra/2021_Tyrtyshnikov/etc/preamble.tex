\geometry{left=20mm,right=20mm,top=25mm,bottom=20mm} % задание полей текста
\usepackage{wrapfig}
%% Стиль колонтитулов
% \fancyhead[RO,LE]{\hyperlink{intro}{Содержание}} % Right odd,  Left even
\fancyhead[LO]{\@lecture}        % Right even, Left odd
\fancyhead[R]{}

\fancyfoot[RO,LE]{\thepage}         % Right odd,  Left even
\fancyfoot[RE,LO]{\CourseName}      % Right even, Left odd
\fancyfoot[C]{}
% Un~comment these to erase foot (and comment footrulewidth renewcommand)
%\fancyfoot{}
%\fancyhead[C]{-~\thepage~-}
\renewcommand{\footrulewidth}{0.4pt}
\usepackage{textcomp}

% Новая команда \lecture{№ лекции}{название}
% После этой команды весь текст до следующей такой же команды будет
% принадлежать конкретной лекции, имя которой будет в колонтитуле каждой страницы
\usepackage{xifthen}
\def\@lecture{}%
\newcommand{\lecture}[2]{
    \ifthenelse{\isempty{#2}}{%
        \def\@lecture{Лекция #1}%
    }{%
        \def\@lecture{Лекция #1: #2}%
    }%
    \section{\@lecture}
}

\def\@lecture{}%
\newcommand{\question}[2]{
    \ifthenelse{\isempty{#2}}{%
        \def\@lecture{Билет #1}%
    }{%
        \def\@lecture{Билет #1: #2}%
    }%
    \section{\@lecture}
}
% ------------ Text settings ------------
%%% Гиппер ссылки
\renewcommand{\linkcolor}{blue}
\renewcommand{\citecolor}{green}
\renewcommand{\filecolor}{magenta}
\renewcommand{\urlcolor}{NavyBlue}

\usepackage{multicol}	   % Для текста в нескольких колонках

% -----------  Images -----------
\graphicspath{{images/}{img/}{figures/}{fig/}}  % Путь к папкам с картинками
\newcommand{\figL}[3]{%      Для быстрой вставки картинок
    \begin{figure}[h!]
        \centering
        \includegraphics[width=#2\textwidth]{#1}
        \label{fig:#3}
    \end{figure}%
}
\newcommand{\fig}[2]{%    
    \begin{figure}[h!]
        \centering
        \includegraphics[width=#2\textwidth]{#1}
    \end{figure}%
}

% ----------- Math short-cats
\newcommand{\vect}{\ensuremath{\mathbf}}
\newcommand{\R}{\ensuremath{\mathbb{R}}}
\newcommand{\N}{\ensuremath{\mathbb{N}}}
\newcommand{\Cx}{\ensuremath{\mathbb{C}}}
\newcommand{\Z}{\ensuremath{\mathbb{Z}}}
\newcommand{\E}{\ensuremath{\mathbb{E}}}
\newcommand{\Q}{\ensuremath{\mathbb{Q}}}
\newcommand{\F}{\ensuremath{\mathbb{F}}}
\def\CB{\mathcal{B}}
\def\CC{\mathcal{C}}
\def\CE{\mathcal{E}}
\def\CR{\mathcal{R}}
\def\CA{\mathcal{A}}
\def\CF{\mathcal{F}}
\def\CG{\mathcal{G}}
\def\CS{\mathcal{S}}
\def\CD{\mathcal{D}}
\def\CH{\mathcal{H}}
\def\CP{\mathcal{P}}
\def\CM{\mathcal{M}}
\def\FB{\mathfrak{B}}

% You can write your commands below
\usepackage{gensymb}
\usepackage{enumitem}
\usepackage{amsmath}
\DeclareMathOperator{\FDU}{FDU}
\DeclareMathOperator{\fl}{fl}
\DeclareMathOperator{\rk}{rk}
\DeclareMathOperator{\im}{im}
\DeclareMathOperator{\argmin}{argmin}

% ----------- Math and theorems -----------
\usepackage[many]{tcolorbox}
\usepackage{mdframed}
\usepackage[dvipsnames]{xcolor}

\newtheorem*{remark}      {Замечание}
\newtheorem*{next0}      {Следствие}
\newtheorem*{next1}      {Следствие 1}
\newtheorem*{next2}      {Следствие 2}
\theoremstyle{definition}
\newtheorem{lemma}{Лемма}[section]
\newtheorem{claim}[lemma]{Утверждение}
\newtheorem{axiom}[lemma]{Аксиома}
\newtheorem{theorem_}[lemma]{Теорема}
\newenvironment{theorem}%
{\begin{mdframed}[backgroundcolor=black!30!white!30]
        % \setlength{\topsep}{-\parskip}\setlength{\partopsep}{0pt}
        \begin{theorem_}}%
            {\end{theorem_}\end{mdframed}}

\theoremstyle{definition}
\newtheorem{definition}[lemma]{Определение}
\tcolorboxenvironment{definition}{
    enhanced,
    borderline={0.8pt}{0pt}{gray!70},
    borderline={0.4pt}{2pt}{black},
    boxrule=0.4pt,
    colback=pink!70!white!30,
    coltitle=black,
    sharp corners
}
\usepackage{ulem}
\newcommand{\mdef}[1]
{\! \textit{\uwave{\textcolor{red!10!black}{#1}}}}
% http://dkhramov.dp.ua/Comp.LatexCyrillicFonts#.XMrWLegzaUk


%https://tex.stackexchange.com/questions/223694/how-to-draw-a-text-box-with-shadow-borders-in-latex

\newtheorem{exercise_}[lemma]{Пример}
\newenvironment{exercise}%
{\begin{tcolorbox}[enhanced,width=\textwidth,center upper,drop fuzzy shadow southwest,
            colframe=red!50!black,colback=orange!100!yellow!50!white!30]
        \begin{exercise_}}%
            {\end{exercise_}\end{tcolorbox}}

\newtheorem{task_}[lemma]{Задача}
\newenvironment{task}%
{\begin{tcolorbox}[enhanced,width=\textwidth,center upper,drop fuzzy shadow southwest,
            colframe=red!50!black,colback=pink!100!yellow!50!white!30]
        \begin{task_}}%
            {\end{task_}\end{tcolorbox}}

\usepackage{dashrule}
\renewenvironment{proof}{\smallskip{\noindent\Large
        \color{red!50!black}\itshape$\lhd$ \normalsize Начало доказательства
        \hdashrule[0.5ex]{0.65\textwidth}{0.5mm}{3mm 3pt 1mm 2pt} }
    \small\itshape}{{\color{red!50!black}
            \hdashrule[0.5ex]{0.66\textwidth}{0.5mm}{3mm 3pt 1mm 2pt}
            \normalsize Конец доказательства \Large$\rhd$}}

\newenvironment{solution}{\smallskip{\noindent\Large
        \color{red!50!black}\itshape$\lhd$ \normalsize Начало решения
        \hdashrule[0.5ex]{0.75\textwidth}{0.5mm}{3mm 3pt 1mm 2pt} }
    \small}{{\color{red!50!black}
            \hdashrule[0.5ex]{0.75\textwidth}{0.5mm}{3mm 3pt 1mm 2pt}
            \normalsize\itshape Конец решения \Large$\rhd$}}

% Обводка кружочком множеств
\usepackage{tikz}
\usetikzlibrary{decorations.pathreplacing,shapes.misc}
\newcommand*\circled[1]{\tikz[baseline=(char.base)]{
        \node[shape=circle,draw,inner sep=2pt] (char) {#1};}}

\counterwithin*{equation}{section}

\renewcommand{\thempfootnote}{\arabic{mpfootnote}}