\section{Основы теории делимости}

\subsection{Повторение изученного}

\begin{example}
	Рассмотрим несколько примеров, демонстрирующих полезность теории колец и полей:
	\begin{enumerate}
		\item Пусть $n = m_1 \dotsm m_s$, где $m_1, \dotsc, m_s$ "--- попарно взаимно простые натуральные числа. Рассмотрим кольцо $\Z_n$ и заметим, что $\Z_n \cong \Z_{m_1} \times \dotsb \times \Z_{m_s}$, поскольку имеет место гомоморфизм $a \mapsto (a \md m_1, \dotsc, a \md m_s)$ с нулевым ядром. Из этого следует, что $\Z_n^* \cong \Z_{m_1}^* \times \dotsb \times \Z_{m_s}^*$. Тогда, в частности, $\phi(n) =  |\Z_n^*| \hm= |\Z_{m_1}^*|\dotsm|\Z_{m_s}^*| \hm= \phi(m_1)\dotsm\phi(m_s)$, где $\phi$ "--- функция Эйлера.
		\item Пусть $F$ "--- поле, $G$ "--- конечная подгруппа в $F^*$. Покажем, что $G$ "--- циклическая. Для произвольного $d\mid |G|$ положим $\psi(d) := |\{g \in G: \ord{g} = d\}|$. Все элементы группы $G$ являются корнями многочлена $x^{|G|} - 1$. Поскольку $d\mid |G|$, то $x^{d} - 1 \mid x^{|G|} - 1$, поэтому все корни многочлена $x^d - 1$ лежат в $G$. Тогда число корней многочлена $x^d - 1$ равно $\sum_{k \mid d} \psi(k)$, и, по формуле обращения Мебиуса, $\psi(d) = \phi(d)$. В частности, $\psi(|G|) = \phi(|G|) \ge 1$, что и означает требумое.
	\end{enumerate}
\end{example}

\begin{definition}
	\textit{Кольцом} называется множество $R$ с определенными на нем бинарными операциями $+$ и $\cdot$ такими, что:
	\begin{enumerate}
		\item $(R, +)$ "--- абелева группа с нейтральным элементом 0
		\item $\forall a, b, c \in R: a(b + c) = ab + ac$, $(a + b)c = ac + bc$ (сложение \textit{дистрибутивно} относительно умножения)
	\end{enumerate}
\end{definition}

\begin{definition}
	Кольцо $R$ называется:
	\begin{itemize}
		\item \textit{ассоциативным кольцом}, если $\forall a, b, c \in R: a(bc) = (ab)c$ (умножение \textit{ассоциативно})
		\item \textit{кольцом с единицей}, если в нем есть нейтральный элемент относительно умножения, обозначаемый через 1
		\item \textit{коммутативным кольцом}, если $R$ "--- ассоциативное кольцо с единицей такое, что $\forall a, b \in R: ab = ba$ (умножение \textit{коммутативно})
	\end{itemize}
\end{definition}

\begin{definition}
	Пусть $K$ "--- коммутативное кольцо. Элемент $a \in K$ называется \textit{обратимым}, если $\exists b \in K: ab = 1$. Тогда $K^* = \{a \in K: a \text{ обратим}\}$ "--- \textit{мультипликативная группа} кольца.
\end{definition}

\begin{definition}
	Пусть $K$ "--- коммутативное кольцо. Элемент $a \in K$ называется \textit{делителем нуля}, если $a \ne 0$ и $\exists b \in K, b \ne 0: ab = 0$. Если в $K$ нет делителей нуля, $K$ называется \textit{областью целостности}.
\end{definition}

\begin{proposition}
	Пусть $K$ "--- область целостности, $c \in K$, $c \ne 0$. Тогда если $ac = bc$, то $a = b$.
\end{proposition}

\begin{proof}
	По условию $ac - bc = (a - b)c = 0$. Тогда, поскольку $K$ "--- область целостности, $a - b = 0 \lra a = b$.
\end{proof}