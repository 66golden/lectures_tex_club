\subsection{Основная теорема теории Галуа}

\textbf{До конца раздела} зафиксируем поле $F$ и его расширение Галуа $L \supset F$.

\begin{theorem}[основная теорема теории Галуа]
	Сопоставление $K \mapsto \Aut_{K}L$ (и обратное к нему сопоставление $H \mapsto L^H$) осуществляет биекцию между полями $K$ такими, что $F \subset K \subset L$, и подгруппами $H \le \Aut_F{L}$, причем для соответствующих полей $K$ и подгрупп $H$ выполнено равенство $[\Aut_FL : H] = [K : F]$. Более того, расширение $L^H \supset F$ нормально $\lra H \normal \Aut_F{L}$.
\end{theorem}

\begin{proof}
	Проверим, что сопоставления $K \mapsto \Aut_{K}L$ и $H \mapsto L^H$ взаимно обратны:
	\begin{itemize}
		\item Пусть $K \mapsto \Aut_{K}L \mapsto L^{\Aut_{K}L}$. Расширение $L \supset K$ нормально в силу нормальности расширения $L \supset F$, а также сепарабельно и конечно, поэтому $L^{\Aut_K{L}} = K$.
		\item Пусть $H \mapsto L^H \mapsto \Aut_{L^H}L$. Очевидно, $H \subset \Aut_{L^H}L$. Предположим, что $H \subsetneq \Aut_{L^H}L$. По теореме о примитивном элементе, $L = L^H(\gamma)$ для некоторого $\gamma \in L$. Рассмотрим многочлен $f(x) := \prod_{h \in H}(x - h(\gamma)) \in L[x]$. Поскольку многочлен $f$ сохраняется под действием любого автоморфизма из $H$, то все его коэффициенты лежат в $L^H$. Но $f(\gamma) = 0$, поэтому минимальный многочлен $m_\gamma \in L^H[x]$ делит $f$. Тогда, поскольку $L \supset L^H$ является расширением Галуа, выполнено $\deg{m_\gamma} \le \deg{f} = |H| < |\Aut_{L^H}L| \hm= [L : L^H] = \deg{m_\gamma}$ --- противоречие.
	\end{itemize}
	
	Значит, сопоставление является биекцией. Равенство $[\Aut_FL : H] = [K : F]$ очевидно, поскольку $|\Aut_KL| = [L : K]$ и $|\Aut_FL| = |L : F|$. Наконец, заметим, что $\forall g \in \Aut_FL$ выполнено $L^{gHg^{-1}} = \{x \in L: \forall h \in H: h(g^{-1}(x)) = g^{-1}(x)\}= \{g(x) : x \in L^H\} = g(L^H)$. Значит, $H \normal \Aut_FL \lra \forall g \in \Aut_FL: g(L^H) = L^H \lra$ расширение $L^H \supset F$ нормально. Последний переход слева направо выполнен потому, что изоморфизм $\phi: F(\alpha) \to F(\widetilde\alpha)$, сохраняющий $F$ и переводящий $\alpha \in L^H$ в сопряженный с ним элемент $\widetilde\alpha \in L$, можно продолжить до некоторого гомоморфизма $\widetilde\phi: L \to \overline{F}$, причем, в силу нормальности расширения $L \supset F$, $\widetilde\phi$ является автоморфизмом поля $L$.
\end{proof}