\subsection{Симметрические многочлены}

\textbf{До конца раздела} зафиксируем поле $F$ и кольцо многочленов $F[x_1, \dotsc, x_n]$.

\begin{definition}
	Группа перестановок $S_n$ действует на множестве $F[x_1, \dotsc, x_n]$ следующим образом: $\forall \sigma \in S_n: \forall f \in F[x_1, \dotsc, x_n]: \sigma(f)(x_1, \dotsc, x_n) := f(x_{\sigma(1)}, \dotsc, x_{\sigma(n)})$. Многочлен $f \in F[x_1, \dotsc, x_n]$ называется \textit{симметрическим}, если он является неподвижной точкой действия, то есть $\forall \sigma \in S_n: \sigma(f) = f$. Множество симметрических многочленов обозначается через $F[x_1, \dotsc, x_n]^{S_n}$.
\end{definition}

\begin{example}
	Рассмотрим несколько примеров симметрических многочленов:
	\begin{enumerate}
		\item $\sigma_k(x_1, \dotsc, x_n) := \sum_{1\le i_1 < \dotsb < i_n \le n}x_{i_1}
		\dotsm x_{i_n}$, где $k \in \N$, "--- \textit{$k$-ый элементарный симметрический многочлен}
		\item $x_1^k + \dotsb + x_n^k$, где $k \in \N$
		\item Обобщение предыдущих примеров "--- многочлен вида $\sum_{\sigma \in S_n}\sigma(f)$ для произвольного $f \in F[x]$
	\end{enumerate}
\end{example}

\begin{theorem}
	$F[x_1, \dotsc, x_n]^{S_n} = F[\sigma_1, \dotsc, \sigma_n]$.
\end{theorem}

\begin{proof}
	Зададим на мономах отношение линейного порядка следующим образом: для любых $\alpha, \beta \in F^*$ неравенство $\alpha x_1^{l_1}\dotsm x_n^{l_n} \preccurlyeq \beta x_1^{m_1}\dotsm x_n^{m_n}$ выполнено тогда и только тогда, когда выполнено неравенство $(l_1 + \dotsb + l_n, l_1, 
	\dotsc, l_n) \le_{lex} (m_1 + \dotsb + m_n, m_1, 
	\dotsc, m_n)$, где $\le_{lex}$ "--- стандартный лексикографический порядок. Ясно, что тогда мономы образуют вполне упорядоченное множество.
	
	Будем доказывать, что любой симметрический многочлен $f \in F[x_1, \dotsc, x_n]^{S_n}$ представляется в виде $h(\sigma_1, \dotsc, \sigma_n)$ для некоторого $h \in F[x_1, \dotsc, x_n]$, индукцией по наибольшему моному в $f$. База, $f = \alpha \in F$, тривиальна.
	
	Пусть теперь $f \ne \alpha \in F$, и $\beta x_1^{m_1}\dotsm x_n^{m_n}$ "--- наибольший моном в $f$. Тогда, в силу симметричности многочлена $f$, выполнены неравенства $m_1 \ge \dotsc \ge m_n$. Рассмотрим $g := \beta\sigma_1^{m_1 - m_2}\dotsm\sigma_{n - 1}^{m_{n - 1} - m_n}\sigma_n^{m_n} \in F[x_1, \dotsc, x_n]^{S_n}$. Он однороден, поэтому моном $\beta x_1^{m_1}\dotsm x_n^{m_n}$ в нем является наибольшим. Значит, в симметрическом многочлене $f - g$ наибольший моном меньше, и к $f - g$ можно применить предположение индукции.
\end{proof}

\begin{example}
	Теорема выше позволяет решать кубические уравнения. Рассмотрим многочлен $x^3 + ax^2 + bx + c \in \Q[x]$ и обозначим его корни через $x_1, x_2, x_3 \in \Cm$. Тогда $\sigma_i(x_1, x_2, x_3) \in \Q$ при $i \in \{1, 2, 3\}$ по теореме Виета, поэтому значение любого симметрического многочлена от $x_1, x_2, x_3$ рационально. Приведем два метода, использующих это наблюдение:
	\begin{enumerate}
		\item Заметим, что многочлен $D := ((x_1 - x_2)(x_2 - x_3)(x_3 - x_1))^2 \in \Q[x_1, x_2, x_3]$ "--- симметрический и потому выражается через $a, b, c$. Тогда, аналогично случаю квадратных уравнений, корни $x_1, x_2, x_3$ можно выразить через значение $D$ и $a, b, c$, хотя формула и получится более громоздкой.
		\item Произведем замену переменных:
		\begin{align*}
			&y_1 := x_1 + x_2 + x_3\\
			&y_2 := x_1 + \omega x_2 + \omega^2 x_3\\
			&y_3 := x_1 + \omega^2 x_2 + \omega x_3
		\end{align*}
	
		Заметим, что при действии группы $S_3$ на множестве $\Cm[x_1, x_2, x_3]$ орбита многочлена $y_2$ имеет вид $S_3(y_2) = \{y_2, \omega y_2, \omega^2 y_2, y_3, \omega y_3, \omega^2 y_3\}$. Тогда $S_3(y_2^3) = S_3(y_3^3) = \{y_2^3, y_3^3\}$, поэтому многочлены $y_2^3 + y_3^2$ и $y_2^3y_3^3$ "--- симметрические и выражаются через $a, b, c$. Значит, нам достаточно решить квадратное уравнение относительно $y_2^3, y_3^3$ и затем линейно выразить $x_1, x_2, x_3$ через $y_1, y_2, y_3$.
	\end{enumerate}
\end{example}

\begin{note}
	Оба метода выше можно обобщить на уравнения четвертой степени, но они станут технически сложными. Однако заметим, что $\{x_1x_2 + x_3x_4, x_1x_3 + x_2x_4, x_1x_4 + x_2x_3\}$ является одной из орбит действия группы $S_4$ на множестве $\Q[x_1, \dotsc, x_4]$. Обозначим эти многочлены через $f_1, f_2, f_3$. Из $f_1, f_2, f_3$ можно сконструировать симметрические многочлены $f_1 + f_2 + f_3$, $f_1f_2 + f_2f_3 + f_3f_1$, $f_1f_2f_3$ и свести задачу к решении кубического уравнения относительно $f_1, f_2, f_3$. Зная значения $f_1, f_2, f_3$, можно найти и значения $x_1, \dotsc, x_4$. Аналогично задачу можно было решить, рассмотрев множество многочленов $\{(x_1 + x_2)(x_3 + x_4), (x_1 + x_3)(x_2 + x_4), (x_1 + x_4)(x_2 + x_3)\}$.
	
	На самом деле, существование орбиты малого размера связано с тем, что в $S_4$ есть нормальная подгруппа $V_4$ малого порядка. Эта идея будет подробно рассмотрена далее.
\end{note}