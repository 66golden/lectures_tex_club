Дифференциальные уравнения, описывающие физические
процессы, всегда содержат некоторые параметры (масса,
упругость и т.д.). Эти параметры в реальных задачах
никогда не могут быть измерены абсолютно точно, т.е.
всегда измеряются с некоторой погрешностью, так что
сами дифференциальные уравнения известны лишь с некоторой степенью точности. Поэтому, для того чтобы
уравнения могли описывать реальные процессы, необходимо, чтобы их решения непрерывно зависели от параметров, т.е. чтобы они мало менялись при малых изменениях параметров.
\bigbreak
Рассмотрим задачу Коши для нормальной системы дифференциальных уравнений в векторном виде т.е. $ \vec{f}=(f_1,\ldots,f_n)$:
\setcounter{equation}{0}
\begin{equation}\label{eq_0}
    \frac{d\vec{y}}{dx}=\vec{f}(x,\vec{y},\vec{\mu}), \qquad \vec{y}(x_0,\vec{\mu})=\vec{y_0}(\vec{\mu})
\end{equation}
где $\mu$ -- параметр, $\mu_0$ -- задан.
\bigbreak
\par \Th  Пусть $\vec{f}(x,\vec{y},\vec{\mu})$ -- непрерывна и удовлетворяет условию Липшица равномерно по $x$ и $\vec{\mu}$, $\forall (x,\vec{y}) \in G \subseteq \R^{n+1}$ и всех $\vec{\mu}$, таких что $|\vec{\mu} - \vec{\mu}_0| \leqslant \delta$. Пусть кроме того, $(x_0,\vec{y_0}) \in G$. \newline Тогда $\exists h > 0 \;\mapsto$ решение $\vec{y}(x,\vec{\mu})$ задачи Коши (\ref{eq_0}) непрерывно по совокупности переменных $(x,\vec{\mu})$ в некоторой области $|x-x_0|\leqslant h$, $|\vec{\mu}-\vec{\mu}_0|\leqslant \delta$.
\bigbreak
\Note Интегральные кривые уравнения в этой теореме образуют семейство кривых, проходящих через точку $(x_0,\vec{y_0})$. Теорема утверждает, что интегральные кривые, отвечающие близким значениям параметра $\vec{\mu}$, близки.