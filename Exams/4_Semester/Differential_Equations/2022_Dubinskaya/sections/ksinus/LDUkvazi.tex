\Def Эти уравнения имеют вид
\[y^{(n)}+a_1y^{(n-1)}+a_2y^{(n-2)}+...+a_n y = f(x),\]
где $f(x)$ квазимногочлен: $f(x) = e^{\mu x} P_m(x)$, $\mu \in \mathbf{C}$ $P_m(x)$ - заданный многочлен степени $m$ с комплексными коэффициентами. 

\Def Характеристическим многочленом $L(x)$ назовём многочлен
\[L(X) = a_n x^n + a_{n-1} x^{n-1} + ... + a_0\]

\Def Дифференциальным оператором $D$ назовём оператор
\[D = \frac{d}{dx}\]

\Note $D^n y = y^{(n)}$ 

Существование и единственность решения следуют из таковых для системы

\begin{equation*}
    \begin{cases}
        \vec{y'}_1 = y_2&\\
        \vec{y'}_2 = y_3&\\
        \ldots&\\
        \vec{y'}_{n-1} = y_n&\\
        \vec{y'}_n = f - a_1 y_1 - a_{2}y_2 - \ldots - a_n y_n
    \end{cases}
\end{equation*}

(здесь $y_1 = y$)
\bigbreak
\Def Если число $\mu$ является корнем характеристического уравнения 
\[L(\lambda) = \lambda^n + a_1 \lambda^{n-1}+...+ a_n = 0\]
то говорят, что в уравнении резонансный случай. Если же $\mu$ не является корнем, то имеем нерезонансный случай.

\Def Дифференциальным многочленом назовём многочлен вида 
\[L(D) = (D-\lambda_1)^{k_1} (D-\lambda_2)^{k_2} ... (D-\lambda_s)^{k_s},\]
Где $k_s$ соответствующие кратности корней характеристического уравнения
\bigbreak
Рассмотрим ЛОУ. Покажем, что если известно некоторое решение $y_0(x)$ ЛНУ, то замена $y = z + y_0$ приводит уравнение к ЛОУ. Воспользуемся представлением левой части через дифференциальный многочлен:

\[L(D)y=L(D)(z+y_0) =L(D)z + L(D) y_0 = L(D)z + f(x) = f(x)\]

Откуда следует, что $L(D)z = 0$, т.е. решение.

Рассмотрим $L(D) y(x) = e^{\mu x} P_m(x)$.
\bigbreak
\Statement $(P_m(x)e^{\lambda x})'_x = Q_m(x)e^{\lambda x}$

\begin{theorem}[О структуре ФСР]
Пусть $\lambda_1, ..., \lambda_k$ корни характеристического многочлена кратности $l_1, ..., l_k$. Тогда набор функций $x^s e^{\lambda_i x}$, где $s = 0,..., l_1-1$, $i = 1, ..., k$ является ФСР для рассматриваемого уравнения
\end{theorem}
Доказано в пункте~<<\nameref{firstthird-anchor}>>.

\begin{theorem}[О структуре решения ЛНУ c правой частью в виде квазимногочлена]
Для рассматриваемого уравнения существует и единственно решение вида
\[y(x) = x^k e^{\mu x} Q_m(x) \]
где $ Q_m(x)$ - многочлен одинаковой с $P_m(x)$ степени $m$, а число $k$ равно кратности корня $\mu$ в уравнении $L(\lambda)=0$ в резонансном случае и $k=0$ в нерезонансном
\end{theorem}

\Proof
Если $\mu \neq 0$, то заменой $y = z e^{\mu x}$ всегда можно избавиться от $e^{\mu x}$ в правой части. В самом деле, по формуле сдвига после замены имеем что \[L(D) y = L(D)(e^{\mu x} z) = e^{\mu x} L(D+\mu) z = e^{\mu x}  P_m(x),\]
откуда $L(D+\mu)z=P_m(x)$.

Таким образом, доказательство теоремы осталось провести для уравнения вида
\[L(D)y=P_m(x)\]

\begin{enumerate}
    \item Нерезонансный случай: $L(\mu) \neq 0$. Пусть
    \[P_m(x) = p_m x^m + ... + p_0\]
    \[Q_m(x) = q_m x^m + ... + q_0\]
    Если подставить и приравнять коэффициенты при одинаковых степенях $x$, получим линейную алгебраическую систему уравнения для опредения неизвестных коэффициентов $q_0, ... q_m$. Матрица системы треугольная с числами $a_n = L(0) \neq 0$, таким образом, все коэффициенты определяются из неё одназначно.
    \item В резонансном случае имеем 
    \[L(\lambda) = \lambda^k (\lambda^{n-k} + a_1 \lambda^{n-k-1} + ...+a_{n-k}) \]
    Следовательно,
    \begin{equation*}
        L(D) = 
        \begin{cases}
           D^n + a_1 D^{n-1} + ... + a_{n-k} D^k, k < n\\
           D^n, k = n
        \end{cases}
    \end{equation*}
    В первом случае замена $D^k y = z$ приводит к уравнению с нерезонансным случаем. Рассмотрим уравнение
    \begin{equation*}
        D^k(y) = 
        \begin{cases}
           R_m(x), k < n\\
           P_m(x), k = n
        \end{cases}
    \end{equation*}
    Взяв нулевые начальные условия для этого уравнения
        \[y(x) = y'(0) = ... = y^{(k-1)} (0) = 0\]
        получим единственное решение вида 
    \[y(x) = x^k Q_m(x)\]
\end{enumerate}
\EndProof

\textbf{О вещественнозначной ФСР}
Для уравнения с вещественнозначными коэффициентами комплексные корни $M(\lambda)$ распадаются на сопряжённые пары одинаковой кратности. Соответствующие им решения легко
заменяются на вещественные функции (по аналогии с однократными действительными корнями $M(\lambda)$).

