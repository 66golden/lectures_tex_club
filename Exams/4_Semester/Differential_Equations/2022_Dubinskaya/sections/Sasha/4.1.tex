\setcounter{equation}{0}

Линейным дифференциальным уравнением с переменными коэфициентами порядка $n$ называется уравнение
\begin{equation}\label{varcoef-eq}
    y^{(n)} + a_1(x)y^{(n-1)} + \ldots + a_n(x)y = f(x)
\end{equation}
где \(x \in I\) и \(a_j(x), f(x)\)  --- непрерывные функции на \(I\). Если \(f(x) \equiv 0\), то такое уравнение называется однородным.

\begin{lemma}\label{varcoef-solsumlemma}
Если  $f(x) = f_1(x) + f_2(x)$ и $y_i(x)$ ~--- решение уравнения (\ref{varcoef-eq}) при $f(x) \equiv f_i(x)$ на $[\alpha; \beta], i = 1, 2$, то функция \(y(x) = y_1(x) + y_2(x)\) является решением уравнения (\ref{varcoef-eq}).
\end{lemma}
\begin{corollary}[Принцип суперпозиции]
Если $y_1(x), y_2(x)$~--- решения однородного уравнения (\ref{varcoef-eq}) (при $f \equiv 0$), то линейная комбинация $y = c_1 \cdot y_1(x) + c_2 \cdot y_2(x)$ также решение однородного уравнения (\ref{varcoef-eq}).
\end{corollary}
Решение уравнения (\ref{varcoef-eq}) всегда можно свести к решению системы линейных уравнений порядка $n$ следующего вида
\begin{equation}\label{varcoef-sys}
    \vec{y'}(x) = A(x)\vec{y}(x) + \vec{f}(x)
\end{equation}
где 
\[
y(x) =
\begin{pmatrix}
y_1(x) \\
\vdots \\
y_n(x) 
\end{pmatrix},
A(x) = \begin{pmatrix}
0 & 1 & 0 & \ldots & 0 \\
0 & 0 & 1 & \ldots & 0 \\
\ldots & \ldots & \ldots & \ldots & \ldots \\
0 & 0 & 0 & \ldots & 1 \\
-a_n(x) & -a_{n-1}(x) & -a_{n-2}(x) & \ldots & -a_1(x)
\end{pmatrix},
\vec{f}(x) =
\begin{pmatrix}
0 \\
\vdots \\
0  \\
f(x)
\end{pmatrix}.
\]

Давайте поймем что происходит в этой системе. Строки матрицы $A$ умножаются на столбец $y(x)$ и полученные суммы сравниваются с вектором $f(x)$. Последняя строчка матрицы $A$ в итоге даст условие 
\[
y'_n(x) = -a_n(x) \cdot y_1(x) + \ldots - a_1(x) \cdot y_n(x) + f(x).
\]
Все остальные строки матрицы дадут условия вида \(y'_i = y_{i + 1}, i \in [1; n-1].\) Получится цепочка, связывающая \(y_1, \ldots y_n\) и их производные.

\begin{lemma}\label{varcoef-equivlemma}
Уравнение (\ref{varcoef-eq}) эквивалентно системе (\ref{varcoef-sys}).
\end{lemma}
\begin{proof}
Пусть \(y = \varphi(x)\)~--- решение (\ref{varcoef-eq}). Положим
\[y_1(x) = \varphi(x), y_2(x) = \varphi'(x), \ldots, y_n(x) = \varphi^{(n-1)}(x).\]
Тогда вектор-функция с компонентами \(\varphi(x), \varphi'(x), \ldots, \varphi^{(n-1)}(x)\) удовлетворяет системе (\ref{varcoef-sys}). Наоборот, если вектор-функция с компонентами \(\varphi(x), \varphi'(x), \ldots, \varphi^{(n-1)}(x)\)~--- решение системы (\ref{varcoef-eq}), то, исключив из (\ref{varcoef-sys}) переменные \(y_2, \ldots,y_n\) получаем, что \(y_1 = \varphi(x)\)~--- решение уравнения (\ref{varcoef-eq}).
\end{proof}

Теперь, благодаря лемме~\ref{varcoef-equivlemma} лемма~\ref{varcoef-solsumlemma} и ее следствие следуют из эквивалентных фактов для системы~(\ref{varcoef-sys}). Принцип суперпозиции и лемма~\ref{varcoef-solsumlemma}  проверяются непосредственной подстановкой и по-сути прямое следствие линейности производной.