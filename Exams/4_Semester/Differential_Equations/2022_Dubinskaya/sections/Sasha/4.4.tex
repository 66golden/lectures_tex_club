Пусть матрица $\text{Ф}(x)$~--- фундаментальная матрица системы~(\ref{varcoef-odn-sys}). Из теоремы~\ref{varcoef-solsysth} получаем самое важное свойство: общее решение~(\ref{varcoef-odn-sys}) записывается в виде \[\vec{y}(x) = \text{Ф}(x) \cdot \vec{C},\]
где $\vec{C}$~--- произвольный числовой вектор размера $n$.

Пусть \(\vec{y}_1(x), \ldots, \vec{y}_n(x)\)~--- система вектор-функций, определенная на $I$ с $n$ компонентами. 

\begin{definition}
Определителем Вронского (или сокращенно вронскианом) системы \[\vec{y}_1(x), \ldots, \vec{y}_n(x)\] называется определитель
\[W(x) \equiv W[\vec{y}_1(x), \ldots, \vec{y}_n(x)] = \det
\begin{Vmatrix}
| &  & |\\
 \vec{y}_1(x) & \ldots &  \vec{y}_n(x)\\
| &  & |
\end{Vmatrix}.\]
\end{definition}

Если \( \vec{y}_1(x), \ldots,  \vec{y}_n(x)\)~--- решения линейной однородной системы~(\ref{varcoef-odn-sys}), то из теоремы~\ref{varcoef-dependencyth} вытекает следующая связь между линейной зависимостью \(\vec{y}_1(x), \ldots, \vec{y}_n(x)\) и обращением в ноль вронскиана. 

\begin{itemize}
    \item Решения~(\ref{varcoef-odn-sys}) \(\vec{y}_1(x), \ldots, \vec{y}_n(x)\)~--- линейно зависимы на $I$ тогда и только тогда когда $W(x) \equiv 0$ на $I$.
    \item Решения~(\ref{varcoef-odn-sys}) \(\vec{y}_1(x), \ldots, \vec{y}_n(x)\)~--- линейно независимы на $I$ тогда и только тогда когда $W(x) \neq 0$ на $I$.
\end{itemize}

Важно, что это верно именно для решений системы~(\ref{varcoef-odn-sys}) и неверно для произвольных вектор-функций.

\begin{theorem}[О структуре решения неоднородной системы]\label{varcoef-neodnor-structure}
Пусть $y_0(x)$~--- некоторое частное решение системы~(\ref{varcoef-sys}) и $\text{Ф}(x)$~--- фундаментальная матрица соотвествующей~(\ref{varcoef-sys}) однородной системы~(\ref{varcoef-odn-sys}). Тогда все решения системы~(\ref{varcoef-sys}) задаются формулой
\[
\vec{y}(x) = \vec{y}_0(x) + \text{Ф}(x) \vec{C},
\]
где $\vec{C}$~--- произвольный числовой вектор размерности $n$.
\end{theorem}
\begin{proof}
В системе~(\ref{varcoef-sys}) сделаем замену 
\[\vec{y}(x) = \vec{z}(x) + \vec{y}_0(x).\]
Тогда получим, что $\vec{z}(x)$ удовлетворяет однородной системе~(\ref{varcoef-odn-sys}). Общее решение системы~(\ref{varcoef-odn-sys}) записывается как 
\begin{equation}
    \vec{z}(x)  = \text{Ф}(x) \vec{C}.
\end{equation}
Из замены следует утверждение теоремы.
\end{proof}
