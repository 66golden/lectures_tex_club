Теперь мы рассматриваем однородную систему
\begin{equation}\label{varcoef-odn-sys}
    \vec{y'}(x) = A(x)\vec{y}(x)
\end{equation}
где \(x \in I\) и \(A(x)\)  --- непрерывны на $I$. Комплекснозначная матрица \(A(x)\) порядка $n$. Решением системы будет являться некоторая комплекснозначная вектор-функция $y$.

Комплекснозначное уравнение можно свести к паре действительнозначных уравнений, поэтому в дальнейшем будут рассматриваться только второй вид уравнений.

Непосредственно проверяется принцип суперпозиции: 
\begin{lemma}[Принцип суперпозиции]
Если $y_1(x), y_2(x)$~--- решения системы~(\ref{varcoef-odn-sys}), то линейная комбинация $y = c_1 \cdot y_1(x) + c_2 \cdot y_2(x)$ также решение.
\end{lemma}

Далее, пусть на действительном промежутке $I$ заданы вектор-функции \(y_1, \ldots, y_k\) с $n$ компонентами.

\begin{definition}
Вектор-функции \(y_1, \ldots, y_k\) называются линейно зависимыми на промежутке $I$, если найдутся числа \(c_1, \ldots, c_k\) одновременно неравные нулю, что
\[
c_1 y_1(x) + \ldots + c_k y_k(x) \equiv 0, \forall x \in I.
\]
В противном случае \(y_1, \ldots, y_k\) называются линейно независимыми.
\end{definition}

\begin{lemma}\label{varcoef-dependencylemma}
Если \(y_1, \ldots, y_k\) линейно зависимы на промежутке $I$, то числовые вектора
\[y_1(x), \ldots, y_k(x), \forall x \in I\] линейно зависимы. Обратное утверждение неверно.
\end{lemma}

\begin{definition}
Любая система из $n$ линейно независимых на $I$ решений системы~(\ref{varcoef-odn-sys}) называется фундаментальной системой решений~(\ref{varcoef-odn-sys}).
\end{definition}

\begin{theorem}\label{varcoef-dependencyth}
Пусть \(y_1(x), \ldots, y_k(x)\)~--- решения линейной однородной системы~(\ref{varcoef-odn-sys}). Эти решения линейно независимы на $I$ тогда и только тогда когда 
\[\forall x_0 \in I, y_1(x_0), \ldots, y_k(x_0)\]
линейно независимы как числовые векторы.
\end{theorem}
\begin{proof}
\fbox{$\Longrightarrow$} Пусть \(y_1(x), \ldots, y_k(x)\)~--- линейно независимые решения линейной однородной системы~(\ref{varcoef-odn-sys}). Если существует \(x_0 \in I\), что \(y_1(x_0), \ldots, y_k(x_0)\) линейно зависимые, то найдется нетривиальная линейная комбинация, что
\[c_1 y_1(x_0) + \ldots + c_k y_k(x_0) = 0.\]
Вектор-функция \[y = c_1 y_1(x) + \ldots + c_k y_k(x)\] является решением системы~(\ref{varcoef-odn-sys}) по принципу суперпозиции и также удовлетворяет начальному условию $y(x_0) = 0$. Но тогда $y(x) \equiv 0$ на $I$ по теореме существовании и единственности. Но тогда \(y_1(x), \ldots, y_k(x)\) линейно зависимы. Противоречие.

\fbox{$\Longleftarrow$} Пусть \(\forall x_0 \in I, y_1(x_0), \ldots, y_k(x_0)\) линейно независимы на $I$. Пусть \(y_1(x), \ldots, y_k(x)\) линейно зависимы, то по лемме~\ref{varcoef-dependencylemma} \(y_1(x_0), \ldots, y_k(x_0)\) также линейно зависимы. Противоречие.
\end{proof}

\begin{theorem}
Для системы~(\ref{varcoef-odn-sys}) фундаментальная система существует и их бесконечное число 
\end{theorem}
\begin{proof}
Фиксируем $x_0 \in I$ и $n$ линейно независимых числовых векторов \(y_1^{(0)}, \ldots, y_n^{(0)}\) с $n$ компонентами. Обозначим через $\varphi_j(x)$ решение системы~(\ref{varcoef-odn-sys}), удовлетворяющее начальному условию $y_j(x_0) = y_j^{(0)}$. По теореме существования и единственности каждое такое решение существует и единственно на $I$. Но тогда система решений $\varphi_j(x)$ образует фундаментальную систему решений. Так как если бы она была линейно зависимой, то и числовые векторы \(y_1^{(0)}, \ldots, y_n^{(0)}\) были линейно зависимы. Точку $x_0 \in I$ и векторы \(y_1^{(0)}, \ldots, y_n^{(0)}\) можно выбирать бесконечным числом способов.
\end{proof}

\begin{theorem}\label{varcoef-solsysth}
Если \(\varphi_1^{(0)}, \ldots, \varphi_n^{(0)}\)~--- фундаментальная система решений системы~(\ref{varcoef-odn-sys}), то любое решение представимо в виде линейной комбинации \(\varphi_1^{(0)}, \ldots, \varphi_n^{(0)}\).
\end{theorem}
\begin{proof}
Пусть $x_0 \in I$, $y$~--- некоторое решение системы. Тогда \(\varphi_1^{(0)}(x_0), \ldots, \varphi_n^{(0)}(x_0)\) линейно независимы по определению и числовой вектор $y(x_0)$ единственным образом выражается через линйеную комбинацию векторов \(\varphi_1^{(0)}, \ldots, \varphi_n^{(0)}\). В силу единственности задачи Коши, коэффициенты линейной комбинации окажутся одними и теми же для всех точек отрезка.
\end{proof}

\begin{definition}
Матрица $\text{Ф}(x)$ у которой столбцы образуют фундаментальную систему решений \(\varphi_1^{(0)}(x_0), \ldots, \varphi_n^{(0)}(x_0)\), называется фундаментальной матрицей системы (\ref{varcoef-odn-sys}).
\end{definition}

Таким образом,
\(\text{Ф}(x) =
\begin{Vmatrix}
\varphi_1^{(0)}(x_0) & \ldots & \varphi_n^{(0)}(x_0)
\end{Vmatrix}
\).

\begin{lemma}
Если $Y_1(x)$ и $Y_2(x)$~--- фундаментальные матрицы одной системы~(\ref{varcoef-odn-sys}), то существует невырожденая числовая матрица $C$, что $Y_1 = Y_2 C$
\end{lemma}