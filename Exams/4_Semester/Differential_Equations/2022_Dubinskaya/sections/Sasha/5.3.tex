Нелинейная автономная система задается

\begin{equation}\label{autonom-nonlin-sys}
    \begin{cases}
    \dot{x_1} = f_1(x_1, x_2)\\
    \dot{x_2} = f_2(x_1, x_2),
    \end{cases}
\end{equation}
где $f_1, f_2$~--- действительные дважды непрерывно дифференцируемые функции в некоторой области $\Omega$.

\begin{definition}
Две определенные на областях $\Omega_1$ и $\Omega_2$ системы будем называть качественно эквивалентными если существует взаимно однозначное и взаимно непрерывное отображение $F$ области $\Omega_1$ на $\Omega_2$, при котором каждая фазовая траектория первой системы с сохранением ориентации переходит в некую фазовую траекторию второй системы и наоборот. Если $F$ непрерывно дифференцируемо, то системы называются дифференцируемо эквивалентными.
\end{definition}

Пусть положение равновесия $x=0$. Разложим $f_1, f_2$ в окрестности положения равновесия по формуле Тейлора с остаточным членом в форме Пеано:
\begin{align*}
    f_1(x_1, x_2) = a_{11} x_1 + a_{12} x_2 + o(|x|),\\
    f_2(x_1, x_2) = a_{21} x_1 + a_{22} x_2 + o(|x|).
\end{align*}

Тогда, линейная однородная система 
\begin{equation}\label{autonom-nonlin-lin}
    \begin{cases}
    \dot{x_1} = a_{11} x_1 + a_{12} x_2\\
    \dot{x_2} = a_{21} x_1 + a_{22} x_2
    \end{cases}
\end{equation}
называется линеаризацией системы~(\ref{autonom-nonlin-sys}) в начале координат.

\begin{theorem}[О линеаризации (б/д)]
Если линеаризация~(\ref{autonom-nonlin-lin}) нелинейной системы~(\ref{autonom-nonlin-sys}) в начале координат $x=0$ является простой автономной системой и $x=0$ не является центром для системы~(\ref{autonom-nonlin-lin}), то в окрестности $x=0$ нелинейная система и ее линеаризация качественно эквивалентны.
\end{theorem}
Другими словами, если собственные значения матрицы линеаризации при $x=0$ имеют действительную часть отличную от нуля, то фазовый портрет~(\ref{autonom-nonlin-lin}) в окрестности $x=0$  получается из фазового портрета~(\ref{autonom-nonlin-sys}) небольшим искривлением.

\begin{theorem}[О выпрямлении траекторий (б/д)]
Пусть $f(x)$~--- непрерывно дифференцируемая вектор-функция в области $\Omega$ и пусть точка $a \in \Omega$ является обыкновенной точкой системы~(\ref{autonom-sys})~--- не точка равновесия. Тогда найдутся окрестность $\Omega_a$ точки $a$ и такая гладкая обратимая замена переменных в $\Omega_a$, что в окрестности $\Omega_a$ система~(\ref{autonom-sys}) примет вид 
\[
\dot{y_i} = 0,\;\; i = \overline{1, n-1},\;\; \dot{y_n} = 1,
\]
а траектории системы~(\ref{autonom-sys}) в окрестности $\Omega_a$ перейдут в отрезки прямых 
\[
y_i = c_i,\;\; i = \overline{1, n-1},\;\; y_n = t + c_n,
\]
где \(c_1, \ldots, c_n\)~--- постоянные.
\end{theorem}