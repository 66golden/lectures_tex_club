\noindent \Def Точка $\brackets{x_0, y_0, p_0} \in G$ называется \textit{особой точкой уравнения}, если в окрестности этой точки решение задачи Коши либо не существует, либо не единственно. В таких точках $\frac{\del F}{\del p} = 0$
\\
\Def Если для особой точки решений задачи Коши не менее двух, то такая точка называется \textit{точкой локальной неединственности}.
\\
\Def Множество точек локальной неединственности $-$ \textit{дискриминантное множество}.
\\
\Def \textit{Особым решением} уравнения $F \brackets{x, y, y'} = 0$ называется такое решение уравнения, для которого любая точка $\brackets{x, y}$ является точкой локальной неединственности. График особого решения в каждой точке касается графика некоторого другого решения. Ясно, что такие решения могут быть найдены только среди дискриминантных кривых.
\bigbreak
\Th Если $\varphi(x)$ -- особое решение $F \brackets{x, y, y'} = 0$, то в каждой точке его интегральной кривой справедлива система:
\begin{equation*}
    \begin{cases}
    F(x,y,p) = 0 \\
    \frac{\partial F}{\partial p} = 0
    \end{cases}
\end{equation*}
\Proof Возьмем точку $(x_0, y_0 = \varphi(x_0), p_0=\varphi'(x_0))$. Первое условие системы выполнено автоматически, так как $\varphi$ -- решение по условию. Пусть второе уравнение не выполнено, то есть $\frac{\partial F}{\partial p}(x_0, y_0, p_0) \neq 0$. Но тогда выполнена теорема $\exists ! \text{решение ЗК}$, иначе говоря, есть окрестность $(x_0, y_0, p_0)$, в которой через нее проходит ровно 1 интегральная кривая, что противоречит определению $\varphi$. А значит, второе уравнение тоже выполнено. \; \EndProof
\bigbreak 
Таким образом, особое решение принадлежит дискриминантному множеству, но не наоборот.