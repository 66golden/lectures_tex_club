Путь у нас есть задача Коши
\begin{center}
    $\begin{cases}
    F \brackets{x, y, y'} = 0; \\
    y\brackets{x_0} = y_0; \\
    y'\brackets{x_0} = p_0
    \end{cases}$
\end{center}

\textbf{Теорема($\exists ! \text{решение ЗК для ур-ния 1-го порядка, не разр. отн. } y'$):}
\newline Пусть $F(x, y, p)$ опрелена и непреывна вместе с $ \frac{\del F}{\del y} $ и $ \frac{\del F}{\del p} $ в некоторой области $G\subseteq \R^3$ и в точке $(x_0, y_0, p_0) \in G $ справедливо $\mathlarger{\frac{\del F}{\del p} \bigg|_{\brackets{x_0, y_0, p_0}}}\!\!\! \neq 0$. \newline Тогда существует $\delta > 0$, такое что на отрезке $[x_0 - \delta; x_0 + \delta]$ существует и единственно решение ЗК. 
\bigbreak
\Proof Так как $\frac{\del F}{\del p} \neq 0$ и частные производные непрерывны в $G$, то по теореме о неявной функции существует окрестность точки $\brackets{x_0, y_0}$ и существует непрерывно дифференцируемая функция $f\brackets{x,y}$, определённая на окрестности $(x_0, y_0)$, такая что $f\brackets{x_0, y_0} = p_0$ и для любой точки из $(x_0, y_0)$ верно равенство $p = f(x, y)$. Тогда задача Коши будет формулироваться так:

\begin{center}
    $\begin{cases}
    y' = f\brackets{x, y}; \\
    y\brackets{x_0} = y_0
    \end{cases}$
\end{center}

Вновь применим теорему о неявной функции, получим $\mathlarger{\frac{\del f}{\del y} = - \frac{\frac{\del F}{\del y} \brackets{x, y, p}}{\frac{\del F}{\del p} \brackets{x, y, p}} \bigg|_{p = f \brackets{x, y}}}$.

Внутри окрестности точки $(x_0, y_0)$ можно взять выпуклую область, на которой для $f(x,y)$ будет выполняться условие Липшица. Тогда получим требуемое по теореме о существовании и единственности решения задачи Коши для уравнения, разрешённого относительно производной. \EndProof