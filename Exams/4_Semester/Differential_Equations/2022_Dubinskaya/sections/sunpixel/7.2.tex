Рассмотрим функционал 

\begin{equation}\label{J_func_free}
J(y) = \int_a^b F \left( x; y(x), y'(x)) \right) dx,
\end{equation}
\\
определённый на множестве
$M = \{ y(x) \in C^1 [a, b] : y(a) = A\}$, где
$F \left( x; y, p \right)$ -- дважды непрерывно дифференцируемая
функция на $[a, b] \times \R^2$.
Задача отличается от простейшей вариационной задачи только тем,
что один из концов (в данном случае $b$) не зафиксирован.

\textbf{Определение.}

Задача нахождения слабого локального экстремума функционала
(\ref{J_func_free}) называется
\textit{задачей со свободным концом}.

\textbf{Теорема.}

Если дважды непрерывно дифференцируемая функция 
$\widehat{y}(x) \in M$ даёт слабый локальный экстремум функционала
(\ref{J_func_free}), то на $[a, b]$ $\widehat{y}(x)$ удовлетворяет
уравнению Эйлера

\begin{equation}\label{euler_eq_free}
\frac{\partial F}{\partial y} - \frac{d}{dx} \left(
\frac{\partial F}{\partial y'} \right) = 0
\end{equation}

и при $x = b$:

\begin{equation}\label{extra_cond}
\left. 
\frac{\partial F}{\partial y'} 
\left[ x; \widehat{y}(x), \widehat{y}'(x) \right]
\right|_{x = b} = 0
\end{equation}

\underline{Доказательство.}

Рассмотрим произвольное допустимое приращение 
$\eta(x) \in C^1 [a, b] : \eta(a) = 0$.

Пусть $\Phi(\alpha) = J(\widehat{y} + \alpha \eta), \alpha \in \R$ -- 
непрерывно дифференицруемая функция аргумента $\alpha$;
при $\alpha = 0$ $\Phi(\alpha)$ имеет экстремум.
Следовательно, (в последнем = меняем местами диф-ние и инт-ние)

$$
0 = \Phi'(0) = \left. \frac{d}{d\alpha} 
J( \widehat{y} + \alpha \eta) \right|_{\alpha = 0} = 
\left. \frac{d}{d\alpha} \int_a^b F \left( x, \widehat{y} + \alpha \eta, 
\widehat{y}' + \alpha \eta' \right) dx \right|_{\alpha = 0} = 
\left. \int_a^b \left[
\frac{\partial F}{\partial y} \eta(x) + 
\frac{\partial F}{\partial y'} \eta'(x)
\right] \right|_{y = \widehat{y}(x)} dx
$$

Проинтегрируем второе слагаемое по частям:

$$
\left. \int_a^b \frac{\partial F}{\partial y'} \eta' 
\right|_{y = \widehat{y}} dx = 
\left. \frac{\partial F(x; \widehat{y}, \widehat{y}')}{\partial y'} 
\right|_{x = b} \cdot \eta(b) - 
\left. \int_a^b \frac{d}{dx} \left(
\frac{\partial F}{\partial y'}
\right) \right|_{y = \widehat{y}} \eta(x) dx
$$

Тогда $\Phi'(0)$ запишется следующим образом:

$$
\Phi'(0) = \left. 
\frac{\partial F(x; \widehat{y}, \widehat{y}')}{\partial y'}
\right|_{x = b} \eta(b) + \left. \int_a^b \left[
\frac{\partial F}{\partial y} - 
\frac{d}{dx} \left( \frac{\partial F}{\partial y'} \right) 
\right] \right|_{y = \widehat{y}} \eta(x) dx = 0
$$

Это выполняется для любого $\eta(x) \in C^1 [a, b] : \eta(a) = 0$.
Это, в частности, означает, что это верно для 
$\eta_1(x) \in C^1 [a, b] : \eta(a) = 0, \eta(b) = 0$. При подставлении
$\eta_1(x)$ получаем, что 

$$
\int_a^b \left[
\frac{\partial F}{\partial y} - 
\frac{d}{dx} \left( \frac{\partial F}{\partial y'} \right) 
\right] \eta(x) dx = 0
$$

Используя основную лемму вариационного исчисления, получим условие 
(\ref{euler_eq_free}). Это означает, что второе слагаемое в выражении 
$\Phi'(0)$ всегда равно нулю. Получаем, что для всех
$\eta(x) \in C^1 [a, b] : \eta(a) = 0$

$$
\Phi'(0) = \left. 
\frac{\partial F(x; \widehat{y}, \widehat{y}')}{\partial y'}
\right|_{x = b} \eta(b) = 0
$$

Так как можно подобрать вариацию, такую что $\eta(b) \neq 0$, то 
необходимым условием слабого локального экстремума является 
условие (\ref{extra_cond}).
\pagebreak

\textbf{Определение.}

Решение уравнения (\ref{euler_eq_free}) называется
\textit{экстремалью задачи со свободным концом}.
Решение уравнения (\ref{euler_eq_free}), которое удовлетворяет
условию $y(a) = A$ и условию (\ref{extra_cond}), называется
\textit{допустимой экстремалью}.

Для задачи, где не закреплены оба конца, можно получить 
аналогичное необходимое условие слабого локального экстремума:

\textbf{Теорема.}

Рассмотрим функционал 

\begin{equation*}
J(y) = \int_a^b F \left( x; y(x), y'(x)) \right) dx,
\end{equation*}
\\
где $y(x) \in C^1 [a, b]$ (то есть оба конца не закреплены) и
$F \left( x; y, p \right)$ -- дважды непрерывно дифференцируемая
функция на $[a, b] \times \R^2$.

Если $\widehat{y}(x) \in C^1 [a, b]$ даёт слабый локальный экстремум
функционала $J(y)$, то $\widehat{y}(x)$ удовлетворяет
следующим условиям:

$$
\frac{\partial F}{\partial y} - 
\frac{d}{dx} \left( \frac{\partial F}{\partial y'} \right) = 0
$$

$$
\left. \frac{\partial F}{\partial y'} \right|_{x = a} = 
\left. \frac{\partial F}{\partial y'} \right|_{x = b} = 0
$$