Пусть функции $F(x; y, p)$ и $G(x; y, p)$ дважды непрерывно
дифференцируемы на $[a, b] \times \R^2$. Рассмотрим функционал

\begin{equation}\label{J_func_isoper}
J(y) = \int_a^b F(x; y, y') dx,
\end{equation}

определённый на множестве ($A, B, l$ -- фиксированные числа)
$$
M = \left\{ y \in C^1 [a, b] : y(a) = A, y(b) = B, 
K(y) = \int_a^b G(x; y, y') dx = l \right\}
$$

Условие $K(y) = l$ называют \textit{условием связи}.

\textbf{Определение.}

Функция $\widehat{y}(x) \in M$ называется называется 
\textit{слабым локальным минимумом (максимумом)} функционала 
(\ref{J_func_isoper}), если 

$$
\exists \varepsilon > 0 : \forall y \in M : 
\| y - \widehat{y} \|_{C^1 [a, b]} < \varepsilon \ \ 
J(y) \geqslant J(\widehat{y}) 
\left( J(y) \leqslant J(\widehat{y}) \right)
$$

\textbf{Определение.}

\textit{Изопериметрической задачей} называется задача нахождения
слабого локального экстремума функционала (\ref{J_func_isoper}).

\textbf{Замечание.}

Изопериметрическая задача похожа на задачу поиска
условного экстремума числовых функций.

Так же, как и для условного экстремума числовых функций,
определим лагранжиан:

$$
L(x, y; \lambda) = F(x, y, y') + \lambda G(x, y, y'), \lambda \in \R
$$

$\lambda$ называется \textit{множителем Лагранжа}.

Также при формулировке необходимого условия потребуем, чтобы 

$$
\delta K(y, \eta) = \int_a^b \left(
\frac{\partial G}{\partial y} \eta(x) + 
\frac{\partial G}{\partial y'} \eta'(x)
\right) dx \ \ \forall \eta \in \mathring{C}^1 [a, b]
$$
\\
не равнялась тождественному нулю, так как иначе 
экстремум функционала может равняться $l$, а значит
условию связи будет удовлетворять только одна функция, и 
изопериметрическая задача будет несодержательной.

\textbf{Теорема (необходимое условие).}

Пусть дважды непрерывно дифференцируемая функция $\widehat{y}(x) \in M$
является решением изопериметрической задачи и пусть
$\delta K (\widehat{y}, \eta) \not \equiv 0$ для всех 
$\eta \in \mathring{C}^1 [a, b]$.
Тогда найдётся такое значение $\lambda \in \R$, что на $[a, b]$
функция $\widehat{y}$ будет удовлетворять уравнению Эйлера

\begin{equation}\label{euler_eq_isoper}
\frac{\partial L}{\partial y} - \frac{d}{dx} \left(
\frac{\partial L}{\partial y'} \right) = 0
\end{equation}

\underline{Доказательство.}

Из условия теоремы следует, что $\exists \eta_0(x) \in \mathring{C}^1 [a, b]$
такая, что $\delta K [ \widehat{y}, \eta_0(x) ] \neq 0$.

Рассмотрим следующие функции:

$$
u = \varphi(\alpha, \beta) = 
J [\widehat{y}(x) + \alpha \eta(x) + \beta \eta_0(x)],
\alpha, \beta \in \R
$$

$$
v = \psi(\alpha, \beta) = 
K [\widehat{y}(x) + \alpha \eta(x) + \beta \eta_0(x)],
\alpha, \beta \in \R
$$

Используя теоремы о непрерывности и дифференцируемости собственных
интегралов, зависящих от параметра, получаем:

$$
\varphi(0, 0) = J(\widehat{y}), \ \ 
\frac{\partial \varphi (0, 0)}{\partial \alpha} = 
\delta J[ \widehat{y}(x), \eta(x)], \ \ 
\frac{\partial \varphi (0, 0)}{\partial \beta} = 
\delta J[ \widehat{y}(x), \eta_0(x)]
$$

$$
\psi(0, 0) = K(\widehat{y}), \ \ 
\frac{\partial \psi (0, 0)}{\partial \alpha} = 
\delta K[ \widehat{y}(x), \eta(x)], \ \ 
\frac{\partial \psi (0, 0)}{\partial \beta} = 
\delta K[ \widehat{y}(x), \eta_0(x)]
$$

Далее можно расписать все первые вариации подобным образом:

$$
\delta J [\widehat{y}, \eta(x)] = 
\int_a^b \left[ 
\left. \frac{\partial F}{\partial y} \right|_{y = \widehat{y}(x)} \eta(x) +
\left. \frac{\partial F}{\partial y'} \right|_{y = \widehat{y}(x)} \eta'(x)
\right] dx
$$

Покажем, что якобиан

$$
\frac{\partial (\varphi, \psi)}{\partial (\alpha, \beta)} \equiv 0
\ \ \forall \eta(x) \in \mathring{C}^1 [a, b]
$$

Предположим, что $\exists \eta_1(x) \in \mathring{C}^1 [a, b]$ такое, что
$\frac{\partial (\varphi, \psi)}{\partial (\alpha, \beta)} \neq 0$.
Это означает, что существует окрестность точки $(0, 0)$,
в которой система

\begin{equation*}
\begin{cases}
u = \varphi(\alpha, \beta) \\
v = \psi(\alpha, \beta)
\end{cases}
\end{equation*}

разрешима относительно $\alpha$ и $\beta$.

Пусть, без ограничения общности, $\widehat{y}$ даёт минимум
функционала (\ref{J_func_isoper}). Тогда рассмотрим такую систему:

\begin{equation*}
\begin{cases}
\varphi(\alpha, \beta) = \varphi(0, 0) - \varepsilon,\  \varepsilon > 0 \\
\psi(\alpha, \beta) = \psi(0, 0)
\end{cases}
\end{equation*}

По доказанному, решение этой системы существует и единственно --
обозначим его $(\widehat{\alpha}, \widehat{\beta})$. Тогда у нас выполнено

$$
\varphi(\widehat{\alpha}, \widehat{\beta}) = \varphi(0, 0) - \varepsilon = 
J(\widehat{y}) - \varepsilon, \ \ 
\psi(\widehat{\alpha}, \widehat{\beta}) = \psi(0, 0) = 
K(\widehat{y}) = l
$$

Это значит, что существует функция 
($\widehat{y}(x) + \widehat{\alpha} \eta_1(x) + \widehat{\beta} \eta_0(x)$)
такая, что

$$
J(\widehat{y}(x) + \widehat{\alpha} \eta_1(x) + \widehat{\beta} \eta_0(x)) =
J(\widehat{y}) - \varepsilon < J(\widehat{y}), \ \ 
K(\widehat{y}(x) + \widehat{\alpha} \eta_1(x) + \widehat{\beta} \eta_0(x)) =
l
$$

Это противоречит тому, что $\widehat{y}$ даёт слабый локальный минимум.
Таким образом, мы доказали, что 

$$
\frac{\partial (\varphi, \psi)}{\partial (\alpha, \beta)} \equiv 0
\ \ \forall \eta(x) \in \mathring{C}^1 [a, b]
$$

Распишем этот якобиан:

$$
\left. \frac{\partial (\varphi, \psi)}{\partial (\alpha, \beta)}
\right|_{\alpha = \beta = 0} = 
\begin{vmatrix}
\delta J(\widehat{y}, \eta) & \delta K(\widehat{y}, \eta) \\
\delta J(\widehat{y}, \eta_0) & \delta K(\widehat{y}, \eta_0)
\end{vmatrix}
\equiv 0 \ \ \forall \eta(x) \in \mathring{C}^1 [a, b]
$$

Положим $\lambda = -\frac{\delta J(\widehat{y}, \eta_0)}
{\delta K(\widehat{y}, \eta_0)}$ (здесь мы пользуемся тем, что
$\delta K(\widehat{y}, \eta_0) \neq 0$). Тогда мы получаем, что


$$
\delta J(\widehat{y}, \eta) + \lambda
\delta K(\widehat{y}, \eta) \equiv 0\ \ 
\forall \eta(x) \in \mathring{C}^1 [a, b]
$$

Расписав первые вариации, получаем:

\begin{multline*}
\delta J(\widehat{y}, \eta) + \lambda \delta K(\widehat{y}, \eta) = 
\int_a^b \left. \left( \frac{\partial F}{\partial y} \eta + 
\frac{\partial F}{\partial y'} \eta' \right) 
\right|_{y = \widehat{y}} dx + \lambda
\int_a^b \left. \left( \frac{\partial G}{\partial y} \eta + 
\frac{\partial G}{\partial y'} \eta' \right) \right|_{y = \widehat{y}} dx =
\\ =
\int_a^b \left. \left[
\left( 
\frac{\partial F}{\partial y} + \lambda \frac{\partial G}{\partial y}
\right) \eta(x) + 
\left( 
\frac{\partial F}{\partial y'} + \lambda \frac{\partial G}{\partial y'}
\right) \eta'(x)
\right] \right|_{y = \widehat{y}} dx = 0
\end{multline*}

Проинтегрируем слагаемое с $\eta'(x)$ по частям:

$$
\int_a^b \left( 
\frac{\partial F}{\partial y'} + \lambda
\frac{\partial G}{\partial y'}
\right) d \eta = 
\underbrace{\left. \left( \frac{\partial F}{\partial y'} + \lambda
\frac{\partial G}{\partial y'} \right) \eta \right|_a^b}
_{= 0, \text{ так как } \eta(a) = \eta(b) = 0} - 
\int_a^b \frac{d}{dx} \left(
\frac{\partial F}{\partial y'} + \lambda
\frac{\partial G}{\partial y'}
\right) \eta dx
$$

Воспользовавшись тем, что 
$\frac{\partial L}{\partial y} = \frac{\partial F}{\partial y} + \lambda
\frac{\partial G}{\partial y}$ и 
$\frac{\partial L}{\partial y'} = \frac{\partial F}{\partial y'} + \lambda
\frac{\partial G}{\partial y'}$, получаем:

$$
\int_a^b \left( \frac{\partial L}{\partial y} - \frac{d}{dx}
\left( \frac{\partial L}{\partial y'} \right)
\right) \eta(x) dx = 0 \ \ 
\forall \eta(x) \in \mathring{C}^1 [a, b]
$$

Используя основную лемму вариационного исчисления, получаем
уравнение (\ref{euler_eq_isoper}).
