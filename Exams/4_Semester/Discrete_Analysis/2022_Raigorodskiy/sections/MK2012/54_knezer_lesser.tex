\Def Кнезеровский граф $-$ G(n, k, 0), также обозначается $KG_{n, k}$, т. е. если $R_n=\overline{1..n}$, то $V=\{ S | S \subseteq R_n\!\! \wedge\!\! |S| = k \}$ и $E=\{ (A, B) | A \in V\!\! \wedge\!\! B \in V\!\! \wedge\!\! A \cap B = \varnothing \}$.

\underline{Простые верхние оценки}: Самая простая оценка: n. Поочерёдно будем брать множества, пересекающие по i для $i \in R_n$. Тогда каждый элемент i-го множества можно покрасить в один и тот же цвет, так как между ними нет рёбер, а всего множеств n. Эту оценку можно немного улучшить: Красим так же, как и раньше до n-k+1. Тогда неиспользованных чисел остаётся только k-1<1 для любой непокрашенной вершины, значит, это меньше, чем полноценная вершина, значит, непокрашенных вершин не осталось. Улучшенная оценка: $\chi(KG_{n, k}) \leqslant n-k+1$.

\underline{Лучшая оценка}: Повторим раскарску, но теперь до n-2k+1. Тогда все оставшиеся веришны - подмножества $R_n \setminus R_{n-2k+1}$. Тогда чисел остаётся всего 2k-1, и значит, любые две веришны пересекаются, значит, можем покрасить их в ещё один цвет и $\chi(KG_{n, k}) \leqslant n-2k+2$.

\textbf{Теорема(Ловаса):} $\chi(KG_{n, k}) = n-2k+2$.