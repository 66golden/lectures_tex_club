Пусть есть граф $G = (V, E)$, $W \subseteq V$.

\Def W - независимое, если $\forall x, y \in W: (x, y) \notin E$

\Def $\alpha (G) = $ мощность любого самого большого независимого $W$ - число независимости графа.

\Def $\varkappa (G)$ (буква читается каппа) Вершинная связность - min количество вершин, удаление которых приведёт к тому, что граф перестаёт быть связным.

\Th Признак Эрдёша - Хва'тала.

$\alpha (G) \leqslant \varkappa (G)$. Тогда G - гамильтонов.

\Example $V = \{ A \subset \{1, 2, \dots, n \} : |A| = 3\}$. Тогда $|V| = C_n^3 \sim \frac{n^3}{6}$. 

$E = \{ (A, B): |A \cap B| = 1\}$; $\forall A \in V$ $deg A = 3 C_{n-3}^2$ $\Rightarrow$ $|E| = \frac{3 C_{n-3}^2 C_n^3}{2} \sim \frac{n^3}{6} \cdot 3 \cdot \frac{n^2}{4} = \frac{n^5}{8}$

$\alpha(G)$: макс. число "троек", попарные пересечения которых не равны 1, т.е. 0 или 2.

\begin{equation*}
\alpha(G) \geqslant (=) 
 \begin{cases}
   n &\text{$n \equiv 0 (4)$}\\
   n-1 &\text{$n \equiv 1 (4)$} \\
   n-2 &\text{$n \equiv 2, 3 (4)$}
 \end{cases}
\end{equation*}
(Конструкция: разбиваем на четвёрки элементов, внутри каждой четвёрки берём все подмн-ва (они пересекаются по 2 элементам всегда), а разные четвёрки пересекаются по 0).

\Th $\alpha(G) \leqslant n$ для графа в примере.

\Proof Зафиксируем $W = \{ A_1, A_2, \dots, A_t \}$ - независимое подмножество. 

Сопоставим каждому $A_i \rightarrow \overline{x_i} = (x^1, \dots, x^n) \in \Z_2^n, x^j \in \{0, 1\}$, причём 1 стоят на позициях, которые принадлежат множеству $A_i$.

$|A_i \cap A_j| = (\overline{x_i}, \overline{x_j})$. Осталось доказать, что $x_1, \dots, x_t$ ЛНЗ, тогда очевидно, что $|W| = t \leqslant n$, в силу размерности пространства.

Докажем, что $x_1, \dots, x_t$ ЛНЗ в $\Z_2^n$. $c_1x_1 + c_2x_2 + \dots + c_tx_t = \overline{0}$. Домножим обе части равенства на $x_1$, получим: $c_1(x_1, x_1) + c_2 (x_2, x_1) + \dots + c_t (x_t, x_1) = 0$; но все скалярные произведения, кроме $(x_1, x_1)$, равны 0 в $\Z_2$, т.к. они равны либо 0, либо 2, а $(x_1, x_1) \equiv 1 (2)$, значит, $c_1 = 0$. Так можно доказать для любого коэффициента, т.е. это тривиальная линейная комбинация. \EndProof

Оценка на $\varkappa(G)$. Зафиксируем вершины $x, y$, $f(x, y)$ - количество их общих соседок. Тогда $\varkappa(G) \geqslant min_{x, y} f(x, y)$ для любого графа (очевидно). 

Для нашего графа G если $x \cap y = 0$, то $f(x, y) = 3 \cdot 3 \cdot (n - 6)$, т.к. нужно, чтобы из первого множества был один из 3х элементов, из 2-ого один из 3х элементов, а 3й элемент отличен от элементов из этих 2х мн-в.

Если $x \cap y = 1$, то $f(x, y) = C_{n-5}^2 + 2 \cdot 2 \cdot (n - 5)$, первое слагаемое - это если "соседи" содержат элемент, по которому пересечение, второе - не содержат.

Аналогично, если $x \cap y = 2$, то $f(x, y) = 2 \cdot C_{n-4}^2 + n - 4$.

Очевидно, что при больших n минимально $9(n-6) > n \geq \alpha(G)$, значит, такие графы гамильтоновы.