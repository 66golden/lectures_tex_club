\Def Графы $G_1 = (V_1, E_1), G_2 = (V_2, E_2), G_1 \cong G_2$ ($G_1$ изоморфен $G_2$), если $\exists$ биекция $\varphi: V_1 \leftrightarrow V_2: \forall x, y: (x, y) \in E_1  (\varphi(x), \varphi(y)) \in E_2$

Проблема: всего различных изоморфных графов $n!$; нет решения проблемы в алгоритмическом ключе для всех графов на n вершинах (мы не можем получить полиномиальную асимптотику). 

Случайный граф $G(n, \frac{1}{2})$; хотим доказать, что есть алгоритм, который работает на почти всех графах за полиномиальное время. На почти всех значит, что доля графов, на которых не работает, стремится к нулю; таким образом, если $K_n$ - графы, на которых алгоритм работает верно, то $|K_n| = 2^{C_n^2(1 + o(1))}$

Авторы алгоритма: Эрдеш, Бабаи, Selkow. 

Каноническая нумерация.

1. Пусть $r = [3log_2n]$. Выберем в нём $r$ вершин наибольшей степени: $d(1) \geqslant d(2) \geqslant d(3) \geqslant \dots \geqslant d(r)$. Если где-то равенство, то граф не попадает в $K_n$ и алгоритм завершёл.

Рассмотрим $i = r+1, \dots , n$. Введём $f(i) = \sum_{j=1}^{r} a(i, j) \cdot 2^j$, где 

\begin{equation*}
a(i, j) = 
 \begin{cases}
   1 &\text{, если $(i, j) \in E$}\\
   0 &\text{, если $(i, j) \notin E$}
 \end{cases}
\end{equation*}

2. Далее упорядочиваем эти коды: $f(r + 1) \geqslant f(r+2) \geqslant \dots \geqslant f(n)$ и если вновь где-то стоит знак равенства, то наш граф не попадает в $K_n$ и алгоритм завершён; иначе добавляем в $K_n$ (и мы занумеровали каноническим способом).

Корректность: \Proof Пусть $C$ - событие: $\forall i: 1, \dots, r+2 d(i) \geqslant d(i+1) + 3 $ ; $C(i, j)$ - событие: при удалении вершин $i, j$ из G старшие по величине r степеней нового графа попарно различны. Получается, что $\forall i, j C \to C_(i, j)$ за счёт запаса.

\Lemma (б/д): $P(\overline{C}) \to 0, n \to \infty$. 

Событие $A(i, j)$: либо $C(i,j)$ не выполнено, либо $G$ отвергается алгоритмом из-за совпадения величин $f(i), f(j)$.

$P(G \notin K_n) \leqslant P(\overline{C}) + P(C \text{ и граф отвергается на 2 этапе})$; осталось убедиться, что $P(C \text{ и граф отвергается на 2 этапе}) \to \infty$, т.к. первое стремится к 0 по лемме. 

$P(C \text{ и граф отвергается на 2 этапе}) \leqslant \sum_{i < j} P(C(i, j) \cap A(i, j))\leqslant \sum_{i < j} P(A(i, j) | C(i, j)) \leqslant C_n^2 \cdot 
(\frac{1}{2})^r < n^2 2^{-3log_2n} = \frac{1}{n} \to 0$ \EndProof