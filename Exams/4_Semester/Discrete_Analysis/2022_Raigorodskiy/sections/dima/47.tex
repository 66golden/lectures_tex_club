\par Все формулировки см. в прошлом билете (46)

\par \textbf{Вывод теоремы из леммы:} \Proof Зафиксируем $a > \frac{5}{6}, n \geq n_0' \geq n_0$. Определим $u=u(n, \alpha)$ - минимальное число, такое что $P(\chi(G) \leq u) > \frac{1}{\ln{n}}$ (для $u=n$ условие точно выполнено. Будем уменьшать $u$ пока оно не нарушится). Тогдв $P(\chi(G) \leq u-1) \leq \frac{1}{\ln{n}}$ (иначе могли бы уменьшить $u$), а значит $P(\chi(G) \geq u) \geq 1-\frac{1}{\ln{n}}$.

\par Введем случайную величину $Y(G):=\min \{k: \exists S \subset \{1, \ldots, n\}, |S|=k: \chi(G|_{\{1, \ldots, n\} \setminus S}) \leq u)\}$ (размер самого маленького кусочка, который можно выкинуть, чтобы граф покрасился в $u$ цветов). Эта случайная величина липшицева по вершинам (если портим что-то в окрестности одной вершины в худшем случае меняется только покраска этой вершины, так что размер минимального кусочка который надо выкинуть изменится не больше чем на 1).

\par Возьмём $a=\sqrt{2(n-1)\ln{\ln{n}}}$. Тогда из неравенства Азумы: $$P(Y-\mathbb{E}Y \leq -a)=P(Y-\mathbb{E}Y \geq a)\leq e^{-\frac{a^2}{2(n-1)}}=\frac{1}{\ln{n}}$$

\par Предположим, что $\mathbb{E}Y\geq a$. Тогда $P(Y-\mathbb{E}Y \leq -a) \leq \frac{1}{\ln{n}}$ (неравенство Азумы). С другой стороны, $P(Y-\mathbb{E}Y \leq -a) = P(Y \leq \mathbb{E}Y-a)$. Так как $\mathbb{E}Y-a \geq 0$, эта вероятность $\geq P(Y \leq 0)$. Это событие равноценно тому, что для того чтобы покрасить в $u$ цветов из графа ничего не надо удалять, то есть эта веротяность $=P(\chi(G) \leq u)>\frac{1}{\ln{n}}$ (последнее неравенство из определения $u$). Получили, что вероятность одновременно и больше и меньше $\frac{1}{\ln{n}}$ - противоречие. Следовательно $\mathbb{E}Y < a$.

$$\frac{1}{\ln{n}} \geq P(Y-\mathbb{E}Y \geq a)=P(Y \geq \mathbb{E}Y + a) \geq P(Y \geq 2a)=$$ 

\par Последнее неравенство следует из того, что $\mathbb{E}Y+a < 2a$.
$$=P(Y \geq 2\sqrt{2(n-1)\ln{\ln{n}}})\underset{n \geq n_0'}{\geq} P(Y > \sqrt{n} \ln{n})$$

\par Последний переход справедлив потому что $2\sqrt{2(n-1)\ln{\ln{n}}}$ асимптотически меньше $\sqrt{n} \ln{n}$. Таким образом получаем три события

$$P(A):=P(\chi(G) \geq u) > 1-\frac{1}{\ln{n}} \text{ (из начала доказательства)}$$
$$P(B):=P(\forall S \subset \{1, \ldots, n\}, |S|\leq \sqrt{n}\ln{n}: \chi(G|_S) \leq 3) \geq 1-\frac{1}{\ln{n}} \text{ (из технической леммы)}$$
$$P(C):=P(Y \leq \sqrt{n} \ln{n}) \geq 1 - \frac{1}{\ln{n}}$$

\par Рассмотрим $G \in A \cap B \cap C$
\begin{enumerate}
    \item $\chi(G) \geq u$, так как $G \in A$
    \item $Y(G) \leq \sqrt{n} \ln{n}$, так как $G \in C$ (то есть можно выделить такое множество $S, |S| \leq \sqrt{n}\ln{n}$, что все кроме него красится в $u$ цветов). 
    \item $G \in B \Rightarrow S$ можно покрасить в 3 цвета $\Rightarrow \chi(G) \leq u+3$    
\end{enumerate}

\par Значит $u \leq \chi(G) \leq u+3$. Так как $P(A \cap B \cap C) = 1 - P(\overline{A} \cup \overline{B} \cup \overline{C}) \geq 1 - \frac{3}{\ln{n}}\rightarrow 1$, то это выполняется асимптотически почти наверное и теорема доказана \EndProof