\par \Note Все обозначения и формулировки см. в прошлом билете (48)

\par \Lemma а.п.н. $\forall S \in V, |V|=n, |S|=m \hookrightarrow \alpha(G|_S) \geq k_1(m)$
\par \Proof Отрицание:
$$P(\exists S \subset V, |S|=m: \alpha(G|_S) < k_1)) \leq C_n^m P(\alpha(G(m, \frac{1}{2})) < k_1)<$$ $$<2^m P(\alpha(G(m, \frac{1}{2})) < k_1)=2^n P(X_{k_1}\leq 0)=*$$
\par Перешли к рассмотрению каждого $S$ в отдельности как графа на $m$ вершинах. В последнем переходе воспользовались соображением из доказательства корректности $k_0$ в прошлом билете. Неравенство Чебышева тут применить не получится. Хотелось бы применить неравенство Азумы, но величина $X_k$ не является липшицевой. 

\par Вспомним величину $Y_k$ из прошлого билеты. Она липшицева по ребрам (чтобы ребро что-то испортило, оно должно лежать полностью внутри какого-то $S_i$. Чтобы оно испортило больше чем на 1 оно должно лежать одновременно в двух $S_i$, но такого не может быть так как их пересечение $\leq 1$). Очевидно, что $Y_k=0 \Leftrightarrow X_k=0$ (нет независимых множеств $\Leftrightarrow$ нет гирлянды). Тогда 
$$*=2^n P(Y_{k_1}=0)=2^nP(-Y_{k_1} \geq 0)=2^nP(\mathbb{E}Y_{k_1}-Y_{k_1} \geq \mathbb{E}Y_{k_1}) \leq 2^n e^{-\frac{(\mathbb{E}Y_{k_1})^2}{2C_m^2}}=*$$

\par По лемме из предыдущего билеты получаем

$$*\leq 2^n e^{-\frac{m^4}{4k_1}(1+o(1))\frac{1}{C_m^2}} = 2^n e^{-\frac{n^2}{4k_1^8 \ln^4 n}(1+o(1)}\rightarrow 0$$

\par Во втором переходе воспользовались определением $m$, а так же тем, что $C_m^2 \sim m^2$. Стремление к 0 происходит так как в показателе экспоненты стоит $-\frac{n^2}{a\ln^b n}$ (так как $k \sim 2 \log_2 n$), что больше чем $n$ в показателе у двойки. \EndProof

\par \textbf{Доказательство теоремы:} \Proof Заметим, что ранее доказывали а.п.н. $\alpha(G) \leq 2\log_2 n \Rightarrow$ а.п.н. $\chi(G) \geq \frac{n}{2\log_2 n}$, то есть нам достаточно доказать только $\chi(G) \leq \frac{n}{2\log_2 n}+o(\frac{n}{\ln{n}})$
\par Обозначим событие из условия прошлой леммы как $A$. Пусть $G \in A$. Тогда так как $n > m$ найдется независимое множество размера $k_1$. Покрасим его в 1 цвет. Будем продолжать процесс, пока остается больше $m$ вершин. Всего сделаем $[\frac{n-m}{k_1}]$ шагов. У нас остается не более $m$ вершин, покрасим каждую из них в свой цвет. Итого количество цветов $$m+[\frac{n-m}{k_1}] \geq \chi(G) \text{, где } m=o(\frac{n}{\ln{n}}) \text{, а } [\frac{n-m}{k_1}] \sim \frac{n}{2\log_2 n} \quad \blacksquare$$