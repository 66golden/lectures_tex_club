$H = (V, E)$ - $n$-однородный гиперграф. (при $n = 2$ это обычный граф)
Пусть $k$ - максимальная степень вершины.

\Def Раскраска множества вершин $V$ гиперграфа $H = (V, E)$ называется правильной, если в этой раскраске все ребра из $E(H)$ не являются одноцветными. Хроматическим числом гиперграфа $H$ называется минимальное число цветов, требуемое для правильной раскраски вершин этого гиперграфа. 
\\
\\
$X(H)$ - хроматическое число гиперграфа \\
При $n = 2 \Rightarrow X(H) \leq k + 1$ (Тривиальная оценка для обычных графов) \\
При $n \geq 3 \Rightarrow X(H) \leq f(kn)$, где $f$ - какая-то функция.

\Th $e \cdot r^{1-n}(n(k-1)+1)  \leq 1 \Rightarrow X(H) \leq r$ (ничего лучшего для $n \geq 3$ сейчас не знают).

\Proof

Красим гиперграф в $r$ цветов

Красим каждую вершину в каждый конкретный цвет с вероятностью $\frac{1}{r}$.

Пусть $e_1, ..., e_{|E|}$ - рёбра гиперграфа

$A_1, ..., A_{|E|}$, где $A_i$ - событие: $e_i$ - одноцветное.

Если докажем, что $P(\bigcap_{i=1}^{|E|} \overline{A_i}) > 0$, получим $X(H) \leq r$, потому что докажем возможность покраски графа в $r$ цветов, не получив одноцветные рёбра. Докажем это:

$p = P(A_i) = r^{1-n}$

$d$ - число зависимостей. $d \leq n \cdot (k - 1)$ (Это суммарная степень всех вершин, помимо выбранного ребра(события))

По условию теоремы $e \cdot r^{1-n}(n(k-1)+1) \leq 1$, следовательно выполняется неравенство $e p (d + 1) \leq 1$. Можем воспользоваться ЛЛЛ и получить $P(\bigcap_{i=1}^{|E|} \overline{A_i}) > 0 \Rightarrow X(H) \leq r  \ \blacksquare$

$(e(n(k-1)+1))^{\frac{1}{n-1}} \leq 1 \Rightarrow X(H) \leq r$ 

$(e(n(k-1)+1))^{\frac{1}{n-1}} \leq 1 \Rightarrow r \geq \left\lceil (e(n(k-1)+1))^{\frac{1}{n-1}} \right\rceil$

И если взять в качестве $r$ значение $\left\lceil (e(n(k-1)+1))^{\frac{1}{n-1}} \right\rceil$, то оно будет удовлетворять условию $r \geq \left\lceil (e(n(k-1)+1))^{\frac{1}{n-1}} \right\rceil$ (переформулированное начальное условие), следовательно, применима теорема. Получим оценку $X(H) \leq \left\lceil (e(n(k-1)+1))^{\frac{1}{n-1}} \right\rceil$.

Если подставить n = 2 (то есть граф обычный), то получим $X(H) \leq \left\lceil e(2(k-1)+1) \right\rceil$

Эта оценка хуже, чем обычная: $X(H) \leq k + 1$, но сопостовимая.