$G(n, 3, 1)$ - полный граф, где вершины это тройки элементов $n$-го множества, а рёбрами соединяем две тройки, если их пересечение ровно один элемент.

$\alpha (G)$ - максимальное независимое множество графа.

$\omega (G)$ - максимальная клика графа.

$V(G)$ - количество вершин в графе.

\Statement $R(s, s) \geqslant \frac{s^3}{6} (1 + o(1))$

\Proof $\alpha (G(n, 3, 1)) \leqslant n$ (доказывалась в прошлом семестре)

$\omega (G(n, 3, 1)) = \left[ \frac{n-1}{2} \right]$

Пример клики размером $\left[ \frac{n-1}{2} \right]$:\\ 
$\{1, 2, 3\}, \{1, 4, 5\}, \{1, 6, 7\}, ...$\\
Оценка на $\omega(G(n, 3, 1))$ доказывается очень просто. Достаточно просто  рассмотреть клики устроенные по другому. 

$V(G(n, 3, 1)) = C_n^3 \thicksim \frac{n^3}{6}$ 

Получили граф на $C_n^3$, у которого $\alpha (G) < n + 1$ и $\omega (G) < n+1$

Следовательно,

$$R(n + 1, n + 1) \geqslant C_n^3$$

$$R(s, s) \geqslant \frac{s^3}{6} (1 + o(1))$$
\EndProof

\Th (Франклина, Уилсона, 1981)

$$R(s, s) > (e^{\frac{1}{4}} + o(1))^{\frac{ln^2(s)}{ln(ln(s))}}$$

\Proof

Пусть $p$ - простое число, $m = p^3,l = p^2$.

Рассмотрим граф:\\
Вершины: $V = \{\overline{x} = (x_1, ..., x_m), x_i \in \{0, 1\}, x_1 + ... + x_m = l \}\}$\\
Рёбра: $E = \{\{\overline{x}, \overline{y}\}: \overline{x} \neq \overline{y}, (\overline{x}, \overline{y}) \equiv 0 (p)\}$

Пусть $s(p) = (\sum_{k=0}^{p} C_m^k) + 1, где n = C_m^l$ ($n$ - сколько всего вершин)

\Lemma 1 (О числе независимости, б/д)
$$\alpha(G) \leqslant \sum_{k=0}^{p-1} C_m^k < s(p)$$

\Lemma 2 (О кликовом числе, б/д)
$$\omega(G) \leqslant \sum_{k=0}^{p} C_m^k < s(p)$$

Только из неравенств в леммах следует: $R(s(p), s(p)) > n$

Если мы докажем, что 
$n = n(s) = (e^{\frac{1}{4}} + o(1))^{\frac{ln^2(s)}{ln(ln(s)))}}$ ($n(s)$ - формула зависимости $n$ от $s$), тогда мы докажем теорему для всех $s = s(p)$ воспользовавшись неравенством $R(s(p), s(p)) > n$.

Если утверждение: $n(s) =  (e^{\frac{1}{4}} + o(1))^{\frac{ln^2(s)}{ln(ln(s)))}}$ для $s = s(p)$, то сможем доказать теорему для произвольного $s = s_2$: \\ 
Для этого найдём $max(p): s(p) \leqslant s_2,$ \\
тогда $R(s_2, s_2) \geqslant R(s(p), s(p)) = (e^{\frac{1}{4}} + o(1))^{\frac{ln^2(s(p))}{ln(ln(s(p))))}}$. Из этого мы хотим вывести: $R(s_2, s_2) \geqslant (e^{\frac{1}{4}} + o(1))^{\frac{ln^2(s_2)}{ln(ln(s_2)))}}$
(По словам лектора, утверждение доказывается нудно, через дважды логарифмирование выражения выше. В частности надо пользоваться тем, что простые числа плотно встречаются в натуральном ряде между $n$ и $n + o(n)$ или $n + O(n^{0.525})$ обязательно есть простое число, без этого мы не сможем доказать, что $ln(s)$ и $ln(s(p))$ ведут себя асимптотичекси одинаково).

Однако продолжим. Докажем, что $n(s) =  (e^{\frac{1}{4}} + o(1))^{\frac{ln^2(s)}{ln(ln(s)))}}$, для $s = s(p)$.

Напомню, что $s(p) = \sum_{k=0}^p C_m^k + 1, n = C_m^l$
$$n = C_{p^3}^{p^2} = \frac{p^3 (p^3 - 1) ... (p^3 - p^2 + 1)}{p^2!} = \frac{(p^{3(1+o(1))})^{p^2}}{(p^{2(1+o(1)})^{p^2}} = p^{3p^2(1 + o(1)) - 2p^2(1 + o(1))} = p^{p^2(1 + o(1))}$$ (факториал раскрыли по формуле Стирлинга) 

Поскольку $s(p) = \sum_{k=0}^p C_m^k + 1$, то

$s < (p + 1) C_m^p + 1$

$s > C_m^p + 1$

$$C_m^p = C_{p^3}^p = \frac{p^3(p^3-1)...(p^3-p+1)}{p!}= p^{3p(p+o(1))-p(1+o(1))} = p^{2p(1+o(1))}$$

$s < (p + 1) C_m^p + 1 = (p+1) p^{2p(1+o(1))} + 1 = p^{2p(1+o(1))}$

$s > C_m^p + 1 = p^{2p(1+o(1))} + 1 = p^{2p(1+o(1))}$, поэтому $s = p^{2p(1+o(1))}$

Получили $s = p^{2p(1+o(1))}, n = p^{p^2(1+o(1))} \Rightarrow$

$ln(s) = 2p(1+o(1))ln(p)$

$ln^2(s) = 4p^2 (1+o(1)))ln^2(p)$

$ln(ln(s)) = ln(2) + ln(p) + ln(1 + o(1)) + ln(ln(p)) = (1+o(1)) ln(p)$

$\frac{ln^2(s)}{ln(ln(s))} = 4p^2 ln(p)(1 + o(1))$

$ln(n) = p^2 ln(p) (1 + o(1))$

Из двух последних равенств следует: $ ln (n) \thicksim \frac{ln^2(s)}{4 ln(ln(s))} \Rightarrow n = (e + o(1))^{\frac{ln^2(s)}{4 ln(ln(s))}}$

Далее, воспользовавшись неравенством $R(s, s) > n$, получим $R(s, s) > (e + o(1))^{\frac{ln^2(s)}{4 ln(ln(s))}}$

\EndProof


