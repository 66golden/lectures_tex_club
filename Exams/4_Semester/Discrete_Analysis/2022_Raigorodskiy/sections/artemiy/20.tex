\Def Гиперграфом называется пара $H = (V, E)$, где $V$ — множество вершин, а $E$ — произвольное подмножество $2^V$, где $|E| \geqslant 2$ (т.е. в отличие от обычного графа, ребро гиперграфа это произвольное неупорядоченное множество вершин).

\Def Гиперграф называется $k$-однородным (для $k \geqslant 2$), если $\forall a \in E, |a| = k$ (Каждой ребро содежит ровно $k$ вершин).

\Def Основные экстремальные величины, рассматриваемые в этом разделе:

$f(n, k, t) = \max\{f \in N : \exists k-\text{однородный гиперграф}\ H = (V, E), |V| = n, |E| = f, \forall A, B \in E : |A \cap B| \geqslant t\}$

$h(n, k, t) = \max\{H \in N : \exists k-\text{однородный гиперграф}\ H = (V, E), |V| = n, |E| = h, \forall A, B \in E : |A \cap B| \leqslant t\}$

$m(n, k, t) = \max\{m \in N : \exists k-\text{однородный гиперграф}\ H = (V, E), |V| = n, |E| = m, \forall A, B \in E : |A \cap B| \neq t\}$

Теорема Эрдёша–Ко–Радо (о максимальном числе рёбер в гиперграфе $1$-пересечений).

\Th. (Эрдёш-Ко-Радо)

\begin{equation*}
f(n, k, 1) = 
\begin{cases}
C_n^k & 2k > n \\
C_{n-1}^{k-1} & 2k \leqslant n
\end{cases}
\end{equation*}


\Proof 

1) Cлучай $2k > n$: можно взять всевозможные рёбра размера $k$, поскольку любые два ребра размера $k$ пересекаются. 

2) Cлучай $2k \leq n$: 

Верхняя оценка $f(n, k, 1) \geq C_{n-1}^{k-1}$ доказывается просто: достаточно рассмотреть пример: $E = \{M \subset \{1, ..., n\}, |M| = k, 1 \in M\}$, $|E| = C_{n-1}^{k-1}$. Даное множество удовлетворяет определению $f$. 

Покажем теперь, что $f(n, k, 1) \leq C_{n-1}^{k-1}$. Рассмотрим произвольную совокупность подмножеств: $F =\{\{F_1, ..., F_s\}, \forall i F_i \subset \{1, ..., n\}, |F_i| = k, \forall i, j: |F_i \cap F_j| \geq 1\}$. Наша цель показать, что $\forall F:$ $s \leq C_{n-1}^{k-1}$, из чего следует $f(n, k, 1) \leq C_{n-1}^{k-1}$, тогда теорема будет доказана. Рассмотрим семейство множеств $A = \{A_1, ..., A_n\}$, где $A_1 = \{1, 2, ..., k\}, A_2 = \{2, ..., k + 1\}, ..., A_n = \{n, 1, ..., k - 1\}$. Докажем сначала следующую лемму:

\Lemma $|F \cap A| \leq k$ (круговой метод Катона) при $2k \leq n$

\Proof Предположим противное: $\exists F: |F \cap A| > k$. Рассмотрим данный $F$. Без огранечения общности скажем, что $A_1 \in F \cap A$. (Иначе сдвинем нумерацию вершин, чтобы это выполнилось). Разобьем $A_i$ на пары следующим образом: $(A_i, A_{n-k+i})$. В силу условия $2k \leq n$, получаем, что два множества в паре не пересекаются. Все элементы $|F \cap A|$ пересекаются друг с другом. Следовательно, пересекаются с $A_1$. Рассмотрим только пары с $(A_2, A_{n-k+2})$ по $(A_k, A_{n})$, потому что это все пары имеющие множество пересекающееся с $A_1$, следовательно, только в них содержаться элементы $|F \cap A|$. Всего пар $k-1$, а $|F \cap A| \geq k+1$ (одно из которых $A_1$). По принципу Дирихле найдётся пара, в которой оба множества из $|F \cap A|$. Противоречие. (множества в паре не пересекаются). Следовательно, $|F \cap A| \leq k$. \EndProof

Изначально $V = \{1, 2, ..., n\}$. Рассмотрим любую перестановку $\sigma \in S_n$. Определим множества
$V_\sigma = \{\sigma(1), ..., \sigma(n)\}$ и $A_\sigma = \{\sigma(A_1), ..., \sigma(A_n)\}$, где $\sigma(A_i)$ равно множеству $\{\sigma(i), \sigma(i+1), ...\}$.

Например, для $n = 7$ и $\sigma$ такой, что $V_{\sigma} = \{2, 5, 1, 3, 4, 6, 7\}$ совокупность $A_{\sigma}$ это множество
$\{\{2, 5, 1\}, \{5, 1, 3\}, ..., \{7, 2, 5\}\}$

$\Lemma |F \cap A_{\sigma}| \leq k$ — доказательство аналогично предыдущей лемме.

\begin{equation*}
\text{Определим индикаторы} I(\sigma, F_i) =
\begin{cases}
1 & F_i \in A_{\sigma}\\
0 & \text{иначе}
\end{cases}
\ \text{и посмотрим на следующую величину:}
\end{equation*}

($F_i$ - это просто список номеров, мы к нему не примеряем перестановку)

$$\sum_{\sigma} \sum_{i=1}^s I(\sigma, F_i) = \sum_{i=1}^s \sum_{\sigma} I(\sigma, F_i)$$

При фиксированной перестановке сумма $\sum_{i=1}^s I(\sigma, F_i) = |A_\sigma \cap F| \leq k$, а значит $\sum_{\sigma} \sum_{i=1}^s I(\sigma, F_i) \leq k \cdot n!$. С другой стороны, для любого фиксированного $i$, всего перестановок $\sigma$, таких что $F_i \in A_{\sigma}$ (т.е. $F_i = \{\sigma(j), \sigma(j+1) ..., \sigma(j+k-1)\}$, для некоторого $j$) ровно $n \cdot k! \cdot (n-k)!$. ($n$ - выбор $j$, $k!$ - различные перестановки элементов $F_i$ среди $\{\sigma(j), \sigma(j+1) ..., \sigma(j+k-1)\}$, $(n-k)!$ - перестановки остальных элементов)

Получаем $\sum_{\sigma} I(\sigma, F_i) = n\cdot k!\cdot (n-k)!$. 

Окончательно получаем:

$$s \cdot n \cdot k! \cdot (n-k)! = \sum_{i=1}^s \sum_{\sigma} I(\sigma, F_i) = \sum_{\sigma} \sum_{i=1}^s I(\sigma, F_i) \leq k \cdot n! \Rightarrow s \leq C_{n-1}^{k-1} \Rightarrow$$
$$f(n, k, 1) \leq C_{n-1}^{k-1} \Rightarrow f(n, k, 1)= C_{n-1}^{k-1}$$

\EndProof