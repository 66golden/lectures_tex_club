\textit{По семинарам. На лекциях не было.}

\Def Числом Рамсея $R(m_1, ..., m_k)$ называется минимальное из таких целых положительных чисел x, что для любой расскраски рёбер графа $K_x$ (полного графа на x вершинах) в $k$ цветов для некоторого $i$ найдётся $m_i$-клика $i$-ого цвета.

\Example

1) $R(1, m, n) = 1$ (граф на одной вершине можно считать за клику размера $1$ первого цвета)

2) $R(2, m, n) = R(m, n)$ (если есть ребро $1$го цвета - клика найдена, иначе будет найдёна клика другого цвета, так как всего $2$ цвета, а вершин $R(m, n)$). 

\Statement

1) $R(m, n, p) \leqslant R(R(m, n), p)$

2) $R(m_1, m_2, ..., m_k) \leqslant 2 - k + R(m_1 - 1, m_2, ..., m_k) + ... + R(m_1, m_2, ..., m_k - 1) $

\Proof

1) Раскрасим граф на $R(R(m, n), p)$ вершинах в 3 цвета. Тогда обязательно найдётся либо клика размера $R(m, n)$ на рёбрах первого и второго цвета (А в ней найдётся либо клика размера $m$ первого цвета, либо клика размера $n$ второго цвета), либо клика размера $p$ третьего цвета. Следовательно, $R(m, n, p) \leqslant R(R(m, n), p)$.
 
2) Рассмотрим граф на $q = 2 - k + R(m_1 - 1, m_2, ..., m_k) + ... + R(m_1, m_2, ..., m_k - 1)$ вершинах и покрасим его в $k$ цветов. Рассмотрим произвольную вершину. По принципу Дирихле у неё есть либо $R(m_1 - 1, m_2, ..., m_k)$ рёбер $1$-го цвета, в этом случае мы всегда сможем в подграфе, состоящем из концов этих рёбер и выбранной вершины, найти клику $1$-го цвета размером $m_1$ или ... или $k$-го цвета размером $m_k$ (в данном случае задачу доказали), либо $R(m_1, m_2 - 1, ..., m_k)$ рёбер $2$-го цвета (аналогичное доказательство нахождения необходимых клик), ... либо $R(m_1, m_2, ..., m_k - 1)$ рёбер $k$-го цвета (аналогичное доказательство нахождения необходимых клик). Во всех случаях были найдены необходимые клики. Следовательно,  $R(m_1, m_2, ..., m_k) \leqslant 2 - k + R(m_1 - 1, m_2, ..., m_k) + ... + R(m_1, m_2, ..., m_k - 1)$. 

\EndProof

\Def Числом Рамсея для гиперграфа $R_l(m_1, ..., m_k)$, при $m_1, ..., m_k \geqslant l$, называется минимальное из таких целых положительных чисел, что для любой расскраски всех $l$-элементных подмножеств $x$-элеменого множества в $k$ цветов найдутся $i$ и подмножество размера $m_i$, у которого все $l$-элементные подмножества покрашены в $i$-ый цвет.

\Statement

$1) R_l(m_1, \ldots, m_k) \leqslant R_l(R_l(m_1, m_2), m_3, ..., m_k)$

$2) R_l(m, n) \leqslant R_{l-1}(R_l(m-1, n), R_l(m, n-1)) + 1.$

\Def

1) Аналогично доказательству уверждения $1$ для обычных графов.

2) Рассмотрим гиперграф $G$ на этих вершинах рёбра которого - все $l$ элементные множества. Рёбра произвольно покрасим в $2$ цвета. Рассмотрим произвольную вершину $q$. Если убрать вершину $q$, то некоторые рёбра станут размером $l-1$ (цвет сохраняется). Рассмотрим только их. В оставшемся графе $G_2$ будет $R_{l-1}(R_l(m-1, n), R_l(m, n-1))$ вершин, а все рёбра будут размером $l-1$ (и все $l-1$-элементные подмножества будут рёбрами). Из количества вершин следует, что в графе $G_2$ найдётся либо $R_l(m-1, n)$ элементная клика графа $G_2$ (множество вершин размером $R_l(m-1, n)$ со всеми возможными внутенними рёбрами (в данном случае рёбрами размером $l-1$)) цвета $1$, либо $R_l(m, n-1)$ клика графа $G_2$ цвета $2$. В первом случае на этих $R_l(m-1, n)$ вершинах найдётся либо $n$-клика $G$ цвета $2$, тогда мы нашли, что хотели, либо $(m-1)$-клика графа $G$ цвета $1$, но мы помним, что эти $(m-1)$ вершины составляют $(m-1)$-клику $G_2$ цвета $1$ (так как дело происходит в $R_l(m-1, n)$ элементной клике графа $G_2$ цвета $1$), поэтому если обратно добавить рёбрам $G_2$ вершину $q$ (помним, что цвет рёбер сохранялся при удалении $q$), то получим $m$-элементную клику в $G$, что мы и искали. Случай $2$ аналогично. \EndProof
