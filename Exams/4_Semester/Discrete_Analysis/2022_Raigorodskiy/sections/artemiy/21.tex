1961. Теорема Эрдёш-Ко-Радо(3 математика), но сделали в 1938 году, однако не знали как применить.

$\forall r, s \exists n_0 = n_0(r, s): \forall n \geqslant n_0: f(n, r, s) = C_{r-s}^{n-s}$

1978. Теорема Франкла: Если $s \geqslant 15, r$ - любое, то $n_0(r, s) = (r-s+1)(s+1)$ (при достаточно большом $s$)

1982 Уилсон: $\forall s, r:\ n_0(r, s) = (r-s+1)(s+1)$. Для $n \geqslant n_0: f(n, k, t) = C_{n-t}^{k-t}$, а для меньших $n$: $f(n, k, t) > C_{n-t}^{k-t}$

Есть тривиальная конструкция, когда $n \leqslant 2k - s - 1$, тогда любые два подмножетсва размера $k$ пересекаются по не менее, чем $s$ элементов $\Rightarrow f(n, k, s) = C_n^k$

1996. Теорема Алсведе–Хачатряна.

Зафиксируем $r, s$. Пусть $n \geqslant 2r - s$. Выберем такое $b \in N$, что
$$(r-s+1)(2+\frac{s-1}{b+1}) \leqslant n < (r-s+1)(2+\frac{s-1}{b})$$

(При b = 0 левая часть становится как в теореме Франка-Уилсона, а правая - бесконечность. Также сводится к ЭКР)

Тогда 
$$f(n, r, s) = \left| \{F \subset \{1, ..., n\}: |F| = r \wedge |F \cap \{1, 2, ..., 2b + s\}| \geqslant b + s\} \right| = \sum_{i=b+s}^{2b+s} C_{2b+s}^i C_{n - 2b - s}^{r-i}$$ 
- Эти множества $r$ элементные и пересекаются хотя бы по $s$ элементам.

\Example $n = 8, r = 4, s = 2$. В этом случае при $b = 1$.    

$$f(8, 4, 2) = \left| \{F \subset \{1, ..., 8\}: |F| = 4, |F \cap \{1, 2, ..., 4\}| \geqslant 3\} \right| = 17$$ 

А если воспользоваться оценкой Эрдёш-Ко-Радо: $f(n, r, s) \geqslant C_{n - s}^{r - s}$

Получим $f(8, 4, 2) \geqslant C_{6}^{2} = 15 < 17$. Оценка по АХ оказалась лучше.

