\textbf{Задача об уклонении}

Пусть задан гиперграф с множеством вершин $V = \{ 1, \ldots, n \}$ и множеством рёбер
$\mathcal{M} = \{ M_1, \ldots, M_m \}$ (гиперграф не обязательно однородный). 
Также $\mathcal{M}$ можно интерпретировать как набор подмножеств 
множества $\{1, \ldots, n \}$.

Пусть $\chi$ -- раскраска вершин в два цвета:
$\chi: \{1, \ldots, n \} \to \{ -1, 1 \}$.

Введём функцию $\chi(M_i) := \sum \limits_{j \in M_i} \chi(j)$. 
Эта функция показывает "неравномерность" покраски ребра $M_i$. 
Также введём функцию разброса 
$disc(\mathcal{M}, \chi) := \max \limits_{M_i \in \mathcal{M}} \left| \chi(M_i) \right|$.
Она показывает "неравномерность" покраски в целом (максимальная "неравномерность" покраски ребра). 
Функция $disc(\mathcal{M}) := \min \limits_{\chi} disc(\mathcal{M}, \chi)$ 
минимизирует "неравномерность" раскраски.

\textbf{Теорема (б/д).}

$$
\forall n \forall \mathcal{M} = \{ M_1, \ldots, M_n \} \ 
disc(\mathcal{M}) \leqslant 6 \sqrt{n}
$$

\textbf{Теорема.}

$$
\forall \mathcal{M} = \{ M_1, \ldots, M_m \} \ 
disc(\mathcal{M}) \leqslant \sqrt{2n \ln (2m)}
$$

\textit{Доказательство.}

Рассмотрим случайную раскраску $\chi$ 
(каждой вершине независимо присвоим цвет $1$ или $-1$ с вероятностью $\frac{1}{2}$).

Можно заметить, что $\chi(M_i)$ -- сумма независимых случайных величин, 
принимающих значения $\pm 1$ с вероятностью $\frac{1}{2}$.
Значит $\chi(M_i)$ можно интерпретировать как случайное блуждание ($k := |M_i|$):
$$
P \left( \left| \chi(M_i) \right| \geqslant a \right) \leqslant
2 e^{-\frac{a^2}{2k}} \leqslant 2 e^{-\frac{a^2}{2n}}
$$

Двойка перед $e^{-\frac{a^2}{2k}}$ появилась из-за модуля. 
Возьмём $a := \sqrt{2n \ln (2m)}$:

$$
P \left( \left| \chi(M_i) \right| \geqslant a \right) \leqslant 
2 e^{-\frac{2n \ln (2m)}{2n}} = \frac{2}{2m} = \frac{1}{m}
$$

Значит, 

$$
P \left( \exists i : \left| \chi(M_i) \right| \geqslant a \right) <
m \cdot \frac{1}{m} = 1 \Rightarrow
P \left( \forall i : \left| \chi(M_i) \right| < a \right) > 0
$$