\textbf{Определение.}
$$
m(n, r, s) := max \{ k : \exists r \text{-однородный гиперграф на } n \text{ вершинах } :
|E| = k \text{ и } \forall A, B \in E \ |A \cap B| \neq s \}
$$

Другая интерпретация:

$m(n, r, s)$ -- максимальное число подмножеств $n$-элементного множества, в каждом из
которых ровно $r$ элементов и среди которых любые два множества пересекаются не по $s$
элементам.

Можно ввести граф $G(n, r, s)$, у которого
$V = \{ A \subset \{1, 2, \ldots, n\} : |A| = r \}$,
$E = \{ (A, B) : |A \cap B| = s \}$.
Тогда $m(n, r, s) = \alpha(G(n, r, s))$.

\textbf{Линейно-алгебраическая оценка для $m(n, 3, 1)$.}

$$
m(n, 3, 1) \leq n
$$

\textbf{Линейно-алгебраическая оценка для $m(n, 5, 2)$ и её асимптотическая неулучшаемость.}

Можно легко получить нижнюю оценку на $m(n, 5, 2)$: зафиксируем первые три элемента, а остальные два произвольно выберем из оставшихся. Тогда получившиеся множества будут пересекаться как минимум по трём элементам. Получается, 

$$
m(n, 5, 2) \geqslant C_{n - 3}^2 \sim \frac{n^2}{2}
$$

Из следующей теоремы видно, что оценки нельзя улучщить асимптотически.

\textbf{Теорема.}

$$
m(n, 5, 2) \leqslant C_n^2 + 2C_n^1 + C_n^0 \sim \frac{n^2}{2}
$$

\textit{Доказательство.}

Пусть $H = (V, E)$. Каждому ребру $A \in E$ сопоставим вектор
$\overline{x_A} = (x^1, \ldots, x^n)$, где $x^j = 1 \Leftrightarrow j \in A$.

Пусть $\{ \overline{x_1}, \ldots \overline{x_t} \}$ -- все векторы, построенные 
по рёбрам гиперграфа.
Каждому вектору $\overline{x_i}$ сопоставим 
формальный многочлен $P_{\overline{x_i}} \in \mathbb{Z}_3 [y_1, \ldots, y_n]$.

$$
P_{\overline{x_i}} = (\overline{x_i}, \overline{y}) \cdot 
((\overline{x_i}, \overline{y}) - 1), \text{ где } \overline{y} = (y_1, \ldots, y_n).
$$

\textit{Пример для понимания}

Если $A = \{1, 2, 4, 5, 8\}$, то 
$\overline{x_A} = (1, 1, 0, 1, 1, 0, 0, 1, 0 \ldots, 0)$, а

$P_{\overline{x_A}}(\overline{y}) = 
(y_1 + y_2 + y_4 + y_5 + y_8) (y_1 + y_2 + y_4 + y_5 + y_8 - 1)$.
\\

Докажем, что $P_{\overline{x_1}}, \ldots, P_{\overline{x_t}}$ линейно независимы над $\mathbb{Z}_3$.

$$
c_1 P_{\overline{x_1}} + \ldots + c_t P_{\overline{x_t}} = 0
\Rightarrow
\forall \overline{y} \in \{ 0, 1 \}^n \ 
c_1 P_{\overline{x_1}}(\overline{y}) + \ldots + c_t P_{\overline{x_t}}(\overline{y}) = 0
$$

Возьмём $\overline{y} = \overline{x_1}$.

$P_{\overline{x_1}}(\overline{x_1}) = 
(\overline{x_1}, \overline{x_1})((\overline{x_1}, \overline{x_1}) - 1) = 
5 \cdot (5 - 1) = 20 \equiv 2 \pmod{3}$.

Для $i \geqslant 2$:

$P_{\overline{x_i}}(\overline{x_1}) = 
(\overline{x_i}, \overline{x_1})((\overline{x_i}, \overline{x_1}) - 1)$.

$(\overline{x_i}, \overline{x_1}) = |A_1 \cap A_i|$. 
Возможные значения $|A_1 \cap A_i|$: 0, 1, 3, 4. По модулю 3 это только 0 и 1.
Видно, что если вместо скалярного произведения мы подставим 0 или 1, то 
$P_{\overline{x_i}}(\overline{x_1}) = 0$.

Из этого следует, что $c_1 = 0$. Аналогично $c_2 = \ldots = c_t = 0$. 
Поэтому $P_{\overline{x_1}}, \ldots, P_{\overline{x_t}}$ линейно независимы над $\mathbb{Z}_3$, а значит $t \leqslant dim \mathbb{Z}_3 [y_1, \ldots, y_n]$.

Какова же размерность пространства $\mathbb{Z}_3 [y_1, \ldots, y_n]$? 
Очевидно, это количество базисных векторов. 
Базисные векторы в $\mathbb{Z}_3 [y_1, \ldots, y_n]$ 
(с учётом того, что все наши многочлены степени не выше 2):

$y_i y_j, i \neq j$ -- их $C_n^2$.

$y_i^2, i = 1, \ldots, n$ -- их $C_n^1$

$y_i, i = 1, \ldots, n$ -- их $C_n^1$

$1$ -- $C_n^0$

Получилось, что $t \leqslant C_n^2 + 2C_n^1 + C_n^0$.