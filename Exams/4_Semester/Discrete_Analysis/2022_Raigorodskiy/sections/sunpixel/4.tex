\textbf{Теорема}
$$
    u_n \sim \sqrt{\frac{\pi}{8}} n^{n - \frac{1}{2}}
$$

\textit{Доказательство.}
Из предыдущего пункта известно, что
\begin{equation*}
    u_n = \sum_{k = 3}^{n} C_n^k \cdot
\frac{(k - 1)!}{2} \cdot k \cdot n^{n - k - 1}
\end{equation*}

Перепишем $k$-й член этой суммы в следующем виде:

\begin{multline*}
C_n^k \cdot \frac{(k - 1)!}{2} \cdot k \cdot n^{n - k - 1} = 
\frac{n(n - 1) \ldots (n - k + 1)}{k!} \cdot \frac{k!}{2} \cdot n^{n - k - 1} = \\
= \frac{n^k}{\cancel{k!}} \left( 1 - \frac{1}{n} \right) \ldots \left( 1 - \frac{k - 1}{n} \right) \cdot
\frac{\cancel{k!}}{2} \cdot n^{n - k - 1} = 
\frac{1}{2} n^{n - 1} \cdot \left( 1 - \frac{1}{n} \right) \ldots \left( 1 - \frac{k - 1}{n} \right) = \\
= \frac{1}{2} n^{n - 1} \cdot 
e^{\ln \left( 1 - \frac{1}{n} \right) + \ldots + \ln \left( 1 - \frac{k - 1}{n} \right)}
\end{multline*}

С одной стороны, 
$$
\frac{1}{2} n^{n - 1} \cdot e^{\ln \left( 1 - \frac{1}{n} \right) + \ldots + \ln \left( 1 - \frac{k - 1}{n} \right)}
\leqslant \frac{1}{2} n^{n - 1} e^{-\frac{1}{n} - \ldots - \frac{k - 1}{n}}
= \frac{1}{2} n^{n - 1} e^{-\frac{k(k - 1)}{2n}}
$$

С другой строны по формуле Тейлора:

$$
\frac{1}{2} n^{n - 1} \cdot e^{\ln \left( 1 - \frac{1}{n} \right) + \ldots + \ln \left( 1 - \frac{k - 1}{n} \right)}
\leqslant \frac{1}{2} n^{n - 1} 
e^{-\frac{1}{n} - \ldots - \frac{k - 1}{n} + O \left( \frac{k^3}{n^2} \right)}
= \frac{1}{2} n^{n - 1} e^{-\frac{k(k - 1)}{2n} + O \left( \frac{k^3}{n^2} \right)}
$$

Если $k \gg n^{\frac{1}{2}}$, то воспользуемся первой оценкой, 
если $k \ll n^{\frac{2}{3}}$, то воспользуемся второй оценкой. 
Для определенности разобьём отрезок от $3$ до $n$ на $[3, n^{0.6}]$ и $[n^{0.6}, n]$:

$$
\sum_{k = 3}^{n} C_n^k \cdot \frac{(k - 1)!}{2} \cdot k \cdot n^{n - k - 1} = 
\underbrace{\sum_{k = 3}^{[n^{0.6}]} C_n^k \cdot \frac{(k - 1)!}{2} \cdot k \cdot n^{n - k - 1}}_{S_1} + 
\underbrace{\sum_{k = [n^{0.6}] + 1}^{n} C_n^k \cdot \frac{(k - 1)!}{2} \cdot k \cdot n^{n - k - 1}}_{S_2}
$$

Оценим $S_2$. Воспользуемся тем, что 
$$
-\frac{k(k - 1)}{2n} \leqslant \frac{-([n^{0.6}] + 1) [n^{0.6}]}{2n} \sim
-\frac{n^{1.2}}{2n} = -\frac{n^{0.2}}{2}
$$

Тогда $S_2$ можно оценить следующим образом:

$$
S_2 \leqslant \frac{1}{2} n^{n - 1} \sum_{[n^{0.6}] + 1}^{n} 
e^{-\frac{k(k - 1)}{2n}} \leqslant
\frac{1}{2} n^{n - 1} \cdot
\underbrace{n \cdot e^{-(1 + o(1)) \frac{n^{0.2}}{2}}}_{\to 0} 
= o \left( n^{n - \frac{1}{2}} \right)
$$

Оценим $S_1$:

\begin{multline*}
S_1 \leqslant \frac{1}{2} n^{n - 1} \sum_{k = 3}^{[n^{0.6}]}
e^{-\frac{k(k - 1)}{2n} + O \left( \frac{k^3}{n^2} \right)} \leqslant
\frac{1}{2} n^{n - 1} \sum_{k = 3}^{[n^{0.6}]}
e^{-\frac{k(k - 1)}{2n} + 
O \left( \underbrace{\frac{(n^{0.6})^3}{n^2}}_{\to 0} \right)} \sim \\
\frac{1}{2} n^{n - 1} \sum_{k = 3}^{[n^{0.6}]} e^{-\frac{k(k - 1)}{2n}} \sim
\frac{1}{2} n^{n - 1} \sum_{k = 3}^{[n^{0.6}]} e^{-\frac{k^2}{2n}}
\end{multline*}

Если мы докажем $\sim$, помеченную вопросом, то мы доказали теорему:

$$
\sum_{k = 3}^{[n^{0.6}]} e^{-\frac{k^2}{2n}} \overset{?}{\sim} 
\sum_{k = 0}^{\infty} e^{-\frac{k^2}{2n}} \sim
\int_{0}^{\infty} e^{-\frac{k^2}{2n}} = \sqrt{\frac{\pi n}{2}}
$$

Понятно, что первые три члена не влияют на асимптотику:
$\sum \limits_{k = 0}^2 e^{-\frac{k^2}{2n}} = o(\sqrt{n})$.

Посмотрим на "хвост" этого ряда:

$$
\sum_{[n^{0.6}] + 1}^{\infty} e^{-\frac{k^2}{2n}} = 
\underbrace{\sum_{[n^{0.6}] + 1}^{n^2}}_{S_3} \ldots + 
\underbrace{\sum_{n^2 + 1}^{\infty} \ldots}_{S_4}
$$

По аналогии с суммой $S_1$, $S_3$ оценим таким образом:

$S_3 < e^{-\frac{n^{0.2}}{2}} \cdot n^2 \to 0$

Чтобы оценить $S_4$, посмотрим на частное двух последовательных членов:

$$
\frac{e^{-\frac{k^2}{2n}}}{e^{-\frac{(k + 1)^2}{2n}}} = 
e^{\frac{(k + 1)^2}{2n} - \frac{k^2}{2n}} = 
e^{\frac{k}{n} + \frac{1}{2n}} > e^n \text{ (так как } k > n^2 \text{)}
$$

Вынесем из $S_4$ первый член за скобки и посчитаем сумму 
бесконечной геометрической прогрессии
(обозначим член последовательности за $a_k(n)$):

$$
S_4 = e^{-\frac{(n^2 + 1)^2}{2n}}
\left( 1 + \underbrace{\frac{a_{n^2 + 2}}{a_{n^2 + 1}}}_{ < e^{-n}} + 
\underbrace{\frac{a_{n^3 + 2}}{a_{n^2 + 2}} 
\cdot \frac{a_{n^2 + 2}}{a_{n^2 + 1}}}_{ < e^{-2n}}
+ \ldots \right) <
\underbrace{e^{-\frac{(n^2 + 1)^2}{2n}}}_{\to 0}
\cdot \underbrace{\frac{1}{1 - e^{-n}}}_{\to 1} \to 0
$$