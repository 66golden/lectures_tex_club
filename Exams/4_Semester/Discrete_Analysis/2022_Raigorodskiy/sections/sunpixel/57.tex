\textbf{Определение.}
$$
m(n, r, s) := max \{ k : \exists r \text{-однородный гиперграф на } n \text{ вершинах } :
|E| = k \text{ и } \forall A, B \in E \ |A \cap B| \neq s \}
$$

Другая интерпретация:

$m(n, r, s)$ -- максимальное число подмножеств $n$-элементного множества, в каждом из
которых ровно $r$ элементов и среди которых любые два множества пересекаются не по $s$
элементам.

\textbf{Явная конструкция для $m(n, 3, 1)$}

Рассмотрим граф $G(n) = (V, E)$, где 
$V = \{ A \subset \{1, 2, \ldots, n\} : |A| = 3\}$ ($|V| = C_n^3$), а

$E = \{ (A, B)\ :\ |A \cap B| = 1\}$.

Тогда $m(n, 3, 1) = \alpha(G(n))$.

Утверждается, что

\begin{equation*}
\alpha(G(n)) = 
\begin{cases}
n & \text{если } n \equiv 0 \pmod 4\\
n - 1 & \text{если } n \equiv 1 \pmod 4\\
n - 2 & \text{иначе}
\end{cases}
\end{equation*}

Оставлено в качестве упражнения. 
Доказательство можно провести индукцией по $n$ (если успею, допишу).

\textbf{Линейно-алгебраическая оценка для $m(n, 3, 1)$}

$$
m(n, 3, 1) = \alpha(G(n)) \leq n
$$

\textit{Доказательство.}
Пусть $W \subset V$ -- независимое множество вершин,
$W = \{ A_1, \ldots, A_t \}$.

$$
\forall i\ |A_i| = 3; \forall i \neq j |A_i \cap A_j| \in \{0, 2\}
$$

Каждому $A_i$ сопоставим вектор $\overline{x_i} = (x^1, \ldots, x^n) \in \mathbb{Z}_2^n$,
в котором $x^j = 1 \Leftrightarrow j \in A_i$.
Тогда мощность пересечения $A_i$ и $A_j$ выражается как скалярное произведение $\overline{x_i}$ и $\overline{x_j}$:
$|A_i \cap A_j| = (\overline{x_i}, \overline{x_j})$.

Докажем, что $\overline{x_1}, \ldots, \overline{x_t}$ 
линейно независимы над $\mathbb{Z}_2^n$.

Пусть $c_1 \overline{x_1} + \ldots + c_t \overline{x_t} = \overline{0}$. Левую и правую часть равенство скалярно умножим на $x_1$ в $\mathbb{Z}_2^n$:

$$
    c_1 \underbrace{(\overline{x_1}, \overline{x_1})}_{\equiv 1} + 
    c_2 \underbrace{(\overline{x_1}, \overline{x_2})}_{\equiv 0} + \ldots +
    c_t \underbrace{(\overline{x_1}, \overline{x_t})}_{\equiv 0} = 0
$$

Получается, $c_1 = 0$. Аналогично $c_2 = c_3 = \ldots = c_t = 0$.

Это означает, что $\overline{x_1}, \ldots, \overline{x_t}$ независимы в $\mathbb{Z}_2^n$, а значит $t \leqslant dim \mathbb{Z}_2^n = n$.