\textbf{Теорема 3.}

$$
m(n) \leqslant (1 + o(1)) \frac{e \ln 2}{4} n^2 2^n
$$

\textit{Доказательство.}

В дальнейшем будем считать, что $n$ -- чётно. Случай с нечётным $n$ аналогичный.

Определим множество вершин нашего гиперграфа таким образом:
$V = \{1, 2, \dots, v\}$, где $v = \frac{n^2}{2}$.

Пусть $m$ -- количество рёбер в гиперграфе. (Цель: доказать, что $m \sim \frac{e \ln 2}{4} n^2 2^n$).

Построим случайный гиперграф: выбираем рёбра с возвращением (согласно классическому определению вероятности). Если некоторые рёбра мы выберем несколько раз, то просто их отождествим.

Пусть $A_i$ -- событие, что мы выбрали $i$-е ребро. Тогда $P(A_i) = \frac{1}{C_v^n}$.

Зафиксируем конкретную раскраску $\chi$ множества вершин в 2 цвета (всего их $2^v$). Пусть в этой раскраске $r$ вершин красного цвета и $b$ -- синего ($r + b = v$).

$$
P(A_1 \text{ одноцветно в } \chi) = \frac{C_r^n + C_b^n}{C_v^n} \geqslant 
\frac{2 C_{\frac{v}{2}}^n}{C_v^n} := p
$$

Здесь мы воспользовались выпуклостью биномиального коэффициента (это свойство было при доказательстве оценок для двудольных чисел Рамсея) и обозначили получившуюся дробь за $p$ (которая, кстати, не зависит от раскраски $\chi$).

$$
P(A_1 \text{ неодноцветно в } \chi) \leqslant 1 - p
$$
Всего у нас $m$ рёбер, поэтому
$$
P(\forall i \ A_i \text{ неодноцветно в } \chi) \leqslant (1 - p)^m
$$
$$
P(\exists \chi \ :\  \forall i \ A_i \text{ неодноцветно в } \chi) \leqslant
2^v (1 - p)^m < 1
$$
$$
P(\forall \chi \  \exists i \ :\ A_i \text{ одноцветно в } \chi) \geqslant
1 - 2^v (1 - p)^m
$$
Если $1 - 2^v (1 - p)^m > 0$, то в любой раскраске есть одноцветное ребро. Теперь распишем это неравенство:

\begin{multline*}
C_v^n = \frac{v (v - 1) \ldots (v - n + 1}{n!} = 
\frac{v^n}{n!} \left( 1 - \frac{1}{v} \right) \ldots \left( 1 - \frac{n - 1}{v} \right) = \\
= \frac{v^n}{n!} e^{-\frac{1}{v} - \ldots - \frac{n - 1}{v} + 
O \left( \frac{1}{v^2} + \ldots + \frac{(n - 1)^2}{v^2} \right)}
= \frac{v^n}{n!} e^{-\frac{n(n - 1)}{2v} + O \left( \frac{n^3}{v^2} \right)}
\end{multline*}

Так как $v = \frac{n^2}{2}$, то $O \left( \frac{n^3}{v^2} \right) = O \left( \frac{1}{n} \right) \to 0, n \to \infty$.

$$
\frac{v^n}{n!} e^{-\frac{n(n - 1)}{2v} + O \left( \frac{n^3}{v^2} \right)}
\sim \frac{v^n}{n!} \cdot e^{-\frac{n^2}{2v} + \frac{n}{2v}}
$$

Так как $v = \frac{n^2}{2}$, то $\frac{n}{2v} \to 0, n \to \infty$. Поэтому

$$
\frac{v^n}{n!} \cdot e^{-\frac{n^2}{2v} + \frac{n}{2v}} \sim
\frac{v^n}{n!} \cdot e^{-\frac{n^2}{2v}} \sim
\frac{v^n}{n!} e^{-1}
$$

Аналогично 

$$
C_{\frac{v}{2}}^{n} \sim \frac{\left( \frac{v}{2} \right)^n}{n!} e^{-2}
$$

Поэтому

$$
p = \frac{2 C_{\frac{v}{2}}^n}{C_v^n} \sim 2^{1 - n} \cdot e^{-1}
$$

$$
2^v \cdot (1 - p)^m \leqslant 2^v \cdot e^{-pm} = 
e^{\frac{n^2}{2} \cdot \ln 2 - 2^{1 - n} \cdot e^{-1} \cdot m}
$$

Так как $m \sim (1 + o(1))\frac{e \ln 2}{4} n^2 2^n$, то

$$
2^{1 - n} \cdot e^{-1} \cdot m \sim
(1 + o(1)) 2^{1 - n} \cdot e^{-1} \cdot \frac{e \ln 2}{4} n^2 2^n = 
(1 + o(1))\frac{n^2}{2} \ln 2
$$

Поэтому 
$$
2^v \cdot (1 - p)^m \leqslant 
e^{\frac{n^2}{2} \cdot \ln 2 - 2^{1 - n} \cdot e^{-1} \cdot m} \sim
e^{\frac{n^2}{2} \cdot \ln 2 - (1 + o(1))\frac{n^2}{2} \ln 2} = 
e^{-o(1)} \to 0
$$

Значит, $2^v \cdot (1 - p)^m < 1$.