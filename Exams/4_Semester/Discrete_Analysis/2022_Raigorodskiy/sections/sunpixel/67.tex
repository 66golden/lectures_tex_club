\textbf{Определение 1.}
\textit{Хроматическое число гиперграфа} -- минимальное количество цветов, в которые можно раскрасить вершины гиперграфа так, чтобы все рёбра были неодноцветными.

\textbf{Определение 2.}
Пусть $H$ -- $n$-однородный гиперграф. 
Пусть $A$ и $B$ -- его рёбра, такие что $|A \cap B| = 1$.
Пусть $\sigma$ -- нумерация вершин $H$.

Назовём пару $(A, B)$ \textit{упорядоченной 2-цепью} в $\sigma$, если номера вершин из $A$
в $\sigma$ предшествуют номеру вершины в $A \cap B$, а номера вершин из $B$ следуют за ним.

Другими словами: пусть у нас $A$ и $B$ пересекаются по общей вершине $v$. Тогда в нумерации $\sigma$ номера всех вершин в $A$ должны быть не больше $v$, а в $B$ -- не меньше $v$.

\textbf{Лемма Плухара.}

$\chi(H) = 2 \Leftrightarrow \exists \sigma \ :$ в ней нет упорядоченных 2-цепей.

\textit{Доказательство.}

$\Rightarrow$. Пусть $\chi(H) = 2$. Пусть количество красных вершин в этой раскраске -- $r$.

Занумеруем сначала все красные вершины (номерами от $1$ до $r$), затем все синие (номерами от $r + 1$ до $|V|$).

Предположим, что в такой нумерации есть упорядоченная 2-цепь $(A, B)$. Посмотрим на цвет вершины, по которой они пересекаются. Если он красный, то и всё ребро $A$ (по определению 2-цепи) красное -- противоречие с правильной раскраской. Если он синий, от и всё ребро $B$ синее -- опять противоречие.

$\Leftarrow$. Пусть $\exists \sigma \ :$ в ней нет упорядоченных 2-цепей.

Красим в каждом ребре последнюю вершину в синий цвет, остальные -- в красный.
Понятно, что ребро не может быть полностью красным (так как последняя вершина в нём должна быть синяя).

Если же ребро $A$ полностью синее, то и первая (вершина с самым маленьким номером) в этом ребре покрашено в синий цвет, а значит является последней вершиной (вершиной с самым большим номером) в другом ребре $B$. Пара $(B, A)$ является упорядоченной 2-цепью, что приводит к противоречию.