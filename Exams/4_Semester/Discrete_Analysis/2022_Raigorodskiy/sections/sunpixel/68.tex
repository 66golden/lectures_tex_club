\textbf{Лемма Плухара.}

$\chi(H) = 2 \Leftrightarrow \exists \sigma \ :$ в ней нет упорядоченных 2-цепей.

\textbf{Теорема.}

Если у $n$-однородного гиперграфа 
$
|E|^2 < \frac{(2n - 1)!}{\left( (n - 1)! \right)^2}
$
, то $\chi(H) = 2$.

Другая формулировка: $m(n) \geqslant \frac{\sqrt{(2n - 1)!}}{(n - 1)!}$

\textit{Доказательство.}

Пусть $H = (V, E)$ -- $n$-однородный гиперграф.

Рассмотрим случайную $\sigma$, то есть $P(\sigma) = \frac{1}{|V|!}$.

Рассмотрим $A, B \in E\ :\ |A \cap B| = 1$ (в общем случае таких рёбер может и не быть).

$$
P(\ (A, B) \text{ -- упорядоченная 2-цепь }) = \frac{\left( (n - 1)! \right)^2}{(2n - 1)!}
$$

Вероятность посчитана так:

Общая вершина $v$ фиксирована, нужно занумеровать остальные вершины в рёбрах $A$ и $B$.
Всего количество способов занумеровать вершины в рёбрах $A$ и $B$ -- $(2n - 1)!$.
$(A, B)$ будет упорядоченной 2-цепью, если номера всех вершин в $A$ будет не больше $v$, в $B$ -- не меньше $v$. Нам важны не номера, а только порядок этих вершин, поэтому благоприятных исходов $\left( (n - 1)! \right)^2$.

Мы не знаем, сколько пар $A, B \in E\ :\ |A \cap B| = 1$ в гиперграфе, однако можем оценить это число сверху $|E|^2$.

$$
P(\exists \ (A, B) \text{ -- упорядоченная 2-цепь }) 
\leqslant |E|^2 \cdot \frac{\left( (n - 1)! \right)^2}{(2n - 1)!}
$$

Если $|E|^2 \cdot \frac{\left( (n - 1)! \right)^2}{(2n - 1)!} < 1$, то существует $\sigma$, такая что в графе нет упорядоченных 2-цепей, значит, по критерию Плухара, $\chi(H) = 2$.