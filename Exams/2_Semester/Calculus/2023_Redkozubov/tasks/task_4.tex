\section{4. Степенные ряды. Теорема Коши-Адамара. Радиус и круг сходимости, равномерная сходимость степенных рядов. Теорема Абеля. Дифференцируемость суммы степенного ряда. Теорема единственности, ряд Тейлора. Пример бесконечно дифференцируемой функции, не разлагающейся в степенной ряд. Достаточное условие разложимости функции в степенной ряд. Ряды Тейлора $e^{x}$, $\sin x$, $\cos x$, $(1 + x)^{\alpha}$, $\ln(1 + x)$.}

\begin{definition}
    \emph{Степенным рядом} называется функциональный ряд вида
    \begin{equation}
        \label{power-series}
        \sum_{n = 0}^\infty a_n (x - x_0)^n,
    \end{equation}
    где $a_n, x_0 \in \R$ и $x$ --- действительная переменная, или $a_n, x_0 \in \mathbb{C}$ и $x$ --- комплексная переменная \emph{(комплексный степенной ряд)}.
\end{definition}


\begin{theorem}[Коши-Адамар]
    \label{cauchy-hadamard}
    Пусть $R = \frac{1}{\overline{\lim}_{n \rightarrow \infty} \sqrt[n]{|a_n|}}$.

    Тогда:
    \begin{enumerate}
        \item при $|x - x_0| < R$ ряд (\ref{power-series}) сходится, причём абсолютно;
        \item при $|x - x_0| > R$ ряд (\ref{power-series}) расходится;
        \item если $r \in (0, R)$, то ряд (\ref{power-series}) равномерно сходится на $\overline{B_r}(x_0) = \{x: |x - x_0| \le r\}$.
    \end{enumerate}

    \begin{proof}
        Пусть $x \neq x_0$, тогда
        \[
            q \coloneqq \overline{\lim_{n \rightarrow \infty}} \sqrt[n]{|a_n (x - x_0)^n|} = |x - x_0| \overline{\lim_{n \rightarrow \infty}} \sqrt[n]{|a_n|} = \frac{|x - x_0|}{R}.
        \]

        Если $|x - x_0| < R$, то $q < 1$ и, значит, по признаку Коши $\sum_{n = 0}^\infty |a_n (x - x_0)^n|$ сходится, то есть, ряд (\ref{power-series}) сходится абсолютно.

        Если $|x - x_0| > R$, то $q > 1$ и, значит, по признаку Коши $n$-й член ряда не стремится к нулю, ряд (\ref{power-series}) расходится и \emph{абсолютно расходится} (то есть, расходится ряд из модулей членов).

        Пусть $r \in (0, R)$. По доказанному ряд (\ref{power-series}) абсолютно сходится в точке $x = x_0 + r$, то есть сходится ряд $\sum_{n = 0}^\infty |a_n|r^n$. Если $|x - x_0| \le r$, то $|a_n (x - x_0)^n| \le |a_n| r^n$. Тогда по признаку Вейерштрасса ряд (\ref{power-series}) равномерно сходится на $B_r(x_0)$.
    \end{proof}
\end{theorem}

\begin{definition}
    Величина $R$ из теоремы (\ref{cauchy-hadamard}) называется \emph{радиусом сходимости} ряда (\ref{power-series}).

    $B_R(x_0) = \{x: |x - x_0| < R\}$ называется \emph{интервалом сходимости} (\emph{кругом сходимости} в комлексной плоскости).
\end{definition}

Из теоремы (\ref{cauchy-hadamard}) получаем:
\begin{corollary}
    Пусть для $R \in [0, +\infty]$ выполнено следующее: при $|x - x_0| < R$ ряд абсолютно сходится и при $|x - x_0| > R$ ряд абсолютно расходится, то $R$ --- радиус сходимости. 
\end{corollary}

\begin{theorem}[Абель]
    \label{abel-power-series}
    Если степенной ряд (\ref{power-series}) сходится в точке $x_{1} \neq x_{0}$, то он сходится равномерно на отрезке с концами $x_{1}, x_{0}$.
\end{theorem}

\begin{proof}
    По условию ряд $\sum_{n = 0}^\infty a_n (x_1 - x_0)^n$ сходится. Рассмотрим последовательность $\{t^{n}\}:$ она монотонна при любом $t \in [0, 1]$ и равномерно ограниченна. По признаку Абеля (\ref{abel-func-series}) ряд $\sum_{n = 0}^\infty a_n (x_1 - x_0)^n t^{n}$ равномерно сходится на $[0, 1]$. Сделав замену $t = \frac{x - x_{0}}{x_{1} - x_{0}}$, получим, что ряд (\ref{power-series}) равномерно сходится на $\{x: x = x_{0} + t(x_{1} - x_{0})\}$.
\end{proof}

\begin{note}
    Если $x_{1} \in B_{R}(x_{0})$, то предыдущая теорема вытекает из теоремы Коши--Адамара (\ref{cauchy-hadamard}), поэтому интерес представляет случай, когда $x_{1}$ лежит на границе круга сходимости.
\end{note}

\begin{lemma}
    \label{lem1-power-series}
    Если ряд (\ref{power-series}) имеет радиус сходимости $R$, то ряд $\sum_{n = 1}^{+\infty}n a_{n}(x - x_{0})^{n - 1}$ также имеет радиус сходимости $R$.
\end{lemma}

\begin{proof}
    Так как $\lim_{n \to +\infty} \sqrt[n]{n} = 1$, то последовательности $\{\sqrt[n]{|a_{n}|}\}$ и $\{\sqrt[n]{n|a_{n}|}\}$ имеют одинаковое множество частичных пределов, значит $\overline{\lim}_{n \to +\infty} \sqrt[n]{|a_{n}|}$ и $\overline{\lim}_{n \to +\infty} \sqrt[n]{n |a_{n}|}$ равны. Тогда по формуле Коши--Адамара ряды $\sum_{n = 0}^{+\infty} a_{n}(x - x_{0})^{n}$ и $\sum_{n = 1}^{+\infty}n a_{n}(x - x_{0})^{n}$ имеют одинаковые радиусы сходимости.

    Ряды $\sum_{n = 1}^{+\infty}n a_{n}(x - x_{0})^{n - 1}$ и $\sum_{n = 1}^{+\infty}n a_{n}(x - x_{0})^{n}$ отличаются при $x \neq x_{0}$ ненулевым множителем (при $x = x_{0}$ оба сходятся). Следовательно, эти ряды сходятся одновременно. Тогда, радиусы сходимости этих рядов также совпадают.
\end{proof}

\begin{theorem}
    \label{th1-power-series}
    Если $f(x) = \sum_{n = 0}^{+\infty} a_{n}(x - x_{0})^{n}$ -- сумма степенного ряда с радиусом сходимости $R > 0$, то функция $f$ бесконечно дифференцируема в $B_{R}(x_{0})$, и для всякого $m \in \N$ выполнено:
    \[f^{(m)}(x) = \sum_{n = m}^{+\infty}n(n-1)\cdot\ldots\cdot(n - m + 1) a_{n}(x - x_{0})^{n - m}.\]
\end{theorem}

\begin{proof}
    По лемме (\ref{lem1-power-series}) при дифференцировании радиус сходимости ряда не меняется, поэтому нам достаточно доказать утверждение для $m = 1$, после чего применить индукцию. 

    Пусть $0 < r < R$. По теореме (\ref{cauchy-hadamard}) исходный ряд и ряд $\sum_{n = 1}^{\infty}n a_{n}(x - x_{0})^{n - 1}$ равномерно сходятся на отрезке $[x_{0} - r, x_{0} + r]$. Обозначим через $g$ сумму продифференцированного ряда. Тогда по следствию о почленном дифференцировании ряда из теоремы (\ref{covergence-3.4}) функция $f$ дифференцируема на $[x_{0} - r, x_{0} + r]$, причем $f' = g$. Так как $r \in (0, R)$ -- любое, то равенство выполняется на $(x_{0} - R, x_{0} + R)$.
\end{proof}

\begin{corollary}[теорема о единственности]
    Если степенные ряды $\sum_{n = 0}^{+\infty} a_{n}(x - x_{0})^{n}$ и $\sum_{n = 0}^{+\infty} b_{n}(x - x_{0})^{n}$ сходятся в круге $B_{\delta}(x_{0})$, и их суммы там совпадают, то $a_{n} = b_{n}$, $n = 0, 1, 2, \ldots$
\end{corollary}

\begin{corollary}
    \label{cor2-power-series}
    Сумма степенного ряда с радиусом сходимости $R > 0$ имеет первообразную $F(x) = C + \sum_{n = 0}^{+\infty} \frac{a_{n}}{n + 1}(x - x_{0})^{n + 1}$ при $|x - x_{0}| < R$.
\end{corollary}

\begin{definition}
    Пусть функция $f$ определена в некоторой окрестности точки $x_{0}$ и в точке $x_{0}$ имеет производные любого порядка. Тогда $\sum_{n = 0}^{+\infty} \frac{f^{(n)}(x_{0})}{n!}(x - x_{0})^{n}$ называется \textit{рядом Тейлора} функции $f$ с центром в точке $x_{0}$. Для $x_{0} = 0$ ряд называют \textit{рядом Маклорена}.
\end{definition}

\begin{example}
    Рассмотрим $f: \R \to \R$, 
    \[f(x) = \begin{cases}
        0, \ x \leq 0; \\
        e^{- \frac{1}{x}}, \ x > 0.
    \end{cases}\]
    
    Существование производных любого порядка в точке $x \neq 0$ следует из теоремы о дифференцировании композиции. Более того, $f^{(n)}(x) = 0$ при $x < 0$ и $f^{(n)}(x) = p_{n}(\frac{1}{x})e^{-\frac{1}{x}}$, где $p_{n}(t)$ -- многочлен степени $2n$. последнее утверждение можно установить по индукции: $p_{0}(t) = 1$ и дифференцирование $f^{(n)}$ дает соотношение $p_{n + 1}(t) = t^{2} [p_{n}(t) - p_{n}'(t)]$.

    Индукцией по $n$ покажем, что $f^{(n)}(0) = 0$. Для $n = 0$ это верно по условию. Если предположить, что $f^{(n)}(0) = 0$, то $(f^{(n)})'_{-}(0) = 0$ и
    \[(f^{(n)})'_{+}(0) = \lim_{h \to +0} \frac{f^{(n)}(h) - f^{(n)}(0)}{h} = \lim_{h \to +0}\frac{p_{n}(\frac{1}{h})e^{-\frac{1}{h}}}{h} = \lim_{t \to +\infty} \frac{t p_{n}(t)}{e^{t}} = 0,\]
    поскольку по правилу Лопиталя $\lim_{t \to +\infty} \frac{t^{m}}{e^{t}} = 0$ для всех $m \in \N_{0}$. Это доказывает, что $f^{(n + 1)}(0) = 0$.

    Таким образом, ряд Маклорена функции $f$ нулевой, но он не сходится к $f$ ни в какой окрестности нуля.
\end{example}

\begin{lemma}[Достаточное условие представимости функции степенным рядом.]
    Если на $(x_{0} - \rho, x_{0} + \rho)$ функция $f$ имеет производные всех порядков и 
    \[\exists C > 0 \ \forall n \in \N_{0} \ \forall x \in (x_{0} - \rho, x_{0} + \rho) \left(\left|f^{(n)}(x)\right| \leq \frac{C n!}{\rho^{n}}\right),\]
    то 
    \[f(x) = \sum_{n = 0}^{+\infty} \frac{f^{(n)}(x_{0})}{n!}(x - x_{0})^{n}\]
    для всех $x \in (x_{0} - \rho, x_{0} + \rho)$.
\end{lemma}

\begin{proof}
    Так как $\sqrt[n]{\frac{f^{(n)}(x)}{n!}} \leq \frac{C^{\frac{1}{n}}}{\rho} \to \frac{1}{\rho}$, то по формуле Коши-Адамара (\ref{cauchy-hadamard}) для $x \in (x_{0} - \rho, x_{0} + \rho)$ найдется $c$ между $x_{0}$ и $x$, что
    \[\left|f(x) - \sum_{k = 0}^{n}\frac{f^{(k)}(x_{0})}{k!}(x - x_{0})^{k}\right| = \left|\frac{f^{(n + 1)}(c)}{(n + 1)!}(x - x_{0})^{n + 1}\right|.\]
    Поскольку $|f^{(n + 1)}(c)| \leq C$, то справедлива оценка:
    \[\left|\frac{f^{(n + 1)}(c)}{(n + 1)!}(x - x_{0})^{(n + 1)}\right| \leq C\left|\frac{x - x_{0}}{\rho}\right|^{n + 1} \to 0,\]
    что завершает доказательство.
\end{proof}

\begin{corollary}
    Ряды Маклорена функций $\exp, \sin, \cos$ сходятся на $\R$ к самим функциям, то есть $\forall x \in \R:$
    \[e^{x} = \sum_{n = 0}^{+\infty} \frac{x^{n}}{n!},\]
    \[\sin(x) = \sum_{n = 0}^{+\infty} \frac{(-1)^{n}}{(2n+1)!}x^{2n + 1},\]
    \[\cos(x) = \sum_{n = 0}^{+\infty} \frac{(-1)^{n}}{(2n)!}x^{2n}.\]
\end{corollary}

\begin{proof}
    Все указанные функции бесконечно дифференцируемы на $\R$, причем $(e^x)^{(n)} = e^x$, $\sin^{(n)}(x) = \sin(x + \frac{\pi n}{2})$, $\cos^{(n)}(x) = \cos(x + \frac{\pi n}{2})$.

    Пусть $\delta > 0$ и $|x| < \delta$. Тогда $(e^x)^{(n)} \leq e^{\delta}$, $|\sin^{(n)}(x)| \leq 1$, $|\cos^{(n)}(x)| \leq 1$.

    Следовательно, по следствию \ref{cor2-power-series} ряды Маклорена этих функций сходятся к самим функциям на $(-\delta, \delta)$. Так как $\delta > 0$ -- любое, то предположение верно и на $\R$.
\end{proof}

\begin{example}
    Так как $\frac{1}{1 + x} = \sum_{n = 1}^{+\infty} (-1)^{n - 1}x^{n - 1}$ при $|x| < 1$, то по (\ref{cor2-power-series})
    
    \[\ln(1 + x) = \sum_{n = 1}^{+\infty}\frac{(-1)^{n - 1} x^{n}}{n}, \ |x| < 1.\]

    Ряд в правой части сходится при $x = 1$, поэтому его сумма непрерывна на $(-1, 1]$ и, значит, равенство имеет место при $x = 1$. Получаем известный нам результат, что $\sum_{n = 1}^{+\infty}\frac{(-1)^{n - 1}}{n} = \ln 2$.
\end{example}

\begin{theorem}[биномиальный ряд]
    Если $\alpha \not\in \N_{0}$ и $C_{\alpha}^{n} = \frac{\alpha \cdot (\alpha - 1) \cdot \ldots \cdot (\alpha - n + 1)}{n!}$, $C_{\alpha}^{0} = 1$, то 
    \[(1 + x)^{\alpha} = \sum_{n = 0}^{+\infty} C_{\alpha}^{n}x^{n}, \ |x| < 1.\]
\end{theorem}

\begin{proof}
    Пусть $f(x) = (1 + x)^{\alpha}$. Тогда $f^{(n)}(x) = \alpha \cdot (\alpha - 1) \cdot \ldots \cdot (\alpha - n + 1) \cdot (1 + x)^{\alpha - 1}$ и, значит, $\frac{f^{(n)}(0)}{n!} = C_{\alpha}^{n}$. При $x \not= 0$:
    \[\lim_{n \to +\infty}\frac{|C_{\alpha}^{n + 1} x^{n + 1}|}{|C_{\alpha}^{n}x^{n}|} = \lim_{n \to +\infty} \frac{|\alpha - n||x|}{n + 1} = |x|.\]
    
    Если $|x| < 1$, то ряд абсолютно сходится по признаку Даламбера.
    Если $|x| > 1$, то ряд абсолютно расходится по признаку Даламбера.
    Следовательно, $R_{\text{сх}} = 1$.

    Обозначим $g(x) = \sum_{n = 0}^{+\infty} C_{\alpha}^{n}x^{n}$, и покажем, что $g \equiv f$ на $(-1, 1)$, т.е. $(1 + x)^{-\alpha} g(x) = 1$ при $x \in (-1, 1)$. Для этого найдем производную функции $(1 + x)^{-\alpha} g(x)$. По теореме (\ref{th1-power-series}) имеем
    
    \[((1 + x)^{-\alpha} g(x))' = (1 + x)^{-\alpha} \sum_{n = 1}^{+\infty}n C_{\alpha}^{n}x^{n - 1} - \alpha(1 + x)^{-\alpha - 1} \sum_{n = 0}^{+\infty} C_{\alpha}^{n}x^{n} = \]
    \[ = (1 + x)^{-\alpha - 1}\left[\sum_{n = 1}^{+\infty}n C_{\alpha}^{n} x^{n - 1} + \sum_{n = 0}^{+\infty}n C_{\alpha}^{n}x^{n} - \alpha \sum_{n = 0}^{+\infty}C_{\alpha}^{n} x^{n}\right].\]

    В первой сумме произведем замену индекса суммирования. После приведения подобных слагаемых получим

    \[((1 + x)^{-\alpha} g(x))' = (1 + x)^{-\alpha - 1}\left[\sum_{n = 0}^{+\infty}(n + 1) C_{\alpha}^{n + 1} x^{n} - \sum_{n = 0}^{+\infty}(\alpha - n) C_{\alpha}^{n}x^{n}\right] = 0.\]

    Отсюда следует, что $(1 + x)^{-\alpha} g(x)$ постоянна на $(-1, 1)$. Из условия $g(0) = 1$ получаем, что $(1 + x)^{-\alpha}g(x) = 1$ для всех $x \in (-1, 1)$.
\end{proof}