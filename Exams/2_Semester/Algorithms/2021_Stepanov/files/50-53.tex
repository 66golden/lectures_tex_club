\setcounter{section}{49}
\section{Алгоритм $A^*$, определение функций $f, g, h$; реализация}

\textit{Применение: } нахождение минимального пути от вершины $s$ до вершины $t$ в графе с неотрицательными рёбрами.\\

$A^*$ работает по принципу алгоритма Дейкстры.

$g(v)$ -- текущее кратчайшее расстояние от $s$ до $v$ (из алгоритма Дейкстры)\\

$h(v)$ -- некоторая оценка на $dist(v, t)$. ($h(v)$ называют \textit{эвристикой})

$f(v) = g(v) + h(v)$\\

Смысл алгоритма: следующую вершину выбираем не по $g(v)$ (как в алгоритме Дейкстры), а по $f(v)$.\\

Реализация (псевдокод): \\

$q$ -- очередь, сравнение элементов по функции $f$.
$g[s] = 0$ -- задали базу для функции $g$
$gr$ -- граф, заданный списком рёбер.

\begin{lstlisting}[mathescape=true]

q.insert(s); // добавляем стартовую вершину

while(!q.empty()) {
v = q.top();
q.pop();

for (edge e : gr[v]) {
    c = g[v] + e.cost + h(to);
    if (c < f[to]) {
        g[e.to] = g[v] + e.cost;
        if (to $\in$ q) {
            q.decreaseKey(...); // уменьшить у вершины $e.to$ значение $f$ до $g[e.to] + h(e.to)$
        } else {
            q.insert(...); // добавить вершину $e.to$ со значением $f = g[e.to] + h(e.to)$
        }
    }
}

\end{lstlisting}


\setcounter{section}{50}
\section{Вырожденные случаи в алгоритме $A^*$: $h \equiv 0$, $h(v) = dist(v, t)$}

1) $h \equiv 0$. Получается, что сравнение идёт только по $g(v)$, а значит алгоритм $A^*$ выраждается в алгоритм Дейкстры.\\

2) $h(v) = dist(v, t)$ -- эвристика всегда каким-то образом знает точное расстояние от $v$ до $t$.\\ Тогда $A^*$ рассматривает почти только оптимальный путь (почти только означает, что алгоритм рассматривает не только вершины, принадлежащие кратчайшему пути, но и их соседей).

\begin{proof}
    Докажем сначала, что если $h$ -- монотонна, то значения $f$ в куче не убывают (у извлекаемых элементов).\\
    Пусть мы раскрываем вершину $v$. Рассмотрим ребро $(v, u)$.\\
    $g(u) = g(v) + cost(v, u)$\\
    С другой стороны, так как $h$ -- монотонна, то $h(v) \leq h(u) + cost(v, u)$, то есть $h(u) \geq h(v) - cost(v, u)$.
    Сложим равенство и неравенство, получим:
    $$
        g(u) + h(u) = f(u) \geq g(v) + h(v) = f(v)
    $$
    Доказано.\\
    Тогда получается, что каждая вершина раскроется не более одного раза. (Предположим противное, тогда получаем, что во второй раз мы её раскрыли со значением $f$ меньшим, чем в первый раз. Противоречие.\\
\end{proof}

Наша эвристика такая, что значения $f$ одинаковы для всех вершин, а значит каждая вершина раскроется не более одного раза. По этому утверждению и принципу работы алгоритма понятно, что он рассматривается почти только оптимальный путь.





\setcounter{section}{51}
\section{Допустимые и монотонные эвристики в алгоритме $A^*$. Примеры монотонных эвристик на разных сетках.}

\begin{definition}
     Эвристика $h(v)$ называется \textit{допустимой}, если $\forall v \: h(v) \leq dist(v, t)$.
\end{definition}

\begin{corollary}
    Если $h(v)$ -- допустимая эвристика, то $h(t) = 0$.
\end{corollary}

\begin{note}[Дополнительно]
    Если эвристика $h(v)$ -- недопустимая, то $A^*$ находит неточный ответ, на практике с помощью недопустимых эвристик можно добиться, чтобы алгоритм работал быстро, при этом ответ не сильно отличался от правильного.
\end{note}

\begin{definition}
     Эвристика $h(v)$ называется \textit{монотонной}, если:\\
     1) $h(t) = 0$
     2) $\forall (u, v) \in E \: h(u) \leq h(v) + cost(u, v)$, где $E$ -- множество рёбер (неравенство треугольника).
\end{definition}

\begin{corollary}
    Монотонная эвристика является допустимой.
\end{corollary}

\textbf{Примеры эвристик}\\

    1) Если граф представляет собой неполную сетку (то есть из каждой точки потенциально есть 4 ребра: вверх, вниз, вправо, влево), при этом некоротых рёбер может не быть (то есть не у всех точек есть ровно 4 ребра), то в качестве эвристики можно использовать Манхэттенское расстояние: $h(v) = |v.x - t.x| + |v.y - t.y|$ (каждую вершину можно задать двумя координатами на плоскости).\\
    
    2) Если в графе также можно ходить по диагонали, то в качестве эвристики можно использовать расстояние Чебышёва: $h(v) = max\{|v.x - t.x|, |v.y = t.y|\}$\\
    
    3) Если можно ходить по плоскости куда угодно, то в качестве эвристики можно использовать Евклидово расстояние: $h(v) = \sqrt{(v.x - t.x)^2 + (v.y - t.y)^2}$.\\
    

\setcounter{section}{52}
\section{Формулировка работоспособности (корректность и время работы) алгоритма $A^*$ в случае монотонной, допустимой или произвольной эвристики. Доказательство для монотонного случая.}

В случае произвольной эвристики $A^*$ может довольно быстро найти хорошее приближение.\\

В случае монотонной и допустимой эвристики алгоритм находит точный минимальный путь, при этом в случае монотонной эвристики алгоритм раскрывает каждую вершину не более 1 раза, в случае допустимой эвристики может работать довольно долго.\\

\begin{proof}[Доказательство (для монотонной эвристики)]
    Пусть $h$ -- монотонная эвристика. Алгоритм $A^*$ на каждом шаге раскрывает вершину с минимальным значением $f$, при этом.\\
    Когда алгоритм дойдет до вершины $t$, он извлечет её из кучи со значением $f(t)$, но $f(t) = g(t) + h(t) = g(t)$ ($h(t) = 0$ по определению монотонной эвристики). Осталось доказать, что $g(t)$ -- не только оценка сверху для минимального пути, но и в точности равна ей. Пусть оптимальный путь $OPT < f(t)$. Тогда бы он извлёкся из кучи раньше, так как значения $f$ у извлекаемых элементов не убывают (доказано в пункте 51). Противоречие.
\end{proof}