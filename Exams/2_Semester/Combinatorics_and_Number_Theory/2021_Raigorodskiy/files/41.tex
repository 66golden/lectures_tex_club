\setcounter{section}{40}
\section{Определение равномерной распределённой последовательности по модулю 1. Пусть последовательность $x_n$ р.р. (mod 1) и m — фиксированное целое число, не равное нулю. Докажите, что последовательность $mx_n$ также р.р. (mod 1). Верно ли, что если m — не целое, то это не верно?}
Послед-ность $x_1, x_2, \dots, x_n, \dots$ - \textbf{равномерно распределена по модулю 1}, если:
\[ \forall a, b \in [0, 1] \lim_{N \to \infty} \frac{|\{i = 1, \dots, N: \{x_i\} \in [a, b)\}|}{N} = b - a \]
\textbf{Теорема}. Если последовательность $x_n$ р.р. (mod 1) и m — фиксированное целое число, не равное нулю, то последовательность $mx_n$ также р.р. (mod 1). \par
$\blacktriangle$
$x_n$ равномерно распределено по модулю 1 по критерию Вейля равносильно тому, что:
\[
\forall h \in \mathbb{Z} \setminus \{ 0 \}: \frac{1}{N}\sum_{n=1}^N e^{2i\pi h x_n} \to 0
\]
Критерий Вейля для $mx_n$, где m - целое число, отличное от 0:
\[
\forall h \in \mathbb{Z} \setminus \{ 0 \}: \frac{1}{N}\sum_{n=1}^N e^{2i\pi h(m x_n)} =  \frac{1}{N}\sum_{n=1}^N e^{2i\pi (hm) x_n}\to 0
\]
Это верно, так как hm - подмножество целых чисел, т.е. для них это выполнялось по критерию Вейля для $x_n$. Виват!
$\blacksquare$ \par
\textbf{Контрпример к важности целостности коэффициента.} См. билет 39: $\alpha n$, где $\alpha$ - иррациональное - равномерно распределённая последовательность. Тогда можно взять последовательность $\sqrt(2)n$, она будет равномерно распределённой, и домножить её на нецелое $m=\sqrt(2)$, получить последовательность $2n$, которая не является равномерно распределённой.