\setcounter{section}{95}
\section{Теорема Лиувилля}
Пусть $\alpha -$ алгебраическое число степени d >=2. Тогда $\exists c = c(\alpha)$ неравенство $|\alpha - \frac{p}{q}| > \frac{c(\alpha)}{q^d} \forall p,q \\ \\ \blacktriangle $БОО можно считать, что $q > 0$, тогда рассмотрим два случая:
\\
1. $|\alpha - \frac{p}{q}| \geq 1 \Longrightarrow $ подойдет $c = 1$\\
2. Считаем, что $|\alpha - \frac{p}{q}| \leq 1$ \\ Заметим, что $|\alpha - \frac{p}{q}| \geq |\frac{p}{q}| - |\alpha| \Longrightarrow |\frac{p}{q}| \leq |\alpha| + 1$ \\ Рассмотрим многочлен, корнем которого является $\alpha$: $a_dx^d + ... + a_0 = \phi(x)$. Этот многочлен не имеет рациональных корней, так как d - степень $\alpha$. Из этого следует, что $\phi(\frac{p}{q}) \neq 0 \\ \phi(\frac{p}{q}) = |a_d(\frac{p}{q})^d + a_{d-1}(\frac{p}{q})^{d-1} + ... + a_0| = |\frac{a_dp^d + a_{d-1}p^{d-1}q + ... + a_0q^d}{q^d}| \geq \frac{1}{q^d}$. 
\\
\\
Теперь рассмотрим над полем комплексных чисел \\
$\phi(\frac{p}{q}) = a_d(x - \alpha)\prod\limits_{i = 2}^d(x - \alpha_i)$\\
$|\phi(\frac{p}{q})| = |a_d||\alpha - \frac{p}{q}|\prod\limits_{i = 2}^d|\alpha_i - \frac{p}{q}| \leq |a_d||\alpha - \frac{p}{q}|\prod\limits_{i = 2}^d(|\alpha_i| +| \frac{p}{q}|) \leq |a_d||\alpha - \frac{p}{q}|\prod\limits_{i = 2}^d(|\alpha_i| + |\alpha| + 1) \Longrightarrow |\alpha -\frac{p}{q}| \geq \frac{1}{q^d} \frac{1}{|a_d||\prod\limits_{i = 2}^d(|\alpha_i| + |\alpha| + 1)} = \frac{1}{q^d} * c(\alpha) \ \blacksquare$


\section{Доказательство иррациональности чисила $e$. Тождество Эрмита}
\textbf{Теорема}\\
e иррационально
\\
$\blacktriangle \ e = \sum\limits_{n = 0}^{\inf} \frac{1}{n!}$. Предположим, e = $\frac{\dots}{k}$ (т.е что е рационально), тогда $Z \ni $e*k! = $A + \frac{1}{k+1} + \frac{1}{(k+1)(k+2)} + ...$. Рассмотрим $\frac{1}{k+1} + \frac{1}{(k+1)(k+2)} + ...$. $\frac{1}{k+1} + \frac{1}{(k+1)(k+2)} + ... < \frac{1}{2} + \frac{1}{4} + .... = \frac{\frac{1}{2}}{1 - \frac{1}{2}} = 1 \Longrightarrow 0 <\frac{1}{k+1} + \frac{1}{(k+1)(k+2)} + ... < 1 \Longrightarrow \frac{1}{k+1} + \frac{1}{(k+1)(k+2)} + ... \in Q \Longrightarrow A + \frac{1}{k+1} + \frac{1}{(k+1)(k+2)} + ... \in Q$. Противоречие $\blacksquare$
\\
\textbf{Тождество Эрмита}\\
Рассмотрим производные многочлена $f(x) $  степени $\nu. \\$
Рассмотрим $\int_0^x f(t)e^{-t}dt$. Будем брать его по частям. $\int_0^x f(t)e^{-t}dt = -f(t)e^{-t}|_0^x + \int_0^x f'(t)e^{-t}dt = f(0) - f(x)e^{-x} + \int_0^x f'(t)e^{-t}dt = f(0) + f'(0) - (f(0) + f'(0))e^{-x} + \int_0^x f''(t)e^{-t}dt$. Заметим, что когда мы продифференциируем больше $\nu$ раз, интеграл обратится в ноль. Таким образом, получим: $\int_0^x f(t)e^{-t}dt = (f(0) + f'(0) + ... + f^{(\nu)}(0)) - (f(x) + f'(x) + ... + f^{(\nu)}(x))e^{-x} = F(0) - F(x)e^{-x}$, где F(x) = $f(x) + f'(x) + ... + f^{(\nu)}(x)$. То равенство, которое мы получили, называется тождством Эрмита


\section{Доказательство трансцендентности числа $e$}

Предположим, e - алгебраическое число, тогда существует многочлен $a_mx^m + ... + a_0, a_i \in Z$, корнем которого является e.\\ 
$e^x\int_0^x f(t)e^{-t}dt =  e^x F(0) - F(x), x = 0,1,2,...,m \\ \\ \sum\limits_{x = 0}^m a_x e^x \int_0^x f(t)e^{-t}dt = -\sum\limits_{x = 0}^m a_xF(x)$. ($\sum\limits_{x = 0}^m e^xa_xF(0) = F(0)(a_me^m + ... + a_0e^0 = 0)$) Воспользуемся тем, что в тождестве Эрмита мы можем использовать абсолютно любой многочлен. Давайте возьмем такой:\\
$f(t) = \frac{1}{(n-1)!}t^{n-1}((t - 1)(t-2)(t-3)...(t-m))^n$
\\
\\
Рассмотрим левую часть равенства.\\
$|\sum\limits_{x = 0}^m a_x e^x \int_0^x f(t)e^{-t}dt| \leq \sum\limits_{x = 0}^m |a_x| e^x \int_0^x|f(t)|e^{-t}dt \leq \sum\limits_{x = 0}^m |a_x| e^m \int_0^x|\frac{m^{mn + n - 1}}{(n-1)!}|e^{-t}dt = \\ \frac{e^m m^{n(m+1) - 1}}{(n-1)!} \sum\limits_{x=0}^m |a_x|\int_0^xe^{-t} = \frac{e^m m^{n(m+1) - 1}}{(n-1)!} \sum\limits_{x=0}^m |a_x|(1 - e^{-x}) < \frac{e^m m^{n(m+1) - 1}}{(n-1)!} \sum\limits_{x=0}^m |a_x| = c_0 \frac{(m^{m+1})^n}{(n-1)!} < 1, n \geq n_0$
\\
\\
Рассмотрим правую часть равенства\\
$| - \sum\limits_{x = 0}^m a_xF(x)|\\ F(x) = f(x) + f'(x) + ... + f^{(\nu)}(x), \nu = n(m + 1) - 1 \\ F(0) = 0 + 0 + ... + 0$((n-1) раз. Покуда мы не возьмем n-1 производную, многочлен f(t) будет зануляться за счет множителя $t^{n-1}) + (-1)^{mn}(m!)^n + n*A$( слагаемое после нулей получается за счет того, что мы избавились от $\frac{t^{n-1}}{(n-1)!}$, а следующее - за счет того, что мы берем производную и по второму множителю $((t - 1)(t-2)(t-3)...(t-m))^n$, за счет чего возникает делимость на n) \\
Теперь рассмотрим F(x)\\
$\forall x = 1,2..m  \ f(x) + f'(x) + ... + f^{(\nu)}(x)$ имеет первые n нулей по той же причине (пока мы не возьмем n производных, множитель $((t - 1)(t-2)(t-3)...(t-m))^n$ будет обнулять функцию. А после производные будут делиться на n). Таким образом, $| - \sum\limits_{x = 0}^m a_xF(x)| = |\sum\limits_{x = 0}^m a_xF(x)| = |(-1)^{mn}(m!)^n*a_0 + nA + nB|, \ \exists n: n > |a_0|, n>n_0, (n, m!) = 1 \Longrightarrow |(-1)^{mn}(m!)^n*a_0 + nA + nB| \neq 0 \Longrightarrow |(-1)^{mn}(m!)^n*a_0 + nA + nB| \geq 1 $ \\ \\
Таким образом, $\exists n: \forall N \geq n: \ $ Правая часть $\geq 1$, левая часть $<1 \Longrightarrow$  равенство не может быть достигнуто. Противоречие $\Longrightarrow e -$ трансцендентно$\blacksquare$