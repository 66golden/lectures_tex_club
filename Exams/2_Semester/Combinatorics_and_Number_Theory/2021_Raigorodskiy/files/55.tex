\setcounter{section}{54}
\section{Матрицы Адамара. (Вторая) конструкция Пэли с квадратичными вычетами при $n = 2p + 2,p = 4m + 1$.}
\par \textbf{Утверждение (свойства кронекеровского произведения):} \begin{enumerate}
    \item $(A \otimes B)^T=A^T \otimes B^T$
    \item $(A \otimes B)(C \otimes D)=(AC) \otimes (BD)$
\end{enumerate}
\par \begin{itemize}
    \item[$\blacktriangle$ 1.] $C=A \otimes B, D = C^T$. Тогда $d_{pb+q, kb+l}=c_{kb+l, pb+q}=a_{kp} b_{lq}=(A)^T_{pk} (B)^T_{ql} \Rightarrow$ по определению $D = A^T \otimes B^T$
    \item[2.] Покажем, что это правда для случаев когда размеры $A , C$ и $B, D$ попарно равны и все матрицы квадратные. Тогда можно просто рассмотреть их произведение как блочных матриц.
    $$\left(
\begin{array}{ccc}
a_{11}B & \ldots & a_{1n}B\\
\vdots & \ddots & \vdots\\
a_{n1}B & \ldots & a_{nn}B
\end{array}
\right) \left(
\begin{array}{ccc}
c_{11}D & \ldots & c_{1n}D\\
\vdots & \ddots & \vdots\\
c_{n1}D & \ldots & c_{nn}D
\end{array}
\right)=\left(
\begin{array}{ccc}
R_{11} & \ldots & R_{1n}\\
\vdots & \ddots & \vdots\\
R_{n1} & \ldots & R_{nn}
\end{array}
\right)$$
\par где $R_{ij}=\sum_{k=1}^n (a_{ik} B)(c_{kj} D)$ (утверждение из Википедии) = $(\sum_{k=1}^n a_{ik} c_{kj})BD=(AC)_{ij} BD$ $\Rightarrow$ по определению получили $(AC) \otimes (BD) \: \blacksquare$
\end{itemize}
\par \textbf{Лемма:} \begin{enumerate}
    \item Если $p \equiv 1 (\md 4)$, то $Q_p$ - симметрична
    \item $Q_pQ_p^T=pE_p-I_p$, где $I_p$ - матрица состоящая полностью из единиц
\end{enumerate} 
\par \begin{itemize}
    \item[$\blacktriangle$ 1.] $$\left(\frac{-1}{p}\right)=(-1)^\frac{p-1}{2}=(-1)^\frac{4k}{2}=1 \Rightarrow \left(\frac{i-j}{p}\right)=\left(\frac{-1}{p}\right) \left(\frac{j-i}{p}\right)= \left(\frac{j-i}{p}\right) \Rightarrow Q_{ij}=Q_{ji}$$ 
    \item[2.] В первой конструкции Пэли мы показали, что скалярное произведение различных строк $Q_p$ равно $-1$. Скалярное произведение строк $i, j$ - это элемент на позиции $i,j$ в $Q_pQ_p^T$. Очевидно, что на диагонали будут стоять числа $p-1$, так как в каждой строке ровно $p-1$ ненулевой элемент, каждый из которых равен $\pm 1$. Таким образом, получается, что $Q_pQ_p^T=pE_p-I_p \: \blacksquare$
\end{itemize}
\par \textbf{II конструкция Пэли:} Пусть $p \equiv 1 (\md 4)$. Если в матрице
$$A=\left(
\begin{array}{cc}
0 & e^T\\
e &  Q_p
\end{array}
\right)$$
где $e$ — столбец из единиц размера $p$, $Q_p$ - матрица Якобсталя порядка $p$, заменить $0$ на матрицу 
$M_0=\left(
\begin{array}{cc}
1 & -1\\
-1 &  -1
\end{array}
\right)$, а$\; \pm 1$ на матрицу $\pm \left(
\begin{array}{cc}
1 & 1\\
1 & -1
\end{array}
\right)=\pm M_1$, то
получится матрица Адамара порядка $2p + 2$.
\par $\blacktriangle$ Найдем $AA^T$ (пригодится нам в будущем). В левом верхнем углу очевидно будет стоять $p$, так как просто перемножили столбец из единиц на строку. Остальные элементы первой строки/столбца будут нулями, так как они равны сумме всех символов Лежандра по $p$. Просто перемножая матрицы заметим, что в оставшемся пространстве у нас получится матрица $I_p+Q_pQ_p^T=I_p+pE_p-I_p=pE_p$ (по пункту 2 леммы). Таким образом, $AA^T=pE_{p+1}$
\par Пусть $H$ - матрица которая получилась после замен. Тогда так как нули находятся только на главной диагонали $$H=A \otimes M_1+E_{p+1} \otimes M_0$$
$$HH^T=(A \otimes M_1 + E_{p+1} \otimes M_0)(A \otimes M_1 + E_{p+1} \otimes M_0)^T=(A \otimes M_1 + E_{p+1} \otimes M_0)(A^T \otimes M_1^T + E_{p+1} \otimes M_0^T)$$
\par Заметим, что $M_1^T=M_1, M_0^T=M_0, M_1M_0=-M_0M_1$
$$HH^T=(A \otimes M_1)(A^T \otimes M_1)+(E_{p+1} \otimes M_0)(A^T \otimes M_1) + (A \otimes M_1)(E_{p+1} \otimes M_0) + (E_{p+1} \otimes M_0)(E_{p+1} \otimes M_0)=$$
$$=(AA^T)\otimes M_1^2 + A^T \otimes (M_0M_1) - A \otimes (M_0M_1) + E_{p+1} \otimes M_0^2$$
\par $A=A^T$ (по пункту 1 леммы), $AA^T=pE_{p+1}$. Матрицы $M_0, M_1$ являются матрицами Адамара $\Rightarrow M_i^2=M_i^TM_i=2E_2$
$$HH^T= pE_{p+1} \otimes 2E_2 + E_{p+1} \otimes 2E_2=2pE_{2p+2} + 2E_{2p+2}=(2p+2)E_{2p+2} \: \blacksquare$$