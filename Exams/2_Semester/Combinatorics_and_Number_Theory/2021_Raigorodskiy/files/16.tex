\setcounter{section}{15}
\section{Существование матриц Адамара при $n = 1$ и $2$. Необходимость делимости на $4$ при $n > 3$. Гипотеза Адамара. Комбинаторная переформулировка гипотезы (через систему подмножеств мощности $\frac{n}{2}$ в множестве из $n - 1$ элемента). Утверждение о плотности матриц Адамара в натуральном ряде (б/д).}
\begin{enumerate}
    \item $n=1$: $(1)$
    \item $n=2$: $\left(
\begin{array}{cc}
1 & 1\\
1 & -1
\end{array}
\right)$
\end{enumerate}
\par \textbf{Утверждение (задача 20.5):} Если $n>2$ - порядок матрицы Адамара, то $n\: \vdots \:4$.
\par $\blacktriangle$ Так как любую матрицу Адамара можно привести к каноническому виду, рассмотрим нормальную форму матрицы Адамара порядка $n$. Из того, что первая строка состоит только из единиц следует то, что во всех остальных строках будет $\frac{n}{2}$ единиц и $\frac{n}{2}$ минус единиц (иначе не будет ортагональности с первой строкой).
\par Рассмотрим две произвольные строчки. Пусть у них на одних и тех же местах стоят $x$ единиц. Тогда мест где в первой строке стоит $-1$, а во второй $1$: $\frac{n}{2}-x$ (столько же когда в первой строке $1$, а во второй $-1$). Получается, что мест где в обоих строках стоит $-1$: $n-(\frac{n}{2}-x)-(\frac{n}{2}-x)-x=x$. Распишем скалярное произведение этих строк: $1 \cdot 1 \cdot x + (-1) \cdot (-1) \cdot x + 1 \cdot (-1) \cdot (\frac{n}{2}-x) + (-1) \cdot 1 \cdot (\frac{n}{2}-x)=4x-n=0 \Rightarrow x = \frac{n}{4} \Rightarrow n\: \vdots \: 4 \: \blacksquare$
\par \textbf{Гипотеза Адамара:} Матрица Адамара существует для любого числа вида $4k$.
\par \textbf{Комбинаторная переформулировка:} В множестве мощности $n$ существует $n-1$ подмножество мощности $\frac{n}{2}$, такие что каждые $2$ подмножества имеют ровно $\frac{n}{4}$ общих элементов.
\par \textbf{Утверждение (о плотности в натуральном ряде):} $\forall n$ на отрезке  $[n;n+O(n^{0.525})]$ есть порядки матриц Адамара. 