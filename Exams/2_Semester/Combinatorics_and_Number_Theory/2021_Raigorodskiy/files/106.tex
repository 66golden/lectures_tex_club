\setcounter{section}{105}
\section{Алгоритм AKS. Определение и неравенства, связывающие $p, r, \log_2 n, t$, группы $G, \mathcal{G}$, многочлена $h(x)$ (б/д). Неравенство $C_{t+l}^{t-1}>n^{\sqrt{t}}$.}
\par \textbf{Неравенства:} $p>r>\log_2^2n$, $\varphi(r)\geq|G|=t>\log_2^2 n$, $\deg h(x) > \ord_r p > 1$
\par \textbf{Утверждение 1:} \begin{enumerate}
    \item Если $a>b$, то $C^k_a>C^k_b \: \forall k$
    \item Если $\frac{n}{2}>a>b$, то $C^a_n>C^b_n$
\end{enumerate}
\par  \begin{itemize}
    \item[$\blacktriangle$ 1.]  В переходе с неравенством добавляем $b-a<0$ к каждому множителю$$C^k_a=\frac{a!}{k! \: (a-k)!}=\frac{(a-k+1)\cdot \ldots \cdot a}{k!}>\frac{(b-k+1) \cdot \ldots \cdot b}{k!}=\frac{b!}{k! \: (b-k)!}=C^k_b$$
    \item[2.] В переходе с неравенством добавляем $a-b>0$ к каждому множителю $$C^a_n=\frac{n!}{a! \: (n-a)!}=\frac{n!}{a! \: b! \: (b+1) \cdot \ldots \cdot (n-a)}>\frac{n!}{a! \: b! \: (a+1) \cdot \ldots \cdot (n-b)}=\frac{n!}{b! \: (n-b)!}$$
\end{itemize}
\par \textbf{Утверждение 2:} $C_{2x+1}^x \geq 2^{x+1}$
\par \begin{itemize}
    \item[$\blacktriangle$ 1.] База: $x=2$ (при $x=1$ неверно, но нам важно на больших) $C_5^2=10 \geq 2^3=8$
    \item[2.] Переход: пусть верно для $x$. Докажем для $x+1$ $$C^{x+1}_{2x+3}=\frac{(2x+3)!}{(x+1)! \: (x+2)!}=\frac{(2x+1)! \: (2x+2)(2x+3)}{(x+1)! \: x! \: (x+1)(x+2)}=C_{2x+1}^x \cdot \frac{(2x+2)(2x+3)}{(x+1)(x+2)}=$$ $$=2C^x_{2x+1} \frac{2x+3}{x+2} > 2C^x_{2x+1} \geq 2^{x+2} \: \blacksquare$$
\end{itemize}
\par \textbf{Лемма 3:} $C_{t+l}^{t-1}>n^{\sqrt{t}}$
\par $\blacktriangle$ Так как $t>\log_2^2 n \Rightarrow t > \sqrt{t} \log_2 n \Rightarrow t \geq [\sqrt{t} \log_2 n]+1$. $l=\sqrt{\varphi(r)}\log_2 n \geq \sqrt{t} \log_2 n$. 
$$C_{t+l}^{t-1}\geq C_{[\sqrt{t}\log_2 n]+1+l}^{[\sqrt{t}\log_2 n]}\geq C_{2[\sqrt{t}\log_2 n]+1}^{[\sqrt{t}\log_2 n]} \geq 2^{[\sqrt{t}\log_2 n]+1} > 2^{\sqrt{t}\log_2 n}=n^{\sqrt{t}} \: \blacksquare$$
\par Так как верны все леммы 1-3 из билетов 104-106, то $n=p^k$, но если $k > 1$, то мы бы остановились еще на шаге 1 нашего алгоритма $\Rightarrow n=p \Rightarrow$ последний шаг алгоритма корректен 