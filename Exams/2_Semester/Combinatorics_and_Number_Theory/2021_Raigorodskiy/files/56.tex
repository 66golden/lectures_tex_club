\setcounter{section}{55}
\section{$(n, M, d)$-коды. Граница Плоткина}
\begin{definition}
    \textit{$(n, M, d)$-кодом} называется код, в котором все кодовые слова имеют длину $n$, $d$ -- минимальное расстояние между словами (в смысле расстояния Хэмминга), $M$ -- количество кодовых слов.
\end{definition}

\begin{theorem}[Граница Плоткина]
    Пусть задан $(n, M, d)$-код. Если $2d > n$, то $M \leq \left[ \frac{2d}{2d - n} \right]$
    \begin{proof}[$\blacktriangle$]
        Пусть $\Vec{a_1}, \Vec{a_2}, ..., \Vec{a_M}$ -- все кодовые слова. Запишем их в матрицу следующим образом (каждый вектор представляется в виде строчки):
        $$\begin{pmatrix}
            \leftarrow \Vec{a_1} \rightarrow \\
            \leftarrow \Vec{a_2} \rightarrow \\
            ...\\
            \leftarrow \Vec{a_M} \rightarrow \\
        \end{pmatrix}$$
        Полученная матрица имеет размер $M \times n$.\\
        Рассмотрим сумму ($a_{i_j}$ -- элемент матрицы)
        $$
        S = \sum \limits_{k=1}^{n} \sum \limits_{1 \leq i < j \leq M} I_{ \{ a_{i_{k}} \neq a_{j_{k}} \} }, \text{где}\ I_{ \{ a_{i_{k}} \neq a_{j_{k}} \} } = 
        \begin{cases}
            1, a_{i_{k}} \neq a_{j_{k}}\\
            0, a_{i_{k}} = a_{j_{k}}
        \end{cases}
        $$
        $S$ можно рассматривать следующим образом: фиксируем столбец $k$ и смотрим количество несовпадающих пар в этом столбце. Пусть $x$ -- число единиц в этом столбце, тогда $M - x$ -- число нулей в этом же столбце. Тогда несовпадающих пар в этом столбце ровно $x(M-x) = Mx - x^2$. Максимум этого значения достигается при $x = \frac{M}{2}$, то есть $x(M-x) \leq \frac{M^2}{4}$. Получаем, что $S \leq n \cdot \frac{M^2}{4}$.\\
        Теперь рассмотрим $S$ с другой стороны, переставив знаки суммирования:\\
        $$
        S = \sum \limits_{1 \leq i < j \leq M} \sum \limits_{k=1}^{n} I_{ \{ a_{i_{k}} \neq a_{j_{k}} \} }
        $$
        Теперь $S$ можно интерпретировать следующим образом: зафиксируем строки $i$ и $j$ (а значит, и соответствующие кодовые слова $\Vec{a_i}$ и $\Vec{a_j}$). Тогда внутренняя сумма $\sum \limits_{k=1}^{n} I_{ \{ a_{i_{k}} \neq a_{j_{k}} \} }$ -- это в точности расстояние Хэмминга $d(\Vec{a_i}, \Vec{a_j})$. По определению $(n, M, d)$-кода $d(\Vec{a_i}, \Vec{a_j}) \geq d$, то есть $S \geq \frac{M(M-1)}{2} \cdot d$, где ($\frac{M(M-1)}{2}$ -- количество пар $(i, j)$).\\
        $$
            \frac{M(M-1)}{2} \cdot d \leq S \leq n \cdot \frac{M^2}{4}
        $$
        $$
            (M - 1)d \leq n \cdot \frac{M}{2}
        $$
        $$
            dM - \frac{nM}{2} \leq d
        $$
        $$
            M(d - \frac{n}{2}) \leq d
        $$
        $$
            M(2d - n) \leq 2d
        $$
        Так как $2d - n$ -- положительное, то можно разделить на него
        $$
            M \leq \frac{2d}{2d - n}
        $$
        Так как $M$ -- натуральное, получаем:
        $$
            M \leq \left[ \frac{2d}{2d - n} \right]
        $$
    \end{proof}
\end{theorem}