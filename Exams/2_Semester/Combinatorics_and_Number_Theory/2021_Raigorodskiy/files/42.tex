\setcounter{section}{41}
\section{Описание алгоритма AKS (6 шагов). Лемма об оценке $r$ (б/д). Оценка сложности алгоритма. Тождество $(X + a)^p = X^p + a (\md p)$.}
\par \textbf{Алгоритм проверки $n$ на простоту:} Agarwal, Kayal, Saxena (AKS) \begin{enumerate}
    \item $n=a^b, b \geq 2 \Rightarrow n$ составное
    \item Ищем наименьшее $r$, такое что $\ord_r n > \log_2^2 n$
    \item Если хотя бы для одного числа $a$ из диапазона $1\ldots r$ выполнено $1 < (a, n) < n \Rightarrow n$ составное ($(a,n)$ := НОД($a, n$))
    \item Если $n \leq r$, то $n$ простое
    \item Если хотя бы для одного числа $a$ в диапазоне $1\ldots l=\sqrt{\varphi(r)} \cdot \log_2 n$ выполнено $(x+a)^n \neq x^n+a \: (\md \: x^r-1, n) \Rightarrow n$ составное
    \item $n$ простое
\end{enumerate}
\par \textbf{Лемма:} $r \leq \max\{3, \lceil \log_2^5 n \rceil\}$
\par \textbf{Сложность:} \begin{enumerate}
    \item $n=a^b \Rightarrow b \leq \log_2 n \Rightarrow$ можно перебрать бинпоиском за $\poly(\log_2 n)$
    \item Из леммы следует, что шаг 2 можно сделать перебором за $\poly(\log_2 n)$
    \item Перебираем числа меньше $r$ и ищем НОД (за логарифм) $\Rightarrow$ этот шаг выполняется за $\poly(\log_2 n)$
    \item $O(1)$
    \item Всего $\poly(\log_2 n)$ итераций. На каждой делаем бинарное возведение в степень $\quad$ ($\poly(\log_2 n)$), как только превышаем $r$ делим на многочлен $x^r-1$ ($\poly(\log n)$, так как степень делимого $\leq 2r$, то есть у него $\poly(\log_2 n)$ коэффициентов)
\end{enumerate}
\par \textbf{Утверждение:} $(x+a)^p = x^p+a \: (\md \: p)$
\par $\blacktriangle$ $$(x+a)^p=x^p+a^p+\sum_{i=1}^{p-1} C_p^i x^{p-i}a^i$$
\par $a^p=a (\md \: p)$ (малая теорема Ферма), $C_p^i=0 \: (\md \: p)$ (доказывалось в прошлом семестре) $\Rightarrow (x+a)^p=x^p + a \: (\md \: p) \: \blacksquare$