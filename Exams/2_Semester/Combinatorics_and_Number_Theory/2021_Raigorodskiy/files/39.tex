\setcounter{section}{38}
\section{Тригонометрические суммы. Критерий Вейля для р.р. (mod 1) (формулировка). Последовательность $\alpha n$ при иррациональном $\alpha$ является р.р. (mod 1). Что происходит при рациональном $\alpha$?}

\textbf{Тригонометрическая сумма} - сумма вида $\sum_{k=1}^N e^{2i\pi kx}$ \par
\textbf{Критерий Вейля}. $x_n$ р.р. mod 1 тогда и только тогда, когда:
\[
\forall h \in \mathbb{Z} \setminus \{ 0 \}: \frac{1}{N}\sum_{n=1}^N e^{2i\pi h x_n} \to 0
\]
$x_n = \alpha n$ при $\alpha \in \mathbb{Q}$ принимает ограниченное количество значений $\Rightarrow$ быть р.р. mod 1 не может. \par
$\alpha \notin \mathbb{Q}$. Применим критерий Вейля:
\[
\forall h \in \mathbb{Z} \setminus \{ 0 \}: \frac{1}{N}\sum_{n=1}^N e^{2i\pi h \alpha n} = \frac{1}{N} e^{2i\pi h \alpha} \frac{e^{2i\pi h \alpha N} - 1}{e^{2i\pi h \alpha} - 1}
\]
Знаменатель не равен нулю в силу иррациональности $\alpha$; по формуле Эйлера ($e^{ix} = cos(x) + isin(x)$) числитель не превышает 2, следовательно, это значение стремится к нулю; критерий Вейля выполняется $\Rightarrow$ последовательность равномерно распределена по модулю 1.