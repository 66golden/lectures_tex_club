\setcounter{section}{39}
\section{Определение равномерной распределённой последовательности по модулю 1. Являются ли р.р. (mod 1) последовательности а) $1, \frac{1}{2}, \frac{1}{3}, \frac{2}{3}, \frac{1}{4}, \frac{2}{4}, \frac{3}{4}, \dots$ б) $\frac{1}{2}, \frac{1}{4}, \frac{3}{4}, \frac{1}{8}, \frac{3}{8}, \frac{5}{8}, \frac{7}{8}, \dots$.}

Послед-ность $x_1, x_2, \dots, x_n, \dots$ - \textbf{равномерно распределена по модулю 1}, если:
\[ \forall a, b \in [0, 1] \lim_{N \to \infty} \frac{|\{i = 1, \dots, N: \{x_i\} \in [a, b)\}|}{N} = b - a \]
а) \textbf{Равномерно распределена mod 1}. \par
$\blacktriangle$
Разбиваем на "блоки" по знаменателям. Тогда количество чисел из N-ого блока, которые попадают в отрезок (b-a) можно оценить как $(b-a)N+C$, где $|C|<2$. Доля чисел от 1 до M, где $\frac{N(N-1)}{2} <= M < \frac{N(N+1)}{2}$ (то есть M находится в блоке N+1), попадающих в этот отрезок - как $\frac{1}{M} (\sum_n [(b-a)n + C] + D)$, где D - это "остаток" из дробей со знаменателем N+1. Так как С, D - это линейные функции от N, то при делении на M они будут стремится к 0, и останется в точности (b-a).
$\blacksquare$ \\
\\
б) \textbf{Не равномерно распределена mod 1}. \par
$\blacktriangle$
Разобъём её на "блоки" по знаменателям. Рассмотрим отрезок $[0; \frac{1}{2}]$. Если смотреть на концы блоков, то там ровно половина попадает в первый отрезок. А если посмотреть на все числа до середины N-ого блока, то доля чисел, попадающих на отрезок $[0; \frac{1}{2}]$ будет $\frac{2}{3}$ (по индукции), что противоречит существованию предела (а, значит, и определению равномерной распределённости по модулю 1).
$\blacksquare$