\setcounter{section}{74}
\section{Критический определитель решётки. Переформулировка теоремы Минковского через критический определитель. Теорема Минковского–Главки и история ее улучшений (б/д). Многомерный октаэдр, его объём.}
\textbf{Критический определитель $\Omega \subset \mathbb{R}^n$}; $\Delta(G)$: $\Delta(G) = \inf\{\det \Lambda:(\Omega \cap \Lambda)\backslash \{0\} = \varnothing\}$ \par
\textbf{Переформулировка теоремы Минковского через критический определитель}. Если $\Omega$ - выпукла и центрально симметрична относительно 0, то $\frac{Vol \Omega}{\Delta(\Omega)} \leqslant 2^n$. \par
Проблема: хочется оценить эту дробь снизу. Но мало ли чего хочется... \par
\textbf{Теорема Минковского-Главки, 1945.} $\forall \Omega \frac{Vol \Omega}{\Delta(\Omega)} \geqslant 1$

\textbf{Теорема Шмидта, Роджерса, 50-60-e гг.} $\geqslant cn$. \par
$O^n = {x : |x_1| + . . . + |x_n| \leqslant 1}$ — \textbf{n-мерный октаэдр (кросс-политоп, ортаэдр} - линейная оболочка векторов (1,0,..0), (0,1,..0), ... \par
\textbf{Объём n-мерного октаэдра}: $Vol(O^n) = \frac{2^n}{n!}$