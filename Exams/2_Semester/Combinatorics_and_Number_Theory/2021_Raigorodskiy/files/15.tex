\setcounter{section}{14}
\section{Матрицы Адамара. Определение. Равносильность попарной ортогональности строчек и попарной ортогональности столбцов. Канонический вид (нормальная форма). Достижение верхней оценки в неравенстве Адамара.}
\par \textbf{Определение:} \textit{Матрицей Адамара} порядка $n$ называется матрица $H_n$ размера $n \times n$ такая, что все
её элементы равны $\pm 1$ и выполнено следующее свойство:
$H_n H_n^T = nE_n$.
Можно переформулировать так: матрица Адамара — это матрица из $\pm 1$, у которой все строки попарно ортогональны.
\par \textbf{Утверждение (задача 20.1):} Ортогональность строк матрицы Адамара равносильна ортогональности ее столбцов
\par $\blacktriangle$ $$H_n H_n^T = nE_n \Rightarrow \frac{H_n}{\sqrt{n}} \frac{H_n^T}{\sqrt{n}} = E_n$$
\par Получили, что $\frac{H_n}{\sqrt{n}}$ и $\frac{H_n^T}{\sqrt{n}}$ - взаимно-обратные $\Rightarrow$ их можно переставить местами. Условие $H_n^T H_n = nE_n$ равносильно ортогональности столбцов (в обратную сторону равносильность доказывается аналогично). $\blacksquare$
\par \textbf{Замечание:} строки/столбцы матрицы Адамара можно менять местами, умножать на -1, при этом она останется матрицей Адамара.
\par \textbf{Определение:} Матрицы Адамара, получаемые друг их друга многократным применением таких операций называются \textit{эквивалентными}. \textit{Каноническим видом (нормальной формой)} матрицы Адамара называется эквивалентная ей матрица, в которой первая строка и первый столбец состоят только из 1.
\par \textbf{Неравенство Адамара:} Если у действительной матрицы $A$ размера $n \times n$ все элементы по модулю
меньше 1, то $|\det A| \leq n^{n/2}$.
\par \textbf{Утверждение (задача 20.2):} Для матриц Адамара в этом неравенстве достигается верхняя оценка, то есть $| \det A| = n^{n/2}$
\par $$\blacktriangle \: \det(n E_n)=\det(H_n  H_n^T)=\det H \det H^T=(\det H)^2 \Rightarrow (\det H_n)^2=n^n \Rightarrow \det H_n = n^{n/2} \: \blacksquare$$