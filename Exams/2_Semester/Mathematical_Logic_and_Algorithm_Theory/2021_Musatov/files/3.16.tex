\subsection{Существование непересекающихся перечислимых множеств, не отделимых никаким разрешимым.}

Два непересекающихся множества X и Y \emph{отделяются} множеством C, если множество C содержит одно из них
и не пересекается с другим.

\textbf{Теорема.} Существуют два непересекающихся перечислимых множества X и Y , которые не отделяются никаким разрешимым множеством.

$\blacktriangle$
В самом деле, пусть d — вычислимая функция, принимающая только значения 0 и 1 и не имеющая всюду определённого вычислимого продолжения (возьмем например $d(x)=U(n,n)$ и заменим все четные значения на 1, а нечетные - на 0). Пусть $X = \{x | d(x) = 1\}$ и $Y = \{x | d(x) = 0\}$.
Легко видеть, что множества X и Y перечислимы (полухарактеристическая функция: запускаем вычисление функции $d$, если вывелось нужное нам число, то выводим 1, иначе зацикливаемся). Пусть они отделяются разрешимым множеством C; будем считать, что C содержит X и не пересекается c Y (если наоборот, перейдём к дополнению). Тогда характеристическая функция множества C (равная 1 внутри C и 0 вне него) продолжает d, что противоречит её выбору $\Rightarrow$ предположение неверно, эти два множества не отделимы.
$\blacksquare$ \\
\\
P.S. Этот результат усиливает утверждение о существовании перечислимого неразрешимого множества (если два множества не отделимы разрешимыми множествами, то ни одно из них не
разрешимо).