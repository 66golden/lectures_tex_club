\subsection{Теорема Банаха-Тарского: равносоставленность сферы и пары сфер}

\begin{definition}
    Назовём две фигуры \textit{равносоставленными}, если каждую из них можно представить в виде объединения конечного числа попарно непересекающихся множеств, так, что количество подмножеств в обоих разбиениях одно и то же, а подмножества с одинаковыми номерами переводятся одно в другое движением (изометрией).
\end{definition}

\begin{proposition}
    Равносоставленность -- отношение эквиваленции.\\
    Рефлексивность и симметричность очевидны.\\
    Транзитивность: рассмотрим множества, получающиеся при попарных пересечениях множеств разбиений $B$, из которых получаются $A$ и $C$.
\end{proposition}

Для доказательства теоремы потребуется определение из теории групп\\

\begin{definition}
    \textit{Свободная группа} -- группа $G$, для которой существует подмножество $S \subset G$ такое, что каждый элемент $G$ записывается единственным образом как произведение конечного числа элементов $S$ и их обратных. (Единственность понимается с точностью до тривиальных комбинаций наподобие $st = su^{-1}ut$).
\end{definition}

\begin{theorem}[Банаха-Тарского]
    Шар и два шара такого же радиуса равносоставлены.
    
    \begin{proof}
        
        Сначала покажем, что сфера и две таких же сферы равносоставлены.\\
        
        I. Полезное замечание для свободных групп преобразований на сфере\\
        
        Расмотрим $\varphi$ и $\psi$ -- преобразования на сфере, которые являются поворотами вокруг оси, проходящий через центр сферы, удовлетворяющие следующим свойствам:\\
        
        1) Никакая степень $\varphi$ и $\psi$ не будет тождественным преобразованием\\
        2) один из поворотов не переводит ось в себя\\
        
        Тогда все повороты $id, \varphi, \psi, \varphi^{-1}, \psi^{-1}, \varphi \psi, \psi \varphi, \varphi^2, ...$ будут различными.\\
        
        Значит, эти преобразования задают свободную группу движений сферы.\\
        
        У каждой точки сферы возникает орбита -- образы под действием соответствующих поворотов. У большинства точек орбита полная: для каждого элемента группы будет один образ, все эти образы разные.\\
        
        Однако у некоторых точек орбита будет неполная: если под действием $\varphi$ точка попала на ось $\psi$ (или она там была с самого начала, то любое количество дальнейших поворотов $\psi$ не изменят результат. Таких исключительных точек -- счётное количество.\\
        
        Идея состоит в следующем: можно из каждой орбиты взять по одной точке, тогда образы такого множества под действием всех поворотов покроют всю сферу (с точностью до неполных орбит -- их точки будут покрыты несколько раз).\\
        
        II. Доказательство утверждения про сферы\\
        
        Для доказательства теоремы Банаха-Тарского возьмём следующие повороты:\\
        
        $\varphi$ -- поворот на $180$, поэтому $\varphi^2 = 1$.\\
        
        $\psi$ -- поворот на $120$, поэтому $\psi^3 = 1$.\\
        
        Оси поворотов расположены в общем положении, так что других соотношений не возникает.\\
        
        Такую орбиту можно разбить на 3 части $A, B, C$:\\
        $A$ равносоставлено с $B \cup C$, то есть $\varphi(A) = b \cup C$)\\
        $A$ равносоставлено с $B$, то есть $\psi(A) = B$\\
        $A$ равносоставлено с $C$, то есть $\psi^2(A) = C$\\
        
        На каждой "нормальной" орбите" (где никакие дополнительные точки не склеиваются) выберем одну точку -- по аксиоме выбора. Покрасим её в цвет $A$, остальные элементы -- согласно вышеуказанным правилам.\\
        
        Вся сфера разбилась на 4 множества: $A, B, C$ и $Q_1$ ($Q_1$ -- множество орбит со склейками).\\
        
        $Q_1$ -- счётное множество, так что его можно повернуть, получив $Q_2$, так что $Q_1 \cap Q_2 = \emptyset, Q_2 \subset A \cup B \cup C$.\\
        
        Пусть $Q_2'$ -- множество, равносоставленное с $Q_2$, лежащее полностью в $C$. Это может быть сделано следующим образом:\\
        Ту часть $Q_2$, которая лежит в $B \cup C$, отправим в $A$.\\
        Ту часть, которая изначально лежала в $A$, отправим в $B$.\\
        Ту часть, которая теперь лежит в $A$, отправим в $C$.\\
        Получим множество целиком в $B \cup C$. отправим его в $A$, и, наконец, в $C$.\\
        
        $$
            U = A \cup B \cup C \cup Q_1 = (A \cup Q_1) \cup (B \cup Q_2) \cup (C \setminus  Q_2)
        $$
        
        $$
            A \cup Q_1 \sim B \cup C \cup Q_1 \sim A \cup C \cup Q_1 \sim B \cup C \cup A \cup Q_1 = U
        $$
        Аналогично, $B \cup Q_2 \sim U$\\
        Получаем, что $U \sim U \cup U \cup (C \setminus Q_2)$\\
        По теореме $U \sim U \cup U$\\
        
        Утверждение про сферы доказано.\\
        
        III. Доказательство утверждения про шары\\
        
        
        Из этого будет следовать и теорема про два шара: шар без центра равносоставлен с двумя шарами без центра.\\
        
        По транзитивности он равносоставлен и с тремя шарами без центра. $А$ -- такое множество с двумя шарами с центром  и одним шаром без центра и еще одной точки.\\
        
        Тогда $A$ -- один шар, $B$ -- два шара, $C$ -- два шара и ещё шар без двух точек.\\
        
        Поскольку $A$ и $C$ равносоставлены, то в силу утверждения $A$ и $B$ равносоставлены.
    \end{proof}
    
\end{theorem}