\subsection{Существование главной универсальной вычислимой функции.}
\par \textbf{Теорема 1:} Существует вычислимая функция двух аргументов, являющаяся универсальной функцией для класса вычислимых функций одного аргумента.
\par $\blacktriangle$ Запишем все программы, вычисляющие функции одного аргумента, в вычислимую последовательность $p_0, p_1, \ldots$ (например, в порядке возрастания их длины). Положим $U(i, x)$ равным результату работы $i$-ой программы на входе $x$. Тогда функция $U$ и будет искомой вычислимой универсальной функцией. $\blacksquare$ 
\par \textbf{Теорема 2:} Существует главная универсальная функция.
\par $\blacktriangle$ Заметим сначала, что существует вычислимая функция трёх аргументов, универсальная для класса вычислимых функций двух аргументов, то есть такая функция $T$, что при фиксации первого аргумента среди функций $T_n(u, v) = T(n, u, v)$ встречаются все вычислимые функции двух аргументов.
\par Такую функцию можно построить так. Фиксируем некоторую вычислимую нумерацию пар, то есть вычислимое взаимно однозначное соответствие $(u, v) \leftrightarrow [u, v]$ между $\mathbb{N} \times \mathbb{N}$ и $\mathbb{N}$; число $[u, v]$, соответствующее паре $(u, v)$, мы будем называть номером этой пары.
\par Если теперь $R$ — двуместная вычислимая универсальная функция
для вычислимых одноместных функций (существует по теореме 1), то вычислимая функция $T$,
определённая формулой $T(n, u, v) = R(n, [u, v])$, будет универсальной для вычислимых двуместных функций. В самом деле, пусть $F$ —
произвольная вычислимая функция двух аргументов. Рассмотрим
вычислимую одноместную функцию $f$, определённую соотношением $f([u, v]) = F(u, v)$. Поскольку $R$ универсальна, найдётся число $n$,
для которого $R(n, x) = f(x)$ при всех $x$. Для этого $n$ выполнены равенства $T(n, u, v) = R(n, [u, v]) = f([u, v]) = F(u, v)$. Итак, универсальная функция
трёх аргументов построена.
\par Теперь используем её для определения главной универсальной
функции $U$ двух аргументов. Положим $U([n, u], v) = T(n, u, v)$ и проверим, что функция U будет главной. Для любой вычислимой функции $V$ двух аргументов можно найти такое $n$, что $V (u, v) = T(n, u, v)$ (так как $T$ - универсальна) для всех $u$ и $v$. Тогда $V (u, v) = U([n, u], v)$ для всех $u$ и $v$ и потому функция $s$, определённая формулой $s(u) = [n, u]$, удовлетворяет требованиям из
определения главной универсальной функции. $\blacksquare$