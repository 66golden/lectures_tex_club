\subsection{Теорема Цермело.}
\par \textbf{Теорема Цермело:} Любое множество можно вполне упорядочить, то есть у любого множества есть равномощное ему вполне упорядоченное множество (ВУМ).
\par $\blacktriangle$ Пусть $\varphi$ - функция из аксиомы выбора для множества $A$. Назовем корректным фрагментом ВУМ $\langle S, \leq_S \rangle$, где $S \subset A$ и $\forall x \in S \: x=\varphi(\{y|y <_S x\})$
\par По теореме о сравнении из двух корректных фрагментов один изоморфен начальному отрезку другого (так как они оба ВУМы). Покажем, что он не только изоморфен, но и равен начальному отрезку другого. 
\par Пусть это не так. Тогда пусть $x$ - минимальный элемент, в котором изоморфизм $f$ дал не то значение. Тогда начальный отрезок $[0;x)$ лежит в обоих корректных фрагментах, а значит $x=\varphi([0;x))=f(x)$ - противоречие.
\par Легко заметить, что объединение корректных фрагментов - это корректный фрагмент, так как если $x$ лежит в объединении, то $x$ лежит в каком-то из корректных фрагментов, а значит равенство $x=\varphi([0;x))$ сохраняется в объединении.
\par Объединим все корректные фрагменты (множество всех корректных фрагментов существует так как оно является подмножеством множества упорядоченных подмножеств $A$). Предположим, что мы получили $B \subset A, B \neq A$. Но тогда мы можем дополнить объединение элементом $\varphi(B)$ и получить корректный фрагмент, больший объединения всех корректных фрагментов - противоречие $\Rightarrow B=A \Rightarrow$ мы смогли вполне упорядочить $A \: \blacksquare$


\subsection{Лемма Цорна.}
\par \textbf{Лемма Цорна:} Пусть $Z$ — частично упорядоченное множество, в котором всякая цепь имеет верхнюю границу. Тогда в этом множестве есть максимальный элемент, и, более того, для любого элемента $a \in Z$ существует элемент $b \geq a$, являющийся максимальным в $Z$.
\par $\blacktriangle$ Пусть дан произвольный элемент $a$. Предположим, что не существует максимального элемента, большего или равного $a$. Это значит, что для любого $b \geq a$ найдётся $c > b$. Тогда $c > a$ и потому найдётся $d > c$ и т. д. Продолжая этот процесс достаточно долго, мы исчерпаем все элементы $Z$ и придём к противоречию.
\par Проведём рассуждение аккуратно. Возьмём вполне упорядоченное множество $I$ достаточно большой мощности (большей, чем
мощность $Z$, например $2^Z$). Построим строго возрастающую функцию $f : I \rightarrow Z$
по трансфинитной рекурсии. Её значение на минимальном элементе $I$ будет равно $a$. Предположим, что мы уже знаем все её значения на всех элементах, меньших некоторого $i$. В силу монотонности эти
значения попарно сравнимы, а значит, образуют цепь. Поэтому существует их верхняя граница $s$, которая, в частности, больше или равна $a$. Возьмём какой-то элемент $t > s$ и положим $f(i) = t$; по построению монотонность сохранится. Тем самым $I$ равномощно части $Z$, что противоречит его выбору.
\par В этом рассуждении, формально говоря, есть пробел: мы одновременно определяем функцию по трансфинитной рекурсии и доказываем её монотонность с помощью трансфинитной индукции. Наше рекурсивное определение имеет смысл, лишь если уже построенная
часть функции монотонна. Формально говоря, можно считать, что следующее значение не определено, если уже построенный участок не монотонен, и получить функцию, определённую на всём $I$ или на начальном отрезке. Если она определена
на некотором начальном отрезке, то она монотонна на нём по построению, поэтому следующее значение тоже определено — противоречие $\blacksquare$