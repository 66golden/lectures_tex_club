\subsection{Существование вычислимой функции, не имеющей всюду определённого вычислимого продолжения.}

\textbf{Теорема 1}.  Существует вычислимая функция d (с натуральными аргументами и значениями), от которой никакая вычислимая функция f не может всюду отличаться: для любой вычислимой функции f найдётся такое число n, что f(n) = d(n) (последнее равенство понимается в том смысле, что либо оба значения f(n) и d(n) не определены, либо оба определены и равны).

$\blacktriangle$
Такова диагональная функция d(n) = U(n, n) (здесь U — вычислимая функция двух аргументов,
универсальная для класса вычислимых функций одного аргумента). Любая вычислимая функция f есть $U_n$ при некотором n и потому $f(n) = U_n(n) = U(n, n) = d(n)$.
$\blacksquare$

\textbf{Теорема 2}.  Существует вычислимая функция, не имеющая всюду определённого вычислимого продолжения.

$\blacktriangle$
Пример: функция $d'(n) = d(n) + 1$, где d — функция из предыдущей теоремы. В самом деле, любое её всюду определённое продолжение всюду отличается от d (в тех местах, где функция d
определена, функция $d'$ на единицу больше d и потому любое продолжение функции $d'$ отличается от d; там, где d не определена, любая всюду определённая функция отличается от d).
$\blacksquare$

P.S. Формально и сама d не имеет вычислимого всюду определённого продолжения.