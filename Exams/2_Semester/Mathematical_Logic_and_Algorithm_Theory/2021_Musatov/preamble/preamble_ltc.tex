\documentclass[a4paper,12pt]{article}

%%% Работа с русским языком
\usepackage{cmap}					% поиск в PDF
\usepackage{mathtext} 				% русские буквы в формулах
\usepackage[T2A]{fontenc}			% кодировка
\usepackage[utf8]{inputenc}			% кодировка исходного текста
\usepackage[english,russian]{babel}	% локализация и переносы
\usepackage{indentfirst}
\frenchspacing

%%% Дополнительная работа с математикой
\usepackage{amsmath,amsfonts,amssymb,amsthm,mathtools} % AMS
\usepackage{icomma} % "Умная" запятая

%%% Шрифты
\usepackage{euscript} % Шрифт Евклид
\usepackage{mathrsfs} % Красивый матшрифт

%%% Свои команды
\usepackage{centernot}
\usepackage{stmaryrd}

\renewcommand{\epsilon}{\ensuremath{\varepsilon}}
\renewcommand{\phi}{\ensuremath{\varphi}}
\renewcommand{\kappa}{\ensuremath{\varkappa}}
\renewcommand{\le}{\ensuremath{\leqslant}}
\renewcommand{\leq}{\ensuremath{\leqslant}}
\renewcommand{\ge}{\ensuremath{\geqslant}}
\renewcommand{\geq}{\ensuremath{\geqslant}}
\renewcommand{\emptyset}{\varnothing}

\DeclareMathOperator{\sgn}{\mathop{sgn}}
\DeclareMathOperator{\nd}{\text{НОД}}
\DeclareMathOperator{\lf}{\text{НОК}}
\DeclareMathOperator{\rk}{rk}
\DeclareMathOperator{\pr}{pr}
\DeclareMathOperator{\im}{Im}
\DeclareMathOperator{\ke}{Ker}
\DeclareMathOperator{\re}{Re}
\DeclareMathOperator{\cha}{char}
\DeclareMathOperator{\ord}{ord}
\DeclareMathOperator{\tr}{tr}
\DeclareMathOperator{\md}{mod}
\DeclareMathOperator{\St}{St}
\DeclareMathOperator{\Aut}{Aut}
\DeclareMathOperator{\Inn}{Inn}
\DeclareMathOperator{\GL}{GL}
\DeclareMathOperator{\SL}{SL}
\DeclareMathOperator{\diag}{diag}
\DeclareMathOperator{\poly}{poly}

\newcommand{\divby}{
	\mathrel{\vbox{\baselineskip.65ex\lineskiplimit0pt\hbox{.}\hbox{.}\hbox{.}}}
}
\newcommand{\notdivby}{\centernot\divby}
\newcommand{\N}{\mathbb{N}}
\newcommand{\Z}{\mathbb{Z}}
\newcommand{\Q}{\mathbb{Q}}
\newcommand{\R}{\mathbb{R}}
\newcommand{\Cm}{\mathbb{C}}
\newcommand{\id}{\mathrm{id}}
\renewcommand\labelitemi{$\triangleright$}
\newcommand{\Chi}{\scalebox{1.05}{\raisebox{\depth}{$\chi$}}}

\let\bs\backslash
\let\mc\mathcal
\let\vect\overline
\let\normal\trianglelefteqslant
\let\lra\Leftrightarrow
\let\ra\Rightarrow
\let\la\Leftarrow
\let\gl\langle
\let\gr\rangle
\let\sd\leftthreetimes

%%% Перенос знаков в формулах (по Львовскому)
\newcommand*{\hm}[1]{#1\nobreak\discretionary{}{\hbox{$\mathsurround=0pt #1$}}{}}

%%% Работа с картинками
\usepackage{graphicx}  % Для вставки рисунков
\setlength\fboxsep{3pt} % Отступ рамки \fbox{} от рисунка
\setlength\fboxrule{1pt} % Толщина линий рамки \fbox{}
\usepackage{wrapfig} % Обтекание рисунков текстом

%%% Работа с таблицами
\usepackage{array,tabularx,tabulary,booktabs} % Дополнительная работа с таблицами
\usepackage{longtable}  % Длинные таблицы
\usepackage{multirow} % Слияние строк в таблице

%%% Теоремы
\theoremstyle{plain}
\newtheorem{theorem}{Теорема}[section]
\newtheorem{lemma}{Лемма}[section]
\newtheorem{proposition}{Утверждение}[section]
\newtheorem*{corollary}{Следствие}
\newtheorem*{exercise}{Упражнение}

\theoremstyle{definition}
\newtheorem{definition}{Определение}[section]
\newtheorem*{note}{Замечание}
\newtheorem*{reminder}{Напоминание}
\newtheorem*{example}{Пример}
\newtheorem*{tasks}{Вопросы и задачи}

\theoremstyle{remark}
\newtheorem*{solution}{Решение}

%%% Оформление страницы
\usepackage{extsizes} % Возможность сделать 14-й шрифт
\usepackage{geometry} % Простой способ задавать поля
\usepackage{setspace} % Интерлиньяж
\usepackage{enumitem}
\setlist{leftmargin=25pt}

\geometry{top=25mm}
\geometry{bottom=30mm}
\geometry{left=20mm}
\geometry{right=20mm}
\singlespacing % \onehalfspacing, \doublespacing

\setlength\parindent{15pt}        % Устанавливает длину красной строки 0pt
\sloppy                           % строго соблюдать границы текста
\linespread{1.3}                  % коэффициент межстрочного интервала
%\setlength{\parskip}{0.5em}      % вертик. интервал между абзацами
%\setcounter{secnumdepth}{0}      % отключение нумерации разделов
%\setcounter{section}{-1}         % Чтобы сделать нумерацию лекций с нуля
\usepackage{multicol}			  % Для текста в нескольких колонках
\usepackage{soulutf8}             % Модификаторы начертания

%%% Колонтитулы
\usepackage{titleps}
\newpagestyle{main}{
	\setheadrule{0.4pt}
	\sethead{\CourseName}{}{\hyperlink{intro}{\;Назад к содержанию}}
	\setfootrule{0.4pt}                       
	\setfoot{ФПМИ МФТИ, \CourseDate}{}{\thepage} 
}
\pagestyle{main}  

%%% Нумерация уравнений
\makeatletter
\def\eqref{\@ifstar\@eqref\@@eqref}
\def\@eqref#1{\textup{\tagform@{\ref*{#1}}}}
\def\@@eqref#1{\textup{\tagform@{\ref{#1}}}}
\makeatother % \eqref* без гиперссылки
\numberwithin{equation}{section}
\mathtoolsset{showonlyrefs=false} % Показывать номера только у тех формул, на которые есть \eqref{} в тексте.

%%% Содержаниие
% \usepackage{tocloft}
% \tocloftpagestyle{main}
% \setlength{\cftsecnumwidth}{2.3em}
% \renewcommand{\cftsecdotsep}{1}
% \renewcommand{\cftsecpresnum}{\hfill}
% \renewcommand{\cftsecaftersnum}{\quad}

%%% Гиперссылки
\usepackage{hyperref}
\usepackage[usenames,dvipsnames,svgnames,table,rgb]{xcolor}
\hypersetup{				% Гиперссылки
	unicode=true,           % русские буквы в раздела PDF
	pdftitle={Заголовок},   % Заголовок
	pdfauthor={Автор},      % Автор
	pdfsubject={Тема},      % Тема
	pdfcreator={Создатель}, % Создатель
	pdfproducer={Производитель}, % Производитель
	pdfkeywords={keyword1} {key2} {key3}, % Ключевые слова
	colorlinks=true,       	% false: ссылки в рамках; true: цветные ссылки
	linkcolor=black!10!blue,         % внутренние ссылки
	citecolor=green,        % на библиографию
	filecolor=magenta,      % на файлы
	urlcolor=NavyBlue,      % на URL
}

\usepackage{tikz}
\usepackage{tikz-cd} % Коммутативные диаграммы
\usepackage{tkz-euclide} % Геометрия
\usepackage{stackengine} % Многострочные тексты

\newcommand{\imp}[2]{(#1\,\,$\ra$\,\,#2)\,\,}
\newcommand{\implr}[2]{(#1\,\,$\lra$\,\,#2)\,\,}

\DeclareRobustCommand{\svdots}{% s for `scaling'
  \, \vcenter{%
    \offinterlineskip
    \hbox{.}
    \vskip0.25\normalbaselineskip
    \hbox{.}
    \vskip0.25\normalbaselineskip
    \hbox{.}%
  }% 
  \,
}

%%% Центрирование заголовков
% \usepackage{sectsty}
% \sectionfont{\centering}
% \subsectionfont{\centering}
% \subsubsectionfont{\centering}