\setcounter{section}{39}
\section{Определение эйлерова цикла. Критерий наличия эйлерова цикла в неориентированном графе.}

\textbf{Определение.} Вершина называется изолированной, если ее степень равна $0$. В случае неориентированного графа это равносильно тому, что из нее не выходит ни одного ребра. В случае ориентированного графа это равносильно тому, что количество входящих ребер и количество исходящих ребер равно 0.

\textbf{Определение.} Эйлеров путь~--- путь, проходящий по всем рёбрам графа и притом только по одному разу.

\textbf{Определение.} Эйлеров цикл~--- цикл, проходящий через каждое ребро графа ровно по одному разу.

\textbf{Определение.} Граф эйлеров, если в нем есть эйлеров цикл.
 
\begin{theorem}[Критерий эйлеровости неориентированного графа]
Неориентированный граф, который становится связным после удаления всех изолированных вершин, является эйлеровым  тогда и только тогда, когда все его вершины имеют четную степень.
\end{theorem}

\begin{proof}
    $\Rightarrow$
    
    Рассмотрим эйлеров обход графа. Заметим, что при попадании в вершину и при выходе из нее мы уменьшаем ее степень на два (помечаем уже пройденые ребра). Кроме того, для стартовой вершины мы уменьшаем ее степень на один в начале обхода эйлерова цикла, и на один при завершении. Следовательно вершин с нечетной степенью быть не может.
    
    $\Leftarrow$
    
    Необходимость мы доказали ранее. Докажем достаточность, используя индукцию по числу вершин $n$.

    База индукции: $n = 0$ цикл существует.

    Предположим что граф, имеющий менее $n$ вершин, степени вершин которого четны, содержит эйлеров цикл. \\
    
    Рассмотрим граф $G$ с $n > 0$ вершинами, степени которых четны. Удалим изолированные вершины. Если такие были, воспользуемся утверждением идукции. В противном случае у нас все еще $n$ вершины, однако теперь мы можем гарантировать, что граф связен. \\
    
    Пусть $v_1$~--- вершина графа. Начнем идти из этой вершины по ребрам графа до тех пор, пока можем (проходим по каждому ребру не более 1 раза). Если в какой-то момент мы не можем пойти дальше, выведем нашу вершину.
    
    \lstinputlisting[language=C++,
    emph={int,char,double,float,unsigned},
    emphstyle={\color{blue}}
    ]{code/40_euler_1.cpp}
    
    Заметим, что у вершин, которые встречаются в процессе обхода, степень каждый раз уменьшается на 2. Ведь если мы вошли в вершину, и она не стартовая, то в этот момент из нее ведет нечетное число непосещенных ребер. Тогда из нее можно выйти.
    
    Поэтому первая выведенная нами вершина будет стартовая, то есть $v_1$. Кроме того, мы получим замкнутый путь (цикл), который начинается и заканчивается в вершине $v_1$.
    
    Назовем этот цикл $C_1$. Если $C_1$ является эйлеровым циклом для $G$, тогда доказательство закончено. Если нет, то пусть $G_2$~--- подграф графа $G$, полученный удалением всех рёбер, принадлежащих $C_1$. Поскольку $C_1$ содержит чётное число рёбер, инцидентных каждой вершине, то каждая вершина подграфа $G_2$ имеет чётную степень. Этот граф разбивается на некоторое количество компонент связности.\\

    Рассмотрим какую-либо компоненту связности $G_2$ (не состоящую из изолированной вершины). Поскольку рассматриваемая компонента связности $G_2$ имеет менее, чем $n$ вершин (как минимум, туда не входит $v_1$, ведь все ее ребра были удалены), а у каждой вершины графа $G_2$ чётная степень, то у каждой компоненты связности $G_2$ существует эйлеров цикл. Пусть для рассматриваемой компоненты связноти это цикл $C_2$. У $C_1$ и $C_2$ имеется общая вершина $a$, так как $G$ cвязен. Давайте объединим $C_1$ и $C_2$ в новый цикл. Для этого нужно, начиная с вершины $a$, обойти $C_1$, вернуться в $a$, затем пройти по $C_2$ и вернуться в $a$. Если новый эйлеров цикл не является эйлеровым циклом для $G$, продолжаем использовать этот процесс, расширяя наш цикл, пока, в конце концов, не получим эйлеров цикл для $G$.
    
\end{proof}

\setcounter{section}{40}
\section{Реализация алгоритма поиска эйлерова цикла.}

\lstinputlisting[language=C++,
emph={int,char,double,float,unsigned},
emphstyle={\color{blue}}
]{code/41_euler_2.cpp}

Асимптотика $O(n + m)$

\setcounter{section}{41}
\section{Критерий наличия эйлерова пути в неориентированном графе.}

\begin{theorem}
Граф $G$ (связный, если убрать изолированные вершины) содержит эйлеров путь тогда и только тогда, когда количество вершин с нечетной степенью меньше или равно двум.
\end{theorem}

\begin{proof}
В одну сторону, очевидно, верно. Если у нас есть путь, то у вершин, отличных от концов, степень четна, а у начальной и конечной вершины степень нечетна.

В другую сторону. Если выполняется условие на степени вершин, то добавим ребро между двумя вершинами с нечетной степенью. В полученном графе есть эйлеров цикл, удаление из которого добавленного ребра даст эйлеров путь.\\

В этом доказательстве мы не рассмотрели случай, когда есть ровно одна вершина с нечетной степенью. Однако такой случай невозможен. По лемме о рукопожатиях в любом графе четное число вершин с нечетной степенью. А один -- нечетное число.
\end{proof}

\begin{theorem}[Лемма о рукопожатиях]
В любом неориентированном графе число вершин с нечетной степенью четно.
\end{theorem}

\begin{proof}
Действительно. Сложим степени всех вершин. Получим некоторое число. Заметим, что оно равно удвоенному числу ребер, ведь каждое ребро в нашей сумме учитывалось два раза -- по одному от каждого из его концов. Отсюда следует, что количество вершин с нечетной степенью четно.
\end{proof}

\setcounter{section}{42}
\section{Критерий наличия эйлерова цикла в ориентированном графе.}

\begin{theorem}[Критерий эйлеровости ориентированного графа]
Ориентированный граф $G$ (который становится сильно связным при удалении изолированных вершин) эйлеров тогда и только тогда, когда для каждой вершины верно, что входная степень равна выходной.
\end{theorem}

\begin{proof}
Абсолютно аналогично доказательству для неориентированного графа.
\end{proof}