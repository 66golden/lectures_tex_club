\section{12. Норма в линейном пространстве. Норма линейного оператора. Вычисление многочлена и аналитической функции от линейного оператора.}

\begin{definition}
    Функция называется \textit{аналитической}, если она представляется сходящимся степенным рядом.
\end{definition}

\begin{definition}
    Функция $||\cdot||: V \to \R$ называется \textit{нормой} если 
    \begin{enumerate}
        \item $||x|| > 0$, если $x \neq 0$,
        \item $|| \lambda x|| = |\lambda| \cdot ||x||$,
        \item $||x + y|| \leq ||x|| + ||y||$.
    \end{enumerate}
\end{definition}

\begin{definition}
    Последовательность векторов $\{x^m\}$ сходится по норме к $x_0$, 
    если $||x^m - x_0|| \to 0$ при $m \to +\infty$.
\end{definition}

\begin{definition}
    Ряд $\displaystyle\sum_{m=1}^{+\infty} x^m$ называется \textit{сходящимся}, если он сходится по норме 
    $S^n = \displaystyle\sum_{m=1}^{n}x^m$.
\end{definition}

\begin{definition}
    Ряд $\displaystyle\sum_{m=1}^{+\infty} x^m$ называется \textit{абсолютно сходящимся}, если сходится ряд 
    $\displaystyle\sum_{m=1}^{+\infty} ||x^m||$.
\end{definition}

\begin{proposition}
    Если ряд $\displaystyle\sum_{m=1}^{+\infty} a_m x^m$ сходится абсолютно, то он сходится, 
    и для сумм верно:
    $$||\displaystyle\sum_{m=1}^{+\infty} x^m|| \leq \displaystyle\sum_{m=1}^{+\infty} ||x^m||.$$
\end{proposition}

\begin{proposition}
    Если ряд $\displaystyle\sum_{m=1}^{+\infty} a_m x^m$ сходится абсолютно и $\phi: \N \to \N$, то 
    ряд $\displaystyle\sum_{m=1}^{+\infty} a_{\phi(m)} x^{\phi(m)}$ сходится и для этих двух рядов 
    верно: $$||\displaystyle\sum_{m=1}^{+\infty} a_{\phi(m)} x^{\phi(m)}|| = 
    ||\displaystyle\sum_{m=1}^{+\infty} a_m x^m||$$
\end{proposition}

\begin{definition}
    \label{def6.8}
    Пусть $\phi: V \to V$, $V$ конечномерно над $\R$ или $\Cm$. Тогда:
    $$||\phi|| \overset{\text{def}}{=} \underset{x \neq 0}{\max} \frac{||\phi(x)||}{||x||} = 
    \underset{||x|| = 1}{\max} \frac{||\phi(x)||}{||x||} = \underset{||x|| = 1}{\max} ||\phi(x)||.$$
\end{definition}

\begin{note}
    Если $\lambda$ -- собственное значение оператора $\phi$, то $||\phi|| \geq \lambda$.
\end{note}

\begin{proposition}[о свойствах нормы оператора]~
    \begin{enumerate}
        \item Определение \ref{def6.8} опеределяет норму в $\mathcal{L}(V)$.
        \item $||\phi(x)|| \leq ||\phi|| \cdot ||x||$.
        \item $||\phi \cdot \psi|| \leq ||\phi|| \cdot ||\psi||$.
    \end{enumerate}
\end{proposition}

\begin{proof}~
    \begin{enumerate}
        \item Докажем неравенство треугольника для нормы:
        \begin{multline*}
            ||\phi + \psi|| \overset{\text{def}}{=} \underset{x \neq 0}{\max} 
            \frac{||(\phi + \psi)(x)||}{||x||} \leq \underset{x \neq 0}{\max} 
            \frac{||\phi(x)|| + ||\psi(x)||}{||x||} \leq \\ \leq \underset{x \neq 0}{\max} 
            \frac{||\phi(x)||}{||x||} + \underset{x \neq 0}{\max} 
            \frac{||\psi(x)||}{||x||} = ||\phi|| + ||\psi||
        \end{multline*} 

        \item Докажем непосредственной проверкой:
        \begin{eqnarray*}
            ||\phi(x)|| = \frac{||\phi(x)||}{||x||} ||x|| \leq  \underset{x \neq 0}{\max} 
            \frac{||\phi(x)||}{||x||} ||x|| = ||\phi|| \cdot ||x||
        \end{eqnarray*}
        \item Докажем непосредственной проверкой:
        \begin{multline*}
            ||\phi \cdot \psi|| = \underset{x \neq 0}{\max} \frac{||\phi \cdot \psi(x)||}{||x||} = 
            \underset{\psi(x) \neq 0}{\max} \frac{||\phi(x)||}{||x||} =
            \underset{\psi(x) \neq 0}{\max} \frac{||\phi \cdot \psi(x)||}{||\psi(x)||} \cdot 
            \frac{||\psi(x)||}{||x||} \leq \\ \leq \underset{\psi(x) \neq 0}{\max} 
            \frac{||\phi \cdot \psi(x)||}{||\psi(x)||} \cdot \underset{\psi(x) \neq 0}{\max} 
            \frac{||\psi(x)||}{||x||} \leq ||\phi|| \cdot ||\psi||
        \end{multline*}
    \end{enumerate}
\end{proof}

\begin{theorem}
    Пусть ряд $f(t) = \displaystyle\sum_{m=1}^{+\infty} a_m t^m$ сходится при $|t| < R$.
    Тогда ряд $\displaystyle\sum_{m=1}^{+\infty} a_m \phi^m$ сходится абсолютно для любого оператора 
    $\phi: ||\phi|| = R_0 < R$. Более того, $f(\phi) = \displaystyle\sum_{m=1}^{+\infty} a_m \phi^m$ 
    - задает линейный оператор в $V$.
\end{theorem}

\begin{proof}
    $\forall x \in V$ докажем, что ряд $\displaystyle\sum_{m=1}^{+\infty} a_m \phi^m(x)$ 
    сходится абсолютно:
    \begin{multline*}
        \displaystyle\sum_{m} |a_m| \cdot ||\phi^m(x)|| \leq \displaystyle\sum_{m} |a_m| \cdot 
        ||\phi^m|| \cdot ||x|| \leq \\ \leq ||x|| \displaystyle\sum_{m} |a_m| \cdot ||\phi^m|| = 
        ||x|| \displaystyle\sum_{m} |a_m| R_0^m \text{\: -- \, сходится при } R_0 < R.   
    \end{multline*}
    Ряд $f(t) = \displaystyle\sum_{m} a_m t^m$ сходится при $|t| < R$, 
    а значит $\displaystyle\sum_{m} |a_m| |t|^m$ сходится при $|t| < R$ по теореме Абеля.
\end{proof}

\begin{note}
    \begin{gather*}
        exp(\phi) = \epsilon + \frac{\phi}{1!} + \dots + \frac{\phi^n}{n!} + \dots, \, R = +\infty \\
        sin(\phi) = \phi - \frac{\phi^3}{3!} + \dots + (-1)^n \frac{\phi^{2n+1}}{(2n+1)!} + \dots, 
        \, R = +\infty \\
        cos(\phi) = \epsilon - \frac{\phi^2}{2!} + \frac{\phi^4}{4!} - \dots + 
        (-1)^n \frac{\phi^{2n}}{(2n)!} + \dots, \, R = +\infty
    \end{gather*}
\end{note}
