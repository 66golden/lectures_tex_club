\section{9. Циклические подпространства. Теорема о нильпотентном операторе. Жорданова нормальная форма и жорданов базис линейного оператора. (Теорема существования жорданова базиса).}

\begin{definition}
    Оператор $\phi: V \to V$ называется \textit{нильпотентным}, если $\exists k \in \N :\: \phi^k = 0$. Наименьшее наутральное число $k$ такое, что $\phi^k = 0$, $\phi^{k-1} \neq 0$, называют \textit{индексом нильпотентности} относительно $\phi$. 
\end{definition}

\begin{definition}
    Пусть $\phi$ -- нильпотентный и $x \in V$ -- вектор, имеющий высоту $k$. 
    Рассмотрим $U = \langle x, \phi(x), \ldots, \phi^{k-1}(x)\rangle$.
    Построенное инвариантное подпространство $U$ называется \textit{циклическим подпространством}, 
    порожденным вектором $x$.
\end{definition}

\begin{theorem}[о нильпотентном операторе]~
    \label{th5.3}

    Пусть $\phi: V \to V$ -- нильпотентный оператор индекса нильпотентности $k$, $x \in V$ -- ненулевой вектор высоты $k$,
    $U = \langle x, \phi(x), \ldots, \phi^{k-1}(x) \rangle$ -- циклическое подпространство, инвариантное $\phi$.
    Тогда найдется $\phi$--инвариантное пространство $W$ дополнительное к $U$ такое, что $V = U \oplus W$.
\end{theorem}

\begin{proof}~

    \begin{enumerate}
        \item Пусть $W$ -- максимальное $\phi$-инвариантное подпространство в $V$, 
        такое что $U \cap W = {0}$. Предположим что $U + W < V$. 
        Тогда найдется ненулевой $a \in V$, такой что $ a \notin U + W$.  Пусть $l$ -- наименьшее 
        значение для которого $z = \phi^{l-1}(a) \notin W + U$, $\phi^l(a) \in W + U$. 
        Такое очевидно найдется так как $a \notin W + U$ и $\phi^{k}(a) = 0 \in W + U$.
        Таким образом в этом пункте мы нашли вектор $z \notin U+W$, такой что $\phi(z) \in U + W$.

        \item Пусть $\phi(z) = \displaystyle\sum_{s = 0}^{k-1} \alpha_s \phi^s(x) + w$, 
        при этом $\phi^s(x) \in U$, $w \in W$. Тогда:
        $$\phi^{k}(z) = \alpha_0 \phi^{k-1}(x) + 0 + \dots + 0 + \phi^{k-1}(w) = 0$$ 
        Тогда $\alpha_0 \phi^{k-1}(x) + \phi^{k-1}(w) = 0$. 
        В силу линейной независимости линейных подпространств 
        $\alpha_o \phi^{k-1}(x) = 0$, $\phi^{k-1}(w) = 0$. 
        При этом в силу того, что $\phi^{k-1}(x) \neq 0$, получаем $\alpha_0 = 0$.

        \item Введем вектор $y = z - \displaystyle\sum_{s = 1}^{k-1} \alpha_s \phi^{s-1}(x) \notin U + W$ 
        (так как $z \notin U+W$, а сумма принадлежит $U$).
        Введем пространство $W' = W + \langle y \rangle$, $\dim W' = \dim W + 1$. 
        Покажем что вновь построенное подпространство так же инвариантно $\phi$:
        $$\phi(y) = \phi(z) - \displaystyle\sum_{s=1}^{k-1} \alpha_s \phi^s (x) = \phi(z) - 
        \displaystyle\sum_{s=0}^{k-1} \alpha_s \phi^s (x) = w \in W.$$

        \item Покажем теперь что $W'$ удовлетворяет условию $U \cap W' = {0}$. 
        Пусть $0 \neq u \in U \cap W'$, $u \notin U \cap W = \{ 0 \}$. 
        Тогда $u$ предствим в виде $u = \widetilde{w} + \lambda y$, $\lambda \neq 0$. Отсюда 
        $y = \frac{1}{\lambda} u - \frac{1}{\lambda} \widetilde{w} \in U + W$. 
        Значит $U \cap W \neq \{0\}$ -- противоречие.
    \end{enumerate}
\end{proof}

\begin{theorem}[о разложении в прямую сумму циклических подпространств для нильпотентного оператора]
    Пусть $\phi: V \to V$, зафиксируем индекс нильпотентности $k$. Тогда существует разложение $V$ 
    в прямую сумму инвариантных циклических подпространств $V = V_1 \oplus V_2 \oplus \ldots \oplus V_s$. 
    При этом количество слагаемых $s = \dim (\ker \phi) = \dim V_0 = geom(0)$.
\end{theorem}

\begin{proof}
    Индукция по $n = \dim V$. 
    \begin{enumerate}
        \item База: $n = 1 \Rightarrow \phi = 0$, $alg(0) = geom(0) = 1$.
        \item Предположение индукции: для пространства $V$ размерности менее $n$ утверждение выполняется. 
        Пусть теперь $\dim V = n$. 
        Тогда существует $x$ высоты $k$ такой что $\phi^k(x) = 0$, $\phi^{k-1}(x) \neq 0$. 
        Пусть $U = \langle x, \phi(x), \dots, \phi^{k-1}(x) \rangle$ -- $\phi$-инвариантное подпространство.
        По теореме \ref{th5.3} существует $\phi$-инвариантное подпространство $W$, такое что  
        $V = U \oplus W$, $\dim W \leq n-1$. Тогда $W$ раскладывается в прямую сумму 
        $\phi$-инвариантных циклических подпространств.
    \end{enumerate}
\end{proof}

\begin{definition}
    Жордановой клеткой, относящейся к  $\lambda \in F$, называется следующая матрица:
    \[J_{k}(\lambda) = \begin{pmatrix}
	\lambda      & 1      & 0      & \dots  & 0\\
		0      & \lambda      & 1      & \dots  & 0\\
		\vdots & \vdots & \vdots & \ddots & \vdots \\
        0      & 0      & \dots      & \lambda  & 1\\
        0      & 0      & 0      & \dots  & \lambda \\
	\end{pmatrix}\]
\end{definition}

\begin{definition}
    Жордановой матрицей называется блочно-диагональная матрица, по главной диагонали которой идут 
    Жордановы клетки, а остальное заполено нулями:
    \[J_{k}(\lambda) = \begin{pmatrix}
        J_{k_1}(\lambda_1)      & \dots      & 0    & 0 \\
        \vdots      & J_{k_2}(\lambda_2)      & \dots   & 0 \\
        0   & \vdots     & \ddots    & \vdots \\
        0      & 0      & \dots    & J_{k_n}(\lambda_n) \\
        \end{pmatrix}\]
\end{definition}

\begin{theorem}[Камиль Жордан]
    Пусть $\phi : V \to V$, $\phi$ -- линейно факторизуем над $F$. 
    Тогда в $V$ существует базис (Жордановый базис), в котором $\phi$ имеет Жорданову матрицу.
\end{theorem}

\begin{note}
    Жорданова матрица определена с точностью до перестановки Жордановых клеток, поэтому базис не единственен в общем случае.
\end{note}

\begin{proof}
    Заметим, что:
    \begin{enumerate}
        \item $V = V^{\lambda_1} \oplus V^{\lambda_2} \oplus \ldots V^{\lambda_k}$ (подпространства инвариантны), где $\lambda_1, \ldots, \lambda_k$ -- все попарно различные собственные значения оператора $\phi$. Тогда в базисе согласованном с таким разложением матрица имеет вид:
         \[A = \begin{pmatrix}
            A_1      & \dots      & 0    & 0 \\
            \vdots      & A_2      & \dots   & 0 \\
            0   & \vdots     & \ddots    & \vdots \\
            0      & 0      & \dots    & A_k \\
            \end{pmatrix}.\]
        \item Для $V^{\lambda_i}$ оператор $\phi_{\lambda_i} = \phi - \lambda_i E$ нильпотентен, а значит V 
        раскладывается в сумму циклических подпространств: $V^{\lambda_i} = \displaystyle\sum_{j = 1}^{geom(\lambda_i)} V_{ij}$.
    \end{enumerate}
    Пусть $\dim V_{ij} = k$. Покажем, что на $V_{ij}$ оператор $\phi$ в подходящем базисе имеет вид $J_k(\lambda_i)$:\\ 
    Пусть $k$ -- индекс нильпотентности $\phi_{\lambda_i}$ на $V_{ij}$, пусть $x$ -- корневой вектор максимальной высоты $k$.\\
    Рассмотрим базис $\langle \phi_{\lambda_i}^{k-1} x, \phi_{\lambda_i}^{k-2} x, \dots, \phi_{\lambda_i}^{1} x \rangle$. 
    Обозначим базисные вектора за $f_{ij}$ следующим образом:
    \begin{gather*}
        f_{i1} = \phi_{\lambda_i}^{k - 1},\\
        f_{i2} = \phi_{\lambda_i}^{k - 2},\\
        \dots
    \end{gather*}
    Подействуем на базис оператором $\phi_{\lambda_i}$. Под действием этого оператора каждый базисный 
    вектор перейдет в предыдущий (первый перейдет в $0$): $\phi_{\lambda_i}(f_{i1}) = \overline{0}, \dots, \phi_{\lambda_i}(f_{ik}) = f_{i(k - 1)}$. 
    Тогда матрица оператора $\phi_{\lambda_i}$ будет иметь в базисе $f$ вид $J_k(0)$.
    Тогда $\phi \vert_{V_{ij}} = \lambda_1 \epsilon + J_k(0) = J_k(\lambda_i)$
    Мы доказали, что в подходящем базисе сужение на подпростанство имеет вид Жордановой клетки. Тогда из
    $V = \displaystyle\sum_{i = 1}^{k} \displaystyle\sum_{j = 1}^{geom(\lambda_i)} V_{ij}$
    вытекает, что матрица оператора в подходящем базисе (Жордановом базисе) имеет вид Жордановой матрицы.
\end{proof}