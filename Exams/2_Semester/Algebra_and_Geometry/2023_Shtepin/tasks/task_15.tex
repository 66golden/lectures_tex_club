\section{15. Симметричные билинейные и квадратичные функции, связь между ними. Поляризационное тождество. Метод Лагранжа приведения квадратичной формы к каноническому виду.}

\begin{definition}
    Пусть $f \in B(V)$, $f: V \times V \to F$. Тогда $\Delta = \{(x, x) \in V \times V\}$ -- 
    \textit{диагональ} в пространстве $V$.
\end{definition}

\begin{definition} 
    Пусть $f \in B^{+}(V)$. \textit{Квадратичной функцией} на $V$ называется произвольное сужение симметричной 
    билинейной функции $f$ на диагональ $\Delta$:
    \begin{gather*}
        q(x) = f(x, y) \vert_{\Delta} = f(x, x): \, V \to F.
    \end{gather*} 
\end{definition}

\begin{agreement}
    Будем обозначать как $Q(V)$ множество всех квадратичных функций на $V$.
\end{agreement}

\begin{theorem}
    Линейные пространства $B^+(V)$ и $Q(V)$ изоморфны, изоморфизм осуществляет отображение $\phi$ 
    сужения на диагональ $\Delta \subset V \times V$.
\end{theorem}

\begin{proof}
    Пусть $\phi: B^+(V) \to Q(V)$, переводящее $f(x, x) \in B^+(V)$ в $q(x) \in Q(V)$. 
    Операции сложения и умножения на скаляр сохраняются. Покажем его биективность:
    \begin{enumerate}
        \item Отображение $\phi$ сюръективно по определению квадратичной функции.  
        \item Проверим инъективность $\phi$. Пусть $\phi(f) = q$, $\phi(g) = q$, покажем, что тогда 
        $f = g$. 
        
        По определению $q(x) = f(x, x)$, тогда:
        \begin{gather*}
            q(x \pm y) = f(x \pm y, x \pm y) = f(x, x) \pm 2 f(x, y) + f(y, y) =
            q(x) \pm 2 f(x, y) + q(y).
        \end{gather*}
        Аналогично $q(x \pm y) = q(x) \pm 2 g(x, y) + q(y)$, откуда:
        \begin{gather*}
            f(x, y) = \frac{1}{4} (q(x+y) - q(x-y)) = g(x, y).
        \end{gather*} 
    \end{enumerate} 
    Таким образом полученное отображение - биекция, сохраняющая необходимые операции, а значит 
    получен изоморфизм между $B^+(V)$ и $Q(V)$.
\end{proof}

\begin{definition}
    Выражение $f(x, y)$ через $q(x)$ и $q(y)$ называется \textit{поляризационным тождеством}.
    Обратное отображение $\psi:\, Q(V) \to B^+(V)$ называется \textit{поляризацией},
    $f(x, y)$ -- \textit{полярной функцией} к $q(x)$.
\end{definition}

\begin{definition}
    Базис в $V$ называется \textit{ортогональным} относительно $f$ если для всех $i$, $j$, $i \neq j$ верно 
    $a_{ij} = f(e_i, e_j) = 0$.
\end{definition}

\begin{theorem}[Лагранжа]
    Всякую билинейную симметричную функцию $f$ и ассоциированную с ней квадратичную функцию 
    подходящим выбором базиса можно привести к диагональному виду.
\end{theorem}

\begin{proof}
    Индукция по размерности пространства.
    \begin{enumerate}
        \item База: при $n=1$ матрица уже имеет диагональный вид.
        \item Предположение индукции: пусть для пространств $V$ размерности меньшей чем $n$ 
        теорема верна. Совершим переход к подпространствам размерности $n+1$.

        Если функция $f$ тождественно нулевая, её матрица так же очевидно диагональная. В случае ненулевой функции $f$ в силу поляризационного тождества функция $q$ так же является ненулевой. Тогда существует вектор $e_1$, такой что $q(e_1) = a_{11} = f(e_1, e_1) \neq 0$. Рассмотрим тогда $U = \langle e_1 \rangle$. Тогда $f \vert_{U}$ невырождена, а значит $V = U \oplus U^{\perp}$. 

        По предположению индукции в $U^{\perp}$ найдется ортогональный относительно сужения 
        $f \vert_{U^{\perp}}$ базис $(e_2, \dots e_n)$. Матрица $A_{U^{\perp}}$ в нем будет иметь вид:
        \begin{gather*}
            A_{U^{\perp}} = \begin{pmatrix}
                \lambda_2  & 0         & \dots  & 0         \\
                0          & \lambda_3 & \dots  & 0         \\
                \vdots     & \vdots    & \ddots & \vdots    \\
                0          & 0         & \dots  & \lambda_n
            \end{pmatrix}
        \end{gather*}
        В силу того, что $U$ и $U^{\perp}$ образуют прямую сумму, равную всему пространству $V$,
        при добавлении в $(e_2, \dots e_n)$ вектора $e_1$ получится базис в $V$, 
        являющийся ортогональным относительно $f$. Матрица $f$ в базисе $(e_1, e_2, \dots e_n)$ имеет 
        вид:
        \begin{gather*}
            A = \begin{pmatrix}
                \lambda_1  & 0         & \dots  & 0         \\
                0          & \lambda_2 & \dots  & 0         \\
                \vdots     & \vdots    & \ddots & \vdots    \\
                0          & 0         & \dots  & \lambda_n
            \end{pmatrix}
        \end{gather*}
        При этом коэффициенты в матрице равны $\lambda_i = q(e_i)$.
    \end{enumerate}
\end{proof}

\begin{definition}
    Пусть $F = \R$. Вид квадратичной функции
    \begin{align*}
        q(x) = x_1^2 + x_2^2 + \dots + x_p^2 - x_{p+1}^2 - \dots - x_{p+q}^2,
    \end{align*} 
    где $p + q = \rk q$, называется \textit{каноническим видом} квадратичной функции в $V$ над $\R$.
\end{definition}

\begin{corollary}
    Если $F = \R$, то всякую квадратическую функцию выбором базиса можно привести к каноническому виду 
    выбором базиса.
\end{corollary}

\begin{algorithm}[Поиск преобразования, приводящего к каноническому виду]~\\
    В нашем базисе $q(x)$ имеет следующее предстваление:
    \begin{gather*}
        q(x) = \sum_{i=1}^{n}\sum_{j=1}^{n} a_{ij}x_i x_j
    \end{gather*}
    Его можно преобразовать к виду: 
    \begin{gather*}
        q(x) = \frac{1}{a_{11}} (a_{11} x_1 + a_{12} x_2 + \dots + a_{1n} x_n)^2 - 
        \sum_{i=2}^{n}\sum_{j=2}^{n} a_{ij}x_i x_j.
    \end{gather*}
    Сумма после вынесения первого слагаемого не содержит $x_1$ ни в одном члене. Обозначим тогда 
    $(a_{11} x_1 + a_{12} x_2 + \dots + a_{1n} x_n)$ за $\xi_1$, который будет являться первым 
    искомым каноническим вектором. После этого $q(x)$ можно записать как:
    \begin{gather*}
        q(x) = \lambda_1  \xi_1^2 + \sum_{i=1}^{n}\sum_{j=1}^{n} a_{ij}x_i x_j.
    \end{gather*}
    Таким образом можно продолжать преобразования суммы до получения разложения в канонический вид:
    \begin{gather*}
        q(x) = \lambda_1  \xi_1^2 + \lambda_2 \xi_2^2 + \dots \lambda \xi_n^n
    \end{gather*}
    При этом столбцы матрицы $S$ будут являться координатами векторов базиса в каноническом базисе.
\end{algorithm}
