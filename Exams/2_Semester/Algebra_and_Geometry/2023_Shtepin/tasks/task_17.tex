\section{17. Положительно определенные квадратичные функции. Критерий Сильвестра. Кососимметрические билинейные функции, приведение их к каноническому виду.}

\begin{proposition}~
    \label{pr10.1}
    \begin{enumerate}
        \item Функция $q(x)$ положительно определена тогда и только тогда когда приводится к каноническому
        виду с матрицей $E$.
        \item Функция $q(x)$ положительно полуопределена тогда и только тогда когда приводится к 
        каноническому виду с матрицей, не имеющей $-1$ на главной диагонали.
    \end{enumerate}
\end{proposition}

\begin{proof}~
    \begin{enumerate}
        \item \begin{enumerate}
            \item Необходимость. 
            
            Пусть $q(x)$ положительно определена. Рассмотрим канонический базис $e$.
            В этом базисе $i$-й элемент матрицы $q$ равен $a_{ii} = q(e_i) > 0$. 
            
            В силу того, что в каноническом 
            базисе матрица может иметь только значения $\pm 1$ и $0$ на главной диагонали, получаем $a_{ii} = 1$.
            Таким образом матрица формы $q$ является единичной.
            \item Достаточность. 
            
            Пусть $q$ приводится к каноническому виду с $E$. Тогда в каноническом базисе: 
            $$q(x) = \xi_1^2 + \, \dots \,+ \xi_n^2, \, \text{ где } \, n = \dim V.$$ 
            Это значит, что для всех $x \neq 0$ верно $q(x) > 0$, так как в каноническом базисе 
            $x$ представляется в виде $x = (\xi_1 \, \xi_2 \, \dots \, \xi_n)^T$. Таким образом 
            $q$ положительно определена.
        \end{enumerate} 

        \item \begin{enumerate}
            \item Необходимость.
            
            Пусть $q(x)$ положительно полуопределена. Тогда в каноническом базисе $i$-й элемент 
            главной диагонали матрицы $q$ равен $a_{ii} = q(x_i) \geq 0$, 
            откуда $a_{ii} \in \{0, 1\}$.

            \item Достаточность.
            
            Пусть в каноническом базисе $a_{ii} \in \{ 0, 1\}$. Тогда $q$ в нем имеет вид:
            $$q(x) = \xi_1^2 + \, \dots \,+ \xi_p^2, \, \text{ где } \, p < \dim V.$$
            Таким образом для всех $x$ верно $q(x) \geq 0$, что значит, что $q$ положительно
            полуопределена.
        \end{enumerate}
    \end{enumerate}
\end{proof}

\begin{lemma}
    \label{pr10.3}
    Пусть $B \in M_n(\R)$ -- квадратная матрица над полем вещественных чисел. Тогда $B$ положительно 
    определена тогда и только тогда, когда существует невырожденная $A \in M_n(\R)$ такая, что 
    $B = A^T A$.
\end{lemma}

\begin{proof}~
    \begin{enumerate}
        \item Необходимость.
        
        Пусть $B$ положительно определена. Тогда она является матрицей некоторой 
        квадратичной функции $q$, что значит, что существует матрица $S = S_{e \to e'}$ такая, что 
        $S^T B S = E$. 
        
        Домножим выражение на $(S^T)^{-1}$ слева и на $S^{-1}$ справа и получим $B = (S^T)^{-1} S^{-1}$.

        Тогда искомая $A$ существует и равна $A = S^{-1}$. 
        \item Достаточность.
        
        Пусть $B = A^T A$, $\det A \neq 0$ (в силу невырожденности $A$). Тогда положим $S = A^{-1}$. 

        В новом базисе $B' = S^TBS = (A^{-1})^T A^T A A^{-1} = E$, откуда $B$ положительно определена по 
        утверждению \ref{pr10.1}.
    \end{enumerate}
\end{proof}

\begin{theorem}[Критерий Сильвестра]
    Пусть $q(x) \in Q(V)$. Тогда верно следующее:
    \begin{enumerate}
        \item Форма $q(x)$ положительно определена тогда и только тогда когда для всех $i$ главный минор 
        положителен: $\Delta_i > 0$.
        \item Форма $q(x)$ отрицательно определена тогда и только тогда когда знаки главных миноров чередуются:
        $\sgn(\Delta_i) = (-1)^i$.
    \end{enumerate}
\end{theorem}

\begin{proof}~
    \begin{enumerate}
        \item \begin{enumerate}
            \item Необходимость.
            
            Пусть $B$ -- матрица квадратичной функции $q(x)$ и $q$ положительно определена.
            Тогда по лемме \ref{pr10.3} верно $B = A^T A$, $\det A \neq 0$. В таком случае: 
            \begin{gather*}
                |B| = |A^T| \cdot |A| = |A|^2 > 0.
            \end{gather*}
            \item Достаточность.
            
            Пусть $\Delta_1 > 0, \dots \Delta_n > 0$. Тогда: 
            \begin{gather*}
                q(x) = \frac{\Delta_0}{\Delta_1} \xi_1^2 + \frac{\Delta_1}{\Delta_2} \xi_2^2 + \dots + 
                \frac{\Delta_{n-1}}{\Delta_n} \xi_n^2,
            \end{gather*}
            что значит, что $q(x)$ положительно определена так как при $x \neq 0$ верно $q(x) > 0$.
        \end{enumerate}
        \item Заметим, что если $q(x)$ положительно определена, то $-q(x)$ отрицательно определена. Пусть 
        $q(x)$ определена отрицательно, тогда $-q(x)$ определена положительно. Выпишем её матрицу:
        \begin{gather*}
            \begin{pmatrix}
                -a_{11} & -a_{12} & \dots  & -a_{2n} \\
                -a_{21} & -a_{22} & \dots  & -a_{2n} \\
                \vdots  & \vdots  & \ddots & \vdots  \\
                -a_{n1} & -a_{n2} & \dots  & -a_{nn}
            \end{pmatrix}
        \end{gather*}

        Тогда $\Delta_1 = -a_{11} > 0$, откуда $a_{11} < 0$. 
        
        Продолжим вычислять миноры:
        $\Delta_2 = \begin{vmatrix}
        -a_{11} & -a_{12}  \\
        -a_{21} & -a_{22}  
        \end{vmatrix} = \begin{vmatrix}
            a_{11} & a_{12}  \\
            a_{21} & a_{22}  
        \end{vmatrix} > 0$. 
        
        Вычисляя аналогично миноры большего размера получим, 
        что знак меняется на каждом шаге, что значит, что $\sgn(\Delta_i) = (-1)^i$.
    \end{enumerate}
\end{proof}

\begin{reminder}
    Билинейная функция $f(x, y)$ называется \textit{кососимметричной}, если для всех $x, y \in V$ верно:
    \begin{enumerate}
        \item $f(x, y) = -f(y, x)$,
        \item $f(x, x) = 0$.
    \end{enumerate}
\end{reminder}

\begin{definition}
    Базис $e = \langle e_1, \dots e_n \rangle$ называется \textit{симплектическим} для билинейной формы $f(x, y)$, 
    если для $S = 1,\dots n$ верно:
    \begin{gather*}
        f(e_{2S - 1}, e_{2S}) = 1 \Rightarrow f(e_{2S}, e_{2S-1}) = -1,
    \end{gather*} а для остальных значений $i, j$ верно $f(e_i, e_j) = 0$. 
    Матрица в таком случае имеет следующий вид: 
    \[A_f = \left(\begin{array}{@{}cccc@{}}
		\cline{1-1}
		\multicolumn{1}{|c|}{A_1} & 0 & \dots & 0\\
		\cline{1-2}
		0 & \multicolumn{1}{|c|}{A_2} & \dots & 0\\
		\cline{2-2}
		\vdots & \vdots & \ddots & \vdots\\
		\cline{4-4}
		0 & 0 & \dots & \multicolumn{1}{|c|}{A_m}\\
		\cline{4-4}
	\end{array}\right),\]
	
	где для всех $i$ матрица $A_i$ нулевая или имеет вид $A_i = 
    \begin{pmatrix}
        0  &1 \\
		-1 &0
    \end{pmatrix}$.
\end{definition}

\begin{theorem}[О каноническом виде кососимметричной билинейной функции]~

    Если $f(x, y)$ -- кососимметричная билинейная функция в $V$, то в $V$ существует симплектический базис.
\end{theorem}

\begin{proof}~
    Докажем по индукции по размерности пространства $V$. 
    \begin{enumerate}
        \item Если $f(x, y) = 0$ для всех $x, y$, то $S = 0$ -- очевидно.
        \item Если $f \neq 0$, то найдутся векторы $e_1, e_2$ такие, что $f(e_1, e_2) = c \neq 0$. 
        
        Рассмотрим тогда векторы $e_1' = e_1$, $e_2' = \frac{e_2}{c}$, для которых верно $f(e_1', e_2') = 1$.
        
        Тогда в $V$ существует невырожденное подпространство $U = \langle e'_1, e'_2 \rangle$,
        в котором матрица будет иметь вид $A_{f \vert_{U}} = \begin{pmatrix}
            0  &1 \\
            -1 &0
        \end{pmatrix}$.
        
        По теореме о невырожденном пространстве $V = U \oplus U^{\perp}$. Таким образом 
        если $\dim V = 2$, то искомый базис получен. Иначе по предположению
        индукции искомый базис найдется для $U^{\perp}$, а значит при объединении с $e'_1$ и $e'_2$ 
        получим базис для $V$.
    \end{enumerate}
\end{proof}