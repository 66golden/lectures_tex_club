\section{1. Кольцо многочленов над полем. Наибольший общий делитель. Алгоритм Евклида. Линейное выражение НОД.}

\begin{reminder}
    \textit{Кольцом} называется множество $R$ с определенными на нем бинарными операциями \textit{сложения} $+ : R \times R \to R$ и \textit{умножения} $\cdot: R \times R \rightarrow R$, удовлетворяющими следующим условиям:
    \begin{itemize}
        \item $(R, +)$ "--- абелева группа, нейтральный элемент в которой обозначается через $0$
        \item $\forall a, b, c \in R: (ab)c = a(bc)$ (ассоциативность умножения)
        \item $\forall a, b, c \in R: a(b + c) = ab + ac$ и $(a + b)c = ac + bc$ (дистрибутивность умножения относительно сложения)
    \end{itemize}
\end{reminder}

\begin{reminder}
    \textit{Полем} называется такое коммутативное кольцо $(F, +, \cdot)$, для которого выполнено равенство $F^* = F\backslash\{0\}$.
\end{reminder}

\begin{definition}
    Последовательность $(a_0, a_1, a_2,\ldots), a_i \in R$ называют \textit{финитной} если 
    $\exists N : \forall n>N \hookrightarrow a_n = 0$, т.е. если начиная с некоторого номера $N$ все значения $a_n$ равны нулю.
\end{definition}

\begin{definition}
    Пусть R -- коммутативное кольцо с единицей. \textit{Многочлен} над R -- финитная последовательность элементов A = $(a_0, a_1, a_2,\ldots), a_i \in R$. Дополнительно будем использовать обозначение $(A)_i = a_i$.
\end{definition}

\begin{definition}
    $R[x]$ -- множество многочленов над кольцом R.
\end{definition}

\begin{definition}
    Пусть $A, B \in R[x]$, тогда верны следующие свойства:
    \begin{enumerate}
        \item $(A + B)_n = (A)_n + (B)_n$,
        \item $(A \cdot B)_n = \displaystyle\sum_{i = 0}^{n}(A)_i \cdot (B)_{n-i}$,
        \item $\lambda \in R \; (\lambda A)_n = \lambda \cdot (A)_n$.
    \end{enumerate}
\end{definition}

\begin{proposition}
    Множество всех многочленов R[x] является коммутативным кольцом с 1. $1 = (1, 0, 0,\ldots)$ -- нейтральный по умножению многочлен.
\end{proposition}

\begin{definition}
    Введем обозначения $x = (0, 1, 0, 0, \ldots)$, $x^2 = (0, 0, 1, 0, \ldots)$ и т.д. Тогда многочлен $A = (a_0, a_1, a_2, \ldots)$ можно записать как $A = a_0 \cdot 1 + a_1 \cdot x + a_2 \cdot x^2 + \ldots$.
\end{definition}

\begin{definition}
    Пусть $P = (a_0, a_1, a_2, \ldots)$ -- многочлен. Последний отличный от нуля коэффициент называется \textit{старшим коэффициентом} многочлена. Номер старшего коэффициента называется степенью многочлена и обозначается как $\deg P$.
\end{definition}

\begin{note}
    Будем считать, что степень нулевого многочлена и только нулевого многочлена не определена.
\end{note}

\begin{reminder}
    Делителями нуля называются такие числа $a$ и $b$, что $a \neq 0$ и $b \neq 0$ но $a \cdot b = 0$.
\end{reminder}

\begin{definition}
    Коммутативное кольцо с единицей называется областью целостности или целостным кольцом если оно 
    не имеет делителей нуля.
\end{definition}

\begin{proposition} 
    В области целостности выполняется правило сокращения:
    $ab = ac, a \neq 0 \Rightarrow b = c$.
\end{proposition}

\begin{proof}
    $a(b-c) = 0$, $a \neq 0 \Rightarrow b-c = 0$.
\end{proof}

\begin{proposition}
    Пусть R -- коммутативное кольцо с единицей, $A, B \in R[x]$, тогда:
    \begin{enumerate}
        \item $\deg(A+B) \leq max(\deg(A), \deg(B))$,
        \item $\deg(A \cdot B) \leq \deg(A) + \deg(B)$,
        \item Если вдобавок R -- область целостности, то $\deg(AB) = \deg(A) + \deg(B)$.
    \end{enumerate}
\end{proposition}

\begin{proof}
    \begin{enumerate}
        \item Обозначим $\deg A = a$, $\deg B = b$. Пусть $n > max(a, b)$, тогда
        $(A+B)_n = (A)_n + (B)_n = 0 + 0 = 0$, а значит $\forall n > max(a, b) \Rightarrow (A+B)_n = 0$. 
        Тогда номер последнего ненулевого элемента не превосходит $max(a, b)$, а значит 
        $deg(A+B) \leqslant max(a, b)$

        \item Пусть $n > a + b$, покажем что $(AB)_n = 0$:
        
        $$(AB)_n = \sum_{i = 0}^{a}(A_i)(B_{n-i}) + \sum_{i = a+1}^{n}(A_i)(B_{n-i}) = 0 + 0 = 0$$

        В первой сумме $B_{n-i} = 0$ во всех слагаемых так как $n > a + b$, а значит $n - i > b$ для
        всех $i$ от $0$ до $a$. Во второй сумма во всех слагаемых $A_i = 0$ так как $i > a$ на всем
        диапазоне суммирования. Таким образом обе суммы равны нулю, а значит $(AB)_n = 0$.

        \item Положим $n = a + b$, тогда:
        
        $$(AB)_{n} = \sum_{i=0}^{a-1} (A)_i(B)_{n-i} + (A)_a(B)_b + \sum_{i=a+1}^{a+b} (A)_i(B)_{n-i}$$

        Аналогично предыдущему пункту первое и третье слагаемое будут нулевыми. 
        При этом $(A)_i \neq 0$ и $(B)_{n-i} = (B)_b \neq 0$,  и в силу целостности 
        $((A)_i(B)_{n-i} \neq 0)$, то есть $(AB)_{n} \neq 0$. Для больших
        чем $n$ номеров сумма будет нулевой из предыдущего пункта, а значит $\deg AB = a$.
    \end{enumerate}
\end{proof}

\begin{theorem}[о существовании деления с остатком]
    Пусть $A, B \in F[x]$, F -- поле, $B \neq 0$. Тогда:
    \begin{enumerate}
        \item Cуществуют $Q, R \in F[x]$ т.ч. $A = QB + R$, где $R = 0$ или $\deg R < \deg B$.
        \item Многочлены R и Q определены однозначно.
    \end{enumerate}
\end{theorem}

\begin{proof}~
    \begin{enumerate}
        \item Индукция по $\deg A$:
    
        Пусть $A = 0$ или $\deg A < \deg B$, тогда очевидно $A = 0 \cdot B +A$.

        Пусть теперь $\deg A \geq \deg B$, и они равны a и b соответственно. Тогда старшие 
        члены равны $HT(A) = \alpha x^a$ и $HT(B) = \beta x^b$. Подберем моном M такой что 
        $HT(A) = M \cdot HT(B)$, например $M = \frac{\alpha}{\beta} \cdot x^{a - b}$.

        Введем обозначение $A' = A - MB$, $\deg A' < \deg A$ по построению $M$. По предположению 
        $A' = Q'B + R'$, где $R' = 0$ или $\deg R' < \deg B$. Тогда 
        $A = A' + MB = Q'B + MB + R' = (Q' + M)B + R'$ -- искомое разложение.

        \item Предположим существуют два разложения $A = Q_1 B + R_1 = Q_2 B + R_2$, многочлены 
        удовлетворяют условиям. 

        $(Q_1 - Q_2)B = R_2 - R_1$. Предположим $Q_1 \neq Q_2$, тогда $\deg((Q_1 - Q_2)B) \geq \deg(B)$.
        При этом $\deg (R_2 - R_1) < \deg B$, а значит мы пришли к противоречию и $Q_1 = Q_2$ 
        и $R_1 = R_2$. 
    \end{enumerate}
\end{proof}

\begin{definition}
    $A$ делится на $B$, если существует такой многочлен $Q$ что $A = QB$. Пишут $A \vdots B$ или 
    $B \vert A$.
\end{definition}

\begin{definition}
    Пусть $f(x)$ и $g(x) \in F[x]$ -- не нулевые одновременно многочлены. Многочлен $d(x) \in F[x]$ 
    называется наибольшим общим делителем (НОД, gcd) если:
    \begin{enumerate}
        \item $d \vert f$, $d \vert g$.
        \item если $d'$ -- общий делитель $f$ и $g$, то $d' \vert d$.
    \end{enumerate}

    Иначе говоря, $\gd(f, g)$ -- такой общий делитель, который делится на любой общий делитель.
\end{definition}

\begin{theorem}[алгоритм Евклида, линейное выражение НОД]
    Пусть $f, g \in F[x]$ и $f, g$ ненулевые одновременно. Тогда существует 
    $d(x) = \gd(f, g) \in F[x]$ и, более того, существуют 
    $u(x), v(x) \in F(x)$, такие что $u(x)f(x) + v(x)g(x) = d(x)$.
\end{theorem}

\begin{proof}
    Пусть без ограничения общности $f(x) = 0$, $g(x) \neq 0$. Тогда $d(x) = g(x)$, $d = 0\cdot f + 1\cdot g$.

    Пусть теперь оба многочлена ненулевые. Тогда можно выполнить цепочку делений многочленов, где 
    на каждом новом шаге делимым и делителем будут становиться делитель и частное предыдущего деления
    соответственно. Таким образом для каждой пары НОД будет сохраняться, так как если делитель кратен 
    некоторому многочлену, то делимое и частное будут кратны ему одновременно. Первые несколько шагов:
    \begin{align*}
        f(x)   & = q_1(x)g(x) + r_1(x), \\
        g(x)   & = q_2(x)r_1(x) + r_2(x), \\
        r_1(x) & = q_3(x)r_2(x) + r_3(x). \\
    \end{align*}
    Продолжая действовать так дойдем до последних двух шагов, после которых остаток будет равен нулю.
    При делении степень остатка меньше степени делителя, а значит, в силу конечности номеров старших 
    членов начальных многочленов, в некоторый момент процесс действительно остановится:
    \begin{align*}
        r_{n-2}(x) & = q_n(x)r_{n-1}(x) + r_n(x), \\
        r_{n-1}(x) & = q_{n+1}(x)r_{n}(x).
    \end{align*}
    Получается, что $\gd(f, g)$ = $r_n$ -- последний ненулевой остаток. Проверим:
    \begin{enumerate}
        \item $r_n \vert r_{n-1}$, $r_n \vert r_{n-2}, \ldots$. 
        Продолжая подниматься наверх, получаем $r_n \vert f$, $r_n \vert g$
        \item Теперь будем спускаться вниз, пусть $d' \vert f$, $d' \vert g$. 
        Таким образом мы дойдем до $d' \vert d$.
    \end{enumerate}
    Покажем, что все остатки $r_1, r_2, \ldots, r_n$ являются линейными комбинациями $f$ и $g$:
    $$r_1 = f - q_1g$$
    $$r_2 = g - q_2r_1 = -q_2f + (1 + q_1q_2)g$$
    Спускаясь вниз и подставляя выражения предыдущих остатков в последующие, получим все разложения.
    Положим $r_{n-2} = u''f + v''g$ и $r_{n-1} = u'f + v'g$. Тогда:
    $$d = r_n = r_{n-2} - q_n r_{n-1} = f(u'' - u'q_n) + g(v'' - v'g_n).$$
    Таким образом, все остатки можно выразить через $f$ и $g$.
\end{proof}
