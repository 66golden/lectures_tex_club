\section{3. Корни многочленов. Теорема Безу. Формальная производная. Кратные корни.}

\begin{definition}
    Элемент поля $F$ является \textit{корнем} многочлена $f$, если $f(c) = 0$.
\end{definition}

\begin{theorem}[Безу]
	Скаляр $a \in F$ является корнем многочлена $P \in F[x]$ $\Leftrightarrow$ $(x - a)\mid P$.
\end{theorem}

\begin{proof}
	Разделим $P$ с остатком на $(x - a)$, то есть выберем $Q, R \in F[x]$ такие, что $P = Q(x - a) \hm{+} R$ и $\deg{R} \ge 0$. Заметим, что $P(a) = R$, тогда выполнены равносильности $P(a) = 0 \Leftrightarrow R = 0 \hm{\Leftrightarrow} (x - a)\mid P$.
\end{proof}

\begin{definition}
    \textit{Формальной производной} многочлена $x^n$ называется $\frac{d}{dx} x^n = n \cdot x^{n-1}$, так же используется обозначение $(x^n)'$. Распространим $\frac{d}{dx}$ на остальные векторы $F[x]$ по линейности. Тогда дифференцирование является линейным опреатором: $\frac{d}{dx}: F[x] \to F[x]$.
\end{definition}

\begin{proposition}
    Формальная производная $\frac{d}{dx}$ удовлетворяет правилу Лейбница: 
    $$(f \cdot g)' = f' \cdot g + f \cdot g'.$$
\end{proposition}

\begin{proof}
    Обе части являются линейными по многочленам $f$ и $g$, поэтому достаточно доказать правило для базисных векторов. Рассмотрим $f = x^m$ и $g = x^l$ -- базисные вектора в $F[x]$. Тогда $(x^m \cdot x^l)' = (x^{m+l})' = (m+l) \cdot x^{m+l-1}$. Так же можно продифференцировать $f$ и $g$ по отдельности: $(x^m)' = m \cdot x^{m-1}$ и $(x^l)' = l \cdot x^{l-1}$. Отсюда очевидно, что равенство действительно выполняется.
\end{proof}

\begin{definition}
    Пусть задан многочлен $f \in F[x]$. Корень $c \in F$ называется корнем многочлена $f$ кратности $k$ ($k\in \N$) если $f(x)$ кратно $(x-c)^k$, но $f(x)$ не кратно $(x-c)^{k+1}$.
\end{definition}

\begin{theorem}[о кратности корня]
    Пусть $F$ -- поле, $f \in F[x]$, $c \in F$ -- корень многочлена $f$. Тогда верно следующее:
    \begin{enumerate}
        \item c -- кратный корень f $\Leftrightarrow$ $f(c) = 0$ и $f'(c) = 0$.
        \item с -- корень кратности $R$ $\Rightarrow$ $f(c) = 0$, $f'(c) = 0$, $\dots$, $f^{(R-1)}(c) = 0$.
    \end{enumerate}
\end{theorem}

\begin{proof}~
    \begin{enumerate}
        \item \begin{enumerate}
            \item Необходимость.
            
            По условию $f(x) = q(x) (x-c)$. Продифференцируем $f$:
            $$f'(x) = q'(x) (x-c) + q(x).$$ 
            Тогда $f'(c) = q(c)$. При этом многочлен $q(x)$ кратен $(x-c)$ в силу того, что $c$ - кратный корень $f$. 
            Таким образом вся производная $f'$ кратна $(x-c)$. 
            
            \item Достаточность.
            
            Пусть $f(c) = f'(c) = 0$, тогда $q(c) = 0$, а значит $q(x)$ кратен $(x-c)$.
            
        \end{enumerate}

        \item Пусть $c$ -- корень кратности $R$. Тогда многочлен $f$ представим в виде $f = q(x) (x-c)^R$, где $q(c) \neq 0$.
        Возьмем производную от $f$:
        $$f'(x) = q'(x) (x-c)^R + R \cdot q(x) (x-c)^{R-1}.$$ 
        Продолжим брать производные. Тогда для 
        k-производной кратность корня $c$ не меньше $R-k$.
    \end{enumerate}
\end{proof}
