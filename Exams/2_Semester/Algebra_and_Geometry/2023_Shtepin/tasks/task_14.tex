\section{14. Билинейные функции. Координатная запись билинейной функции. Матрица билинейной функции и ее изменение при замене базиса. Ортогональное дополнение к подпространству относительно симметричной (кососимметричной) билинейной функции и его свойства.}

\begin{definition}
    Пусть $V$ -- линейное пространство над $F$. Функция $f: V \times V \to F$ называется \textit{билинейной}, если выполняются следующие условия:
    \begin{enumerate}
        \item Аддитивность по первому аргументу $f(x_1 + x_2, y) = f(x_1, y) + f(x_2, y)$.
        \item Линейность по первому аргументу $f(\lambda x, y) = \lambda f(x, y)$.
        \item Аддитивность по второму аргументу.
        \item Линейность по второму аргументу.
    \end{enumerate}
\end{definition}

\begin{definition}
    Если $x, y \in F^n$, то выражение $\displaystyle\sum_{i=1}^{n} \displaystyle\sum_{j=1}^{m} a_{ij} x_i y_j$ называется \textit{билинейной формой} от координатных столбцов $x$ и $y$. Билинейная форма сама является билинейной функцией: $F^n \times F^n \to F$.
\end{definition}

\begin{proposition}
    Если $f(x, y)$ -- билинейная функция $V \times V \to F$, то она может быть записана в виде билинейной формы от координат $x$ и $y$ при добавлении коэффициентов $a_{ij} = f(e_i, e_j)$ - значения функции $f$ на базисных векторах. 
\end{proposition}

\begin{proof}
    Пусть $f(x, y)$ - билинейная функция, $e = (e_1, \dots e_n)$ - базис в $V$. Запишем разложения векторов $x$ и $y$ по базису:
    \begin{align*}
        x = \sum_{i=1}^{n} x_i e_i && y = \sum_{j=1}^{n} y_j e_j
    \end{align*}
    Тогда верно следующее:
    \begin{gather*}
        f(x, y) = f(\sum_{i=1}^{n} x_i e_i, \sum_{j=1}^{n} y_j e_j) = \sum_{i=1}^{n} \sum_{j=1}^{n} x_i y_j f(e_i, e_j) = \sum_{i=1}^{n} \sum_{j=1}^{n} a_{ij} x_i y_j
    \end{gather*}
\end{proof}

\begin{proposition}
    Пусть $f(x, y)$ -- билинейная функция в $V$. $e$, $e'$ -- базисы в $V$. $A$, $A'$ -- матрицы билинейной формы $f$ в этих базисах. Тогда $A' = S^T A S$, где $S$ -- матрица перехода между $e$ и $e'$. 
\end{proposition}

\begin{proof}
    Пусть $x$ и $y$ имеют в $e$ координаты $\alpha$ и $\beta$ соответственно.\\ Было доказано, что $\alpha = S \alpha'$, $\alpha` = s^{-1} \alpha$, $\beta = S \beta'$. Тогда:
    $$f(x, y) = x^T A y = \alpha^T A \beta = (S \alpha')^T A (S \beta) = (\alpha')^T S^T A S \beta' = (\alpha')^T A' \beta'.$$ Из последнего равенства сразу следует, что $S^T A S = A'$.
\end{proof}

\begin{definition}
    Билинейная функция $f(x, y)$ называется \textit{симметричной} если для всех $x, y \in V$ верно 
    $f(x, y) = f(y, x)$.
\end{definition}

\begin{definition}
    Билинейная функция $f(x, y)$ называется \textit{кососимметричной}, если для всех $x, y \in V$ верно:
    \begin{enumerate}
        \item $f(x, y) = -f(y, x)$,
        \item $f(x, x) = 0$.
    \end{enumerate}
\end{definition}

\begin{agreement}
    Многие утверждения, доказываемые в этом разделе верны и для симметричных и для кососимметричных 
    функций. Чтобы показать, что функция $f$ лежит в $B^+$ или в $B^-$ будем использовать 
    обозначение $f \in B^{\pm}$.
\end{agreement}

\begin{definition}
    \label{def8.5}
    Пусть $f \in B^{\pm}(V)$. Тогда ядром $f$ является: 
    $$\ker f = \{x \in V \vert \, \forall y \in V \hookrightarrow f(x, y) = 0\} = 
    \{y \in V \vert \, \forall x \in V \hookrightarrow f(x, y) = 0\},$$ 
    где сначала выписано левое ядро, а затем правое ядро, и они равны.
\end{definition}

\begin{definition}
    Пусть $U \leq V$. \textit{Ортогональным дополнением} к $U$ относительно функции $f \in B^{\pm}(V)$ 
    называется подпространство $U^{\perp} = \{y \in V \,\vert \, \forall x \in U \hookrightarrow 
    f(x, y) = 0\}$.
\end{definition}

\begin{definition}
    Подпространство $U \leq V$ назовем невырожденным относительно функции $f \in B^{\pm}(V)$, если 
    сужение $f$ на $U$ невырожденно.
\end{definition}

\begin{theorem}
    \label{th8.1}
    Пусть $U \leq V$, $f \in B^{\pm}(V)$. Тогда $U$ невырожденно относительно $f$ тогда и только 
    тогда когда $V$ раскладывается в прямую сумму подпространств: $V = U \oplus U^{\perp}$.
\end{theorem}

\begin{proof}~
    \begin{enumerate}
        \item Необходимость. Пусть $f \vert_{U}$ невырождено. Покажем, что тогда 
        $\ker f \vert_{U} = \{0\}$.
        \begin{multline*}
            \ker f \vert_{U} = \{y \in U \vert \, \forall x \in U \hookrightarrow f(x, y) = 0\} = \\
            = \{y \in V \vert \, \forall x \in U \hookrightarrow f(x, y) = 0\} \cap U 
            = U^{\perp} \cap U = \{0\},
        \end{multline*}
        где первое равенство получено по определению ядра $f$ над $U$, а третье по определению 
        ортогонального дополнения. Из соображений размерностей подпространств получим:
        \begin{multline*}
            dim (U + U^{\perp}) = \dim U + \dim U^{\perp} - \dim (U \cap U^{\perp}) = \\ =
            \dim U + \dim U^{\perp} \geq \dim U + \dim V - \dim U = \dim V.
        \end{multline*} 
        Так как $U + U^{\perp} \leq V$, получаем равенство размерностей $\dim (U + U^{\perp}) = \dim V$, 
        а значит и равенство подпространств: $U + U^{\perp} = V$. 
        
        По теореме о характеристике прямой суммы получаем $V = U \oplus U^{\perp}$.

        \item Пусть $V = U \oplus U^{\perp}$. Но $\ker (f \vert_{U}) = U \cap U^{\perp} = \{0\}$, 
        а значит $f$ невырождена на $U$.
    \end{enumerate}
\end{proof}
