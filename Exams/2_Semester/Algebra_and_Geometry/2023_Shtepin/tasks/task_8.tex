\section{8. Корневое подпространство линейного оператора. Свойства корневых подпространств. Разложение пространства в прямую сумму корневых подпространств (случай, когда характеристический многочлен линейного оператора раскладывается на линейные множители).}

\begin{definition}
    $\phi : V \to V$, $V$ -- линейное пространство над полем $F$, $\lambda \in f$.
    Вектор $x \in V$ называют \textit{корневым} для $\phi$, отвечающим $\lambda \in F$, если 
    $\exists k \in \N \; : (\phi - \lambda \epsilon)^{k}x = 0$.
\end{definition}

\begin{definition}
    Число $k$ -- \textit{высота корневого вектора} $x$, отвечающего $\lambda$ из $F$, если $k$ -- наименьшее 
    число такое, что $(\phi - \lambda \epsilon)^k x = 0$. 
    Будем считать, что нулевой вектор имеет высоту 0.
\end{definition}

\begin{definition}
    Подпространство $ V^{\lambda}$ -- множество всех корневых векторов для $\phi$, относящихся к $\lambda$. $ V^{\lambda}$ называется \textit{корневым подпространством} для оператора $\phi$ относящегося к $\lambda$.
\end{definition}

\begin{definition}
    Подпространство $W$ называется \textit{дополнительным} к $V^{\lambda}$, 
    если их пересечение состоит только из нуля.
\end{definition}

\begin{theorem}[о свойствах корневых подпространств]~
    \label{th5.1}

    Пусть $V^{\lambda}$ -- корневое подпространство для $\phi$ отвечающее $\lambda$. Тогда
    \begin{enumerate}
        \item $V^{\lambda}$ инвариантно относительно $\phi$.
        \item Подпространство $V^{\lambda}$ имеет единственное собственное значение $\lambda$.
        \item Если $W$ -- тоже инвариантное относительно $\phi$ подпространство, при этом являющееся 
        дополнительным к $V^{\lambda}$, то на $W$ оператор 
        $\phi - \lambda \epsilon$ действует невырожденным образом.
    \end{enumerate} 
\end{theorem}

\begin{proof}~
    \begin{enumerate}
        \item Пусть m -- максимальная высота векторов $x \in V^{\lambda}$, в силу конечномерности 
              $V^{\lambda}$ такая существует и является конечным числом.
              Тогда $V^{\lambda} = \ker (\phi - \lambda \epsilon)^m$. 

              Операторы $\phi$ и $\epsilon$
              коммутируют с $\phi$, а значит и оператор $(\phi - \lambda \epsilon)^m$ коммутирует с 
              $\phi$. Таким образом, можно записать 
              $(\phi - \lambda \epsilon)^m \phi = \phi (\phi - \lambda \epsilon)^m$. 
              Тогда по теореме о коммутирующих линейных операторах получаем, что $\ker (\phi - \lambda \epsilon)^m$ 
              инвариантно относительно $\phi$. 
        \item Докажем от противного, пусть в $V^{\lambda}$ найдется ненулевой собственный вектор $x$ 
              с собственным значением $\mu \neq \lambda$, то есть $\phi(x) = \mu x$. Применим к этому 
              вектору оператор $(\phi - \lambda \epsilon)$:  
              $$(\phi - \lambda \epsilon) x = \phi(x) - (\lambda \epsilon)(x) = (\mu - \lambda) x.$$ 
              Тогда при многократном применении 
              получим $(\phi - \lambda \epsilon)^m x = (\mu - \lambda)^m x = 0$,
              так как $x \in V^{\lambda}$ и должен аннулироваться. Тогда $\mu - \lambda = 0$, 
              что дает противоречие.
        \item По условию $V$ представляется как $V = V^{\lambda} \oplus W$. При этом подпространства 
              $V^{\lambda}$ и $W$ инвариантны относительно $\phi$, а значит, согласно утверждению 
              \ref{prop4.2}, они так же инвариантны относительно $(\phi - \lambda \epsilon)$. 
              Нам нужно доказать, что $\phi - \lambda \epsilon$ невырожден на $W$, то есть что 
              $\ker (\phi - \lambda \epsilon) \vert_{W} = \{0\}$.

              Докажем от противного, пусть $\exists x \neq 0$ такое что 
              $x \in \ker (\phi - \lambda \epsilon) \vert_{W}$. 
              Отсюда следует, что вектор $x$ лежит в пространстве $W$, так как лежит в ядре сужения
              оператора на это подпространство.

              Однако $(\phi - \lambda \epsilon) x = 0$, а значит х -- собственный для $\phi$ 
              с собственным значением $\lambda$. Тогда вектор $x$ так же лежит и в пространстве 
              $V^{\lambda}$, что приводит к 
              противоречию с тем, что по условию $V^{\lambda} \cap W = \{0\}$.
    \end{enumerate}
\end{proof}

\begin{corollary}
    Корневое подпространство $V^{\lambda}$ -- максимальное по включению инвариантное подпространство, 
    на котором $\phi$ имеет единственное собственное значение $\lambda$.
\end{corollary}

\begin{theorem}[о разложении пространства V в прямую сумму корневых]~
    \label{th5.2}

    Пусть $\phi \in \mathcal{L}$, $\phi$ -- линейно факторизуем над $F$.
    Тогда пространство $V$ есть прямая сумма корневых подпространств: 
    $V = V^{\lambda_1} \oplus V^{\lambda_2} \oplus \ldots \oplus V^{\lambda_k}$, где все $\lambda_i$ попарно различны.
\end{theorem}

\begin{proof}
    По условию $\phi$ линейно факторизуем, а значит
    $\chi_{\phi}(t) = \displaystyle\prod_{i= 1}^{k} (\lambda_i - t)^{m_i}$. Многочлены $(\lambda_i - t)^{m_i}$ попарно взаимно просты 
    из попарной различности $\lambda_i$, поэтому по следствию из теоремы о взаимно простых делителях аннулирующего многочлена \footnote{Пусть $\phi \in \mathcal{L}(V)$, $f$ -- аннулирующий многочлен для $\phi$, такой что $f$ 
    раскладывается в произведение $f = f_1 \cdot f_2 \dots f_n$ попарно взаимно-простых многочленов.
    Тогда $V$ раскладывется в прямую сумму $V = V_1 \oplus V_2 \oplus \dots V_n$, 
    где $V_i = \ker f_i(\phi)$ -- инвариантные подпространства.} можно заключить:
    $$V = \ker (\phi - \lambda_1 \epsilon)^{m_1} \oplus \ker (\phi - \lambda_2 \epsilon)^{m_2} 
    \oplus \dots \oplus \ker (\phi - \lambda_k \epsilon)^{m_k}.$$
    При этом $\ker (\phi - \lambda_i \epsilon)^{m_i} \subseteq V^{\lambda_i}$ для всех $i$,
    а значит вектор $x \in V$ представим в виде суммы 
    $x = x_1 + \ldots + x_k$, где $x_i \in V^{\lambda_i}$.
    Отсюда очевидно, что пространство $V$ является суммой подпространств: 
    $$V = V^{\lambda_1} + V^{\lambda_2} + \dots V^{\lambda_k}.$$ 
    
    Осталось доказать что $V^{\lambda_i} \subseteq \ker(\phi - \lambda_i \epsilon)^{m_i}$, 
    в таком случае сумма будет прямой. Докажем от противного, пусть существует индекс $i$ такой, 
    что $\ker (\phi - \lambda_i \epsilon) \leq V^{\lambda_i}$. Тогда найдется вектор 
    $x \in V^{\lambda_i}$ такой, что он не лежит в ядре. Обозначим высоту $x$ за $M > m_i$, тогда:
    $$\chi_{\phi}(\phi) x = \left(\displaystyle\prod_{j \neq i} (\phi - \lambda_j \epsilon)^{m_j}\right) \cdot 
    (\phi - \lambda_i \epsilon)^{m_i} x = \displaystyle\prod_{j \neq i} 
    ((\phi - \lambda_j \epsilon)^{m_j})x' \neq 0.$$ 
    Если найдется такой $j$ что $(\phi - \lambda_j \epsilon)x = 0$, то у $x'$ есть собственное значение
    $\lambda_j$, что приводит к противоречию с пунктом 2 теоремы \ref{th5.1}. 
    В противном случае возникает противоречие с 
    $\chi_{\phi}(\phi) = 0$ по теореме \ref{th4.4} (Гамильтона-Кэли). 
    Таким образом, $V^{\lambda_i} = \ker(\phi - \lambda_i \epsilon)^{m_i}$, а значит $V$ представляется 
    в виде прямой суммы $V^{\lambda_i}$:

    $$V = V^{\lambda_1} \oplus V^{\lambda_2} \oplus \, \dots \, \oplus V^{\lambda_k}.$$
\end{proof}
