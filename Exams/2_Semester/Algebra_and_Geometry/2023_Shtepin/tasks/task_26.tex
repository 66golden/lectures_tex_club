\section{26. Приведение квадратичной формы в пространстве со скалярным произведением к главным осям. Одновременное приведение пары квадратичных форм к диагональному виду.}

\begin{theorem}[о приведении к главным осям]
	Пусть $V$ "--- евклидово (эрмитово) пространство,  $q \in \mathcal{Q}(V)$. Тогда в $V$ существует ортонормированный базис $e$, в котором $q$ имеет диагональный вид.
\end{theorem}

\begin{proof}
	Пусть $b \in \mathcal{B}^+(V)$ "--- $\theta$-линейная форма, полярная к $q$. Тогда $\exists \phi \in \mathcal{L}(V)$ такой, что $b(\overline{u}, \overline{v}) = (\phi(\overline{u}), \overline{v})$. При этом:
	\[(\phi(\overline{u}), \overline{v}) = b(\overline{u}, \overline{v}) = \overline{b(\overline{v}, \overline{u})} = \overline{(\phi(\overline{v}), \overline{u})} = (\overline{u}, \phi(\overline{v}))\]
	
	Значит, $\phi$ "--- самосопряженный, и в $V$ существует ортонормированный базис $e$, в котором $\phi$ диагонализуем. Тогда если $\phi \leftrightarrow_e A$, то $b \leftrightarrow_e A^T$ и $q \leftrightarrow_e A^T$, поэтому форма $q$ тоже имеет диагональную матрицу в базисе $e$.
\end{proof}

\begin{note}
	Напротив, если в ортонормированном базисе $e$ матрица формы $q$ диагональна, то и матрица оператора $\phi$ диагональна и, следовательно, задана однозначно собственными значениями $\phi$. Значит, диагональный вид $q$ в ортонормированном базисе определен однозначно.
\end{note}

\begin{theorem}
	Пусть $V$ "--- линейное пространство над $\mathbb{R}$ (над $\mathbb{C}$), $q_1, q_2 \hm{\in} \mathcal{Q}(V)$, и $q_2$ положительно определена. Тогда в $V$ существует такой базис $e$, в котором матрицы форм $q_1$ и $q_2$ диагональны.
\end{theorem}

\begin{proof}
	Пусть $b$ "--- $\theta$-линейная форма, полярная к $q_2$. Тогда $b$  можно объявить $b$ скалярным (эрмитовым скалярным) произведением на $V$. В полученном евклидовом (эрмитовом) пространстве форма $q_1$ приводится к главным осям в некотором ортонормированном базисе $e$. Поскольку базис $e$ "--- ортонормированный, то в этом же базисе $q_2$ имеет диагональный вид $E$.
\end{proof}

\begin{note}
	Требование положительной определенности в теореме существенно.
\end{note}