\section{5. Инвариантные подпространства. Собственные векторы и собственные значения. Характеристический многочлен и его свойства. Инвариантность следа и определителя матрицы оператора.}

\begin{definition}
    Пусть $V$ -- линейное пространство, $\phi: V \to V$. Подпространство $U \leq V$ называется 
    \textit{инвариантным}, если выполняется $\forall x \in U \; \phi(x) \in U$. Другими словами, действие 
    оператора $\phi$ на вектор из $U$ не выводит его за пределы $U$, а значит 
    $\phi(U) \subset U \Leftrightarrow \phi(U) \leq U$.
\end{definition}

\begin{proposition}
    Пусть $\phi: V \to V$ -- линейный оператор, $U$ -- инвариантное подпространство. Тогда в базисе, 
    согласованном с $U$, оператор $\phi$ имеет матрицу с левым нижним углом нулей:
    \[\phi(A) = \left(\begin{array}{@{}c|c@{}}
		A & B\\
		\hline
		0 & C
	\end{array}\right)\]
    Здесь $A \in M_k(F)$, $k = \dim U$.
\end{proposition}

\begin{proof}
    $U$ инвариантно относительно $\phi$, а значит $\phi(e_1), \phi(e_2), \dots \phi(e_k) \in U$.
    Тогда для базисного вектора из $U$ ненулевыми могут быть только первые $k$ элементов 
    соответствующего ему столбца.
\end{proof}

\begin{note}
    Блок нулей в левом нижнем углу означает, что $\phi(e_1), \phi(e_2), \dots \phi(e_k) \in U$, 
    а значит подпространство $U$ является инвариантным относительно $\phi$.
\end{note}

\begin{definition}
    Пусть $\phi: V \to V$. Ненулевой вектор
    $x \in V: \phi(x) = \lambda x$ называется \textit{собственным вектором} оператора $\phi$, 
    отвечающим собственному значению $\lambda$.
\end{definition}

\begin{definition}
    Число $\lambda \in F$ называется \textit{собственным значением} оператора $\phi$, если 
    $\exists x \in V: x \neq 0,\; \phi(x) = \lambda x$,
    то есть если некоторое $x$ отвечает $\lambda$.
\end{definition}

\begin{definition}
    \textit{Характеристическим многочленом} оператора $\phi: V \to V$ называется определитель $|A - \lambda E| = \chi_A(\lambda)$, где $A$ -- матрица $\phi$ в произвольном базисе.
\end{definition}

\begin{theorem}
    Верны следующие свойства характеристического многочлена:
    \begin{enumerate}
        \item Корни $\chi(\lambda)$ принадлежащие полю $F$ и только они являются собственными 
        значениями $\phi$.
        \item Многочлен $\chi(\lambda)$ не зависит от выбора базиса.
    \end{enumerate} 
\end{theorem}

\begin{proof}
    \begin{enumerate}
        \item Пусть $\lambda_0$ -- корень $\chi_{\phi}(\lambda) \lra \chi_{\phi}(\lambda_0) = 0 \lra \det(A - \lambda_{0}E) = 0 \lra $ система $A - \lambda_{0}E$ имеет ненулевое решение при $x_0 \neq 0$ $\lra \phi x_{0} = \lambda_0 x_0 \lra \lambda_0$ -- собственное значение $\phi$. В силу равносильных переходов, обратное утверждение тоже верно. 
        \item Наряду с $e$ выберем базис $f$, обозначим за $S$ матрицу перехода между ними: 
        $S = S_{e \to f}$. Тогда $\phi \xleftrightarrow[e]{} A$, $\phi \xleftrightarrow[f]{} B$, 
        $B = S^{-1}AS$. Верна следующая цепочка равенств:
        \begin{eqnarray}
            \chi_b(\lambda) = |B - \lambda E| = |S^{-1}AS - \lambda E| 
            = |S^{-1}AS - S^{-1} \lambda E S| = \\ = |S^{-1}(A - \lambda E)S| 
            = |S^{-1}| \cdot |A - \lambda E| \cdot |S| = |A - \lambda E| = \chi_a(\lambda).
        \end{eqnarray}
        Таким образом, характеристический многочлен одинаков для всех базисов.
    \end{enumerate}
\end{proof}

\begin{corollary}
    От выбора базиса не зависят так же коэффициенты характеристического многочлена, в частности $det A$ и $tr A$, поэтому часто пишут $det \phi$ и $tr \phi$ соответственно.
\end{corollary}