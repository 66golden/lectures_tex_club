\section{18. Полуторалинейные формы в комплексном линейном пространстве. Эрмитовы полуторалинейные и квадратичные формы, связь мужду ними. Приведение их к каноническому виду. Закон инерции для эрмитовых квадратичных форм. Критерий Сильвестра.}

\begin{definition}
    Если рассматривать $V$ над $\Cm$, то в $V$ не бывает положительных функций в привычном нам виде.
    Для сравнения функции с $0$ на комплексных значениях будем считать, что если  $q(x) > 0$, то 
    $q(ix) = f(ix, ix) = -f(x,x) = -q(x) < 0$.
\end{definition}

\begin{definition}
    \textit{Полуторалинейными функциями} будем называть такие $f: V \times V \to \Cm$, для которых верны: 
    \begin{enumerate}
        \item Аддитивность по первому аргументу: $f(x_1 + x_2, y) = f(x_1, y) + f(x_2, y)$,
        \item Однородность по первому аргументу: $f(\lambda x, y) = \lambda f(x, y)$ для всех $\lambda \in \Cm$,
        \item Аддитивность по второму аргументу: $f(x, y_1 + y_2) = f(x, y_1) + f(x, y_2)$.
        \item $f(x, \lambda y) = \overline{\lambda} f(x, y)$.
    \end{enumerate}
\end{definition}

\begin{definition}
    Пусть $f$ -- полуторалинейная функция на $V$, $e$ -- базис в $V$, и векторы $x, y \in V$ имеют 
    координаты $x \leftrightarrow (x_1, x_2, \dots x_n)^T$, $y \leftrightarrow (y_1, y_2, \dots y_n)$.
    \textit{Полуторалинейной формой} от $x$, $y$ называют:
    \begin{gather*}
        f(x, y) = \sum_{i=1}^{n} \sum_{j=1}^{n} a_{ij} x_i \overline{y_j} = x^T A \overline{y}.
    \end{gather*} 
\end{definition}

\begin{proposition}
    Пусть $f$ -- полуторалинейная функция в $V$, $e$, $f$ -- базисы в $V$, $S$ -- матрица перехода 
    $S = S_{e \to f}$ и функция $f$ представляется в базисах $V$ матрицами
    $f \underset{e}{\leftrightarrow} A$, $f  \underset{f}{\leftrightarrow} B$, то $B = S^T A \overline{S}$.
\end{proposition}

\begin{proof}
    В базисе $e$ функция $f$ выражается как $f(x, y) = x^T A \overline{y}$. При переходе к базису 
    $f$ получим $x = S x'$, $y =  S y'$. Тогда:
    $$f(x, y) = (Sx')^T A \overline{(Sy')} = (x')^T S^TA \overline{S} \overline{y'} = (x')^TB\overline{y'},$$
    откуда $B = S^TA\overline{S}$.
\end{proof}

\begin{definition}
    Полуторалинейная функция $f(x, y)$ называется \textit{эрмитовой} (или \textit{эрмитово-симметричной}) если для всех 
    $x, y \in V$ верно $f(x, y) = \overline{f(y, x)}$. Матрица называется эрмитово-симметричной если 
    $A = \overline{A^T}$. 
\end{definition}

\begin{note}
    Комплексное сопряжение $\overline{A}$ к матрице $A$ стоит воспринимать как замену всех её элементов 
    на комплексно-сопряженные к ним.
\end{note}

\begin{proposition}
    Полуторалинейная функция $f$ эрмитова тогда и только тогда, когда в произвольном базисе $e$ её 
    матрица эрмитова.
\end{proposition}

\begin{proof}~
    \begin{enumerate}
        \item Необходимость. Пусть $f$ эрмитова. Тогда верно: \begin{gather*}
            a_{ij} = f(e_i, e_j) = \overline{f(e_j, e_i)} = \overline{a_{ji}}.
        \end{gather*} Отсюда следует $A = \overline{A^T}$.
        \item Достаточность. Пусть $A = \overline{A^T}$, откуда $A^T = \overline{A}$.
        Тогда: 
        \begin{gather*}
            f(x, y) = (x^T A \overline{y}) = (x^T A \overline{y})^T = \overline{y^T} A^T x = 
            \overline{y^T} A^T \overline{\overline{x}} = \overline{y^T A \overline{x}} = \overline{f(y, x)}.
        \end{gather*}
    \end{enumerate}
\end{proof}

\begin{definition}
    Пусть $\Delta = \{(x, x) \vert x \in V\}$ -- диагональ декартового квадрата. Тогда функция $q: V \to \Cm$ 
    назвается эрмитовой квадратичной функцией $q(x) = f(x, x) = f \vert_{\Delta}$, где $f$ -- эрмитова 
    симметричная функция.
\end{definition}

\begin{theorem}[О существовании канонического базиса]~

    Пусть $q$ -- эрмитова квадратичная функция (или соответствующая ей эрмитова симметричная функция $f$).
    Тогда в $V$ существует базис $e$, в котором матрица $q(f)$ диагональна, причем на главной диагонали 
    стоят числа $\pm 1$ и $0$.
\end{theorem}

\begin{idea}
    Пусть $q \neq 0$. Тогда в $V$ существует такой ненулевой вектор $e_1$, что $q(e_1) \neq 0$.
    Без ограничения общности можно перейти к $q(e_1) = \pm 1$. \\ Тогда можно рассмотреть пространство 
    $U = \langle e_1 \rangle$ и ортогональное дополнение к нему, образующие прямую сумму.
\end{idea}

\begin{proposition}[Закон инерции для квадратичных эрмитовых функций]
    Пусть $e$ - произвольный канонический базис для $q(x)$ и пусть $p, q$ - положительнй и отрицательный
    индексы инерции относительно $e$. Тогда:
    \begin{enumerate}
        \item $p = \max \{\dim U | U \leq V: q \vert_{U} \text{ -- положительно опеределена}\}$.
        \item $q = \max \{\dim U | U \leq V: q \vert_{U} \text{ -- отрицательно опеределена}\}$.
        \item $p$ и $q$ не зависят от выбора канонического базиса.
    \end{enumerate}
\end{proposition}

\begin{proof}
    Доказательство аналогично билинейному случаю.
\end{proof}

\begin{proposition}[Аналог критерия Сильвестра]
    Пусть $q(x) \in H(V)$ -- эрмитова квадратичная функция, $A$ -- её матрица в произвольном базисе, где 
    выполняется условие эрмитовости $\tilde{A^T} = A$. Тогда:
    \begin{enumerate}
        \item $q(x)$ положительно определена тогда и только тогда, когда $\Delta_1 > 0$, $\Delta_2 > 0$, $\dots$, 
        $\Delta_n > 0$.
        \item $q(x)$ отрицательно определена тогда и только тогда, когда $\Delta_1 < 0$, $\Delta_2 > 0$, $\dots$, 
        $\sgn (\Delta_n) = (-1)^n$.
    \end{enumerate}
\end{proposition}

\begin{proof}
    Доказательство аналогично билинейному случаю.
\end{proof}