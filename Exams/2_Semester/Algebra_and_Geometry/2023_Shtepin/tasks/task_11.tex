\section{11. Аннулирующий и минимальный многочлен линейного оператора. Связь минимального многочлена с жордановой нормальной формой.}

\begin{definition}
    $\phi: V \to V$, $P \in F[t]$ называется \textit{аннулирующим} для оператора $\phi$, если $P(\phi) = 0$ (иначе говоря: $\forall x \in V \hookrightarrow P(\phi) = \overline{0}$).
\end{definition}

\begin{note}
    Если $\dim V = n$, то у любого $\phi$ существует аннулирующий многочлен.
\end{note}

\begin{proof}
    Если $\phi$ соответствует матрица $A$ размером $n$ на $n$ и $\dim M_n(F) = n^2$. \\
    Тогда если рассмотреть все матрицы вида $E, A, A^2, \dots, A^{n^2}$, то существуют $\alpha_i \in F: \sum_{i = 0}^{n^2} \alpha_i A^i = 0$, тогда аннулирующий многочлен выглядит как $P = \sum_{i = 0}^{n^2} \alpha_i t^i$.
\end{proof}

\begin{definition}
    Аннулирующий многочлен для $\phi$ минимальной возможной степени называется \textit{минимальным многочленом} оператора $\phi$ и обозначается: $\mu_{\phi}$.
\end{definition}

\begin{theorem}
    \label{th4.5}
    Пусть $\phi: V \to V$, $\mu(t)$ -- минимальный многочлен $\phi$ и пусть $P(t)$ -- аннулирующий многочлен оператора $\phi$. Тогда $P \vdots \mu$.
\end{theorem}

\begin{proof}
    Пусть $P(t) = Q(t) \cdot \mu(t) + R(t)$, $\deg R < \deg \mu$ или $R = 0$. \\
    От противного, пусть $R \neq 0$ тогда выразим этот остаток из предыдущего выражения: 
    $R(\phi) = P(\phi) - Q(\phi) \cdot \mu (\phi) = 0$ -- так как аннулирующий и минимальный 
    многочлены зануляются, то и остаток равен нулю. Противоречие. Значит, $\mu(t) \vert P(t)$.
\end{proof}

\begin{corollary}
    Минимальный многочлен линейного оператора $\phi$ определяется с точностью до ассоциированности.
\end{corollary}

\begin{proposition}
    Пусть матрица отображения имеет вид Жордановой клетки: $J = J_k(\lambda)$. 
    Тогда его минимальный многочлен имеет вид $\mu_j(x) = (x - \lambda)^k$.
\end{proposition}

\begin{proof}
    Так как матрица отображения имеет вид Жордановой клетки, 
    его характеристический многочлен $\chi_J(x)$ представляется как:
    $$\chi_J(x) = (\lambda - x)^k = (-1)^k (x - \lambda)^k \sim (x - \lambda)^k.$$
    По теореме \ref{th4.4} Гамильтона-Кэли $\mu_J \vert \chi_J$, значит $\mu_J (x) = (x - \lambda)^t$, $t \leq k$.
    Если $t < k$, то $(J - \lambda \epsilon)^t \neq 0$, 
    что приводит к противоречию с определением минимального многочлена $\mu_J(J) = 0$.  
    Таким образом $t$ не может быть меньше $k$, а значит $t = k$. 
\end{proof}
