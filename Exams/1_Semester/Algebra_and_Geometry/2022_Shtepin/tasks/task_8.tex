\subsection{Плоскость в пространстве, различные способы задания, их эквивалентность. Условие параллельности двух плоскостей. Направляющий вектор пересечения двух плоскостей. Пучок плоскостей.}
    
    \begin{definition}
    	Пусть $\nu \subset P_3$ "--- плоскость, $\overline{a}, \overline{b} \in V_3$ "--- неколлинеарные векторы, представители которых лежат в $\nu$, $M \in l$, и в декартовой системе координат $(O, e)$ в $P_3$ выполнены соотношения $\overline{a} \leftrightarrow_{e} \alpha$, $\overline{b} \leftrightarrow_{e} \beta$, $M \leftrightarrow_{(O, e)} (x_0, y_0, z_0)^T$, $\overline{r_0} := \overline{OM}$.
    	\begin{itemize}
    		\item \textit{Векторно-параметрическим уравнением плоскости} называется следующее семейство уравнений:
    		\[\overline{r} = \overline{r_0} + t\overline{a} + s\overline{b},~t, s \in \mathbb{R}\]
    		
    		\item \textit{Параметрическим уравнением плоскости} называется следующее семейство систем:
    		\[\left\{
    		\begin{aligned}
    			x = x_0 + t\alpha_1 + s\beta_1\\
    			y = y_0 + t\alpha_2 + s\beta_2\\
    			z = z_0 + t\alpha_3 + s\beta_3
    		\end{aligned}
    		\right.,~s, t \in \R
    		\]
    	\end{itemize}
    \end{definition}
    
    \begin{note}
    	Множество точек $X \in P_3$ таких, что $X \leftrightarrow_{(O, e)} (x, y, z)^T$, $\overline{r} := \overline{OX}$, являющихся решениями любого из уравнений прямой выше, совпадает с плоскостью $\nu$. Действительно, $X \in \nu \lra MX \parallel \nu \lra \overline{MX} \text{ компланарен системе } (\overline{a}, \overline{b}) \lra \overline{MX} \text{ выражается через } \overline{a}, \overline{b}$.
    \end{note}
    
    \begin{note}
    	Векторно-параметрическое уравнение плоскости можно также переписать в следующем виде:
    	\[(\overline{r} - \overline{r_0}, \overline{a}, \overline{b}) = 0\]
    	
    	Перепишем это уравнение, положив $\gamma := (\overline{r_0}, \overline{a}, \overline{b})$:
    	\[(\overline{r}, \overline{a}, \overline{b}) = \gamma\]
    	
    	Если расписать смешанное произведение $(\overline{r}, \overline{a}, \overline{b})$ как определитель соответствующей матрицы, можно получить еще одно уравнение плоскости, определенное ниже.
    \end{note}
    
    \begin{definition}
    	Пусть $A, B, C, D \in \R$, $A^2+B^2+C^2 \ne 0$. \textit{Общим уравнением плоскости} называется следующее уравнение:
    	\[Ax + By + Cz + D = 0\]
    \end{definition}
    
    \begin{definition}
    	\textit{Вектором нормали} плоскости $\nu \subset P_3$ называется вектор $\overline{n} \in V_3$, $\overline{n} \ne \overline 0$, представителем которого является направленный отрезок, ортогональный плоскости $\nu$.
    \end{definition}
    
    \begin{definition}
    	Пусть $\nu \subset P_3$ "--- плоскость с вектором нормали $\overline{n} \in V_3$, и пусть $M \in \nu$, $\overline{r_0} := \overline{OM}$. \textit{Нормальным уравнением плоскости} называется следующее уравнение:
    	\[(\overline{r} - \overline{r_0}, \overline{n}) = 0\]
    \end{definition}
    
    \begin{note}
    	Уравнения различного типа, задающие плоскость, эквивалентны: из каждого из них можно получить любое другое.
    \end{note}
    
    \begin{note}
    	В прямоугольной декартовой системе координат $(O, e)$ в $P_3$ нормальное уравнение плоскости преобразуется к виду $Ax+By+Cz\hm{+}D=0$, поэтому вектором нормали этой плоскости является вектор $\overline{n} \in V_3$ такой, что $\overline{n} \leftrightarrow_{e} (A, B, C)^T$.
    \end{note}
    
    \begin{proposition}
    	Пусть в декартовой системе координат $(O, e)$ в $P_3$ плоскости $\nu_1, \nu_2$ заданы общими уравнениями $A_1x+B_1y+C_1z+D_1 = 0$, $A_2x+B_2y+C_2z+D_2 = 0$. Тогда:
    	\begin{itemize}
    		\item $\nu_1 \cap \nu_2 \ne \emptyset$ и $\nu_1 \ne \nu_2 \Leftrightarrow \overline{n_1} \nparallel \overline{n_2}$
    		\item $\nu_1 \parallel \nu_2 \text{ и } \nu_1 \ne \nu_2 \Leftrightarrow \overline{n_1} \parallel \overline{n_2} \text{, но уравнения плоскостей непропорциональны}$
    		\item $\nu_1 = \nu_2 \Leftrightarrow \text{уравнения плоскостей пропорциональны}$
    	\end{itemize}
    \end{proposition}
    
    \begin{proof}~
    	\begin{itemize}
    		\item Пусть $\overline{n_1} \nparallel \overline{n_2}$, тогда без ограничения общности столбцы $(A_1, B_1)^T$ и $(A_2, B_2)^T$ непропорциональны. Рассмотрим следующую систему уравнений относительно $x$ и $y$:
    		\[\left\{
    		\begin{aligned}
    		A_1x + B_1y = -C_1z -D_1\\
    		A_2x + B_2y = -C_2z -D_2
    		\end{aligned}
    		\right.
    		\]
    		По правилу Крамера, эта система имеет единственное решение при любом $z \in \R$. Значит, плоскости имеют общие точки, но не все их точки общие, и это означает, что они пересекаются.
    		
    		\item Пусть $\overline{n_1} \parallel \overline{n_2}$ и уравнения непропорциональны. Поскольку столбцы $(A_1, B_1, C_1)^T$ и $(A_2, B_2, C_2)^T$ пропорциональны из коллинеарности векторов $\overline{n_1}$ и $\overline{n_2}$, можно без ограничения общности считать, что $A_1 = A_2$, $B_1 = B_2$, $C_1 = C_2$, но $D_1 \ne D_2$. Тогда уравнения плоскостей не имеют общих решений, откуда $\nu_1 \parallel \nu_2 \text{ и } \nu_1 \ne \nu_2$.
    		
    		\item Пусть уравнения плоскостей пропорциональны, тогда можно считать, что они совпадают. Тогда совпадают и множества их решений, то есть $\nu_1 = \nu_2$.\qedhere
    	\end{itemize}
    \end{proof}
    
    \begin{proposition}
    	Пусть в декартовой системе координат $(O, e)$ пересекающиеся плоскости $\nu_1, \nu_2$ заданы уравнениями $A_1x+B_1y+C_1z+D_1 = 0$, $A_2x+B_2y+C_2z+D_2 = 0$. Тогда направляющим вектором прямой их пересечения является вектор $\overline v \in V_3$ такой, что:
    	\[\overline{v} \leftrightarrow_{e} \left(
    	\begin{vmatrix}
    	B_1&C_1\\
    	B_2&C_2
    	\end{vmatrix}, \begin{vmatrix}
    	C_1&A_1\\
    	C_2&A_2
    	\end{vmatrix}, \begin{vmatrix}
    	A_1&B_1\\
    	A_2&B_2
    	\end{vmatrix}\right)^T\]
    \end{proposition}
    
    \begin{proof}~
    	\begin{enumerate}
    		\item Поскольку $\nu_1 \nparallel \nu_2$, то хотя бы один из определителей, указанных в координатном столбце вектора $\overline{v}$, ненулевой, откуда $\overline v \ne \overline 0$.
    		\item Заметим, что выполнено следующее равенство:
    		\[A_1\begin{vmatrix}B_1&C_1\\B_2&C_2\end{vmatrix}+
    		B_1\begin{vmatrix}C_1&A_1\\C_2&A_2\end{vmatrix}+
    		C_1\begin{vmatrix}A_1&B_1\\A_2&B_2\end{vmatrix} = 
    		\begin{vmatrix}A_1&B_1&C_1\\A_1&B_1&C_1\\A_2&B_2&C_2\end{vmatrix}\]
    		
    		Поскольку определитель матрицы не меняется при транспонировании, выполнено следующее:
    		\[\begin{vmatrix}A_1&B_1&C_1\\A_1&B_1&C_1\\A_2&B_2&C_2\end{vmatrix} = \begin{vmatrix}A_1&A_1&A_2\\B_1&B_1&B_2\\C_1&C_1&C_2\end{vmatrix} = 0\]
    		
    		Определитель в правой части равенства равен $0$, поскольку он соответствует ориентированному объему от тройки векторов, среди которых есть два одинаковых. Получено следующее равенство:
    		\[A_1\begin{vmatrix}B_1&C_1\\B_2&C_2\end{vmatrix}+
    		B_1\begin{vmatrix}C_1&A_1\\C_2&A_2\end{vmatrix}+
    		C_1\begin{vmatrix}A_1&B_1\\A_2&B_2\end{vmatrix} = 0\]
    		
    		Значит, по критерию параллельности вектора и плоскости, $\overline{v} \parallel \nu_1$.
    		\item Аналогично пункту $(2)$, выполнено $\overline{v} \parallel \nu_2$.\qedhere
    	\end{enumerate}
    \end{proof}
    
    \begin{definition}
    	\textit{Пучком плоскостей} называется либо множество всех плоскостей в $P_3$, проходящих через фиксированную прямую $l \subset P_3$, либо множество всех плоскостей, параллельных фиксированной плоскости $\nu \subset P_3$.
    \end{definition}
    
    \begin{note}
    	Любые две плоскости в $P_3$ ровно в одном пучке.
    \end{note}
    
    \begin{theorem}
    	Пусть в декартовой системе координат $(O, e)$ в $P_3$ различные плоскости $\nu_1, \nu_2$ заданы уравнениями $A_1x\hm{+}B_1y+C_1z+D_1=0$, $A_2x+B_2y+C_2z+D_2=0$. Тогда плоскость $\nu \subset P_2$ лежит в одном пучке с плоскостями $\nu_1$ и $\nu_2$ $\lra$ плоскость $\nu$ задается уравнением следующего вида при некоторых $\alpha_1, \alpha_2 \in \R$:
    	\[\alpha_1(A_1x+B_1y+C_1z+D_1) + \alpha_2(A_2x+B_2y+C_2z+D_2) = 0\]
    \end{theorem}
    
    \begin{proof}~
    	\begin{itemize}
    		\item[$\la$] Возможны два случая:
    		\begin{enumerate}
    			\item Если $\nu_1 \cap \nu_2 = l \subset P_3$, то координаты каждой точки $P$ на прямой $l$ удовлетворяют требуемому уравнению, откуда $l \subset \nu$.
    			\item Если $\nu_1 \parallel \nu_2$, то из требуемого уравнения сопутствующий вектор плоскости $\nu$ параллелен сопутствующим векторам плоскостей $\nu_1$ и $\nu_2$. В этом случае уравнение задает плоскость не при всех $\alpha_1, \alpha_2 \in \R$, но если задает, то лежащую в данном пучке.
    		\end{enumerate}
    		
    		\item[$\ra$] Возможны два случая:
    		\begin{enumerate}
    			\item Если $\nu \cap \nu_1 \cap \nu_2 = l \subset P_3$, то выберем на $\nu$ точку $Q \not\in l$, $Q \leftrightarrow_{(O, e)} (x_0, y_0, z_0)^T$. Тогда $Q$ удовлетворяет уравнению с коэффициентами $\alpha_1 := A_2x_0+B_2y_0+C_2z_0 + D_2$, $\alpha_2 := -(A_1x_0+B_1y_0+C_1z_0 + D_1)$. Хотя бы один из коэффициентов ненулевой, поскольку $Q$ лежит не более, чем на одной из плоскостей $\nu_1$, $\nu_2$. Значит, такое уравнение задает $\nu$, так как ему удовлетворяют все точки прямой $l$ и точка, не лежащая на $l$.
    			
    			\item Если $\nu \parallel \nu_1 \parallel \nu_2$, то аналогичным образом выберем любую точку $Q \in \nu$ и соответствующие коэффициенты, тогда полученное уравнение задает $\nu$ при условии, что оно задает плоскость. Но оно всегда задает плоскость, поскольку множество его решений непусто и не содержит хотя бы одну из плоскостей $\nu_1, \nu_2$.\qedhere
    		\end{enumerate}
    	\end{itemize}
    \end{proof}