\subsection{Миноры и их алгебраические дополнения. Теорема о произведении минора на его алгебраическое дополнение. Теорема Лапласа. Разложение определителя по строке, столбцу. Определитель Вандермонда. Теорема Крамера. Формула обратной матрицы.}

\begin{definition}
	Пусть $A \in M_n(F)$. \textit{Минором} порядка $k$ матрицы $A$ называется определитель некоторой ее подматрицы размера $k \times k$.
\end{definition}

\begin{note}
	Теорему о базисном миноре можно переформулировать так: ранг матрицы $A \in M_{n \times k}(F)$ равен наибольшему из порядков его ненулевых миноров.
\end{note}

\begin{definition}
	Пусть $A = (a_{ij}) \in M_n(F)$, $i, j \in \{1, \dotsc, n\}$.
	\begin{itemize}
		\item Минором, \textit{дополнительным} к элементу $a_{ij}$, называется величина $M_{ij} := \det{A'}$, где матрица $A'$ получена из $A$ удалением $i$-й строки и $j$-го столбца
		\item \textit{Алгебраическим дополнением} к элементу $a_{ij}$ называется величина $A_{ij} := (-1)^{i + j}M_{ij}$
	\end{itemize}
\end{definition}

\begin{theorem}[о произведении минора на его алгебраическое дополнение]
    Пусть $A \in M_{n \times n}(F)$. Произведение любого слагаемого минора $M_{j_{1}, ..., j_{k}}^{i_{1}, ..., i_{k}}$ на любое слагаемое его алгебраического дополнения $A_{j_{1}, ..., j_{k}}^{i_{1}, ..., i_{k}}$ является слагаемым $\det(A)$.
\end{theorem}

\begin{proof}
    Рассмотрим $A_{0} = A = \left(\begin{array}{@{}c|c|c@{}}
	* & 0 & *\\
	\hline
	0 & M & 0\\
	\hline
	* & 0 & *
	\end{array}\right)$. Покажем, что $M A = \det(A_{0})$. Применяя элементарные преобразования строк, добьемся того, что минор $M$ располагается в первых $k$ строках. Так, мы проделали $(i_{1} - 1) + (i_{2} - 2) + ... + (i_{k} - k)$ элементарных преобразований II типа. Аналогично, сделаем $(j_{1} - 1) + (j_{2} - 2) + ... + (j_{k} - k)$ элементарных преобразований II типа и получим $\det(A) = (-1)^{i_1 + ... + i_k + j_1 + ... + j_k} \det(A_{0})$, где $A' = A = \left(\begin{array}{@{}c|c@{}}
    		M & 0\\
    		\hline
    		0 & \overline{M}
    	\end{array}\right)$. Тогда $\det(A') = M \overline{M}$, $\det(A_0) = M (-1)^{S_{m}} \overline{M} = M A$. Так, мы показали, что $M A$ тоже входит в состав $\det(A)$.
\end{proof}

\begin{theorem}[Лапласа]
    Пусть $A \in M_{n \times n}(F)$ и зафиксированы строки $i_{1}, ..., i_{k}$. Тогда определитель матрицы $A$ равен сумме всевозможных произведений миноров $k$--ого порядка, расположенных в этих строках, на их алгебраические дополнения.
    \[\det(A) = \sum_{j_{1} < ... < j_{k}} M_{j_{1}, ..., j_{k}}^{i_{1}, ..., i_{k}} A_{j_{1}, ..., j_{k}}^{i_{1}, ..., i_{k}}\]
\end{theorem}

\begin{proof}
    Рассмотрим произвольное слагаемое в этой сумме $M_{j_{1}, ..., j_{k}}^{i_{1}, ..., i_{k}} A_{j_{1}, ..., j_{k}}^{i_{1}, ..., i_{k}}$. Минор $M_{j_{1}, ..., j_{k}}^{i_{1}, ..., i_{k}}$ содержит $k!$ слагаемых, $A_{j_{1}, ..., j_{k}}^{i_{1}, ..., i_{k}}$ содержит $(n-k)!$ слагаемых. Получаем $k! (n-k)!$ слагаемых, входящих в $\det(A)$. Выбрать минор $M$ мы можем $C_{n}^{k}$ способами.
    \[C_{n}^{k} k! (n-k)! = n!.\]
    Тем самым мы получили все $n!$ слагаемых определителя $\det(A)$.
\end{proof}

\begin{theorem}[о разложении по строке или столбцу]
	Пусть $A = (a_{ij}) \hm{\in} M_n(F)$. Тогда выполнены следующие равенства:
	\[\det{A} = \sum_{i = 1}^na_{ij}A_{ij} = \sum_{j = 1}^na_{ij}A_{ij}\]
\end{theorem}

\begin{proof}
	Докажем без ограничения общности вторую формулу, поскольку первая может быть получена из второй транспонированием. Представим $i$-ю строку матрицы $A$ в следующем виде:
	\[a_{i*} = (a_{i1}, a_{i2}, \dots, a_{in}) = (a_{i1}, 0, \dots, 0) + (0, a_{i2}, \dots, 0) + \dots + (0, 0, \dots, a_{in})\]
	
	Тогда, в силу линейности определителя как функции от строк $A$, получим:
	\[det{A} = a_{i1}A_{i1} + a_{i2}A_{i2} + \dots + a_{in}A_{in}\qedhere\]
\end{proof}

\begin{theorem}[определитель Вандермонда]
    Определителем Вандермонда называется
    \[\begin{vmatrix}
    1 & x_1 & ... & x_{1}^{n-1} \\
    1 & x_2 & ... & x_{2}^{n-1} \\
    ... & ... & ... & ... \\
    1 & x_n & ... & x_{n}^{n-1} \\
    \end{vmatrix} = \prod_{1 \leq i \leq j \leq n} (x_j - x_i).\]
\end{theorem}

\begin{proof}
    Индукция по размеру матрицы $m$.
    База индукции: $m = 2$ -- очевидно. Пусть утверждение верно для размера $m - 1$. Тогда, вычтем из последнего столбца предпоследний, умноженный на $x_{1}$, из $m-2$-го -- $m-3$-й, умноженный на $x_{1}$, из $i$-го -- $i-1$-й, умноженный на $x_{1}$ и так далее для всех столбцов. Эти преобразования не меняют определитель матрицы. Получим
    \[\begin{vmatrix}
    1 & 0 & 0 & ... & 0 \\
    1 & x_2 - x_1 & x_{2}(x_{2} - x_1) & ... & x_{2}^{m-2}(x_2 - x_1) \\
    ... & ... & ... & ... & ... \\
    1 & x_m - x_1 & x_{m}(x_{m} - x_1) & ... & x_{m}^{m-2}(x_m - x_1) \\
    \end{vmatrix}\]
    Раскладывая этот определитель по элементам первой строки, получаем, что он равен следующему определителю:
    \[\begin{vmatrix}
    x_2 - x_1 & x_2 (x_2 - x_1) & ... & x_{2}^{m-2}(x_2 - x_1) \\
    ... & ... & ... & ... \\
    x_m - x_1 & x_m (x_m - x_1) & ... & x_{m}^{m-2}(x_m - x_1) \\
    \end{vmatrix}\]
    Для всех $i$ от $1$ до $m-1$ вынесем из $i$-й строки множитель $x_{i+1} - x_1$. Получим
    \[(x_2 - x_1)(x_3 - x_1)...(x_m - x_1) \begin{vmatrix}
    1 & x_2 & ... & x_{2}^{m-1} \\
    ... & ... & ... & ... \\
    1 & x_m & ... & x_{m}^{m-1} \\
    \end{vmatrix}\]
    Подставим значение имеющегося в предыдущей формуле определителя, известного из индукционного предположения:
    \[(x_2 - x_1)(x_3 - x_1)...(x_m - x_1) \prod_{2 \leq i \leq j \leq m} (x_j - x_i) = \prod_{1 \leq i \leq j \leq m} (x_j - x_i).\]
\end{proof}

\begin{theorem}[Крамера]
	Пусть $A \in M_n(F)$, причем $\Delta := \det{A} \ne 0$, $b \in F^n$. Для каждого $i \in \nset{n}$ положим $\Delta_i := \det (a_{*1},\dots,a_{*i-1},b,a_{*i+1},\dots,a_{*n})$. Тогда система $Ax = b$ имеет единственное решение $x$, и это решение имеет следующий вид:
	\[x = \left(\frac{\Delta_1}{\Delta}, \dotsc, \frac{\Delta_n}{\Delta}\right)^T\]
\end{theorem}

\begin{proof}
	Матрица $A$ невырожденна и потому обратима, тогда $x := A^{-1}b$ "--- единственное решение системы. Заметим, что для этого решения и каждого $i \in \nset{n}$ выполнены следующие равенства:
	\begin{multline*}
		\Delta_i = \det\left(a_{*1},\dots,a_{*i-1},\sum_{j = 1}^{n}x_j(a_{*j}),a_{*i+1},\dots,a_{*n}\right) =
		\\
		= \sum_{j = 1}^{n}x_j\det\left(a_{*1},\dots,a_{*i-1},a_{*j},a_{*i+1},\dots,a_{*n}\right)
		=
		\\
		=
		x_i\det\left(a_{*1},\dots,a_{*i-1},a_{*i},a_{*i+1},\dots,a_{*n}\right) = x_i\Delta
	\end{multline*}
	
	Таким образом, для любого $i \in \nset{n}$ выполнено $x_i = \frac{\Delta_i}{\Delta}$.
\end{proof}

\begin{proposition}
	Пусть $A \in M_n(F)$, $\Delta := \det{A} = 0$, но существует $i \in \{1, \dots, n\}$ такое, что $\Delta_i \hm{\ne} 0$. Тогда система несовместна.
\end{proposition}

\begin{proof}
	Поскольку $\Delta = 0$, то $A$ вырожденна, то есть $\rk{A} < n$. При этом существует $i \in \{1, \dots, n\}$ такой, что $\Delta_i \ne 0$, поэтому в $(A|b)$ существует система из $n$ линейно независимых столбцов, тогда $\rk(A|b) > \rk{A}$. Значит, по теореме Кронекера-Капелли, система несовместна.
\end{proof}

\begin{corollary}[формула Крамера]
	Пусть $A = (a_{ij}) \in M_n(F)$ "--- обратимая матрица, и пусть $B = (b_{ij}) \in M_n(F)$ "--- обратная к ней матрица. Тогда для любых $i, j \in \nset{n}$ выполнено следующее равенство:
	\[b_{ij} = \frac{A_{ji}}{\det{A}}\]
    upd: $A_{ij}$ -- это не $a_{ij}$.
\end{corollary}

\begin{proof}
	Каждый столбец $b_{*j}$ матрицы $B$ является единственным решением системы линейных уравнений $Ab_{*j} = e_{*j}$, где $e_{*j}$ "--- $j$-й столбец единичной матрицы. Тогда:
	\[b_{ij} = \frac{\det(a_{*1}, \dots, a_{*i-1},e_{*j},a_{*i+1}, \dots, a_{*n})}{\det{A}}\]
	
	По уже доказанному утверждению, определитель в выражении выше равен $A_{ji}$.
\end{proof}