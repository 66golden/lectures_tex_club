\subsection{Вывод общего уравнения касательной к кривой второго порядка. Касательные к эллипсу, параболе и гиперболе.}
    
    \begin{definition}
    	\textit{Касательной} к кривой $C$ в точке $A \in C$ называется предельное положение секущей $AB$, $B \in C$, при $B \to A$.
    \end{definition}

    \begin{definition}
        \textit{Особая точка} кривой второго порядка -- это её центр, если центр принадлежит кривой. Кривые с особыми точками:
        \begin{itemize}
            \item Пара пересекающихся действительных прямых;
            \item Пара пересекающихся мнимых прямых;
            \item Пара совпавших действительных прямых.
        \end{itemize}
    \end{definition}

    \begin{note}
        В особой точке касательная не определена.
    \end{note}

    \begin{note}
        Пусть точка -- не особая. Если эта точка лежит на прямой, входящей в состав кривой $C$, то эта прямая -- касательная. 
    \end{note}

    Далее будем рассматривать $C$ -- эллипс, гипербола или парабола.\\
    
    \begin{theorem}[Вывод общего уравнения касательной]
            Пусть $F(x,y) := A x^2 + 2B xy + C y^2 + 2Dx + 2Ey + F = 0$, $M(x_{0}, y_{0}) \in C$.\\
        Пусть $F_{1}(x, y) = Ax + By + D$, $F_{2}(x, y) = Bx + Cy + E$. Секущая $l:$ $x = x_0 + \alpha t$, $y = y_0 + \beta t$. Заметим, что при $t = 0$ $M \in C \cap l$. Найдем вторую точку пересечения.\\
        \[A(x_{0} + \alpha t)^2 + 2B(x_{0} + \alpha t)(y_{0} + \beta t) + C(y_0 + \beta t)^2 + 2D(x_0 + \alpha t) + 2E(y_0 + \beta t) + F = 0\]
        \[t^{2}(A \alpha^2 + 2B\alpha\beta + C\beta^2) + 2t(A x_0 \alpha + B x_0 \beta + B y_0 \alpha + C y_{0} \beta + D\alpha + E\beta) + F(x_{0}, y_{0}) = 0\]
        \begin{enumerate}
            \item $A\alpha^2 + 2B\alpha\beta + C\beta^2 = 0 \Rightarrow \overline{v}\begin{pmatrix}
    		\alpha\\ \beta
            \end{pmatrix}$, удовлетворяющие этому условию, называются асимптотическими направлениями. \\
            $\delta < 0$ -- 2 направления;\\
            $\delta = 0$ -- 1 направление;\\
            $\delta > 0$ -- 0 направлений;
            \item Пусть $\begin{pmatrix}
    		\alpha\\ \beta
            \end{pmatrix}$ -- не асимптотическое. Тогда существует $t'$ -- корень и $t' \neq t_0$ и существует $M_{1} \in l \cap C$. Если $M \to M_0$, то $0$ -- двукратный корень. Но для этого 
            \[Ax_0 \alpha + B x_0 \beta + B y_0 \alpha + C y_0 \beta + D \alpha + E \beta = 0\]
            \[\alpha F_{1}(x_{0}, y_{0}) + \beta F_{2}(x_{0}, y_{0}) = 0,\]
            но оба $F_1$ и $F_2$ не равны нулю одновременно, иначе $M$ -- особая точка. Тогда
            \[ k: \frac{x - x_0}{-F_{2}(x_{0}, y_{0})} = \frac{y - y_0}{F_{1}(x_{0}, y_{0})}\]
            -- уравнение касательной в точке $M$. 
        \end{enumerate}
    \end{theorem}