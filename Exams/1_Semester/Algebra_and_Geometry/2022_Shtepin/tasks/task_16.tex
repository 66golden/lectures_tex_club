\subsection{Поле комплексных чисел. Модуль и аргумент комплексного числа.}

        Пусть $F$ -- поле, такое что $x^2 + y^2 = 0$ имеет в $F$ только тривиальное решение $x = y = 0$. (I)
    
    \begin{proposition}
            Поле $F$, обладающее свойством (I), можно вложить в поле $K$ так, что $\dim_{F}K = 2$ и в поле $K$ уравнение $x^2 = -1$ имеет решение.
    \end{proposition}
    
    \begin{proof}
            Рассмотрим следующее множество матриц:
    	\[K := \left\{\begin{pmatrix}
    	a & b\\
    	-b & a
    	\end{pmatrix} \in M_2(F)\right\}\]
        \begin{enumerate}
    		\item Непосредственная проверка позволяет убедиться, что $(K, +)$ является подгруппой в $(M_2(F), +)$, причем $K$ замкнуто относительно умножения и содержит нейтральный относительно умножения элемент --- матрицу $E \in M_2(F)$. Значит, $K$ является подкольцом в $M_2(F)$.
    		
    		\item Покажем теперь, что $K$ "--- поле. Для этого следует проверить, что $K^* = K \backslash \{0\}$. Действительно, если $a, b \in F$, и эти элементы не равны нулю одновременно, то без ограничения общности $b \ne 0$, тогда:
    		\[\begin{vmatrix}
    		a & b\\
    		-b & a
    		\end{vmatrix} = a^{2} + b^{2} = b^2(1 + (ab^{-1})^2) \ne 0\]
    		
    		Итак, согласно формуле Крамера, матрица выше обратима, причем выполнено следующее равенство:
    		\[\begin{pmatrix}
    		a & b\\
    		-b & a
    		\end{pmatrix}^{-1} = \frac{1}{a^2 + b^2}\begin{pmatrix}
    		a & -b\\
    		b & a
    		\end{pmatrix} \in K\]
    		
    		\item Поле $K$ содержит подполе $F' := \{aE: a \in F\}$, изоморфное полю $F$. Легко проверить, что операции с его элементами этого подполя соответствуют операциям с элементами поля $F$.
    		
    		\item В поле $K$ есть элемент $i$ следующего вида:
    		\[i := \begin{pmatrix}
    		0 & 1\\
    		-1 & 0
    		\end{pmatrix} \in K\]
    		
    		Тогда $i^2 = (-1)E$, и матрица $(-1)E$ соответствует числу $-1$ в подполе $F'$.
    	\end{enumerate}
    	
    	Получено требуемое.
    \end{proof}
    
    \begin{corollary}
    	Если $F = \mathbb{R}$, то полученное поле изоморфно $\mathbb{C}$, причем изоморфизм имеет следующий вид:
    	\[\begin{pmatrix}
    	a & b\\
    	-b & a
    	\end{pmatrix} \mapsto a +bi\]
    \end{corollary}

    \begin{definition}
        \textit{Модулем} комплексного числа $z = x + iy$ называется неотрицательное вещественное число $\sqrt{x^2 + y^2}$.
    \end{definition}

    \begin{definition}
        Пусть угол $\phi$ -- угол между положительным направлением оси абсцисс и направлением из начала координат на $z$. Угол $\phi$ называется \textit{аргументом} числа $z$ и обозначается $\arg z = \phi$. При заданном $r = |z|$ углы, отличающиеся на целое кратное $2\pi$, соответствуют одному и тому же комплексному числу.
    \end{definition}

    \begin{note}
        Аргумент не определен для числа $0$ с модулем $|0| = 0$.
        Отношения больше/меньше бессмысленны в применении к комплексным числам, то есть их нельзя соединять знаком неравенства: в отличие от вещественных чисел, \textit{комплексные числа не упорядочены}.
    \end{note}