\subsection{Инварианты кривой второго порядка.}

    $P(x, y) \hm{=} Ax^2 + 2Bxy + Cy^2 + 2Dx + 2Ey + F$.

    \begin{definition}
        \textit{Инвариантом} кривой второго порядка называется $F(A, ..., F)$, которая сохраняет свое значение при переходе от одной ПДСК к другой.
    \end{definition}

    \begin{theorem}
        Следующие величины являются инвариантами кривой второго порядка:
        \[\Delta = \begin{vmatrix}
            A & B & D \\
            B & C & E \\
            D & E & F
        \end{vmatrix}, \ \delta = \begin{vmatrix}
            A & B \\
            B & C
        \end{vmatrix}, \ I = A + C\]
    \end{theorem}

    \begin{proof}
        Запишем матричное уравнение кривой второго порядка
        \[\begin{pmatrix}
            x & y & z
        \end{pmatrix} \cdot \begin{pmatrix}
            A & B & D \\
            B & C & E \\
            D & E & F
        \end{pmatrix} \cdot \begin{pmatrix}
            x \\
            y \\
            z
        \end{pmatrix}\bigg|_{z = 1} = 0. \ \text{(I)}\]
        Так же запишем формулу перехода
        \[\begin{pmatrix}
    		x\\ y
            \end{pmatrix} = \begin{pmatrix}
                \cos\phi & -\sin\phi \\
                \sin\phi & \cos\phi
            \end{pmatrix} \begin{pmatrix}
    		x'\\ y'
            \end{pmatrix} + \begin{pmatrix}
    		\gamma_{1}\\ \gamma_{2}
            \end{pmatrix}, \ \text{(1)}\]
        от $(0, \overline{e_{1}}, \overline{e_{2}})$ к $(0', \overline{e_{1}}', \overline{e_{2}}')$. Наряду с (1) рассмотрим преобразование в пространстве
        \[\begin{pmatrix}
    		x \\
                y \\
                z
            \end{pmatrix} =  \begin{pmatrix}
                \cos\phi & -\sin\phi & \gamma_1 \\
                \sin\phi & \cos\phi & \gamma_2 \\
                0 & 0 & 1
            \end{pmatrix} \begin{pmatrix}
    		x' \\
                y' \\
                z'
            \end{pmatrix}, \ \text{(2)}\]
        при этом $z = z' = 1$. После преобразования в пространстве (2) ур-ие (I) примет вид:
        \[\begin{pmatrix}
            x' & y' & z'
        \end{pmatrix} \left(\begin{pmatrix}
            \cos\phi & \sin\phi & 0 \\
            -\sin\phi & \cos\phi & 0 \\
            \gamma_1 & \gamma_2 & 1
        \end{pmatrix} \begin{pmatrix}
            A & B & D \\
            B & C & E \\
            D & E & F
        \end{pmatrix} \begin{pmatrix}
            \cos\phi & -\sin\phi & \gamma_1 \\
            \sin\phi & \cos\phi & \gamma_2 \\
            0 & 0 & 1
        \end{pmatrix}\right) \begin{pmatrix}
            x' \\
            y' \\
            z'
        \end{pmatrix}\bigg|_{z' = 1} = 0.\]
        Обозначим $R' = \begin{pmatrix}
            \cos\phi & \sin\phi & 0 \\
            -\sin\phi & \cos\phi & 0 \\
            \gamma_1 & \gamma_2 & 1
        \end{pmatrix}$, $R = \begin{pmatrix}
            \cos\phi & -\sin\phi & \gamma_1 \\
            \sin\phi & \cos\phi & \gamma_2 \\
            0 & 0 & 1
        \end{pmatrix}$. Тогда
        \[\Delta = \begin{vmatrix}
            R' \begin{pmatrix}
                A & B & D \\
                B & C & E \\
                D & E & F
            \end{pmatrix} R
        \end{vmatrix} = |R'|\Delta |R| = \Delta.\]
        Обозначим $R(-\phi) = \begin{pmatrix}
            \cos\phi & \sin\phi \\
            -\sin\phi & \cos\phi
        \end{pmatrix}$, $R(\phi) = \begin{pmatrix}
                \cos\phi & -\sin\phi \\
                \sin\phi & \cos\phi
            \end{pmatrix}$. Тогда
        \[\delta' = \begin{vmatrix}
            R(-\phi) \begin{pmatrix}
                A & B \\
                B & C
            \end{pmatrix} R(\phi)
        \end{vmatrix} = \begin{vmatrix}
            R(-\phi)
        \end{vmatrix} \delta \begin{vmatrix}
            R(\phi)
        \end{vmatrix} = \delta.\]
        Заметим, что $I = A + C = tr\begin{pmatrix}
            A & B \\
            B & C
        \end{pmatrix}$ ($tr(A)$ -- след матрицы $\lra$ сумма элементов на главной диагонали). Тогда
        \[I' = tr\begin{pmatrix}
            A' & B' \\
            B' & C'
        \end{pmatrix} = tr\left(R(-\phi)\begin{pmatrix}
            A & B \\
            B & C
        \end{pmatrix} R(\phi)\right) = I.\]
    \end{proof}

    \begin{proposition}{Классификация кривых второго порядка относительно инвариантов.} \\
        \begin{itemize}
    		\item Кривые эллиптического типа:
    		\begin{enumerate}
    			\item $\frac{x^2}{a^2} + \frac{y^2}{b^2} = 1$, $a \ge b > 0$, "--- \textit{эллипс} $\lra \delta > 0, \ I\cdot\Delta < 0.$
    			\item $\frac{x^2}{a^2} + \frac{y^2}{b^2} = 0$, $a \ge b > 0$, "--- \textit{пара мнимых прямых} $\lra \delta > 0, \ \Delta = 0.$
    			\item $\frac{x^2}{a^2} + \frac{y^2}{b^2} = -1$, $a \ge b > 0$, "--- \textit{мнимый эллипс} $\lra \delta > 0, \ I\cdot\Delta > 0.$
    		\end{enumerate}
    	
    		\item Кривые гиперболического типа:
    		\begin{enumerate}
    			\item $\frac{x^2}{a^2} - \frac{y^2}{b^2} = 1$, $a, b > 0$, "--- \textit{гипербола} $\lra \delta < 0, \ \Delta \neq 0.$
    			\item $\frac{x^2}{a^2} - \frac{y^2}{b^2} = 0$, $a, b > 0$, "--- \textit{пара пересекающихся прямых} $\lra \delta < 0, \ \Delta = 0.$
    		\end{enumerate}
    	
    		\item Кривые параболического типа:
    		\begin{enumerate}
    			\item $y^2 = 2px$, $p > 0$, "--- \textit{парабола} $\lra \delta = 0, \ \Delta \neq 0.$
    			\item $\frac{y^2}{a^2} = 1$, $a > 0$, "--- \textit{пара параллельных прямых} $\ra \delta = 0, \ \Delta = 0.$
    			\item $\frac{y^2}{a^2} = 0$, $a > 0$, "--- \textit{пара совпадающих прямых} $\ra \delta = 0, \ \Delta = 0.$
    			\item $\frac{y^2}{a^2} = -1$, $a > 0$, "--- \textit{пара мнимых параллельных прямых} $\ra \delta = 0, \ \Delta = 0.$
    		\end{enumerate}
    	\end{itemize}
    \end{proposition}
