\subsection{Понятние об уравнении множества. Алгебраические линии и поверхности. Пересечение и объединение алгебраических линий (поверхностей). Сохранение порядка при переходе к другой системе координат.}

    \begin{definition}
    	\textit{Одночленом}, или \textit{мономом}, от переменных $x_1, \dotsc, x_n$ называется выражение вида $\alpha x_1^{k_1} \dotsm x_n^{k_n}$, где $\alpha \in \mathbb{R}$, $k_1, \dotsc, k_n \in \mathbb{N} \cup \{0\}$. \textit{Многочленом}, или \textit{полиномом}, от переменных $x_1, \dotsc, x_n$ называется линейная комбинация одночленов от $x_1, \dotsc, x_n$.
    \end{definition}
    
    \begin{definition}
    	\textit{Несократимой записью} многочлена $P(x_1, \dotsc, x_n)$ называется представление этого многочлена в виде линейной комбинации одночленов $\alpha x_1^{k_1} \dotsm x_n^{k_n}$ с ненулевыми коэффициентами $\alpha$ и попарно различными наборами степеней $k_1, \dotsc, k_n$.
    \end{definition}
    
    \begin{corollary}
    	Несократимая запись многочлена $P(x_1, \dotsc, x_n)$ единственна.
    \end{corollary}
    
    \begin{proof}
    	Предположим, что у $P(x_1,\dots, x_n)$ есть две различных несократимых записи $P_1$ и $P_2$. Тогда несократимая запись разности $P_1 - P_2$ содержит хотя бы один моном, но эта же запись должна быть тождественно нулевой, что невозможно.
    \end{proof}
    
    \begin{definition}
    	\textit{Степенью одночлена} $\alpha x_1^{k_1} \dotsm x_n^{k_n}$ с ненулевым коэффициентом $\alpha$ называется число $k_1 + \dotsb + k_n$. \textit{Степенью многочлена} называется наибольшая из степеней одночленов, входящих в его несократимую запись. Обозначение "--- $\deg{P}$.
    \end{definition}
    
    \begin{proposition}
    	Для любых многочленов $P, Q$ выполнено следующее неравенство:
    	\[\deg{(P + Q)} \le \max\{\deg{P}, \deg{Q}\}\]
    \end{proposition}
    
    \begin{proof}
    	Сложим несократимые записи многочленов $P$ и $Q$. Приводя подобные слагаемые, получим несократимую запись многочлена $P+Q$. В ней не будет мономов степени, превосходящей $\max\{\deg{P}, \deg{Q}\}$.
    \end{proof}
    
    \begin{proposition}
    	Для любых многочленов $P, Q$ выполнено следующее равенство:
    	\[\deg{PQ} = \deg{P} + \deg{Q}\]
    \end{proposition}
    
    \begin{proof}
    	Перемножим несократимые записи многочленов $P$ и $Q$, получим сумму мономов со степенями, не превосходящими $\deg{P} + \deg{Q}$, поэтому $\deg{(PQ)} \hm{\le} \deg{P} + \deg{Q}$. Далее рассмотрим в несократимой записи $P$ моном $ax_1^{\alpha_1}\dotsm x_n^{\alpha_n}$, $a \ne 0$, удовлетворяющий следующим условиям:
    	\begin{itemize}
    		\item $\alpha_1 + \dots + \alpha_n = \deg{P}$, то есть моном имеет наибольшую степень
    		\item Среди всех мономов, удовлетворяющих предыдущему пункту, показатель степени $\alpha_1$ у данного монома наибольший
    		\item Среди всех мономов, удовлетворяющих предыдущему пункту, показатель степени $\alpha_2$ у данного монома наибольший, и так далее
    	\end{itemize}
    	
    	Аналогичным образом выберем в $Q$ моном $bx_1^{\beta_1}\dotsm x_n^{\beta_n}$, $b \ne 0$. Произведение выбранных мономов дает моном $abx_1^{\alpha_1+\beta_1}\dots x_n^{\alpha_n+\beta_n}$, $ab \hm{\ne} 0$. Пусть моном с такими же показателями степеней появился как произведение мономов $cx_1^{\gamma_1}\dots x_n^{\gamma_n}$, $c \ne 0$, из $P$ и $dx_1^{\delta_1}\dots x_n^{\delta_n}$, $d \ne 0$, из $Q$, тогда:
    	\begin{itemize}
    		\item $\gamma_1 + \dots + \gamma_n \le \alpha_1 + \dots + \alpha_1$ и $\delta_1 + \dots + \delta_n \le \beta_1 + \dots + \beta_1$, поэтому в обоих неравенствах имеет место равенство
    		\item $\gamma_1 \le \alpha_1$ и $\delta_1 \le \beta_1$, поэтому в обоих неравенствах имеет место равенство
    		\item $\gamma_2 \le \alpha_2$ и $\delta_2 \le \beta_2$, поэтому в обоих неравенствах имеет место равенство, и так далее
    	\end{itemize}
    	
    	Таким образом, все степени в данных парах мономов совпадают, тогда, в силу несократимости записей, совпадают и эти мономы. Значит, после приведения подобных слагаемых моном $abx_1^{\alpha_1+\beta_1}\dots x_n^{\alpha_n+\beta_n}$, $ab \hm{\ne} 0$, степени $\deg{P} + \deg{Q}$ сократиться не мог, откуда $\deg{(PQ)} = \deg{P} + \deg{Q}$.
    \end{proof}

    \begin{definition}
            \textit{Алгебраической кривой} (или \textit{поверхностью}) называется множество
        $M$ в $V_{2}$ (в $V_{3}$), заданное уравнением:
        \[P(x, y) = 0 \ (P(x, y, z) = 0),\]
        где $P$ -- многочлен над $\R$ отличный от $0$.
    \end{definition}

    \begin{definition}
            \textit{Порядком} алгебраической кривой $L$ называется наименьшая из возможных
        степеней многочленов $P(x, y)$, задающих $L$.
    \end{definition}

    \begin{proposition}
    	Объединение и пересечение алгебраических кривых также являются алгебраическими кривыми.
    \end{proposition}
    
    \begin{proof}
    	Пусть две кривые задаются многочленами $P_1(x, y)$ и $P_2(x, y)$ соответственно. Тогда объединение кривых задается следующим уравнением:
    	\[
    	P_1(x, y)P_2(x, y) = 0\]
    	
    	Пересечение кривых задается следующим уравнением:
    	\[
    	(P_1(x, y))^2 + (P_2(x, y))^2 = 0
    	\]
    	
    	Видно, что оба полученных выражения также являются многочленами.
    \end{proof}

    \begin{proposition}
            Порядок алгебраической кривой (поверхности) не зависит от выбора декартовой системы координат в $V_{2}$ (в $V_{3}$).
    \end{proposition}

    \begin{proof}
        Пусть алгебраическая линия $L$ имеет в системе координат $(O, \overline{e_{1}}, \overline{e_{2}})$ уравнение $P(x, y) = 0$ и порядок $N$. Перейдем к системе координат $(O, \overline{e_{1}}', \overline{e_{2}}')$:
        \[
            \begin{pmatrix}
    		x\\ y
            \end{pmatrix} =  S_{e \to e'} \begin{pmatrix}
    		x'\\ y'
            \end{pmatrix} + \begin{pmatrix}
    		\alpha_{1}\\ \alpha_{2}
            \end{pmatrix},
        \]
        поэтому уравнение линии $L$ в "новой" системе координат будет
        \[P(S_{11}x' + S_{12}y' + \alpha_{1}, \ S_{21}x' + S_{22}y' + \alpha_{2}) = 0.\]
        Отсюда следует, что $N \geq N'$, то есть при переходе к "новой" системе координат порядок алгебраической кривой не может повыситься. Применяя аналогичные рассуждения для обратного перехода от $(O, \overline{e_{1}}', \overline{e_{2}}')$ к $(O, \overline{e_{1}}, \overline{e_{2}})$, получим $N \leq N'$ и, окончательно, $N = N'$.
    \end{proof}