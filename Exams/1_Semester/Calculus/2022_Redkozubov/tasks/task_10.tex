\section{10. Первообразная. Неопределнный интеграл и его свойства. Интегрирование рациональных функций.}

    \begin{definition}
        Пусть $f$ определена на промежутке $I$. Функция $F: I \to \R$ называется \textit{первообразной}
        функции $f$ на $I$, если $F$ дифференцируема на $I$ и $\forall x \in I \ F'(x) = f(x)$.
    \end{definition}
    
    \begin{theorem}{Описание множества первообразных.}\\
        Если $F$ -- первообразная функции $f$ на промежутке $I$, то $F+C$, где $C$ -- константа,
        также является первообразной $f$ на $I$. Если $F_1, F_2$ -- первообразные $f$ на $I$,
        то $F_1-F_2$ -- постоянна на $I$.
    \end{theorem}
    
    \begin{proof}
        \[(F+C)' = F' + C' = f + 0 = f\]
        \[(F_1 - F_2)' = F'_1 - F'_2 = f - f = 0\]
        Следовательно, функция постоянна $F_1 - F_2 = C$, где $C$ -- константа.
    \end{proof}
    
    \begin{definition}
        Произвольная первообразная функции $f$ на промежутке $I$ называется
        \textit{неопределенным интергалом} функции $f$ на $I$ и обозначается \(\int f(x) \,dx\) или \(\int f \,dx \).
        Операция перехода от данной $f$ к первообразной называется \textit{интегрированием}.
    \end{definition}
    
    \begin{theorem}{Свойства неопределенного интеграла.}
        \begin{enumerate}
            \item Если существует \(\int f \,dx \) на $I$, то \((\int f \,dx)' = f\) на $I$.
            \item Если существует \(\int f \,dx \) и \(\int g \,dx \) на $I, \alpha, \beta \in \R$, то
                существует \(\int (\alpha f + \beta g) \,dx = \alpha \int f \,dx + \beta \int g \,dx + C\).
            \item (Формула интеграла по частям) Если функции $u$ и $v$ дифференцируемы на промежутке $I$
                и существует \(\int vu' \,dx\), то на $I$ существует \(\int uv' \,dx = uv - \int vu' \,dx + C\).
                \begin{proof}
                    Правая часть имеет вид $F(x) + C$. тогда $F$ дифференцируема на $I$ и $F' = u'v + uv' - vu' = uv'$.
                    Следовательно, $F$ -- первообразная $uv'$.
                \end{proof}
                
                \begin{note}
                    Традиционная запись \(\int u \,dv = uv - \int v \,du\).
                \end{note}
            \item (Интегрирование подстановкой) Если $F$ -- первообразная функции $f$ на промежутке $I$,
                $\phi$ -- дифференцируема на промежутке $J$ и $\phi(J) \subset I$, то существует на $J$:
                \[\int f(\phi(t))\phi'(t) \,dt = F(\phi(t)) + C\]
                \begin{note}
                    Если дополнительно $\phi$ -- строго монотонна но $J$, то из предыдущей формулы следует,
                    что на $\phi(J)$ существует 
                    \[\int f(x) \,dx = \int f(\phi(t)) \phi'(t) \,dt|_{x = \phi(t)} + C\]
                \end{note}
            \item (Формула интегрирования обратной функции) Если $f$ на $I$ имеет конечную, не равную 0 производную и $F$ -- первообразная $f$ на $I$, то для обратной функции на $f(I)$ существует
                \[\int f^{-1}(y) \,dy = yf^{-1}(y) - F(f^{-1}(y)) + C\]
        \end{enumerate}
    \end{theorem}
    
    \begin{proof}
        Проверка всех перечисленных равенств производится дифференцированием на указанных промежутках.
    \end{proof}
    
    \begin{definition}
        \textit{Рациональной дробью} называется частное двух многочленов.
        Рациональные функции вида $\frac{A}{(x-a)^{n}} \ (A \neq 0), \ \frac{Mx + N}{(x^{2} + px + q)^{n}}, \  M \text{ или } N \neq 0, \  \frac{p^2}{4} - q < 0$
        называются \textit{элементарными (простейшими)} рациональными дробями.
    \end{definition}
    
    Всякая правильная рациональная дробь $\frac{P(x)}{Q(x)}$ имеет единственное разложение на элементраные дроби с точностью до порядка слагаемых.
    Покажем, как интегрируются элементарные дроби.
    
    \begin{enumerate}
        \item $\int \frac{A}{x-a} \,dx = A \ln|x-a| + C$
        \item $\int \frac{A}{(x-a)^{n}} \,dx = - \frac{A}{(n-1)(x-a)^{n-1}} + C$
        \item $\int \frac{Mx+N}{x^2 + px + q} \, dx = \frac{M}{2} \int \frac{2x + p}{x^2 + px + q} \,dx + (N-\frac{Mp}{2})\int \frac{1}{x^2 + px + q} \,dx + C_{1} = \\ \frac{M}{2}\int \frac{d(x^2 + px + q)}{x^2 + px + q} + (N - \frac{Mp}{2})\int \frac{d(x + \frac{p}{2})}{(x^2 + \frac{p}{2})^2 + q - \frac{p^2}{4}} + C_{1} = \frac{M}{2}\ln(x^2 + px + q) + \frac{N - \frac{Mp}{2}}{\sqrt{q - \frac{p^2}{4}}}\cdot \arctg \frac{x + \frac{p}{2}}{\sqrt{q - \frac{p^2}{4}}} + C_{2}$
        \item $\int \frac{Mx + N}{(x^2 + px + q)^{n}} \, dx = \frac{M}{2} \int \frac{2x+p}{(x^2 + px + q)^{n}} \, dx + (N - \frac{Mp}{2}) \int \frac{1}{(x^2 + px + q)^{n}}) \, dx + C_{1} = - \frac{M}{2(n-1)(x^2 + px + q)^{n-1}} + (N - \frac{Mp}{2}) \int \frac{d(x + \frac{p}{2})}{((x + \frac{p}{2})^{2} + q - \frac{p^2}{4})^{n}} + C_{2}$.\\
        Заменой $t = x + \frac{p}{2}$ и $a = \sqrt{q - \frac{p^2}{4}}$ последний интеграл сводится к $J_{n} = \int \frac{dt}{(t^2 + a^2)^{n}}$. Проитегрируем $J_{n}$ по частям, положив $u = \frac{1}{(t^2 + a^2)^{n}}, \ v = t$.
        Тогда 
        \[J_{n} = \frac{t}{(t^2 + a^2)^{n}} + 2n \int \frac{t^2}{(t^2 + a^2)^{n+1}} \, dt + C_{1} = \frac{t}{(t^2 + a^2)^{n}} + 2n J_{n} - 2n a^2 J_{n+1} + C_{2}\]
        \[J_{n+1} = \frac{1}{2n a^2} [\frac{t}{(t^2+a^2)^{n}} + (2n-1)J_{n}] + C_{3}, \ J_{1} = \frac{1}{a} \arctg(\frac{t}{a}) + C.\]
    \end{enumerate}
    
    Все элементарные функции возможно проинтегрировать за конечное число операций.
    
    \begin{theorem}
        Об интегрировании рациональных дробей.\\
        Неопределенный интеграл от рациональной дроби выражается через рациональные функции (быть может многочлены), $\ln, \arctg$ и, следовательно, является элементарной функцией.
    \end{theorem}