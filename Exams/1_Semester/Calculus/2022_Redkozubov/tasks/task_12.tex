\section{12. Евклидово пространство $\R^{m}$. Предел и производная вектор--функции. Теорема Лагранжа для вектор--функций. Параметризованная кривая в $\R^{m}$. Длина кривой. Аддитивность длины кривой. Достаточное условие спрямляемости. Дифференцируемость переменной длины дуги кривой. Натуральная параметризация.}

    Пусть $\R^{m} = \{x = (x_{1}, x_{2}, ... , x_{m})^{T}, x \in \R\}$. Множество $\R^m$ является векторным пространством относительно операций сложения и умножения на число:\\
    1) $(x_{1}, ... , x_{m})^{T} + (y_{1}, ... , y_{m})^{T} = (x_{1} + y_{1}, ... , x_{m} + y_{m})^{T}$,\\
    2) $\alpha (x_{1}, ... , x_{m})^{T} = (\alpha x_{1}, ... , \alpha x_{m})^{T}$.\\
    Функция $(\cdot, \cdot): \R^{m} \times \R^{m} \to \R, (x, y) = \sum_{i = 1}^{m}x_{i}y_{i}$ удовлетворяет свойствам:
    \begin{enumerate}
        \item $\forall x \in \R^{m}: (x, x) \geq 0, (x, x) = 0 \lra x = 0$;
        \item $\forall x, y \in \R^{m}: (x, y) = (y, x)$;
        \item $\forall x, y, z \in \R^{m}: \forall \alpha, \beta \in \R (\alpha x + \beta y, z) = \alpha(x, z) + \beta(y, z)$.
    \end{enumerate}
    
    Всякая функция $(\cdot, \cdot)$ на векторном пространстве (над $\R$), удовлетворяющее свойствам 1-3, называется \textit{скалярным произведением} на векторном пространстве (над $\R$). Векторное пространство со скалярным произведением называется \textit{евклидовым пространством}. Т.о, $\R^{m}$ -- евклидово пространство.\\
    Будем рассматривать функции $\gamma: E \to \R^{m}$, где $E \subset \R$ (\textit{вектор-функция}). Поскольку $\gamma(t) = (\gamma_{1}(t) , ... , \gamma_{m}(t))^{T}, t \in E$, то задание вектор-функции $\gamma \lra$ заданию на $E$ $m$ числовых функций $t \mapsto \gamma_{i}(t)$ (\textit{$i$--ая координатная функция $\gamma$}).
    
    \begin{definition}
        Пусть $\gamma: E \to \R^{m}, \ a \in \R^{m}$. Вектор $a$ называется \textit{пределом} функции $\gamma$ в точке $t_{0}$, если $t_{0}$ -- предельная точка $E$ и 
        \[\forall \epsilon > 0 \ \exists \delta > 0 \ \forall t \in \overset{o}{B}_{\delta}(t_{0}) \cap E \ (|\gamma(t) - a| < \epsilon)\]
    \end{definition}
    
    Пишут $a = \lim_{t \to t_{0}} \gamma(t)$. Аналогично вводится понятие непрерывности вектор-функции.
    
    \begin{theorem}
        Пусть $\gamma: E \to \R^{m}, \ \gamma(t) = (\gamma_{1}(t), ... , \gamma_{m}(t))^{T}$. Вектор $a = (a_{1}, ... , a_{m})^{T}$ является пределом функции $\gamma$ при $t \to t_{0}$, тогда и только тогда, когда $\gamma_{i}(t) \to a_{i}$ при $t \to t_{0}$ для каждого $i = 1, ... , m$.
    \end{theorem}
    
    \begin{proof}
        Для любого $x \in \R^{m}$ верно неравенство
        \[|x_{i}| \leq \sqrt{x_{1}^2 + ... + x_{m}^2} \leq |x_{1}| + ... + |x_{m}|.\]
        Осталось применить его к $x = \gamma(t) - a$.
    \end{proof}
    
    \begin{corollary}{Об операциях с пределами.}\\
        Пусть $\gamma, \tilde\gamma: E \to \R^m, \ f: E \to \R$. Если существуют $\lim_{t \to t_{0}}\gamma(t) = a, \ \lim_{t \to t_{0}}\tilde\gamma(t) = b$ и $\lim_{t \to t_{0}} f(t) = c$, то существуют:
        \begin{enumerate}
            \item $\lim_{t \to t_{0}}(\gamma(t) + \tilde\gamma(t)) = a + b$;
            \item $\lim_{t \to t_{0}} f(t)\gamma(t) = ca$;
            \item $\lim_{t \to t_{0}}(\gamma(t), \tilde\gamma(t)) = (a, b)$.
        \end{enumerate}
    \end{corollary}
    
    \begin{definition}
        Пусть функция $\gamma: I \to \R^m$ определена на промежутке $I$ и $t_{0} \in I$. Вектор $\gamma'(t_{0})$ называется \textit{производной} функции $\gamma$ в точке $t_{0}$, если $\gamma'(t_{0}) = \lim_{t \to t_{0}}\frac{\gamma(t) - \gamma(t_{0})}{t - t_{0}}$. При этом будем говорить, что функция $\gamma$ \textit{дифференцируема} в точке $t_{0}$.
    \end{definition}
    
    \begin{corollary}
        Пусть функция $\gamma: I \to \R^m$ определена на промежутке $I$, $\gamma(t) = (\gamma_{1}(t)\text{, ... , } \gamma_{m}(t))^{T}$. Функция $\gamma$ дифференцируема в точке $t_{0}$, тогда и только тогда, когда в этой точке дифференцируемы все ее координатные функции, причем $\gamma'(t_{0}) = (\gamma_{1}'(t_{0})\text{, ... , }\gamma_{m}'(t_{0}))^{T}$.
    \end{corollary}
    
    \begin{corollary}
        Пусть в точке $t_{0}$ дифференцируемы функции $\gamma, \tilde\gamma: I \to \R^m$ и $f: I \to \R$. Тогда в этой точке дифференцируемы функции $\gamma + \tilde\gamma, \ f\gamma$, скалярное произведение $(\gamma, \tilde\gamma)$, причем:\\
        1) $(\gamma + \tilde\gamma)' = \gamma' + \tilde\gamma'$;\\
        2) $(f\gamma)' = f'\gamma + f\gamma'$;\\
        3) $(\gamma, \tilde\gamma)' = (\gamma', \tilde\gamma) + (\gamma, \tilde\gamma')$.
    \end{corollary}
    
    \begin{corollary}
        Пусть $J, I$ -- промежутки, функции $h: J \to I$ и $\gamma: I \to \R^m$ дифференцируемы в точках $u_{0}$ и $t_{0} = h(u_{0})$ соответственно. Тогда в точке $u_{0}$ дифференцируема композиция $\gamma\circ h: J \to \R^m$ и $(\gamma\circ h)'(u_{0}) = h'(u_{0})\gamma'(t_{0})$.
    \end{corollary}
    
    \begin{theorem}{Лагранжа для вектор-функций.}\\
        Если функция $\gamma: [a, b] \to \R^m$ непрерывна на $[a, b]$ и дифференцируема на $(a, b)$, то существует $\xi \in (a, b)$, что $|\gamma(b) - \gamma(a)| \leq |\gamma'(\xi)|(b - a)$.
    \end{theorem}
    
    \begin{proof}
        Если $\gamma(b) = \gamma(a)$, то неравенство очевидно. Пусть $\gamma(b) \neq \gamma(a)$. Положим $e = \frac{\gamma(b) - \gamma(a)}{|\gamma(b) - \gamma(a)|}$, тогда $|e| = 1$ и 
        \[|\gamma(b) - \gamma(a)| = (\gamma(b) - \gamma(a), e) = (\gamma(b), e) - (\gamma(a), e).\]
        Рассмотрим на $[a, b]$ функцию $f(t) = (\gamma(t), e)$. По теореме Лагранжа $f(b) - f(a) = f'(\xi)(b - a)$ для некоторой точки $\xi \in (a, b)$. Поскольку $f(b) - f(a) = |\gamma(b) - \gamma(a)|$, $f'(t) = (\gamma'(t), e)$, $|f'(\xi)| \leq |\gamma'(\xi)|$, получаем требуемое.
    \end{proof}
    
    \begin{definition}
        \textit{Параметризованной кривой} в $\R^m$ называется непрерывная функция $\gamma: [a, b] \to \R^m$. При этом $\gamma(a)$ называется \textit{началом}, $\gamma(b)$ -- \textit{концом}, а множество $\gamma([a, b])$ -- \textit{носителем} параметризованной кривой $\gamma$.\\
        Параметризованная кривая $\gamma^{-}: [a, b] \to \R^m$, $\gamma^{-}(t) = \gamma(a + b - t)$, называется \textit{противоположной} к $\gamma$.
    \end{definition}
    
    \begin{definition}
        Две параметризованные кривые $\gamma: [a, b] \to \R^m$ и $\tilde\gamma: [c, d] \to \R^m$ называются \textit{эквивалентными}, если существует непрерывная строго возрастающая функция $h$, отображающая отрезок $[c, d]$ на $[a, b]$, такая что $\tilde\gamma = \gamma \circ h$. Аргумент кривой называется \textit{параметром}, а функция $h$ -- \textit{заменой параметра}.
    \end{definition}
    
    \begin{definition}
        \textit{Кривой} Г называется класс эквивалентности параметризованных кривых. Каждый представитель класса называется \textit{параметризацией} кривой Г.
    \end{definition}
    
    Введенное понятие кривой является слишком широким и, в частности, содержит примеры (кривые Пеано), не согласующиеся с интуитивным представлением о кривых как одномерных объектах. Желая исключить из рассмотрения подобные примеры, на параметризации накладывают дополнительные условия.
    
    \begin{definition}
        Параметризованная кривая $\gamma$ называется \textit{простой}, если $\gamma$ инъекция.
    \end{definition}
    
    \begin{definition}
        Пусть $r \in \N$. Параметризованная кривая $\gamma: [a, b] \to \R^m$, $\gamma(t) = (\gamma_{1}(t), ... , \gamma_{m}(t))^{T}$, называется $C^{r}$-\textit{гладкой}, если все $\gamma_{i} \in C^{r}([a, b])$, т.е. производные $\gamma_{i}^{(k)}$ определены и непрерывны на $(a, b)$ и существуют конечные $\gamma_{i}^{(k)}(a + 0)$ и $\gamma_{i}^{(k)}(b - 0)$, $k = 1, ... , r$. Класс $C^{\infty}$ рассматривается как пересечение всех классов $C^{r}$. При этом $\gamma'(t) = (\gamma_{1}'(t), ... , \gamma_{m}'(t))^{T}$ называется \textit{вектором скорости} кривой $\gamma$. Параметризованная кривая $\gamma: [a, b] \to \R^m$ называется \textit{кусочно-гладкой}, если существует разбиение $T = \{t_{i}\}_{i = 0}^{n}$, что кривая $\gamma|_{[t_{i-1}, t_{i}]}$ гладкая для каждого $i = 1, ... , n$.
    \end{definition}
    
    \begin{definition}
        Параметризованная кривая $\gamma: [a, b] \to \R^m$ называется \textit{регулярной} в точке $t_{0}$, если существует $\gamma'(t_{0}) \neq 0$. Параметризованная кривая регулярна, если она регулярна в каждой точке.
    \end{definition}
    
    \begin{definition}
        Кривая Г называется $C^{r}$-\textit{гладкой} (\textit{регулярной}), если у нее имеется хотя бы одна $C^{r}$-гладкая параметризация.
    \end{definition}
    
    Для параметризованной кривой $\gamma: [a, b] \to \R^m$ и разбиения $T = \{t_{i}\}_{i=0}^{n}$ отрезка $[a, b]$ определим число $L(\gamma, T) = \sum_{i=1}^{n} |\gamma(t_{i}) - \gamma(t_{i-1})|$, геометрический смысл которого -- это длина ломаной $\gamma(t_{0})\gamma(t_{1})...\gamma(t_{n})$.
    
    \begin{definition}
        \textit{Длиной} параметризованной кривой $\gamma: [a, b] \to \R^m$ называется величина 
        \[L(\gamma) = \sup_{T}L(\gamma, T),\]
        где супремум берется по всем разбиениям $T$ отрезка $[a ,b]$. Параметризованная кривая называется \textit{спрямляемой}, если ее длина конечна.
    \end{definition}
    
    \begin{lemma}
        Если параметризованные кривые $\tilde\gamma$ и $\gamma$ эквивалентны и $\gamma$ спрямляема, то $\tilde\gamma$ спрямляема, причем $L(\tilde\gamma) = L(\gamma)$.
    \end{lemma}
    
    \begin{proof}
        По условию $\tilde\gamma = \gamma \circ h$ для некоторой замены параметра $h: [c, d] \to [a, b]$. Так как $h$ строго возрастает и концевые точки переводит в концевые, то разбиение $T = \{u_{i}\}_{i = 0}^{n}$ отрезка $[c, d]$ определяет разбиение $h(T) = \{h(u_{i})\}_{i = 0}^{n}$ отрезка $[a, b]$, и наоборот. Таким образом, $h$ индуцирует биекцию между разбиениями отрезков $[c, d]$ и $[a, b]$, причем $L(\tilde\gamma, T) = L(\gamma, h(T))$. Следовательно, $L(\tilde\gamma) = L(\gamma)$.
    \end{proof}
    
    \begin{note}
        Таким образом, корректно определена длина кривой как длина любой ее параметризации.
    \end{note}
    
    \begin{lemma}{Аддитивность длины кривой.}\\
        Если кривая $\gamma: [a, b] \to \R^m$ спрямляема и $c \in (a, b)$, то также спрямляемы кривые $\gamma|_{[a, c]}$ и $\gamma|_{[c, b]}$, причем $L(\gamma) = L(\gamma|_{[a, c]}) + L(\gamma|_{[c, b]})$.
    \end{lemma}
    
    \begin{proof}
        Введем обозначения $\gamma_{1} = \gamma|_{[a, c]}$, $\gamma_{2} = \gamma|_{[c, b]}$. Пусть $T_{1}$ и $T_{2}$ -- разбиения отрезков $[a, c]$ и $[c, b]$ соответственно. Тогда $T = T_{1} \cup T_{2}$ -- разбиение $[a, b]$, и $L(\gamma_{1}, T_{1}) + L(\gamma_{2}, T_{2}) = L(\gamma, T) \leq L(\gamma)$. Переходя к супремуму по всем разбиениям $T_{1}$, затем по $T_{2}$, получим $L(\gamma_{1}) + L(\gamma_{2}) \leq L(\gamma)$.\\
        Пусть теперь $T$ -- разбиение $[a, b]$. Положим $T' = T \cup \{c\}$, $T_{1} = T' \cap [a, c]$ и $T_{2} = T' \cap [c, b]$, тогда $T_{1}$ -- разбиение $[a, c]$, $T_{2}$ -- разбиение $[c, b]$ и $L(\gamma, T) \leq L(\gamma, T') = L(\gamma_{1}, T_{1}) + L(\gamma_{2}, T_{2}) \leq L(\gamma_{1}) + L(\gamma_{2})$. Переходя к супремуму по всем разбиениям $T$ получим $L(\gamma) \leq L(\gamma_{1}) + L(\gamma_{2})$. 
    \end{proof}
    
    \begin{definition}
        Пусть $\gamma: [a, b] \to \R^m$ спрямляема. Функция $s(t) = L(\gamma|_{[a, t]})$ называется \textit{переменной длиной} дуги $\gamma$.
    \end{definition}
    
    Если спрямляемая кривая $\gamma: [a, b] \to \R^m$ не стационарна, то функция $s$ имеет обратную $t: [0, l] \to [a, b]$, $t = t(s)$, которая удовлетворяет условиям замены параметра. Параметризованная кривая $\sigma(s) = \gamma(t(s))$, $s \in [0, l]$, эквивалентная $\gamma$, называется \textit{натуральной параметризацией}, а ее аргумент -- \textit{натуральным параметром}.
    
    \begin{lemma}{Достаточное условие спрямляемости.}\\
        Всякая $C^{-1}$-гладкая параметризованная кривая $\gamma: [a, b] \to \R^m$ спрямляема, причем $L(\gamma) \leq \sup_{[a, b]} |\gamma'| \cdot (b - a)$.
    \end{lemma}
    
    \begin{proof}
        Пусть $M = \sup_{[a, b]}|\gamma'|$. Так как функция $|\gamma'|$ непрерывна, то $M \in \R$. Если $T = \{t_{i}\}_{i = 0}^{n}$ -- разбиение $[a, b]$, то по теореме Лагранжа для вектор-функций найдется такое $\xi_{i} \in (t_{i-1}, t_{i})$, что $|\gamma(t_{i}) - \gamma(t_{i-1})| \leq |\gamma'(\xi_{i})|\cdot(t_{i} - t_{i - 1})$. Поэтому $L(\gamma, T) \leq M \sum_{i = 1}^{n}(t_{i} - t_{i - 1}) = M(b - a)$. Осталось перейти к супремуму по всем разбиениям $T$. 
    \end{proof}
    
    \begin{theorem}
        Для $C^{-1}$-гладкой параметризованной кривой $\gamma: [a, b] \to \R^m$ выполнено $s'(t) = |\gamma'(t)|$.
    \end{theorem}
    
    \begin{proof}
        Пусть $a \leq t_{0} < t \leq b$. Имеем $s(t) - s(t_{0}) = L(\gamma|_{[t_{0}, t]})$, тогда из аддитивности длины кривой и достаточного условия спрямляемости
        \[|\gamma(t) - \gamma(t_{0})| \leq s(t) - s(t_{0}) \leq \sup_{[a, b]} |\gamma'|(t - t_{0}).\]
        По теореме Вейерштрасса $\sup_{[a, b]} |\gamma'| = |\gamma'(\xi_{t})|$ для некоторого $\xi_{t} \in (t_{0}, t)$. Поэтому 
        \[\lvert\frac{\gamma(t) - \gamma(t_{0})}{t - t_{0}}\rvert \leq \frac{s(t) - s(t_{0})}{t - t_{0}} \leq |\gamma'(\xi_{t})|.\]
        Перейдя к пределу при $t \to t_{0} + 0$, получим $s_{+}'(t_{0}) = |\gamma'(t_{0})|$. Аналогично устанавливается, что $s_{-}'(t_{0}) = |\gamma'(t_{0})|$.
    \end{proof}
    
    \begin{corollary}
        Всякая гладкая регулярная кривая имеет натуральную параметризацию.
    \end{corollary}
    
    \begin{proof}
        По определению кривая имеет гладкую параметризацию $\gamma$ с $\gamma' \neq 0$. Тогда по теореме $s' = |\gamma'| > 0$. Следовательно, функция $s$ обратима.
    \end{proof}
    
    \begin{corollary}
        Гладкая регулярная параметризованная кривая $\gamma: [0, l] \to \R^m$ является натуральной параметризацией тогда и только тогда, когда $|\gamma'(s)| = 1$ для всех $s \in [0, l]$.
    \end{corollary}
    
    \begin{proof}
        Если $\gamma$ -- натуральная параметризация, то $|\gamma'(s)| = s' = 1$. Обратно, если $|\gamma'(t)| = 1$, то по теореме для переменной длины дуги $s(t) = t + C$. Так как $s(t) = L(\gamma|_{[0, t]})$, то $C = s(0) = 0$ и, значит, параметр $t$ натуральный.
    \end{proof}