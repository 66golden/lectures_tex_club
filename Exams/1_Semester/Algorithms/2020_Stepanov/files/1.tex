\section{Асимптотические обозначения: $O, \Omega, \Theta$. Независимость от стартового индекса. Мастер-теорема (б/д).}
\par \textbf{Определение:} Пусть $f,g: \; \mathbb{N} \rightarrow \mathbb{N}$. Тогда $f=O(g)$, если существует $c, N$, такие что $\forall n \in N \; \hookrightarrow n \geqslant N \Rightarrow f(n) \leqslant c \cdot g(n)$.
\par \textbf{Определение:} Пусть $f,g: \; \mathbb{N} \rightarrow \mathbb{N}$. Тогда $f=\Omega(g)$, если существует $c, N$, такие что $\forall n \in N \; \hookrightarrow n \geqslant N \Rightarrow f(n) \geqslant c \cdot g(n)$.
\par \textbf{Замечание:} $f = \Omega(g) \Leftrightarrow g = O(f)$
\par \textbf{Определение:} Пусть $f,g: \; \mathbb{N} \rightarrow \mathbb{N}$. Тогда $f=\Theta(g)$, если существует $c_1,c_2, N$, такие что $\forall n \in N \; \hookrightarrow \\ n \geqslant N \Rightarrow c_1 \cdot g(n) \leqslant f(n) \leqslant c_2 \cdot g(n)$.
\par \textbf{Замечание:} $f = \Theta(g) \Leftrightarrow f = O(g)$ и $f = \Omega(g)$
\par \textbf{Утверждение:} $f = O(g) \Leftrightarrow \exists c: \forall n \in \mathbb{N} \hookrightarrow f(n) \leqslant c \cdot g(n)$ 
\par \begin{itemize}
    \item[$\blacktriangle \Leftarrow$] Очевидно, достаточно положить $N=1$
    \item[$\Rightarrow$] Пусть $\forall n \geqslant N \hookrightarrow f(n) \leqslant c \cdot g(n)$. Определим $c'=\max \{c, \frac{f(1)}{g(1)}, \ldots, \frac{f(N)}{g(N)}\}$ ($g$ не обращается в 0 так как результат натуральный). Тогда \begin{enumerate}
        \item $n \geqslant N \Rightarrow f(n) \leqslant c \cdot g(n) \leqslant c' \cdot g(n)$
        \item $n < N \Rightarrow c' \geqslant \frac{f(n)}{g(n)} \Rightarrow f(n) \leqslant c' \cdot g(n) \; \blacksquare$
    \end{enumerate}
\end{itemize}
\par Для $\Omega$ и $\Theta$ справедливы аналогичные утверждения.
\par \textbf{Мастер-теорема (с лекции):} Пусть $T: \mathbb{N} \rightarrow \mathbb{N}$ с условием $T(n)=a \cdot T(\frac{n}{b})+f(n)$, $a=const, a \geqslant 1; \\ b=const, b \geqslant 1, f: \mathbb{N} \rightarrow \mathbb{N}$. Тогда \begin{enumerate}
    \item Если $\exists \varepsilon > 0$, такое что $f(n)=O(n^{\log_b a - \varepsilon})$, то $T(n)=\Theta(n^{\log_b a})$
    \item Если $f(n)=\Theta(n^{\log_b a})$, то $T(n)=\Theta(n^{\log_b a} \log n)$
    \item Если $\exists \varepsilon > 0$, такое что $f(n)=\Omega(n^{\log_b a + \varepsilon})$, причем $\exists c < 1$, такое что $a \cdot f(\frac{n}{b}) \leqslant c \cdot f(n)$ для всех $n$, начиная с некоторого номера, то $T(n)=\Theta(f(n))$
\end{enumerate}
\par \textbf{Мастер-теорема (из интернета):} Пусть имеется рекуррентное соотношение:
$$T(n)=\left\{
\begin{array}{ccc}
a \cdot T(\frac{n}{b})+O(n^c), n>1\\
O(1),n=1\\
\end{array}
\right., \text{ где } a \in \mathbb{N}, b \in \mathbb{R}, b>1, c \in R^+.$$

Тогда асимптотическое решение имеет вид: \begin{enumerate}
    \item Если $c>\log_b a$, то $T(n)=O(n^c)$
    \item Если $c=\log_b a$, то $T(n)=O(n^c \log n)$
    \item Если $c<\log_b a$, то $T(n)=O(n^{\log_b a})$
\end{enumerate}