\setcounter{section}{9}
\section{Нахождение арифметического выражения в обратной польской записи.}
\par \textbf{Определение:} Пусть $A$ - \textit{множество всех корректных выражения в обратной польской записи}
\begin{enumerate}
    \item Если $x \in \mathbb{Z}$, то $x \in A$
    \item Если $\varphi, \psi \in A$, то $\varphi\psi+, \varphi\psi-, \varphi\psi\cdot \in A$
\end{enumerate}
\par \textbf{Пример:} 7 2 3 + $\cdot$
\par \textbf{Определение:} Рассмотрим отображение $E: A \rightarrow \mathbb{Z}$ - \textit{значение выражения} \begin{enumerate}
    \item Если $x \in \mathbb{Z}$, то $E(x)=x$
    \item Если $\varphi, \psi \in A$, то $E(\varphi\psi *)=E(\varphi)* E(\psi)$, где $* \in \{+, -, \cdot \}$
\end{enumerate} 
\par \textbf{Теорема:} Формулы в обратной польской записи обладают однозначностью разбора
\par \textbf{Договоренность:} Будем считать, что типы не переполняются
\lstinputlisting[language=C++,
emph={int,char,double,float,unsigned},
emphstyle={\color{blue}}
]{code/10_polish.cpp}
\par \textbf{Утверждение:} После выполнение программы в стеке лежит $E(s)$
\par $\blacktriangle$ Докажем индукцией по $s$ \begin{enumerate}
    \item $s$ - число $\Rightarrow$ всё корректно
    \item $s=\varphi\psi *$. По предположению индукции, сначала обработается $\varphi$ и в стек будет помещено его значение, потом $\psi$ и только затем будет произведена операция $\Rightarrow$ всё корректно $\blacksquare$
\end{enumerate}

Рассмотрим скобочную последовательность, на которую алгоритм выдаёт true. Алгоритм сопоставил каждой открывающейся скобке одного типа закрывающуюся скобку того же типа. Причём они обязательно идут в правильном порядке. 

Для доказательства факта используем индукцию.
Рассмотрим пары соответсвующих скобок в порядке закрытия пары:

База: Если между парой скобок нет других скобок, то последовательность от одной скобки до другой - правильная.

Шаг: Если между парой скобок (назовём их $a$ и $b$ соответственно) есть непустая подстрока, то все скобки из подстроки уже были рассмотрены индукцией, так как открывающиеся скобки в подстроке были позже, чем $a$ добавлены в стек и по правилу стека должны были раньше из него выйти, а значит они уже были рассмотрены индукцией. Аналогично с закрывающимеся скобками из подстроки - они идут раньше, чем $b$, следовательно, по правилу стека им ставили в соответствие открывающиеся скобки, которые были добавленны позже $a$. (окрывающаяся скобка не могла быть добавленны раньше $a$, потому что $a$ перегородила бы ей выход.) По предположению индукции получаем, что подстрока состоит из одной или нескольких ПСП => сама подстрока ПСП. => Подпоследовательность от $a$ до $b$ - правильная. 

В итоге получим, что вся последовательность ПСП. Доказано.