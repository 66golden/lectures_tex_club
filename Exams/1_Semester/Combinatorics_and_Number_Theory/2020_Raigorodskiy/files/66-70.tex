\setcounter{section}{65}

\section{Проблема Эрдеша–Гинзбурга–Зива (формулировка для d = 2). Контрпример. История проблемы. Теорема Шевалле (формулировка).}
\subsection*{Проблема Эрдеша–Гинзбурга–Зива (d = 2):}
$$(a_1, b_1), \; (a_2, b_2), ..., (a_m, b_m) \in \mathbb{Z}^2$$
$$\exists I \subset \{ 1, 2, ..., m\}, |I|=n: \; \sum_{i \in I}a_i \equiv 0(m); \sum_{i \in I}b_i \equiv 0(m)$$
Необходимо найти минимальное $m$, при котором условие выполняется, как функцию от $n$. Простая оценка:
$$m \geq 4n-3$$
\par \textbf{Контрпример} для $m=4n-4$: (0, 0) $n-1$ раз, (0, 1) $n-1$ раз, (1, 0) $n-1$ раз, (1, 1) $n-1$ раз 
\subsubsection*{История проблемы:}
\begin{enumerate}
    \item 1983 - Гипотеза Кемница: $m=4n-3$
    \item 1990 - Алон и Дубинер доказали, что $m \leq 6n-5, \; n \geq n_0$
    \item 2000 - Роньяи доказал, что $m \leq 4n-2$
    \item 2005 - Райер доказал, что $m=4n-3$
\end{enumerate}
\par \textbf{Теорема Шевалле:} если $deg \; F < n$, то число решений уравнения $F(x_1, ..., x_n) \equiv 0 (p): \; N_p \equiv 0(p)$

\section{Теорема Шевалле. Сведение к доказательству сравнения суммы.}
\par $\blacktriangle$ Заметим, что $N_p \equiv \sum_{x_1=1}^p...\sum_{x_n=1}^p(1-F^{p-1}(x_1, ..., x_n)) \; (p)$, так как мы складываем столько единиц, сколько есть наборов, где $F \equiv 0(p)$, а остальные обнуляются по малой теореме Ферма. 
\par Разобьём на 2 суммы. Очевидно, что сумма с 1 делится на $p$, поэтому нам остается только доказать, что $\sum\limits_{x_1=1}^p...\sum\limits_{x_n=1}^p(F^{p-1}(x_1, ..., x_n) \equiv 0 (p)$
\par Многочлен $F^{p-1}$ можно представить как сумму мономов вида $x_1^{\alpha_1}...x_n^{\alpha_n}$, где $\alpha_1 + ... + \alpha_n < n(p-1)$ (так как степень многочлена могла увеличиться не более чем в $p-1$ раз). Тогда, если мы докажем, что любая сумма вида $\sum\limits_{x_1=1}^p...\sum\limits_{x_n=1}^px_1^{\alpha_1}...x_n^{\alpha_n} \equiv 0 (p)$, то всё доказано. $\blacksquare$

\section{Теорема Шевалле (формулировка). Доказательство сравнения с суммой. Теорема Варнинга (формулировка).}
\par $\blacktriangle$ $\sum\limits_{x_1=1}^p...\sum\limits_{x_n=1}^px_1^{\alpha_1}...x_n^{\alpha_n} = \left(\sum\limits_{x_1=1}^px_1^{\alpha_1}\right)...\left(\sum\limits_{x_n=1}^px_n^{\alpha_n}\right)$ (очевидно). Тогда нам достаточно показать, что хотя бы одна из этих скобок сравнима с 0 по модулю $p$. Если $\exists i: \; \alpha_i = 0$, то $\sum\limits_{x_i=1}^px_i^{\alpha_i} \equiv 0(p)$. При $p=2$ такое $\alpha_i$ точно найдется по принципу Дирихле ($\alpha_1 + ...+\alpha_n < n$)
\par Пусть $p \geq 3$ и $\forall i \; \alpha_i \geq 1$. Тогда $\exists i: \; 1 \leq \alpha_i \leq p-2$ (если такого нет, то все $\alpha_i \geq p-1$, и $\alpha_1+...+\alpha_n \geq n(p-1)$ - противоречие).
\par \textbf{Утверждение: } $\exists a: \; (a,p)=1$ и $a^{\alpha_i} \not\equiv 1(p)$ ($1 \leq \alpha_i \leq p-2)$ (будет доказано в весеннем семестре)
\par Обозначим $S = \sum\limits_{x_i=1}^p x_i^{\alpha_i}$. Тогда по утверждению существует такое $a$, что
$$a^{\alpha_i}S=\sum_{x_i=1}^p (a x_i)^{\alpha_i}$$
\par Так как $a$ и $p$ взаимно просты $a x_i$ это просто какая-то перестановка $\mathbb{Z}_p$. Следовательно
$$a^{\alpha_i}S \equiv S (p)$$
$$(a^{\alpha_i}-1)S \equiv 0 (p)$$
\par В силу нашего выбора $a$ первая скобка не сравнима с 0 по модулю $p$. Тогда $S \equiv 0 (p) \; \blacksquare$ 
\par \textbf{Теорема Варнинга: } Если $deg F < n$ и $F(0, ..., 0) \equiv 0(p)$, то $\exists(x_1, ..., x_n) \neq (0, ..., 0): \; F(x_1, ..., x_n) \equiv 0 (p)$

\section{Проблема Эрдеша–Гинзбурга–Зива при d = 2 и n = p: нижняя и верхние оценки (формулировка). Теорема Варнинга (формулировка). Доказательство основной леммы.}
\par \textbf{Обобщенная теорема Варнинга: } Пусть $F_1, ..., F_k$ - многочлены, причём $deg F_1 +...+deg F_k < n$. Рассмотрим систему:
$$\left\{
\begin{array}{ccc}
F_1(x_1, ..., x_n) \equiv 0(p)\\
...\\
F_k(x_1, ..., x_n) \equiv 0(p)\\
\end{array}
\right. $$
Если ей удовлетворяет набор (0, ..., 0), то существует набор $(x_1, ..., x_n) \not\equiv (0, ..., 0)$, удовлетворяющий системе.
\par \textbf{Основная лемма: } Пусть $(a_1, b_1), ..., (a_{3p}, b_{3p})$ - наборы, такие что $\sum\limits_{i=1}^{3p}a_i \equiv \sum\limits_{i=1}^{3p}b_i \equiv 0(p)$. Тогда $\exists I \subset \{1, ..., 3p\}: \; |I|=p, \; \sum\limits_{i \in I}^{p}a_i \equiv \sum\limits_{i \in I}^{p}b_i \equiv 0 (p)$
\par $\blacktriangle$ Рассмотрим некоторый набор переменных $x_1, ..., x_{3p-1}$ и систему
$$\left\{
\begin{array}{ccc}
F_1(x_1, ..., x_{3p-1}) =\sum\limits_{i=1}^{3p-1}a_i x_i^{p-1} \equiv 0(p)\\
F_2(x_1, ..., x_{3p-1}) =\sum\limits_{i=1}^{3p-1}a_i x_i^{p-1} \equiv 0(p)\\
F_3(x_1, ..., x_{3p-1}) =\sum\limits_{i=1}^{3p-1}x_i^{p-1} \equiv 0(p)\\
\end{array}
\right. $$
\par Очевидно, что $deg F_1 + deg F_2 + deg F_3 = 3p - 3 < 3p-1$, набор (0, ..., 0) удовлетворяет системе $\Rightarrow$ по обобщённой теореме Варнинга существует набор $(x_1, ..., x_{3p-1}) \not\equiv (0, ..., 0)$, удовлетворяющий системе.
\par Рассмотрим $J \subset \{1, ..., 3p-1\}: \; \forall j \in J \; x_j \not\equiv 0$. Тогда
$$F_1(x_1, ..., x_{3p-1})=\sum_{i=1}^{3p-1}a_i x_i^{p-1} \equiv \sum_{j \in J}a_j \equiv 0(p) \mbox{ (по малой теореме Ферма)}$$
$$F_2(x_1, ..., x_{3p-1})=\sum_{i=1}^{3p-1}b_i x_i^{p-1} \equiv \sum_{j \in J}b_j \equiv 0(p)$$
$$F_3(x_1, ..., x_{3p-1})=\sum_{i=1}^{3p-1}a_i x_i^{p-1} \equiv \sum_{j \in J}1 = |J| \equiv 0(p)$$
Из последнего равенства следует, что 
\begin{enumerate}
    \item либо $|J|=p$, тогда всё доказано
    \item либо $|J|=2p$, тогда рассмотрим множество $I=\{1, ..., 3p\} \setminus J$. Получается, что $|I|=p, \; \sum\limits_{i \in I} a_i = \sum\limits_{i=1}^{3p}a_i-\sum\limits_{j \in J} a_j \equiv 0(p)$, так как каждое из слагаемых сравнимо с 0 по модулю $p$ (аналогично для $b$) $\blacksquare$
\end{enumerate}

\section{Проблема Эрдеша–Гинзбурга–Зива при d = 2 и n = p: нижняя и верхние оценки (формулировка). Основная лемма (б/д), вывод из неё теоремы Роньяи.}
\par \textbf{Теорема Роньяи для n=p}: $m \leq 4p-2$
\par $\blacktriangle$ Зафиксируем $m=4p-2$. Пусть нельзя выбрать набор, удовлетворяющий проблеме. Тогда $\forall I: \; |I|=p$ либо $\sum\limits_{i \in I}a_i \not\equiv 0(p)$, либо $\sum\limits_{i \in I}b_i \not\equiv 0(p)$. Тогда по основной лемме это же условие выполнено и для $I: \; |I|=3p$.
\par Рассмотрим функцию
$$F(x_1, ..., x_m)=\left(\left(\sum_{i = 1}^m a_i x_i\right)^{p-1}-1\right)\left(\left(\sum_{i = 1}^m b_i x_i\right)^{p-1}-1\right)\left(\left(\sum_{i = 1}^m  x_i\right)^{p-1}-1\right)\left(\sigma_p(x_1, ..., x_m)-2\right)$$
где $\sigma_p(x_1, ..., x_m) = \sum\limits_{I \subset \{1..m\},\\ |I|=p}(\prod\limits_{i \in I} x_i)$ - симметрический многочлен (сумма всех произведений по $p$ множителей)
\par Подставим вместо $x_1, ..., x_n$ нули и единицы
\begin{enumerate}
    \item единиц среди аргументов $p$ или $3p$ штук. Тогда в силу предположения о том, что теорема неверна одна из первых двух скобок $\equiv 0(p)$ по малой теореме Ферма
    \item число единиц не делится на $p$, то зануляется 3 скобка по малой теореме Ферма
    \item единиц $2p$ штук: $\sigma_p(x_1, ..., x_n) = C_{2p}^p \equiv 2 (p)$ (доказано в билете 64) $\Rightarrow$ последняя скобка $\equiv 0(p)$
    \item все переменные равны нулю $\Rightarrow F=2$ 
\end{enumerate}
\par Раскроем скобки. Заменим слагаемое $C x_1^{\alpha_1}...x_n^{\alpha_n}$ на $C x_1...x_n$ и получим некоторую функцию $F'$. Тогда значение функции не изменится (так как $0^k=0$ и $1^k=1$). 
\par \textbf{Утверждение: } $F'(x_1, ..., x_m)=2(1-x_1)...(1-x_m)$
\par $\blacktriangle$ Докажем, что мономы $x_{i_1}...x_{i_k}$ образуют базис в пространстве функций $\{0, 1\}^m \rightarrow \mathbb{Z}_p$. Их линейная независимость очевидна. Также очевидно, что любую функцию можно выразить в базисе характеристических функций:
$$\chi_u(v)=\left\{
\begin{array}{ccc}
1, u=v\\
0, u \neq v\\
\end{array}
\right. = \prod\limits_{i:u_i=1} v_i \prod\limits_{j:u_j=0} (1-v_j)$$
\par Если раскрыть правую часть, то получим линейную комбинацию мономов $\Rightarrow$ любую функцию можно выразить через эти мономы $\Rightarrow$ они образуют базис. Так как разложение по базису единственное, то функция $F'$ представляется именно так $\blacksquare$
$$deg F'=m=4p-2 \leq deg F \mbox{ (т.к. степень не могла увеличиться)}$$
$$deg F = (p-1)+(p-1)+(p-1)+p=4p-3 \mbox{ (сложили степени скобок)}$$
Получаем, что $4p-2 \leq 4p-3$ - противоречие $\Rightarrow$ теорема верна $\blacksquare$