\setcounter{section}{25}

\section{Правило сложения. Правило умножения. Примеры. Принцип Дирихле. Пример на принцип Дирихле.}
Предположим, что у нас имеются 2 множества (здесь и далее предполагаем, что рассматриваемые множества конечны, если не оговорено обратного): A = $\{ a_1, a_2, \dots, a_n \}$ и B = $\{ b_1, b_2, \dots, b_m, \}$. \par
\textbf{Базовые принципы комбинаторики:} \\
1. \textbf{Правило суммы}: Количество способов выбрать один объект из A или B в предположении, что $A \cap B = \varnothing$, равно n + m. \\
2. \textbf{Правило произведения}: Количество способов выбрать один объект из A и к нему в пару один объект из B равно $n*m$.\\
3. \textbf{Принцип Дирихле}: Предположим, имеется n ящиков и $n+1$ кролик, которые сидят в этих ящиках. Тогда найдется ящик, в котором сидит $\geqslant 2$ кролика. \\
\textbf{Обобщённый принцип Дирихле}: Если $nk + 1$ элементов разбить на n множеств, то хотя бы в одном множестве содержится $k+1$ элемент.  \\
\textbf{Формальная запись}: Пусть задано отображение $f: A \rightarrow B$ на конечных множествах A и B, причём $|A| > |B|$. Тогда отображение f неинъективно. \\ \par
\textbf{Задача}: Найти количество шестизначных чисел с различными цифрами.\par \textbf{Решение}: Для первой цифры есть 9 возможных вариантов (все, кроме нуля), для второй цифры 9 вариантов (все, кроме первой), для третьей 8 (все, кроме первых двух), и так далее. Тогда по правилу произведения всего количество шестизначных чисел с различными цифрами: $9*9*8*7*6*5$. \par
\textbf{Задача}: Пусть есть квадрат со стороной 2, внутри которого выбраны пять точек. Доказать, что внутри него найдутся две точки, такие, что расстояние между ними не превосходит корня из двойки. \par
$\blacktriangle$
Рассмотрим 4 квадратика 1x1. По принципу Дирихле, найдется квадратик, внутри или на границе которого 2 точки. Эти точки — искомые.
$\blacksquare$

\section{Принцип Дирихле. Оценки мощности множества попарно неортогональных $(-1, 0, 1)$-векторов: верхняя оценка величиной 140 и нижняя оценка величиной 70.}
Рассмотрим множество $V = \{ \overline{x} = (x_1, \dots, x_n) : x_i \in \{1; 0; -1 \}, x_1^2 + \dots + x_n^2 = 4\}$. $|V| = C_8^4*2^4 = 1120$. Рассмотрим произвольное $W \subset V: \forall \overline{x}, \overline{y} \in W: (x, y) \neq 0$. Max$|W|$ - ? \\ \par
\textbf{Нижняя оценка}: Пусть $x_1 = 1$, а остальные координаты либо 0, либо 1. Такое множество подходит под условие W. Аналогично построим множество векторов, где $x_1 = -1$, остальные координаты либо 0, либо -1. Это множество тоже подходит под условие W, причём условие сохраняется даже для объединения этих множеств. Тогда нижняя оценка: max$|W|$ $\geqslant 2*C_7^3 = 2*35 = 70$. \\ \par
\textbf{Верхняя оценка}: Разобьем множество V на подмонжества-ящики, внутри которых любые два вектора гарантированно дают ноль. Первый ящик: (1, 1, 1, 1, 0, 0, 0, 0), (1, -1, -1, 1, 0, 0, 0, 0), (1, -1, 1, -1, 0, 0, 0, 0), (1, 1, -1, -1, 0, 0, 0, 0) и ещё четыре вектора, которые получены из первых четырёх, поменяв местами $x_5$ и $x_1$ и.т.д. Второй ящик: (1, -1, -1, -1, 0, 0, 0, 0), (1, -1, 1, 1, 0, 0, 0, 0), (1, 1, -1, 1, 0, 0, 0, 0), (1, 1, 1, -1, 0, 0, 0, 0) и ещё четыре вектора, полученные аналогичным способом. Третий ящик - векторы первого, домноженные на -1, четвёртый - векторы второго, домноженные на -1. Всего способов выбрать четыре нуля - 35, тогда всего таких ящиков $35*4 = 140$. Два вектора не могут быть в одном ящике, значит, max$|W|$ $\leqslant 140$.