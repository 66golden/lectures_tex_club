\setcounter{section}{13}

\section{Теорема Кантора для множеств}
\par \textbf{Теорема Кантора:} Любое множество $A$ менее мощно, чем его булеан $2^A$.
\par $\blacktriangle$ Вначале заметим, что $A$ не более мощно, чем $2^A$. Действительно, $A$ равномощно множеству всех одноэлементных подмножеств $A$. Осталось доказать, что $A \not\cong 2^A$.
\par Предположим противное, т.е. что существует биекция $f : A \rightarrow 2^A$. Рассмотрим множество $M = \{x \; | \; x \not\in f(x)\}$. Это множество корректно определено, поскольку $f(x) \in 2^A$, т.е. $f(x) \subset A$, а значит вопрос о том, принадлежит ли $x$ множеству $f(x)$, правомерен.
Поскольку $f$ является биекцией, само множество $M$ равно $f(x_0)$ для некоторого $x_0$.
Есть два варианта: $x_0 \in M$ и $x_0 \not\in M$. Оба приводят к противоречию.
\par Действительно, если $x_0 \in M$, то $x \not\in f(x)$ для $x = x_0$, т.е. $x_0 \not\in f(x_0)$, т.е. $x_0 \not\in M$ в силу $f(x_0) = M$, что противоречит предположению. Если же $x_0 \not\in M$, то $x \not\in f(x)$
неверно для $x = x_0$, т.е. $x_0 \in f(x_0)$, откуда $x_0 \in M$, снова имеем противоречие. Таким образом, противоречие возникает во всех случаях, и теорема доказана. $\blacksquare$

\section{Континуальные множества. Доказательство того, что плоскость, пространство, $\mathbb{R}^k$ и $\mathbb{R}^{\mathbb{N}}$ являются континуальными. Формулировка \\ континуум-гипотезы.}
\par Множества, равномощные множеству действительных чисел, называются \textbf{континуальными}.
\par \textbf{Вспомогательная лемма: } Множество бесконечных последовательностей из 0 и 1 континуально
\par $\blacktriangle$ Построим биекцию между этим множеством и отрезком $[0,1]$. Выберем число. Если оно лежит в левой половине отрезка, то допишем в последовательность 0, иначе - 1. Те же действия повторим с отрезком в который попало число и так далее. Очевидно, что построенное отображение является биекцией.
\par Отрезок $[0,1]$ равномощен интервалу (0,1), так как они бесконечны и второе множество получено исключением конечного числа точек. А интервал, в свою очередь, равномощен $\mathbb{R}$, биекция: $tg(\pi (x - \frac{1}{2}))$. Получаем $\{0, 1\}^{\mathbb{N}} \cong [0,1] \cong (0,1) \cong \mathbb{R} \; \blacksquare$ 

\par \textbf{Утверждение:} $\mathbb{R} \cong \mathbb{R}^n, \; n \in \mathbb{N}$
\par $\blacktriangle$ Так как $\mathbb{R} \cong \{0, 1\}^{\mathbb{N}}$, можно доказать данное утрверждение заменив $\mathbb{R}$ на $\{0, 1\}^{\mathbb{N}}$. Построим биекцию: набору из $n$ последовательностей сопоставим последовательность $a_1^1 a_1^2...a_1^n a_2^1 ...$ (верхний индекс - номер последовательности, нижний - номер элемента). Очевидно, что данное соответствие взаимно-однозначно $\Rightarrow$ эти множества равномощны $\blacksquare$
\par \textbf{Утверждение:} $\mathbb{R}^{\mathbb{N}} \cong \mathbb{R}$
\par $\blacktriangle$ По свойствам возведения множества в степень другого множества:
$\mathbb{R}^{\mathbb{N}} \cong (2^{\mathbb{N}})^{\mathbb{N}} \cong 2^{\mathbb{N} \times \mathbb{N}} \cong 2^{\mathbb{N}} \cong \mathbb{R} \; \blacksquare$
\par \textbf{Континуум-гипотеза: } Любое бесконечное подмножество
$\mathbb{R}$ либо счётно, либо континуально (иными словами, нет множеств $A$, удовлетворяющих соотношению $\mathbb{N} \lesssim A \lesssim \mathbb{R}$).