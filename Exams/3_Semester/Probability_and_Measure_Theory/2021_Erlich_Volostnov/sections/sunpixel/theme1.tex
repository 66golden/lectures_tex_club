Предмет исследования теории вероятностей -- случайный эксперимент, он должен удовлетворять трём требованиям:

1) повторяемость -- должна быть возможность повторить этот эксперимент в тех же условиях много раз

2) отсутствие детерминистической регулярности -- у эксперимента должно быть несколько исходов

3) статистическая устойчивость частот -- если исследуем частоты события в двух разных сериях (серии экспериментов предполагают достаточно большое количество повторений), то получившиеся частоты должны быть близки.

Вероятностным пространством называется тройка $(\Omega, \mathcal{F}, p)$, где:

$\Omega$ -- пространство элементарных исходов

$\mathcal{F}$ -- множество событий

$p$ -- вероятностная мера


\begin{tabular}{c|c}
    Реальность & Математика \\
    \hline
    Результаты случайного эксперимента & $\omega_1, \ldots, \omega_n, \ldots$ 
    -- элементарные исходы \\
    \hline
    Событие & Событие $A \subset \Omega, \{ A \} = \mathcal{F}$ \\
    \hline
    Частота $N(A)$ & $p(A)$
    
\end{tabular}

Можно сказать, что вероятность -- идеализированное понятие частоты.

$p: \mathcal{F} \to [0, 1]$ удовлетворяет следующим условиям:

1) $p(\Omega) = 1$

2) $\forall A, B \in \mathcal{F} \ p(A \bigsqcup B) = p(A) + p(B)$.
