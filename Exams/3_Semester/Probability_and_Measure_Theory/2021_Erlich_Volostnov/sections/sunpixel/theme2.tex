\Def

\textit{Дискретной вероятностной моделью} называется математическая модель, в которой $\Omega$ не более чем счётно. В этом случае $\mathcal{F} = 2^{\Omega}$.

Классическая теория вероятностей занимается математическими моделями дискретной теории вероятностей, в которой элементарные исходы равновероятны.

Пусть $\forall \omega \in \Omega \ p(\omega) = c$

Случай, когда $|\Omega| < \infty$.

$$
1 = p(\Omega) = p(\bigsqcup_{\omega \in \Omega} w) = 
\sum_{\omega \in \Omega} p(\omega) = c \cdot |\Omega|
$$

Получается, что $p(\omega) = \frac{1}{|\Omega|}$, а значит
$\forall A \in \mathcal{F} \ p(A) = \frac{|A|}{|\Omega|}$

Докажем, что в классической теории вероятностей не может быть счётной $\Omega$.

Возьмём произвольное конечное $B \in \mathcal{F}$ ($B \subset \Omega$).
Тогда

$$
1 = p(\Omega) \geq p(\bigsqcup_{\omega \in B} \omega) = 
\sum_{\omega \in B} p(\omega) = c \cdot |B|
$$

Так как $B$ можно выбирать сколь угодно большое, получаем противоречие с тем, что $c \cdot |B| \leq 1$.

\subsection*{Урновые схемы}

В урне находится $M$ белых шаров ($1, \ldots, M$) 
и $N - M$ чёрных шаров ($M + 1, \ldots, N$).

Вытаскиваем $n$ шаров.

1) С возвращением, с порядком

$\omega = (i_1, \ldots, i_n)$, $|\Omega| = N^n$.

2) Без возвращения, с порядком

$\omega = (i_1, \ldots, i_n)$, $|\Omega| = A_N^n$

3) Без возвращения, неупорядоченный выбор

$\omega = \{ i_1, \ldots, i_n \}$ 
или $\omega = (i_1, \ldots, i_n), i_1 < i_2 < \ldots < i_n$.

$|\Omega| = C_N^n$

4) С возвращением, неупорядоченный выбор

$\omega = (j_1, \ldots j_n)$, $j_i$ -- количество появлений $i$-го шара

$|\Omega| = C_{N + n - 1}^{N - 1} = C_{N + n - 1}^{n}$

