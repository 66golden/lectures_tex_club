
\Th (о структуре измеримых множеств)\\
$S$ - полукольцо с единицей $E$.\\
$m$ - $\sigma$-аддитивная мера на $S$.\\
$\mu$ - лебегово продолжение $m$\\
$M$ - множество измеримых множеств\\
Тогда $\forall A \in M$:
$$A = (\bigcap_{i=1}^{\infty} \bigcup_{j=1}^{\infty} A_{ij}) \backslash A_0,$$ причём 
\begin{enumerate}
    \item $A_{ij} \in R(S); \ \forall i \ A_{i1} \subseteq A_{i2} \subseteq A_{i3} \subseteq \dots $
    \item $B_i = \bigcup_{j=1}^{\infty} A_{ij}; \ B_1 \supseteq B_2 \supseteq B_3 \supseteq \dots$
    \item $A_0 \in M; \ \mu(A_0) = 0; \ A_0 \subseteq (\bigcap_{i=1}^{\infty} \bigcup_{j=1}^{\infty} A_{ij})$
\end{enumerate}

\Proof

$\mu(A) = \mu^*(A) \Rightarrow$(по следствию теоремы о внешней мере)\\ 
$\forall \epsilon = \frac{1}{n} \ \exists C_n = \bigcup_{j=1}^{\infty} D_{nj} \supseteq A$ (покрытие $A$), т.ч. $D_{nj} \in S, \sum_{j=1}^{\infty} m(D_{nj}) < \mu(A) + \frac{1}{n} \ (1)$

$$C_n \supset A \Rightarrow \mu(A) < \mu(C_n) \ (2)$$

(1): $$\mu(A) > \sum m(D_{nj}) - \frac{1}{n} = \sum \mu(D_{nj}) - \frac{1}{n} \geq \mu(\bigcup D_{nj}) - \frac{1}{n} = \mu(C_n) - \frac{1}{n} \ (3)$$

(2) \& (3): $$\mu(C_n \backslash A) < \frac{1}{n};$$

Пусть $B_i = \bigcap^i_{n=1} C_n \Rightarrow B_1 \supseteq B_2 \supseteq B_3 \supseteq \dots$

$$\forall k \ A \subset (\cap^k_{n=1} C_n) = B_k;\ \mu(B_k \backslash A) \leq \mu(C_k \backslash A) < \frac{1}{k}; \ B_1 \backslash A \supseteq B_2 \backslash A \supset \dots$$

По полученным свойствам и непрерывности меры:

$0 = \lim_{i \rightarrow \infty} \mu(B_i \backslash A) = \mu(\bigcap^{\infty}_{i=1} (B_i \backslash A)) = \mu((\bigcap^{\infty}_{i=1} B_i) \backslash A)$

Тогда пусть $A_0 = (\bigcap^{\infty}_{i=1} B_i) \backslash A, \ \mu(A_0) = 0$

$$B_i = \bigcap^i_{n=1} C_n;\ C_n = \bigcup^{\infty}_{j=1} D_{nj};\ D_{nj} \in S$$

$$B_i = \bigcap^i_{n=1} \bigcup^{\infty}_{j=1} D_{nj} = \bigcup^{\infty}_{j = 1} \bigcap^{i}_{n = 1} D_{nj}; $$

Пусть $A_{ij} = \bigcup^{j}_{r = 1} \bigcap^{i}_{n = 1} D_{nr} \in R(S) \Rightarrow A_{i1} \subseteq A_{i2} \subseteq A_{i3} \subseteq \dots $

$$\bigcup^{\infty}_{j = 1} A_{ij} = \bigcup^{\infty}_{j = 1} (\bigcup^{j}_{r = 1} \bigcap^{i}_{n = 1} D_{nr}) = \bigcup^{\infty}_{j = 1} \bigcap^{i}_{n = 1} D_{nj} = B_i\ (4) \Rightarrow$$

$$A_0 = (\bigcap^{\infty}_{i=1} B_i) \backslash A = (\bigcap_{i=1}^{\infty} \bigcup_{j=1}^{\infty} A_{ij}) \backslash A \Rightarrow A_0 \subseteq (\bigcap_{i=1}^{\infty} \bigcup_{j=1}^{\infty} A_{ij})$$

Подставляем (4) в $A = (\bigcap^{\infty}_{i=1} B_i) \backslash A_0$:
$$A = (\bigcap_{i=1}^{\infty} \bigcup_{j=1}^{\infty} A_{ij}) \backslash A_0$$



\EndProof

\subsection*{Теорема Каратеодори.}

Пусть $m$ - $\sigma$-конечная мера на полукольце S. (Полукольцо не обязательно с единицей. Если единицы нет, то её можно представить в как: $X = \bigsqcup^{\infty}_{i=1} X_i;\ X_i \in S$)
Тогда $\exists!$ $\sigma$-конечная "мера" (см. замечание) $\mu$ на $\sigma(S)$, согласованная с мерой $m$, т.е. $\forall A \in S: m(A) = \mu(A)$

\textbf{Замечание:} То что мы обычно называли мерой $m: S \rightarrow [0, +\infty)$ = конечная мера. Но иногда мера $m: S \rightarrow [0, +\infty]$ может принимать бесконечные значения (как, например, $\mu(\mathbb{R}) = +\infty$), такую меру мы обозначили как "мера".

\Proof

1) Существование меры $\mu$ мы доказали на прошлой лекции. Пример $\mu$ - лебегово продолжение $m$.

2) Докажем единственность. Разберём 2 случая:

\begin{enumerate}

\item Пусть $S$ - полукольцо с единицей $E$:

Пусть $\mu$ - лебегово продолжение. $\mu'$ - другая мера на $\sigma(S)$ согласованная с $m$ (предполагаем, что она есть).

Так как мы ранее доказывали, что продолжение на полукольце единственное, то:

$\forall A \in R(S), \ \mu(A) = \mu'(A) = \nu(A)$

$\forall A \in \sigma(S) \Rightarrow$(поскольку множество измеримых по лебегу множеств является $\sigma$-алгеброй, а $\sigma(S)$ - минимальная $\sigma$-алгебра)

$A \in M \Rightarrow$ (по теореме о структуре измеримых множеств)

$A = (\bigcap_{i=1}^{\infty} \bigcup_{j=1}^{\infty} A_{ij}) \backslash A_0,\ \mu(A_0) = 0,$

$A_{ij} \in R(S) \Rightarrow A_0 = A \backslash (\bigcap \bigcup A_{ij}) \in \sigma(S)$

\textbf{Докажем утверждение:} $\mu'(A_0) = 0$

\Proof
$\mu^*(A_0) = \mu(A_0) = 0 \Rightarrow$ (существует покрытие $A_0$)

$A_0 \subset \bigcup^{\infty}_{i=1} C_i,\ $где $ C_i \in S$, такие что $\forall \epsilon > 0,\ \sum m(C_i) < \epsilon$, 

т.к. $C_i \in S$, то $\forall \epsilon > 0,\ \mu'(A_0) \leq \sum \mu'(C_i) = \sum m(C_i) < \epsilon$

Простремив $\epsilon \rightarrow 0$, получаем, что $\mu'$ не может быть положительной. 

\EndProof

Мы знаем, что $A_0 \subset \bigcap \bigcup A_{ij}$ и $A \sqcup A_0 = (\bigcap_{i=1}^{\infty} \bigcup_{j=1}^{\infty} A_{ij})$, но тогда

$\mu'(A) + \mu'(A_0) = \mu'((\bigcap_{i=1}^{\infty} \bigcup_{j=1}^{\infty} A_{ij})) =$

Но можества $A_{ij}$ вложенные (по предыдущей о структуре измеримых множеств), поэтому по непрерывности $\sigma$-адитивной меры, это можно записать как

$= \lim_{i \rightarrow \infty} \lim_{j \rightarrow \infty} \mu'(A_{ij})$ 

Аналогично для $\mu$:

$\mu(A) + \mu(A_0) = \mu((\bigcap_{i=1}^{\infty} \bigcup_{j=1}^{\infty} A_{ij})) = \lim_{i \rightarrow \infty} \lim_{j \rightarrow \infty} \mu(A_{ij})$ 

Т.к. мы знаем, что $\mu'(A_0) = 0 = \mu(A_0)$ и $\mu(A_{ij}) = \mu'(A_{ij})$ (потому что $A_{ij} \in R(S)$), то $\mu(A) = \mu'(A)$

Доказали теорему для первого случая.

\item Пусть $S$ - полукольцо без единицы:

$X \not \in S,\ X = \bigsqcup^{\infty}_{i=1} X_i,\ X_i \in S$

Рассмотрим $S_i = \{ A \cap X_i | A \in S \}$

\textbf{Докажем утверждение:}
$\forall A \in \sigma(S)\ \ (A \cap X_i) \in \sigma(S_i)$

\Proof

Пусть $F = \{A \in 2^X$, т.ч. $\forall i\ A \cap X_i \in \sigma(S_i)\}$, тогда

1) $F - \sigma-$алгебра с единицей $X$, потому что:

\begin{enumerate}
\item Замкнутость относительно пересечения: $A \cap X_i \in \sigma(S_i), B \cap X_i \in \sigma(S_i) \Rightarrow (A \cap B) \cap X_i \in \sigma(S_i)$

\item Замкнутость относительно бесконечного объединения: $A_1, A_2, \dots, A_n,\dots \in F \Rightarrow \bigcup A_i \in F$, такие что $\forall i \ A_j \cap X_i \in \sigma(S_i) \Rightarrow \forall i\ (\bigcup A_j) \cap X_i \in \sigma(S_i)$

\item Замкнутость относительно взятия дополнения: Если $A \cap X_i \in \sigma(S_i)$, то $\overline{A} \cap X_i = X_i \backslash (A \cap X_i) \in \sigma(S_i)$

\item $X \cap X_i = X_i \in \sigma(S_i) \Rightarrow X \in F$ - единица $\sigma-$алгебра $F$. 

\end{enumerate}

2) $S \subset F$:

$\forall A \in S\ A \cap X_i \in S_i \subset \sigma(S_i)$

$1) + 2) \Rightarrow \sigma(S) \subseteq F \Rightarrow \forall A \in \sigma(S)\ A \cap X_i \in \sigma(S_i)$

\EndProof

$S_i$ - полукольцо с единицей $X_i \Rightarrow\ $ (по предыдущему пункту теоремы)

$\forall A \in \sigma(S_j),\ \mu(A) = \mu'(A) \ (5)$

$\forall A \in \sigma(S) $, выполняется $\ A = \bigsqcup^{\infty}_{i=1} (A \cap X_i)$, где (по доказаному только что утверждению) $(A \cap X_i) \in \sigma(S_i) \Rightarrow ($по$ \ (5))\ \mu(A \cap X_i) = \mu'(A \cap X_i)$

По $\sigma$-адитивности мер:

$\mu(A) = \sum_{i=1}^{\infty} \mu(A \cap X_i) = \sum_{i=1}^{\infty} \mu'(A \cap X_i) = \mu'(A)$

Теорема доказана.
\EndProof
\end{enumerate}
