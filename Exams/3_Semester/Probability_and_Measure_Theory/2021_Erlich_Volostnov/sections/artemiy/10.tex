
\Def Пусть $S$ — полукольцо. Функция $m : S \rightarrow [0, +\infty)$ называется \textit{конечной мерой} на $S$, если выполняется свойство \textit{аддитивности}:

$\forall A, A_1, \dots , A_n \in S\ (\bigsqcup^n_{i=1} A_i = A \Rightarrow m(A) = \sum^n_{i=1} m(A_i))$

\Def $m$ называется \textit{$\sigma$-аддитивной конечной мерой} на полукольце $S$, если:
\begin{enumerate}
    \item $m$ — конечная мера на $S$
    \item $\forall A, \{A_i\}^{\infty}_{i=1} \in S\ (\bigsqcup^{\infty}_{i=1} A_i = A \Rightarrow m(A) = \sum^{\infty}_{i=1} m(A_i))$ (ряд должен сходиться)
\end{enumerate}

\Conseq (из теоремы о продолжении меры на кольцо, они будет ниже).
\begin{enumerate}
    \item Если $\bigsqcup_{i=1}^{\infty} A_i \subseteq A$, то 
$$m(A) \geq \sum^{\infty}_{i=1} m(A_i)$$
    \item Если $A \subseteq
\bigcup_{i=1}^n A_i$, то 
$$m(A) \leq \sum^n_{i=1} m(A_i)$$
    \item $m$ - $\sigma$-аддитивна, если $A \subseteq \bigcup^{\infty}_{i=1} A_i$, то $$m(A) \leq \sum^{\infty}_{i=1} m(A_i)$$
\end{enumerate}
\Proof
Пусть $\nu$ - продолжение меры $m$ на $R(S)$.
\begin{enumerate}
    \item $m(A) = \nu(A) = \nu((\bigsqcup^n_{i=1} A_i) \sqcup (A \backslash (\bigsqcup^n_{i=1} A_i))) = \sum^n_{i = 1} \nu(A_i) + \nu(A \backslash (\bigsqcup^n_{i=1} A_i)) \geq \sum^n_{i = 1} \nu(A_i) = \sum^n_{i = 1} m(A_i)$. Следовательно, по теореме о предельном переходе:
    $$m(A) \geq \sum^{\infty}_{i=1} m(A_i)$$
    
    \item $B_1 = A_1 \cap A$.
    $B_i = (A_i / \bigcup_{j = 1}^{i-1} A_j) \cap A$
    Тогда $A = \bigsqcup_{i=1}^n B_i$
    
    $$\nu(A) = \sum^{n}_{i = 1}\nu(B_i) \leq \sum^{n}_{i = 1}\nu(A_i)$$
    
    \item Абослютно такое же. Но будет бесконесчно много $B_i$, а вместо $n$ должно быть $\infty$. 
    
\end{enumerate}

\leftbar

\LemmaN{1} Если $m$— мера на полукольце $S$, и множества $A, A_1, \dots , A_n \in S$ таковы, что $A \subset
\bigcup_{i=1}^n A_i$, то 
$$m(A) \leq \sum^n_{i=1} m(A_i)$$

\Proof По ранее доказанной лемме существует такой набор попарно непересекающихся множеств $B_1, \dots, B_q$, что любое множество из $\{A, A_1, \dots , A_n\}$ представимо как объединение некоторых элементов набора (можно считать, что $\forall B_i\ \exists A_j\ (B_i \subseteq A_j ))$. Далее,

$$\bigcup^n_{i=1} A_i = \bigsqcup_{i=1}^q B_i \supset A$$

$$\sum^n_{i=1} m(A_i) = \sum^n_{i=1} \sum_{j:B_j \subseteq A-i} m(B_j) \geq \sum^q_{k=1} m(B_k) \geq m(A)$$ \EndProof

\LemmaN{2} Если $m$ — мера на полукольце $S$, и множества $A, A_1, \dots , A_n \in S$ таковы, что $A_i$ попарно не пересекаются, и $A_i \subset A$, то 
$$m(A) \geq \sum^n_{i=1} m(A_i)$$

\Proof По определению полукольца набор $\{A_i\}$ можно “дополнить до $A$”, а дальше всё очевидно. \EndProof

\Conseq С помощью предельного перехода данную лемму можно обобщить на бесконечный набор $\{A_i\}$.
\endleftbar

\Example 

(Классическая мера Лебега на полукольце промежутков и ее сигмааддитивность) Промежутки в $\mathbb{R}$ образуют полукольцо, на котором длина промежутка является мерой.

\Th Длина промежутка — $\sigma$-аддитивная мера.

%\begin{enumerate}
\Proof 

1) Пусть $\lfloor a, b \rceil = \bigsqcup^{n}_{i=1} \lfloor a_i, b_i \rceil$ — некоторый промежуток. ($a = a_1, b_1 = a_2, b_2 = a_3, \dots, b_n = b$)

Тогда $m(\lfloor a, b \rceil) = b - a = \sum^n_{i=1}(b_i - a_i) = \sum^n_{i=1} m(\lfloor a_i, b_i \rceil)$

2) Пусть $\lfloor a, b\rceil = \bigsqcup^{\infty}_{i=1} \lfloor a_i, b_i \rceil$ — некоторый промежуток.

Формулировка леммы о конечном покрытии (принцип Гейне-Бореля): 

$I$ - отрезок, $\{U_{\alpha}\}$ - семейство интервалов. $I \in \bigcup U_{\alpha} \rightarrow \exists U_1, \dots, U_n$, т.ч. $I \subset \bigcup_{i=0}^n U_i$

Продолжим,

\begin{enumerate}

\item Возьмём такой отрезок $[\alpha, \beta] \subseteq \lfloor a, b \rceil$, что $m[\alpha, \beta] > m \lfloor a, b\rceil - \frac{\epsilon}{2}$.

\item Определим такие интервалы $(\alpha_i, \beta_i) \supseteq \lfloor a_i, b_i\rceil,$ что $m(\alpha_i, \beta_i) < m\lfloor a_i, b_i\rceil + \frac{\epsilon}{2^{i+1}}.$

\item Так как $[\alpha, \beta] \subseteq \cup^{\infty}_{i=1} (\alpha_i, \beta_i)$, то по лемме Гейне-Бореля существует конечное подпокрытие (пусть это будут интервалы $\{(\alpha_i, \beta_i)\}^k_{i=1})$
\end{enumerate}

Далее,
$$m\lfloor a, b\rceil < m[\alpha, \beta] + \frac{\epsilon}{2} \leq \sum^k_{j=1} m(\alpha_j, \beta_j) + \frac{\epsilon}{2}$$ 
$$\leq
\sum^{\infty}_{j=1} m(\alpha_j, \beta_j) + \frac{\epsilon}{2} \leq \sum^{\infty}_{j=1} m \lfloor a_i, b_i\rceil + \epsilon$$

Устремляя $\epsilon$ к нулю, получаем $m\lfloor a, b\rceil \leq \sum^{\infty}_{j=1} m\lfloor a_i, b_i\rceil$. Равенство $m\lfloor a, b\rceil = \sum^{\infty}_{j=1} m\lfloor a_i, b_i\rceil$ получается по следствию 1 в начале билета (или по следствию из леммы 2). \EndProof
%\end{enumerate}

\leftbar
\Example Полукольцо полуинтервалов $(a, b]$, мерой на котором является $m(a, b] = \phi(b) - \phi(a)$, где $\phi$ — ограниченная неубывающая непрерывная справа функция.

\Th Указанная мера является $\sigma$-аддитивной.

Доказательство для ограниченных полуинтервалов. \Proof Каждый полуинтервал
$(A_i, B_i]$ из разбиения можно покрыть интервалом $(\alpha_i, \beta_i)$, где $\alpha_i = A_i и \beta_i-B_i < \frac{\epsilon}{2} i;$ эти интервалы покрывают произвольный подотрезок. Дальнейшее доказательство аналогично предыдущему. \EndProof
\endleftbar
