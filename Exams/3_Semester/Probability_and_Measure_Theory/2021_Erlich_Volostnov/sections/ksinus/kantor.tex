
\Idea рассмотрим отрезок $[0, 1]$, выкидываем центральную треть, а потом выкидываем центральную треть из оставшихся третей и так далее.

Формально: пусть $C_0 = [0, 1]$, $C_1 = [0, \frac{1}{3}]\cup[\frac{2}{3}, 1]$, ...: $C_n = \cup [a_n, b_n] \rightarrow C_{n+1} = \cup([a_n, c_n]\cup[d_n, b_n])$, где $c_n = \frac{2a_n+b_n}{3}$ и $d_n = \frac{a_n+2b_n}{3}$.

\Def \textbf{Канторово множество $C$} - пересечение всех $C_n$: $\cap_{n=1}^\infty C_n$.

Пользуясь непрерывностью меры и вложенностью $C_i$, вычислим Лебегову меру:\\
1. $C_0 \supset C_1 \supset C_2 \supset ...$, тогда $\lim\limits_{n\to +\infty} (\frac{2}{3})^n=0$\\
2. Замкнуто как пересечение замкнутых\\
3. Континуально (изоморфно $\mathds{R}$), так как когда мы выкидываем трети от отрезков, мы выкидываем все числа, у которых последовательно первое, второе, третье и далее число в троичной записи соответственно равно единице. А значит, в результате, в множестве кантора лежат все числа, которые записываются нулями и двойками, значит, имеем континуальное количество.

Построим \textbf{канторову лестницу}.\\
Начнём с определения 'заготовки'. Пусть $T$ \textendash\; множество концов отрезков $[a_n, b_n]$, тогда пусть $\overline{\varphi}: T \rightarrow [0, 1]$: \\
1. База: $\overline{\varphi}(0) = 0,\; \overline{\varphi}(1) = 1$ \\
2. $\overline{\varphi}(c) =  \overline{\varphi}(d) = \frac{\varphi(a) + \varphi(b)}{2}$\\
3. $\overline{\varphi}(x)$ не убывает\\
4. $\overline{\varphi}(x)$ принимает все значения вида $\frac{k}{2^n}$.

Доопределим для остальных точек интервала $[0, 1]$:
$\varphi(x) = \sup\limits_{y \leq x,\; y \in T} \; \overline{\varphi}(y)$. \\
1. Неубывающая \\
2. В точках из T: $\overline{\varphi}(x) = \varphi(x)$ (что следует из монотонности вспомогательной функции).

\Th \\
1. $\mathds{B}(\mathds{R}) \subsetneq \mathds{M}$: измеримые по Борелю не совпадают с измеримыми по Лебегу \\
2. $\exists$ непрерывная $f: [0, 1]  \rightarrow  [0, 1]$ и  $\exists$ измеримая по Лебегу $g: g(f(x))$ не измерима по Лебегу \\
Замечание: $f(g(x))$ измерима по теореме о композиции.\\
3. $\exists$ измеримое по Лебегу множество $E$, такое что его прообраз $f^{-1}(E)$ не измерим по Лебегу. 

\Proof Рассмотрим $\psi = \frac{x+\phi(x)}{2}$, непрерывная и монотонная, а значит и биекция. Обозначим обратную $\psi^{-1} = f$. Тогда: \\
1. $\lambda(C_0) = \frac{1}{2}$ \\
2. $\lambda(\psi(\bar{C})) = \frac{1}{2}\lambda(\bar{C}) = \frac{1}{2}$ \\
Имеем, что $\psi$ переводит интервал в интервал $[0,1]$, который лежит в каком-то подмножестве $C`$ с мерой $\frac{1}{2}$, поскольку множество имеет положительную меру Лебега, воспользуемся утверждением о том, что оно содержит неизмеримое подмножество $E`$. Тогда: \\
1. $E$ измеримо, но не Борелевское: $\psi(E) = E`$ (доказали пункты 1 и 3).\\
2. В качестве $g$ выберем индикатор множества $E$, тогда $g$ измерима, но композиция $g(f)$ это индикатор незимеримого $E`$, а приведен пример для пункта 2. \EndProof
