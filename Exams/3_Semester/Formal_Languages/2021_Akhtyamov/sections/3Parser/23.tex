\subsection{23. Алгоритм Эрли синтаксического разбора для КС-грамматик: доказательство полноты.}

\Statement Алгоритм Эрли является полным.

\Proof Рассмотрим слово $w$. Если слово $w$ выводимо в грамматике $G = \langle N, \Sigma, P, S \rangle$, то верно, что $S \vdash w$. Пусть $i = 0$, $j = |w|$. Тогда существуют $\varphi, \psi \in \brackets{N \cup \Sigma}^*$, что $\varphi \vdash w[0 : 0] = \varepsilon$, $S \vdash w[0 : |w|] = w$, что $S' \vdash \varphi S' \psi \vdash S' \psi \vdash_1 S \psi$. Укажем явно $\varphi$ и $\psi$: $\varphi = \varepsilon$, $\psi = \varepsilon$, тогда выполняется, что:

\begin{center}
    $S' \vdash_1 S \vdash w$
\end{center}

По сути, только что расписали подробно $S' \vdash w$. $A$ соответствует нетерминальному символу $S'$ (вспомогательному стартовому нетерминалу), $\alpha$ соответствует нетерминальному символу $S$, из которого выводится слово $w$, $\beta$ соответствует пустому слову $\varepsilon$. По основной лемме об инварианте, доказанной ранее, ситуация $\brackets{S' \rightarrow S \cdot, 0} \in D_{|w|}$ тогда и только тогда, когда $S' \vdash w$. Так как из основной леммы следует корректность алгоритма Эрли, то все возможные ситуации будут рассмотрены, и если слово $w$ выводимо в грамматике $G$, то это эквивалентно тому, что будет рассмотрена ситуация $\brackets{S' \rightarrow S \cdot, 0} \in D_{|w|}$, и будет выведено, что $w \in L \brackets{G}$. Если слово $w$ не является выводимым в грамматике $G$, то ситуация $\brackets{S' \rightarrow S \cdot, 0} \notin D_{|w|}$, и будет выведено, что $w \notin L \brackets{G}$. Значит, алгоритм Эрли является полным. \EndProof