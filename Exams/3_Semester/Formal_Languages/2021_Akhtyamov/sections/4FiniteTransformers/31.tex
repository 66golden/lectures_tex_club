\subsection{31. Замкнутость класса автоматных языков относительно конечных преобразований.}

\newcommand{\idR}{\left. id \right|_{R}}

\textbf{Теорема. }

Если $L$ -- автоматный язык, $\psi$ -- конечное преобразование, то
$\psi(L)$ -- автоматный язык.

\underline{Доказательство}

Так как $\psi$ -- конечное преобразование, то по теореме Нива
существуют такие $\eta, R, \varphi$, что 
$\psi = \eta \circ \idR \circ \varphi^{-1}$, 
где $\eta, \varphi$ -- неудлиняющие гомоморфизмы,
$R$ -- регулярный (автоматный) язык.

Введём обозначения:

1) $\varphi^{-1} (L) = L_1$

2) $\idR (L_1) = L_2$

3) $\eta (L_2) = L_3$.

Тогда нужно доказать, что $L_1, L_2, L_3$ -- автоматные языки.

1) $\varphi$ -- неудлиняющий гомоморфизм, а значит 
$\forall x \in \Sigma \ \varphi(x) = \varepsilon \text{ или } \varphi(x) = a \in \Sigma$

Обозначим 

$\Gamma_{\varepsilon} = \{ u \in \Sigma \ |\ \varphi(u) = \varepsilon \}$

$\Gamma_a = \{ u \in \Sigma \ |\ \varphi(u) = a \}$

Так как $L$ -- автоматный язык, то для него существует задающий его автомат, на рёбрах которого написана ровно одна буква. Преобразуем этот автомат так, чтобы он распознавал $L_1$:

У каждой вершины сделаем петли по $\Gamma_{\varepsilon}$

Если был переход по $a$, то заменим его переходами по $\Gamma_a$.

2) $L_2 = L_1 \cap R$. Воспользуемся тем, что пересечение автоматных языков -- автоматный язык.

3) Построим автомат для $L_2$ и заменим переходы по $a$ на $\eta(a)$.