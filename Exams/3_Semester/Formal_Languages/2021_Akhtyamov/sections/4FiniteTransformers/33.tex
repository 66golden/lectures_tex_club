\subsection{33. Лемма о разрастании для конечных преобразований. Примеры соответствий, не задаваемых конечными преобразованиями.}

\textbf{Лемма.}

Пусть $\psi$ -- конечное преобразование. Тогда:

$$
\exists p:\ \forall (u, v) \in \psi : |u| + |v| \geq p\ 
\exists x_1, y_1, z_1 \in \Sigma^{*},
x_2, y_2, z_2 \in \Gamma^{*}\ :\ 
u = x_1 y_1 z_1, v = x_2 y_2 z_2:
$$
$$
|y_1| + |y_2| > 0, |x_1 y_1| + |x_2 y_2| \leq p\ :
\forall k \in \N \ (x_1 y_1^k z_1, x_2 y_2^k z_2) \in \psi
$$

\underline{Доказательство}

Пусть $M$ -- конечный преобразователь с однобуквенными переходами, задающий конечное преобразование $\psi$.

Положим $p = |Q|$. Условие $|u| + |v| \geq p$ означается, что мы посетили $\geq p + 1$ состояние, а значит $\exists q \in Q$, которое посетили дважды (если их несколько, рассмотрим самую первую).

Пусть $\angles{q_0, x_1 y_1 z_1, \varepsilon} \vdash \angles{q, y_1 z_1, x_2}$,
$\angles{q, y_1 z_1, x_2} \vdash \angles{q, z_1, x_2 y_2}$,
$\angles{q, z_1, x_2 y_2} \vdash \angles{q_f, \varepsilon, x_2 y_2 z_2}$.

Тогда $|y_1| + |y_2| > 0$ (так как $M$ -- конечный преобразователь с однобуквенными переходами) и $|x_1 y_1| + |x_2 y_2| \leq p$ (иначе $q$ -- не первое пересечение по состояниям, так как нашёлся бы другой цикл).

Тогда из-за условия $\angles{q, y_1 z_1, x_2} \vdash \angles{q, z_1, x_2 y_2}$ можно бесконечно "накачивать" $y_1$ и $y_2$. 
Более формально, $\forall k \in \N \ (x_1 y_1^k z_1, x_2 y_2^k z_2) \in \psi$.

\textbf{Пример соответствия, не задаваемого конечным преобразованием}

$\psi: w \to w^R$

Докажем, что это преобразование нельзя задать конечным преобразованием.

Зафиксируем произвольное $p$. Тогда $(a^p b^p, b^p a^p) \in \psi$.

$a^p b^p = x_1 y_1 z_1, b^p a^p = x_2 y_2 z_2$

$|x_1 y_1| + |x_2 y_2| \leq p \Rightarrow x_1 y_1 = a^l, x_2 y_2 = b^m$

Пусть $y_1 = a^t, t > 0$. Тогда $x_1 y_1^2 z_1 = a^{p + t} b^p$.

$|x_2 y_2^2 z_2|_a = p < p + t$, то есть
$(x_1 y_1^2 z_1, x_2 y_2^2 z_2) \not \in \psi$.