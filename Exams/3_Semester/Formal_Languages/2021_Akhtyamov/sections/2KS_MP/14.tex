\subsection{14. Алгоритм Кока-Янгера-Касами синтаксического разбора для КС-грамматик.}

Алгоритм Кока-Янгера-Касами принимает на вход грамматику $G$ в нормальной форме Хомского и слово $w$. В качестве выхода выдаётся информация, принадлежит ли слово $w$ языку $L$, который задаётся грамматикой $G$.

\subsubsection*{Идея алгоритма}

Пусть $G = \angles{N, \Sigma, P, S}$ и $|w| = n$.

Будем решать задачу динамическим программированием. Заведем трехмерный массив $d$ размером $|N| \times n \times n$, состоящий из логических значений, и $d[A][i][j]=true$  тогда и только тогда, когда из нетерминала A правилами грамматики можно вывести подстроку $w[i \ldots j]$.

Рассмотрим все пары $\{\angles{j,i}\ |\ j - i = m\}$, где $m$ -- константа и $m < n$.

1) $i = j$ . Инициализируем массив для всех нетерминалов, из которых выводится какой-либо символ строки $w$. В таком случае $d[A][i][i] = true$, если в грамматике $G$ присутствует правило $A \rightarrow w[i]$. Иначе $d[A][i][i] = false$.

2) $i \neq j$. Значения для всех нетерминалов и 
пар $\{\angles{j',i'}\ |\ j' - i' < m\}$ уже вычислены, поэтому
$$
    d[A][i][j] = \bigvee \limits_{A \rightarrow BC} 
    \bigvee \limits_{k = i}^{j - 1} d[B][i][k] \wedge d[C][k + 1][j]
$$
То есть, подстроку $w[i \ldots j]$ можно вывести из нетерминала $A$, если существует продукция вида $A \rightarrow BC$ и такое $k$, что подстрока $w[i \ldots k]$ выводима из $B$, а подстрока $w[k + 1 \ldots j]$ выводится из $C$. Ответом будет значение $d[S][1][n]$.

\subsubsection*{Доказательство корректности}

Проведём индукцию по длине слова.

\textbf{База.} Пусть $j = i$. Тогда $dp[A][i][j] = True$ появилась на этапе инициализации, что по построению равносильно тому, что правило $\brackets{A \rightarrow w[i]} \in P$, что эквивалентно тому, что $A \vdash w[i : j]$, $j = i$, то есть $A \vdash w[i]$.

\textbf{Переход.} Пусть $dp[A][i][j] = True$. Тогда по построению и предположению индукции существуют число $k$ и нетерминальные символы $B$ и $C$, такие что $A \vdash_1 BC$, $B \vdash w[i:k]$, $C \vdash w[k+1:j]$, откуда $dp[B][i][k] = dp[C][k+1][j] = True$, так как $A \vdash_1 BC$, то $\brackets{A \rightarrow BC} \in P$, $A \vdash w[i : j]$.

\subsubsection*{Асимптотика}

Обработка правил вида $A \rightarrow w[i]$ выполняется за $O(n \cdot |P|)$.

Проход по всем подстрокам выполняется за $O(n^2)$. В обработке одной подстроки присутствует цикл по всем правилам вывода и по всем разбиениям на две подстроки, следовательно обработка работает за $O(n \cdot |P|)$. В итоге получаем конечную сложность $O(n^3 \cdot |P|)$.

