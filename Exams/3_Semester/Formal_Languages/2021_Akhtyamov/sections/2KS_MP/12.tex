\subsection{12. Устранение бесполезных вспомогательных символов и eps-правил для КС-грамматик.}


\Def Символ $Y \in N$ называется порождающим, если существует слово $w \in \Sigma^*$, такое что $Y \vdash w$.

\Def Символ $D \in N$ называется достижимым, если существуют некоторые $\varphi$, $\psi \in \brackets{N \cup \Sigma}^*$, такие что $S \vdash \varphi D \psi$.

\Def Символ $U \in N$ называется бесполезным, если он непорождающий или недостижимый.

\Def Символ $E \in N$ называется $\varepsilon$-порождающим, если $E \vdash \varepsilon$.

\Statement Для любой контекстно-свободной грамматики существует эквивалентная КС-грамматика без бесполезных символов.

\Proof Приведём алгоритм преобразования КС-грамматики: сначала найдём непорождающие символы, удалим их из грамматики, затем найдём и удалим недостижимые символы.

Пусть $G_1$ — грамматика, преобразованная из $G$ путём удаления непорождающих символов и всех правил, содержащих непорождающие символы. Покажем, почему $L \brackets{G} = L \brackets{G_1}$. $L \brackets{G_1} \subset L \brackets{G}$, так как количество правил уменьшается. Пусть $w \in L \brackets{G} \setminus L \brackets{G_1}$, тогда в дереве вывода есть непорождающий символ $C$:

\begin{center}
    $S \vdash \varphi C \psi \vdash w$, $w = xyz$
\end{center}

Но $\varphi \vdash x$, $C \vdash y$, $\psi \vdash z$, откуда $C$ — порождающий символ. Противоречие.

Пусть $G_2$ — грамматика, преобразованная из $G_1$ путём удаления всех недостижимых символов и правил, содержащих их. Покажем, почему $L \brackets{G_1} = L \brackets{G_2}$. $L \brackets{G_2} \subset L \brackets{G_1}$, так как количество правил уменьшается. Пусть $w \in L \brackets{G_1} \setminus L \brackets{G_2}$, тогда существует недостижимый символ $D$, что $S \vdash \varphi D \psi \vdash w$, но тогда $D$ по определению является достижимым. Противоречие.

Проверим, что не появилось новых непорождающих символов. Пусть $B$ — непорождающий символ в $G_2$, значит $B$ достижим в $G_1$. Тогда $B$ — порождающий символ в $G_1$, то есть $B \vdash_{G_1} u$. Так как $B$ стал непорождающим после удаления недостижимых символов, значит, что на пути вывода $B\vdash u$ был недостижимый символ $C$, чего не может быть так как есть путь $S \rightarrow B \rightarrow C$ - противоречие $\blacksquare$


Про удаление $\varepsilon$-порождающих символов см. билет 13