\subsection{11. Контекстно-свободные грамматики и языки. Примеры контекстно-свободных языков. Замкнутость и незамкнутость КС-языков относительно теоретико-множественных операций (можно пользоваться примерами не КС-языков).}

\par \Def \textit{Контекстно-свободная грамматика} - грамматика, все правила которой имеют вид $A \rightarrow \alpha,$
где $A \in N, \alpha \in (N \cup \Sigma)^*$. Язык называют \textit{контекстно-свободным}, если он задается контекстно-свободной грамматикой.

\par \Example КС-язык: $L = \{a^nb^mc^m\}$. Грамматика: $S \rightarrow AT, A \rightarrow aA, A \rightarrow \varepsilon, T \rightarrow bTc, T \rightarrow \varepsilon$
\par \Example Не КС-язык (понадобится для доказательств): $L=\{a^n b^n c^n\}$
\par $\blacktriangle$ Зафиксируем $p$ в лемме о разрастании. Рассмотрим $w=a^p b^p c^p=xuyvz, |uv|>0, |uyv|\leq p$. Заметим, что в $uyv$ и $uv$ не может быть трех разных букв из $a,b,c$. Не умаляя общности $|uyv|_c=0, |uyv|_b>0$. Рассмотрим $k=2$: $$|w'|_b=|xu^2yv^2z|_b=|xuyvz|_b+|uv|_b>p$$ $$|xu^2yv^2z|_c=p+|uv|_c=p$$
\par Следовательно, $|w'|_b \neq |w'|_c$, а значит $w'$ не лежит в языке и выполнено отрицание леммы о разрастании, то есть язык не является КС $\blacksquare$
\par \Statement КС-грамматики замкнуты относительно объединения и конкатенации
\par \begin{itemize}
    \item[$\blacktriangle$ 1.] Объединение: конструктивно построим объединение грамматик: создадим стартовую вершину $S'$ и два правила $S' \rightarrow S_1, S' \rightarrow S_2$, где $S_i$ - стартовый нетерминал первой или второй грамматики соответственно. Если нетерминальные символы в грамматиках совпадают переименуем их в одной из грамматик.
    \item[2.] Конкатенация: аналогично, построим КС-грамматику для языка $L_1 L_2$, добавив правило $\brackets{S' \rightarrow S_1 S_2}$. $\blacksquare$
\end{itemize}

\par \Statement КС-грамматики не замкнуты относительно пересечения и дополнения
\par \begin{itemize}
    \item[$\blacktriangle$ 1.] Пересечение: $$L_1=\{a^n b^m c^m\},$$ $$L_2=\{a^n b^n c^m\}$$ $$L_1 \cap L_2 = \{a^n b^n c^n \} \text{ - не КС-язык}$$
    \item[2.] Дополнение: $$L=\{a^nb^nc^n\} \text{ - не КС-язык}$$ $$\overline{L} = \{\text{easy cases}\} \cup \{a^k b^l c^m | k \neq l \vee l \neq m \vee k \neq m\}$$
    Оба языка из объединения - КС $\Rightarrow \overline{L}$ - КС. Но $L=\overline{\overline{L}}$ не КС $\Rightarrow$ множество КС языков не замкнуто относительно дополнения $\blacksquare$
\end{itemize}