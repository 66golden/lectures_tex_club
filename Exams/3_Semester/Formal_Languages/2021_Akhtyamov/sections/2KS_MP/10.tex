\subsection{10. Иерархия Хомского. Праволинейные грамматики. Праволинейные языки. Теорема о совпадении классов автоматных и праволинейных языков.}

\subsubsection*{Иерархия Хомского}

$A, B \in N, \; \alpha, \varphi, \psi \in \brackets{N \cup \Sigma}^{*}, \, w \in \Sigma^*$

\begin{center}
    \centering
    \begin{tabular}{|c|c|p{250px}|}
        \hline
         Грамматики &
         Правила &
         Автоматы \\
         \hline
         Праволинейные &
         $A \rightarrow wB, \ A \rightarrow w$ &
         НКА \\
         \hline
         Контекстно-свободные &
         $A \rightarrow \alpha$ &
         Автоматы с магазинной памятью \\
         \hline
         Контекстно-зависимые &
         $\varphi A \psi \rightarrow \varphi \alpha \psi$ &
         Линейно-ограниченные недетерминированные автоматы, машины Тьюринга \\
         \hline
         Порождающие &
         Любые &
         Машины Тьюринга \\
         \hline
    \end{tabular}
\end{center}

\subsubsection*{Порождающие грамматики}

\Def
\textit{Порождающая грамматика}: $G = \{N, \Sigma, P, S\}$, где:

1) $N$ -- множество вспомогательных (нетерминальных) символов, $\abs{N} < \infty$.

2) $\Sigma$ -- алфавит -- множество (терминальных) символов, $\abs{\Sigma} < \infty$, $\Sigma \cap N = \varnothing$.

3) $S \in N$ -- стартовый нетерминал

4) $P \subset \brackets{N \cup \Sigma}^{+} \times \brackets{N \cup \Sigma}^{*}$ -- множество правил, $\abs{P} < \infty$.

Правила грамматики грамматики задаются следующим образом:
$$
    \alpha \rightarrow \beta \in P, 
    \alpha \in \brackets{N \cup \Sigma}^{+},
    \beta \in \brackets{N \cup \Sigma}^{*}
$$
То есть $\alpha$ должен содержать хотя бы один символ — либо терминальный, либо нетерминальный, а $\beta$ может быть равным $\varepsilon$, то есть пустому слову. \\

В следующих определениях $G = \{N, \Sigma, P, S\}$ -- порождающая грамматика.

\Def
Наименьшее рефлексивное транзитивное отношение $\vdash_{G}$ 
называется \textit{отношением выводимости} в грамматике $G$, если 
$$
    \forall \brackets{\alpha \rightarrow \beta} \in P, \;\;
    \forall \varphi, \psi \in \brackets{N \cup \Sigma}^{*}:
    \varphi \alpha \psi \vdash_{G} \varphi \beta \psi
$$
\textit{По сути, это операция замены левой части на правую часть несколько раз, возможно, нуль.}

\Def
Слово $w \in \Sigma^{*}$ называется \textit{выводимым} в грамматике $G$,
если $S \vdash_{G} w$, то есть из стартового символа достижимо слово $w$.

\Def 
Язык $L$ \textit{распознаётся} грамматикой $G$, если $L = \{w \in \Sigma^{*}\ |\ S \vdash_{G} w\}$, то есть язык $L$ состоит из таких слов, которые выводимы в грамматике $G$. Если $L$ распознаётся грамматикой $G$, то его обозначают как $L \brackets{G}$.

\subsubsection*{Праволинейные грамматики и праволинейные языки}

\Def Грамматика $G$ называется праволинейной, если правила из $P$ имеют вид либо $A \rightarrow wB$, либо $A \rightarrow w$, где $A$, $B$ — нетерминальные символы, то есть $A$, $B \in N$, и $w \in \Sigma^*$.

\Def Язык $L(G)$ называется праволинейным, если грамматика $G$, которой распознаётся (задаётся) язык $L$, является праволинейной.


\subsubsection*{Теорема о совпадении классов автоматных и праволинейных языков}

\Th Множество автоматных языков равно множеству языков, 
задаваемых праволинейными грамматиками.

\Proof \textbf{Грамматика $\rightarrow$ Автомат:} Состояния в автомате $=$ нетерминалы в грамматике $+$ сток

Пусть дан язык, задаваемый праволинейной грамматикой: $\mathcal{L} = \mathcal{L}(G),\ G = \{N, \Sigma, P, S\}$

Построим $M = \{N \cup \{q_f\}, \Sigma, \Delta', S, \{q_f\}\}$, где

$\Delta' = \{ \{A, w\} \rightarrow B \text{, если}\ A \rightarrow wB \in P  \} \cup
\{\{A, w\} \rightarrow q_f \text{, если}\ A \rightarrow w \in P \}$

Надо показать, что $\mathcal{L}(G) = \mathcal{L}(M)$

\underline{$\mathcal{L}(G) \subset \mathcal{L}(M)$}:
Индукция по длине вывода грамматики (количество $\vdash$)

$$
w \in \mathcal{L}(G) \Rightarrow 
S \vdash_{G} w  \Rightarrow ..?.. \Rightarrow
\{S, w\} \vdash_{M} \{q_f, \varepsilon\} \Rightarrow
w \in \mathcal{L}(M)\ \qquad  (*)
$$

Покажем, что если $C \vdash_{G} wD$ (за $k$ шагов), 
то $\{C, w\} \vdash_{M} \{D, \varepsilon\}$.

Индукция по $k$:

$k = 0$:
$$
C \vdash_{G} C \Rightarrow 
w = \varepsilon \Rightarrow
\{C, \varepsilon\} \vdash_{M} \{C, \varepsilon\}
$$

Переход:
$$
C \vdash wD \text{ за } k \text{ шагов } \Rightarrow
\exists E, u, v\ :\ C \vdash_{G} uE \text{ за } k - 1 \text{ шаг },
uE \vdash_{G} uvD \text{ за 1 шаг } (w = uv) 
$$

Тогда по предположению индукции
$$
\{C, u\} \vdash_M \{E, \varepsilon\},
\{E, v\} \vdash_M \{D, \varepsilon\} \Rightarrow
\{C, uv\} \vdash_M \{D, \varepsilon\} \text{ (по транзитивности)}
$$

Если $C \vdash_{G} w$, то $\{C, w\} \vdash_{M} \{q_f, \varepsilon\}$

Так как из $q_f$ нет переходов, то $\exists E : C \vdash_{G} uE$ (за $k - 1$),
$uE \vdash_{G} uv$ (за 1) \ ($E \rightarrow v \in P$)

Тогда $\{C, uv\} \vdash_{M} \{E, \} \vdash \{q_f, \varepsilon\}$

Чтобы доказать $(*)$, достаточно вместо $C$ подставить $S$

$\{S, w\} \vdash \{q_f, \varepsilon\} \Rightarrow w \in \mathcal{L}(M)$

\underline{$\mathcal{L}(M) \subset \mathcal{L}(G)$}:
$w \in \mathcal{L}(M) \Rightarrow \{S, w\} \vdash_{M} \{q_f, \varepsilon\}$

Покажем, что если $\{C, w\} \vdash_{M} \{D, \varepsilon\}$, 
то $C \vdash_{G} wD$

Индукция по длине пути в автомате:

База: $k = 0$
$$
\{C, w\} \vdash_{M} 
\{D, \varepsilon\},  \;\; C = D, \, w = \varepsilon \;\Rightarrow\;
C \vdash_{G} C
$$

Переход:
$$
\exists E, u : \langle C, w\rangle \vdash_{M} \langle E, u\rangle \vdash_{M} \{D, \varepsilon\}
$$
$$
w = uv \Rightarrow \text{ по предположению индукции: }
C \vdash_{G} uE, \; E \vdash vD \Rightarrow
C \vdash_{G} uvD = wD
$$

Так как из $q_f$ нет переходов, то $\{S, w\} \vdash \{q_f, \varepsilon\} \Rightarrow$

$\exists E, u, v: \{S, uv\} \vdash_{M} 
\{E, v\} \vdash_{M} 
\{q_f, \varepsilon\} \Rightarrow
\{S, u\} \vdash \{E, \varepsilon\} \Rightarrow
S \vdash_{G} uE
$
$$
\{E, v\} \vdash_{M} \{q_f, \varepsilon\} \Rightarrow
E \vdash_{G} v
$$
$$
S \vdash_{G} uE, \; E \vdash_{G} v \Rightarrow S \vdash_{G} uv
$$

\textbf{Автомат $\rightarrow$ Грамматика:}
Идея: возьмём автомат с 1 завершающим состоянием $q_{stock}$ (и из этого $q_{stock}$ нет переходов)

$M = \{Q, \Sigma, \Delta, q_0, \{q_{stock}\}\}$

$G = \{Q \setminus \{q_{stock}\}, \Sigma, P, q_0\}$

$$
P = \{q_1 \vdash wq_2\ |\ q_2 \neq q_{stock} \text{ и } 
\{q_1, w\} \rightarrow q_2 \in \Delta \} \cup
\{q_1 \rightarrow w\ |\ \{q_1, w\} \rightarrow q_{stock} \in \Delta \}
$$

Доказательство аналогично \textit{Грамматика $\rightarrow$ Автомат}