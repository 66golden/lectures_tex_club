\subsection{7. Минимальный ДКА, алгоритм построения.}
Мы хотим преобразовать автомат так, чтобы его состояния были попарно неэквивалентны. Но у нас есть только слова. Как быть? Введем эквивалентность по словам фиксированной длины.

\Def $q_1 \underset{n}{\sim} q_2$, если для любого слова $w$ : $\arrowvert w \arrowvert \leq n$:
$$\Delta(q_1, w) \in F \Longleftrightarrow \Delta(q_2, w) \in F$$.
Это отношение эквивалентности (аналогично своему $M$-аналогу), поэтому можно ввести $Q/_{\underset{n}{\sim}}$.\\

\Lemma $q_1 \underset{\arrowvert Q \arrowvert - 2}{\sim} q_2 \Longrightarrow q_1 \sim q_2  $

\Proof Если $q_1 \underset{i + 1}{\sim} q_2 \Longrightarrow q_1 \underset{i}{\sim} q_2$, тогда $\arrowvert Q /_{\underset{i}{\sim}} \arrowvert \leqslant \arrowvert Q /_{\underset{i + 1}{\sim}} \arrowvert$.

Покажем, что если $\arrowvert Q /_{\underset{i}{\sim}} \arrowvert = \arrowvert Q /_{\underset{i + 1}{\sim}} \arrowvert$, то $\arrowvert Q /_{\underset{i + 1}{\sim}} \arrowvert = \arrowvert Q /_{\underset{i + 2}{\sim}} \arrowvert$. Пусть существуют состояния $q_1$ и $q_2$. такие что $q_1 \underset{i + 2}{\nsim} q_2$, $q_1 \underset{i + 1}{\sim} q_2$.

Тогда $\exists u, \arrowvert u \arrowvert \leqslant i + 2$, что без ограничения общности $\Delta(q_1, u) \in F$, $\Delta (q_2, u) \notin F$. Заметим, что должно быть верно, что $\arrowvert u \arrowvert = i + 2$, иначе возникнет противоречие с $(i + 1)$-эквивалентностью. Тогда пусть $u = av$, где $a$ -- некоторый символ. 

Рассмотрим $p_1 = \Delta(q_1, a), p_2 = \Delta(q_2, a) \Longrightarrow \Delta(p_1,v)\in F, \; \Delta(p_2,v)\notin F$, при этом $|v| \leqslant i + 1$

Тогда $p_1 \underset{i + 1}{\nsim} p_2$ так как существует слово длины $i+1$, которое их различает $\Longrightarrow p_1 \underset{i}{\nsim} p_2 \Longrightarrow q_1 \underset{i + 1}{\nsim} q_2$ Противоречие.

Отсюда следует, что если $\arrowvert Q /_{\underset{i}{\sim}} \arrowvert = \arrowvert Q /_{\underset{i + 1}{\sim}} \arrowvert$, то $\arrowvert Q /_{\underset{i + 1}{\sim}} \arrowvert = \arrowvert Q /_{\underset{i + 2}{\sim}} \arrowvert$. И так для всех $i$. Поэтому если мы остановились, новые классы эквивалентности не появятся. Понятно, что мы можем увеличить класс эквивалентности O($|Q|$) раз

\EndProof

\textit{Поэтому наш алгоритм выглядит так:}

Множество $Q /_{\sim_0}$ представляет собой $\{ F, Q \setminus F \}$, то есть это множество из множества завершающих состояний и множества состояний, не являющихся завершающими.

До тех пор, пока количество классов эквивалентности меняется, увеличиваем длину слов, по которым проверяем эквивалентность.

Пусть у нас была длина $n$, мы перешли к $n + 1$. Рассмотрим какой-то класс эквивалентности. 

Посмотрим на переход по первой букве. Заметим тогда, что нам останется пройти $n$ символов. Все эти вершины (на расстоянии $1$) мы уже распределили по классам эквивалентности на основе слов длины $n$, поэтому нужно для каждого сотояния найти множество классов эквивалентности его соседей. Если множества различаются, то эти состояния теперь разойдутся по разным классам. Совпадают, значит, останутся в одном классе. 

В соответствии с этим получим новое распределение состояний по классам.

