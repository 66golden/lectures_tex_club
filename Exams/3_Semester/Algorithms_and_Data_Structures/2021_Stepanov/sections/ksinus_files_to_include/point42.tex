Есть некоторый многоугольник — замкнутая несамопересекающаяся ломаная. Нужно найти его триангуляцию — разбиение многоугольника на треугольники с помощью диагоналей такое, что все эти треугольники лежат в многоугольники. Диагонали здесь являются отрезками между несоседними вершинами, полностью лежащие во внутренности многоугольника.

Вершина $v_i$ называется ухом, если диагональ $v_{i-1}v_{i+1}$ лежит строго во внутренней области многоугольника P

Лемма о двух ушах: У любого простого n-вершинного многоугольника P всегда существует два не пересекающихся между собой уха.

Доказательство будем вести по индукции. Базовый случай: $n=4$. Предположим для всех многоугольников, количество вершин в которых не больше $n$, теорема верна. Рассмотрим многоугольник P, в котором $n+1$ вершина. Далее возможны два случая:
\begin{enumerate}
    \item Случай, когда $v_i$ является ухом в P. \\
    Произвольная выпуклая вершина $v_i$ многоугольника P является ухом. Отрезав это ухо, мы уменьшим число вершин P на одну. В результате, получим n-вершинный многоугольник P'. По предположению индукции у него существует два непересекающихся уха. Учитывая, что уши P' являются ушами и P, несложно заметить, что для P теорема верна.
    \item Произвольная выпуклая вершина $v_i$ многоугольника P не является ухом. \\
    В таком случае в треугольнике $\Delta v_{i-1}v_iv_{i+1}$ лежат вершины, принадлежащие P. Из этих вершин выберем вершину q, которая будет ближе всего к $v_i$. Проведём отрезок Q, который разделит P на два многоугольника: $P_1$ и $P_2$. В каждом из них будет не более n вершин, следовательно у каждого будет по два непересекающихся уха. Даже если предположить, что ухо из $P_1$ и ухо из $P_2$ будут пересекаться по стороне $v_{iq}$, в P всё равно будет не менее двух непересекающихся ушей.
\end{enumerate}