Рассмотрим алгоритм построения триагнуляции.

Поддерживаем вершины многоугольника в двусвязном списке. Далее напишем процедуру $FindEar$, которая проводит отрезок между $v_{i + 1}$ и $v_{i - 1}$ и смотрит, попали ли ещё точки внутри. Если нет, то это ухо, иначе найдём точку внутри треугольника, и проведём диагональ. Точек найдётся не более, чем $\frac{n}{2} + 1$. Далее запустимся рекурсивно. Найдём асимптотику, пользуясь мастер-теоремой: $T (n) = \Theta(n) + T (\frac{n}{2}) \Longrightarrow T (n) = \Theta (n)$.

Алгоритм весь будет работать за $\Theta (n^2)$. Существует алгоритм за $O (n \log n)$, который, скорее всего, рассмотрен не будет.

Поиск в Боре:
При решении этой задачи, обход бора совершается из его корня по рёбрам, отмеченным символами строки $S$, пока возможно. Если с последним символом $S$ мы приходим в терминальную вершину, то $S$ — слово из словаря. Если в какой-то момент ребра, отмеченного нужным символом, не находится, то строки $S$ в словаре нет. Ясно, что это занимает $O(|S|)$ времени. Таким образом, бор — это эффективный способ хранить словарь и искать в нем слова.