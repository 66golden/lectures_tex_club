
\subsubsection*{Точка и вектор}

В $R^2$ можно задать декартовую систему координат, обладающей осями абсцисс и ординат. Каждую точку можно задать двумя координатами $(a, b)$, обозначающие проекцию на каждую из осей. Каждой точке можно сопоставить радиус вектор из $(0, 0)$ в $(a, b)$. Для точки заведём структуру point:

\begin{lstlisting}
struct point {
  int x, y;
};
\end{lstlisting}


Пусть у нас есть вектор $\vec{v}$ из точки $p$ в точку $q$. Его можно задать как $(q.x - p.x, q.y - p.y)$. Для структуры point можно написать оператор «минус»:

\begin{lstlisting}
point operator - (const point& other) const {
  return {x - other.x, y - other.y}
}
\end{lstlisting}

Можно также реализовать оператор «плюс», соответствующее сложению векторов, а также оператор «звёздочка» умножения на скаляр.

\subsubsection*{Прямая}

Далее нам пригодится примитив «прямая», и мы хотим, чтобы она однозначно задавалась. Удобно её представлять в виде $y = kx + b$, но есть проблема: вертикальные прямые так не задаются. Можно было бы повернуть на случайный угол $\alpha$. Но с целыми числами возможна проблема. Поэтому будем представлять прямую в виде $ax + by + c = 0$:

\begin{lstlisting}
struct line {
  int a, b, c;
};
\end{lstlisting}

Но есть проблема: прямая может задаваться неоднозначно. Чтобы убедиться в этом, достаточно умножить обе части на $\lambda$. 

Удобно строить прямую по двум точкам:

\begin{lstlisting}
line (const point& p, const point& q) {
  if (p == q) {
    exception;
  }
  a = p.y - q.y;
  b = q.x - p.x;
  c = p.x * q.y - p.y * q.x;
}
\end{lstlisting}

Действительно, если есть две точки $p(x_1, y_1)$, $q(x_2, y_2)$, то $a = y_1 - y_2$, $b = x_2 - x_1$, $c = x_1 y_2 - x_2 y_1$. Доказывается подстановкой точек $p$ и $q$ в уравнение $ax + by + c = 0$;

С прямой $ax + by + c = 0$ связаны два вектора: вектор нормали $\vec{n} (a, b)$ и направляющий вектор $\vec{l} (b, -a)$, можно взять как $(-b, a)$. 
