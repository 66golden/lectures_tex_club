
\subsubsection*{Примитив окружности}

Окружность можно хранить как пару: точка $p$, отвечающая за центр окружности, и её радиус $r$.

\subsubsection*{Пересечение окружности и прямой}

Пересечение окружности и прямой $l$ есть тогда и только тогда, когда расстояние между центром окружности и прямой $l$ не превосходит $r$. Можно полагать, что мы ищем решение системы уравнений:

\begin{center}
    $\begin{cases}
    ax + by + c = 0; \\
    (x - x_0)^2 + (y - y_0)^2 = r^2
    \end{cases}$
\end{center}

Найдём $q$ - проекцию точки $p$ на прямую. (для этого можно опустить перпендикуляр из центра окружности на прямую.) Далее отступим от проекции $q$ на некоторое расстояние по обе стороны. Для этого найдём направляющий вектор прямой $l$, нормируем его. Найдём $k = \sqrt{r^2 - {dist}^2 (p, q)}$, где $dist (p, q) = \sqrt{(p.x - q.x)^2 + (p.y - q.y)^2}$. Находим $q + k \vec{v}$, $q - k \vec{v}$, где $\vec{v}$ — направляющий вектор $l$ единичной длины. Это и есть искомые точки.

\subsubsection*{Пересечение двух окружностей}

Пусть заданы две окружности: одна с центром в точке $(x_1, y_1)$ и радиусом $r_1$; другая с центром в точке $(x_2, y_2)$ и радиусом $r_2$. Можно полагать, что мы ищем решение системы уравнений:

\begin{center}
    $\begin{cases}
        (x - x_1)^2 + (y - y_1)^2 = r_1^2; \\
        (x - x_2)^2 + (y - y_2)^2  = r_2^2.
    \end{cases}$
\end{center}

Вычтем второе уравнение из первого.

\begin{center}
    $\begin{cases}
        (2x_2-2x_1)x + (2y_2-2y_1)y + (x^2_1 - x^2_2 + y^2_1 - y^2_2 + r^2_2 - r_1^2) = 0; \\
        (x - x_2)^2 + (y - y_2)^2  = r_2^2.
    \end{cases}$
\end{center}

Свели задачу к поиску точки пересечения окружности и прямой.

Здесь не обойтись без чисел с плавающей точкой.
